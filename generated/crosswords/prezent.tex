
\documentclass[11pt]{article}
\usepackage[utf8]{inputenc}

\usepackage[unboxed]{cwpuzzle}
\usepackage[top=15mm,bottom=15mm,left=15mm,right=3cm,twoside]{geometry}
\usepackage{adjustbox}

\newcommand\drarr{$\rightarrow \!\!\!\!\! \downarrow$}
\newcommand\rarr{$\rightarrow$}
\newcommand\darr{$\downarrow$}

\begin{document}



\newpage%\section*{Krzyżówka 1}

\noindent\begin{Puzzle}{24}{24}|*	|*	|*	|*	|*	|*	|*	|*	|*	|*	|*	|*	|*	|*	|*	|[1][S]\darr	|*	|[2][S]\drarr	|t	|a	|j	|g	|a	|*	|*	|.
|*	|*	|*	|*	|*	|*	|*	|*	|*	|[3][S]\rarr	|g	|a	|l	|l	|u	|p	|*	|g	|*	|*	|[4][S]\darr	|[5][S]\darr	|*	|[6][S]\darr	|*	|.
|*	|*	|*	|*	|*	|*	|*	|*	|*	|*	|*	|[7][S]\darr	|[8][S]\darr	|*	|*	|o	|[9][S]\darr	|o	|*	|*	|a	|m	|*	|g	|*	|.
|*	|*	|*	|*	|*	|*	|*	|*	|*	|*	|*	|p	|d	|*	|*	|d	|k	|o	|*	|[10][S]\darr	|r	|i	|[11][S]\darr	|a	|[12][S]\darr	|.
|*	|*	|*	|*	|*	|*	|*	|*	|*	|[13][S]\rarr	|d	|i	|r	|k	|*	|o	|a	|l	|[14][S]\darr	|o	|t	|k	|c	|m	|b	|.
|*	|*	|*	|*	|*	|*	|*	|*	|*	|*	|*	|a	|o	|*	|[15][S]\darr	|l	|t	|e	|b	|b	|y	|r	|z	|b	|ł	|.
|*	|*	|*	|*	|*	|*	|*	|*	|*	|*	|*	|s	|b	|[16][S]\drarr	|b	|o	|j	|*	|r	|m	|l	|o	|y	|i	|y	|.
|*	|*	|*	|*	|*	|*	|*	|*	|*	|*	|*	|e	|i	|k	|u	|g	|a	|*	|o	|u	|e	|p	|t	|t	|s	|.
|*	|*	|*	|*	|*	|*	|*	|*	|*	|*	|[17][S]\drarr	|k	|a	|o	|l	|i	|n	|*	|w	|r	|r	|a	|a	|[][,]{ }	|k	|.
|*	|*	|*	|*	|*	|*	|*	|*	|*	|*	|c	|[][,]{ }	|z	|w	|w	|a	|g	|*	|i	|o	|i	|ń	|n	|v	|o	|.
|*	|*	|*	|*	|*	|*	|*	|*	|*	|*	|a	|w	|g	|a	|a	|*	|*	|[18][S]\darr	|c	|w	|a	|s	|i	|i	|t	|.
|*	|*	|*	|*	|*	|*	|*	|*	|*	|*	|b	|o	|*	|l	|*	|[19][S]\rarr	|ł	|o	|z	|a	|*	|t	|e	|l	|k	|.
|*	|*	|*	|*	|[20][S]\rarr	|n	|u	|b	|u	|k	|*	|d	|*	|*	|*	|*	|*	|p	|*	|n	|*	|w	|[][,]{ }	|l	|a	|.
|*	|*	|*	|[21][S]\rarr	|z	|a	|b	|u	|r	|z	|e	|n	|i	|e	|[][,]{ }	|n	|e	|r	|w	|i	|c	|o	|w	|e	|*	|.
|*	|[22][S]\drarr	|d	|w	|u	|d	|z	|i	|e	|s	|t	|o	|p	|a	|r	|o	|l	|a	|t	|e	|k	|*	|y	|m	|[23][S]\darr	|.
|[24][S]\drarr	|k	|i	|e	|l	|o	|n	|*	|[25][S]\rarr	|b	|o	|l	|i	|m	|ó	|w	|*	|w	|*	|*	|*	|*	|b	|s	|o	|.
|k	|r	|[26][S]\rarr	|c	|z	|t	|e	|r	|o	|w	|i	|o	|s	|ł	|ó	|w	|k	|a	|*	|*	|*	|*	|i	|o	|s	|.
|a	|e	|*	|*	|*	|[27][S]\rarr	|d	|o	|k	|ł	|a	|d	|n	|o	|ś	|ć	|*	|*	|*	|*	|*	|*	|ó	|n	|s	|.
|r	|d	|*	|*	|*	|[28][S]\rarr	|s	|z	|e	|w	|r	|o	|*	|[29][S]\rarr	|s	|u	|b	|k	|u	|l	|t	|u	|r	|a	|*	|.
|o	|e	|*	|*	|*	|*	|*	|*	|*	|[30][S]\rarr	|l	|w	|i	|a	|[][,]{ }	|s	|p	|ó	|ł	|k	|a	|*	|c	|*	|*	|.
|t	|n	|*	|*	|[31][S]\rarr	|l	|a	|w	|o	|n	|i	|c	|h	|a	|*	|*	|[32][S]\rarr	|d	|u	|a	|l	|i	|z	|m	|*	|.
|a	|s	|[33][S]\rarr	|l	|o	|g	|i	|k	|a	|[][,]{ }	|w	|o	|l	|n	|a	|*	|*	|*	|[34][S]\rarr	|w	|a	|f	|e	|l	|*	|.
|ż	|*	|[35][S]\rarr	|r	|e	|n	|t	|a	|[][,]{ }	|i	|n	|w	|a	|l	|i	|d	|z	|k	|a	|*	|*	|*	|*	|*	|*	|.
|*	|[36][S]\rarr	|z	|a	|j	|ę	|c	|i	|a	|[][,]{ }	|w	|y	|r	|ó	|w	|n	|a	|w	|c	|z	|e	|*	|*	|*	|*	|.
|*	|*	|[37][S]\rarr	|h	|y	|d	|r	|o	|f	|i	|l	|*	|*	|*	|*	|*	|*	|*	|*	|*	|*	|*	|*	|*	|*	|.\end{Puzzle}

\newpage

\begin{PuzzleClues}{\textbf{Poziome}\\}\Clue{2}{}{obszar północnej Ameryki, Europy i Azji, porośnięty borealnym lasem iglastym}
\Clue{3}{}{amerykański socjolog i publicysta (1901-84); założyciel instytutu Gallupa}
\Clue{13}{}{rodzaj sztyletu}
\Clue{16}{}{żartobliwie o strachu, lęku; współcześnie tylko jako człon frazymieć boja}
\Clue{17}{}{skała osadowa zawierająca w swym składzie głównie kaolinit, a także m.in. kwarc i mikę}
\Clue{19}{}{wierzba szara, eurazjatycki krzew wysokości do 5 m}
\Clue{20}{}{wyprawiona miękka skóra, najczęściej cielęca, o aksamitnej powierzchni, używana do wyrobu wierzchów obuwia, galanterii skórzanej}
\Clue{21}{}{zaburzenie psychiczne}
\Clue{22}{}{osoba mająca więcej niż 20 lat, przed 30 rokiem życia}
\Clue{24}{}{kieliszek, niewielkie naczynie używane najczęściej do picia wódki}
\Clue{25}{}{wieś w Polsce położona w województwie łódzkim, w powiecie skierniewickim, siedziba gminy Bolimów}
\Clue{26}{}{łódka mieszcząca czterech wioślarzy}
\Clue{27}{}{cecha rzeczy, wykonania, rezultatu, wytworu: to, że coś jest wykonane lub zrealizowane z dbałością o szczegóły}
\Clue{28}{}{GIEMZA}
\Clue{29}{}{wzory, normy, zasady obowiązujące w grupie społecznej o tej samej nazwie, odmienne od wzorców przyjętych przez ogół społeczeństwa}
\Clue{30}{}{spółka osobowa, w której jeden ze wspólników jest wyłączony od udziału w stratach, a drugi od udziału w zyskach}
\Clue{31}{}{białoruski taniec ludowy utrzymany w metrum parzystym}
\Clue{32}{}{podział elementu systemu władzy na dwie, względnie niezależne instytucje}
\Clue{33}{}{logika wolna od założeń ontologicznych, takich jak założenie niepustości dziedziny}
\Clue{34}{}{delikatne, łamliwe ciasto w kształcie cienkich, okrągłych lub kwadratowych płatków}
\Clue{35}{}{świadczenie pieniężne uzyskiwane z powodu niezdolności do pracy w wyniku choroby lub kalectwa}
\Clue{36}{}{dodatkowe zajęcia szkolne, podaczas których uczniowie mają pod okiem nauczyciela nadrabiać zaległości w nauce}
\Clue{37}{}{substancja, która łatwo bierze udział w tworzeniu lepkich wodnych roztworów koloidowych}\end{PuzzleClues}

\begin{PuzzleClues}{\textbf{Pionowe}\\}\Clue{1}{}{dziedzina medycyny zajmująca się diagnozowaniem i leczeniem (także chirurgicznym) chorób stopy i stawu skokowo-goleniowego; podolodzy szczególnie często leczą sportowców}
\Clue{2}{}{miasto w środkowej Anglii nad rzeką Ouse}
\Clue{4}{}{nauka o budowie dział i innych środków ogniowych}
\Clue{5}{}{państwo o niewielkim obszarze terytorialnym i o niewielkiej liczbie ludności}
\Clue{6}{}{wariant gambitu królewskiego, w którym czarne chcą natychmiastowo wykorzystać osłabienie królewskiego skrzydła białych}
\Clue{7}{}{luźna skała osadowa, osadzona na przedpolach moren czołowych przez wody lodowcowe rozmywające morenę czołową i płynące od czoła topniejącego lodowca}
\Clue{8}{}{niewielki utwór literacki}
\Clue{9}{}{roślina zielna z motylkowatych uprawiana dla jadalnych nasion i na zielony nawóz}
\Clue{10}{}{ciągła konstrukcja warowna lub umocnieniowa w postaci muru wykonanego z kamienia lub cegły}
\Clue{11}{}{sposób czytania, w którym część tekstu zostaje pominięta przez czytającego}
\Clue{12}{}{Fulgensia - rodzaj grzybów z rodziny złotorostowatych (Teloschistaceae)}
\Clue{14}{}{Tadeusz (1847-1928); anatomopatolog, profesor UJ; zarazek duru brzusznego}
\Clue{15}{}{1) wieloletnia roślina ze złożonych; słonecznik bulwiasty, 2) zgrubiały wierzchołek pędu podziemnego z pączkami, z którego mogą rozwijać się młode pędy}
\Clue{16}{}{amerykański ptak z rodziny tęgosterów}
\Clue{17}{}{powóz czterokołowy będący odmianą kabrioletu}
\Clue{18}{}{zbiór elementów scenografii, otoczenia, dekoracji, muzyki, efektów itp. osadzonych w jakimś stylu i towarzyszących jakiemuś wydarzeniu}
\Clue{22}{}{żartobliwie: duży fiat}
\Clue{23}{}{miasto w środkowej Holandii; 48,8 tys. mieszkańców (1982 r.)}
\Clue{24}{}{profilowanie; badanie charakterystycznych zjawisk występujących w otworze wiertniczym}\end{PuzzleClues}\newpage%\section*{Krzyżówka 2}

\noindent\begin{Puzzle}{22}{33}|*	|*	|*	|*	|*	|*	|*	|*	|*	|*	|[1][S]\drarr	|p	|e	|t	|r	|e	|l	|o	|w	|a	|t	|e	|*	|.
|*	|*	|*	|*	|*	|*	|*	|*	|*	|*	|b	|*	|[2][S]\drarr	|k	|o	|ź	|l	|a	|k	|*	|*	|[3][S]\darr	|*	|.
|*	|*	|*	|*	|*	|*	|*	|*	|*	|[4][S]\drarr	|r	|a	|k	|*	|*	|[5][S]\darr	|*	|[6][S]\drarr	|g	|i	|p	|s	|*	|.
|*	|*	|*	|*	|*	|*	|[7][S]\darr	|*	|[8][S]\darr	|r	|u	|*	|u	|*	|[9][S]\drarr	|s	|z	|k	|i	|c	|*	|i	|*	|.
|*	|*	|*	|*	|*	|*	|o	|*	|v	|e	|k	|[10][S]\darr	|s	|[11][S]\rarr	|s	|z	|u	|r	|e	|*	|[12][S]\darr	|e	|[13][S]\darr	|.
|*	|*	|*	|*	|*	|[14][S]\darr	|p	|*	|i	|g	|*	|s	|z	|*	|z	|y	|*	|z	|*	|[15][S]\darr	|n	|ć	|e	|.
|[16][S]\drarr	|s	|e	|n	|s	|o	|r	|*	|o	|u	|*	|t	|y	|*	|a	|l	|[17][S]\drarr	|a	|r	|k	|a	|*	|r	|.
|n	|*	|*	|*	|*	|b	|z	|*	|l	|ł	|*	|r	|t	|[18][S]\darr	|t	|k	|k	|k	|[19][S]\darr	|r	|d	|*	|o	|.
|i	|*	|*	|*	|*	|s	|ę	|*	|a	|a	|*	|a	|a	|c	|a	|r	|r	|ó	|c	|y	|u	|*	|t	|.
|e	|*	|*	|*	|*	|z	|d	|*	|[][,]{ }	|[][,]{ }	|[20][S]\darr	|c	|*	|z	|[][,]{ }	|e	|a	|w	|z	|l	|r	|*	|o	|.
|k	|*	|*	|*	|*	|c	|z	|*	|d	|o	|l	|h	|*	|a	|d	|t	|j	|k	|a	|[][,]{ }	|*	|[21][S]\darr	|m	|.
|o	|*	|*	|*	|*	|z	|i	|*	|a	|d	|a	|[][,]{ }	|[22][S]\darr	|k	|e	|*	|a	|a	|r	|a	|[23][S]\darr	|g	|a	|.
|n	|*	|*	|*	|*	|y	|k	|*	|[][,]{ }	|r	|m	|b	|ś	|a	|j	|*	|n	|[][,]{ }	|e	|n	|w	|e	|n	|.
|f	|*	|*	|*	|*	|m	|*	|[24][S]\drarr	|s	|y	|b	|i	|l	|l	|a	|*	|k	|b	|c	|t	|i	|r	|[][,]{ }	|.
|l	|*	|*	|*	|*	|u	|*	|c	|p	|w	|e	|e	|i	|a	|n	|*	|a	|i	|z	|a	|e	|m	|g	|.
|i	|*	|*	|*	|*	|r	|*	|e	|a	|a	|r	|r	|z	|k	|i	|*	|*	|a	|k	|r	|r	|a	|a	|.
|k	|*	|*	|[25][S]\darr	|*	|e	|*	|l	|l	|n	|t	|n	|g	|a	|r	|*	|[26][S]\darr	|ł	|a	|k	|z	|n	|w	|.
|t	|[27][S]\rarr	|c	|o	|o	|k	|*	|a	|l	|i	|*	|y	|a	|*	|y	|*	|z	|o	|[][,]{ }	|t	|c	|i	|ę	|.
|o	|*	|*	|c	|*	|*	|*	|*	|a	|a	|*	|*	|w	|*	|*	|*	|a	|b	|m	|y	|h	|s	|d	|.
|w	|[28][S]\rarr	|k	|i	|s	|z	|k	|a	|*	|*	|*	|*	|k	|*	|*	|*	|s	|r	|i	|c	|e	|t	|z	|.
|o	|[29][S]\rarr	|k	|o	|n	|s	|e	|k	|r	|a	|c	|j	|a	|*	|*	|*	|u	|z	|e	|z	|n	|k	|i	|.
|ś	|*	|*	|*	|*	|*	|[30][S]\drarr	|l	|i	|m	|m	|u	|*	|*	|*	|*	|w	|u	|s	|n	|e	|a	|a	|.
|ć	|[31][S]\rarr	|b	|ą	|k	|[][,]{ }	|p	|s	|t	|r	|y	|*	|*	|*	|[32][S]\rarr	|h	|a	|c	|z	|y	|k	|*	|r	|.
|*	|*	|*	|*	|[33][S]\rarr	|w	|a	|r	|g	|a	|[][,]{ }	|d	|o	|l	|n	|a	|*	|h	|a	|*	|*	|*	|z	|.
|*	|[34][S]\drarr	|k	|o	|s	|t	|r	|z	|e	|w	|a	|[][,]{ }	|t	|a	|t	|r	|z	|a	|ń	|s	|k	|a	|*	|.
|[35][S]\drarr	|a	|s	|y	|m	|i	|l	|o	|w	|a	|n	|i	|e	|[][,]{ }	|s	|i	|ę	|*	|c	|*	|*	|[36][S]\darr	|*	|.
|o	|f	|*	|*	|[37][S]\rarr	|h	|a	|m	|b	|u	|r	|c	|z	|y	|k	|*	|*	|*	|o	|*	|[38][S]\darr	|r	|*	|.
|b	|*	|*	|*	|*	|[39][S]\rarr	|m	|o	|l	|a	|l	|n	|o	|ś	|ć	|*	|*	|*	|w	|*	|r	|y	|*	|.
|s	|[40][S]\rarr	|s	|y	|s	|t	|e	|m	|a	|t	|y	|k	|a	|*	|[41][S]\rarr	|d	|e	|k	|a	|l	|o	|g	|*	|.
|a	|[42][S]\rarr	|g	|ł	|ó	|w	|n	|o	|d	|o	|w	|o	|d	|z	|ą	|c	|y	|*	|*	|*	|k	|i	|*	|.
|d	|*	|*	|[43][S]\rarr	|r	|e	|t	|r	|o	|g	|r	|a	|d	|a	|c	|j	|a	|*	|*	|*	|i	|e	|*	|.
|a	|[44][S]\rarr	|l	|o	|b	|o	|*	|*	|*	|*	|*	|*	|*	|*	|[45][S]\rarr	|b	|u	|c	|h	|n	|e	|r	|*	|.
|*	|*	|*	|*	|*	|*	|*	|*	|*	|*	|*	|*	|*	|*	|*	|*	|*	|*	|*	|*	|t	|*	|*	|.
|*	|*	|*	|*	|*	|*	|*	|*	|*	|*	|*	|*	|*	|*	|*	|*	|*	|*	|*	|*	|*	|*	|*	|.\end{Puzzle}

\newpage

\begin{PuzzleClues}{\textbf{Poziome}\\}\Clue{1}{}{burzykowate, Procellariidae - rodzina ptaków z rzędu rurkonosych; obejmuje gatunki oceaniczne zamieszkujące otwarte morza całego świata}
\Clue{2}{}{mocne, ciemne piwo dolnej fermentacji}
\Clue{4}{}{jakieś negatywne zjawisko, coś, co jest równie szkodliwe i niszczące, jak poważna choroba}
\Clue{6}{}{trudna, kłopotliwa i nieprzyjemna sytuacja, problem}
\Clue{9}{}{ogólne ujęcie jakiegoś tematu (problemu, zagadnienia) w formie artykułu, pracy naukowej bądź innej wypowiedzi}
\Clue{11}{}{niewielki szybik stosowany w robotach geologiczno-poszukiwawczych}
\Clue{16}{}{wyspecjalizowana struktura organizmu mająca zdolność do odbierania bodźców z zewnątrz}
\Clue{17}{}{archaiczne określenie skrzyni, szczególnie w biblii; drewniany lub metalowy pojemnik o kształcie prostopadłościanu z wypukłym wiekiem}
\Clue{24}{}{wieszczka, przeprowiadająca przyszłość}
\Clue{27}{}{angielski żeglarz i jeden z najwybitniejszych odkrywców (1728-79); zginął na Hawajach w starciu z krajowcami}
\Clue{28}{}{rodzaj wyrobu wędliniarskiego - mieszanka krwi, podrobów, kaszy i mięsa w jelicie zwierzęcym}
\Clue{29}{}{śluby służby Bogu składane w zakonie lub przez osobę świecką}
\Clue{30}{}{godność urzędnicza sprawowana w starożytnej Asyrii, związana z administracją i przyznawana na jeden rok}
\Clue{31}{}{Botaurus pinnatus - gatunek ptaka z rodziny czaplowatych (Ardeidae)}
\Clue{32}{}{wykonana najczęściej z metalu część osprzętu wędkarskiego, zaostrzona na końcu, która, dzięki zaczepionej na niej przynęcie, służy do zahaczenia ryby przez wędkarza, a następnie, dzięki żyłce do niej przyczepionej, wyciągnięcia jej na brzeg}
\Clue{33}{}{labium inferius, jedna z warg ust człowieka}
\Clue{34}{}{Festuca tatrae - gatunek trawy z rodziny wiechlinowatych}
\Clue{35}{}{przystosowywanie się do nowych warunków i życia w jakimś miejscu}
\Clue{37}{}{mieszkaniec Hamburga}
\Clue{39}{}{liczba moli danej substancji chemicznej rozpuszczona w 1 kg danego rozpuszczalnika}
\Clue{40}{}{dział nauki zajmujący się klasyfikacją obiektów lub zjawisk}
\Clue{41}{}{w przenośni: zbiór obowiązujących norm, reguł, zasad}
\Clue{42}{}{najwyższy dowódca}
\Clue{43}{}{ruch ciała niebieskiego, które pozornie obraca się lub porusza po orbicie w kierunku przeciwnym, niż większość ciał w danym układzie orbitalnym}
\Clue{44}{}{rodzaj słodkiego jabłka, owoc z drzewa odmiany o tej samej nazwie}
\Clue{45}{}{(1813-37), niemiecki dramatopisarz i działacz rewolucyjny, prekursor dramatu naturalistycznego; „Woyzeck”, „Śmierć Dantona”}\end{PuzzleClues}

\begin{PuzzleClues}{\textbf{Pionowe}\\}\Clue{1}{}{nawierzchnia ulicy, placu lub chodnika utwardzona za pomocą układania na niej warstwy przylegających do siebie ściśle kamieni, kostek kamiennych, betonowych lub ceramicznych}
\Clue{2}{}{członek grupy etnicznej Kuszytów, zamieszkujących wschodnią Afrykę}
\Clue{3}{}{rodzaj plecionki wytworzonej ze sznurka lub nici}
\Clue{4}{}{w logice, reguła przekształcania jednych formuł zdaniowych w inne formuły zdaniowe przyjmowana na gruncie rachunku zdań}
\Clue{5}{}{kolor, ubarwienie przypominające szylkret - tworzywo; najczęściej czerń lub brąz z rudymi lub złotymi plamami}
\Clue{6}{}{Gerygone inornata - gatunek ptaka z rodziny buszówkowatych (Acanthizidae)}
\Clue{7}{}{Sitona - rodzaj chrząszcza z rodziny ryjkowcowatych}
\Clue{8}{}{dawny instrument smyczkowy z grupy chordofonów, odmiana małej wiolonczeli}
\Clue{9}{}{coś, co przynosi cierpienie i czego nie można uniknąć}
\Clue{10}{}{u zwierząt forma biernej reakcji na atak lub zagrożenie atakiem powodująca całkowity bezruch atakowanego osobnika}
\Clue{12}{}{prowincja w płn. Maroku}
\Clue{13}{}{mężczyzna, który odbywa stosunki płciowe z kobietami jedynie w swoich wyobrażeniach}
\Clue{14}{}{człowiek, który nie stosuje się do zasad kultury ani higieny, nadużywa alkoholu, często przy tym z marginesu społecznego; menel, obdartus, pijaczyna, łajza, wandal}
\Clue{15}{}{Euphausia superba - gatunek skorupiaka, najbardziej znany spośród wszystkich gatunków kryla; długość ciała do 60-65 mm; ciężar do 2 gramów, żyje od 2 do 10 lat}
\Clue{16}{}{niewywoływanie konfliktów}
\Clue{17}{}{krajan-kobieta}
\Clue{18}{}{amerykański ptak z rodziny czubaczy}
\Clue{19}{}{Physcomitrella hamperi - gatunek mchu z rodziny skrętkowatych}
\Clue{20}{}{niemiecki matematyk, fizyk i astronom (1728-77); prace z matematyki fotometrii, refrakcji światła}
\Clue{21}{}{neofilolożka, znawczyni języka niemieckiego}
\Clue{22}{}{zabawa, polegająca na ślizganiu się po powierzchni lodu z rozbiegu}
\Clue{23}{}{rodzaj wieczka, pokrywki}
\Clue{24}{}{powieściopisarz hiszpański, ur. 1916r; „Rodzina Pascuala Duarte”, „Ul” - Nobel '89}
\Clue{25}{}{zdrobniale i pieszczotliwie o oku}
\Clue{26}{}{wysuwana płyta stanowiąca zamknięcie kanału spalinowego lub powietrznego}
\Clue{30}{}{ludzie, którzy są w parlamencie, pracują tam, znajdują się chwilowo w jego siedzibie itp}
\Clue{34}{}{skrót/symbol waluty afgani}
\Clue{35}{}{zespół aktorów, którzy występują w danym filmie, spektaklu, itp}
\Clue{36}{}{rzeźbiarz (1841-1913) rzeźby o tematyce portretowej i religijnej}
\Clue{38}{}{kosmopolityczny mech; tworzy płaskie, zbite darnie na pniach drzew i skałach}\end{PuzzleClues}\newpage%\section*{Krzyżówka 3}

\noindent\begin{Puzzle}{18}{27}|*	|*	|*	|*	|[1][S]\drarr	|p	|s	|y	|c	|h	|u	|s	|z	|k	|a	|*	|*	|[2][S]\darr	|*	|.
|*	|*	|[3][S]\rarr	|m	|a	|c	|h	|*	|*	|*	|*	|*	|*	|*	|[4][S]\darr	|[5][S]\darr	|*	|c	|[6][S]\darr	|.
|[7][S]\rarr	|c	|h	|i	|p	|p	|e	|r	|*	|*	|*	|*	|*	|*	|i	|p	|*	|m	|j	|.
|*	|[8][S]\darr	|*	|*	|o	|[9][S]\darr	|*	|[10][S]\drarr	|b	|o	|l	|a	|*	|[11][S]\rarr	|r	|o	|n	|*	|ę	|.
|[12][S]\drarr	|k	|r	|ó	|l	|i	|c	|z	|ę	|*	|*	|[13][S]\rarr	|s	|z	|a	|d	|r	|*	|c	|.
|b	|a	|*	|*	|o	|w	|*	|s	|[14][S]\drarr	|c	|z	|e	|s	|a	|n	|k	|a	|*	|z	|.
|i	|p	|[15][S]\darr	|*	|g	|a	|*	|u	|b	|[16][S]\drarr	|f	|r	|*	|[17][S]\darr	|k	|o	|*	|*	|m	|.
|c	|i	|p	|*	|i	|*	|*	|w	|r	|w	|*	|[18][S]\darr	|*	|t	|a	|w	|*	|[19][S]\darr	|i	|.
|i	|t	|e	|*	|a	|*	|*	|n	|ą	|i	|*	|a	|*	|a	|*	|a	|[20][S]\darr	|k	|e	|.
|e	|u	|r	|*	|*	|*	|[21][S]\darr	|i	|z	|e	|[22][S]\rarr	|p	|e	|r	|m	|*	|f	|u	|ń	|.
|[][,]{ }	|ł	|k	|[23][S]\rarr	|s	|u	|k	|a	|*	|l	|*	|l	|[24][S]\drarr	|l	|u	|p	|u	|s	|*	|.
|p	|a	|o	|*	|*	|*	|o	|*	|*	|o	|*	|i	|b	|i	|*	|*	|n	|a	|*	|.
|o	|*	|w	|[25][S]\rarr	|b	|u	|l	|w	|a	|r	|e	|k	|*	|c	|*	|*	|*	|c	|*	|.
|k	|[26][S]\rarr	|s	|t	|a	|r	|o	|s	|t	|a	|*	|a	|[27][S]\darr	|a	|*	|[28][S]\darr	|*	|z	|*	|.
|ł	|[29][S]\drarr	|k	|u	|r	|u	|r	|u	|*	|s	|*	|c	|m	|*	|[30][S]\darr	|c	|*	|[][,]{ }	|*	|.
|o	|p	|i	|[31][S]\rarr	|c	|h	|y	|b	|*	|o	|*	|j	|u	|*	|f	|h	|*	|ż	|*	|.
|n	|b	|*	|*	|*	|*	|s	|*	|*	|w	|*	|a	|s	|*	|e	|ł	|[32][S]\darr	|ó	|*	|.
|ó	|*	|*	|*	|*	|[33][S]\drarr	|t	|r	|i	|o	|*	|*	|s	|*	|r	|o	|j	|ł	|*	|.
|w	|[34][S]\rarr	|r	|a	|m	|p	|a	|*	|*	|ś	|*	|*	|a	|*	|t	|p	|e	|t	|*	|.
|*	|*	|[35][S]\drarr	|o	|s	|a	|*	|*	|*	|ć	|*	|*	|k	|*	|y	|i	|d	|o	|*	|.
|*	|[36][S]\rarr	|m	|a	|t	|r	|o	|n	|a	|*	|*	|*	|a	|*	|c	|ę	|n	|n	|*	|.
|[37][S]\drarr	|h	|e	|l	|i	|o	|g	|r	|a	|f	|i	|a	|*	|[38][S]\darr	|z	|c	|a	|o	|*	|.
|j	|[39][S]\rarr	|n	|e	|o	|l	|i	|t	|*	|*	|*	|*	|*	|o	|n	|o	|n	|g	|*	|.
|a	|[40][S]\rarr	|t	|a	|s	|i	|e	|m	|n	|i	|c	|a	|*	|k	|o	|ś	|e	|i	|*	|.
|g	|*	|y	|*	|[41][S]\rarr	|s	|c	|e	|n	|i	|c	|z	|n	|o	|ś	|ć	|*	|*	|*	|.
|ł	|[42][S]\rarr	|k	|a	|r	|t	|a	|*	|*	|*	|*	|*	|*	|*	|ć	|*	|*	|*	|*	|.
|a	|*	|*	|*	|*	|*	|[43][S]\rarr	|k	|o	|n	|w	|i	|k	|t	|*	|*	|*	|*	|*	|.
|*	|*	|*	|*	|*	|*	|*	|*	|*	|*	|*	|*	|*	|*	|*	|*	|*	|*	|*	|.\end{Puzzle}

\newpage

\begin{PuzzleClues}{\textbf{Poziome}\\}\Clue{1}{}{szpital psychiatryczny w ZSRR, miejsce, do którego trafiali nie tylko pacjenci chorzy psychicznie, ale też dysydenci, miejsce, które nie miało dobrej opinii, w którym raczej nie należało liczyć na skuteczną kurację, a które było miejscem wieloletniego odosobnienia dla osób uznawanych za element niepożądany}
\Clue{3}{}{pozaukładowa jednostka prędkości liniowej równa prędkości rozchodzenia się dźwięku w powietrzu}
\Clue{7}{}{rodzaj kija golfowego; odmiana puttera, ale z większym nachyleniem łopatki kija}
\Clue{10}{}{broń myśliwska składająca się z jednego lub paru rzemieni, używana do polowania przez Indian Ameryki Płd. BOLAS}
\Clue{11}{}{kod ISO 4217 leja rumuńskiego}
\Clue{12}{}{młode królika}
\Clue{13}{}{Adam (1890-1959) syn Jana, malarz, działał we Francji i USA, ilustracje do 'W pustyni i w puszczy'}
\Clue{14}{}{rodzaj taśmy z długich włókien wełnianych}
\Clue{16}{}{w chemii: symbol fransu}
\Clue{22}{}{kraina historyczna w północnej Rosji przeduralskiej, między Peczorą, Wyczegdą i Kamą a Uralem}
\Clue{23}{}{samica psa}
\Clue{24}{}{WILK}
\Clue{25}{}{niewielki bulwar}
\Clue{26}{}{urzędnik, namiestnik prowincji, tytuł istniał już od XIV w}
\Clue{29}{}{ropucha olbrzymia, aga, Rhinella marina - gatunek płaza z rodziny ropuchowatych, żyjący w Ameryce Środkowej i Południowej, introdukowany na licznych wyspach Oceanii i Karaibów}
\Clue{31}{}{w gwarze łowieckiej: szczecina na grzbiecie dzika}
\Clue{33}{}{zespół złożony z 3 instrumentalistów lub kompozycja na trzy różnorodne instrumenty}
\Clue{34}{}{pomost ułatwiający ładowanie towarów do samochodu lub wagonu kolejowego}
\Clue{35}{}{owad błonkoskrzydły z rodziny osowatych, o charakterystycznym, kontrastowym ubarwieniu w postaci żółtych pasów na czarnym odwłoku, osiągający rozmiary od bardzo małych (7 mm) do stosunkowo dużych (5cm)}
\Clue{36}{}{w starożytnym Rzymie: mężatka, która cieszyła się nieposzlakowaną opinią, obdarzana była powszechnym szacunkiem, stawiana jako wzór czystości obyczajów}
\Clue{37}{}{technika fotograficzna wynaleziona przez Josepha Nicéphore'a Niépce'a na początku lat 20. XIX wieku}
\Clue{39}{}{młodsza epoka kamienia, epoka kamienia gładzonego - ostatni okres epoki kamienia (poprzedzający epokę brązu)}
\Clue{40}{}{trawa morska - podwodna bylina o wstęgowatych liściach, surowiec tapicerski i opakunkowy}
\Clue{41}{}{to, że coś jest charakterystyczne dla występów na scenie}
\Clue{42}{}{rodzaj dokumentu, środek płatniczy}
\Clue{43}{}{przyklasztorna zwykle szkoła z internatem}\end{PuzzleClues}

\begin{PuzzleClues}{\textbf{Pionowe}\\}\Clue{1}{}{przemowa (wygłoszona lub napisana) zawierająca obronę lub pochwałę}
\Clue{2}{}{w chemii: symbol kiuru}
\Clue{4}{}{mieszkanka Iranu, kobieta pochodzenia irańskiego}
\Clue{5}{}{łow. brązowy pas na piersi kuropatwy}
\Clue{6}{}{Hordeolum - torbielowata infekcja powieki, ropień powodowany przez zakażenie gronkowcowe; usytuowany jest na brzegach powiek, gruczołów przyrzęsowych i tarczkowych}
\Clue{8}{}{grupa osób decydujących o nadaniu jakiegoś odznaczenia (orderu, tytułu lub nagrody)}
\Clue{9}{}{eurazjatycki gatunek wierzby, w Polsce pospolita, srebrzyste bazie, pędy na obręcze do beczek}
\Clue{10}{}{przenośnik ślizgowy}
\Clue{12}{}{oddawanie czci, kłanianie się}
\Clue{14}{}{ciemny, ciepły kolor, który można opisać jako ciemny pomarańczowy lub żółty, powstający także jako wynik mieszania mało intensywnych barwników: czerwonego z zielonym, pomarańczowego z niebieskim lub żółtego z fioletem (wówczas kolor ma status barwy złamanej lub martwej)}
\Clue{15}{}{kompozytor i pedagog (1901-1990); utwory symfoniczne, wokalne, balety; 'Swantewit'}
\Clue{16}{}{cecha człowieka, którego przodków da się przyporządkować do różnych ras etnicznych}
\Clue{17}{}{deska szerokości kilkunastu centymetrów i długości około 1,5 metra z wyciętymi dwoma szczelinami, w które wchodzą dwie deski przymocowane na ruchomym bolcu - całość pracuje jak duże nożyce; łamanie łodyg lnu (międlenie) tarlicą ma na celu oczyszczenie ich i uzyskanie długich włókien}
\Clue{18}{}{program użytkowy}
\Clue{19}{}{Crypturellus noctivagus - gatunek ptaka z rodziny kusaczy (Tinamidae)}
\Clue{20}{}{ubaw, zgrywa, chwilowa radość, wybryk, który dostarcza zabawy (wyraz z angielskiego, wymawiany z angielska)}
\Clue{21}{}{osoba, która zajmuje się nakładaniem kolorów na ilustracje komiksowe i inne}
\Clue{24}{}{w chemii: symbol boru}
\Clue{27}{}{popularne w kuchni greckiej zapiekane danie przygotowywane na bazie bakłażana, pomidorów oraz mielonego mięsa}
\Clue{28}{}{cecha tego, co chłopięce - typowe dla chłopca, takie jak u chłopca}
\Clue{29}{}{w chemii: symbol ołowiu}
\Clue{30}{}{cecha człowieka zwinnego, energicznego, ruchliwego}
\Clue{32}{}{poena concordiae, poena compositionis - w średniowiecznym prawie polskim opłata należna sądowi w przypadku zawarcia ugody po rozpoczęciu procesu sądowego}
\Clue{33}{}{Zygophyllum - rodzaj kilkuletniej rośliny krzewiastej z rodziny bodziszkowatych, występującej we wszystkich strefach klimatycznych}
\Clue{35}{}{krótki płaszcz z obszyciem zarzucany na dolman na lewe ramię, zapinany pod szyję, używany przez węgierskich huzarów i w wojsku polskim od czasów S. Batorego do XVIII w}
\Clue{37}{}{objaw jaglicy; grudka jagliczna zbudowana z limfocytów i plazmocytów}
\Clue{38}{}{wzrok w chwili patrzenia, wyrażający emocje; występuje w liczbie mnogiej}\end{PuzzleClues}\newpage%\section*{Krzyżówka 4}

\noindent\begin{Puzzle}{17}{33}|*	|*	|*	|*	|*	|*	|*	|*	|*	|*	|*	|[1][S]\darr	|*	|*	|*	|*	|*	|*	|.
|*	|*	|*	|*	|*	|*	|*	|*	|*	|*	|*	|f	|*	|*	|*	|*	|[2][S]\darr	|[3][S]\darr	|.
|*	|*	|*	|*	|[4][S]\darr	|*	|*	|*	|[5][S]\drarr	|b	|r	|u	|z	|d	|a	|*	|ś	|k	|.
|[6][S]\rarr	|w	|r	|ą	|b	|*	|[7][S]\drarr	|r	|w	|d	|*	|s	|*	|[8][S]\darr	|*	|*	|m	|a	|.
|*	|*	|[9][S]\darr	|[10][S]\darr	|e	|*	|d	|[11][S]\darr	|i	|[12][S]\darr	|*	|z	|*	|u	|*	|*	|i	|z	|.
|*	|[13][S]\darr	|ż	|h	|z	|[14][S]\darr	|z	|o	|l	|k	|*	|i	|*	|p	|*	|*	|g	|a	|.
|*	|d	|ó	|i	|s	|p	|i	|h	|c	|n	|*	|k	|*	|r	|*	|*	|o	|r	|.
|*	|w	|ł	|p	|z	|e	|o	|y	|z	|a	|[15][S]\darr	|i	|[16][S]\darr	|a	|*	|*	|w	|k	|.
|*	|u	|t	|e	|k	|r	|b	|d	|y	|j	|k	|*	|r	|w	|*	|[17][S]\darr	|n	|a	|.
|[18][S]\rarr	|k	|o	|r	|o	|ł	|a	|z	|[][,]{ }	|p	|a	|p	|u	|a	|s	|k	|i	|*	|.
|*	|ą	|d	|f	|d	|o	|c	|t	|d	|a	|n	|*	|b	|*	|*	|a	|c	|[19][S]\darr	|.
|*	|t	|z	|o	|o	|w	|z	|w	|ó	|*	|a	|*	|e	|[20][S]\darr	|*	|z	|a	|b	|.
|*	|[][,]{ }	|i	|k	|w	|c	|*	|o	|ł	|[21][S]\drarr	|r	|o	|l	|a	|d	|a	|*	|a	|.
|*	|s	|ó	|a	|o	|e	|*	|*	|*	|g	|e	|*	|*	|u	|*	|r	|*	|s	|.
|*	|f	|b	|l	|ś	|*	|[22][S]\drarr	|p	|ó	|ł	|k	|r	|y	|t	|e	|k	|*	|k	|.
|*	|e	|[][,]{ }	|n	|ć	|*	|z	|*	|[23][S]\darr	|ó	|*	|*	|*	|o	|*	|a	|*	|a	|.
|*	|r	|n	|a	|*	|*	|ł	|[24][S]\darr	|o	|w	|*	|*	|*	|m	|[25][S]\darr	|[][,]{ }	|[26][S]\darr	|k	|.
|*	|y	|i	|*	|*	|*	|*	|t	|ś	|k	|[27][S]\rarr	|p	|ł	|o	|t	|e	|k	|*	|.
|*	|c	|z	|[28][S]\rarr	|r	|u	|t	|y	|n	|a	|*	|[29][S]\darr	|*	|b	|r	|g	|a	|*	|.
|*	|z	|i	|*	|*	|*	|*	|s	|i	|*	|[30][S]\darr	|p	|[31][S]\darr	|i	|i	|i	|n	|*	|.
|*	|n	|n	|[32][S]\rarr	|k	|u	|l	|i	|k	|*	|s	|e	|d	|l	|u	|p	|g	|*	|.
|*	|y	|n	|*	|*	|*	|*	|ą	|*	|[33][S]\darr	|z	|ł	|e	|i	|m	|s	|u	|*	|.
|*	|*	|y	|[34][S]\darr	|[35][S]\drarr	|n	|o	|c	|*	|t	|a	|z	|r	|s	|f	|k	|r	|*	|.
|*	|*	|*	|p	|d	|*	|*	|z	|*	|a	|ł	|a	|y	|t	|a	|a	|z	|*	|.
|*	|*	|*	|o	|y	|*	|*	|ł	|[36][S]\darr	|r	|o	|t	|w	|a	|t	|*	|ą	|*	|.
|*	|*	|*	|k	|n	|[37][S]\rarr	|ł	|o	|p	|a	|t	|k	|a	|*	|o	|*	|t	|*	|.
|[38][S]\rarr	|a	|l	|l	|o	|c	|h	|t	|o	|n	|*	|a	|t	|*	|r	|*	|k	|*	|.
|*	|*	|*	|a	|d	|*	|*	|ó	|l	|*	|[39][S]\darr	|*	|y	|*	|*	|*	|o	|*	|.
|*	|*	|*	|t	|a	|*	|*	|w	|w	|*	|s	|[40][S]\rarr	|w	|e	|l	|s	|*	|*	|.
|*	|*	|*	|*	|*	|*	|[41][S]\rarr	|k	|i	|s	|z	|k	|a	|*	|*	|*	|*	|*	|.
|*	|[42][S]\rarr	|a	|s	|p	|i	|r	|a	|n	|t	|k	|a	|*	|*	|*	|*	|*	|*	|.
|[43][S]\rarr	|p	|l	|u	|d	|r	|y	|*	|i	|*	|ł	|*	|*	|*	|*	|*	|*	|*	|.
|*	|*	|[44][S]\rarr	|p	|a	|p	|r	|o	|t	|k	|o	|w	|a	|t	|e	|*	|*	|*	|.
|*	|[45][S]\rarr	|z	|a	|ł	|o	|g	|a	|*	|*	|*	|*	|*	|*	|*	|*	|*	|*	|.\end{Puzzle}

\newpage

\begin{PuzzleClues}{\textbf{Poziome}\\}\Clue{5}{}{bardzo głęboka zmarszczka}
\Clue{6}{}{odpowiedni o ukształtowanie wgłębienie na obwodzie tarczy wirnika turbiny lub sprężarki}
\Clue{7}{}{międzywojenny, turystyczno-szkolny, polski samolot}
\Clue{18}{}{Cormobates placens - gatunek ptaka z rodziny korołazów (Climacteridae) występujący we wschodniej Australii i Nowej Gwinei}
\Clue{21}{}{wędlina, która składa się z kawałków mięsa (często połączonych galaretą) uformowanych w kształt rolady, rodzaj rulonu (blok w przekroju okrągły, o stosunkowo dużej średnicy)}
\Clue{22}{}{dwukonny pojazd podróżna z budą nad tylną częścią nadwozia, używany w Polsce w XVII/XVIII w}
\Clue{27}{}{urządzenie do suszenia siana składające się z drążków lub drutów rozpiętych na końcach}
\Clue{28}{}{monotonia, powtarzalność zdarzeń}
\Clue{32}{}{kosmopolityczny ptak podmokłych terenów z mew-siewek, o długim łukowatym dziobie, w Polsce chroniony}
\Clue{35}{}{pora doby, w której Słońce znajduje się minimum 18o pod horyzontem}
\Clue{37}{}{tyle, ile się mieści na łopatce (małej łopacie)}
\Clue{38}{}{osoba niepochodząca z miejsca, w którym mieszka}
\Clue{40}{}{miasto w Austrii (Górna Austria) nad rzeką Traun, w pobliżu wydobycie gazu ziemnego}
\Clue{41}{}{zsiadłe mleko}
\Clue{42}{}{kandydat; osoba pretendująca do czegoś}
\Clue{43}{}{krótkie bufiaste spodnie z sukna lub aksamitu sięgające do połowy ud; w Polsce rozpowszechnione w XVII-XVIII w; każde spodnie sięgające powyżej kolan}
\Clue{44}{}{Polypodiaceae - rodzina roślin należąca do klasy paproci; zalicza się do niej ok. 1000 gatunków, zgrupowanych w 50 rodzajach; jedna z czternastu rodzin w obrębie rzędu paprotkowców (Polypodiales) wchodzącego w skład klasy paprocie (Polypodiopsida); wg Crescent Bloom należy do rzędu paprotnikowców}
\Clue{45}{}{zespół ludzi wykonujących wspólnie jakąś pracę, zwłaszcza zespół pracowników fabryki, zakładu}\end{PuzzleClues}

\begin{PuzzleClues}{\textbf{Pionowe}\\}\Clue{1}{}{miasto i port w Japonii na wyspie Hondo}
\Clue{2}{}{lekkie działo o długiej lufie, używane w XVI/XVIII w}
\Clue{3}{}{kazarka rdzawa, Tadorna ferruginea - gatunek dużego ptaka wodnego z rodziny kaczkowatych (Anatidae);  zamieszkuje wschodnią część basenu Morza Śródziemnego, Morza Czarnego, pas w Azji Środkowej po Amur oraz północnozachodnią Afrykę, po Wyżynę Abisyńską}
\Clue{4}{}{cecha czegoś, co przebiega bez szkód, bez wypadków, bez zdarzeń skutkujących zniszczeniami, uszkodzeniami, stratą}
\Clue{5}{}{wykopany dół służący jako pułapka na zwierzęta}
\Clue{7}{}{waleń o szczękach wyciągniętych w kształt dzioba poławiany dla mięsa i tłuszczu}
\Clue{8}{}{całokształt zabiegów stosowanych w produkcji roślinnej, obejmujących uprawę roli: nawożenie, siew i sadzenie roślin, pielęgnację, zbiór i przechowywanie plonów}
\Clue{9}{}{Syma torotoro  - gatunek leśnego ptaka z rzędu kraskowych (Coraciiformes), z rodziny zimorodkowatych (Alcedinidae), z podrodziny łowców (Halcyoninae)}
\Clue{10}{}{odległość do punktu, na który ustawiono ostrość w aparacie fotograficznym, aby uzyskać ciągnącą się do nieskończoności głębię ostrości (za tym punktem); hiperfokalna nie jest stała i zależy od ogniskowej i wartości przysłony}
\Clue{11}{}{coś ohydnego, obrzydliwego, coś, co jest wstrętne, odrzuca, budzi obrzydzenie}
\Clue{12}{}{osoby znajdujące się w knajpie - restauracji o mało zobowiązującej atmosferze, nie bardzo eleganckiej}
\Clue{13}{}{figura geometryczna, część sfery ograniczona dwiema płaszczyznami przechodzącymi przez jej środek}
\Clue{14}{}{rusałkowate, południcowate, południce, Nymphalidae - rodzina z grupy motyli dziennych; zawiera 5 700 gatunków rozpowszechnionych na całym świecie (w Polsce 75), m.in. rusałki, przeplatki, dostojki, mieniaki}
\Clue{15}{}{podgatunek kulczyka; samce żółtozielone, samice brunatno-szare; w hodowli domowej otrzymano wiele ras śpiewających i ozdobnych}
\Clue{16}{}{nazwa waluty obowiązującej w Rosji i na niektórych terenach, obecnie lub dawniej, zależnych od Rosji (dawniej także: ZSRR)}
\Clue{17}{}{gęsiówka egipska, gęś egipska, Alopochen aegyptiaca - gatunek dużego ptaka wodnego z rodziny kaczkowatych (Anatidae), zamieszkujący subsaharyjską Afrykę, dolinę Nilu oraz basen Morza Śródziemnego}
\Clue{19}{}{dowódca wojsk mongolskich w podbitym kraju}
\Clue{20}{}{człowiek, którego hobby jest uprawianie sportu samochodowego}
\Clue{21}{}{górna część szyny, po której toczą się koła pojazdu}
\Clue{22}{}{skrót/symbol złotego}
\Clue{23}{}{dwuchwytowy nóż z ostrzem prostym lub wygiętym}
\Clue{24}{}{banknot o nominale tysiąca złotych}
\Clue{25}{}{w starożytnym Rzymie: wódz odprawiający triumf}
\Clue{26}{}{młode kangura}
\Clue{29}{}{kaulerpa, Caulerpa - przedstawiciel międzyzwrotnikowych, morskich glonów z klasy watkowych (Ulvophyceae)}
\Clue{30}{}{potrawa sporządzana na zimno z pokrojonego śledzia oraz ugotowanych ziemniaków, jaj, ogórka kiszonego, cebuli, skwarek boczku wędzonego i przypraw z dodatkiem musztardy}
\Clue{31}{}{rodzaj instrumentu finansowego, który uzależniony jest od instrumentu bazowego; używane przeważnie w liczbie mnogiej}
\Clue{33}{}{sposób walki polegający na uderzeniu własnym samolotem, okrętem lub pojazdem w samolot, okręt lub pojazd nieprzyjaciela w celu jego uszkodzenia lub zniszczenia}
\Clue{34}{}{TENT, CELT; daszek z płótna nad pokładem statku}
\Clue{35}{}{katoda wtórna lampy elektronowej}
\Clue{36}{}{zmiękczony, czyli plastyfikowany i granulowany, polichlorek winylu}
\Clue{39}{}{nieorganiczny materiał najczęściej przezroczysty wytwarzany w hutach z krzemianów, schłodzony do stanu stałego bez krystalizacji, najszęściej używany do wyrobu naczyń, szyb, przedmiotów artystycznych}\end{PuzzleClues}\newpage%\section*{Krzyżówka 5}

\noindent\begin{Puzzle}{24}{23}|*	|*	|*	|*	|*	|*	|*	|*	|*	|*	|*	|*	|*	|*	|*	|*	|*	|*	|*	|[1][S]\drarr	|c	|u	|*	|*	|[2][S]\darr	|.
|*	|*	|[3][S]\darr	|*	|*	|*	|*	|*	|*	|*	|*	|[4][S]\drarr	|m	|a	|t	|c	|z	|y	|n	|o	|ś	|ć	|*	|*	|p	|.
|*	|*	|b	|*	|*	|*	|*	|[5][S]\rarr	|c	|z	|y	|s	|t	|a	|[][,]{ }	|a	|l	|e	|k	|s	|j	|a	|*	|*	|o	|.
|*	|*	|o	|*	|[6][S]\drarr	|r	|ę	|k	|a	|[][,]{ }	|b	|ł	|o	|g	|o	|s	|ł	|a	|w	|i	|ą	|c	|a	|*	|d	|.
|[7][S]\drarr	|c	|y	|n	|a	|m	|o	|n	|[][,]{ }	|b	|i	|a	|ł	|y	|*	|[8][S]\rarr	|t	|r	|z	|e	|p	|o	|t	|*	|o	|.
|p	|*	|f	|*	|b	|*	|[9][S]\rarr	|r	|e	|t	|a	|b	|u	|l	|u	|m	|*	|*	|*	|m	|[10][S]\darr	|[11][S]\darr	|*	|*	|b	|.
|a	|[12][S]\darr	|r	|[13][S]\darr	|b	|*	|[14][S]\rarr	|s	|ł	|o	|n	|e	|c	|z	|n	|i	|c	|a	|*	|n	|k	|l	|[15][S]\darr	|*	|r	|.
|r	|b	|i	|s	|e	|[16][S]\darr	|*	|[17][S]\darr	|*	|[18][S]\darr	|*	|u	|[19][S]\darr	|[20][S]\darr	|*	|*	|*	|[21][S]\darr	|*	|a	|l	|a	|l	|[22][S]\darr	|a	|.
|k	|a	|e	|t	|*	|s	|*	|d	|[23][S]\darr	|m	|[24][S]\darr	|s	|a	|m	|[25][S]\rarr	|m	|y	|k	|*	|s	|u	|p	|a	|p	|z	|.
|i	|t	|n	|r	|[26][S]\drarr	|k	|r	|y	|n	|i	|c	|z	|n	|i	|k	|[][,]{ }	|g	|i	|ę	|t	|k	|i	|*	|r	|i	|.
|e	|[][,]{ }	|d	|z	|n	|u	|[27][S]\darr	|s	|o	|r	|e	|*	|t	|l	|*	|*	|*	|n	|*	|k	|*	|l	|[28][S]\darr	|z	|e	|.
|t	|m	|*	|a	|o	|l	|s	|k	|g	|i	|r	|*	|y	|i	|*	|[29][S]\rarr	|r	|o	|c	|a	|i	|l	|l	|e	|*	|.
|*	|i	|[30][S]\darr	|ł	|w	|*	|i	|*	|a	|s	|a	|[31][S]\drarr	|f	|a	|l	|l	|*	|[][,]{ }	|*	|*	|*	|i	|r	|ł	|*	|.
|[32][S]\drarr	|c	|z	|k	|a	|ł	|o	|w	|*	|z	|m	|j	|u	|g	|*	|*	|[33][S]\rarr	|f	|a	|r	|a	|*	|d	|o	|*	|.
|m	|w	|r	|a	|k	|*	|s	|*	|*	|j	|*	|u	|t	|r	|*	|[34][S]\drarr	|k	|a	|r	|m	|i	|n	|*	|ż	|*	|.
|*	|a	|*	|*	|*	|*	|t	|[35][S]\rarr	|f	|a	|s	|t	|b	|a	|c	|k	|*	|m	|*	|[36][S]\darr	|[37][S]\darr	|*	|[38][S]\darr	|e	|*	|.
|*	|*	|*	|*	|*	|[39][S]\rarr	|r	|y	|j	|*	|*	|a	|o	|m	|*	|o	|*	|i	|[40][S]\darr	|k	|m	|*	|z	|ń	|*	|.
|*	|[41][S]\rarr	|z	|e	|t	|t	|a	|b	|a	|j	|t	|*	|l	|*	|*	|m	|*	|l	|a	|o	|e	|[42][S]\darr	|ą	|s	|*	|.
|[43][S]\rarr	|e	|f	|e	|k	|t	|[][,]{ }	|s	|k	|a	|l	|i	|*	|*	|*	|b	|*	|i	|d	|n	|l	|f	|b	|t	|*	|.
|[44][S]\rarr	|g	|n	|i	|e	|w	|k	|o	|w	|o	|*	|*	|[45][S]\rarr	|i	|n	|i	|c	|j	|a	|t	|o	|r	|*	|w	|*	|.
|*	|*	|*	|*	|[46][S]\rarr	|p	|r	|z	|y	|l	|ż	|e	|ń	|c	|e	|*	|*	|n	|m	|o	|d	|o	|*	|o	|*	|.
|[47][S]\rarr	|b	|r	|ą	|z	|o	|w	|y	|[][,]{ }	|m	|e	|d	|a	|l	|*	|*	|*	|e	|s	|*	|i	|n	|*	|*	|*	|.
|*	|*	|[48][S]\rarr	|t	|a	|p	|i	|r	|[][,]{ }	|i	|n	|d	|y	|j	|s	|k	|i	|*	|*	|*	|a	|t	|*	|*	|*	|.
|[49][S]\rarr	|p	|a	|l	|l	|a	|*	|*	|*	|[50][S]\rarr	|u	|t	|r	|z	|y	|m	|a	|n	|i	|e	|*	|*	|*	|*	|*	|.\end{Puzzle}

\newpage

\begin{PuzzleClues}{\textbf{Poziome}\\}\Clue{1}{}{w chemii: symbol miedzi}
\Clue{4}{}{to, że ktoś jest matką, pozostawanie czyjąś matką; macierzyństwo}
\Clue{5}{}{rodzaj aleksji, w którym dotknięta nią osoba zachowała zdolność pisania, ale nie potrafi odczytać słów, które napisała}
\Clue{6}{}{porażenie strony dłoniowej ręki takie, że zginają się tylko palce IV i V, przez co nie można zacisnąć dłoni w pięść}
\Clue{7}{}{przyprawa wytwarzana z wysuszonej kory korzybiela białego}
\Clue{8}{}{szybkie bicie serca}
\Clue{9}{}{wertykalna dekoracja ołtarza w kościele zazwyczaj zdobiona płaskorzeźbami i/lub malowana}
\Clue{14}{}{przedstawiciel pierwotniaków wodnych}
\Clue{25}{}{sprytne, przemyślne posunięcie, działanie}
\Clue{26}{}{Nitella flexilis - kosmopolityczny gatunek ramienicy z rodzaju krynicznik}
\Clue{29}{}{ornament naśladujący kształty małżowin i muszli, charakterystyczny dla rokoka}
\Clue{31}{}{kompozytor austriacki (1873-1925); przedstawiciel operetki wiedeńskiej}
\Clue{32}{}{pilot radziecki (1904-1938), dokonał przelotu nad biegunem północnym wraz z Bajdykowem i Bielakowem w 1937 r., oblatywacz}
\Clue{33}{}{kościół parafialny}
\Clue{34}{}{jasnoczerwony barwnik otrzymywany z koszenili, mający m. in. zastosowanie w malarstwie}
\Clue{35}{}{samochód o nadwoziu typu fastback}
\Clue{39}{}{usta - wargi i jama gębowa człowieka}
\Clue{41}{}{jednostka używana w informatyce oznaczająca (zgodnie z zaleceniami IEC) tryliard bajtów (1 ZB = 10E+21 B)}
\Clue{43}{}{miara, określająca zmiany produkcji relatywne do obniżenia kosztów produkcji i zwiększonego wykorzystania środków}
\Clue{44}{}{Gniewkowo - wieś}
\Clue{45}{}{czynnik (związek chemiczny lub zjawisko fizyczne) zapoczątkowujący lub przyspieszający reakcję chemiczną}
\Clue{46}{}{rząd małych owadów; drapieżne lub wysysają soki roślin, niektóre szkodniki, np. wciornastek}
\Clue{47}{}{medal przyznawany za trzecie miejsce na podium}
\Clue{48}{}{tapir malajski, tapir czaprakowy, tapir azjatycki, Tapirus indicus - ssak z rodziny tapirowatych; zamieszkuje Mjanmę, Indochiny oraz Sumatrę}
\Clue{49}{}{Schoenoplectus Palla - rodzaj roślin kłączowych z rodziny ciborowatych}
\Clue{50}{}{środki do życia}\end{PuzzleClues}

\begin{PuzzleClues}{\textbf{Pionowe}\\}\Clue{1}{}{przyjęcie z okazji czyichś osiemnastych urodzin}
\Clue{2}{}{metoda malarska polegająca na równomiernym pokrywaniu obrazu drobnymi punktami czystej farby}
\Clue{3}{}{określenie partnera, używane głównie w okresie nastoletnim; chłopak; sympatia}
\Clue{4}{}{uczeń, który ma słabe wyniki w nauce}
\Clue{6}{}{fizyk niemiecki (1840-1905); konstruktor przyrządów optycznych, rozwinął teorię mikroskopu}
\Clue{7}{}{posadzka ułożona z drewnianych klepek}
\Clue{10}{}{pilot szybowcowy, brązowy medalista Mistrzostw Świata w 1972 r}
\Clue{11}{}{LAPILLE; typ materiału piroklastycznego wyrzucanego podczas erupcji wulkanu; Wielkość lapilli wynosi od 2 mm do 64 mm. Mają one kształt zbliżony do bomb wulkanicznych. Zazwyczaj mają kształt kropli lub są kuliste, wrzecionowate i lekko skręcone przez ruch obrotowy w czasie spadania.}
\Clue{12}{}{uroczystość żydowska organizowana dla dziewczynki w 12. lub 13. roku jej życia związana z wejściem w okres dorosłości}
\Clue{13}{}{wskazówka, zwykle rodzaj pręcika, który wskazuje odczyt urządzenia pomiarowego}
\Clue{15}{}{w chemii: symbol lantanu}
\Clue{16}{}{jednoosobowa łódź półwyścigowa}
\Clue{17}{}{sprzęt lekkoatletyczny używany podczas rzutu dyskiem}
\Clue{18}{}{Mirischia - rodzaj dinozaura z grupy celurozaurów; żył w okresie wczesnej kredy na terenach Ameryki Południowej}
\Clue{19}{}{sposób gry w piłkę nożną polegający nie na dążeniu do zdobycia bramki, ale na dążeniu do tego, by przeciwnik nie zdobył bramki, przede wszystkim przez zaniechanie gry w ataku, wzmocnienie gry w obronie, przetrzymywanie piłki w grze pasywnej}
\Clue{20}{}{jednostka masy równa 10 kilogramom}
\Clue{21}{}{filmy, które mogą oglądać zarówno dorośli, jak i dzieci; najczęściej są to filmy przygodowe, animowane}
\Clue{22}{}{fakt bycia zwierzchnikiem, przełożonym}
\Clue{23}{}{kończyna dolna u człowieka}
\Clue{24}{}{eseista niemiecki (1915-72), szkice popularnonaukowe o dziejach dawnych cywilizacji; „Bogowie, groby i uczeni”}
\Clue{26}{}{geolog i paleontolog (1880-1940); badacz utworów kredowych w Polsce}
\Clue{27}{}{osoba płci żeńskiej, którą ktoś traktuje, jak siostrę, mimo braku więzów krwi}
\Clue{28}{}{kod ISO 4217 dolara liberyjskiego}
\Clue{30}{}{w chemii: symbol cyrkonu}
\Clue{31}{}{włókno produkowane z łodyg rośliny juty}
\Clue{32}{}{skrót, symbol jednostki - metra}
\Clue{34}{}{typ nadwozia samochodu osobowego o wydłużonej tylnej części}
\Clue{36}{}{rachunek założony w danym banku, wykaz pieniędzy złożonych w banku, wykaz należności własciciela rachunku, innych operacji wykonanych za pośrednictwem pieniędzy związanych z bankiem, rodzaj usługi bankowej}
\Clue{37}{}{istota, natura, charakter czegoś, który jest przepojony jakimś nastrojem}
\Clue{38}{}{występ w kole zębatym lub zębatce}
\Clue{40}{}{chemik amerykański (1889-1971); prace dotyczące katalitycznego uwodorniania}
\Clue{42}{}{zespół operacyjny wojsk składający się z kilku armii}\end{PuzzleClues}\newpage%\section*{Krzyżówka 6}

\noindent\begin{Puzzle}{24}{26}|*	|[1][S]\darr	|*	|*	|*	|*	|*	|*	|*	|*	|*	|*	|[2][S]\darr	|[3][S]\drarr	|r	|u	|r	|e	|c	|z	|n	|i	|k	|*	|[4][S]\darr	|.
|*	|u	|*	|*	|[5][S]\rarr	|k	|o	|m	|p	|a	|s	|[][,]{ }	|g	|e	|o	|l	|o	|g	|i	|c	|z	|n	|y	|*	|u	|.
|*	|ś	|*	|*	|[6][S]\rarr	|p	|o	|d	|e	|j	|r	|z	|e	|n	|i	|e	|*	|*	|*	|*	|[7][S]\darr	|*	|*	|*	|p	|.
|*	|c	|[8][S]\drarr	|d	|e	|m	|*	|*	|*	|*	|*	|*	|m	|z	|[9][S]\darr	|*	|*	|[10][S]\darr	|[11][S]\rarr	|b	|e	|l	|l	|*	|r	|.
|*	|i	|k	|[12][S]\drarr	|i	|b	|a	|d	|y	|t	|a	|*	|m	|y	|e	|*	|*	|p	|[13][S]\darr	|*	|u	|*	|[14][S]\darr	|*	|a	|.
|*	|e	|r	|m	|*	|*	|*	|[15][S]\darr	|*	|*	|*	|*	|e	|m	|l	|[16][S]\rarr	|p	|i	|l	|o	|t	|*	|p	|[17][S]\darr	|w	|.
|*	|[][,]{ }	|z	|i	|*	|*	|[18][S]\drarr	|k	|u	|l	|a	|*	|r	|[][,]{ }	|e	|[19][S]\darr	|*	|s	|e	|[20][S]\darr	|e	|*	|t	|k	|n	|.
|*	|n	|e	|e	|*	|*	|k	|l	|*	|*	|*	|*	|a	|p	|m	|ś	|*	|m	|n	|t	|r	|*	|a	|l	|i	|.
|*	|a	|w	|c	|[21][S]\darr	|*	|o	|i	|*	|*	|[22][S]\darr	|*	|*	|r	|e	|c	|*	|o	|t	|r	|p	|*	|s	|e	|e	|.
|*	|d	|u	|z	|k	|*	|ś	|n	|*	|*	|n	|[23][S]\drarr	|k	|o	|n	|i	|ś	|*	|o	|z	|a	|*	|z	|r	|n	|.
|*	|[][,]{ }	|s	|[][,]{ }	|o	|[24][S]\darr	|l	|k	|*	|[25][S]\darr	|a	|s	|*	|t	|t	|e	|[26][S]\darr	|*	|*	|m	|*	|[27][S]\darr	|n	|[][,]{ }	|i	|.
|*	|ł	|z	|p	|ń	|m	|a	|i	|*	|s	|r	|u	|*	|e	|*	|ż	|f	|[28][S]\darr	|[29][S]\darr	|i	|[30][S]\darr	|s	|i	|p	|e	|.
|*	|a	|k	|ó	|[][,]{ }	|i	|w	|e	|*	|t	|o	|m	|*	|o	|[31][S]\drarr	|k	|r	|o	|p	|e	|c	|z	|k	|a	|*	|.
|[32][S]\drarr	|b	|a	|ł	|a	|k	|i	|r	|i	|e	|w	|*	|[33][S]\darr	|l	|ś	|a	|e	|n	|r	|l	|z	|y	|[][,]{ }	|r	|*	|.
|k	|ą	|[][,]{ }	|t	|n	|r	|e	|*	|[34][S]\darr	|r	|i	|[35][S]\darr	|c	|i	|w	|[][,]{ }	|u	|a	|z	|[][,]{ }	|ą	|f	|j	|a	|*	|.
|o	|*	|c	|o	|g	|o	|c	|*	|p	|e	|s	|h	|e	|t	|i	|h	|d	|g	|e	|o	|s	|r	|a	|f	|*	|.
|n	|*	|u	|r	|l	|s	|*	|[36][S]\drarr	|r	|o	|t	|o	|d	|y	|n	|a	|*	|e	|w	|g	|t	|[][,]{ }	|w	|i	|*	|.
|w	|*	|d	|a	|o	|o	|*	|t	|ó	|t	|o	|n	|r	|c	|i	|m	|*	|r	|i	|r	|k	|p	|a	|l	|*	|.
|a	|*	|o	|r	|a	|c	|*	|a	|b	|y	|ś	|o	|ó	|z	|a	|i	|*	|[][,]{ }	|d	|o	|a	|ł	|j	|a	|*	|.
|l	|*	|w	|ę	|r	|z	|*	|l	|a	|p	|ć	|r	|w	|n	|k	|l	|*	|p	|y	|d	|[][,]{ }	|o	|s	|n	|*	|.
|i	|*	|n	|c	|a	|e	|*	|b	|*	|i	|*	|*	|k	|y	|*	|t	|[37][S]\darr	|e	|w	|o	|a	|t	|k	|y	|*	|.
|j	|*	|a	|z	|b	|w	|[38][S]\rarr	|o	|r	|a	|n	|i	|a	|*	|*	|o	|r	|r	|a	|w	|l	|o	|i	|*	|*	|.
|k	|*	|*	|n	|s	|k	|*	|t	|*	|*	|*	|*	|*	|*	|*	|n	|a	|s	|n	|y	|f	|w	|*	|*	|*	|.
|a	|*	|*	|y	|k	|a	|*	|*	|*	|[39][S]\rarr	|d	|o	|t	|h	|r	|a	|c	|k	|i	|*	|a	|y	|*	|*	|*	|.
|*	|*	|*	|*	|i	|*	|[40][S]\rarr	|c	|i	|a	|s	|n	|o	|ś	|ć	|*	|k	|i	|e	|*	|*	|*	|*	|*	|*	|.
|*	|*	|*	|*	|*	|*	|*	|*	|[41][S]\rarr	|c	|z	|a	|w	|y	|c	|z	|a	|*	|*	|*	|*	|*	|*	|*	|*	|.
|*	|*	|*	|*	|*	|*	|*	|*	|*	|*	|*	|*	|*	|*	|*	|*	|*	|*	|*	|*	|*	|*	|*	|*	|*	|.\end{Puzzle}

\newpage

\begin{PuzzleClues}{\textbf{Poziome}\\}\Clue{3}{}{TUBIFEKS}
\Clue{5}{}{przyrząd stosowany do podstawowych pomiarów podczas geologicznych prac w terenie, zawierający oprócz kompasu magnetycznego z celownikiem i podziałką, także klinometr, klizymetr i zestaw poziomic}
\Clue{6}{}{posądzenie kogoś o jakiś czyn}
\Clue{8}{}{w biologii: populacja lokalna, populacja genetyczna, populacja geograficzna - grupa osobników jednego gatunku zasiedlająca jednolity obszar}
\Clue{11}{}{angielski pedagog, kapelan wojskowy (1753-1832); wprowadził metodę wzajemnego nauczania}
\Clue{12}{}{członek muzułmańskiej sekty chardżytów utworzonej w VII w}
\Clue{16}{}{urządzenie do zdalnego sterowania jakimś innym urządzeniem lub zespołem urządzeń}
\Clue{18}{}{nabój wystrzeliwany z broni palnej}
\Clue{23}{}{pieszczotliwie o koniu}
\Clue{31}{}{zdrobniale o kropce - znaku diakrytycznym}
\Clue{32}{}{kompozytor rosyjski (1837-1910); symfonie, utwory fortepianowe, poematy symfoniczne; 'Tamara', 'Rus'}
\Clue{36}{}{WIROLOT}
\Clue{38}{}{prowincja w środkowej części R.P.A, powierzchnia 129 tyś. km2, ośrodek administracyjny Bloemfontein}
\Clue{39}{}{język, którym posługują się członkowie ludu Dothraki z uniwersum cyklu powieściowegoPieść lodu i ognia oraz serialuGra o tron}
\Clue{40}{}{to, że coś - zwłaszcza ubranie lub but - jest ciasne, nie jest luźne}
\Clue{41}{}{łosoś pacyficzny o wadze do 50 kg; czerwone, wysoko cenione mięso}\end{PuzzleClues}

\begin{PuzzleClues}{\textbf{Pionowe}\\}\Clue{1}{}{miasto w północno-zachodnich Czechach, stolica kraju usteckiego, okręgu terytorialnego kraj północnoczeski i powiatu Uście nad Łabą}
\Clue{2}{}{prowincja w Nilfgaardzie, z książek Andrzeja Sapkowskiego z cykluWiedźmin}
\Clue{3}{}{proteaza - enzym z klasy hydrolaz katalizujący proteolizę oraz w większości przypadków hydrolizę wiązania estrowego}
\Clue{4}{}{określenie wyróżnionego w jakiś sposób elementu prawa podmiotowego - przeważnie chodzi o bliższe wyznaczenie zachowania objętego prawem podmiotowym}
\Clue{7}{}{Euterpe - palma z rodziny arekowatych}
\Clue{8}{}{Weigela Florida - krzew ozdobny o charakterystycznych, różowych i gęsto rosnących kwiatach}
\Clue{9}{}{żywioł, jeden z podstawowych składników świata materialnego}
\Clue{10}{}{umiejętność wyrażania myśli znakami graficznymi, pisanie}
\Clue{12}{}{miecz, którego długa rękojeść pozwala na walkę oburącz}
\Clue{13}{}{określenie wykonawcze: powoli}
\Clue{14}{}{Selenocosmia javanensis - gatunek pająka z rodziny ptasznikowatych}
\Clue{15}{}{mieszanina margli lub wapnia z gliną, półprodukt przy produkcji cementu portlandzkiego}
\Clue{17}{}{ogół duchownych pracujących w jakiejś parafii}
\Clue{18}{}{coś krzywego, koślawego}
\Clue{19}{}{ścieżka w grafie przebiegająca przez wszystkie jego wierzchołki dokładnie raz}
\Clue{20}{}{Bombus hortorum - owad z rodziny pszczołowatych}
\Clue{21}{}{angloarab - jedna z ras koni gorącokrwistych, pochodząca od konia angielskiego skrzyżowanego z koniem arabskim, hodowana głównie we Francji, Wielkiej Brytanii oraz w Polsce; cechuje ją gorący temperament i inteligencja, dzieki czemu odnosi ogromne sukcesy w sporcie jeździeckim}
\Clue{22}{}{cecha zwierzęcia jeździeckiego, które jest trudne w prowadzeniu, okiełznaniu, jest porywcze, samowolne, nieprzewidywalne; najczęściej jest to cecha konia}
\Clue{23}{}{Silurus glanis, sum europejski - gatunek ryby z rodziny sumowatych (Siluridae)}
\Clue{24}{}{gwiazda działająca jak mikrosoczewka w sytuacji, gdy dwie gwiazdy i Ziemia znajdą się prawie na jednej linii, bliższa gwiazda może ogniskować światło drugiej gwiazdy w taki sposób, że dla obserwatora na Ziemi jej blask wzrasta w charakterystyczny sposób}
\Clue{25}{}{ciągłe powtarzanie bezcelowych lub rytualnych ruchów, postaw ciała}
\Clue{26}{}{chemik (1835-92); prace z chemii organicznej}
\Clue{27}{}{rodzaj szyfru polegający na zapisaniu kolejnych liter tekstu jawnego na zmianę w ustalonej ilości rzędach}
\Clue{28}{}{Equus hemionus onager - podgatunek osła azjatyckiego z rodziny koniowatych; żyje w niewielkich stadach na stepach południowo-zachodniej Azji}
\Clue{29}{}{wytwór umysłu, antycypacja}
\Clue{30}{}{składa się z dwóch protonów i dwóch neutronów, ma ładunek dodatni i jest identyczna z jądrem atomu izotopu 4He, więc często oznacza się ją jako He2+}
\Clue{31}{}{świnia domowa}
\Clue{32}{}{roślina z rodziny szparagowatych}
\Clue{33}{}{północnoamerykański ptak z jemiołuszek}
\Clue{34}{}{badanie, doświadczenie, test służące weryfikacji hipotezy, dążeniu do zanegowania lub potwierdzenia czegoś}
\Clue{35}{}{poczucie godności osobistej, szacunku do siebie}
\Clue{36}{}{angielski archeolog, chemik i matematyk (1800-77); wynalazca talbotypu (technika fotograficzna negatywowo-pozytywowa)}
\Clue{37}{}{owca śruboroga, tucerna - jedna z ras owcy hodowanej głównie na Węgrzech, użytkowana trojako - dostarczała mięsa, mleka i wełny; pierwotnie występowała na stepach i obszarach trawiastych}\end{PuzzleClues}\newpage%\section*{Krzyżówka 7}

\noindent\begin{Puzzle}{19}{31}|*	|*	|*	|*	|*	|*	|*	|*	|*	|*	|*	|[1][S]\drarr	|t	|e	|r	|o	|f	|i	|t	|*	|.
|*	|*	|[2][S]\rarr	|o	|g	|n	|i	|w	|o	|[][,]{ }	|w	|z	|o	|r	|c	|o	|w	|e	|*	|*	|.
|*	|[3][S]\rarr	|d	|a	|w	|k	|a	|[][,]{ }	|p	|r	|o	|g	|o	|w	|a	|*	|*	|[4][S]\darr	|*	|*	|.
|*	|*	|*	|*	|*	|*	|*	|*	|[5][S]\drarr	|d	|r	|a	|k	|o	|n	|i	|d	|y	|*	|*	|.
|*	|[6][S]\drarr	|z	|o	|s	|t	|e	|r	|a	|[][,]{ }	|d	|r	|o	|b	|n	|a	|*	|o	|*	|*	|.
|*	|f	|*	|[7][S]\rarr	|s	|a	|m	|u	|r	|k	|a	|*	|*	|[8][S]\darr	|[9][S]\darr	|*	|[10][S]\darr	|r	|*	|*	|.
|*	|e	|*	|[11][S]\darr	|[12][S]\drarr	|m	|a	|n	|g	|a	|b	|a	|*	|m	|t	|*	|s	|k	|[13][S]\darr	|*	|.
|*	|c	|[14][S]\drarr	|p	|l	|u	|d	|r	|y	|*	|*	|*	|*	|e	|r	|*	|t	|*	|c	|*	|.
|*	|h	|b	|ł	|i	|[15][S]\darr	|*	|[16][S]\drarr	|r	|a	|w	|i	|o	|l	|i	|*	|r	|*	|y	|*	|.
|*	|m	|u	|o	|b	|p	|*	|h	|o	|*	|*	|*	|*	|k	|n	|*	|a	|*	|p	|*	|.
|*	|i	|r	|m	|r	|r	|*	|o	|z	|[17][S]\drarr	|n	|i	|ż	|*	|i	|*	|t	|[18][S]\darr	|r	|*	|.
|*	|s	|u	|y	|a	|e	|*	|m	|a	|c	|*	|*	|*	|*	|t	|[19][S]\darr	|y	|k	|y	|*	|.
|*	|t	|n	|k	|*	|z	|*	|a	|u	|h	|[20][S]\darr	|[21][S]\darr	|[22][S]\darr	|[23][S]\darr	|r	|ż	|[][,]{ }	|l	|s	|*	|.
|*	|r	|d	|*	|[24][S]\darr	|b	|*	|g	|r	|o	|z	|g	|t	|r	|o	|y	|m	|e	|o	|*	|.
|[25][S]\drarr	|z	|u	|g	|d	|i	|d	|i	|*	|r	|a	|r	|ę	|a	|f	|w	|o	|s	|w	|*	|.
|r	|*	|k	|[26][S]\darr	|ó	|t	|[27][S]\darr	|u	|*	|ą	|s	|z	|t	|k	|e	|o	|r	|z	|a	|*	|.
|o	|[28][S]\darr	|*	|k	|ł	|e	|p	|m	|*	|g	|ł	|e	|n	|[][,]{ }	|n	|t	|a	|c	|t	|*	|.
|s	|o	|*	|w	|*	|r	|i	|*	|[29][S]\darr	|i	|o	|b	|i	|s	|o	|n	|l	|z	|e	|*	|.
|s	|s	|[30][S]\drarr	|a	|m	|i	|n	|o	|p	|e	|n	|i	|c	|y	|l	|i	|n	|a	|*	|*	|.
|*	|t	|t	|s	|[31][S]\darr	|u	|g	|*	|r	|w	|a	|e	|a	|g	|*	|k	|e	|k	|*	|*	|.
|*	|o	|r	|[][,]{ }	|p	|m	|w	|*	|o	|[][,]{ }	|[][,]{ }	|ń	|[][,]{ }	|n	|*	|[][,]{ }	|*	|[][,]{ }	|*	|*	|.
|[32][S]\drarr	|j	|u	|m	|a	|*	|i	|*	|w	|s	|d	|[][,]{ }	|s	|a	|*	|k	|*	|a	|*	|*	|.
|a	|a	|t	|i	|s	|*	|n	|*	|a	|t	|y	|b	|z	|ł	|*	|o	|*	|z	|*	|*	|.
|b	|*	|e	|g	|k	|*	|[][,]{ }	|*	|n	|r	|m	|i	|c	|o	|[33][S]\darr	|r	|*	|j	|*	|*	|.
|e	|*	|ń	|d	|u	|*	|m	|*	|s	|z	|n	|o	|z	|w	|e	|e	|[34][S]\darr	|a	|*	|*	|.
|n	|*	|*	|a	|d	|*	|a	|*	|a	|e	|a	|d	|ę	|y	|v	|a	|d	|t	|*	|*	|.
|b	|*	|*	|ł	|z	|*	|ł	|*	|l	|l	|*	|r	|k	|*	|a	|ń	|e	|y	|*	|*	|.
|e	|*	|*	|o	|t	|*	|y	|*	|s	|c	|*	|o	|o	|*	|n	|s	|h	|c	|*	|*	|.
|r	|*	|*	|w	|w	|*	|*	|*	|k	|z	|*	|w	|w	|[35][S]\drarr	|s	|k	|o	|k	|*	|*	|.
|g	|*	|*	|y	|o	|*	|*	|*	|i	|a	|*	|y	|a	|c	|*	|i	|t	|i	|*	|*	|.
|*	|*	|*	|*	|*	|*	|*	|*	|*	|*	|*	|*	|*	|*	|*	|*	|i	|*	|*	|*	|.
|*	|*	|*	|*	|*	|*	|*	|*	|*	|*	|*	|*	|*	|*	|*	|*	|*	|*	|*	|*	|.\end{Puzzle}

\newpage

\begin{PuzzleClues}{\textbf{Poziome}\\}\Clue{1}{}{theros phyton - jedna z form życiowych roślin; oznacza roślinę przechodzącą cały cykl rozwojowy (od wykiełkowania z nasiona do wydania własnych nasion) w ciągu jednego okresu wegetacyjnego, później ginącą}
\Clue{2}{}{ogniwo o napięciu bliskim 1 V stosowane jako wzorzec jednostki miary napięcia elektrycznego}
\Clue{3}{}{graniczna ilość substancji, lub promieniowania jonizującego, wywołująca pierwsze dostrzegalne zmiany w organizmie}
\Clue{5}{}{Giacobinidy, - rój meteorów związany z orbitą komety Giacobiniego-Zinnera}
\Clue{6}{}{Zostera noltii - gatunek z rodziny zosterowatych}
\Clue{7}{}{zdrobniale: samura}
\Clue{12}{}{małpa z rodziny koczkodanów}
\Clue{14}{}{krótkie bufiaste spodnie z sukna lub aksamitu sięgające do połowy ud; w Polsce rozpowszechnione w XVII-XVIII w; każde spodnie sięgające powyżej kolan}
\Clue{16}{}{rodzaj włoskich pierożków}
\Clue{17}{}{niskie ciśnienie atmosferyczne}
\Clue{25}{}{miasto w zach. Gruzji na Nizinie Kolchidzkiej}
\Clue{30}{}{antybiotyk ß-laktamowy, pochodna penicyliny z jedną grupą aminową; modelowym przykładem jest ampicylina}
\Clue{32}{}{określenie kradzieży dokonanej w czasach PRL-u za zachodnią granicą Polski połączonej z przetransportowaniem łupu do Polski}
\Clue{35}{}{słabo umięśniona część kończyny dolnej ptaków, której szkielet tworzy silnie wydłużona kość skokowa}\end{PuzzleClues}

\begin{PuzzleClues}{\textbf{Pionowe}\\}\Clue{1}{}{strata metalu w czasie procesu technologicznego}
\Clue{4}{}{miasto w USA (Pensylwania) ośrodek handlowy regionu rolniczego}
\Clue{5}{}{Argyrosaurus - rodzaj zauropoda z grupy tytanozaurów; żył w epoce późnej kredy na terenach Ameryki Południowej}
\Clue{6}{}{nauczyciel fechtunku, szermierki}
\Clue{8}{}{miasto w Austrii (Dolna Austria) nad Dunajem}
\Clue{9}{}{organiczny związek chemiczny z grupy związków nitrowych i fenoli}
\Clue{10}{}{szkoda, która nie dotyczy bezpośrednio właności pkorzywdzonego, lecz jego komfortu psychicznego i fizycznego, samopoczucia bądź opinii}
\Clue{11}{}{FLOKS}
\Clue{12}{}{WAGA}
\Clue{13}{}{Cupressaceae - rodzina żywicznych drzew i krzewów nagonasiennych}
\Clue{14}{}{Tamias sibiricus - gatunek gryzonia z rodziny wiewiórkowatych; występuje w północnej Azji od centralnej Rosji do Chin, Korei i północnej Japonii (Hokkaido), introdukowany do parków centralnej i wschodniej Europy}
\Clue{15}{}{część świątyni mieszcząca główny ołtarz, oddzielona od nawy balustradą}
\Clue{16}{}{hołd składany przez duchowieństwo nowemu biskupowi}
\Clue{17}{}{w wojsku polskim XV-XVI wieku jeden z dwóch obok chorągwi kopijniczych rodzajów jazdy}
\Clue{18}{}{Anastomus oscitans - gatunek ptaka z rodziny bocianowatych (Ciconiidae)}
\Clue{19}{}{Thuja koraiensis - gatunek z rodziny cyprysowatych}
\Clue{20}{}{działanie przedsięwzięte w celu ukrycia prawdziwych intencji czy celów działającego}
\Clue{21}{}{struktura anatomiczna kości biodrowej, która rozpoczyna się kolcem biodrowym przednim górnym, a kończy kolcem biodrowym tylnym górnym}
\Clue{22}{}{główny pień tętniczy doprowadzający krew do twarzoczaszki}
\Clue{23}{}{Pacifastacus leniusculus - jeden z gatunków raków pochodzących z Ameryki Północnej, a obecnie występujących także w Europie, w tym w Polsce; został sprowadzony w latach sześćdziesiątych do Szwecji, a później do innych krajów europejskich, powodem jego introdukcji była odporność na dżumę raczą, która powodowała wtedy masowe śnięcia rodzimych gatunków raków}
\Clue{24}{}{w anatomii: zagłębienie w ciele człowieka lub zwierzęcia (np. dół pachowy)}
\Clue{25}{}{astronauta amerykański na pokładzie Atlantisa w 1991 r}
\Clue{26}{}{kwas z grupy alfa-hydroksylowych, który w naturze występuje w migdałach}
\Clue{27}{}{Eudyptula minor minor - nominatywny podgatunek ptaka, wyróżniony w obrębie gatunku pingwin mały (Eudyptula minor); zamieszkuje południową część Wyspy Południowej (Nowa Zelandia)}
\Clue{28}{}{miejsce stałego przebywania zwierzyny}
\Clue{29}{}{język, którego używa się lub używało się w Prowansji; najczęściej mówi się tak o językach oksytańskich, ale może to być też nazwa któregoś z prowansalskich dialektów francuskiego}
\Clue{30}{}{samiec pszczoły miodnej}
\Clue{31}{}{coś, co wzbudza obrzydzenie, wstręt poprzez inne cechy niż brzydota}
\Clue{32}{}{miasto w Niemczech, w Bawarii}
\Clue{33}{}{amerykański pianista i kompozytor (1912-1988); współpracownik Davisa}
\Clue{34}{}{(1911 -62), tadżycki pisarz i krytyk literacki}
\Clue{35}{}{symbol chemiczny węgla}\end{PuzzleClues}\newpage%\section*{Krzyżówka 8}

\noindent\begin{Puzzle}{21}{26}|*	|*	|*	|*	|[1][S]\darr	|*	|*	|[2][S]\darr	|*	|*	|*	|[3][S]\drarr	|j	|o	|h	|n	|s	|o	|n	|*	|*	|[4][S]\darr	|.
|*	|*	|*	|[5][S]\darr	|w	|*	|[6][S]\drarr	|o	|k	|r	|e	|s	|[][,]{ }	|w	|ę	|g	|l	|o	|w	|y	|*	|b	|.
|*	|*	|*	|o	|y	|*	|p	|b	|*	|*	|*	|u	|[7][S]\darr	|*	|*	|*	|[8][S]\drarr	|h	|g	|*	|[9][S]\darr	|i	|.
|*	|*	|[10][S]\darr	|k	|r	|*	|t	|i	|*	|*	|*	|l	|w	|[11][S]\rarr	|g	|h	|e	|e	|*	|*	|k	|o	|.
|*	|*	|k	|r	|ó	|*	|*	|*	|[12][S]\darr	|*	|*	|f	|e	|[13][S]\rarr	|k	|a	|s	|a	|k	|*	|ł	|g	|.
|*	|[14][S]\darr	|o	|e	|b	|[15][S]\rarr	|g	|a	|l	|o	|n	|i	|k	|*	|*	|*	|c	|[16][S]\darr	|*	|*	|u	|e	|.
|*	|w	|s	|s	|[][,]{ }	|*	|[17][S]\darr	|*	|e	|*	|*	|d	|t	|*	|*	|*	|h	|d	|*	|*	|s	|o	|.
|*	|a	|t	|[][,]{ }	|m	|[18][S]\drarr	|w	|e	|n	|e	|t	|*	|o	|*	|*	|*	|a	|o	|*	|*	|a	|c	|.
|[19][S]\drarr	|p	|r	|z	|e	|ż	|y	|c	|i	|e	|*	|*	|r	|*	|*	|[20][S]\darr	|t	|r	|*	|[21][S]\darr	|k	|e	|.
|n	|i	|z	|a	|d	|e	|g	|*	|u	|[22][S]\darr	|[23][S]\darr	|*	|[][,]{ }	|*	|*	|r	|o	|m	|*	|u	|[][,]{ }	|n	|.
|o	|e	|e	|s	|y	|g	|r	|*	|s	|b	|r	|*	|l	|[24][S]\darr	|*	|a	|l	|i	|[25][S]\darr	|p	|r	|o	|.
|w	|ń	|w	|i	|c	|l	|y	|*	|z	|o	|a	|[26][S]\darr	|o	|w	|*	|k	|o	|t	|b	|r	|o	|z	|.
|i	|[][,]{ }	|a	|ł	|z	|a	|w	|*	|e	|n	|j	|s	|s	|e	|*	|o	|g	|o	|r	|a	|s	|a	|.
|n	|k	|[][,]{ }	|k	|n	|r	|*	|[27][S]\drarr	|k	|o	|c	|i	|o	|ł	|[][,]{ }	|w	|i	|r	|o	|w	|y	|*	|.
|y	|r	|n	|o	|y	|e	|*	|g	|*	|b	|a	|l	|w	|n	|*	|c	|a	|i	|d	|a	|j	|[28][S]\darr	|.
|[][,]{ }	|y	|i	|w	|*	|k	|*	|n	|*	|o	|*	|n	|y	|a	|[29][S]\rarr	|e	|*	|u	|o	|*	|s	|k	|.
|b	|s	|e	|y	|*	|*	|*	|i	|*	|*	|*	|i	|*	|*	|*	|*	|*	|m	|w	|[30][S]\darr	|k	|i	|.
|r	|t	|d	|*	|*	|*	|*	|o	|[31][S]\rarr	|s	|e	|k	|s	|t	|a	|n	|s	|*	|s	|m	|i	|c	|.
|d	|a	|ź	|[32][S]\rarr	|p	|r	|ą	|t	|n	|i	|k	|[][,]{ }	|b	|a	|w	|a	|r	|s	|k	|i	|*	|z	|.
|o	|l	|w	|*	|*	|*	|*	|*	|*	|*	|[33][S]\rarr	|p	|r	|o	|m	|i	|e	|n	|i	|c	|a	|*	|.
|w	|i	|i	|[34][S]\rarr	|m	|e	|t	|a	|j	|ę	|z	|y	|k	|o	|w	|o	|ś	|ć	|*	|h	|*	|*	|.
|s	|c	|e	|*	|*	|*	|*	|*	|[35][S]\rarr	|k	|o	|ł	|o	|w	|r	|ó	|t	|*	|*	|o	|*	|*	|.
|k	|z	|d	|*	|*	|*	|*	|*	|*	|[36][S]\rarr	|g	|o	|p	|u	|r	|a	|*	|[37][S]\rarr	|g	|a	|z	|*	|.
|i	|n	|z	|[38][S]\rarr	|l	|e	|k	|[][,]{ }	|o	|d	|t	|w	|ó	|r	|c	|z	|y	|*	|*	|c	|*	|*	|.
|e	|y	|i	|*	|*	|*	|[39][S]\rarr	|v	|e	|v	|e	|y	|*	|[40][S]\rarr	|p	|e	|l	|i	|k	|a	|n	|*	|.
|*	|*	|a	|*	|*	|[41][S]\rarr	|o	|c	|e	|n	|a	|*	|*	|*	|*	|[42][S]\rarr	|p	|l	|a	|n	|*	|*	|.
|*	|*	|*	|*	|[43][S]\rarr	|k	|w	|a	|s	|[][,]{ }	|m	|i	|g	|d	|a	|ł	|o	|w	|y	|*	|*	|*	|.\end{Puzzle}

\newpage

\begin{PuzzleClues}{\textbf{Poziome}\\}\Clue{3}{}{ur. w 1921 r. astronom amerykański, współtwórca współczesnej fotometrii gwiazd}
\Clue{6}{}{piąty okres w paleozoiku}
\Clue{8}{}{w chemii: symbol rtęci}
\Clue{11}{}{rodzaj masła klarowanego; wywodzi się z Indii}
\Clue{13}{}{długa bluzka damska wkładana przez głowę, noszona na spódnicy}
\Clue{15}{}{zdrobniale o galonie amerynkańskim, jednostce objętości ciał}
\Clue{18}{}{przedstawiciel ludu iliryjskiego zamieszkującego północną część Italii, najprawdopodobniej przybyłego z terenów Europy Środkowej około 950 r. p.n.e. (czasami uznawanego za odrębną grupę indoeuropejską)}
\Clue{19}{}{znalezienie się - w sensie metaforycznym - w pewnym stanie, sytuacji; doświadczenie czegoś}
\Clue{27}{}{owalne zagłębienie o wygładzonych ścianach, wyżłobione w stropie lub ścianie korytarza jaskini, w miejscach zawirowań płynącej pod ciśnieniem wody krasowej, która wypełniała całą objętość korytarza jaskini}
\Clue{29}{}{liczba niewymierna, będąca podstawą logarytmu naturalnego; można ją definiować na kilka różnych sposobów}
\Clue{31}{}{kątomierz lusterkowy, optyczny przyrząd nawigacyjny, stosowany niegdyś w żeglarstwie i astronomii, służący do mierzenia wysokości ciał niebieskich nad horyzontem, a także kątów poziomych i pionowych pomiędzy obiektami widocznymi na Ziemi}
\Clue{32}{}{Bryum algovicum - gatunek mchu z rodziny prątnikowatych}
\Clue{33}{}{przewlekła choroba zakaźna ludzi i zwierząt (głównie bydła)}
\Clue{34}{}{w językoznawstwie: kategoria właściwa zjawiskom, które w sferze komunikacji mają charakter metajęzykowy}
\Clue{35}{}{w gimnastyce; ruch ciała około osi przyrządu, z podporu do podporu}
\Clue{36}{}{indyjska brama w kształcie wielokondygnacyjnej wieży}
\Clue{37}{}{pedał gazu}
\Clue{38}{}{lek, który ma taki sam skład i takie samo działanie, jak lek oryginalny}
\Clue{39}{}{miasto w Szwajcarii nad Jeziorem Genewskim (Vaud), ośrodek wypoczynkowo-turystyczny}
\Clue{40}{}{rybożerny ptak o potężnym dziobie ze skórzastym workiem i palcach spiętych błoną; zamieszkuje wody słodkie i słone strefy podzwrotnikowej i umiarkowanej}
\Clue{41}{}{decyzja, stwierdzenie, pokazujące znaczenie czegoś, stopień rozwoju, zgodność z jakimiś normami}
\Clue{42}{}{graficzna, schematyczna i umowna reprezentacja przestrzeni (np. rysunek), która odwzorowuje układ elementów w przestrzeni; mapa}
\Clue{43}{}{kwas z grupy alfa-hydroksylowych, który w naturze występuje w migdałach}\end{PuzzleClues}

\begin{PuzzleClues}{\textbf{Pionowe}\\}\Clue{1}{}{narzędzie, przyrząd, aparat, sprzęt, materiał lub inny artykuł wykorzystywany do diagnozowania, zapobiegania lub leczenia}
\Clue{2}{}{japoński, zwykle jedwabny pas przepasujący kimono}
\Clue{3}{}{tioeter - siarkoorganiczny związek chemiczny, siarkowy analog eterów o wzorze ogólnym R-S-R (R ? H)}
\Clue{4}{}{populacja roślin i zwierząt leśnych wraz z elementami przyrody nieożywionej, które stanowią środowisko abiotyczne danej wspólnoty życiowej}
\Clue{5}{}{czas, w którym pracownik pobiera zasiłek z tytułu niezdolności do pracy}
\Clue{6}{}{punkt typograficzny - miara wielkości czcionek i innych elementów typograficznych }
\Clue{7}{}{wektor, którego składowe są zmiennymi losowymi}
\Clue{8}{}{część doktryny religijnej dotycząca pośmiertnych losów człowieka, ludzkości i świata}
\Clue{9}{}{koń gorącokrwisty, który powstał ze skrzyżowania kłusaka orłowskiego z kłusakiem amerykańskim;  bardzo dobrze sprawdza się w wyścigach}
\Clue{10}{}{Festuca scoparia - gatunek trawy z rodziny wiechlinowatych}
\Clue{12}{}{człowiek, który jest leniwy, chętnie unika trudu lub po prostu odpoczywa; też: o człowieku, który za mało się stara}
\Clue{14}{}{czysty wapień, skała węglanowa, metamorficzna, drobno lub gruboblastyczna, niekiedy o wyraźnym uwarstwieniu, zbita, o różnorodnym zabarwieniu}
\Clue{16}{}{sypialnia zakonników w klasztorze, zwykle pod kapitularzem}
\Clue{17}{}{człowiek, który uchodzi za farciarza, ktoś fajny, taki, który zawsze dobrze wychodzi na swoich przedsięwzięciach}
\Clue{18}{}{paź żeglarz, witeź żeglarz, Iphiclides podalirius, Papilio podalirius - motyl dzienny zaliczany do rodziny paziowatych; ma ubarwienie skrzydeł jasnożółte z czarnymi, klinowato zwężającymi się pręgami}
\Clue{19}{}{wieś położona w Polsce, administracyjnie w województwie wielkopolskim, w powiecie kolskim, w gminie Babiak}
\Clue{20}{}{PANCERZOWCE}
\Clue{21}{}{rośliny uprawiane na danym terenie}
\Clue{22}{}{szympans karłowaty, Pan paniscus - gatunek małpy człekokształtnej z rodziny człowiekowatych (Hominidae); zamieszkuje jedynie lewe (południowe) dorzecze rzeki Kongo w Afryce Centralnej (Demokratyczna Republika Konga), jednakże poszczególne populacje są od siebie oddzielone}
\Clue{23}{}{staropolska nazwa członka rady miejskiej}
\Clue{24}{}{przędza, włóczka wełniana}
\Clue{25}{}{Canal; włoski malarz i rytownik (1697-1768) wuj i nauczyciel Belotta}
\Clue{26}{}{silnik spalinowy na paliwo stałe doprowadzane w postaci pyłu}
\Clue{27}{}{ubytek przekroju ciągnionego lub walcowanego materiału}
\Clue{28}{}{dzieło o niskiej wartości artystycznej i estetycznej}
\Clue{30}{}{stan w płd Meksyku nad Oceanem Spokojnym, powierzchnia 59,9 tyś. km2, stolica Morelia}\end{PuzzleClues}\newpage%\section*{Krzyżówka 10}

\noindent\begin{Puzzle}{18}{31}|*	|*	|[1][S]\drarr	|w	|i	|n	|o	|[][,]{ }	|o	|w	|o	|c	|o	|w	|e	|*	|*	|*	|*	|.
|*	|*	|z	|*	|*	|[2][S]\darr	|*	|[3][S]\drarr	|w	|s	|k	|a	|z	|ó	|w	|k	|a	|*	|*	|.
|*	|*	|e	|*	|*	|j	|[4][S]\rarr	|k	|ą	|t	|[][,]{ }	|p	|a	|d	|a	|n	|i	|a	|*	|.
|*	|*	|s	|*	|*	|a	|[5][S]\drarr	|ł	|u	|k	|[][,]{ }	|o	|p	|o	|r	|o	|w	|y	|*	|.
|*	|*	|p	|*	|*	|z	|z	|y	|*	|*	|*	|*	|*	|[6][S]\drarr	|k	|u	|m	|*	|[7][S]\darr	|.
|*	|*	|ó	|*	|*	|[][,]{ }	|i	|k	|*	|*	|[8][S]\rarr	|t	|o	|w	|o	|t	|*	|*	|i	|.
|*	|*	|ł	|*	|[9][S]\drarr	|r	|o	|c	|z	|e	|k	|*	|[10][S]\drarr	|s	|e	|k	|t	|*	|b	|.
|*	|*	|[][,]{ }	|[11][S]\drarr	|k	|u	|b	|i	|s	|z	|*	|*	|g	|t	|*	|*	|*	|*	|i	|.
|*	|*	|s	|e	|a	|c	|e	|n	|[12][S]\darr	|*	|*	|*	|d	|r	|*	|*	|[13][S]\darr	|*	|s	|.
|*	|*	|t	|s	|m	|h	|r	|a	|s	|*	|[14][S]\darr	|[15][S]\darr	|a	|z	|*	|*	|m	|*	|[][,]{ }	|.
|*	|[16][S]\darr	|e	|e	|u	|o	|*	|[][,]{ }	|i	|*	|s	|c	|ń	|ą	|*	|[17][S]\darr	|o	|*	|c	|.
|*	|a	|v	|s	|f	|m	|*	|k	|n	|*	|t	|h	|s	|s	|*	|t	|d	|*	|z	|.
|*	|l	|e	|o	|l	|y	|[18][S]\drarr	|o	|g	|ó	|r	|e	|k	|*	|*	|o	|u	|*	|a	|.
|*	|t	|n	|w	|e	|*	|z	|ń	|e	|*	|ó	|r	|a	|*	|*	|p	|l	|*	|r	|.
|*	|r	|s	|i	|t	|*	|a	|c	|r	|*	|j	|u	|[][,]{ }	|*	|*	|a	|a	|*	|n	|.
|*	|u	|e	|e	|*	|*	|z	|z	|*	|*	|*	|b	|w	|*	|[19][S]\darr	|z	|c	|*	|o	|.
|*	|i	|n	|c	|*	|*	|u	|y	|*	|*	|[20][S]\darr	|i	|ó	|[21][S]\darr	|g	|[][,]{ }	|j	|*	|p	|.
|*	|z	|a	|*	|*	|*	|l	|s	|*	|*	|m	|n	|d	|k	|ł	|d	|a	|[22][S]\darr	|i	|.
|*	|m	|[][S]-	|*	|*	|*	|e	|t	|*	|*	|i	|e	|k	|u	|o	|y	|[][,]{ }	|b	|ó	|.
|*	|[][,]{ }	|j	|*	|[23][S]\darr	|*	|ń	|a	|*	|*	|m	|k	|a	|b	|w	|m	|f	|r	|r	|.
|[24][S]\drarr	|k	|o	|ź	|l	|a	|k	|*	|*	|*	|e	|*	|*	|r	|i	|n	|a	|a	|y	|.
|z	|r	|h	|*	|i	|[25][S]\rarr	|a	|h	|i	|s	|t	|o	|r	|y	|c	|y	|z	|m	|*	|.
|b	|e	|n	|*	|m	|*	|*	|[26][S]\rarr	|w	|t	|y	|c	|z	|k	|a	|*	|y	|a	|*	|.
|a	|w	|s	|[27][S]\drarr	|f	|l	|o	|r	|e	|n	|c	|j	|a	|*	|*	|[28][S]\darr	|*	|*	|*	|.
|w	|n	|o	|w	|o	|*	|[29][S]\drarr	|s	|k	|r	|z	|e	|l	|o	|n	|o	|g	|i	|*	|.
|i	|i	|n	|a	|c	|*	|s	|[30][S]\darr	|[31][S]\drarr	|p	|n	|e	|u	|m	|o	|k	|o	|k	|*	|.
|e	|a	|a	|i	|y	|[32][S]\drarr	|m	|e	|t	|r	|o	|p	|o	|l	|i	|t	|a	|*	|*	|.
|n	|c	|*	|t	|t	|u	|a	|d	|a	|*	|ś	|*	|*	|*	|*	|e	|*	|*	|*	|.
|i	|z	|*	|a	|[][,]{ }	|d	|l	|e	|k	|*	|ć	|[33][S]\rarr	|y	|b	|i	|t	|*	|*	|*	|.
|e	|y	|*	|r	|b	|a	|e	|a	|t	|*	|*	|*	|*	|*	|*	|*	|*	|*	|*	|.
|*	|*	|*	|a	|*	|r	|c	|*	|*	|*	|*	|*	|*	|*	|*	|*	|*	|*	|*	|.
|*	|*	|*	|*	|*	|*	|*	|*	|*	|*	|*	|*	|*	|*	|*	|*	|*	|*	|*	|.\end{Puzzle}

\newpage

\begin{PuzzleClues}{\textbf{Poziome}\\}\Clue{1}{}{tani napój alkoholowy o dużej zawartości siarczyn i syntetycznych barwników}
\Clue{3}{}{element przyrządu pomiarowego analogowego (wskazówkowego), przesuwający się nad (lub obok) narysowanej skali}
\Clue{4}{}{kąt pomiędzy promieniem padającym na powierzchnię rozgraniczającej dwa ośrodki a normalną do tej powierzchni w punkcie padania promienia}
\Clue{5}{}{element konstrukcyjny wykorzystywany w sklepieniach w celu  przenoszenia na zewnątrz ciężaru własnego i dźwiganego}
\Clue{6}{}{ojciec chrzestny czyjegoś dziecka lub ojciec dziecka, które jest czyimś chrześniakiem}
\Clue{8}{}{smar maszynowy do łożysk i powierzchni ślizgowych}
\Clue{9}{}{uroczystość organizowana w dniu pierwszych urodzin dziecka}
\Clue{10}{}{wino musujące}
\Clue{11}{}{(1848-1929), poeta ludowy na Śląsku Cieszyńskim, nauczyciel; „Z niwy śląskiej”, „Pamiętnik starego nauczyciela”}
\Clue{18}{}{typ starego autobusu, busa lub większego samochodu, który kształtem przypomina dużego ogórka - warzywo}
\Clue{24}{}{rodzaj mocnego piwa dolnej fermentacji warzonego ze słodu jęczmiennego}
\Clue{25}{}{ahistoryzm - w naukach o kulturze: metoda badania zjawisk, polegająca na ujmowaniu ich w perspektywie ahistorycznej, tzn. z pominięciem historycznego kontekstu lub uwarunkowań}
\Clue{26}{}{dodatkowy moduł do programu komputerowego, który rozszerza możliwości wyjściowego produktu}
\Clue{27}{}{miasto w środkowych Włoszech, nad Arno, u stóp Apeninów, stolica Toskanii i prowincji Florencja, 366 tys. mieszkańców (2006)}
\Clue{29}{}{Branchiopoda - gromada skorupiaków słodkowodnych; ciało wyraźnie segmentowane - duża liczba segmentów; odnóża liczne, dwugałęziste, służą jednocześnie do poruszania się, oddychania i do pobierania pokarmu}
\Clue{31}{}{gram-dodatnia tlenowa bakteria należąca do paciorkowców alfa-hemolizujących}
\Clue{32}{}{wyższy dostojnik w Kościele katolickim i prawosławnym}
\Clue{33}{}{jednostka informacji oznaczająca 10\textasciicircum24 bitów}\end{PuzzleClues}

\begin{PuzzleClues}{\textbf{Pionowe}\\}\Clue{1}{}{forma rumienia wielopostaciowego, zmianami pierwotnymi są nietrwałe pęcherze pojawiające się na błonach śluzowych, głównie jamy ustnej i narządów płciowych, pęcherze pękając tworzą bolesne nadżerki, które utrudniają odżywianie się; chorobie towarzyszą niespecyficzne objawy ogólne, takie jak gorączka czy bóle stawowe}
\Clue{2}{}{rodzaj jazu charakteryzujący się tym, że w przeciwieństwie do jazów stałych, jazy ruchome wyposażone są zgodnie ze swą nazwą w elementy ruchome - zamknięcia}
\Clue{3}{}{grudka, brodawka skórna, powstająca w wyniku przerostu naskórka, zlokalizowana w okolicy odbytu, kanału odbytu i okolicach narządów płciowych}
\Clue{5}{}{piłkarz, napastnik ŁKS-u Łódź i reprezentacji grywał we Francji i Hiszpanii, ostatnio Amica Wronki}
\Clue{6}{}{nagła zmiana}
\Clue{7}{}{Threskiornis moluccus - gatunek dużego ptaka z rodziny ibisowatych (Threskiornithidae); zamieszkuje Australię, Nową Gwineę, Małe Wyspy Sundajskie i Moluki}
\Clue{9}{}{wybuch pod ziemią pocisku artyleryjskiego bez utworzenia leja}
\Clue{10}{}{likier z drobinami złota tradycyjnie produkowany w Gdańsku}
\Clue{11}{}{ESESMAN}
\Clue{12}{}{mechanik amerykański (1811-75); znacznie udoskonalił maszyny do szycia i rozpoczął ich seryjną produkcję}
\Clue{13}{}{kodowanie informacji w fali nośnej przez zmianę jej chwilowej fazy, w zależności od sygnału wejściowego}
\Clue{14}{}{ubranie posiadane przez kogoś, często noszone w danym zestawieniu}
\Clue{15}{}{dziecko ze skrzydełkami; aniołek (głównie barokowy, rzeźbiony)}
\Clue{16}{}{genetycznie uwarunkowany altruizm przejawiany w stosunku do osobnika spokrewnionego}
\Clue{17}{}{odmiana kwarcu o ciemnej barwie (od brązowej do czarnej) i specyficznym, dymnym połysku; określenie handlowe, niepoprawne w mineralogii}
\Clue{18}{}{o kobiecie miłej, kochanej}
\Clue{19}{}{górna część jakiegoś przedmiotu, kulistym kształtem przypominająca głowę}
\Clue{20}{}{naśladownictwo rzeczywistości, natury w sztuce, głównie w literaturze}
\Clue{21}{}{wieloosobowe pomieszczenie mieszkalne dla załogi na statku}
\Clue{22}{}{WROTA}
\Clue{23}{}{limfocyt odpowiedzialny za humoralną odpowiedź odpornościową, czyli wytwarzanie przeciwciał}
\Clue{24}{}{uchronienie, oszczędzenie, wybawienie}
\Clue{27}{}{miasto w Nowej Zelandii na Wyspie Północnej}
\Clue{28}{}{zgrupowanie ośmiu cząstek elementarnych}
\Clue{29}{}{tłuszcz wytopiony z mięsa (np. gęsiego czy słoniny) z dodatkiem przypraw oraz często skwarków czy cebuli; używany do smarowania pieczywa}
\Clue{30}{}{miasto w płd.-zach. Kamerunie nad rzeką Sanaca; huta aluminium, elektrownia wodna}
\Clue{31}{}{jednostka podziału metrycznego w utworze muzycznym lub w jego części}
\Clue{32}{}{miara szybkości zrywu w czasie}\end{PuzzleClues}\newpage%\section*{Krzyżówka 11}

\noindent\begin{Puzzle}{18}{31}|*	|*	|*	|*	|*	|*	|[1][S]\drarr	|s	|z	|o	|p	|*	|[2][S]\darr	|*	|[3][S]\drarr	|ż	|a	|l	|*	|.
|*	|*	|*	|*	|*	|*	|p	|*	|[4][S]\rarr	|p	|o	|n	|s	|*	|s	|*	|*	|*	|*	|.
|*	|*	|[5][S]\drarr	|p	|ó	|ł	|r	|o	|z	|k	|r	|o	|k	|*	|c	|*	|[6][S]\darr	|*	|*	|.
|*	|[7][S]\drarr	|o	|p	|e	|r	|a	|t	|o	|r	|[][,]{ }	|m	|o	|d	|a	|l	|n	|y	|*	|.
|*	|c	|l	|*	|*	|*	|ż	|*	|*	|[8][S]\darr	|*	|*	|r	|*	|r	|*	|e	|[9][S]\darr	|*	|.
|*	|h	|e	|*	|*	|*	|m	|*	|[10][S]\darr	|a	|[11][S]\darr	|*	|p	|*	|l	|[12][S]\darr	|o	|t	|*	|.
|*	|o	|s	|*	|*	|*	|o	|*	|m	|n	|w	|*	|u	|*	|a	|m	|n	|a	|[13][S]\darr	|.
|*	|i	|z	|*	|*	|*	|*	|*	|a	|a	|i	|*	|c	|*	|t	|u	|*	|e	|w	|.
|*	|n	|y	|*	|*	|*	|*	|*	|ł	|l	|e	|[14][S]\darr	|h	|*	|t	|z	|*	|l	|y	|.
|*	|k	|c	|*	|*	|*	|*	|[15][S]\drarr	|p	|i	|ż	|m	|o	|w	|i	|e	|c	|*	|p	|.
|*	|o	|e	|*	|*	|*	|*	|l	|k	|z	|y	|u	|w	|[16][S]\darr	|*	|u	|[17][S]\darr	|[18][S]\darr	|a	|.
|*	|w	|*	|*	|*	|*	|*	|u	|a	|a	|c	|r	|a	|p	|*	|m	|ł	|d	|r	|.
|*	|o	|[19][S]\drarr	|w	|y	|j	|e	|c	|*	|*	|a	|z	|t	|o	|*	|*	|e	|o	|z	|.
|*	|ś	|m	|[20][S]\rarr	|t	|u	|l	|e	|j	|a	|*	|a	|e	|d	|*	|[21][S]\darr	|m	|m	|a	|.
|*	|ć	|a	|*	|[22][S]\darr	|[23][S]\rarr	|u	|r	|m	|i	|a	|*	|*	|s	|*	|k	|k	|e	|c	|.
|*	|*	|m	|*	|e	|[24][S]\rarr	|r	|n	|*	|[25][S]\darr	|[26][S]\darr	|[27][S]\darr	|*	|a	|[28][S]\darr	|r	|o	|n	|z	|.
|[29][S]\drarr	|f	|u	|n	|k	|c	|j	|a	|[][,]{ }	|s	|c	|h	|o	|d	|k	|o	|w	|a	|*	|.
|r	|*	|t	|[30][S]\drarr	|s	|ą	|d	|*	|*	|t	|z	|a	|[31][S]\darr	|k	|o	|k	|s	|[][,]{ }	|*	|.
|o	|*	|[][,]{ }	|p	|k	|*	|*	|*	|*	|a	|e	|w	|b	|a	|s	|o	|z	|m	|*	|.
|p	|*	|k	|i	|l	|*	|*	|*	|*	|l	|p	|e	|a	|[][,]{ }	|z	|d	|c	|a	|*	|.
|u	|*	|o	|a	|u	|[32][S]\darr	|*	|*	|[33][S]\darr	|o	|i	|l	|c	|p	|t	|y	|z	|g	|*	|.
|c	|*	|l	|s	|z	|k	|*	|*	|m	|r	|g	|*	|h	|o	|[][,]{ }	|l	|y	|n	|*	|.
|h	|*	|u	|z	|y	|l	|*	|*	|i	|y	|a	|[34][S]\darr	|a	|c	|z	|[][,]{ }	|z	|e	|*	|.
|a	|*	|m	|c	|w	|i	|[35][S]\darr	|*	|n	|t	|*	|l	|n	|h	|m	|k	|n	|t	|*	|.
|[][,]{ }	|*	|b	|z	|i	|m	|o	|*	|ó	|*	|[36][S]\darr	|o	|a	|w	|i	|u	|a	|y	|*	|.
|d	|*	|i	|y	|z	|o	|g	|[37][S]\drarr	|g	|o	|t	|t	|l	|i	|e	|b	|*	|c	|*	|.
|ę	|*	|j	|s	|m	|n	|i	|l	|*	|*	|u	|n	|i	|a	|n	|a	|*	|z	|*	|.
|b	|*	|s	|t	|*	|t	|e	|u	|*	|*	|ł	|i	|a	|s	|n	|ń	|*	|n	|*	|.
|o	|[38][S]\rarr	|k	|o	|m	|ó	|r	|k	|a	|*	|a	|k	|*	|t	|y	|s	|*	|a	|*	|.
|w	|*	|i	|ś	|*	|w	|*	|a	|*	|*	|*	|*	|*	|a	|*	|k	|*	|*	|*	|.
|a	|*	|*	|ć	|*	|*	|*	|*	|*	|*	|*	|*	|*	|*	|*	|i	|*	|*	|*	|.
|*	|*	|*	|*	|[39][S]\rarr	|b	|o	|u	|r	|d	|i	|c	|h	|o	|n	|*	|*	|*	|*	|.\end{Puzzle}

\newpage

\begin{PuzzleClues}{\textbf{Poziome}\\}\Clue{1}{}{nadrzewny ssak drapieżny o cennym futrze}
\Clue{3}{}{słaba sytuacja, zwykle taka, w której coś się nie udało, taka, która może być powodem czyjegoś wstydu, taka, której lepiej nie komentować}
\Clue{4}{}{Jean, ur. w 1761 r. astronom francuski, odkrywca wielu komet, dyrektor obserwatorium we Florencji}
\Clue{5}{}{postawa stojąca, w której nogi rozstawione są na boki, mniej więcej na szerokość bioder}
\Clue{7}{}{symbol używany w różnego typu logikach modalnych, nadający swoim argumentom specjalne znaczenie}
\Clue{15}{}{ptak z rodziny kaczkowatych; Ameryka Płd, Afryka, Azja, Australia}
\Clue{19}{}{nazwa małpy z grupy szerokonosych, żyjącej w lasach Ameryki Południowej; małpa ta swoją nazwę zawdzięcza wydawaniu głosów przypominających ryki i przeciągłe wycia}
\Clue{20}{}{element konstrukcyjny w postaci krótkiego odcinka rury z osadzeniami wewnętrznymi lub zewnętrznymi}
\Clue{23}{}{REZAIJE-bezodpływowe jezioro w Iranie, na wysokości 1275 m powierzchnia 5,2 - 6 tyś. km2, głębokość do 15 m}
\Clue{24}{}{w chemii: symbol radonu}
\Clue{29}{}{funkcja, która jest stała na określonych przedziałach}
\Clue{30}{}{rozprawa, proces sądowy, także czynność polegająca na sądzeniu kogoś za jakieś przewinienia}
\Clue{37}{}{Leopold (1879-1934) malarz, rysownik i grafik; rysunki o tematyce żołnierskiej, akwaforty, litografie}
\Clue{38}{}{sześciokątny element pszczelego plastra}
\Clue{39}{}{malarz, członek Grupy Krakowskiej (1918-87) obrazy abstrakcyjno-surrealistyczne, wybitny kolorysta}\end{PuzzleClues}

\begin{PuzzleClues}{\textbf{Pionowe}\\}\Clue{1}{}{prażone lub pieczone ziarno, najczęściej żyta lub pszenicy}
\Clue{2}{}{Chelydridae - rodzina żółwi z podrzędu żółwi skrytoszyjnych; największe żółwie słodkowodne}
\Clue{3}{}{Alessandro (1660-1725); ojciec Domenica; kompozytor włoski; opery, oratoria, kantaty, msze, pasje, motety}
\Clue{5}{}{miasto w województwie podkarpackim, w powiecie lubaczowskim, położone na Płaskowyżu Tarnogrodzkim, nad rzeczką Przerwą, dopływem Lubaczówki}
\Clue{6}{}{potocznie, często z ironią lub żartobliwie: członek neokatechumenatu - katolickiej wspólnoty religijnej}
\Clue{7}{}{cecha czegoś, co przypomina choinkę, szczególnie choinkę bożonarodzeniową - czyli cecha czegoś ozdobnego, kolorowego, czasem tandetnego lub nadmiernie pstrokatego, co niekoniecznie pasuje do okoliczności i otoczenia}
\Clue{8}{}{opis zjawiska lub wyjaśnienie, będące wynikiem badań}
\Clue{9}{}{chińska jednostka wagi srebra, dawniej także jednostka monetarna}
\Clue{10}{}{mała poręczna butelka, w której mieści się ok. ćwierć litra alkoholu}
\Clue{11}{}{wysoka wieża}
\Clue{12}{}{instytucja powołana do gromadzenia, badania oraz opieki nad obiektami posiadającymi pewną wartość historyczną bądź artystyczną}
\Clue{13}{}{urządzenie do wyparzania czegoś - czyszczenia, dezynfekowania, sterylizowania za pomocą wysokiej temperatury}
\Clue{14}{}{tytuł będący odpowiednikiem grzecznościowegopan}
\Clue{15}{}{kanton w środku Szwajcarii, obszar 1,5 tyś. km2, stolica Lucerna}
\Clue{16}{}{gatunek grzybów należący do rodziny muchomorowatych (Amanitaceae)}
\Clue{17}{}{kultura Łemków}
\Clue{18}{}{spontanicznie namagnesowany obszar w ferromagnetykach lub ferrimagnetykach}
\Clue{19}{}{Mammuthus columbi - najlepiej poznany gatunek mamuta, wywodzący się od wcześniejszego mamuta południowego; zasiedlał obszary od południowej Kanady po Florydę, Meksyk, Nikaraguę i Honduras, był więc jedynym gatunkiem właściwych słoni, jaki wkroczył na obszary Ameryki Południowej}
\Clue{21}{}{Crocodylus rhombifer - gatunek gada z rodziny krokodylowatych, występujący tylko w dwóch miejscach na Kubie - na Bagnie Zapata na północnym zachodzie i na bagnach wyspy Isla de Juventud}
\Clue{22}{}{zjawisko polegające na tym, że coś nie jest ogólnodostępne}
\Clue{25}{}{francuski malarz, grafik i scenograf ur. w 1919 r.; ekspresyjne obrazy abstrakcyjne w ciemnej tonacji}
\Clue{26}{}{Colius - rodzaj ptaka z rodziny czepig}
\Clue{27}{}{kompozytor i dyrygent ur. w 1936 r., utwory orkiestralne, kameralne, wokalno-instrumentalne}
\Clue{28}{}{koszt zmienny przedsiębiorcy wykazujący reakcję na zmiany wolumenu działalności przedsiębiorstwa}
\Clue{29}{}{Anaxyrus quercicus syn. Bufo quercicus - gatunek płaza bezogonowego z rodziny ropuchowatych, występujący w Ameryce Północnej, od obszarów Północnej Karoliny do Florydy i Luizjany}
\Clue{30}{}{to, że coś jest piaszczyste - występuje tam dużo piasku}
\Clue{31}{}{rytuał, uroczystość na cześć boga Bachusa}
\Clue{32}{}{wschodnia, duża dzielnica Sosnowca}
\Clue{33}{}{ryba z gromady kręgoustych przypominająca węgorza}
\Clue{34}{}{inżynier specjalizacji lotniczej}
\Clue{35}{}{erotoman, kobieciarz; mężczyzna, który jest nie do wyczerpania w kwestii flirtów i podbojów seksualnych}
\Clue{36}{}{miasto obwodowe w europejskiej części Federacji Rosyjskiej w Podmoskiewskim Zagłębiu Węglowym; samowary}
\Clue{37}{}{puste miejsce, wyrwa, przerwa}\end{PuzzleClues}\newpage%\section*{Krzyżówka 12}

\noindent\begin{Puzzle}{23}{33}|*	|*	|*	|*	|*	|[1][S]\drarr	|l	|i	|c	|o	|w	|a	|n	|i	|e	|*	|[2][S]\darr	|*	|*	|[3][S]\drarr	|p	|u	|*	|*	|.
|*	|[4][S]\darr	|*	|*	|*	|t	|[5][S]\rarr	|a	|g	|l	|o	|m	|e	|r	|a	|c	|j	|a	|*	|k	|*	|*	|*	|[6][S]\darr	|.
|[7][S]\rarr	|b	|a	|l	|k	|o	|n	|*	|[8][S]\rarr	|l	|e	|g	|n	|i	|c	|z	|a	|n	|k	|a	|*	|*	|*	|ł	|.
|*	|l	|*	|[9][S]\rarr	|c	|y	|b	|u	|s	|z	|e	|k	|*	|[10][S]\rarr	|ł	|y	|s	|k	|*	|l	|[11][S]\darr	|*	|*	|a	|.
|*	|i	|*	|*	|*	|a	|[12][S]\rarr	|o	|d	|t	|l	|e	|n	|i	|a	|c	|z	|*	|*	|a	|t	|[13][S]\darr	|*	|b	|.
|*	|ż	|*	|*	|[14][S]\darr	|m	|*	|*	|[15][S]\rarr	|c	|i	|e	|r	|p	|i	|ą	|c	|y	|*	|t	|o	|m	|*	|ę	|.
|*	|s	|*	|[16][S]\drarr	|s	|a	|m	|o	|c	|h	|o	|d	|z	|i	|a	|r	|z	|*	|*	|e	|p	|a	|*	|d	|.
|*	|z	|*	|i	|a	|*	|*	|*	|*	|*	|*	|*	|*	|[17][S]\darr	|[18][S]\rarr	|b	|u	|l	|w	|a	|*	|s	|*	|ź	|.
|*	|o	|[19][S]\darr	|o	|m	|[20][S]\drarr	|a	|z	|o	|t	|o	|b	|a	|k	|t	|e	|r	|*	|*	|[][,]{ }	|*	|z	|*	|[][,]{ }	|.
|*	|ś	|a	|u	|o	|t	|*	|[21][S]\darr	|[22][S]\drarr	|z	|ł	|o	|t	|o	|l	|i	|k	|i	|*	|l	|*	|y	|*	|t	|.
|*	|ć	|t	|i	|l	|u	|*	|c	|t	|*	|*	|*	|[23][S]\darr	|ś	|*	|*	|a	|[24][S]\darr	|*	|a	|*	|n	|*	|r	|.
|*	|*	|m	|t	|o	|r	|*	|e	|e	|*	|*	|*	|k	|n	|*	|*	|[][,]{ }	|o	|*	|n	|*	|a	|*	|ą	|.
|*	|[25][S]\darr	|o	|o	|t	|b	|[26][S]\rarr	|n	|o	|r	|m	|a	|l	|i	|a	|*	|z	|b	|[27][S]\darr	|c	|*	|[][,]{ }	|*	|b	|.
|*	|f	|s	|s	|[][,]{ }	|a	|*	|a	|l	|*	|*	|*	|ę	|k	|*	|*	|[][,]{ }	|i	|k	|e	|[28][S]\darr	|s	|*	|i	|.
|*	|i	|f	|*	|w	|n	|*	|r	|o	|*	|*	|*	|p	|*	|[29][S]\darr	|*	|i	|e	|r	|t	|i	|a	|*	|ą	|.
|*	|l	|e	|[30][S]\darr	|ą	|*	|*	|*	|g	|[31][S]\rarr	|f	|r	|a	|*	|s	|*	|b	|k	|e	|o	|n	|m	|[32][S]\darr	|c	|.
|*	|o	|r	|p	|s	|*	|[33][S]\drarr	|p	|i	|r	|a	|n	|*	|*	|t	|*	|i	|t	|w	|l	|f	|o	|m	|y	|.
|*	|z	|a	|e	|k	|[34][S]\rarr	|z	|n	|a	|m	|i	|ę	|[][,]{ }	|n	|a	|c	|z	|y	|n	|i	|o	|w	|e	|*	|.
|*	|o	|[][,]{ }	|n	|o	|*	|b	|*	|[][,]{ }	|[35][S]\rarr	|l	|e	|h	|a	|r	|*	|y	|w	|i	|s	|r	|z	|k	|*	|.
|*	|f	|n	|e	|k	|*	|*	|[36][S]\darr	|f	|[37][S]\rarr	|p	|l	|u	|t	|o	|n	|*	|n	|a	|t	|m	|b	|s	|[38][S]\darr	|.
|*	|[][,]{ }	|o	|l	|a	|[39][S]\rarr	|a	|l	|u	|w	|i	|u	|m	|*	|s	|*	|*	|o	|k	|n	|a	|u	|y	|i	|.
|[40][S]\drarr	|p	|r	|o	|d	|u	|c	|e	|n	|t	|*	|*	|*	|*	|ą	|*	|*	|ś	|*	|a	|c	|d	|k	|s	|.
|s	|r	|m	|p	|ł	|*	|*	|w	|d	|*	|*	|[41][S]\rarr	|l	|i	|d	|o	|*	|ć	|*	|*	|j	|n	|a	|k	|.
|t	|z	|a	|a	|u	|[42][S]\drarr	|b	|i	|a	|ł	|o	|b	|r	|z	|e	|g	|i	|*	|*	|*	|a	|a	|n	|r	|.
|r	|y	|l	|[][,]{ }	|b	|s	|*	|n	|m	|*	|[43][S]\rarr	|t	|r	|z	|c	|i	|ń	|s	|k	|o	|*	|*	|k	|a	|.
|y	|r	|n	|c	|o	|t	|*	|*	|e	|*	|*	|[44][S]\darr	|*	|[45][S]\rarr	|z	|b	|r	|o	|j	|ó	|w	|k	|a	|*	|.
|j	|o	|a	|z	|w	|r	|*	|*	|n	|*	|*	|k	|[46][S]\rarr	|g	|a	|r	|l	|a	|n	|d	|*	|*	|*	|[47][S]\darr	|.
|o	|d	|*	|a	|y	|u	|*	|*	|t	|[48][S]\rarr	|w	|i	|e	|r	|n	|y	|*	|*	|*	|*	|*	|*	|*	|p	|.
|*	|y	|*	|r	|*	|g	|[49][S]\rarr	|ł	|a	|d	|o	|w	|a	|r	|k	|a	|*	|*	|*	|*	|*	|*	|*	|u	|.
|*	|*	|*	|n	|*	|*	|*	|*	|l	|*	|[50][S]\darr	|i	|[51][S]\rarr	|m	|a	|t	|u	|s	|z	|k	|a	|*	|*	|l	|.
|[52][S]\rarr	|c	|h	|a	|m	|p	|i	|o	|n	|a	|t	|*	|*	|*	|*	|*	|*	|*	|*	|*	|*	|*	|*	|a	|.
|*	|*	|*	|*	|[53][S]\rarr	|u	|s	|t	|a	|w	|o	|d	|a	|w	|c	|a	|[][,]{ }	|z	|w	|y	|k	|ł	|y	|*	|.
|[54][S]\rarr	|f	|e	|n	|e	|l	|o	|n	|*	|*	|r	|*	|*	|*	|*	|*	|*	|*	|*	|*	|*	|*	|*	|*	|.
|*	|*	|*	|*	|*	|*	|*	|*	|*	|*	|*	|*	|*	|*	|*	|*	|*	|*	|*	|*	|*	|*	|*	|*	|.\end{Puzzle}

\newpage

\begin{PuzzleClues}{\textbf{Poziome}\\}\Clue{1}{}{pokrywanie powierzchni ścian warstwą okładzinową tzw oblicówką}
\Clue{3}{}{D: poliuretan, PUR - polimer powstający w wyniku addycyjnej polimeryzacji  wielofunkcyjnych izocyjanianów do amin i alkoholi; w jego głównym łańcuchu występuje ugrupowanie uretanowe}
\Clue{5}{}{obszar o intensywnej zabudowie, zespół miast skupionych wokół największego miasta będącego centrum}
\Clue{7}{}{element architektoniczny, przestrzeń wystająca poza mury budynku, często ogrodzona balustradą}
\Clue{8}{}{mieszkanka Legnicy}
\Clue{9}{}{zdrobniale o cybuchu (części fajki)}
\Clue{10}{}{błysk, krótkie silne światło}
\Clue{12}{}{substancja chemiczna, której zadaniem jest ochrona metalu przed utlenianiem}
\Clue{15}{}{ten, który cierpi z jakiegoś powodu}
\Clue{16}{}{człowiek uprawiający sport samochodowy}
\Clue{18}{}{1) wieloletnia roślina ze złożonych; słonecznik bulwiasty, 2) zgrubiały wierzchołek pędu podziemnego z pączkami, z którego mogą rozwijać się młode pędy}
\Clue{20}{}{Gram-ujemna proteobakteria z rodzaju Azotobacter, czyli najczęściej reprezentująca jedną z dwóch rodzin w ramach rzędu Pseudomonadales; bakteria beztlenowa, która wiąże azot z powietrza}
\Clue{22}{}{rodzina błonkówek z grupy żądłówek, drapieżne, ubarwione}
\Clue{26}{}{drobne znormalizowane wyroby metalowe, np. śruby, wkręty, nity}
\Clue{31}{}{umowa, w ramach której dwaj kontrahenci ustalają wysokość stopy procentowej, która będzie obowiązywała w przyszłości dla określonej kwoty wyrażonej w walucie transakcji dla z góry ustalonego okresu}
\Clue{33}{}{miasto w Słowienii nad Morzem Adriatyckim na płw. Istria}
\Clue{34}{}{uszkodzenie lub zmiana w drobnych naczyniach krwionośnych, które powstaje najczęściej jako efekt występowania problemów hormonalnych, nadciśnienia tętniczego i problemów z krążeniem; może mieć charakter zarówno wrodzony, jak i nabyty}
\Clue{35}{}{kompozytor węgierski (1870-1948); przedstawiciel operetki wiedeńskiej 'Wesoła wdówka', 'Kraina uśmiechu'}
\Clue{37}{}{pierwiastek 94 układu okresowego pierwiastków}
\Clue{39}{}{osady powstające w procesie akumulacji na skutek działalności wód płynących}
\Clue{40}{}{w branży filmowej: główny inwestor produkcji filmowej zatrudniający wszystkie potrzebne osoby do jego produkcji i zapewniający sprzęt oraz środki materialne}
\Clue{41}{}{piaszczysty wał oddzielający lagunę od morza; mierzeja}
\Clue{42}{}{Białobrzegi - wieś}
\Clue{43}{}{wieś w Polsce położona w województwie dolnośląskim, w powiecie jeleniogórskim, w gminie Janowice Wielkie, na pograniczu Rudaw Janowickich i Kotliny Jeleniogórskiej w Sudetach Zachodnich}
\Clue{45}{}{Merganetta armata - gatunek ptaka z rodziny kaczkowatych (Anatidae), występujący w Andach; jedyny gatunek wyróżniony w obrębie plemienia  zbrojówek (Merganettini)}
\Clue{46}{}{polski niszczyciel z okresu II wojny światowej}
\Clue{48}{}{człowiek, który jest aktywnym wyznawcą jakiejś religii, członkiem wspólnoty wyznaniowej, uczestnikiem nabożeństw itp}
\Clue{49}{}{maszyna robocza służąca do ładowania czegoś, używana najczęściej na budowie lub w rolnictwie}
\Clue{51}{}{ojczyzna, kraj rodzinny}
\Clue{52}{}{turniej sportowy na wysokim poziomie zaawansowania}
\Clue{53}{}{podmiot władzy państwowej, który jest odpowiedzialny za ustanowienie ustawy innej niż konstytucyjna}
\Clue{54}{}{(1651-1715), francuski pisarz, pedagog i kaznodzieja, zwolennik kwietyzmu}\end{PuzzleClues}

\begin{PuzzleClues}{\textbf{Pionowe}\\}\Clue{1}{}{zatoka Morza Japońskiego u wybrzeży Honsiu, główny port Toyama}
\Clue{2}{}{Podarcis pityusensis - gatunek gada z rodziny jaszczurkowatych, endemit Balearów i pobliskich małych, przybrzeżnych wysp na wybrzeżu Hiszpanii}
\Clue{3}{}{Calathea lancifolia - gatunek rośliny z rodziny marantowatych}
\Clue{4}{}{bliższe pokrewieństwo (najczęściej w kontekście prawa do własności)}
\Clue{6}{}{Cygnus buccinator - gatunek dużego ptaka wodnego z rodziny kaczkowatych (Anatidae); żyje w Ameryce Północnej}
\Clue{11}{}{kod ISO 4217 pa'angi}
\Clue{13}{}{maszyna, która sama wytwarza swoje wzbudzenie}
\Clue{14}{}{samolot pasażerski lub transportowy o typowej szerokości kadłuba wynoszącej od 3 do 4 metrów}
\Clue{16}{}{miasto w Peru, port nad Amazonką}
\Clue{17}{}{HOACYN}
\Clue{19}{}{pozaukładowa jednostka miary ciśnienia, równa ciśnieniu 760 milimetrów słupa rtęci (mm Hg) w temperaturze 273,15 K (0 °C), przy normalnym przyspieszeniu ziemskim}
\Clue{20}{}{nakrycie głowy uformowane z długiego pasa lekkiej materii owiniętej wokół głowy noszone przez mężczyzn w krajach muzułmańskich i Indiach - ZAWÓJ}
\Clue{21}{}{muzyka do tańca cenar}
\Clue{22}{}{dział teologii chrześcijańskiej, zajmujący się zbieraniem i badaniem argumentów, potwierdzających zasadność wiary chrześcijańskiej}
\Clue{23}{}{samica łosia}
\Clue{24}{}{cecha kogoś, kto rozumuje obiektywnie, w ocenach i decyzjach kieruje się faktami, a nie sympatiami lub własnym interesem, uwzględnia różnorodne informacje}
\Clue{25}{}{myśliciel zajmujący się filozofią przyrody}
\Clue{27}{}{jeden z grupy organizmów, mających wspólnego przodka, połączonych więzami krwi}
\Clue{28}{}{termin interdyscyplinarny, definiowany różnie w różnych dziedzinach nauki; najogólniej - właściwość pewnych obiektów, relacja między elementami zbiorów pewnych obiektów, której istotą jest zmniejszanie niepewności (nieokreśloności)}
\Clue{29}{}{mieszkanka Starego Sącza}
\Clue{30}{}{Penelope jacquacu granti - podgatunek ptaka wyróżniony w obrębie gatunku penelopa zielonawa (Penelope jacquacu)}
\Clue{32}{}{mieszkanka Meksyku, kobieta pochodzenia meksykańskiego}
\Clue{33}{}{skrót od zettabitu; jednostka informacji oznaczająca 10\textasciicircum21 bitów}
\Clue{36}{}{ur. 1910r, poeta, tłumacz; „Słowa dla ludzi”, „U przyjaciół”}
\Clue{38}{}{(elektryczna) krótkotrwałe zjawisko jaskrawego świecenia charakteryzujące wyładowanie elektryczne}
\Clue{40}{}{wychodząca z powszechnego użycia nazwa relacji rodzinnej zachodzącej w stosunku do krewnego, który jest bratem ojca}
\Clue{42}{}{(garbarski) nóż oburęczny służący do strugania skór}
\Clue{44}{}{nielotny ptak wielkości kury domowej, zamieszkuje lasy Nowej Zelandii, chroniony, wymierający - nielot}
\Clue{47}{}{waluta Botswany}
\Clue{50}{}{Th - promieniotwórczy pierwiastek chemiczny z grupy aktynowców}\end{PuzzleClues}\newpage%\section*{Krzyżówka 13}

\noindent\begin{Puzzle}{24}{21}|*	|*	|*	|*	|*	|*	|[1][S]\darr	|*	|*	|*	|[2][S]\darr	|*	|[3][S]\darr	|[4][S]\darr	|*	|*	|[5][S]\drarr	|r	|a	|k	|i	|*	|[6][S]\darr	|*	|*	|.
|*	|*	|[7][S]\rarr	|s	|t	|o	|l	|i	|k	|*	|f	|*	|s	|i	|[8][S]\darr	|[9][S]\drarr	|o	|f	|i	|c	|j	|a	|ł	|*	|*	|.
|*	|*	|[10][S]\rarr	|k	|e	|r	|a	|t	|y	|n	|a	|*	|u	|m	|p	|b	|b	|*	|[11][S]\drarr	|t	|a	|b	|u	|n	|*	|.
|*	|[12][S]\rarr	|w	|a	|n	|i	|l	|i	|a	|*	|u	|*	|c	|i	|a	|l	|r	|[13][S]\drarr	|r	|f	|*	|*	|p	|*	|[14][S]\darr	|.
|*	|[15][S]\rarr	|c	|z	|a	|j	|k	|a	|*	|*	|n	|[16][S]\drarr	|h	|e	|r	|e	|z	|j	|a	|*	|*	|*	|e	|*	|l	|.
|*	|[17][S]\rarr	|m	|a	|r	|t	|a	|b	|a	|n	|*	|t	|y	|n	|a	|s	|ą	|a	|j	|*	|*	|*	|k	|*	|u	|.
|*	|*	|*	|*	|*	|[18][S]\rarr	|r	|z	|a	|z	|*	|h	|[][,]{ }	|n	|l	|k	|d	|s	|o	|*	|*	|*	|[][,]{ }	|*	|d	|.
|*	|*	|*	|[19][S]\rarr	|t	|u	|s	|z	|k	|a	|*	|b	|d	|i	|i	|o	|e	|k	|w	|*	|*	|*	|s	|*	|y	|.
|*	|*	|*	|[20][S]\drarr	|n	|e	|t	|b	|a	|l	|l	|*	|o	|c	|p	|t	|k	|ó	|a	|*	|*	|[21][S]\darr	|e	|*	|[][,]{ }	|.
|*	|*	|[22][S]\rarr	|s	|t	|e	|w	|a	|r	|t	|*	|*	|k	|t	|s	|k	|[][,]{ }	|ł	|t	|*	|*	|b	|r	|[23][S]\darr	|t	|.
|*	|*	|*	|p	|[24][S]\rarr	|m	|o	|r	|e	|n	|c	|i	|*	|w	|a	|a	|o	|e	|e	|[25][S]\darr	|*	|a	|y	|p	|u	|.
|[26][S]\rarr	|p	|o	|r	|a	|*	|*	|*	|*	|[27][S]\rarr	|s	|u	|m	|o	|*	|*	|r	|c	|*	|z	|*	|r	|c	|r	|r	|.
|*	|*	|*	|z	|*	|[28][S]\rarr	|p	|e	|w	|n	|o	|ś	|ć	|*	|*	|*	|m	|z	|*	|a	|*	|b	|y	|z	|a	|.
|*	|[29][S]\rarr	|l	|e	|m	|i	|e	|s	|z	|*	|[30][S]\rarr	|m	|e	|z	|a	|n	|i	|n	|*	|b	|*	|a	|t	|e	|ń	|.
|*	|*	|[31][S]\rarr	|c	|h	|o	|r	|o	|b	|o	|w	|o	|ś	|ć	|*	|*	|a	|i	|[32][S]\darr	|a	|*	|d	|o	|r	|s	|.
|*	|*	|[33][S]\rarr	|z	|ł	|a	|[][,]{ }	|w	|i	|a	|r	|a	|*	|*	|*	|*	|ń	|k	|w	|w	|*	|o	|w	|y	|k	|.
|*	|*	|[34][S]\rarr	|n	|a	|k	|o	|*	|[35][S]\rarr	|b	|i	|n	|c	|h	|o	|i	|s	|*	|a	|i	|*	|s	|y	|w	|i	|.
|[36][S]\drarr	|ś	|w	|i	|s	|t	|u	|ł	|a	|*	|*	|*	|[37][S]\rarr	|b	|l	|o	|k	|a	|d	|a	|*	|y	|*	|k	|e	|.
|d	|[38][S]\rarr	|t	|a	|r	|c	|z	|k	|a	|[][,]{ }	|s	|y	|g	|n	|a	|l	|i	|z	|a	|c	|y	|j	|n	|a	|*	|.
|y	|*	|*	|*	|*	|*	|[39][S]\rarr	|w	|a	|p	|i	|e	|n	|n	|i	|k	|*	|*	|*	|z	|*	|k	|*	|*	|*	|.
|l	|[40][S]\rarr	|f	|r	|a	|c	|h	|t	|[][,]{ }	|d	|y	|s	|t	|a	|n	|s	|o	|w	|y	|*	|*	|a	|*	|*	|*	|.
|*	|*	|[41][S]\rarr	|p	|o	|t	|e	|n	|c	|j	|a	|ł	|[][,]{ }	|i	|g	|l	|i	|c	|o	|w	|y	|*	|*	|*	|*	|.\end{Puzzle}

\newpage

\begin{PuzzleClues}{\textbf{Poziome}\\}\Clue{5}{}{wierszowany utwór literacki, który da się odczytać na dwa sposoby - od lewej do prawej i od prawej do lewej, a czasem również od początku do końca i od końca do początku}
\Clue{7}{}{grupa osób siedzących przy jednym stole}
\Clue{9}{}{urzędnik w zaborze austriackim}
\Clue{10}{}{nierozpuszczalne w wodzie białko fibrylarne wytwarzane przez keratynocyty}
\Clue{11}{}{gaz bojowy o działaniu paralityczno-drgawkowym}
\Clue{12}{}{aromatyczna przyprawa do ciast, lodów, likierów}
\Clue{13}{}{skrót/symbol franka rwandyjskiego}
\Clue{15}{}{gołąb należący do grupy gołębi barwnych}
\Clue{16}{}{wyrażenie, stwierdzenie, niezgodne z powszechnie uznanymi normami, ideami}
\Clue{17}{}{zatoka Morza Andamańskiego u wybrzeży Birmy, głębokość do 20 m, główny port Rangun}
\Clue{18}{}{ślad po przejściu piły przez materiał}
\Clue{19}{}{pozbawione głowy i wypatroszone ciało ryby}
\Clue{20}{}{gra zespołowa, uznawana za poprzednika koszykówki; rozgrywana w siedmoosobowych drużynach, które grając przeciwko sobie próbują zdobyć punkty umieszczając piłkę w koszu nieposiadającym tablicy}
\Clue{22}{}{astronauta amerykański; lotna pokładzie Challengera w 1984r}
\Clue{24}{}{miasto wydobycia rud miedzi w USA (Arizona)}
\Clue{26}{}{okres}
\Clue{27}{}{japoński sport narodowy znany od początku VIII wieku, który stylem walki przypomina zapasy}
\Clue{28}{}{cecha czegoś, w czym przejawia się to, że ktoś dobrze, stabilnie się czuje w obliczu czegoś}
\Clue{29}{}{płyta, za pomocą której ciągnik przemieszcza odspojone fragmenty ziemi}
\Clue{30}{}{półpiętro, antresola}
\Clue{31}{}{liczba chorych w danej chwili na konkretną chorobę na określoną liczbę ludzi}
\Clue{33}{}{w prawie: sytuacja, gdy ktoś wie (albo nie wie, choć powinien wiedzieć), że jakieś prawo mu nie przysługuje}
\Clue{34}{}{najczęściej końska lub cielęca skóra wyprawiona przez garbowanie chromowe, używana do wyrobu cholewek obuwia dziecięcego, torebek, pasków itp}
\Clue{35}{}{kompozytor franko-flamandzki (1400-1460); przedstawiciel szkoły burgundzkiej}
\Clue{36}{}{broń, z której podmuchem wyrzuca się drobne pociski}
\Clue{37}{}{wstawienie w skład drukarski dowolnych czcionek tam, gdzie tekst ma być nieczytelny lub ma znaleźć się z czasem inny zapis}
\Clue{38}{}{niewielka okrągła tarcza osadzona na trzonku, używana do kierowania ruchem pojazdów}
\Clue{39}{}{zakład, w którym ze skał wapiennych uzyskuje się wapno}
\Clue{40}{}{opłata za przewóz towarów obliczana według stosunku części podróży rzeczywiście przebytej przez ładunek do całej umówionej podrózy}
\Clue{41}{}{przejściowa zmiana potencjału błonowego komórki, związana z przekazywaniem informacji}\end{PuzzleClues}

\begin{PuzzleClues}{\textbf{Pionowe}\\}\Clue{1}{}{wydział uczelni teatralnej lub kierunek studiów zajmujący się lalkarstwem jako sztuką teatralną}
\Clue{2}{}{stroitalski bożek leśny, przedstawiany jako brodaty mężczyzna z koźlimi rogami i kopytami, utożsamiany z greckim satyrem}
\Clue{3}{}{rodzaj budowli hydrotechnicznej w porcie wodnym, najczęściej w stoczni; wąski basen portowy ze szczelnymi wrotami oraz urządzeniami wypompowującymi z jego wnętrza wodę}
\Clue{4}{}{onomastyka, dział językoznawstwa zajmujący się badaniem nazw własnych (zwanych też onimami), czyli odnoszących się do konkretnych przedmiotów, a nie ich klasy (por. gr. ónoma, oznaczającego imię)}
\Clue{5}{}{jeden z prawnie usankcjonowanych obrządków kościoła katolickiego wschodniego, praktykowany głównie w Armenii}
\Clue{6}{}{skała metamorficzna powstała z niewielkiego przeobrażenia łupka ilastego bądź mułowca w warunkach niskich temperatur (200-400 °C) i niezbyt wysokich ciśnień (bardzo niski i niski stopień metamorfizmu według Winklera)}
\Clue{8}{}{opuszczenie, pominięcie - figura retoryczna polegające na zwróceniu uwagi na obiekt poprzez pozorne pominięcie go}
\Clue{9}{}{rodzaj owada błonkoskrzydłego z rodziny bleskotkowatych}
\Clue{11}{}{płaszczkowate, płaszczki właściwe, Rajidae - rodzina ryb chrzęstnoszkieletowych z rzędu rajokształtnych (Rajiformes); występują w wodach oceanicznyvh całego świata, od zimnych do bardzo ciepłych; w ciepłych wodach są spotykane w głębszych, chłodniejszych warstwach}
\Clue{13}{}{glistnik jaskółcze ziele, Chelidonium majus - gatunek byliny z rodziny makowatych; nazwa regionalna używana na białostocczyźnie}
\Clue{14}{}{rzekoma grupa ludów, do których mieli się zaliczać Hunowie, Mongołowie czy Turcy; w XIX i XX wieku określano tym mianem ludy z rodzin i podrodzin językowych tureckiej, ugrofińskiej, a nawet drawidyjskiej (na południu Indii) oraz mówiące językiem japońskim i językiem koreańskim (przez współczesnych naukowców pojęcie uważane jest za przestarzałe)}
\Clue{16}{}{kod ISO 4217 waluty baht}
\Clue{20}{}{sprzeczność}
\Clue{21}{}{mieszkanka Barbadosu, kobieta pochodzenia barbadoskiego}
\Clue{23}{}{czynność wykonywana przy uprawie roślin, polegająca na usuwaniu roślin, przerzedzaniu ich}
\Clue{25}{}{osoba, której zadaniem jest zabawianie innych, np. publiczności przed występem wykonawcy}
\Clue{32}{}{uszkodzenie lub niedostatek wpływające negatywnie na ogólny stan czegoś poprzez obniżenie jego wartości}
\Clue{36}{}{długi i ciężki kloc, najczęściej drewniany, służący do podtrzymywania, mocowania}\end{PuzzleClues}\newpage%\section*{Krzyżówka 15}

\noindent\begin{Puzzle}{24}{31}|*	|*	|*	|*	|*	|*	|*	|*	|*	|*	|*	|*	|*	|*	|*	|*	|*	|*	|*	|*	|*	|*	|*	|[1][S]\darr	|*	|.
|*	|*	|*	|*	|*	|*	|*	|*	|*	|*	|*	|*	|*	|*	|*	|[2][S]\drarr	|s	|z	|p	|i	|c	|z	|a	|k	|*	|.
|*	|*	|*	|*	|*	|*	|*	|*	|[3][S]\rarr	|o	|r	|z	|e	|ł	|[][,]{ }	|s	|t	|e	|p	|o	|w	|y	|*	|a	|[4][S]\darr	|.
|[5][S]\rarr	|a	|n	|o	|m	|a	|l	|i	|a	|[][,]{ }	|m	|a	|g	|n	|e	|t	|y	|c	|z	|n	|a	|*	|*	|r	|m	|.
|[6][S]\drarr	|d	|z	|i	|e	|r	|z	|y	|k	|[][,]{ }	|s	|e	|n	|e	|g	|a	|l	|s	|k	|i	|*	|*	|*	|a	|a	|.
|f	|*	|*	|[7][S]\darr	|*	|[8][S]\darr	|[9][S]\rarr	|b	|ą	|k	|[][,]{ }	|a	|u	|s	|t	|r	|a	|l	|i	|j	|s	|k	|i	|*	|z	|.
|r	|[10][S]\darr	|*	|s	|[11][S]\rarr	|b	|r	|a	|d	|l	|e	|y	|*	|*	|[12][S]\rarr	|s	|z	|k	|a	|r	|a	|d	|a	|*	|a	|.
|e	|j	|*	|o	|[13][S]\rarr	|l	|i	|*	|*	|*	|*	|*	|[14][S]\rarr	|p	|r	|z	|e	|j	|e	|m	|c	|a	|*	|*	|j	|.
|g	|e	|*	|c	|*	|a	|*	|*	|[15][S]\rarr	|m	|a	|n	|d	|a	|r	|y	|ń	|s	|k	|i	|*	|*	|[16][S]\darr	|*	|a	|.
|a	|d	|*	|z	|*	|s	|[17][S]\drarr	|k	|o	|z	|i	|b	|r	|ó	|d	|[][,]{ }	|w	|s	|c	|h	|o	|d	|n	|i	|*	|.
|t	|n	|*	|e	|*	|z	|f	|*	|[18][S]\darr	|*	|*	|[19][S]\rarr	|k	|l	|u	|c	|z	|*	|*	|[20][S]\rarr	|d	|r	|u	|k	|*	|.
|a	|o	|*	|n	|*	|k	|o	|[21][S]\drarr	|b	|r	|z	|e	|s	|z	|c	|z	|e	|*	|[22][S]\rarr	|p	|a	|l	|m	|a	|*	|.
|[][,]{ }	|p	|*	|i	|*	|o	|n	|s	|u	|*	|[23][S]\darr	|*	|*	|[24][S]\darr	|[25][S]\darr	|ł	|*	|*	|*	|*	|[26][S]\drarr	|i	|e	|p	|*	|.
|ś	|r	|*	|c	|[27][S]\darr	|s	|o	|c	|r	|*	|s	|*	|[28][S]\darr	|t	|p	|o	|*	|*	|*	|*	|c	|*	|r	|[29][S]\darr	|*	|.
|r	|z	|*	|z	|p	|k	|g	|*	|a	|[30][S]\darr	|t	|[31][S]\darr	|p	|u	|i	|w	|*	|[32][S]\darr	|*	|*	|h	|[33][S]\darr	|y	|g	|*	|.
|e	|e	|*	|e	|a	|r	|r	|[34][S]\drarr	|k	|l	|e	|t	|e	|r	|k	|i	|*	|d	|*	|[35][S]\darr	|o	|g	|c	|l	|[36][S]\darr	|.
|d	|b	|[37][S]\darr	|k	|p	|z	|a	|k	|[][,]{ }	|e	|e	|e	|r	|k	|e	|e	|*	|e	|*	|r	|c	|ł	|z	|i	|t	|.
|n	|i	|k	|[][,]{ }	|i	|e	|f	|r	|s	|w	|n	|n	|l	|o	|*	|k	|*	|p	|*	|o	|h	|ó	|n	|n	|r	|.
|i	|e	|l	|a	|e	|l	|*	|o	|t	|a	|*	|r	|o	|t	|[38][S]\darr	|*	|*	|r	|*	|l	|o	|d	|o	|i	|z	|.
|a	|g	|a	|l	|r	|n	|[39][S]\drarr	|k	|o	|r	|w	|i	|n	|*	|r	|*	|*	|e	|*	|n	|ł	|[][,]{ }	|ś	|a	|e	|.
|*	|o	|w	|o	|o	|e	|n	|*	|ł	|*	|*	|*	|*	|*	|u	|*	|*	|c	|*	|i	|e	|n	|ć	|s	|p	|.
|[40][S]\drarr	|w	|i	|e	|ś	|[][,]{ }	|d	|r	|o	|b	|n	|o	|s	|z	|l	|a	|c	|h	|e	|c	|k	|a	|*	|t	|a	|.
|e	|o	|a	|s	|n	|w	|o	|[41][S]\drarr	|w	|a	|l	|a	|b	|i	|a	|[][,]{ }	|d	|a	|m	|a	|*	|r	|*	|o	|c	|.
|p	|ś	|t	|o	|i	|ł	|l	|s	|y	|[42][S]\darr	|*	|[43][S]\darr	|*	|[44][S]\darr	|d	|*	|*	|*	|*	|*	|*	|k	|*	|ś	|z	|.
|i	|ć	|u	|w	|c	|a	|a	|u	|*	|t	|[45][S]\darr	|t	|[46][S]\rarr	|k	|a	|m	|e	|r	|a	|l	|n	|o	|ś	|ć	|*	|.
|t	|*	|r	|a	|a	|ś	|*	|s	|*	|y	|z	|r	|*	|l	|*	|*	|[47][S]\rarr	|k	|o	|p	|y	|t	|o	|*	|*	|.
|e	|*	|k	|t	|*	|c	|*	|e	|*	|c	|w	|e	|*	|u	|*	|*	|[48][S]\rarr	|b	|e	|r	|r	|y	|*	|*	|*	|.
|t	|*	|a	|y	|*	|i	|*	|ł	|*	|z	|r	|s	|*	|c	|*	|[49][S]\rarr	|l	|u	|k	|r	|e	|c	|j	|a	|*	|.
|*	|*	|*	|*	|*	|w	|*	|*	|*	|k	|o	|e	|*	|h	|*	|*	|[50][S]\rarr	|p	|ł	|e	|s	|z	|k	|a	|*	|.
|[51][S]\rarr	|k	|u	|r	|i	|e	|r	|[][,]{ }	|t	|a	|t	|r	|z	|a	|ń	|s	|k	|i	|*	|[52][S]\rarr	|m	|n	|t	|*	|*	|.
|[53][S]\rarr	|k	|r	|ó	|l	|*	|[54][S]\rarr	|a	|f	|*	|*	|*	|*	|*	|*	|*	|*	|*	|[55][S]\rarr	|f	|r	|y	|z	|*	|*	|.
|*	|*	|[56][S]\rarr	|p	|l	|a	|n	|e	|t	|a	|[][,]{ }	|o	|c	|e	|a	|n	|i	|c	|z	|n	|a	|*	|*	|*	|*	|.\end{Puzzle}

\newpage

\begin{PuzzleClues}{\textbf{Poziome}\\}\Clue{2}{}{samiec jelenia z pierwszym porożem}
\Clue{3}{}{Aquila nipalensis nipalensis - nominatywny podgatunek ptaka wyróżniony w obrębie gatunku orzeł stepowy (Aquila nipalensis); występuje na obszarze od wschodniego Kazachstanu do północnych Chin}
\Clue{5}{}{obszar, w którym obserwuje się nasilenie magnetyzmu ziemskiego spowodowane występowaniem minerałów o właściwościach magnetycznych}
\Clue{6}{}{Laniarius nigerrimus - gatunek ptaka z rodziny dzierzbików (Malaconotidae), wcześniej klasyfikowanych jako podrodzina dzierzbowatych (Laniidae); występuje endemicznie w Somalii}
\Clue{9}{}{Botaurus poiciloptilus - gatunek ptaka z rodziny czaplowatych (Ardeidae)}
\Clue{11}{}{James, ur. w 1693r. astronom angielski, odkrył aberrację światła i mutację osi Ziemi}
\Clue{12}{}{ktoś szkaradny, bardzo nieładny lub coś szkaradnego, bardzo nieładnego; ktoś lub coś, co ma jakąś ujemną cechę w wyglądzie, ta cecha może być permanentna lub chwilowa, może też być dyskusyjna}
\Clue{13}{}{w chemii: symbol litu}
\Clue{14}{}{w łowiectwie - rodzaj ogara, potrafiącego wywęszyć świeży ślad zwierzyny}
\Clue{15}{}{oficjalny standard chińskiego języka mówionego w ChRL i język urzędowy tego państwa}
\Clue{17}{}{Tragopogon orientalis - gatunek roślin należący do rodziny astrowatych}
\Clue{19}{}{przedmiot służący do otwierania zamków lub kłódek}
\Clue{20}{}{wykonanie odbitek tekstu lub ilustracji przy użyciu formy drukarskiej}
\Clue{21}{}{miasto w południowej Polsce, w województwie małopolskim, w powiecie oświęcimskim, siedziba gminy miejsko-wiejskiej Brzeszcze}
\Clue{22}{}{drzewiasta roślina z klasy jednoliściennych, uprawiana zależnie od gatunku dla owoców, oleju, jadalnej mączki, soku, włókna bądź drewna}
\Clue{26}{}{kod ISO 4217 funta irlandzkiego}
\Clue{34}{}{obuwie do wysokogórskiej wspinaczki}
\Clue{39}{}{Szymanowska; siostra K. Szymanowskiego (1881-1938); śpiewaczka i pedagog}
\Clue{40}{}{wieś, w której szlachta o niskim statusie społecznym stanowi większość mieszkańców}
\Clue{41}{}{Macropus eugenii - gatunek torbacza z rodziny kangurowatych; zamieszkuje Australię Południową (okolice Cleve), południowo-zachodnią część Australii Zachodniej, wyspy Houtman Abrolhos, Garden, Wyspę Kangura, archipelag Recherche, introdukowany na wyspę Kawau i w region Rotorua}
\Clue{46}{}{mały rozmiar, np. pomieszczenia}
\Clue{47}{}{puszka rogowa u ssaków kopytnych}
\Clue{48}{}{prowincja historyczna we Francji, obecnie departament Cher i Indre}
\Clue{49}{}{wysuszony korzeń lukrecji, z którego pozyskiwany jest wyciąg zawierający słodką substancję, stosowany w przemyśle farmaceutycznym, kosmetycznym i cukierniczym}
\Clue{50}{}{maleńki metalicznie niebieski chrząszcz szkodnik roślin krzyżowych głównie kapusty, rzepaku}
\Clue{51}{}{osoba zajmująca się dostarczaniem tajnych informacji i dokumentów oraz przeprowadzaniem ludzi przez granicę w rejonie Tatr w okresie II wojny światowej}
\Clue{52}{}{kod ISO 4217 tugrika}
\Clue{53}{}{figura w kartach do gry, druga co do starszeństwa po asie (w niektórych grach, np. w tysiącu - trzecia z kolei po asie i dziesiątce), starsza od damy}
\Clue{54}{}{skrót/symbol waluty afgani}
\Clue{55}{}{deska z drewna drzewa liściastego używana do wyrobu boazerii i desek parkietowych}
\Clue{56}{}{hipotetyczny typ planety, której powierzchnia jest pokryta całkowicie wodami oceanu}\end{PuzzleClues}

\begin{PuzzleClues}{\textbf{Pionowe}\\}\Clue{1}{}{środek wychowawczy stosowany do dyscyplinowania dzieci}
\Clue{2}{}{człowiek, który nie jest już młody ani w średnim wieku}
\Clue{4}{}{kolorowy, niekształtny rysunek o abstrakcyjnych motywach}
\Clue{6}{}{Fregata minor - gatunek ptaka z rodziny fregatowatych (Fregatidae)}
\Clue{7}{}{Aloina aloides - gatunek mchu z rodziny płoniwowatych}
\Clue{8}{}{podgromada morskich i słodkowodnych małży o parzystych skrzelach, których nici są połączone mostkami}
\Clue{10}{}{podanie informacji urządzeniu tylko raz bez konieczności powtarzania}
\Clue{16}{}{liczebność; liczbowość}
\Clue{17}{}{prototyp gramofonu, przyrząd do mechanicznego zapisywania dźwięku}
\Clue{18}{}{duży, jadalny burak o intensywnie czerwonym korzeniu oraz czerwonawych liściach; uprawiany jest na barszcz i jako jarzyna}
\Clue{21}{}{w chemii: symbol skandu}
\Clue{23}{}{Antonio (1832-77) malarz chilijski, przedstawiciel romantyzmu}
\Clue{24}{}{gatunek domowego gołębia wydającego głos podobny do stłumionego warkotu bębna; lot ociężały}
\Clue{25}{}{podróżnik amerykański (1779-1813); badacz Missisipi, Meksyku, Teksasu}
\Clue{26}{}{niewielki  lub określany z pozytywnym nacechowaniem chochoł do ochrony roślin przed zimnem}
\Clue{27}{}{zazwyczaj ozdobne pudełko, etui, pojemnik, który służy do przechowywania papierosów, cygar}
\Clue{28}{}{włókno poliamidowe używane do wyrobu dzianin i tkanin bieliźnianych}
\Clue{29}{}{to, że coś jest gliniaste - ma taką konsystencję jak glina}
\Clue{30}{}{zakrzywiona rurka, która umożliwa przelanie cieczy z naczynia}
\Clue{31}{}{miasto we Włoszech (Umbria) w Apeninach nad rzeką Nerą: hutnictwo żelaza, przemysł maszynowy}
\Clue{32}{}{chandra, gorszy nastrój, chwilowy kiepski humor, smutek, przygnębienie}
\Clue{33}{}{zaburzenia związane z ciałem i psychiką, rejestrowane u osób po zaprzestaniu przyjmowania narkotyków}
\Clue{34}{}{posunięcie, postepek, czyn; coś, co ktoś robi, aby coś osiągnąć}
\Clue{35}{}{motyl nocny z rodziny sówek, larwy żerują nocą na roślinach zielnych; szkodnik}
\Clue{36}{}{osoba zajmująca się czyszczeniem lnu}
\Clue{37}{}{ludzkie uzębienie}
\Clue{38}{}{szybki pasaż w muzyce wokalnej}
\Clue{39}{}{miasto w płn. Zambii; ośrodek górnictwa i hutnictwa metali nieżelaznych w Pasie Miedzionośnym}
\Clue{40}{}{wyrażenie obraźliwe dla kogoś}
\Clue{41}{}{gryzoń z rodziny wiewiórkowatych}
\Clue{42}{}{skok o tyczce, konkurencja lekkoatletyczna, w której zawodnik pokonuje przeszkodę przy pomocy tyczki}
\Clue{43}{}{człowiek, który zawodowo tresuje, trenuje zwierzęta, np. w szkole dla psów, dawniej we dworze możnowładcy lub współcześnie w cyrku}
\Clue{44}{}{osoba gruba, niezdarna, niezgrabna}
\Clue{45}{}{oddanie, zwrócenie, spłata}\end{PuzzleClues}\newpage%\section*{Krzyżówka 16}

\noindent\begin{Puzzle}{22}{29}|*	|*	|[1][S]\drarr	|s	|z	|o	|s	|t	|a	|k	|o	|w	|i	|c	|z	|*	|*	|[2][S]\drarr	|m	|a	|ł	|y	|*	|.
|[3][S]\drarr	|s	|p	|r	|a	|w	|s	|t	|w	|o	|[][,]{ }	|k	|i	|e	|r	|o	|w	|n	|i	|c	|z	|e	|*	|.
|m	|*	|r	|[4][S]\darr	|*	|*	|*	|*	|*	|*	|*	|*	|*	|[5][S]\rarr	|k	|o	|m	|i	|z	|m	|*	|*	|*	|.
|a	|*	|a	|k	|*	|[6][S]\drarr	|a	|s	|t	|e	|r	|[][,]{ }	|g	|a	|w	|ę	|d	|k	|a	|*	|*	|*	|*	|.
|k	|[7][S]\drarr	|p	|ę	|d	|z	|l	|i	|c	|z	|e	|k	|[][,]{ }	|z	|i	|e	|l	|o	|n	|a	|w	|y	|*	|.
|a	|g	|t	|d	|*	|j	|[8][S]\darr	|[9][S]\darr	|[10][S]\darr	|*	|[11][S]\rarr	|w	|ę	|d	|z	|i	|d	|ł	|o	|*	|[12][S]\darr	|*	|*	|.
|k	|a	|a	|z	|*	|a	|p	|m	|b	|[13][S]\drarr	|z	|w	|i	|n	|g	|l	|i	|a	|n	|i	|z	|m	|*	|.
|[][,]{ }	|l	|k	|i	|[14][S]\rarr	|d	|o	|r	|y	|g	|n	|a	|t	|*	|*	|*	|*	|j	|[15][S]\darr	|*	|j	|*	|*	|.
|c	|i	|*	|e	|*	|l	|ż	|o	|l	|o	|[16][S]\darr	|*	|[17][S]\darr	|[18][S]\drarr	|d	|u	|p	|e	|r	|k	|a	|*	|*	|.
|z	|s	|*	|r	|*	|i	|o	|w	|i	|ł	|g	|*	|l	|b	|*	|*	|[19][S]\darr	|w	|ó	|*	|w	|*	|[20][S]\darr	|.
|u	|y	|*	|z	|*	|w	|g	|i	|c	|ą	|u	|[21][S]\drarr	|u	|s	|k	|o	|k	|*	|ż	|*	|i	|*	|s	|.
|b	|j	|*	|a	|*	|o	|a	|e	|a	|b	|i	|p	|n	|d	|*	|*	|o	|[22][S]\darr	|e	|[23][S]\darr	|s	|*	|k	|.
|a	|c	|*	|w	|*	|ś	|*	|c	|*	|e	|r	|e	|i	|*	|*	|*	|l	|m	|k	|k	|k	|[24][S]\darr	|r	|.
|t	|z	|[25][S]\darr	|k	|*	|ć	|*	|*	|[26][S]\rarr	|k	|a	|r	|t	|a	|*	|*	|b	|u	|*	|r	|o	|o	|z	|.
|y	|y	|k	|a	|*	|*	|*	|*	|*	|*	|*	|f	|*	|*	|*	|*	|a	|f	|*	|z	|[][,]{ }	|s	|e	|.
|*	|k	|i	|*	|[27][S]\rarr	|p	|e	|t	|r	|e	|l	|o	|w	|a	|t	|e	|*	|a	|[28][S]\darr	|y	|n	|a	|l	|.
|*	|*	|n	|[29][S]\drarr	|t	|a	|r	|n	|o	|g	|ó	|r	|z	|a	|n	|i	|n	|*	|o	|ż	|a	|d	|a	|.
|[30][S]\drarr	|d	|o	|m	|[][,]{ }	|g	|r	|y	|*	|*	|[31][S]\rarr	|a	|n	|d	|r	|o	|m	|e	|d	|y	|d	|y	|*	|.
|e	|*	|*	|o	|[32][S]\darr	|*	|*	|[33][S]\darr	|[34][S]\drarr	|d	|u	|c	|h	|*	|*	|*	|*	|*	|p	|k	|p	|[][,]{ }	|[35][S]\darr	|.
|l	|*	|[36][S]\drarr	|ż	|e	|g	|l	|a	|r	|z	|[][,]{ }	|j	|a	|c	|h	|t	|o	|w	|y	|*	|r	|ś	|p	|.
|e	|*	|g	|d	|s	|*	|*	|k	|o	|[37][S]\rarr	|m	|a	|t	|n	|i	|a	|*	|*	|l	|*	|z	|c	|i	|.
|k	|*	|a	|ż	|p	|*	|*	|s	|q	|*	|*	|*	|*	|*	|*	|*	|*	|*	|a	|[38][S]\darr	|y	|i	|l	|.
|t	|*	|z	|e	|a	|[39][S]\rarr	|m	|o	|u	|n	|d	|o	|u	|*	|*	|*	|*	|*	|k	|c	|r	|e	|n	|.
|r	|*	|ó	|ń	|d	|[40][S]\rarr	|z	|l	|e	|w	|o	|z	|m	|y	|w	|a	|k	|*	|*	|r	|o	|k	|i	|.
|o	|*	|w	|*	|o	|*	|*	|e	|f	|*	|*	|*	|*	|*	|*	|*	|*	|*	|*	|c	|d	|o	|c	|.
|n	|*	|k	|*	|n	|*	|[41][S]\rarr	|m	|o	|r	|f	|o	|n	|o	|l	|o	|g	|i	|a	|*	|z	|w	|z	|.
|i	|*	|a	|*	|*	|[42][S]\rarr	|k	|a	|r	|t	|k	|ó	|w	|k	|a	|*	|*	|*	|*	|*	|o	|e	|e	|.
|k	|*	|*	|*	|*	|*	|*	|*	|t	|*	|*	|*	|*	|*	|[43][S]\rarr	|d	|e	|l	|f	|i	|n	|*	|k	|.
|a	|*	|[44][S]\rarr	|o	|s	|a	|d	|a	|*	|*	|*	|*	|*	|*	|[45][S]\rarr	|p	|a	|r	|k	|i	|e	|t	|*	|.
|*	|*	|[46][S]\rarr	|p	|o	|w	|r	|ó	|z	|e	|k	|[][,]{ }	|n	|a	|s	|i	|e	|n	|n	|y	|*	|*	|*	|.\end{Puzzle}

\newpage

\begin{PuzzleClues}{\textbf{Poziome}\\}\Clue{1}{}{kompozytor radziecki (1906-1975); symfonie, koncerty, balety, utwory fortepianowe, opery; 'Nos', 'Katarzyna Izmajłowa'}
\Clue{2}{}{dziecko, chłopczyk, zazwyczaj w wieku mniej niż 6 lat}
\Clue{3}{}{forma przestępstwa polegająca na kierowaniu wykonaniem czynu zabronionego przez inną osobę lub osoby}
\Clue{5}{}{kategoria estetyczna, dotycząca zjawisk i dzieł, zdolnych wywoływać śmiech lub rozbawienie}
\Clue{6}{}{Aster amellus - gatunek rośliny należący do rodziny astrowatych; występuje w stanie dzikim w środkowej, południowej i wschodniej Europie oraz w niektórych częściach Azji (Kaukaz, Turcja, Czelabińsk, Kazachstan)}
\Clue{7}{}{Syntrichia virescens - gatunek mchu z rodziny płoniwowatych; w Polsce objęty ścisłą ochroną gatunkową}
\Clue{11}{}{element ogłowia wkładany koniowi do pyska w celu przekazywania mu sygnałów i kierowania nim}
\Clue{13}{}{nauka teologii protestanckiej odrzucająca kult świętych, rozumienie Eucharystii jako ofiary, celibat, zakony i władzę papieża}
\Clue{14}{}{Dorygnathus - pterozaur z rodziny Rhamphorhynchidae; zamieszkiwał dzisiejszą Europę we wczesnej jurze, około 190 milionów lat temu}
\Clue{18}{}{dupcia, laseczka, kobieta potrzegana jako obiekt seksualny}
\Clue{21}{}{urwisko; ostry, pionowy spadek terenu}
\Clue{26}{}{prostokątny kartonik oznaczony umownymi znakami, używany do gry lub do wróżenia}
\Clue{27}{}{burzykowate, Procellariidae - rodzina ptaków z rzędu rurkonosych; obejmuje gatunki oceaniczne zamieszkujące otwarte morza całego świata}
\Clue{29}{}{mieszkaniec Tarnowskich Gór}
\Clue{30}{}{miejsce, gdzie uprawia się gry hazardowe}
\Clue{31}{}{Bielidy}
\Clue{34}{}{w religii, mitologii: demon, siła nadprzyrodzona, która opiekowała się osobami, miejscami lub też mogła mieć wpływ (zły lub dobry) na życie człowieka}
\Clue{36}{}{osoba, która dysponuje patentem żeglarza jachtowego}
\Clue{37}{}{środkowa część sieciowych narzędzi rybackich, w której gromadzą się ryby}
\Clue{39}{}{miasto w płd. Czadzie, nad rzeką Logone, ośrodek handlowy regionu uprawy bawełny}
\Clue{40}{}{funkcjonalny odpowiednik umywalki (ujęcie wody) zamontowany w kuchni, służący głównie do mycia naczyń oraz produktów spożywczych i usuwania zbędnych płynów, pozostałych z gotowania potraw}
\Clue{41}{}{dyscyplina naukowa, dział morfologii badający wykorzystanie środków fonologicznych w systemie morfologicznym danego języka}
\Clue{42}{}{krótka forma sprawdzianu wiedzy, obowiązująca w różnych szkołach i praktycznie na każdym poziomie edukacji}
\Clue{43}{}{zwierzę morskie z rzędu wielorybów uzębionych; łatwo się oswaja i jest bardzo inteligentne}
\Clue{44}{}{rowek w tylnej części drzewca strzały dostosowany do grubości cięciwy w łuku}
\Clue{45}{}{konstrukcja z drewna, którą montuje się na odwrotnej stronie obrazu malowanego na desce, by zapobiec jej wypaczaniu}
\Clue{46}{}{struktura anatomiczna, o długości 15-20 cm, którą stanowią wszystkie twory wchodzące i wychodzące z moszny, a następnie przechodzące przez kanał pachwinowy}\end{PuzzleClues}

\begin{PuzzleClues}{\textbf{Pionowe}\\}\Clue{1}{}{Archaeopteryx - rodzaj późnojurajskego teropoda z rodziny Archeopteryksów; odkryte dotąd skamieniałości archeopteryksów pochodzą z okresu 150,8-145,5 milionów lat temu}
\Clue{2}{}{kosmonauta radziecki na pokładzie Wostoka 3 w 1962 r., pierwszy zespołowy lot z Popowem}
\Clue{3}{}{czarny makak czubaty, czubaty pawian, Macaca nigra, Cynopithecus niger -  ssak z rodziny makakowatych; gatunek endemiczny występujący wyłącznie na dwóch wyspach Indonezji: Sulawesi na północy wyspy oraz na wyspie Pulau Bacan, gdzie został sprowadzony przez człowieka w 1867 roku}
\Clue{4}{}{wirusowa choroba roślin, której objawem jest marszczenie i zwijanie się liści, a w następstwie skarłowacenie rośliny}
\Clue{6}{}{zdolność do wywoływania chorób zakaźnych}
\Clue{7}{}{mieszkaniec Galicji - wspólnoty autonomicznej lub krainy historycznej i regionu geograficznego w Hiszpanii}
\Clue{8}{}{pożar, szczególnie taki o wielkim zasięgu i powodujący ogromne zniszczenia}
\Clue{9}{}{muzykolog, ksiądz ur. w 1919 r., profesor KUL}
\Clue{10}{}{Artemisia - rodzaj roślin z rodziny astrowatych, liczący 300 gatunków; występują głównie na obszarach suchych klimatu umiarkowanego Europy, Azji i Ameryki Północnej}
\Clue{12}{}{określenie używane wobec zjawisk przypisywanych działaniu sił innych niż prawa natury i niemożliwych do wytłumaczenia w sposób naukowy}
\Clue{13}{}{pieszczotliwie: kochanie, ptaszyna, człowiek, dla którego ma się szczególnie dużo czułości, przy okazji: zwykle ktoś o łagodnym usposobieniu}
\Clue{15}{}{czułek na ciele bezkręgowców}
\Clue{16}{}{zachodnioindyjski instrument muzyczny}
\Clue{17}{}{grunt pochodzący z Księżyca}
\Clue{18}{}{kod ISO 4217 dolara bahamskiego}
\Clue{19}{}{część broni strzeleckiej służąca do oparcia broni o ramię strzelca}
\Clue{20}{}{SLOTY}
\Clue{21}{}{otwory wykonane w jakimś materiale, spełniające różne funkcje, np. ułatwienie przesuwania, odrywania, wpinania w segregator, mocowania, przepuszczania jakichś substancji, zasiewu, programowania procesów obliczeniowych itp}
\Clue{22}{}{łącznik w połączeniu śrubowym służący do spinania rur}
\Clue{23}{}{każde 10 lat z życia człowieka}
\Clue{24}{}{produkt (odpad) oczyszczania ścieków; o.ś. powstają w wyniku procesów fizycznych, fizyczno-chemicznych i biologicznych zachodzących w oczyszczalniach ścieków; mieszanina wody i ciał stałych oddzielonych z różnych typów wody w rezultacie procesów naturalnych i sztucznych}
\Clue{25}{}{film, seans filmowy}
\Clue{28}{}{w górnictwie: rodzaj prądownicy stosowanej do wtryskiwania wody w otwór strzałowy}
\Clue{29}{}{wyrostek kości czołowej u pustorożców, na których jest osadzona pochwa rogowa}
\Clue{30}{}{nauka techniczna, zajmująca się obwodami elektrycznymi, w których są obecne elementy aktywne: diody, lampy próżniowe, tranzystory}
\Clue{32}{}{duży, dwuręczny miecz używany przez niemieckich lancknechtów i szwajcarską piechotę}
\Clue{33}{}{błona komórkowa wokół aksonu}
\Clue{34}{}{ostry ser owczy, niebieskawy, żyłkowany}
\Clue{35}{}{zdrobniale - pilnik: narzędzie służące do piłowania, czyli skrawania z obrabianej powierzchni cienkiej warstwy o grubości od 0,01 do 1 mm}
\Clue{36}{}{rodzaj leczniczej kąpieli, podczas której chory jest zanużony w wodzie z rozpuszczonym dwutlenkiem węgla; stosowana w leczeniu chorób serca}
\Clue{38}{}{kod ISO 4217 colona}\end{PuzzleClues}\newpage%\section*{Krzyżówka 17}

\noindent\begin{Puzzle}{23}{27}|*	|*	|*	|*	|*	|*	|*	|*	|*	|*	|*	|*	|*	|*	|*	|*	|[1][S]\drarr	|p	|e	|n	|s	|*	|*	|*	|.
|*	|*	|*	|[2][S]\drarr	|s	|k	|o	|c	|z	|k	|o	|w	|c	|e	|*	|[3][S]\darr	|b	|[4][S]\drarr	|k	|o	|ł	|o	|*	|*	|.
|*	|*	|*	|p	|*	|*	|[5][S]\darr	|[6][S]\rarr	|k	|a	|c	|z	|k	|a	|[][,]{ }	|d	|o	|m	|o	|w	|a	|*	|*	|*	|.
|*	|*	|*	|o	|*	|[7][S]\drarr	|m	|a	|r	|s	|z	|*	|*	|[8][S]\darr	|*	|*	|k	|i	|[9][S]\darr	|*	|*	|[10][S]\darr	|*	|[11][S]\darr	|.
|*	|*	|*	|w	|[12][S]\drarr	|p	|a	|r	|l	|a	|m	|e	|n	|t	|*	|*	|s	|ó	|s	|*	|[13][S]\darr	|b	|*	|b	|.
|*	|*	|[14][S]\rarr	|s	|m	|i	|t	|h	|*	|*	|*	|*	|*	|o	|[15][S]\darr	|*	|e	|d	|z	|*	|p	|e	|*	|a	|.
|*	|*	|[16][S]\darr	|t	|h	|e	|*	|*	|*	|*	|*	|*	|*	|n	|j	|*	|r	|[][,]{ }	|y	|*	|i	|d	|*	|t	|.
|*	|*	|h	|a	|*	|n	|*	|[17][S]\rarr	|a	|k	|r	|y	|b	|i	|a	|*	|k	|k	|l	|[18][S]\darr	|a	|u	|[19][S]\darr	|e	|.
|*	|[20][S]\drarr	|a	|n	|t	|i	|f	|a	|*	|*	|*	|*	|*	|k	|z	|*	|a	|a	|d	|f	|s	|i	|e	|r	|.
|*	|p	|k	|i	|[21][S]\rarr	|s	|u	|c	|h	|a	|r	|e	|k	|*	|*	|*	|*	|p	|k	|r	|e	|n	|l	|i	|.
|*	|a	|*	|e	|*	|t	|*	|*	|*	|*	|*	|*	|[22][S]\rarr	|p	|u	|c	|h	|a	|r	|e	|k	|*	|e	|a	|.
|*	|n	|[23][S]\rarr	|c	|z	|o	|ł	|o	|w	|n	|i	|c	|a	|*	|*	|*	|*	|n	|e	|g	|[][,]{ }	|[24][S]\darr	|k	|[][,]{ }	|.
|*	|f	|*	|[][,]{ }	|*	|ś	|*	|*	|*	|*	|*	|*	|*	|*	|*	|*	|*	|i	|t	|a	|p	|l	|t	|a	|.
|*	|i	|*	|ś	|*	|ć	|*	|[25][S]\rarr	|ż	|r	|o	|n	|k	|o	|w	|a	|t	|e	|*	|t	|u	|e	|r	|a	|.
|*	|ł	|[26][S]\rarr	|l	|i	|*	|*	|*	|*	|*	|*	|*	|*	|[27][S]\drarr	|u	|r	|o	|c	|z	|y	|s	|k	|o	|*	|.
|*	|o	|*	|ą	|*	|*	|*	|[28][S]\rarr	|b	|r	|a	|t	|e	|k	|*	|*	|*	|*	|*	|*	|t	|[][S].	|n	|*	|.
|*	|w	|[29][S]\drarr	|s	|u	|w	|n	|i	|c	|a	|[][,]{ }	|b	|r	|a	|m	|o	|w	|a	|*	|*	|y	|[][,]{ }	|[][,]{ }	|[30][S]\darr	|.
|*	|c	|u	|k	|[31][S]\rarr	|l	|e	|k	|k	|o	|m	|y	|ś	|l	|n	|o	|ś	|ć	|*	|*	|n	|w	|w	|f	|.
|*	|y	|s	|i	|[32][S]\rarr	|b	|e	|z	|p	|ł	|c	|i	|o	|w	|i	|e	|c	|*	|*	|*	|i	|e	|a	|l	|.
|*	|*	|z	|*	|*	|*	|*	|[33][S]\rarr	|c	|z	|o	|s	|n	|a	|c	|z	|e	|k	|*	|*	|*	|t	|l	|o	|.
|[34][S]\rarr	|j	|a	|g	|o	|d	|n	|i	|k	|*	|[35][S]\rarr	|p	|o	|r	|z	|ą	|d	|n	|o	|ś	|ć	|*	|e	|r	|.
|[36][S]\rarr	|o	|n	|t	|o	|*	|[37][S]\drarr	|t	|r	|a	|n	|s	|m	|i	|s	|j	|a	|*	|*	|*	|*	|*	|n	|e	|.
|*	|[38][S]\drarr	|k	|u	|p	|o	|n	|*	|[39][S]\rarr	|p	|a	|w	|i	|a	|n	|[][,]{ }	|ż	|o	|ł	|t	|y	|*	|c	|n	|.
|[40][S]\drarr	|r	|a	|p	|e	|ć	|*	|*	|*	|*	|*	|*	|[41][S]\rarr	|n	|e	|w	|s	|m	|a	|n	|*	|*	|y	|c	|.
|a	|a	|*	|*	|*	|*	|*	|[42][S]\rarr	|b	|e	|k	|o	|n	|i	|a	|k	|*	|*	|*	|*	|*	|*	|j	|k	|.
|l	|z	|[43][S]\rarr	|k	|o	|r	|m	|o	|r	|a	|n	|[][,]{ }	|i	|n	|d	|y	|j	|s	|k	|i	|*	|*	|n	|i	|.
|t	|*	|*	|[44][S]\rarr	|o	|r	|e	|o	|p	|i	|t	|e	|k	|*	|*	|*	|[45][S]\rarr	|s	|z	|ó	|s	|t	|y	|*	|.
|*	|*	|*	|*	|*	|*	|*	|*	|*	|*	|*	|[46][S]\rarr	|i	|r	|r	|a	|d	|i	|a	|c	|j	|a	|*	|*	|.\end{Puzzle}

\newpage

\begin{PuzzleClues}{\textbf{Poziome}\\}\Clue{1}{}{jednostka zdawkowa w Wielkiej Brytanii i w wielu terytoriach zależnych; 1/100 funta}
\Clue{2}{}{Chytridiomycota - typ (gromada), jedna z głównych linii rozwojowych grzybów; jest to prawdopodobnie takson parafiletyczny}
\Clue{4}{}{część maszyny, urządzenia}
\Clue{6}{}{ptak gospodarski z rodziny kaczkowatych, udomowiona forma kaczki krzyżówki (Anas platyrhynchos) i kaczki piżmowej (Cairina moschata) oraz ich niepłodne mieszańce (mulardy)}
\Clue{7}{}{rodzaj demonstracji (skierowanej przeciw czemuś, coś promującej lub upamiętniającej), w której duża grupa ludzi wspólnie przemieszcza się w pochodzie przebiegającym określoną trasą}
\Clue{12}{}{miejsce, siedziba parlamentu, budynek}
\Clue{14}{}{amerykańska śpiewaczka jazzowa (1894-1937); jedna z najwybitniejszych wykonawczyń bluesa zwana 'cesarzową bluesa'}
\Clue{17}{}{staranność, skrupulatność, szczególnie jako cecha pracy filologa}
\Clue{20}{}{pewna nieformalna grupa osób (akcja, koło, środowisko), które różnymi metodami (poprzez muzykę, sport, publicystykę, street art itp.) sprzeciwiają się tendencjom skrajnie prawicowym}
\Clue{21}{}{kromka specjalnie wysuszonego pieczywa, dietetyczna i praktyczna rzecz do jedzenia (czasem: specjalny wypiek, który również jest suchy i długo pozostaje zdatny do spożycia, lecz nie przypomina kromki)}
\Clue{22}{}{tyle, ile zmieści się w pucharku - małym pucharze}
\Clue{23}{}{rodzaj belki (nie zawsze z drewna) w przedniej części konstrukcji wagonu lub parowozu}
\Clue{25}{}{Mutillidae - rodzina os; bezskrzydłe samice gatunków tej rodziny przypominają mrówki; larwy tych os są parazytoidami błonkówek zakładających gniazda podziemne}
\Clue{26}{}{w chemii: symbol litu}
\Clue{27}{}{teren odludny, najczęsciej bagnisty, leśny i trudno dostępny}
\Clue{28}{}{dwuletnia roślina ozdobna o różnobarwnych, aksamitnych kwiatach, otrzymana ze skrzyżowania dzikich gatunków fiołka}
\Clue{29}{}{suwnica najczęściej instalowana na zewnętrznych składowiskach lub pochylniach stoczniowych, składająca się z pomostu zawieszonego na dwóch bocznych mostach, wiążących dwie lub większą liczbę bram}
\Clue{31}{}{to, że jakieś postępowanie jest lekkomyślne}
\Clue{32}{}{człowiek nijaki, którego osobowość jest bezbarwna}
\Clue{33}{}{Alliaria - monotypowy rodzaj dwuletniej rośliny z rodziny kapustowatych}
\Clue{34}{}{Paramythia montium - gatunek małego ptaka z rodziny jagodników (Paramythiidae); jedyny przedstawiciel rodzaju Paramythia; jest endemitem wysokich gór Nowej Gwinei}
\Clue{35}{}{cecha jakiejś rzeczy: to, że coś jest dobrze wykonane, trwałe itp}
\Clue{36}{}{jezioro w środkowej Finlandii}
\Clue{37}{}{przenośnie: przekazanie czegoś, fakt przeniesienia się, przejścia tworów (najczęściej niematerialnych, np. umysłowych) między bytami, podmiotami}
\Clue{38}{}{blankiet do wypełnienia, na podstawie którego uczestniczy się w grze, losowaniu}
\Clue{39}{}{pawian zielony, pawian masajski, babuin, Papio cynocephalus - ssak z rzędu naczelnych, zamieszkujący tereny południowej, środkowej i wschodniej Afryki}
\Clue{40}{}{rzemień, taśma do podtrzymywania szabli przy pasie}
\Clue{41}{}{człowiek, który zbiera i redaguje newsy, najnowsze wiadomości do gazety, portalu internetowego itp}
\Clue{42}{}{młoda świnia o wadze 84-95 kg}
\Clue{43}{}{Phalacrocorax fuscicollis - gatunek ptaka z rodziny kormoranów (Phalacrocoracidae); zamieszkuje obszary słodkowodnych mokradeł na terenie Indii oraz Sri Lanki}
\Clue{44}{}{Oreopithecus bamboli - kopalny gatunek wąskonosej małpy, żyjącej w miocenie}
\Clue{45}{}{szósty dzień (najczęściej bieżącego lub przyszłego) miesiąca}
\Clue{46}{}{irradiancja - w radiometrii strumień promieniowania na jednostkę powierzchni}\end{PuzzleClues}

\begin{PuzzleClues}{\textbf{Pionowe}\\}\Clue{1}{}{suczka boksera}
\Clue{2}{}{uczestnik jednego lub kilku powstań śląskich}
\Clue{3}{}{litera alfabetu używana w numeracji porządkowej}
\Clue{4}{}{miód ściągnięty bezpośrednio z plastra, tradycyjną metodą (miody takie produkowano przed wynalezieniem miodarki)}
\Clue{5}{}{podoficerski stopień w Marynarce Wojennej}
\Clue{7}{}{fizyczna właściwość niektórych substancji, zazwyczaj cieczy, np. szamponu do włosów, płynu do mycia naczyń, wody}
\Clue{8}{}{płynny kosmetyk służący do pielęgnacji cery}
\Clue{9}{}{masa rogowa otrzymywana z płytek pancerza kostnego żółwi morskich; stosowana w jubilerstwie i w wyrobie galanterii ozdobnej}
\Clue{10}{}{bliskowschodni koczownik}
\Clue{11}{}{bateria w kształcie walca o długości ok. 51 mm. powszechnie używane w sprzęcie elektronicznym}
\Clue{12}{}{wielokrotność henra (H) - jednostki indukcyjności oraz permeancji (przewodności magnetycznej) w układzie SI; jest równy 10E-3 H}
\Clue{13}{}{syntetyczny kamień wytworzony ze szkła awenturynowego z kawałkami miedzi, zabarwiony na miedziowozłoty z połyskującymi na złoto kryształkami}
\Clue{15}{}{budowla piętrząca wodę w rzece lub na kanale wznoszona w poprzek koryta; stała lub ruchoma}
\Clue{16}{}{zakrzywiony pręt metalowy służący do zaczepiania, trzymania lub zawieszania na nim przedmiotów}
\Clue{18}{}{fregatowate, Fregatidae - rodzina ptaków z rzędu głuptakowych (Suliformes)}
\Clue{19}{}{elektron znajdujący się na ostatniej, najbardziej zewnętrznej powłoce atomu, która nazywana jest powłoką walencyjną}
\Clue{20}{}{żołnierze słynnej dywizji piechoty, którzy w czasie walk o Moskwę zagrodzili drogę wojskom niemieckim na szosie Moskwa-Wołokołamsk}
\Clue{24}{}{skrót odlekarz weterynarii}
\Clue{27}{}{mieszkaniec Kalwarii Zebrzydowskiej}
\Clue{29}{}{papacha, uszatka - czapka futrzana z nausznikami}
\Clue{30}{}{dialekt języka włoskiego używany w rejonie Florencji}
\Clue{37}{}{w chemii: symbol azotu}
\Clue{38}{}{cios, uderzenie}
\Clue{40}{}{klawisz używany do zmiany funkcji innych klawiszy, używany w różnych kombinacjach z innymi klawiszami}\end{PuzzleClues}\newpage%\section*{Krzyżówka 18}

\noindent\begin{Puzzle}{24}{28}|*	|*	|*	|*	|*	|*	|*	|*	|*	|*	|*	|*	|*	|*	|*	|*	|[1][S]\drarr	|k	|o	|r	|y	|t	|k	|o	|*	|.
|*	|*	|[2][S]\darr	|*	|*	|[3][S]\rarr	|d	|z	|i	|w	|o	|o	|k	|[][,]{ }	|ł	|u	|p	|k	|o	|w	|y	|*	|*	|*	|*	|.
|*	|*	|p	|[4][S]\drarr	|m	|a	|s	|z	|y	|n	|a	|[][,]{ }	|w	|y	|p	|o	|r	|o	|w	|a	|*	|*	|*	|*	|*	|.
|*	|*	|r	|g	|*	|[5][S]\darr	|[6][S]\darr	|*	|*	|*	|*	|*	|*	|*	|*	|[7][S]\drarr	|o	|c	|i	|o	|s	|*	|*	|[8][S]\darr	|*	|.
|*	|*	|y	|a	|[9][S]\rarr	|l	|i	|b	|e	|r	|a	|ł	|*	|[10][S]\darr	|*	|w	|m	|*	|*	|*	|*	|[11][S]\darr	|[12][S]\darr	|k	|*	|.
|*	|*	|m	|z	|[13][S]\darr	|i	|m	|*	|*	|*	|*	|*	|*	|k	|[14][S]\drarr	|s	|i	|e	|c	|i	|ó	|w	|k	|a	|*	|.
|*	|*	|i	|*	|f	|g	|p	|*	|*	|*	|*	|*	|*	|e	|k	|p	|e	|*	|[15][S]\darr	|[16][S]\darr	|[17][S]\darr	|i	|ł	|p	|*	|.
|*	|*	|t	|*	|l	|a	|o	|*	|*	|*	|*	|[18][S]\darr	|*	|k	|a	|ó	|n	|[19][S]\darr	|p	|p	|d	|r	|a	|o	|*	|.
|*	|*	|y	|[20][S]\darr	|i	|b	|t	|*	|*	|*	|*	|w	|*	|s	|r	|ł	|i	|w	|o	|l	|r	|t	|m	|k	|*	|.
|*	|[21][S]\rarr	|w	|z	|n	|i	|e	|s	|i	|e	|n	|i	|e	|*	|ł	|r	|o	|ę	|ł	|a	|a	|u	|s	|*	|*	|.
|*	|*	|n	|w	|t	|n	|n	|[22][S]\drarr	|k	|o	|j	|e	|c	|*	|o	|a	|w	|g	|u	|n	|g	|a	|t	|*	|*	|.
|*	|*	|o	|i	|*	|o	|t	|d	|[23][S]\drarr	|f	|i	|l	|t	|r	|*	|d	|a	|i	|d	|e	|a	|l	|w	|*	|*	|.
|*	|*	|ś	|ą	|*	|*	|*	|o	|d	|[24][S]\darr	|*	|k	|[25][S]\rarr	|o	|p	|o	|n	|e	|n	|t	|*	|i	|o	|*	|*	|.
|*	|[26][S]\darr	|ć	|z	|[27][S]\rarr	|c	|y	|j	|a	|n	|i	|a	|n	|*	|*	|ś	|i	|l	|i	|a	|*	|z	|[][,]{ }	|*	|*	|.
|*	|s	|*	|k	|*	|*	|*	|ś	|c	|a	|*	|[][,]{ }	|[28][S]\darr	|*	|*	|ć	|e	|[][,]{ }	|k	|[][,]{ }	|*	|a	|l	|*	|*	|.
|*	|k	|*	|o	|*	|*	|*	|c	|h	|n	|*	|s	|i	|*	|[29][S]\darr	|*	|*	|k	|[][,]{ }	|o	|*	|c	|u	|*	|*	|.
|*	|u	|*	|w	|[30][S]\darr	|*	|*	|i	|*	|h	|*	|e	|n	|*	|f	|[31][S]\darr	|*	|e	|m	|k	|*	|j	|s	|*	|*	|.
|*	|b	|*	|i	|c	|*	|*	|e	|*	|e	|*	|r	|t	|*	|u	|d	|*	|n	|a	|o	|*	|a	|t	|*	|*	|.
|*	|a	|*	|e	|h	|*	|*	|*	|*	|*	|*	|b	|e	|*	|g	|e	|*	|e	|g	|ł	|*	|*	|r	|*	|*	|.
|*	|n	|[32][S]\drarr	|c	|a	|n	|t	|u	|s	|[][,]{ }	|f	|i	|r	|m	|u	|s	|*	|l	|n	|o	|*	|[33][S]\darr	|a	|*	|*	|.
|*	|i	|z	|*	|m	|[34][S]\drarr	|d	|r	|u	|c	|i	|a	|r	|z	|*	|e	|*	|s	|e	|p	|*	|t	|c	|*	|*	|.
|[35][S]\drarr	|e	|g	|z	|e	|m	|p	|l	|a	|r	|z	|*	|e	|*	|*	|r	|*	|k	|t	|o	|[36][S]\darr	|y	|y	|*	|*	|.
|d	|c	|r	|*	|o	|a	|*	|[37][S]\rarr	|j	|a	|b	|ł	|k	|o	|*	|*	|*	|i	|y	|d	|g	|c	|j	|*	|*	|.
|r	|*	|e	|*	|f	|k	|*	|[38][S]\rarr	|p	|r	|z	|e	|s	|a	|d	|a	|*	|*	|c	|w	|l	|z	|n	|*	|*	|.
|a	|*	|d	|*	|i	|u	|*	|*	|*	|*	|*	|*	|*	|*	|*	|*	|*	|*	|z	|ó	|o	|k	|e	|*	|*	|.
|g	|*	|e	|*	|t	|l	|*	|*	|*	|*	|*	|*	|*	|*	|*	|*	|*	|*	|n	|j	|b	|a	|*	|*	|*	|.
|*	|*	|k	|*	|*	|a	|[39][S]\rarr	|c	|z	|ł	|o	|w	|i	|e	|k	|[][,]{ }	|c	|z	|y	|n	|u	|*	|*	|*	|*	|.
|*	|*	|*	|*	|*	|*	|*	|[40][S]\rarr	|i	|l	|u	|m	|i	|n	|a	|c	|j	|a	|*	|a	|s	|*	|*	|*	|*	|.
|[41][S]\rarr	|ś	|w	|i	|e	|r	|z	|b	|o	|w	|i	|e	|c	|[][,]{ }	|u	|s	|z	|n	|y	|*	|*	|*	|*	|*	|*	|.\end{Puzzle}

\newpage

\begin{PuzzleClues}{\textbf{Poziome}\\}\Clue{1}{}{przyrząd biurowy, pojemnik na ołówki, długopisy itp}
\Clue{3}{}{Melidectes ochromelas - gatunek ptaka z rodziny miodojadów (Meliphagidae) występujący na Nowej Gwinei i Archipelagu Bismarcka}
\Clue{4}{}{maszyna robocza, której działanie polega na wypieraniu dawek płynu z przestrzeni ssawnej do tłocznej}
\Clue{7}{}{zrąb; boczna ściana wyrobiska górniczego}
\Clue{9}{}{człowiek popierający liberalizm, także należący do partii liberalnej}
\Clue{14}{}{bilet sieciowy}
\Clue{21}{}{fragment terenu wznoszący się ponad otaczający równinny obszar}
\Clue{22}{}{mebel z ograniczoną przestrzenią dla małego dziecka, w którym może przebywać i bawić się}
\Clue{23}{}{część papierosa filtrująca zawarte w tytoniu substancje smoliste}
\Clue{25}{}{przeciwnik, osoba, która się czemuś lub komuś przeciwstawia}
\Clue{27}{}{sól kwasu cyjanowego}
\Clue{32}{}{linia melodyczna, do której dokomponowuje się kontrapunkt}
\Clue{34}{}{ktoś, kto naprawia samochody po kosztach, wykorzystując najtańsze rozwiązania, przede wszystkim stare/ złe jakościowo części}
\Clue{35}{}{okaz, organizm, osobnik, jednostka}
\Clue{37}{}{kula lub krążek stanowiące zakończenie topu masztu}
\Clue{38}{}{nadmiar, to, że ktoś przesadza}
\Clue{39}{}{osoba, która działa, często robiąc coś ważnego dla społeczeństwa, w którym żyje}
\Clue{40}{}{w edytorstwie: ozdoba na karcie tytułowej książki będąca kompozycją graficzną, figuralną lub ornamentalną}
\Clue{41}{}{Otodectes cynotis - gatunek roztocza z rodziny świerzbowców}\end{PuzzleClues}

\begin{PuzzleClues}{\textbf{Pionowe}\\}\Clue{1}{}{strumień cząstek lub fal wysyłanych przez ciało}
\Clue{2}{}{to, że ktoś jest prymitywny, prostacki; cecha kogoś prymitywnego}
\Clue{4}{}{paliwo}
\Clue{5}{}{Ligabueino - rodzaj dinozaura z grupy ceratozaurów; żył w okresie wczesnej kredy na terenach Ameryki Południowej, osiągał około 70 centymetrów długości ciała i 30 centymetrów wysokości}
\Clue{6}{}{mężczyzna niemogący czemuś podołać, niezdolny do działania}
\Clue{7}{}{radość wspólna, dzielona z kimś}
\Clue{8}{}{włókno otrzymywane z różnych gatunków drzew wełniakowatych, miękkie, nieprzemakalne, odznaczające się właściwością utrzymywania na wodzie dość dużego ciężaru, używane do wypełniania wodnego sprzętu ratunkowego}
\Clue{10}{}{ciasto biszkoptowe z bakaliami}
\Clue{11}{}{szerokie pojęcie odnoszące się do abstrakcji zasobów w różnych aspektach informatyki; szczególnie: użycie maszyn wirtualnych}
\Clue{12}{}{zatajenie współpracy z służbami komunistycznymi podczas postępowania lustracyjnego}
\Clue{13}{}{miasto w USA (Michigan) duży ośrodek przemysłu samochodowego}
\Clue{14}{}{fotel charakteryzujący się esowato wygiętymi, krzyżującymi się nogami; był znany w starożytnym Egipcie i Rzymie, później był powszechny w średniowiecznych Włoszech}
\Clue{15}{}{południk przechodzący przez bieguny magnetyczne}
\Clue{16}{}{planeta orbitująca wokół obu składników gwiazdy podwójnej}
\Clue{17}{}{jednostka pływająca, której przeznaczeniem jest pogłębianie akwenów morskich i śródlądowych}
\Clue{18}{}{serbska idea nacjonalistyczna wykorzystywana od XIX wieku, wyrażająca plan przywrócenia Serbii ziem dawniej posiadanych, w szczególności w okresie panowania dynastii Nemaniczów}
\Clue{19}{}{węgiel kamienny sapropelowy złożony głównie z licznych spor (sporynitu) oraz rozproszonego inertynitu}
\Clue{20}{}{osoba, która należy do związku - organizacji zrzeszającej ludzi o podobnych poglądach, trzymających się wspólnych zasad}
\Clue{22}{}{osiągnąć jakąś granicę w czasie}
\Clue{23}{}{górna, najwyższa część budynku, mająca za zadanie przykrycie i osłanianie go przed opadami atmosferycznymi}
\Clue{24}{}{miasto w Chinach na Nizinie Chińskiej}
\Clue{26}{}{z podziwem o kimś, kto świetnie sobie radzi, jest sprytny, utalentowany, budzi zazdrość}
\Clue{28}{}{najwyższy dostojnik państwowy sprawujący niektóre funkcje monarsze w czasie bezkrólewia}
\Clue{29}{}{takifugu - ryba morska z rodziny rozdymkowatych, której mięso jest bardzo silną trucizną}
\Clue{30}{}{jedna z form życiowych roślin, w której pączki umożliwiające odtworzenie się rośliny w przyszłym sezonie wegetacyjnym znajdują się ponad ziemią, ale nie wyżej niż 0,3-0,5 m}
\Clue{31}{}{potrawa (najczęściej słodka) jadana zazwyczaj na zakończenie posiłku głównego lub na podwieczorek czy drugie śniadanie}
\Clue{32}{}{bardzo popularna i lubiana postać - skrzat domowy z serii książek i filmów o Harrym Potterze}
\Clue{33}{}{długi, cienki pręt, palik, słupek}
\Clue{34}{}{ur. w 1930 r., pilot szybowcowy, samolotowy i śmigłowcowy, mistrz świata w szybownictwie, zdobywca medalu Lilienthala}
\Clue{35}{}{stylówa, przebranie i makijaż, zwykle zmieniające całkowicie czyjś wygląd; pojęcie wywodzi się w polszczyźnie z kultury crossdresserów i dragqueenów, ale dzisiaj bywa używane na nazwanie także damskiej stylizacji}
\Clue{36}{}{model kulisty Ziemi lub innego ciała niebieskiego}\end{PuzzleClues}\newpage%\section*{Krzyżówka 19}

\noindent\begin{Puzzle}{24}{32}|*	|[1][S]\darr	|[2][S]\drarr	|s	|y	|n	|t	|a	|k	|t	|y	|k	|a	|*	|[3][S]\drarr	|r	|o	|c	|z	|n	|i	|c	|a	|*	|*	|.
|*	|g	|w	|*	|*	|*	|[4][S]\drarr	|z	|a	|k	|ł	|a	|d	|[][,]{ }	|p	|o	|p	|r	|a	|w	|c	|z	|y	|*	|[5][S]\darr	|.
|*	|r	|s	|*	|*	|*	|k	|[6][S]\rarr	|c	|y	|k	|l	|[][,]{ }	|k	|r	|e	|b	|s	|a	|*	|[7][S]\darr	|*	|[8][S]\darr	|*	|s	|.
|*	|u	|p	|*	|[9][S]\rarr	|w	|o	|l	|e	|j	|*	|*	|[10][S]\darr	|[11][S]\darr	|o	|*	|*	|*	|*	|[12][S]\darr	|s	|*	|p	|*	|i	|.
|*	|b	|ó	|[13][S]\rarr	|l	|i	|s	|e	|k	|*	|*	|[14][S]\darr	|t	|d	|m	|*	|*	|*	|*	|u	|t	|[15][S]\darr	|r	|*	|ł	|.
|*	|o	|ł	|*	|[16][S]\rarr	|k	|a	|t	|z	|*	|[17][S]\darr	|m	|i	|z	|i	|*	|*	|*	|[18][S]\darr	|ż	|o	|p	|z	|[19][S]\darr	|a	|.
|*	|s	|ś	|[20][S]\darr	|*	|[21][S]\drarr	|c	|h	|l	|a	|p	|a	|n	|i	|e	|[][,]{ }	|j	|ę	|z	|y	|k	|i	|e	|m	|*	|.
|[22][S]\drarr	|k	|r	|o	|p	|l	|i	|k	|*	|[23][S]\darr	|ó	|m	|n	|a	|n	|[24][S]\darr	|[25][S]\darr	|[26][S]\darr	|ł	|c	|ł	|n	|w	|o	|*	|.
|k	|ó	|o	|b	|[27][S]\darr	|e	|e	|*	|*	|b	|ł	|o	|*	|ł	|i	|s	|o	|n	|o	|i	|o	|g	|ó	|d	|*	|.
|a	|r	|d	|s	|d	|w	|c	|*	|*	|r	|d	|n	|[28][S]\darr	|*	|o	|t	|p	|o	|t	|e	|s	|w	|d	|r	|*	|.
|r	|c	|k	|ł	|i	|y	|[][,]{ }	|*	|*	|a	|z	|a	|p	|[29][S]\darr	|w	|r	|c	|l	|o	|[][,]{ }	|a	|i	|[][,]{ }	|a	|*	|.
|d	|e	|o	|u	|a	|[][,]{ }	|b	|*	|[30][S]\darr	|n	|i	|*	|r	|g	|a	|u	|j	|a	|u	|n	|[][,]{ }	|n	|s	|s	|*	|.
|y	|*	|w	|g	|g	|p	|u	|[31][S]\drarr	|m	|i	|e	|s	|z	|a	|n	|k	|a	|*	|c	|i	|ł	|[][,]{ }	|c	|z	|*	|.
|n	|*	|o	|a	|o	|r	|c	|b	|i	|e	|c	|*	|e	|r	|i	|t	|[][,]{ }	|*	|h	|e	|u	|b	|h	|e	|*	|.
|a	|*	|ś	|*	|n	|o	|h	|e	|c	|[][,]{ }	|k	|*	|k	|n	|e	|u	|b	|*	|[][,]{ }	|k	|s	|i	|e	|k	|*	|.
|ł	|[32][S]\darr	|ć	|[33][S]\darr	|a	|s	|a	|z	|h	|p	|o	|*	|u	|i	|[][,]{ }	|r	|a	|*	|s	|o	|k	|a	|m	|[][,]{ }	|*	|.
|e	|s	|*	|j	|l	|t	|r	|c	|a	|o	|*	|*	|p	|r	|n	|a	|r	|*	|z	|m	|o	|ł	|a	|r	|*	|.
|k	|t	|*	|e	|i	|y	|s	|z	|ł	|p	|*	|[34][S]\darr	|k	|o	|a	|[][,]{ }	|i	|*	|a	|e	|w	|o	|t	|o	|*	|.
|*	|r	|*	|d	|z	|*	|k	|e	|*	|r	|*	|s	|a	|w	|d	|z	|e	|*	|r	|r	|a	|o	|o	|z	|*	|.
|*	|u	|[35][S]\rarr	|n	|a	|w	|i	|l	|ż	|a	|c	|z	|*	|a	|c	|a	|r	|*	|o	|c	|t	|k	|w	|c	|*	|.
|[36][S]\drarr	|g	|r	|o	|c	|h	|*	|*	|*	|w	|[37][S]\darr	|a	|*	|n	|z	|t	|o	|*	|c	|y	|a	|i	|y	|h	|*	|.
|c	|*	|*	|w	|j	|*	|[38][S]\darr	|*	|*	|k	|d	|f	|*	|i	|e	|r	|w	|*	|z	|j	|*	|*	|*	|o	|*	|.
|f	|*	|*	|y	|a	|[39][S]\darr	|w	|*	|*	|i	|n	|r	|[40][S]\darr	|e	|r	|u	|a	|*	|e	|n	|*	|*	|[41][S]\darr	|d	|*	|.
|*	|*	|*	|m	|*	|n	|e	|*	|*	|*	|o	|a	|d	|*	|w	|d	|*	|*	|l	|e	|*	|*	|w	|n	|*	|.
|[42][S]\rarr	|u	|n	|i	|w	|e	|r	|s	|u	|m	|*	|n	|i	|*	|o	|n	|*	|*	|n	|*	|*	|*	|a	|i	|*	|.
|[43][S]\drarr	|g	|r	|a	|d	|o	|b	|i	|c	|i	|e	|*	|a	|*	|n	|i	|*	|*	|y	|*	|*	|*	|r	|k	|*	|.
|n	|[44][S]\rarr	|p	|r	|o	|f	|e	|s	|j	|a	|*	|*	|l	|*	|e	|e	|*	|*	|*	|*	|*	|*	|n	|o	|*	|.
|a	|*	|*	|o	|*	|i	|n	|*	|[45][S]\rarr	|z	|a	|l	|e	|w	|*	|n	|*	|*	|*	|*	|*	|*	|a	|w	|*	|.
|j	|*	|*	|w	|*	|t	|a	|*	|*	|[46][S]\rarr	|b	|i	|k	|i	|n	|i	|a	|r	|s	|t	|w	|o	|*	|i	|*	|.
|a	|*	|*	|o	|*	|a	|*	|*	|*	|*	|*	|*	|t	|[47][S]\rarr	|s	|a	|t	|e	|l	|i	|t	|a	|*	|e	|*	|.
|d	|*	|*	|ś	|*	|*	|*	|[48][S]\rarr	|f	|u	|t	|r	|y	|n	|a	|*	|*	|*	|*	|*	|*	|*	|*	|c	|*	|.
|a	|*	|*	|ć	|[49][S]\rarr	|p	|l	|a	|s	|k	|a	|n	|k	|a	|[][,]{ }	|s	|e	|r	|o	|w	|a	|*	|*	|*	|*	|.
|*	|*	|*	|*	|*	|*	|*	|*	|*	|*	|*	|*	|*	|*	|*	|*	|*	|*	|*	|*	|*	|*	|*	|*	|*	|.\end{Puzzle}

\newpage

\begin{PuzzleClues}{\textbf{Poziome}\\}\Clue{2}{}{dział językoznawstwa, który zajmuje się budową wypowiedzeń.}
\Clue{3}{}{obchody którejś rocznicy jakiegoś wydarzenia}
\Clue{4}{}{placówka resocjalizacyjna dla nieletnich od 11 do 18 roku życia, skierowanych ze schroniska dla nieletnich}
\Clue{6}{}{cykliczny szereg reakcji biochemicznych, stanowiący końcowy etap metabolizmu aerobów}
\Clue{9}{}{uderzenie piłki tenisowej grane z powietrza przed odbiciem się piłki od podłoża}
\Clue{13}{}{zdrobniale: lis - wyprawione futro z lisa}
\Clue{16}{}{ur. w 1911 r. elektrofizjolog angielski; laureat nagrody Nobla}
\Clue{21}{}{mówienie za dużo, niepotrzebnie, ponad miarę}
\Clue{22}{}{Mimulus - roślina z rodziny trędownikowatych rosnąca w Ameryce w strefie umiarkowanej}
\Clue{31}{}{różne fizyczne obiekty (np. ciecze, gazy, nasiona) pomieszane celowo w określonych proporcjach, z myślą o praktycznym zastosowaniu; materia, substancja, która jest zrobiona z więcej niż jednego składnika}
\Clue{35}{}{naczynie zawieszane na grzejnikach wypełnione wodą w celu nawilżenia powietrza}
\Clue{36}{}{warzywo strączkowe; ziarno grochu zwyczajnego}
\Clue{42}{}{wszechświat, wszystko, co fizycznie istnieje: cała przestrzeń, czas, wszystkie formy materii i energii oraz prawa fizyki i stałe fizyczne określające ich zachowanie}
\Clue{43}{}{obfity grad ze szczególnie dużymi gradzinami}
\Clue{44}{}{zbiór zadań (zespół czynności) wyodrębnionych w wyniku społecznego podziału pracy, będących świadczeniami na rzecz innych osób, wykonywanych stale lub z niewielkimi zmianami przez poszczególne osoby i wymagających odpowiednich kwalifikacji (wiedzy i umiejętności), zdobytych w wyniku kształcenia lub praktyki; zajęcie, którego ktoś się wyuczył}
\Clue{45}{}{płytka zatoka oddzielona od morza mierzeją}
\Clue{46}{}{styl bycia typowy dla bikiniarza}
\Clue{47}{}{państwo lub jakakolwiek organizacja zależna w jakiś sposób od innej, często większej i prowadzącej agresywną politykę}
\Clue{48}{}{rama w otworze okiennym lub drzwiowym}
\Clue{49}{}{regionalne danie z Podkarpacia, zapiekanka ze zmielonego twarogu, jaj i kminku}\end{PuzzleClues}

\begin{PuzzleClues}{\textbf{Pionowe}\\}\Clue{1}{}{Ceratomorpha - podrząd ssaków z rzędu nieparzystokopytnych; należą do niego rodziny: nosorożcowate i tapirowate}
\Clue{2}{}{cecha obiektów, które mają wspólny środek}
\Clue{3}{}{promieniowanie elektromagnetyczne o długości fal pomiędzy światłem widzialnym a falami radiowymi}
\Clue{4}{}{Iris bucharica - garunek rośliny należący do rodziny kosaćcowatych}
\Clue{5}{}{energia, którą dysponuje człowiek, witalność, tyle zdrowia i zapału, ile ma w danej chwili}
\Clue{7}{}{Bromus squarrosus - gatunek rośliny z rodziny wiechlinowatych}
\Clue{8}{}{przewód izolowany stosowany do połączeń wewnętrznych w urządzeniach telekomunikacyjnych}
\Clue{10}{}{jezioro w Norwegii w Górach Skandynawskich}
\Clue{11}{}{osoby pracujące w danej jednostce organizacyjnej}
\Clue{12}{}{jeden z warunków licencji otwartej, który nie zezwala na wykorzystywanie dzieł w sposób komercyjny}
\Clue{14}{}{najczęściej żartobliwie o pieniądzach, środkach płatniczych}
\Clue{15}{}{pingwin Adeli, Pygoscelis adeliae - gatunek dużego, nielotnego ptaka wodnego z rodziny pingwinów (Spheniscidae), zamieszkujący wody wokół Antarktydy; gnieździ się na Antarktydzie oraz Orkadach Południowych i Szetlandach Południowych}
\Clue{17}{}{osoba niedojrzała emocjonalnie}
\Clue{18}{}{Ptilotula plumula - gatunek ptaka z rodziny miodojadów (Meliphagidae) występujący w Australii i na Nowej Gwinei}
\Clue{19}{}{modraszek orion, Scolitantides orion - motyl z rodziny modraszkowatych; występuje zarówno na nizinach, jak i w górach; jest gatunkiem chronionym}
\Clue{20}{}{część personelu instytucji lub firmy, odpowiedzialna za obsługiwanie urządzeń lub ludzi (klientów bądź petentów)}
\Clue{21}{}{cios prosty wyprowadzony z lewej ręki}
\Clue{22}{}{barwna, chińska ryba z rodziny karpiowatych}
\Clue{23}{}{uwzględnianie, branie pod uwagę}
\Clue{24}{}{podział ludności w grupie według ich zatrudnienia}
\Clue{25}{}{rodzaj opcji egzotycznej, której wykonanie zależy od tego czy cena instrumentu bazowego osiagnie lub przekroczy ustalony poziom}
\Clue{26}{}{miasto w płd. Włoszech (Kampania) na wschód od Neapolu}
\Clue{27}{}{zmiana macierzy kwadratowej w diagonalną}
\Clue{28}{}{kobieta zajmująca się drobnym handlem, sprzedająca jakieś towary na bardzo małą skalę}
\Clue{29}{}{dekorowanie potraw i naczyń, na których są podane}
\Clue{30}{}{stosunkowo łatwy przedmiot w szkole, taki, który wydaje się mało obciążający, ale z drugiej strony: są z niego oceny}
\Clue{31}{}{człowiek bezczelny}
\Clue{32}{}{(węglowy) urabiarka do węgla odwalająca ostrzem urobek}
\Clue{33}{}{cecha czegoś, co ma jeden wymiar fizyczny}
\Clue{34}{}{bulwiasta bylina z kosaćcowatych; chroniony}
\Clue{36}{}{skrót oznaczający kaliforn}
\Clue{37}{}{pogardliwie: kompletny ignorant, zero; stosowane także w odniesieniu do osoby zgniłej moralnie}
\Clue{38}{}{ameryk. roślina zielna lub półkrzew, w Polsce pospolita, lecznicza i ozdobna}
\Clue{39}{}{osoba, która przyjęła nową wiarę albo niedawno zaangażowała się w głoszenie jakiejś idei}
\Clue{40}{}{człowiek biegły w dyskutowaniu}
\Clue{41}{}{miasto w Bułgarii ośrodek administracyjny okręgu Warna nad Morzem Czarnym, największy bułgarski port morski}
\Clue{43}{}{nimfa wodna z mitologii greckiej; związana z wodami środlądowymi}\end{PuzzleClues}\newpage%\section*{Krzyżówka 20}

\noindent\begin{Puzzle}{20}{23}|*	|*	|*	|*	|*	|*	|*	|*	|[1][S]\drarr	|r	|e	|f	|*	|*	|*	|*	|*	|*	|*	|*	|*	|.
|*	|*	|*	|*	|*	|[2][S]\rarr	|h	|a	|n	|s	|u	|e	|z	|j	|a	|*	|*	|*	|*	|*	|*	|.
|[3][S]\drarr	|s	|p	|e	|c	|j	|a	|l	|i	|z	|a	|n	|t	|k	|a	|*	|*	|*	|*	|*	|*	|.
|k	|[4][S]\rarr	|z	|g	|i	|ę	|c	|i	|e	|[][,]{ }	|ł	|o	|k	|c	|i	|o	|w	|e	|*	|*	|*	|.
|e	|[5][S]\darr	|[6][S]\darr	|[7][S]\rarr	|o	|c	|z	|e	|p	|*	|*	|[8][S]\drarr	|d	|o	|m	|i	|n	|i	|u	|m	|*	|.
|r	|l	|z	|[9][S]\rarr	|w	|e	|l	|u	|r	|*	|[10][S]\rarr	|k	|ó	|ł	|k	|o	|*	|*	|*	|[11][S]\darr	|*	|.
|*	|u	|r	|*	|*	|[12][S]\darr	|*	|[13][S]\darr	|z	|[14][S]\rarr	|ł	|u	|t	|*	|[15][S]\darr	|*	|*	|[16][S]\darr	|[17][S]\darr	|f	|*	|.
|[18][S]\drarr	|k	|a	|j	|a	|k	|*	|s	|y	|*	|*	|m	|[19][S]\darr	|[20][S]\darr	|j	|*	|*	|a	|p	|r	|*	|.
|w	|*	|z	|[21][S]\drarr	|b	|o	|n	|e	|s	|*	|*	|b	|t	|h	|a	|*	|*	|f	|o	|a	|*	|.
|y	|*	|*	|n	|*	|n	|[22][S]\darr	|n	|t	|[23][S]\rarr	|b	|e	|r	|i	|n	|g	|*	|i	|l	|n	|*	|.
|t	|*	|[24][S]\darr	|e	|*	|c	|m	|t	|o	|[25][S]\rarr	|k	|r	|o	|k	|o	|d	|y	|l	|e	|k	|*	|.
|r	|*	|ś	|*	|[26][S]\darr	|e	|i	|e	|s	|[27][S]\darr	|[28][S]\darr	|l	|g	|o	|w	|*	|*	|i	|[][,]{ }	|[][,]{ }	|*	|.
|z	|*	|m	|[29][S]\darr	|r	|p	|ł	|n	|o	|m	|t	|a	|l	|r	|i	|*	|*	|a	|b	|m	|*	|.
|e	|*	|i	|p	|y	|t	|o	|c	|w	|i	|u	|n	|o	|a	|e	|*	|*	|c	|e	|o	|*	|.
|s	|*	|e	|ł	|t	|*	|r	|j	|a	|z	|l	|d	|d	|*	|c	|[30][S]\darr	|*	|j	|z	|n	|*	|.
|z	|[31][S]\darr	|c	|a	|[][,]{ }	|[32][S]\drarr	|z	|a	|l	|a	|u	|*	|y	|*	|*	|n	|*	|a	|ź	|a	|*	|.
|c	|c	|i	|t	|r	|o	|ę	|*	|n	|n	|*	|*	|t	|*	|[33][S]\darr	|o	|*	|*	|r	|k	|*	|.
|z	|z	|u	|o	|z	|r	|b	|*	|o	|t	|[34][S]\darr	|*	|a	|*	|d	|r	|[35][S]\darr	|[36][S]\darr	|ó	|i	|*	|.
|k	|e	|s	|w	|y	|d	|o	|*	|ś	|r	|y	|*	|*	|*	|z	|i	|f	|m	|d	|j	|*	|.
|a	|r	|z	|i	|m	|a	|w	|*	|ć	|o	|s	|*	|*	|*	|i	|t	|a	|a	|ł	|s	|*	|.
|*	|e	|k	|e	|s	|*	|e	|*	|*	|p	|t	|*	|*	|[37][S]\rarr	|w	|o	|j	|ł	|o	|k	|*	|.
|[38][S]\rarr	|p	|a	|c	|k	|a	|*	|[39][S]\rarr	|p	|i	|a	|s	|t	|*	|e	|*	|k	|a	|w	|i	|*	|.
|*	|*	|*	|*	|i	|*	|*	|[40][S]\rarr	|ł	|a	|d	|o	|w	|a	|r	|k	|a	|*	|e	|*	|*	|.
|[41][S]\rarr	|e	|f	|a	|*	|[42][S]\rarr	|r	|[][S]2	|[][S]0	|*	|*	|*	|*	|*	|*	|*	|*	|*	|*	|*	|*	|.\end{Puzzle}

\newpage

\begin{PuzzleClues}{\textbf{Poziome}\\}\Clue{1}{}{poprzeczny rząd krótkich linek przymocowanych do żagla w jego dolnej części}
\Clue{2}{}{Hanssuesia - rodzaj dinozaura z rodziny pachycefalozaurów, żyjący w okresie późnej kredy na terenach Ameryki Północnej; długość ciała 3 m, wysokość 1,2 m, ciężar 60 kg}
\Clue{3}{}{lekarka, która robi specjalizację w jakimś zakresie, taka, która jeszcze nie nabyła uprawnień specjalistki, nie ukończyła specjalizacji}
\Clue{4}{}{miejsce na ręce człowieka, w którym ramię łączy się z przedramieniem}
\Clue{7}{}{pozioma belka wiążąca górne końce słupów budowy}
\Clue{8}{}{w prawie rzymskim: własność prywatna - prawo do nieograniczonego korzystania z rzeczy i rozporządzania nimi we własnym interesie}
\Clue{9}{}{dość ogólna nazwa tkanin z grupy pluszów, czyli tkanin z okrywą włókienną (włosową); czasem nazywa się tak specyficznie plusz osnowowy, czyli taki, w którym okrywa runowa tworzona jest z osnowy}
\Clue{10}{}{małe koło tj. grono ludzi}
\Clue{14}{}{dawna jednostka miary masy, używana w Europie od średniowiecza do końca XIX wieku, stosowana przez kupców i mincerzy do określania wagi monet i wagi probierczej srebra; wartość łuta w zależności od stulecia i miejsca wynosiła od 10 do 50 gramów}
\Clue{18}{}{sztywna lub składana łódź turystyczna lub sportowa poruszana wiosłem}
\Clue{21}{}{klaskani instrument w kształcie australijskich bumerangów}
\Clue{23}{}{duński żeglarz i odkrywca (1681-1741); odkrył brzegi Alaski i Aleuty}
\Clue{25}{}{zdrobniale: krokodyl - należący do rzędu krokodyli drapieżny gad o wydłużonym ciele pokrytym tarczkami rogowymi, płaskiej głowie z długim pyskiem i krótkich nogach, żyjący w słodkich wodach krajów tropikalnych}
\Clue{32}{}{miasto w płd. Rumunii, ośrodek administracyjny okręgu Salaj; przemysł maszynowy i drzewny}
\Clue{37}{}{niskogatunkowy filc używany do wyrobu obuwia tekstylnego, na chodniki, jako materiał izolacyjny}
\Clue{38}{}{PACA}
\Clue{39}{}{władca z dynastii Piastów wywodzącej się od Piasta Kołodzieja}
\Clue{40}{}{urządzenie służące do wprowadzenia energii elektrycznej do  akumulatora elektrycznego}
\Clue{41}{}{biblijna jednostka objętości materiałów sypkich}
\Clue{42}{}{gruba i okrągła bateria o napięciu 1,5 V}\end{PuzzleClues}

\begin{PuzzleClues}{\textbf{Pionowe}\\}\Clue{1}{}{to, że coś jest nieprzystosowane, niedostosowane do czegoś innego, nie pasuje do tego, nie odpowiada temu}
\Clue{3}{}{kauczuk syntetyczny styrenowy polskiej produkcji}
\Clue{5}{}{zamykany pokrywą otwór w pokładzie statku, służący do załadunku i wyładunku towarów, doprowadzania powietrza, światła itp}
\Clue{6}{}{w ogrodnictwie: tzw. komponent szczepienia; część szlachetnej rośliny (zazwyczaj jednoroczny pęd), którą naszczepia się na podkładce}
\Clue{8}{}{zimny sos do mięsa przyrządzany z galaretki porzeczkowej z dodatkiem chrzanu, papryki itp}
\Clue{11}{}{dawna waluta księstwa Monako, zastąpiona w 2002 roku przez Euro}
\Clue{12}{}{w archiwistyce: wersja dokumentu, na podstawie której przygotowuje się czystopis}
\Clue{13}{}{maksyma - krótka, zwykle jednozdaniowa wypowiedź, wyrażająca ogólną prawdę, mądrość dotyczącą życia człowieka}
\Clue{15}{}{krzewinka z motylkowatych o żółtych kwiatach, z pędów i kwiatów otrzymuje się żółty barwnik}
\Clue{16}{}{przyjęcie, włączenie jakiejś osoby lub organizacji do istniejącego już towarzystwa, stowarzyszenia}
\Clue{17}{}{pole wektorowe, w przypadku którego nie można wskazać w przestrzeni początku działania jego sił}
\Clue{18}{}{Astata - kosmopolityczny rodzaj drapieżnej błonkówki z podrodziny grzebaczowatych}
\Clue{19}{}{członek prymitywnego plemienia ludzi zamieszkujących jaskinie}
\Clue{20}{}{ORZESZNIK niezwykle twarde drzewo amerykańskie z rodziny orzechowatych, drewno używane w przemyśle lotniczym i do produkcji nart}
\Clue{21}{}{w chemii: symbol neonu}
\Clue{22}{}{Ginkgoopsida, Ginkgopsida - klasa roślin należąca do typu (gromady) nagonasiennych; szczątki kopalne znane są od dolnego permu, szczyt rozwoju grupa ta osiągnęła w mezozoiku}
\Clue{24}{}{pośmieciuszka, DZIERLATKA}
\Clue{26}{}{tradycyjny sposób sprawowania mszy wg obrządku łacińskiego}
\Clue{27}{}{niechęć do ludzi i lęk przed nimi, unikanie ich}
\Clue{28}{}{gatunek wielbłąda}
\Clue{29}{}{zespół kadłuba statku powietrznego wraz ze skrzydłami, usterzeniem, podwoziem i mechanizmami sterującymi}
\Clue{30}{}{modlitewne rytualne recytacje w japońskiej religii shinto wygłaszane w celu wybłagania u bogów deszczu i urodzaju}
\Clue{31}{}{skorupa ceramiczna, stanowiąca właściwą ściankę przedmiotu; wypalona, lecz nie szkliwiona i nie malowana}
\Clue{32}{}{w dawnej Polsce nazwa wojsk tatarskich}
\Clue{33}{}{falisty deseń na lufach strzelb ze stali damasceńskiej}
\Clue{34}{}{miasto i port w Szwecji nad Morzem Bałtyckim; prom do Świnoujścia}
\Clue{35}{}{kieł dzika}
\Clue{36}{}{młoda urocza kobieta, dziewczyna}\end{PuzzleClues}\newpage%\section*{Krzyżówka 21}

\noindent\begin{Puzzle}{24}{29}|*	|*	|*	|*	|[1][S]\drarr	|g	|o	|l	|[][,]{ }	|s	|a	|m	|o	|b	|ó	|j	|c	|z	|y	|*	|*	|*	|*	|[2][S]\darr	|*	|.
|*	|[3][S]\drarr	|w	|y	|k	|ł	|a	|d	|[][,]{ }	|h	|a	|b	|i	|l	|i	|t	|a	|c	|y	|j	|n	|y	|*	|r	|[4][S]\darr	|.
|*	|m	|[5][S]\rarr	|k	|a	|l	|a	|t	|e	|a	|[][,]{ }	|b	|u	|r	|l	|e	|[][,]{ }	|m	|a	|r	|x	|a	|*	|z	|j	|.
|*	|i	|*	|[6][S]\rarr	|m	|a	|t	|e	|m	|a	|t	|y	|k	|*	|*	|*	|*	|*	|*	|*	|*	|*	|*	|ą	|e	|.
|*	|g	|[7][S]\rarr	|w	|i	|c	|e	|s	|z	|e	|f	|*	|*	|*	|*	|*	|*	|*	|*	|*	|*	|*	|*	|d	|z	|.
|*	|d	|[8][S]\drarr	|m	|e	|l	|o	|d	|y	|k	|a	|[][,]{ }	|k	|a	|n	|t	|y	|l	|e	|n	|o	|w	|a	|*	|i	|.
|*	|a	|s	|*	|ń	|*	|*	|[9][S]\drarr	|l	|e	|p	|*	|[10][S]\drarr	|s	|t	|o	|n	|k	|a	|*	|[11][S]\drarr	|p	|u	|*	|o	|.
|*	|ł	|z	|*	|[][,]{ }	|*	|*	|a	|*	|[12][S]\rarr	|m	|o	|d	|n	|i	|a	|r	|z	|*	|*	|z	|[13][S]\darr	|*	|[14][S]\darr	|r	|.
|*	|e	|a	|*	|w	|*	|*	|u	|*	|[15][S]\rarr	|k	|l	|e	|e	|*	|[16][S]\drarr	|m	|e	|s	|m	|e	|r	|*	|a	|o	|.
|*	|k	|b	|*	|ę	|[17][S]\rarr	|s	|t	|r	|y	|j	|o	|s	|t	|w	|o	|*	|*	|[18][S]\darr	|*	|y	|ó	|[19][S]\darr	|d	|[][,]{ }	|.
|*	|[][,]{ }	|a	|*	|g	|[20][S]\rarr	|m	|o	|n	|g	|o	|ł	|k	|a	|*	|b	|*	|[21][S]\darr	|p	|*	|e	|ż	|p	|e	|m	|.
|*	|g	|s	|[22][S]\darr	|i	|*	|*	|m	|[23][S]\rarr	|s	|i	|z	|a	|l	|*	|l	|*	|p	|i	|*	|r	|a	|e	|n	|e	|.
|*	|a	|*	|j	|e	|[24][S]\drarr	|d	|a	|m	|s	|k	|i	|[][,]{ }	|k	|r	|a	|w	|i	|e	|c	|*	|[][,]{ }	|r	|o	|a	|.
|*	|r	|[25][S]\rarr	|a	|l	|b	|a	|t	|r	|o	|s	|*	|ś	|*	|*	|c	|*	|z	|s	|[26][S]\darr	|[27][S]\darr	|r	|s	|w	|n	|.
|*	|d	|*	|r	|n	|a	|*	|y	|*	|*	|[28][S]\rarr	|i	|n	|w	|a	|z	|j	|a	|*	|p	|z	|ó	|p	|i	|d	|.
|*	|ł	|[29][S]\darr	|z	|y	|r	|[30][S]\rarr	|k	|o	|w	|a	|l	|i	|k	|*	|k	|[31][S]\darr	|*	|[32][S]\darr	|i	|g	|ż	|e	|r	|r	|.
|*	|o	|f	|y	|*	|y	|*	|a	|*	|*	|*	|[33][S]\drarr	|e	|s	|*	|o	|k	|*	|k	|s	|o	|a	|k	|u	|o	|.
|[34][S]\drarr	|w	|i	|n	|a	|*	|*	|[][,]{ }	|*	|*	|[35][S]\darr	|m	|ż	|*	|*	|w	|a	|[36][S]\darr	|o	|e	|r	|ń	|t	|s	|w	|.
|p	|y	|r	|k	|*	|*	|[37][S]\rarr	|p	|a	|t	|r	|o	|n	|*	|[38][S]\darr	|a	|t	|u	|p	|m	|z	|c	|y	|[][,]{ }	|e	|.
|ę	|*	|m	|a	|*	|*	|[39][S]\rarr	|o	|p	|ł	|a	|t	|a	|[][,]{ }	|s	|t	|o	|s	|u	|n	|k	|o	|w	|a	|*	|.
|c	|*	|a	|*	|[40][S]\drarr	|g	|i	|g	|*	|*	|m	|y	|*	|*	|e	|e	|n	|k	|ł	|o	|n	|w	|a	|d	|*	|.
|h	|*	|m	|*	|a	|[41][S]\rarr	|b	|o	|j	|*	|o	|w	|*	|*	|m	|*	|*	|o	|a	|ś	|i	|a	|*	|[][S]-	|*	|.
|e	|*	|e	|*	|m	|*	|[42][S]\rarr	|d	|r	|g	|n	|i	|e	|n	|i	|e	|*	|k	|[][,]{ }	|ć	|e	|*	|*	|[][S]3	|*	|.
|r	|*	|n	|[43][S]\darr	|a	|[44][S]\rarr	|w	|o	|l	|e	|*	|c	|[45][S]\darr	|[46][S]\rarr	|t	|o	|k	|*	|l	|*	|n	|*	|*	|[][S]6	|*	|.
|z	|*	|t	|z	|d	|*	|[47][S]\rarr	|w	|a	|l	|*	|z	|k	|*	|k	|[48][S]\rarr	|p	|ł	|o	|m	|i	|e	|ń	|*	|*	|.
|n	|*	|*	|o	|e	|[49][S]\rarr	|j	|a	|s	|i	|e	|n	|i	|c	|a	|*	|[50][S]\rarr	|a	|d	|r	|e	|s	|a	|t	|*	|.
|i	|[51][S]\rarr	|k	|l	|u	|c	|z	|*	|*	|*	|*	|o	|l	|*	|*	|*	|[52][S]\rarr	|t	|o	|n	|*	|*	|*	|*	|*	|.
|c	|*	|*	|l	|s	|*	|*	|*	|*	|*	|*	|ś	|k	|*	|[53][S]\rarr	|o	|ł	|a	|w	|i	|a	|n	|k	|a	|*	|.
|a	|*	|*	|*	|*	|*	|*	|*	|*	|*	|*	|ć	|a	|*	|*	|[54][S]\rarr	|b	|i	|a	|ł	|o	|g	|o	|n	|*	|.
|*	|[55][S]\rarr	|s	|t	|r	|ó	|ż	|ó	|w	|k	|a	|*	|*	|*	|*	|*	|*	|*	|*	|*	|*	|*	|*	|*	|*	|.\end{Puzzle}

\newpage

\begin{PuzzleClues}{\textbf{Poziome}\\}\Clue{1}{}{w sportach zespołowych (w szczególności w piłce nożnej) skuteczny strzał w kierunku własnej bramki, przypadkowe uderzenie lub zmiana toru lotu piłki, którego efektem jest zdobycie gola (punktu) dla drużyny przeciwnej}
\Clue{3}{}{wykład wygłaszany na temat związany z pracą habilitacyjną, jeden z etapów nadawania tytułu dra habilitacyjnego}
\Clue{5}{}{Calathea burle-marxii - gatunek rośliny z rodziny marantowatych}
\Clue{6}{}{nauczyciel matematyki}
\Clue{7}{}{zastępca dyrektora, szefa, kierownika jakiejś placówki, instytucji}
\Clue{8}{}{typ melodyki w dziele muzycznym charakteryzujący się śpiewnym (lirycznym) przebiegiem melodii (bez dużych skoków, zawierający małą liczbę nut o krótkich wartościach, z zastosowaniem w przeważającym stopniu artykulacji legato), wolnym tempem, w odróżnieniu od melodyki figuracyjnej}
\Clue{9}{}{przenośnie: coś kuszącego i zdradliwego}
\Clue{10}{}{ameryk, chrząszcz, groźny szkodnik ziemniaka}
\Clue{11}{}{D: poliuretan, PUR - polimer powstający w wyniku addycyjnej polimeryzacji  wielofunkcyjnych izocyjanianów do amin i alkoholi; w jego głównym łańcuchu występuje ugrupowanie uretanowe}
\Clue{12}{}{mężczyzna zajmujący się zawodowo projektowaniem, szyciem i sprzedażą damskich ubrań}
\Clue{15}{}{szwajcarski malarz, grafik i teoretyk sztuki (1879-1940); wpływ kubizmu i surrealizmu}
\Clue{16}{}{lekarz niemiecki (1734-1815); twórca mesmeryzmu opartego na działaniu rzekomego 'magnetyzmu zwierzęcego'}
\Clue{17}{}{stryj oraz jego żona}
\Clue{20}{}{mieszkanka Mongolii, kobieta pochodzenia mongolskiego}
\Clue{23}{}{mocne włókno otrzymywane z liści agawy odznaczające się dużą odpornością na wpływy atmosferyczne, używane do wyrobu lin, worków, szpagatu itp}
\Clue{24}{}{mężczyzna, który ma same córki}
\Clue{25}{}{ptak wodny; poszczególne gatunki tego ptaka klasyfikowane są w taksonomii biologicznej w obrębie rodziny albatrosów (Diomedeidae)}
\Clue{28}{}{gwałtowne rozprzestrzenienie się czegoś lub kogoś, nagłe i masowe pojawienie się czegoś lub kogoś, natłok}
\Clue{30}{}{bargiel - ptak leśny z rzędu wróblowatych, owadożerny, chroniony}
\Clue{33}{}{w chemii: symbol einsteinu}
\Clue{34}{}{odpowiedzialność za niewłaściwe zachowanie}
\Clue{37}{}{szablon; tektura lub deseczka z wyciętym deseniem, który przez zamalowanie przenosi się na dekorowaną płaszczyznę}
\Clue{39}{}{opłata pobierana w sprawach o prawa majątkowe w wysokości obliczonej jako procent wartości przedmiotu sporu lub wartości przedmiotu zaskarżenia}
\Clue{40}{}{jednokonny powozik używany w XI X w. Anglii}
\Clue{41}{}{żartobliwie o strachu, lęku; współcześnie tylko jako człon frazymieć boja}
\Clue{42}{}{nagłe uczucie, poruszenie, np. drgnienie serca}
\Clue{44}{}{powiększenie tarczycy}
\Clue{46}{}{klepisko w stodole}
\Clue{47}{}{WALEŃ; wieloryb gładkoskóry}
\Clue{48}{}{przenośnie: uczucie, które tli się lub płonie, jest, żyje w kimś}
\Clue{49}{}{wieś w Polsce położona w województwie małopolskim, w powiecie myślenickim, w gminie Myślenice}
\Clue{50}{}{osoba, do której zaadresowano przesyłkę, list lub kartkę}
\Clue{51}{}{narzędzie do dokręcania lub odkręcania śrub, nakrętek itp}
\Clue{52}{}{w muzyce: miara odległości pomiędzy dźwiękami w skali}
\Clue{53}{}{mieszkanka Oławy}
\Clue{54}{}{hodowlana rasa gołębia polnego}
\Clue{55}{}{żona stróża}\end{PuzzleClues}

\begin{PuzzleClues}{\textbf{Pionowe}\\}\Clue{1}{}{podstawa czegoś, początek jakichś wielkich zmian}
\Clue{2}{}{naczelny wykonawczy i zarządzający organ państwa}
\Clue{3}{}{skupisko tkanki chłonnej w gardle, element pierścienia Waldeyera}
\Clue{4}{}{jezioro powstające między wałami brzegowymi aktywnego i opuszczonego (starorzecze) koryta rzeki meandrującej}
\Clue{8}{}{przerwa w tokowaniu cietrzewi}
\Clue{9}{}{system - zespół urządzeń stosowanych w budownictwie; jego zadaniem jest regulacja ilości ciepła służącego do ogrzewnia budynku, dostarczanego do instalacji wewnętrznej centralnego ogrzewania}
\Clue{10}{}{płat nieprzytwierdzonego do podłoża śniegu, który przesuwa się po oblodzonym podłożu; może stanowić sama lawinę śnieżną lub powodować poruszanie się innych kawałków śniegu, tworzących lawinę}
\Clue{11}{}{pisarz czeski, neoromantyk (1841-1901), powieści, opowiadania, poezje, nowele, dramaty}
\Clue{13}{}{każda kobieta, która należy do kółka różańcowego}
\Clue{14}{}{rodzaj adenowirusa wywołującego infekcje dróg oddechowych}
\Clue{16}{}{oblaczki, Syntomidae - rodzina motyli, rozprzestrzeniona na wszystkich kontynentach; najchętniej zamieszkują wszakże obszary gorące}
\Clue{18}{}{policjant}
\Clue{19}{}{otwarty, rozległy widok, panorama}
\Clue{21}{}{miasto we Włoszech (Toskania) nad rzeką Arno, międzynarodowe znaczenie turystyczne, tzw. krzywa wieża}
\Clue{22}{}{zdrobniale o jarzynie - jadalnej części jakiejś rośliny}
\Clue{24}{}{botanik niemiecki (1831-88); twórca podstaw mikologii}
\Clue{26}{}{to, że coś odbywa się pisemnie, na piśmie, za pomocą pisma}
\Clue{27}{}{trwały żal połączony z poczuciem goryczy i beznadziejności}
\Clue{29}{}{sfera niebieska: sfera, w której znajduje się obserwator i na której obserwuje się położenia ciał niebieskich; sklepienie niebieskie, niebo}
\Clue{31}{}{Marek Porcjusz Katon, ur. 234 r.p.n.e., zm. 149 r.p.n.e., mówca, polityk i pisarz rzymski}
\Clue{32}{}{kopulasta lub płaska pokrywa (mniejsza niż lądolód) wieloletniego śniegu lub lodu, pokrywająca wierzchowinę górską lub płaską powierzchnię lądową np. obszar arktycznej wyspy}
\Clue{33}{}{sposób ukształtowania utworu muzycznego wedle motywów}
\Clue{34}{}{Physocarpus - rodzaj roślin należący do rodziny różowatych (Rosaceae)}
\Clue{35}{}{lekarz francuski (1886-1963); immunolog, odkrył szczepionkę anatoksynę błoniczną i tężcową}
\Clue{36}{}{struktura skalna powstała w wyniku pęknięcia i przesunięcia się skał względem siebie}
\Clue{38}{}{przedstawicielka grupy ludów posługujących się językami należącymi do rodziny języków semickich, zamieszkującej Bliski Wschód i północną Afrykę}
\Clue{40}{}{bezodpływowe słone jezioro w środkowej Australii}
\Clue{43}{}{ur.1924 r. prawnik, profesor Uniwersytetu Jagiellońskiego, prezes Trybunału Konstytucyjnego}
\Clue{45}{}{Clupeonella - rodzaj ryby z rodziny śledziowatych}\end{PuzzleClues}\newpage%\section*{Krzyżówka 22}

\noindent\begin{Puzzle}{23}{27}|*	|*	|*	|[1][S]\drarr	|r	|g	|*	|[2][S]\drarr	|c	|z	|a	|p	|l	|a	|[][,]{ }	|a	|r	|a	|b	|s	|k	|a	|*	|[3][S]\darr	|.
|*	|*	|*	|a	|[4][S]\rarr	|ż	|a	|b	|n	|i	|c	|a	|*	|[5][S]\drarr	|g	|a	|ć	|*	|[6][S]\darr	|*	|[7][S]\darr	|*	|*	|l	|.
|*	|[8][S]\drarr	|b	|r	|a	|v	|a	|i	|s	|*	|[9][S]\drarr	|p	|a	|t	|i	|o	|*	|*	|t	|*	|o	|[10][S]\darr	|[11][S]\darr	|i	|.
|[12][S]\rarr	|l	|a	|t	|e	|n	|*	|s	|*	|*	|w	|[13][S]\rarr	|p	|u	|r	|i	|*	|*	|u	|*	|w	|b	|p	|g	|.
|*	|i	|*	|y	|*	|[14][S]\darr	|*	|[][S](	|*	|*	|a	|[15][S]\darr	|[16][S]\rarr	|p	|a	|r	|t	|*	|s	|*	|o	|a	|r	|a	|.
|*	|n	|[17][S]\drarr	|l	|a	|s	|k	|a	|*	|[18][S]\drarr	|d	|e	|k	|o	|m	|p	|e	|n	|s	|a	|c	|j	|a	|*	|.
|*	|k	|n	|e	|*	|o	|*	|z	|*	|k	|a	|d	|[19][S]\darr	|l	|[20][S]\drarr	|t	|o	|w	|o	|t	|*	|e	|s	|*	|.
|*	|e	|a	|r	|[21][S]\darr	|ł	|*	|o	|*	|u	|*	|i	|u	|e	|o	|*	|*	|*	|r	|[22][S]\darr	|*	|c	|a	|*	|.
|*	|*	|j	|i	|n	|t	|*	|t	|*	|r	|[23][S]\darr	|a	|r	|w	|c	|[24][S]\darr	|*	|*	|*	|t	|[25][S]\darr	|z	|[][,]{ }	|*	|.
|*	|*	|ś	|a	|i	|a	|[26][S]\drarr	|a	|r	|a	|b	|k	|a	|*	|h	|m	|*	|[27][S]\drarr	|c	|e	|t	|n	|o	|*	|.
|*	|*	|w	|*	|e	|n	|p	|n	|[28][S]\darr	|n	|o	|a	|n	|*	|o	|a	|*	|s	|*	|l	|e	|o	|d	|*	|.
|*	|*	|i	|*	|z	|*	|e	|[][S])	|s	|c	|d	|r	|o	|*	|j	|r	|*	|n	|*	|e	|a	|ś	|[][,]{ }	|*	|.
|*	|*	|ę	|*	|a	|*	|r	|[][,]{ }	|a	|i	|e	|a	|w	|*	|n	|y	|*	|i	|*	|t	|t	|ć	|o	|*	|.
|*	|*	|t	|*	|u	|*	|y	|s	|m	|k	|*	|n	|i	|[29][S]\darr	|i	|n	|*	|c	|*	|e	|r	|*	|w	|*	|.
|*	|*	|s	|*	|w	|*	|h	|t	|o	|*	|*	|*	|e	|o	|k	|a	|*	|a	|*	|c	|[][,]{ }	|*	|o	|*	|.
|*	|*	|z	|*	|a	|*	|e	|r	|i	|[30][S]\drarr	|d	|u	|c	|h	|*	|t	|*	|*	|*	|h	|m	|*	|c	|*	|.
|*	|*	|y	|*	|ż	|*	|l	|o	|s	|s	|*	|[31][S]\darr	|*	|r	|[32][S]\darr	|a	|*	|[33][S]\darr	|[34][S]\darr	|n	|u	|*	|ó	|*	|.
|*	|*	|[][,]{ }	|*	|e	|[35][S]\rarr	|i	|n	|t	|e	|r	|e	|s	|i	|k	|*	|[36][S]\darr	|a	|m	|i	|z	|*	|w	|*	|.
|*	|*	|s	|*	|n	|*	|u	|t	|n	|n	|[37][S]\darr	|m	|*	|d	|n	|*	|k	|s	|e	|k	|y	|*	|*	|*	|.
|*	|*	|a	|[38][S]\darr	|i	|*	|m	|u	|i	|a	|o	|i	|*	|*	|o	|*	|u	|p	|l	|a	|c	|*	|*	|*	|.
|*	|[39][S]\drarr	|k	|w	|e	|f	|*	|*	|e	|t	|d	|r	|*	|*	|t	|*	|r	|i	|o	|*	|z	|[40][S]\darr	|*	|*	|.
|*	|d	|r	|a	|*	|*	|*	|*	|n	|*	|p	|*	|*	|*	|*	|*	|u	|r	|d	|*	|n	|k	|*	|*	|.
|*	|*	|a	|z	|*	|[41][S]\rarr	|w	|e	|i	|p	|a	|*	|[42][S]\rarr	|p	|u	|l	|s	|a	|r	|*	|y	|a	|*	|*	|.
|*	|[43][S]\rarr	|m	|o	|w	|a	|[][,]{ }	|b	|e	|z	|d	|ź	|w	|i	|ę	|c	|z	|n	|a	|*	|*	|f	|*	|*	|.
|[44][S]\rarr	|m	|e	|n	|s	|a	|*	|*	|*	|*	|*	|*	|*	|*	|*	|*	|*	|t	|m	|*	|*	|t	|*	|*	|.
|*	|*	|n	|*	|*	|*	|*	|*	|*	|*	|*	|*	|*	|*	|*	|*	|*	|*	|a	|[45][S]\rarr	|j	|a	|z	|*	|.
|*	|[46][S]\rarr	|t	|ę	|ż	|y	|c	|z	|k	|a	|*	|*	|[47][S]\rarr	|p	|r	|z	|e	|s	|t	|r	|o	|n	|*	|*	|.
|*	|*	|*	|*	|*	|*	|*	|*	|*	|*	|*	|*	|*	|*	|*	|*	|*	|*	|*	|*	|*	|*	|*	|*	|.\end{Puzzle}

\newpage

\begin{PuzzleClues}{\textbf{Poziome}\\}\Clue{1}{}{w chemii: symbol pierwiastka roentgen}
\Clue{2}{}{Egretta gularis schistacea - podgatunek czapli rafowej (Egretta gularis)}
\Clue{4}{}{NAWĘD; głębinowa denna ryba z rodziny wędkarzy o długości do 1,5 m}
\Clue{5}{}{materiał do gacenia, czyli uszczelniania budynku, np. mech, słoma, liście}
\Clue{8}{}{francuski krystalograf, fizyk i matematyk (1811-63); udowodnił występowanie w kryształach 14 typów sieci przestrzennych}
\Clue{9}{}{wewnętrzny dziedziniec pałaców i domów hiszpańskich}
\Clue{12}{}{w archeologii - zespół cech kulturowych charakterystycznych dla ludności celtyckiej}
\Clue{13}{}{miasto w Indiach (Orisa) port nad Zatoką Bengalską}
\Clue{16}{}{grube, praktyczne i tanie płótno zwłaszcza konopne lub lniane}
\Clue{17}{}{spawadełko łukowe o małej mocy do spawania bardzo cienkich elementów}
\Clue{18}{}{załamanie się adaptacyjnych funkcji narządu w organizmie, objawiające się jego niewydolnością}
\Clue{20}{}{smar maszynowy do łożysk i powierzchni ślizgowych}
\Clue{26}{}{klacz czystej krwi arabskiej}
\Clue{27}{}{dawne określenie liczby parzystej (podzielnej przez 2)}
\Clue{30}{}{człowiek o określonych namiętnościach, usposobieniu}
\Clue{35}{}{mały zakład handlowy, niewielkie przedsiębiorstwo}
\Clue{39}{}{zasłona na twarz muzułmańskich kobiet - CZARCZAF}
\Clue{41}{}{górnicze miasto i port w Australii na zach. wybrzeży półwyspu Jork wydobycie boksytów}
\Clue{42}{}{gwiazda neutronowa wysyłająca w regularnych, niewielkich odstępach czasu impulsy promieniowania elektromagnetycznego (najczęściej promieniowanie radiowe)}
\Clue{43}{}{wada wymowy charakteryzująca się brakiem realizacji głosek dźwięcznych i zastępowaniu ich ich bezdźwięcznymi odpowiednikami, np.papcia zamiastbabcia}
\Clue{44}{}{świadczenia kleru parafialnego na utrzymanie biskupa i jego domowników}
\Clue{45}{}{budowla piętrząca wodę w rzece lub na kanale wznoszona w poprzek koryta; stała lub ruchoma}
\Clue{46}{}{choroba zwierząt, związana z niedoborami wapnia; zapadają na nią przede wszystkim młode suki tuż po porodzie}
\Clue{47}{}{część dużego pieca}\end{PuzzleClues}

\begin{PuzzleClues}{\textbf{Pionowe}\\}\Clue{1}{}{uzbrojenie armi artyleryjskiej}
\Clue{2}{}{nieorganiczny związek chemiczny, sól kwasu azotowego i strontu}
\Clue{3}{}{w systemie rozgrywek sportowych, szebel, ranga, klasa drużyn, które rywalizują i są na podobnym poziomie}
\Clue{5}{}{radziecki konstruktor lotniczy (1888-1972), organizator przemysłu lotniczego w ZSRR}
\Clue{6}{}{gatunek jedwabnika dębowego z rodziny pawic}
\Clue{7}{}{jadalna część rośliny, najczęściej owoc lub owocostan drzewa czy krzewu owocowego, którą można spożywać na surowo i która charakteryzuje się słodko-kwaskowatym smakiem, jest mięsista i soczysta}
\Clue{8}{}{pedagog austriacki (1884-1938); twórca tzw. nauczania łącznego}
\Clue{9}{}{uszkodzenie lub niedostatek wpływające negatywnie na ogólny stan czegoś poprzez obniżenie jego wartości}
\Clue{10}{}{cecha kogoś lub czegoś, co wydaje się wspaniałe, doskonałe, niezwykle piękne}
\Clue{11}{}{prasa służąca do wyciskania soku z owoców}
\Clue{14}{}{fizyk (1897-1959); specjalista w dziedzinie fizyki jądrowej, organizator Instytutu Badań Jądrowych}
\Clue{15}{}{formacja geologiczna z ostatniego okresu neoprotozoiku}
\Clue{17}{}{hostia będąca symbolem Chrystusa}
\Clue{18}{}{taniec francuski}
\Clue{19}{}{każdy promieniotwórczy pierwiastek chemiczny, o liczbie atomowej 92 lub większej}
\Clue{20}{}{mszyca powodująca na gałęziach świerka duże, szyszkowate wyrosła, przechodzi rozwój na modrzewiu i świerku}
\Clue{21}{}{nieświadome przeoczenie czegoś, pominięcie czegoś, zapomnienie o czymś}
\Clue{22}{}{technika przesyłania informacji drogą kablową lub radiową}
\Clue{23}{}{astronom niemiecki (1747-1826) rozpowszechnił prawidłowość w odległościach kolejnych planet od Słońca}
\Clue{24}{}{zalewa na bazie octu do marynowania przetworów, by zachowały trwałość}
\Clue{25}{}{przedstawienie, podczas którego oprócz tekstu mówionego wykorzystuje się śpiew i muzykę}
\Clue{26}{}{punkt orbity okołosłonecznej ciała niebieskiego położony najbliżej Słońca}
\Clue{27}{}{osada dyszla u wozu}
\Clue{28}{}{istnienie niezależnie od niczego, np. samoistnienie Boga, diabła, narodów}
\Clue{29}{}{OCHRYDA miasto w Macedonii nad Jeziorem Ochrydzkim; kąpielisko i ośrodek turystyczny}
\Clue{30}{}{budynek rządowy, w którym odbywają się posiedzenia senatu}
\Clue{31}{}{w okresie powstawania państwa muzułmańskiego dowódca wojskowy, a następnie wielkorządca podbitej prowincji z nadania kalifa, sprawujący jednocześnie władzę wojskową, administracyjną i finansową}
\Clue{32}{}{mały chłopiec, brzdąc}
\Clue{33}{}{kandydat, osoba ubiegająca się o coś, kandydująca do czegoś}
\Clue{34}{}{gatunek powieści, nasycony patetyczno-sentymentalnymi efektami oraz wątkiem miłosnym}
\Clue{36}{}{moneta w Imperium Osmańskim (tur. gurusz)}
\Clue{37}{}{pozostałość po czymś, uboczny rezultat jakiegoś działania}
\Clue{38}{}{ozdobne naczynie na cięte kwiaty}
\Clue{39}{}{nazwa literowa drugiego dźwięku w gamie, także od niej bierze oznaczenie tonacja, której toniką jest d}
\Clue{40}{}{luźne ubranie wierzchnie z rękawami, dziś używane jako ubranie robocze lub część stroju ludowego}\end{PuzzleClues}\newpage%\section*{Krzyżówka 24}

\noindent\begin{Puzzle}{22}{25}|*	|*	|*	|*	|*	|*	|[1][S]\drarr	|a	|n	|g	|l	|i	|s	|t	|k	|a	|*	|*	|*	|*	|[2][S]\darr	|[3][S]\darr	|*	|.
|*	|*	|[4][S]\drarr	|k	|o	|n	|s	|t	|r	|u	|k	|t	|y	|w	|n	|o	|ś	|ć	|*	|*	|h	|t	|*	|.
|[5][S]\drarr	|m	|e	|l	|o	|d	|i	|a	|*	|*	|*	|*	|*	|*	|*	|*	|*	|[6][S]\drarr	|k	|o	|r	|a	|*	|.
|ł	|[7][S]\drarr	|p	|r	|e	|f	|e	|r	|e	|n	|c	|j	|a	|*	|[8][S]\drarr	|c	|*	|d	|*	|[9][S]\darr	|k	|n	|*	|.
|u	|p	|a	|[10][S]\rarr	|k	|a	|w	|a	|l	|e	|r	|*	|*	|*	|t	|*	|[11][S]\darr	|w	|*	|e	|*	|d	|*	|.
|k	|i	|n	|*	|[12][S]\rarr	|g	|e	|r	|e	|n	|u	|k	|*	|*	|o	|*	|b	|u	|[13][S]\darr	|p	|*	|e	|*	|.
|[][,]{ }	|l	|a	|*	|[14][S]\rarr	|o	|c	|z	|e	|r	|e	|t	|*	|*	|m	|[15][S]\darr	|i	|s	|k	|e	|[16][S]\darr	|m	|*	|.
|b	|i	|s	|*	|[17][S]\drarr	|c	|z	|o	|p	|*	|*	|[18][S]\drarr	|s	|h	|i	|t	|s	|t	|o	|r	|m	|*	|*	|.
|r	|c	|t	|*	|b	|*	|k	|*	|[19][S]\rarr	|l	|o	|c	|o	|*	|c	|o	|t	|u	|ń	|n	|o	|*	|*	|.
|w	|z	|r	|[20][S]\drarr	|a	|p	|a	|r	|a	|t	|*	|z	|*	|*	|e	|p	|o	|z	|[][,]{ }	|a	|c	|*	|*	|.
|i	|a	|o	|k	|t	|*	|[][,]{ }	|*	|*	|[21][S]\rarr	|r	|y	|t	|m	|*	|i	|r	|ł	|t	|y	|[][,]{ }	|*	|*	|.
|o	|n	|f	|o	|*	|[22][S]\rarr	|p	|i	|k	|*	|[23][S]\rarr	|n	|o	|s	|e	|k	|*	|o	|r	|*	|c	|*	|*	|.
|w	|k	|a	|ś	|[24][S]\darr	|[25][S]\rarr	|u	|n	|i	|ż	|o	|n	|o	|ś	|ć	|*	|*	|t	|a	|*	|z	|*	|*	|.
|y	|a	|*	|ć	|k	|*	|s	|[26][S]\rarr	|p	|o	|w	|i	|e	|w	|*	|*	|*	|ó	|k	|*	|y	|*	|*	|.
|*	|*	|*	|[][,]{ }	|a	|*	|t	|*	|[27][S]\rarr	|e	|n	|k	|o	|d	|e	|r	|*	|w	|e	|*	|n	|*	|*	|.
|*	|[28][S]\drarr	|d	|u	|s	|z	|y	|c	|z	|k	|a	|*	|*	|*	|*	|*	|*	|k	|ń	|*	|n	|*	|*	|.
|*	|g	|*	|d	|z	|*	|n	|*	|*	|*	|*	|[29][S]\darr	|*	|*	|[30][S]\darr	|*	|[31][S]\darr	|a	|s	|*	|a	|*	|*	|.
|*	|o	|*	|o	|m	|*	|n	|*	|*	|*	|*	|m	|*	|*	|p	|*	|r	|*	|k	|*	|*	|*	|*	|.
|[32][S]\drarr	|d	|ź	|w	|i	|g	|a	|c	|z	|[][,]{ }	|d	|a	|c	|h	|o	|w	|y	|*	|i	|*	|*	|*	|*	|.
|s	|n	|*	|a	|r	|*	|*	|[33][S]\rarr	|f	|i	|l	|c	|*	|*	|l	|*	|t	|*	|*	|*	|*	|*	|*	|.
|y	|o	|*	|*	|c	|[34][S]\rarr	|k	|o	|m	|o	|r	|a	|[][,]{ }	|m	|i	|n	|o	|w	|a	|*	|*	|*	|*	|.
|l	|ś	|*	|*	|z	|*	|*	|*	|[35][S]\rarr	|p	|a	|k	|i	|e	|t	|*	|w	|*	|*	|*	|*	|*	|*	|.
|w	|ć	|[36][S]\rarr	|ż	|y	|ł	|k	|a	|*	|*	|*	|*	|*	|*	|y	|[37][S]\rarr	|n	|a	|g	|o	|ś	|ć	|*	|.
|e	|*	|[38][S]\rarr	|s	|k	|r	|z	|y	|d	|ł	|a	|*	|*	|[39][S]\rarr	|k	|s	|i	|ę	|g	|a	|r	|z	|*	|.
|k	|*	|*	|*	|*	|*	|[40][S]\rarr	|t	|e	|s	|t	|[][,]{ }	|p	|ł	|a	|t	|k	|o	|w	|y	|*	|*	|*	|.
|*	|*	|*	|*	|[41][S]\rarr	|k	|r	|z	|y	|ż	|u	|l	|e	|c	|*	|*	|*	|*	|*	|*	|*	|*	|*	|.\end{Puzzle}

\newpage

\begin{PuzzleClues}{\textbf{Poziome}\\}\Clue{1}{}{studentka filologii angielskiej}
\Clue{4}{}{to, że coś jest skuteczne i pouczające; np. konstruktywność krytyki}
\Clue{5}{}{uporządkowany szereg dźwięków, które tworzą pewną całość; melodia może być rozpatrywana jako element dzieła muzycznego, ale także - na zasadzie metonimii - może być nazwą samego utworu muzycznego}
\Clue{6}{}{zewnętrzna warstwa tkanki korzeni lub łodyg (pni), głównie gdy są zdrewniałe}
\Clue{7}{}{opcja, warunki, które wyróżniają jakąś grupę odbiorców, są korzystniejsze dla określonej grupy albo w określonych okoliczościach}
\Clue{8}{}{format pomocniczy w drukarstwie}
\Clue{10}{}{mężczyzna, który nie ma żony}
\Clue{12}{}{Litocranius walleri - gatunek antylopy z rodziny krętorogich, jedyny przedstawiciel rodzaju Litocranius; żyje we wschodniej Afryce}
\Clue{14}{}{zarośle roślin, szuwar przybrzeżny; zwykle w liczbie mnogiej}
\Clue{17}{}{występ pięty masztu}
\Clue{18}{}{specjalnie podsycana awantura, w której ktoś bierze udział dla zabawy (żeby kogoś wyprowadzić z równowagi) albo po to, by zwrócić uwagę na jakiś istotny fakt}
\Clue{19}{}{określenie w nutach unieważniające poprzednia znak wskazujący, że należy grać o oktawę wyżej lub niżej}
\Clue{20}{}{zespół osób, instytucji, które sprawują władzę w określonej dziedzinie życia publicznego, w państwie itp}
\Clue{21}{}{ugrupowanie artystyczne działające w latach 1922-32 w Warszawie, skupiało malarzy, grafików i rzeźbiarzy}
\Clue{22}{}{wolny koniec rei, gafla lub bomu}
\Clue{23}{}{część okularów, chroniąca nos przed otarciami, wykonana najczęściej z miękkiego, przezroczystego plastiku}
\Clue{25}{}{przesadna grzeczność, cecha zachowania, w którym ktoś stara się umniejszyć}
\Clue{26}{}{pojedynczy podmuch powietrza, jedna porcja wiatru}
\Clue{27}{}{urządzenie elektryczne, przetwornik, który służy do kodowania informacji}
\Clue{28}{}{człowiek; osoba}
\Clue{32}{}{w budownictwie jest to podstawowy element nośny konstrukcji dachu (więźby dachowej) przenoszący obciążenia na podpory główne (ściany lub słupy)}
\Clue{33}{}{wyrób włókienniczy z nieuporządkowanych włókien wełny lub sierści z domieszką wiskozy lub bawełny poddawanych spilśnieniu}
\Clue{34}{}{pomieszczenie kończące chodnik minowy znajdujące się pod lub w pobliżu elementu fortyfikacji przeznaczonego do zniszczenia}
\Clue{35}{}{w informatyce, telekomunikacji - zestaw danych; jednostka informacji}
\Clue{36}{}{nerw liściowy, który zawiera wiązki przewodzące}
\Clue{37}{}{cecha, fakt, że coś jest nagie - pozbawione osłony czy ozdób, widać to, bo nie jest zasłonięte, przeziera}
\Clue{38}{}{przenośnie: opieka, nadzór ze strony kogoś lub czegoś}
\Clue{39}{}{właściciel oficyny wydawniczej; określenie używane w odniesieniu do wydawców epoki nowożytnej}
\Clue{40}{}{test wykorzystywany w diagnostyce alergologicznej, polegający na naklejeniu badanemu na skórę pleców lub ramion plastra z naniesionym alergenem; test ten służy przede wszystkim wykrywaniu alergii kontaktowej}
\Clue{41}{}{metalowe wiązanie łańcuchów kotwicznych chroniące przed ich splątaniem}\end{PuzzleClues}

\begin{PuzzleClues}{\textbf{Pionowe}\\}\Clue{1}{}{Charadrius leschenaultii leschenaultii - nominatywny podgatunek ptaka wyróżniony w obrębie gatunku sieweczka pustynna (Charadrius leschenaultii)}
\Clue{2}{}{kod ISO 4217 kuny}
\Clue{3}{}{zaprzęg psów w podbiegunowych podróżach}
\Clue{4}{}{anastrofa - powtórzenie wyrazów w odwróconym porządku}
\Clue{5}{}{struktura kostna tworzona przez kość czołową, poprzecznie biegnący wał położony nad oczodołem; określenie niepoprawne z medycznego punktu widzenia}
\Clue{6}{}{banknot lub moneta (dawniej i czasem współcześnie w formie monety okolicznościowej) o wartości 200 zł}
\Clue{7}{}{mieszkanka Pilicy}
\Clue{8}{}{wieś w Polsce położona w województwie dolnośląskim, w powiecie wrocławskim, w gminie Jordanów Śląski}
\Clue{9}{}{miasto we Francji (Szampania) nad Marną}
\Clue{11}{}{polska nazwa handlowa przędzy o wysokiej puszystości i mniejszej od elastoru elastyczności}
\Clue{13}{}{traken - rasa konia pochodzenia pruskiego, którego przodkiem jest schweiken - silny pruski kuc, wyhodowany przez zakon krzyżacki; konie wszechstronnego użytku - potrafią świetnie skakać, sprawdzają się w konkurencjach WKKW oraz w ujeżdżeniu}
\Clue{15}{}{wodnik; słodkowodny pająk, buduje podwodne pajęczyny, gniazdo w kształcie dzwonu, w którym gromadzi powietrze}
\Clue{16}{}{w układach prądu przemiennego (również prądu zmiennego) część mocy, którą odbiornik pobiera ze źródła i zamienia na pracę lub ciepło}
\Clue{17}{}{najprostszy rodzaj wędki o długości od 2 do kilkunastu metrów}
\Clue{18}{}{czynnik iloczynu - element mnożony w działaniu mnożenia}
\Clue{20}{}{kość kończyny dolnej będąca elementem wspierającym tkanki miękkie uda}
\Clue{24}{}{mieszkaniec Kaszmiru}
\Clue{28}{}{tytuł przynoszący honor}
\Clue{29}{}{przyrząd służący do mierzenia długości, używany w kowalstwie}
\Clue{30}{}{działalność grupy społecznej, partii, osoby}
\Clue{31}{}{artysta, który zajmuje się rytowaniem - tworzeniem różnych wytworów graficznych takich, jak miedzioryty, drzeworyty, linoryty itd}
\Clue{32}{}{impreza sylwestrowa}\end{PuzzleClues}\newpage%\section*{Krzyżówka 25}

\noindent\begin{Puzzle}{25}{26}|*	|*	|*	|*	|*	|*	|*	|*	|*	|*	|*	|*	|*	|*	|[1][S]\drarr	|o	|b	|c	|i	|ą	|ż	|n	|i	|k	|*	|[2][S]\darr	|.
|*	|*	|*	|*	|*	|*	|[3][S]\rarr	|ł	|y	|ż	|w	|i	|a	|r	|k	|a	|[][,]{ }	|s	|z	|y	|b	|k	|a	|*	|[4][S]\darr	|j	|.
|*	|*	|*	|*	|[5][S]\darr	|*	|*	|[6][S]\rarr	|p	|o	|r	|t	|u	|g	|a	|l	|k	|a	|*	|*	|*	|[7][S]\darr	|*	|*	|p	|e	|.
|*	|*	|[8][S]\darr	|*	|s	|*	|[9][S]\rarr	|c	|z	|a	|r	|n	|a	|[][,]{ }	|r	|o	|b	|o	|t	|a	|*	|o	|*	|*	|r	|d	|.
|*	|*	|s	|*	|i	|*	|*	|*	|*	|*	|[10][S]\rarr	|a	|s	|e	|m	|b	|l	|e	|r	|*	|*	|c	|*	|*	|z	|y	|.
|*	|*	|i	|*	|n	|*	|*	|*	|*	|*	|*	|*	|[11][S]\rarr	|z	|e	|w	|n	|ę	|t	|r	|z	|e	|*	|*	|o	|n	|.
|*	|*	|ł	|*	|i	|[12][S]\rarr	|l	|i	|s	|t	|[][,]{ }	|o	|k	|ó	|l	|n	|y	|*	|*	|*	|*	|t	|*	|*	|d	|y	|.
|[13][S]\rarr	|ł	|o	|p	|a	|t	|a	|*	|*	|*	|[14][S]\drarr	|z	|a	|w	|i	|k	|ł	|a	|n	|i	|e	|*	|*	|*	|o	|[][,]{ }	|.
|*	|*	|w	|*	|c	|*	|[15][S]\rarr	|m	|a	|r	|s	|z	|r	|u	|t	|y	|z	|a	|c	|j	|a	|*	|*	|*	|w	|p	|.
|*	|*	|n	|*	|z	|*	|[16][S]\drarr	|g	|r	|a	|t	|k	|a	|*	|a	|*	|*	|*	|*	|*	|*	|*	|*	|*	|n	|i	|.
|*	|*	|i	|*	|e	|*	|m	|[17][S]\rarr	|z	|g	|r	|e	|d	|y	|*	|*	|*	|*	|[18][S]\drarr	|ż	|y	|ł	|a	|*	|i	|e	|.
|*	|*	|k	|[19][S]\darr	|k	|*	|i	|*	|[20][S]\rarr	|f	|e	|l	|c	|*	|*	|[21][S]\darr	|*	|*	|i	|*	|*	|*	|*	|*	|k	|r	|.
|*	|*	|[][,]{ }	|m	|[][,]{ }	|*	|r	|*	|*	|*	|p	|*	|[22][S]\rarr	|k	|o	|p	|r	|o	|l	|i	|t	|*	|[23][S]\darr	|[24][S]\darr	|[][,]{ }	|ś	|.
|*	|*	|p	|i	|p	|[25][S]\drarr	|t	|r	|z	|m	|i	|e	|l	|[][,]{ }	|s	|t	|e	|p	|o	|w	|y	|*	|a	|b	|p	|c	|.
|*	|*	|n	|ę	|l	|o	|*	|*	|*	|*	|t	|*	|*	|*	|*	|*	|*	|*	|r	|*	|*	|*	|g	|o	|r	|i	|.
|*	|*	|e	|d	|a	|c	|*	|[26][S]\rarr	|s	|k	|o	|r	|p	|u	|c	|h	|o	|w	|a	|t	|e	|*	|r	|l	|a	|e	|.
|*	|*	|u	|l	|m	|z	|*	|*	|*	|*	|s	|[27][S]\rarr	|s	|e	|r	|d	|u	|s	|z	|k	|o	|*	|e	|s	|c	|ń	|.
|*	|*	|m	|a	|i	|k	|*	|*	|*	|[28][S]\rarr	|o	|r	|g	|a	|n	|i	|z	|m	|[][,]{ }	|w	|y	|ż	|s	|z	|y	|*	|.
|*	|[29][S]\drarr	|a	|k	|s	|o	|l	|o	|t	|l	|*	|[30][S]\rarr	|d	|e	|g	|e	|n	|e	|r	|a	|t	|*	|t	|e	|*	|*	|.
|*	|a	|t	|*	|t	|*	|*	|[31][S]\drarr	|m	|i	|e	|s	|i	|ą	|c	|[][,]{ }	|s	|m	|o	|c	|z	|y	|*	|w	|*	|*	|.
|*	|u	|y	|*	|y	|*	|*	|m	|*	|*	|*	|[32][S]\rarr	|s	|z	|t	|r	|a	|n	|d	|o	|w	|a	|n	|i	|e	|*	|.
|*	|d	|c	|*	|*	|*	|*	|a	|*	|*	|[33][S]\rarr	|k	|s	|i	|ą	|ż	|ę	|[][,]{ }	|z	|[][,]{ }	|b	|a	|j	|k	|i	|*	|.
|*	|y	|z	|*	|[34][S]\rarr	|g	|o	|r	|y	|l	|[][,]{ }	|c	|r	|o	|s	|s	|[][,]{ }	|r	|i	|v	|e	|r	|*	|*	|*	|*	|.
|*	|t	|n	|[35][S]\rarr	|t	|r	|o	|c	|k	|i	|*	|*	|*	|[36][S]\rarr	|z	|i	|e	|m	|n	|i	|a	|k	|*	|*	|*	|*	|.
|*	|o	|y	|*	|*	|*	|*	|h	|[37][S]\rarr	|w	|o	|d	|a	|[][,]{ }	|u	|t	|l	|e	|n	|i	|o	|n	|a	|*	|*	|*	|.
|*	|r	|*	|[38][S]\rarr	|p	|o	|j	|e	|m	|n	|i	|c	|z	|e	|k	|*	|[39][S]\rarr	|b	|y	|s	|t	|r	|o	|ś	|ć	|*	|.
|*	|*	|[40][S]\rarr	|m	|r	|o	|k	|*	|[41][S]\rarr	|p	|r	|z	|y	|d	|r	|o	|ż	|e	|*	|*	|*	|*	|*	|*	|*	|*	|.\end{Puzzle}

\newpage

\begin{PuzzleClues}{\textbf{Poziome}\\}\Clue{1}{}{element sztangi}
\Clue{3}{}{kobieta uprawiająca łyżwiarstwo szybkie}
\Clue{6}{}{mieszkanka Portugalii, kobieta pochodzenia portugalskiego}
\Clue{9}{}{praca, która brudzi, np. praca kominiarza, drukarza; przenośnie}
\Clue{10}{}{niskopoziomowy język programowania bazujący na podstawowych operacjach procesora}
\Clue{11}{}{coś, co znajduje się na zewnątrz}
\Clue{12}{}{rodzaj pisma okólnego, występującego zwłaszcza w związkach, zrzeszeniach i innych organizacjach, których członkowie mają równorzędne prawa, a pismo jest kierowane do każdego z członków}
\Clue{13}{}{ręczne narzędzie do kopania składające się z blachy stalowej osadzonej na stylisku}
\Clue{14}{}{sytuacja, stosunek, sprawa trudna do rozwiązania lub zrozumienia, niejasna}
\Clue{15}{}{wyznaczanie optymalnych tras przewozowych}
\Clue{16}{}{specjalna okazja, wyjątkowo interesująca oferta, bardzo atrakcyjne, korzystne, sprzyjające okoliczności, szansa na coś, nie do przegapienia}
\Clue{17}{}{rodzice, także wszyscy starsi ludzie}
\Clue{18}{}{osoba szczupła, sprawna fizycznie, witalna i energiczna}
\Clue{20}{}{wrąb}
\Clue{22}{}{bogata w substancje mineralne, korzystna dla rozwoju roślin wydalina zwierząt żyjących w glebie}
\Clue{25}{}{Bombus laesus - gatunek z rodziny pszczołowatych, zaliczany do rodzaju trzmiel (Bombus) w podrodzinie pszczół właściwych}
\Clue{26}{}{Chelydridae - rodzina żółwi z podrzędu żółwi skrytoszyjnych; największe żółwie słodkowodne}
\Clue{27}{}{Dicentra - rodzaj bylin z rodziny makowatych}
\Clue{28}{}{człowiek lub zwierzę wyposażone w szkielet, którego osią jest kręgosłup łączący się z czaszką}
\Clue{29}{}{aksolotl meksykański}
\Clue{30}{}{człowiek zwyrodniały, odbiegający charakterologicznie od przeciętnego członka społeczeństwa}
\Clue{31}{}{średni czas, w jakim Księżyc przechodzi przez ten sam węzeł orbity, trwający 27 dni 5 godzin 5 minut i 35,9 sekundy (27,21222 dnia)}
\Clue{32}{}{zamierzone, celowe wejście statku na mieliznę co uchronić może przed zatonięciem}
\Clue{33}{}{ideał mężczyzny, ktoś, na kogo czeka kobieta, mężczyzna, w którym można się zakochać i z którym można być}
\Clue{34}{}{Gorilla gorilla diehli - podgatunek goryla zachodniego; występuje w Nigerii i Kamerunie}
\Clue{35}{}{rewolucjonista rosyjski, jeden z twórców i przywódców RFSRR i ZSRR}
\Clue{36}{}{KARTOFEL - bylina z rodziny psiankowatych pochodząca z Ameryki Płd}
\Clue{37}{}{komercyjnie dostępny roztwór nadtlenku wodoru w wodzie o stężeniu wynoszącym 3\%}
\Clue{38}{}{zawartość pojemniczka, małego pojemnika}
\Clue{39}{}{cecha czegoś, co szybko się porusza, płynie itp.; np. bystrość rzeki}
\Clue{40}{}{przenośnie: beznadzieja, smutek, żal, zło, czas, okoliczności, w których nie ma radości, jest tylko przygnębienie}
\Clue{41}{}{pobocze, pas przy drodze, często zarośnięty roślinnością polną}\end{PuzzleClues}

\begin{PuzzleClues}{\textbf{Pionowe}\\}\Clue{1}{}{członek katolickiego zakonu założonego przez krzyżowca Bertolda z Kalabrii w 1155 r}
\Clue{2}{}{fikcyjny artefakt ze stworzonej przez J. R. R. Tolkiena mitologii Śródziemia, najpotężniejszy z Pierścieni Władzy, zniszczony w ogniu Góry Przeznaczenia}
\Clue{4}{}{pracownik przewyższający innych pod względem wydajności}
\Clue{5}{}{Claravis pretiosa - gatunek ptaka z rodziny gołębiowatych (Columbidae)}
\Clue{7}{}{porcja octu, wodnego roztworu kwasu octowego; określona ilość octu, zazwyczaj szklana butelka}
\Clue{8}{}{rodzaj siłownika napędzanego gazowym czynnikiem roboczym, najczęściej odpowiednio przygotowanym powietrzem}
\Clue{14}{}{określenie wykonawcze; hałaśliwie}
\Clue{16}{}{krzew obszaru śródziemnomorskiego o drobnych zimotrwałych liściach - myrtol}
\Clue{18}{}{współczynnik stosowany w systemach społecznych niektórych krajów, ustalenie progu podatkowego zgodnie z zasadą, że im większa rodzina, tym ten próg jest niższy}
\Clue{19}{}{przyrząd do ręcznego międlenia skór}
\Clue{21}{}{w chemii: symbol platyny}
\Clue{23}{}{Ribes uva-crispa, porzeczka agrest - gatunek krzewu z rodziny agrestowatych; roślina uprawiana z uwagi na swoje owoce}
\Clue{24}{}{rosyjski komunista z okresu od 1903 roku (kiedy to bolszewicy uznali się za większościową frakcję Socjaldemokratycznej Partii Robotniczej Rosji) do powstania Związku Radzieckiego}
\Clue{25}{}{coś, co jest dla kogoś wyjątkowo istotne, o co się dba, pielęgnuje się to}
\Clue{29}{}{członek sądu kościelnego (może być świecki) przygotowujący materiał procesowy}
\Clue{31}{}{kraina we Francji u podnóża Masywy Centralnego}\end{PuzzleClues}\newpage%\section*{Krzyżówka 28}

\noindent\begin{Puzzle}{16}{16}|*	|*	|*	|*	|*	|*	|[1][S]\drarr	|w	|c	|i	|s	|k	|*	|*	|*	|*	|*	|.
|[2][S]\drarr	|g	|r	|e	|n	|a	|d	|a	|*	|*	|*	|*	|[3][S]\darr	|*	|*	|[4][S]\darr	|*	|.
|k	|*	|[5][S]\darr	|[6][S]\rarr	|p	|r	|z	|e	|w	|ó	|d	|*	|m	|*	|*	|k	|*	|.
|o	|[7][S]\drarr	|p	|ł	|ó	|c	|i	|e	|n	|n	|i	|c	|a	|*	|*	|o	|*	|.
|r	|p	|r	|*	|*	|*	|k	|[8][S]\drarr	|m	|a	|k	|a	|r	|t	|*	|z	|*	|.
|y	|r	|o	|*	|[9][S]\drarr	|b	|a	|r	|*	|*	|*	|*	|i	|[10][S]\darr	|[11][S]\darr	|i	|*	|.
|t	|o	|d	|*	|u	|*	|[][,]{ }	|z	|*	|*	|*	|*	|o	|b	|d	|u	|*	|.
|a	|t	|u	|[12][S]\drarr	|b	|i	|l	|a	|r	|d	|*	|*	|n	|a	|z	|ł	|*	|.
|r	|o	|k	|o	|o	|[13][S]\drarr	|o	|d	|r	|o	|d	|z	|e	|n	|i	|e	|*	|.
|z	|r	|t	|s	|l	|j	|k	|k	|*	|*	|*	|*	|t	|i	|e	|k	|*	|.
|o	|o	|y	|p	|e	|u	|a	|o	|*	|*	|*	|*	|k	|a	|c	|*	|*	|.
|w	|z	|w	|a	|w	|n	|t	|ś	|[14][S]\rarr	|a	|l	|b	|a	|*	|i	|[15][S]\darr	|*	|.
|y	|a	|n	|ł	|a	|g	|o	|ć	|[16][S]\rarr	|e	|l	|f	|*	|*	|u	|d	|*	|.
|*	|u	|o	|o	|n	|*	|r	|*	|[17][S]\rarr	|s	|h	|e	|m	|i	|c	|i	|*	|.
|*	|r	|ś	|ś	|i	|[18][S]\rarr	|k	|o	|l	|u	|g	|o	|*	|*	|h	|r	|*	|.
|*	|*	|ć	|ć	|e	|[19][S]\rarr	|a	|m	|i	|d	|e	|k	|*	|*	|*	|t	|*	|.
|*	|*	|*	|*	|*	|*	|*	|*	|*	|*	|*	|*	|*	|*	|*	|*	|*	|.\end{Puzzle}

\newpage

\begin{PuzzleClues}{\textbf{Poziome}\\}\Clue{1}{}{wciśnięcie, dociśnięcie czegoś siłą}
\Clue{2}{}{kraina w płd Hiszpanii nad wybrzeżem Morza Śródziemnego}
\Clue{6}{}{część obwodu elektrycznego (np. drut), która łączy źródło energii z odbiornikiem}
\Clue{7}{}{warsztat, w którym zajmowano się wyrabianiem płótna}
\Clue{8}{}{malarz austriacki (1840-84) kompozycje historyczne i alegoryczne o charakterze eklektycznym}
\Clue{9}{}{ludzie znajdujący się w barze - lokalu gastronomicznym, w którym konsumuje się zamówione wcześniej dania i napoje alkoholowe, głównie piwa oraz drinki i wódkę}
\Clue{12}{}{stół do gry w bilard}
\Clue{13}{}{technika produkowania oleodruków}
\Clue{14}{}{długa, biała szata liturgiczna, noszona przez wszystkich duchownych i usługujących wszystkich stopni w czasie liturgii obrządku łacińskiego oraz przez duchownych wielu kościołów protestanckich}
\Clue{16}{}{stworzenie rodem z mitologii germańskich (ale też wielu innych - bardzo rozpowszechniony wizerunek to elfy irlandzkie), skrzydlaty jasny człekokształtny duszek, chochlik, miewa magiczne moce}
\Clue{17}{}{jeden z ludów sprzymierzonych ze Stygijczykami w cyklu opowiadań o Conanie Barbarzyńcy Roberta Howarda}
\Clue{18}{}{lotokot filipiński, kaguan, lotokot, Cynocephalus volans - ssak łożyskowy z rzędu latawców; zamieszkuje lasy deszczowe na Filipinach, spotykany jest również na plantacjach bananowych, kokosowych i kauczukowych}
\Clue{19}{}{nieorganiczny związek chemiczny o wzorze ogólnym MNH2, gdzieM oznacza jon metalu; jest to sól, w której anion został utworzony przez oderwanie jednego z atomów wodoru od cząsteczki amoniaku}\end{PuzzleClues}

\begin{PuzzleClues}{\textbf{Pionowe}\\}\Clue{1}{}{kobieta, która zajęła lokal bez zgody właściciela}
\Clue{2}{}{osoba, która obsługuje dany korytarz, np. w więzieniu, w pociągu}
\Clue{3}{}{lalka teatralna poruszana od góry za pomocą nitek lub drucików zawieszonych na tzw. krzyżaku}
\Clue{4}{}{KOMARNICA - muchówka pokrojem zbliżona do komara lecz znacznie większa, larwy podgryzają korzenie roślin}
\Clue{5}{}{dawanie dużej ilości jakiegoś produktu; cecha roślin, zwierząt, terenu itd}
\Clue{7}{}{Protorosaurus jest nazwą najstarszego przedstawiciela grupy Archosauromorpha, żyjącego w późnym permie; jego szczątki odkryto w Niemczech już w roku 1706}
\Clue{8}{}{coś, co pojawia się lub występuje nieczęsto}
\Clue{9}{}{współczucie}
\Clue{10}{}{pękaty, kulisty przedmiot, najczęściej naczynie}
\Clue{11}{}{z pobłażaniem o człowieku młodym, postępującym nierozsądnie, jak dziecko}
\Clue{12}{}{to, że coś się dzieje powoli, bez dynamizmu}
\Clue{13}{}{JUNGA; chłopiec okrętowy}
\Clue{15}{}{dyscyplina rowerowa, która polega na wykonywaniu skoków i różnych ewolucji w powietrzu na specjalnie przygotowanym torze z zastosowaniem rowerów BMX lub MTB}\end{PuzzleClues}\newpage%\section*{Krzyżówka 29}

\noindent\begin{Puzzle}{18}{29}|*	|*	|[1][S]\drarr	|b	|e	|l	|l	|a	|d	|o	|n	|a	|*	|[2][S]\drarr	|l	|e	|w	|o	|*	|.
|*	|[3][S]\darr	|k	|[4][S]\darr	|[5][S]\drarr	|g	|a	|z	|a	|*	|[6][S]\darr	|*	|*	|w	|[7][S]\drarr	|ć	|m	|a	|*	|.
|[8][S]\drarr	|p	|r	|o	|b	|a	|b	|i	|l	|i	|z	|m	|*	|e	|ż	|[9][S]\darr	|[10][S]\darr	|[11][S]\darr	|*	|.
|c	|r	|ą	|b	|o	|*	|*	|*	|*	|*	|ę	|*	|*	|k	|ó	|p	|a	|g	|*	|.
|h	|z	|g	|i	|r	|*	|*	|[12][S]\rarr	|s	|i	|b	|u	|*	|t	|ł	|a	|m	|r	|*	|.
|o	|y	|l	|e	|o	|*	|[13][S]\darr	|[14][S]\darr	|[15][S]\drarr	|w	|a	|r	|t	|o	|w	|n	|i	|a	|*	|.
|n	|s	|i	|g	|k	|*	|s	|r	|h	|*	|t	|[16][S]\darr	|*	|r	|[][,]{ }	|a	|d	|b	|*	|.
|d	|t	|k	|*	|r	|[17][S]\drarr	|t	|u	|r	|i	|e	|c	|*	|[][,]{ }	|n	|s	|*	|*	|*	|.
|r	|r	|*	|*	|z	|d	|r	|c	|a	|[18][S]\darr	|k	|e	|*	|w	|o	|o	|[19][S]\darr	|[20][S]\darr	|*	|.
|o	|ó	|*	|*	|e	|w	|a	|h	|b	|t	|*	|r	|*	|a	|w	|n	|b	|g	|[21][S]\darr	|.
|s	|j	|[22][S]\darr	|*	|m	|u	|c	|a	|a	|a	|[23][S]\darr	|k	|[24][S]\darr	|h	|o	|i	|y	|r	|p	|.
|t	|*	|d	|*	|*	|w	|h	|n	|l	|r	|h	|i	|m	|a	|g	|c	|s	|u	|a	|.
|e	|*	|y	|*	|*	|i	|[][,]{ }	|i	|*	|n	|u	|e	|e	|d	|w	|*	|t	|b	|c	|.
|o	|*	|p	|*	|*	|j	|n	|e	|*	|i	|n	|w	|s	|ł	|i	|[25][S]\darr	|r	|o	|h	|.
|z	|*	|t	|[26][S]\darr	|[27][S]\drarr	|k	|a	|c	|z	|k	|a	|*	|m	|o	|n	|w	|z	|d	|o	|.
|a	|*	|y	|ś	|h	|a	|[][,]{ }	|*	|*	|*	|*	|*	|e	|w	|e	|i	|y	|z	|ł	|.
|u	|*	|k	|n	|e	|[][,]{ }	|w	|*	|*	|[28][S]\rarr	|s	|y	|r	|y	|j	|s	|k	|i	|*	|.
|r	|*	|[][,]{ }	|i	|r	|a	|r	|*	|*	|*	|*	|[29][S]\darr	|*	|*	|s	|k	|*	|ó	|*	|.
|*	|*	|k	|e	|z	|n	|ó	|*	|*	|[30][S]\rarr	|w	|d	|ó	|w	|k	|a	|*	|b	|*	|.
|*	|*	|o	|ż	|l	|t	|b	|*	|[31][S]\drarr	|c	|z	|e	|r	|n	|i	|c	|a	|*	|*	|.
|*	|*	|n	|y	|*	|a	|l	|*	|p	|[32][S]\rarr	|p	|l	|a	|n	|*	|z	|*	|*	|*	|.
|*	|*	|s	|c	|[33][S]\darr	|r	|e	|*	|i	|*	|*	|f	|[34][S]\darr	|*	|*	|a	|*	|*	|*	|.
|*	|*	|u	|a	|w	|k	|*	|[35][S]\drarr	|r	|o	|o	|i	|b	|o	|s	|*	|*	|*	|*	|.
|*	|*	|l	|[][,]{ }	|i	|t	|*	|c	|a	|*	|[36][S]\rarr	|n	|a	|r	|o	|g	|i	|*	|*	|.
|*	|[37][S]\darr	|a	|d	|e	|y	|*	|z	|c	|[38][S]\rarr	|r	|o	|s	|ł	|o	|ś	|ć	|*	|*	|.
|[39][S]\drarr	|k	|r	|u	|p	|c	|z	|a	|t	|k	|a	|*	|a	|*	|*	|*	|*	|*	|*	|.
|g	|o	|n	|ż	|r	|z	|*	|j	|w	|*	|[40][S]\rarr	|g	|ł	|o	|w	|i	|c	|a	|*	|.
|ę	|d	|y	|a	|z	|n	|[41][S]\rarr	|k	|o	|l	|o	|r	|y	|s	|t	|a	|*	|*	|*	|.
|ś	|*	|*	|*	|*	|a	|*	|a	|*	|*	|[42][S]\rarr	|o	|k	|n	|ó	|w	|k	|a	|*	|.
|*	|[43][S]\rarr	|r	|ó	|g	|*	|*	|*	|*	|*	|*	|*	|*	|*	|*	|*	|*	|*	|*	|.\end{Puzzle}

\newpage

\begin{PuzzleClues}{\textbf{Poziome}\\}\Clue{1}{}{wyciąg z owoców pokrzyku wilczej jagody, który zawiera różne alkaloidy, przez co jest bardzo ceniony w medycynie}
\Clue{2}{}{lewa strona}
\Clue{5}{}{prowincja w południowym Mozambiku o powierzchni 75 539 km2; stolicą prowincji jest Xai-Xai oddalone od Maputo (stolica Mozambiku) o 224 km}
\Clue{7}{}{mrok, ciemność, brak światła}
\Clue{8}{}{system etyczny stwierdzający, że w razie wątpliwości co do zastosowania zasad moralnych w danej sytuacji można iść za opinią prawdopodobną, choćby nawet opinia przeciwna była od niej prawdopodobniejsza}
\Clue{12}{}{miasto w Malezji na Bornego (Sarawał) port nad rzeką Rajang, port lotniczy}
\Clue{15}{}{miejsce - budynek lub pomieszczenie - w którym jest straż, np. żołnierze pełniący wartę}
\Clue{17}{}{kraina historyczna w środkowej Słowacji}
\Clue{27}{}{potrawa z mięsa kaczego}
\Clue{28}{}{makrojęzyk używany w Iraku i w Syrii, posiadający dwie odmiany: asyryjski oraz chaldejski}
\Clue{30}{}{afrykański ptak z rodziny wiktaczy, uprawiają pasożytnictwo lęgowe podobnie jak kukułki}
\Clue{31}{}{kaczka czernica, Aythya fuligula - gatunek średniego ptaka wodnego z rodziny kaczkowatych (Anatidae); zamieszkuje północną Eurazję w pasie od Islandii i Wielkiej Brytanii po Kamczatkę, na południu osiąga Europę Środkową, północną Mongolię i Hokkaido - zasadniczo jest to pas pomiędzy 70°N a 48°N}
\Clue{32}{}{graficzna, schematyczna i umowna reprezentacja przestrzeni (np. rysunek), która odwzorowuje układ elementów w przestrzeni; mapa}
\Clue{35}{}{susz z liści rooibosa, który służy do sporządzania naparu do picia}
\Clue{36}{}{jadalne wnętrzności zwierząt}
\Clue{38}{}{to, że ktoś jest rosły, potężnie zbudowany, wysoki, umięśniony itp}
\Clue{39}{}{porcja krupczatki, mąki pszennej, o zwiększonym ziarnie; określona ilość tego produktu, zazwyczaj papierowa torebka o pojemności np. 1 kg}
\Clue{40}{}{zakaźna, mało zaraźliwa choroba wirusowa bydła oraz bawołów}
\Clue{41}{}{osoba koloryzująca i sztucznie upiększająca swoje opowiadania}
\Clue{42}{}{gat. chronionej jaskółki gniazda na zewnątrz budynków}
\Clue{43}{}{zawartość rogu, naczynia lub pojemnika w kształcie zwierzęcego rogu}\end{PuzzleClues}

\begin{PuzzleClues}{\textbf{Pionowe}\\}\Clue{1}{}{ruchoma, mosiężna rurka włączana do obiegu instrumentu dętego blaszanego za pomocą wentyla; dzięki niej możliwe jest uzyskanie niższego dźwięku}
\Clue{2}{}{wektor genetyczny, który może utrzymać się i replikować w dwóch rodzajach komórek}
\Clue{3}{}{ozdoba czegoś; to, co zdobi coś innego}
\Clue{4}{}{powszechna obecność czegoś, funkcjonowanie}
\Clue{5}{}{szkło o wysokiej wytrzymałości chemicznej i temperaturowej zawierające telenek krzemu}
\Clue{6}{}{australijski ptak z rzędu wróblowatych, rodzina altanników}
\Clue{7}{}{miękkoskórek dwupazurzasty, żółw miękkoskóry, żółw dwupazurzasty, Carettochelys insculpta - gatunek gada z podrzędu żółwi skrytoszyjnych, jedyny przedstawiciel rodziny miękkoskórkowatych (Carettochelyidae); występuje w Nowej Gwinei i północnej Australii}
\Clue{8}{}{Chondrosteosaurus - rodzaj czworonożnego, roślinożernego zauropoda z rodziny kamarazaurów; żył w epoce wczesnej kredy na terenach obecnej Europy}
\Clue{9}{}{urządzenie elektroniczne wyprodukowane przez firmę Panasonic}
\Clue{10}{}{organiczny związek chemiczny zawierający grupę amidową}
\Clue{11}{}{bardzo twarde drewno pozyskiwane z drzewa o tej samej nazwie, wykorzystywane gł. w rzemiośle}
\Clue{13}{}{kukła odstraszająca ptaki z pól i sadów, w nowoczesnej formie także straszak na ptaki niemający kształtu kukły}
\Clue{14}{}{pulchny, lekko słodki, tradycyjny kujawski racuch z ciasta drożdżowego}
\Clue{15}{}{pisarz czeski, ur. 1914r, nowele, powieści; „Pociągi pod specjalnym nadzorem” „Postrzyżyny”, „Bar świata”, „Chrzest”}
\Clue{16}{}{świątynia obrządku prawosławnego lub katolickich obrządków wschodnich}
\Clue{17}{}{diksonia antarktyczna, Dicksonia antarctica - nazwa zwyczajowa gatunku paproci z rodziny diksoniowatych, pochodzącego z Australii}
\Clue{18}{}{pilnik o nacięciu punktowym do obróbki miękkich materiałów}
\Clue{19}{}{wspólna nazwa niewielkich, słodkowodnych ryb kąsaczowatych z Ameryki Południowej, w większości z rodzaju Hyphessobrycon}
\Clue{20}{}{GRABOŁUSK; ptak leśno-parkowy z rodziny łuszczaków o mocnym dziobie zdolnym do rozłupywania pestek, chroniony; Eurazja, płn. Afryka}
\Clue{21}{}{mocno osadzony na nabrzeżu lub równie mocno przymocowany do szkieletu jednostki pływającej (najczęściej na jej pokładzie) element spełniający rolę uchwytu}
\Clue{22}{}{powstający IV wieku n.e. rodzaj dyptyku; wytwarzany z okazji objęcia urzędu przez konsula, wykonywany najczęściej z kości słoniowej z wyrytym w środku portretem lub monogramem konsula}
\Clue{23}{}{zatoka Oceanu Atlantyckiego u północnych wybrzeży Islandii}
\Clue{24}{}{lekarz niemiecki (1734-1815); twórca mesmeryzmu opartego na działaniu rzekomego 'magnetyzmu zwierzęcego'}
\Clue{25}{}{Lagostomus maximus - gryzoń z rodziny szynszylowatych, jedyny żyjący przedstawiciel rodzaju Lagostomus; zamieszkuje południowo-wschodnią Boliwię, skrajne tereny zachodniego Paragwaju, większą część Argentyny aż do prowincji Rio Negro}
\Clue{26}{}{śnieżyca większa, Chen caerulescens - gatunek dużego ptaka wodnego z rodziny kaczkowatych (Anatidae)}
\Clue{27}{}{żydowski pisarz i publicysta (1860-1904), działał w Wiedniu, twórca i ideolog syjonizmu}
\Clue{29}{}{szermierz włoski, czterokrotny mistrz i dwukrotny wicemistrz olimpijski z Helsinek, Melbourne, Rzymu i Tokio}
\Clue{31}{}{korsarstwo, rozbójnictwo morskie}
\Clue{33}{}{wytrzebiony knur}
\Clue{34}{}{dziecko, zwłaszcza żywe, urwisek}
\Clue{35}{}{zwrotna, pełnomorska kozacka łódź bojowa o długości do 20 m i szerokości do 4 m}
\Clue{37}{}{ciąg składników sygnału (kombinacji sygnałów elementarnych, np. kropek i kresek, impulsów prądu, symboli) oraz reguła ich przyporządkowania składnikom wiadomości (np. znakom pisma)}
\Clue{39}{}{potrawa z mięsa gęsiego}\end{PuzzleClues}\newpage%\section*{Krzyżówka 30}

\noindent\begin{Puzzle}{19}{26}|*	|[1][S]\drarr	|k	|ą	|t	|*	|[2][S]\drarr	|s	|p	|ł	|u	|k	|i	|w	|a	|c	|z	|*	|*	|*	|.
|*	|m	|*	|[3][S]\rarr	|ł	|u	|k	|[][,]{ }	|r	|o	|m	|a	|ń	|s	|k	|i	|*	|*	|[4][S]\darr	|*	|.
|[5][S]\drarr	|i	|m	|i	|g	|r	|a	|n	|t	|*	|[6][S]\rarr	|s	|i	|e	|r	|p	|i	|e	|c	|*	|.
|m	|e	|[7][S]\rarr	|k	|i	|e	|l	|*	|*	|*	|[8][S]\darr	|*	|*	|[9][S]\darr	|*	|[10][S]\darr	|*	|*	|h	|*	|.
|r	|d	|*	|*	|*	|[11][S]\rarr	|k	|n	|o	|c	|k	|o	|u	|t	|*	|d	|*	|*	|r	|*	|.
|ó	|z	|*	|*	|[12][S]\rarr	|ł	|a	|d	|o	|w	|a	|n	|i	|e	|*	|y	|[13][S]\darr	|*	|y	|*	|.
|w	|i	|[14][S]\darr	|*	|[15][S]\darr	|*	|*	|[16][S]\drarr	|m	|o	|s	|t	|e	|k	|*	|s	|t	|[17][S]\darr	|s	|[18][S]\darr	|.
|k	|a	|z	|[19][S]\darr	|l	|[20][S]\darr	|[21][S]\drarr	|s	|e	|*	|y	|[22][S]\darr	|[23][S]\darr	|s	|*	|p	|r	|s	|t	|b	|.
|a	|n	|d	|s	|a	|c	|u	|z	|*	|[24][S]\darr	|t	|i	|s	|t	|*	|u	|ó	|i	|o	|a	|.
|*	|k	|r	|t	|n	|z	|c	|p	|[25][S]\drarr	|m	|a	|n	|t	|o	|*	|t	|j	|l	|f	|g	|.
|*	|a	|o	|r	|g	|ó	|h	|a	|t	|a	|*	|t	|a	|l	|[26][S]\darr	|a	|l	|n	|a	|a	|.
|*	|*	|w	|y	|l	|ł	|w	|j	|w	|*	|*	|e	|r	|i	|c	|c	|i	|i	|n	|c	|.
|*	|*	|o	|j	|e	|n	|y	|z	|d	|[27][S]\drarr	|b	|r	|y	|t	|y	|j	|s	|k	|i	|*	|.
|*	|[28][S]\drarr	|t	|o	|y	|o	|t	|a	|*	|f	|*	|m	|*	|*	|n	|a	|t	|[][,]{ }	|a	|*	|.
|*	|k	|n	|*	|*	|*	|*	|*	|*	|e	|*	|e	|*	|*	|k	|*	|[][,]{ }	|s	|*	|*	|.
|*	|o	|e	|*	|*	|[29][S]\rarr	|c	|z	|a	|r	|c	|z	|a	|f	|*	|*	|ż	|p	|*	|*	|.
|*	|n	|*	|[30][S]\rarr	|s	|a	|r	|y	|i	|g	|*	|z	|*	|*	|*	|*	|ó	|a	|*	|*	|.
|[31][S]\rarr	|s	|k	|r	|u	|b	|*	|[32][S]\rarr	|n	|a	|t	|o	|l	|i	|n	|*	|ł	|l	|[33][S]\darr	|*	|.
|*	|y	|[34][S]\rarr	|l	|e	|c	|y	|t	|y	|n	|a	|*	|[35][S]\rarr	|g	|e	|n	|t	|i	|l	|*	|.
|[36][S]\drarr	|l	|e	|n	|i	|u	|c	|h	|[][,]{ }	|o	|s	|p	|a	|ł	|y	|*	|y	|n	|o	|*	|.
|l	|i	|*	|[37][S]\drarr	|ż	|a	|b	|i	|ś	|c	|i	|e	|k	|*	|*	|*	|*	|o	|c	|*	|.
|u	|u	|*	|h	|*	|[38][S]\rarr	|u	|c	|i	|e	|c	|z	|k	|a	|*	|*	|[39][S]\darr	|w	|h	|*	|.
|k	|m	|*	|a	|[40][S]\drarr	|b	|r	|y	|t	|f	|a	|n	|k	|a	|*	|*	|p	|y	|*	|*	|.
|k	|*	|*	|s	|s	|*	|*	|*	|[41][S]\rarr	|a	|g	|e	|n	|e	|z	|j	|a	|*	|*	|*	|.
|a	|*	|*	|ł	|ą	|*	|[42][S]\rarr	|c	|e	|l	|o	|w	|n	|i	|k	|*	|s	|*	|*	|*	|.
|*	|*	|[43][S]\rarr	|o	|d	|p	|u	|s	|t	|*	|*	|*	|*	|*	|*	|*	|*	|*	|*	|*	|.
|*	|*	|*	|*	|*	|*	|*	|*	|*	|*	|*	|*	|*	|*	|*	|*	|*	|*	|*	|*	|.\end{Puzzle}

\newpage

\begin{PuzzleClues}{\textbf{Poziome}\\}\Clue{1}{}{miejsce, w którym się mieszka}
\Clue{2}{}{urządzenie do przeładowywania warzyw z wozów lub wagonów za pomocą strumienia wody}
\Clue{3}{}{łuk budowli, który ma kształt półkola}
\Clue{5}{}{osoba, która przybyła zza granicy do innego kraju w celu osiedlenia się}
\Clue{6}{}{gągoł północny, Bucephala islandica - gatunek ptaka z rodziny kaczkowatych (Anatidae); gniazduje głównie w północno-zachodnich rejonach Ameryki Północnej, a także na Islandii}
\Clue{7}{}{KILONIA; miasto w Niemczech, stolica Szlezwiku-Holsztynu}
\Clue{11}{}{słowo używane w zapisie, czyt. nokaut}
\Clue{12}{}{wprowadzanie naboju do komory nabojowej lub do lufy broni palnej}
\Clue{16}{}{element tokarki zawierający prowadnice}
\Clue{21}{}{w chemii: symbol selenu}
\Clue{25}{}{lanie, akt przemocy fizycznej, najczęściej: wiele silnych klapsów wymierzanych w pupę dziecka, czasem też inna forma bicia - także wśród dorosłych}
\Clue{27}{}{odmiana języka angielskiego używana przez mieszkańców Wielkiej Brytanii}
\Clue{28}{}{marka samochodu; japoński koncern motoryzacyjny Toyota Motor Corporation}
\Clue{29}{}{tradycyjna część ubioru muzułmańskich kobiet; zasłona okrywająca głowę wraz z twarzą, z otworem na oczy}
\Clue{30}{}{(1921-81), duński poeta, prozaik i eseista; „Przedpiekle”, „Nie zapomnij”}
\Clue{31}{}{australijska odmiana buszu; twardolistne zarośla wiecznie zielonych, krzewiastych gatunków}
\Clue{32}{}{zespół pałacowo-parkowy w Warszawie}
\Clue{34}{}{organiczny związek zaliczany do fosfolipidów, w którym reszta fosforanowa zestryfikowana jest choliną}
\Clue{35}{}{sokół złowiony w lecie, dający się łatwo ułożyć}
\Clue{36}{}{lori wysmukły, Loris tardigradus - gatunek niewielkiego ssaka z rodziny lorisowatych zamieszkujący Sri Lankę; z powodu powolnych ruchów zwany jest leniwcem}
\Clue{37}{}{drobna bylina stojących i powoli płynących wód. liście okrągłe, pływające, kwiaty białe, wonne}
\Clue{38}{}{przen. coś, w czym szuka się pomocy i pocieszenia}
\Clue{40}{}{zawartość brytfanki, metalowego naczynia służącego do pieczenia mięsa lub ciast}
\Clue{41}{}{zaburzenie rozwojowe polegające na niewykształceniu się zawiązka narządu, a w konsekwencji także tegoż narządu}
\Clue{42}{}{przyrząd w aparacie służący do wyznaczania granic kadru}
\Clue{43}{}{lokalna uroczystość kościelna organizowana z okazji święta patrona danego kościoła, zazwyczaj połączona z kiermaszem, festynem}\end{PuzzleClues}

\begin{PuzzleClues}{\textbf{Pionowe}\\}\Clue{1}{}{GNIEWOSZ}
\Clue{2}{}{stworzenie czegoś na wzór istniejącego już dzieła, odtworzenie}
\Clue{4}{}{zjawienie się Chrystusa po zmartwychwstaniu}
\Clue{5}{}{bardzo pracowita osoba}
\Clue{8}{}{przedstawiciel jednego ze starożytnych ludów Mezopotamii}
\Clue{9}{}{tworzywo sztuczne złożone z warstw płótna nasyconego żywicą fenolową i sprasowanego w podwyższonej temperaturze}
\Clue{10}{}{dysputa}
\Clue{13}{}{Trillium luteum - gatunek rośliny zielnej z rodziny melantkowatych}
\Clue{14}{}{składka na ubezpieczenie zdrowotne odprowadzana do zakładu ubezpieczeń publicznych}
\Clue{15}{}{amerykański heliofizyk i aerodynamik (1834-1906), badał podczerwoną część widma Słońca}
\Clue{16}{}{śląski deser przygotowywany z jajek kurzych}
\Clue{17}{}{silnik wykorzystujący sprężanie i rozprężanie gazu do wytworzenia momentu obrotowego lub siły}
\Clue{18}{}{zatoka Morza Południowochińskiego u wybrzeży Filipin (Płw. Bataan)}
\Clue{19}{}{wychodząca z powszechnego użycia nazwa relacji rodzinnej zachodzącej w stosunku do krewnego, który jest bratem ojca}
\Clue{20}{}{łódź wykonana z jednego pnia drzewa; wypalona lub wydłubana, płaskodenna, poruszana wiosłami lub drągiem}
\Clue{21}{}{element wyposażenia pojazdu służący pasażerom do trzymania się w czasie jazdy}
\Clue{22}{}{krótkie przejście, które łączy części utworu cyklicznego np. symfonii, dialogów}
\Clue{23}{}{czyjś ojciec, tato}
\Clue{24}{}{jednostka prędkości odnosząca się do obiektów poruszających się w płynie lub poruszających się płynów}
\Clue{25}{}{kod ISO 4217 dolara tajwańskiego}
\Clue{26}{}{związek chemiczny oparty na cynku, głównie stosowany w przemyśle}
\Clue{27}{}{Ferganocephale - rodzaj roślinożernego dinozaura z rodziny pachycefalozaurów, żyjący w okresie jury na terenach Azji; długość ciała 2 m, wysokość 70 cm, ciężar 50 kg}
\Clue{28}{}{narada kilku lekarzy w celu ustalenia u chorego prawidłowego rozpoznania choroby i właściwego leczenia}
\Clue{33}{}{podziemna część dużej budowli (zamku, kościoła) służąca zwłaszcza w czasach pokoju jako więzienie}
\Clue{36}{}{miasto we Włoszech (Toskania) 91,2 tys. mieszkańców (1981)}
\Clue{37}{}{wyrażenie, które umożliwia odróżnienie swoich od obcych, stosowane w wojsku czy w harcerstwie}
\Clue{39}{}{część maszyny, która służy do poruszania lub przemieszczania czegoś}
\Clue{40}{}{siedziba organu wymiaru sprawiedliwości}\end{PuzzleClues}\newpage%\section*{Krzyżówka 31}

\noindent\begin{Puzzle}{22}{26}|*	|*	|*	|*	|*	|*	|*	|*	|*	|[1][S]\drarr	|k	|a	|r	|c	|z	|o	|w	|i	|s	|k	|o	|*	|*	|.
|*	|*	|*	|*	|*	|*	|[2][S]\rarr	|p	|r	|z	|y	|s	|t	|a	|ń	|[][,]{ }	|m	|o	|r	|s	|k	|a	|*	|.
|*	|[3][S]\rarr	|p	|r	|o	|t	|e	|g	|o	|w	|a	|n	|y	|*	|*	|*	|*	|*	|[4][S]\drarr	|t	|ł	|o	|*	|.
|[5][S]\rarr	|w	|i	|e	|l	|k	|a	|[][,]{ }	|p	|ł	|y	|t	|a	|*	|*	|[6][S]\rarr	|d	|u	|g	|o	|ń	|*	|*	|.
|*	|[7][S]\darr	|[8][S]\rarr	|g	|o	|ł	|o	|g	|ł	|ó	|w	|*	|*	|[9][S]\rarr	|z	|i	|e	|l	|e	|*	|[10][S]\darr	|*	|*	|.
|*	|s	|*	|[11][S]\rarr	|p	|u	|ł	|a	|p	|k	|a	|*	|*	|[12][S]\darr	|[13][S]\darr	|*	|*	|[14][S]\rarr	|n	|o	|r	|i	|*	|.
|*	|t	|[15][S]\rarr	|ś	|m	|i	|e	|t	|a	|n	|k	|a	|*	|m	|k	|*	|*	|*	|[][,]{ }	|*	|o	|[16][S]\darr	|[17][S]\darr	|.
|[18][S]\rarr	|r	|a	|y	|*	|*	|[19][S]\darr	|[20][S]\darr	|*	|i	|*	|[21][S]\drarr	|t	|u	|w	|i	|m	|*	|h	|*	|z	|w	|c	|.
|*	|a	|[22][S]\darr	|*	|*	|*	|k	|k	|[23][S]\darr	|e	|*	|l	|*	|n	|a	|[24][S]\darr	|*	|*	|o	|*	|k	|y	|y	|.
|*	|ż	|k	|[25][S]\darr	|[26][S]\rarr	|t	|o	|u	|r	|n	|a	|i	|*	|d	|t	|z	|*	|*	|m	|*	|r	|s	|p	|.
|*	|n	|u	|g	|[27][S]\darr	|*	|z	|t	|a	|i	|[28][S]\darr	|w	|[29][S]\darr	|u	|e	|ł	|*	|*	|e	|*	|u	|p	|r	|.
|*	|i	|b	|ł	|s	|*	|a	|e	|d	|e	|p	|i	|s	|r	|r	|o	|[30][S]\darr	|[31][S]\darr	|o	|*	|s	|y	|z	|.
|[32][S]\rarr	|k	|r	|o	|k	|i	|*	|r	|i	|[][,]{ }	|o	|s	|z	|o	|o	|t	|j	|z	|t	|*	|z	|[][,]{ }	|y	|.
|*	|[][,]{ }	|a	|w	|a	|*	|*	|*	|o	|t	|d	|t	|y	|w	|d	|e	|e	|i	|y	|[33][S]\darr	|e	|s	|k	|.
|*	|m	|k	|a	|l	|[34][S]\drarr	|f	|i	|l	|o	|z	|o	|f	|i	|a	|[][,]{ }	|j	|ę	|z	|y	|k	|a	|*	|.
|*	|i	|*	|*	|n	|p	|*	|[35][S]\darr	|a	|r	|i	|n	|o	|e	|w	|m	|m	|b	|n	|o	|[][,]{ }	|l	|*	|.
|*	|e	|[36][S]\rarr	|p	|i	|a	|n	|a	|*	|b	|a	|i	|n	|c	|c	|y	|o	|a	|y	|r	|d	|o	|*	|.
|*	|j	|*	|*	|k	|w	|*	|t	|*	|i	|ł	|a	|*	|*	|a	|ś	|ś	|*	|*	|k	|r	|m	|*	|.
|*	|s	|*	|*	|*	|*	|*	|*	|*	|e	|*	|*	|*	|*	|*	|l	|ć	|*	|*	|*	|o	|o	|*	|.
|[37][S]\drarr	|k	|r	|z	|y	|ż	|[][,]{ }	|m	|a	|l	|t	|a	|ń	|s	|k	|i	|*	|*	|*	|*	|b	|n	|*	|.
|l	|i	|*	|*	|[38][S]\rarr	|m	|i	|ę	|s	|o	|ż	|e	|r	|c	|a	|*	|[39][S]\rarr	|b	|r	|o	|n	|a	|*	|.
|e	|*	|[40][S]\drarr	|s	|z	|a	|c	|h	|o	|w	|n	|i	|c	|a	|*	|[41][S]\rarr	|u	|s	|t	|n	|y	|*	|*	|.
|*	|[42][S]\rarr	|k	|o	|l	|e	|j	|*	|[43][S]\rarr	|a	|n	|o	|m	|o	|d	|o	|n	|t	|y	|*	|*	|*	|*	|.
|[44][S]\rarr	|h	|o	|m	|o	|z	|y	|g	|o	|t	|a	|[][,]{ }	|r	|e	|c	|e	|s	|y	|w	|n	|a	|*	|*	|.
|[45][S]\rarr	|u	|k	|ł	|a	|d	|[][,]{ }	|n	|i	|e	|s	|t	|a	|c	|j	|o	|n	|a	|r	|n	|y	|*	|*	|.
|*	|[46][S]\rarr	|s	|z	|u	|k	|a	|c	|z	|*	|*	|*	|*	|*	|*	|*	|*	|*	|*	|*	|*	|*	|*	|.
|*	|*	|*	|*	|*	|*	|*	|*	|*	|*	|*	|*	|*	|*	|*	|*	|*	|*	|*	|*	|*	|*	|*	|.\end{Puzzle}

\newpage

\begin{PuzzleClues}{\textbf{Poziome}\\}\Clue{1}{}{miejsce w lesie, z którego wyrąbano lub z którego wyrąbuje się drzewa, miejsce wykarczowane}
\Clue{2}{}{przystań nad morzem lub morskimi wodami śródlądowymi dla niewielkich jednostek pływających, objemująca infrastrukturę portową i akweny, ustanawiana przez dyrektora urzędu morskiego; termin prawniczy}
\Clue{3}{}{ktoś, kto jest traktowany w specjalny sposób, faworyzowany z powodu znajomości lub koneksji}
\Clue{4}{}{coś mało ważnego lub ktoś mało ważny}
\Clue{5}{}{osiedle złożone z bloków wybudowanych w technice wielkiej płyty}
\Clue{6}{}{diugoń, piersiopławka, Dugong dugon - morski ssak zaliczany do rzędu syren, jedyny współcześnie żyjący przedstawiciel rodziny diugoniowatych; zamieszkuje region indopacyficzny - od wschodnich wybrzeży Afryki i Morza Czerwonego po Australię, Wyspy Marshalla, Wyspy Salomona i Nową Kaledonię}
\Clue{8}{}{Pityriasis gymnocephala - gatunek ptaka, jedynego przedstawiciela monotypowej rodziny gołogłowów (Pityriaseidae), enigmatyczny i rzadki mieszkaniec lasów deszczowych Borneo}
\Clue{9}{}{stosunkowo rzadko spotykana nazwa marihuany}
\Clue{11}{}{Przyrząd do łapania zwierząt}
\Clue{14}{}{japońska nazwa określająca różne gatunki jadalnych wodorostów morskich z rodzaju Porphyra z gromady krasnorostów}
\Clue{15}{}{porcja słodkiej śmietany możliwa do nabycia w sklepie spożywczym}
\Clue{18}{}{ameryk malarz i fotografik (1890-1976) współzałożyciel grupy New York Dada}
\Clue{21}{}{Irena, ur. 1900r, siostra Juliana, poetka i tłumaczka, wspomnienia, zbiory wierszy}
\Clue{26}{}{miasto w Belgii nad Skaldą w V w. jeden z głównych ośrodków Franków salickich}
\Clue{32}{}{odgłos chodzenia}
\Clue{34}{}{dział filozofii podejmujący problem natury, pochodzenia oraz użycia języka}
\Clue{36}{}{mieszanina dyspersyjna, w której ośrodkiem rozpraszającym jest ciecz lub ciało stałe, a fazą rozproszoną gaz}
\Clue{37}{}{rozpowsechniony symbol, modyfikacja równoramiennego krzyża (oparty na krzyżu greckim; jego ramiona rozszerzają się jednak od środka i są rozwidlone na końcu, tradycyjnie jest biały, choć nie zawsze)}
\Clue{38}{}{zwierzę odżywiające się mięsem innych zwierząt (kręgowców)}
\Clue{39}{}{stosowana w średniowieczu drewniana krata, najczęściej okuta żelazem, zamykająca warowną bramę}
\Clue{40}{}{Fritillaria - rodzaj roślin należący do rodziny liliowatych (Liliaceae)}
\Clue{41}{}{egzamin ustny, przebiegający w formie rozmowy}
\Clue{42}{}{pociąg, lokomotywa i wagony}
\Clue{43}{}{Anomodontia - podrząd gadów ssakokształtnych z rzędu terapsydów; żyły od środkowego permu do późnego triasu na wszystkich kontynentach z wyjątkiem Australii}
\Clue{44}{}{organizm (komórka), w którym oba allele danego genu są recesywne (aa)}
\Clue{45}{}{układ, którego wyjście zależy wprost od czasu}
\Clue{46}{}{przyrząd do wyznaczania trasy zakopanego kabla}\end{PuzzleClues}

\begin{PuzzleClues}{\textbf{Pionowe}\\}\Clue{1}{}{wrodzona choroba uwarunkowana genetycznie polegająca na zaburzeniu wydzielania przez gruczoły zewnątrzwydzielnicze}
\Clue{4}{}{gen kotrolujący rozwój morfologiczny poszczególnych części ciała w początkowych stadiach rozwoju zarodkowego, zarówno u bezkręgowców jak i kręgowców}
\Clue{7}{}{mundurowy, funkcjonariusz straży miejskiej - instytucji powołanej w celu utrzymywania porządku publicznego i bezpieczeństwa ludzi, która podlega lokalnemu samorządowi, a nie władzy centralnej}
\Clue{10}{}{Tyrophagus putrescentiae - gatunek roztocza z rodziny rozkruszków}
\Clue{12}{}{policjant ubrany w mundur}
\Clue{13}{}{nazwa osob lub firmy odpowiedzialnej za zakwaterowanie, najczęściej spotykana w regulacjach prawnych dotyczących turystyki}
\Clue{16}{}{państwo wyspiarskie w południowo-wschodniej Oceanii}
\Clue{17}{}{żywiczlin, kieliszniak, Tetraclinis, Callitris - rodzaj drzew z rodziny cyprysowatych (Cupressaceae); obejmuje 16 gatunków występujących w Australii oraz na Nowej Kaledonii}
\Clue{19}{}{żartobliwie o więzieniu}
\Clue{20}{}{jednomasztowy jacht żaglowy mający dwa albo trzy przednie żagle (sztaksle)}
\Clue{21}{}{palma o wachlarzowatych liściach z kolczastymi ogonkami, w Polsce roślina doniczkowa}
\Clue{22}{}{męski kontusz z płótna lub sukna noszony zwykle przez służbę w XVIII w}
\Clue{23}{}{aparat radiowy umieszczony w jednej skrzynce z magnetofonem}
\Clue{24}{}{rodzaj pamiętnika popularny w latach 80.-90. XX wieku}
\Clue{25}{}{o górnej zaokrąglonej części czegoś}
\Clue{27}{}{robotnik pracujący w kopalni odkrywkowej}
\Clue{28}{}{podział czegoś na części, elementy}
\Clue{29}{}{cienka, przezroczysta i przewiewna tkanina jedwabna lub bawełniana o splocie płóciennym, używana na bieliznę, suknie, szale itp}
\Clue{30}{}{nieznajoma kobieta, zazwyczaj - starsza lub przy kości; słowo używane dziś żartobliwie}
\Clue{31}{}{drobny ptak leśno-parkowy z rzędu wróblowatych o barwnym upierzeniu, w Polsce chroniona}
\Clue{33}{}{miasto w Anglii nad rzeką Ouse, ośrodek administracyjny hrabstwa North Yorkshire}
\Clue{34}{}{ptak z rodziny kurowatych (Phasianidae)}
\Clue{35}{}{jednostka zdawkowa w Laosie; 1/100 kipa}
\Clue{37}{}{skrót/symbol funta egipskiego}
\Clue{40}{}{steryd anaboliczny stosowany przez sportowców}\end{PuzzleClues}\newpage%\section*{Krzyżówka 32}

\noindent\begin{Puzzle}{23}{29}|*	|*	|*	|*	|[1][S]\darr	|*	|[2][S]\darr	|[3][S]\drarr	|s	|a	|l	|a	|[][,]{ }	|p	|l	|e	|n	|a	|r	|n	|a	|*	|*	|*	|.
|*	|*	|*	|*	|g	|[4][S]\darr	|l	|w	|[5][S]\darr	|[6][S]\darr	|*	|*	|*	|*	|*	|*	|*	|*	|*	|[7][S]\darr	|[8][S]\darr	|*	|*	|*	|.
|*	|[9][S]\darr	|*	|*	|i	|t	|u	|a	|r	|z	|*	|*	|[10][S]\darr	|*	|*	|*	|*	|[11][S]\drarr	|p	|i	|g	|w	|a	|*	|.
|*	|s	|*	|[12][S]\rarr	|g	|o	|t	|l	|a	|n	|d	|i	|a	|*	|[13][S]\darr	|*	|[14][S]\darr	|b	|[15][S]\drarr	|g	|a	|z	|*	|*	|.
|*	|i	|*	|*	|*	|r	|ó	|o	|k	|*	|*	|*	|l	|[16][S]\darr	|p	|*	|f	|u	|h	|ł	|l	|*	|[17][S]\darr	|*	|.
|*	|o	|*	|*	|*	|*	|w	|r	|i	|*	|*	|*	|a	|s	|a	|[18][S]\darr	|e	|j	|u	|a	|a	|*	|d	|*	|.
|[19][S]\rarr	|ł	|u	|n	|n	|i	|k	|*	|*	|*	|*	|[20][S]\drarr	|w	|o	|d	|o	|l	|o	|t	|*	|n	|*	|o	|*	|.
|*	|o	|[21][S]\rarr	|m	|a	|g	|a	|d	|a	|n	|*	|g	|i	|n	|d	|k	|d	|w	|t	|*	|g	|*	|d	|*	|.
|*	|*	|*	|*	|*	|*	|*	|*	|*	|*	|*	|o	|t	|k	|l	|r	|f	|i	|o	|*	|a	|*	|a	|*	|.
|*	|[22][S]\darr	|[23][S]\darr	|[24][S]\drarr	|b	|i	|o	|g	|e	|n	|*	|n	|a	|o	|e	|ę	|e	|s	|n	|*	|l	|*	|t	|*	|.
|*	|i	|s	|c	|*	|*	|[25][S]\drarr	|v	|e	|r	|g	|a	|*	|j	|*	|t	|b	|k	|*	|*	|[][,]{ }	|*	|e	|*	|.
|*	|n	|ł	|h	|*	|*	|w	|[26][S]\rarr	|s	|u	|r	|m	|i	|a	|*	|[][,]{ }	|e	|o	|[27][S]\darr	|*	|w	|*	|k	|*	|.
|[28][S]\drarr	|k	|o	|l	|o	|r	|y	|m	|e	|t	|r	|i	|a	|*	|*	|s	|l	|*	|c	|*	|i	|*	|[][,]{ }	|*	|.
|d	|w	|w	|a	|[29][S]\rarr	|s	|p	|r	|z	|ę	|t	|*	|*	|*	|*	|z	|*	|*	|e	|*	|ę	|*	|b	|*	|.
|r	|i	|i	|p	|[30][S]\rarr	|p	|ł	|ó	|z	|*	|[31][S]\rarr	|s	|z	|r	|a	|p	|n	|e	|l	|*	|k	|[32][S]\darr	|r	|*	|.
|a	|z	|a	|n	|[33][S]\rarr	|s	|a	|r	|o	|n	|g	|*	|*	|[34][S]\rarr	|r	|i	|z	|e	|*	|*	|s	|p	|a	|*	|.
|m	|y	|ń	|i	|[35][S]\rarr	|ś	|w	|i	|e	|r	|k	|[][,]{ }	|v	|e	|i	|t	|c	|h	|a	|*	|z	|i	|n	|*	|.
|a	|c	|s	|ę	|*	|*	|e	|*	|[36][S]\rarr	|s	|u	|l	|f	|o	|n	|a	|m	|i	|d	|*	|y	|a	|ż	|*	|.
|t	|y	|k	|c	|*	|*	|k	|[37][S]\darr	|*	|*	|[38][S]\rarr	|p	|a	|r	|a	|l	|a	|l	|i	|a	|*	|c	|o	|*	|.
|y	|j	|o	|i	|*	|*	|*	|m	|*	|*	|[39][S]\rarr	|r	|y	|s	|u	|n	|e	|k	|*	|*	|*	|e	|w	|*	|.
|c	|n	|ś	|e	|*	|[40][S]\rarr	|r	|o	|z	|w	|ó	|j	|[][,]{ }	|p	|s	|y	|c	|h	|i	|c	|z	|n	|y	|*	|.
|z	|o	|ć	|[][,]{ }	|*	|*	|*	|s	|[41][S]\rarr	|c	|z	|o	|s	|k	|a	|*	|[42][S]\rarr	|l	|e	|g	|a	|t	|*	|*	|.
|n	|ś	|*	|j	|[43][S]\rarr	|p	|a	|t	|r	|z	|a	|ł	|k	|i	|*	|*	|*	|[44][S]\rarr	|d	|a	|n	|i	|e	|*	|.
|o	|ć	|*	|ę	|*	|*	|*	|*	|*	|*	|*	|*	|*	|*	|*	|*	|[45][S]\rarr	|s	|i	|l	|i	|n	|g	|*	|.
|ś	|*	|[46][S]\rarr	|z	|w	|r	|o	|t	|*	|*	|*	|*	|*	|*	|[47][S]\rarr	|p	|e	|r	|g	|a	|m	|i	|n	|*	|.
|ć	|[48][S]\rarr	|l	|o	|d	|o	|w	|i	|e	|c	|[][,]{ }	|h	|i	|m	|a	|l	|a	|j	|s	|k	|i	|*	|*	|*	|.
|*	|[49][S]\rarr	|p	|r	|o	|t	|o	|k	|ó	|ł	|[][,]{ }	|a	|k	|c	|e	|s	|y	|j	|n	|y	|*	|*	|*	|*	|.
|*	|*	|[50][S]\rarr	|e	|l	|e	|g	|a	|n	|t	|*	|*	|*	|*	|*	|*	|*	|*	|*	|*	|*	|*	|*	|*	|.
|[51][S]\rarr	|k	|o	|m	|i	|s	|j	|a	|[][,]{ }	|b	|u	|d	|ż	|e	|t	|o	|w	|a	|*	|*	|*	|*	|*	|*	|.
|*	|*	|*	|*	|*	|*	|*	|*	|*	|*	|*	|*	|*	|*	|*	|*	|*	|*	|*	|*	|*	|*	|*	|*	|.\end{Puzzle}

\newpage

\begin{PuzzleClues}{\textbf{Poziome}\\}\Clue{3}{}{sala do obrad plenarnych odbywających się w gmachu ważnej instytucji}
\Clue{11}{}{owoc (jabłkowaty, wielopestkowiec) pigwy pospolitej}
\Clue{12}{}{wyspa na Bałtyku należąca do Szwecji}
\Clue{15}{}{pedał gazu}
\Clue{19}{}{radziecki pojazd kosmiczny wysłany w kierunku Księżyca dla dokonania badań}
\Clue{20}{}{hydropłat; statek wodny z płatami nośnymi pod kadłubem; osiąga znaczne szybkości}
\Clue{21}{}{miasto obwodowe w azjatyckiej części Federacji Rosyjskiej, port nad Morzem Ochockim}
\Clue{24}{}{związek lub pierwiastek chemiczny (albo ich kombinacje), substancja, która występuje w środowisku naturalnie albo jest do niego dostarczana (np. w postaci nawozów, karm, pasz) i która zapewnia warunki do życia i rozwoju określonych organizmów; najczęściej wyraz używany jest w liczbie mnogiej}
\Clue{25}{}{pisarz włoski (1840-1922), przedstawiciel weryzmu; „Rodzina Malavogliów”, „Rycerskość wieśniacza”}
\Clue{26}{}{katalpa, Catalpa - rodzaj roślin należących do rodziny bignoniowatych; należy do niego 11 gatunków pochodzących z Azji i Ameryki Północnej}
\Clue{28}{}{dział psychofizyki (optyki) zajmujący się ilościowym opisem i charakterystyką barw postrzeganych przez człowieka lub zwierzęta}
\Clue{29}{}{przedmiot użytkowy, np. mebel, kuchenka}
\Clue{30}{}{część pługa ułatwiająca jego prowadzenie}
\Clue{31}{}{pocisk artyleryjski używany do rażenia ludzi}
\Clue{33}{}{rodzaj spódnicy upiętej z jednego płata tkaniny, którą noszą kobiety i mężczyźni na Malajach, Jawie i w Tajlandii}
\Clue{34}{}{miasto w płn.-wsch. Turcji, port nad Morzem Czarnym}
\Clue{35}{}{Picea neoveitchii - gatunek z rodziny sosnowatych}
\Clue{36}{}{sulfamid - organiczny związek chemiczny będący amidem kwasu organosulfonowego}
\Clue{38}{}{wada wymowy polegająca na substytucji (zastępowaniu) jednej głoski inną, np. klowa zamiast krowa}
\Clue{39}{}{każda ilustracja, coś, co jest narysowane, niesie informację poprzez graficzne przedstawienie}
\Clue{40}{}{zmiany, jakie zachodzą w umyśle człowieka od momentu urodzenia aż do dorosłości}
\Clue{41}{}{zatoka Morza Barentsa u wybrzeży Federacji Rosyjskiej (Płw. Kanin)}
\Clue{42}{}{w Kościele katolickim duchowny występujący w imieniu papieża, jego przedstawiciel}
\Clue{43}{}{żartobliwie: przyrząd optyczny, zbudowany z pary szkieł i oprawy, umożliwiającej umocowanie szkieł przed oczami, najczęściej za pomocą zauszników, służący, z reguły, do poprawiania ostrości widzenia, osłabionej przez chorobę lub uraz oka albo przez wiek człowieka}
\Clue{44}{}{część posiłku; dania są podawane kolejno, składają się na nie konkretne potrawy}
\Clue{45}{}{bezodpływowe, słone jezioro w Chinach, na Wyżynie Tybetańskiej}
\Clue{46}{}{w komunikacji: pewien wyraz lub konstrukcja wyrazowa o określonej funkcji (najczęściej fatycznej) lub znaczeniu}
\Clue{47}{}{potocznie: papier pergaminowy}
\Clue{48}{}{lodowiec górski, który w swym systemie zasilania ma kilka łączących się strumieni lodowych - rozgałęzień, z których każdy jest co najmniej lodowcem dolinnym}
\Clue{49}{}{umowa regulująca zasady członkostwa jakiegoś kraju we wspólnocie międzynarodowej}
\Clue{50}{}{elegancki mężczyzna}
\Clue{51}{}{organ zajmujący się systemem finansowym państwa}\end{PuzzleClues}

\begin{PuzzleClues}{\textbf{Pionowe}\\}\Clue{1}{}{jednokonny powozik używany w XIX w. Anglii}
\Clue{2}{}{płyn, który ułatwia płynięcie lutu}
\Clue{3}{}{rzeźbiarz włoski ur. w 1906 r., kompozycje abstrakcyjne}
\Clue{4}{}{droga ułożona z szyn umocowanych na podłożu za pomocą podkładów}
\Clue{5}{}{wierszowany utwór literacki, który da się odczytać na dwa sposoby - od lewej do prawej i od prawej do lewej, a czasem również od początku do końca i od końca do początku}
\Clue{6}{}{w chemii: symbol cynku}
\Clue{7}{}{(gramofonowa) element przenoszący przebiegi akustyczne, zapisane na płycie za pomocą rowków na system drgający adaptera}
\Clue{8}{}{Alpinia galanga - gatunek okazałej byliny zaliczanej do rodziny imbirowatych}
\Clue{9}{}{wieś, wioska, osada rolnicza; miejsce zamieszkane przez ludność rolniczą}
\Clue{10}{}{NUSAJRYTA; członek sekty muzułmańskiej założonej w IX w.; skrajny odłam szyitów}
\Clue{11}{}{coś, co jest rozbujane, na przykład wody morza podczas burzy}
\Clue{13}{}{urządzenie wejścia komputera służące do sterowania w grach, posiadające okrągłe pokrętło i jeden lub więcej przycisków}
\Clue{14}{}{niemiecki, podoficerski stopień wojskowy odpowiednik polskiego sierżanta (wachmistrza)}
\Clue{15}{}{szkocki geolog i lekarz (1726-97); twórca podstaw nowoczesnej geologii}
\Clue{16}{}{smaczny owoc (wielopestkowiec) flaszowca purpurowego}
\Clue{17}{}{dodatek do pensji pracowników określonych branż, mający za zadanie wyrównanie wynagrodzeń pracowników tych branż, bez względu na rodzaj umowy, na podstawie której są zatrudnieni}
\Clue{18}{}{jednostka pływająca zaprojektowana lub przystosowana do użytku jako pływający szpital}
\Clue{20}{}{siedziba bobra}
\Clue{22}{}{cecha działań, w których główny organ procesowy (np. sędzia) pełni również funkcje śledcze, oskarżycielskie i funkcje obrony}
\Clue{23}{}{to, że coś ma słowiańskie korzenie, pochodzi od Słowian}
\Clue{24}{}{powiedzenie za dużo, ponad miarę}
\Clue{25}{}{słodkowodny i lądowy wirek z trójdzielnym jelitem}
\Clue{27}{}{miejsce przeznaczenia}
\Clue{28}{}{to, że coś zawiera trudną (często smutną) sytuację lub wiele takich sytuacji}
\Clue{32}{}{architekt włoski (1881-1960), przebudowa osi Bazyliki Św. Piotra w Rzymie}
\Clue{37}{}{miasto w Czechach (kraj północno-czeski) ośrodek eksploatacji węgla kamiennego, hutnictwo żelaza}\end{PuzzleClues}\newpage%\section*{Krzyżówka 33}

\noindent\begin{Puzzle}{23}{22}|*	|*	|*	|[1][S]\darr	|*	|[2][S]\drarr	|j	|a	|j	|k	|o	|[][,]{ }	|n	|a	|[][,]{ }	|t	|w	|a	|r	|d	|o	|*	|*	|*	|.
|*	|[3][S]\drarr	|t	|a	|j	|n	|e	|[][,]{ }	|n	|a	|u	|c	|z	|a	|n	|i	|e	|*	|*	|*	|*	|[4][S]\darr	|*	|[5][S]\darr	|.
|*	|r	|*	|b	|[6][S]\drarr	|o	|b	|r	|ą	|b	|*	|[7][S]\drarr	|k	|a	|p	|i	|t	|u	|ł	|a	|*	|m	|[8][S]\darr	|d	|.
|*	|u	|[9][S]\darr	|a	|d	|r	|*	|*	|[10][S]\darr	|[11][S]\drarr	|o	|k	|a	|*	|*	|*	|[12][S]\darr	|[13][S]\rarr	|r	|a	|k	|ó	|w	|*	|.
|*	|m	|s	|k	|y	|k	|*	|[14][S]\darr	|u	|d	|*	|l	|[15][S]\rarr	|c	|a	|y	|l	|e	|y	|*	|*	|r	|a	|*	|.
|*	|s	|a	|*	|s	|a	|*	|t	|l	|e	|[16][S]\rarr	|e	|*	|*	|[17][S]\rarr	|n	|i	|t	|*	|[18][S]\darr	|*	|*	|t	|*	|.
|[19][S]\drarr	|z	|n	|a	|k	|*	|*	|u	|t	|b	|[20][S]\drarr	|s	|i	|r	|w	|e	|n	|t	|*	|w	|*	|*	|t	|*	|.
|j	|t	|t	|[21][S]\darr	|o	|*	|[22][S]\darr	|ń	|r	|i	|b	|z	|*	|[23][S]\drarr	|p	|*	|o	|*	|[24][S]\darr	|i	|[25][S]\darr	|*	|e	|*	|.
|e	|y	|o	|k	|g	|*	|l	|c	|a	|u	|i	|c	|*	|w	|[26][S]\darr	|*	|r	|*	|b	|l	|p	|[27][S]\darr	|a	|*	|.
|d	|k	|s	|o	|r	|*	|e	|z	|p	|t	|c	|z	|*	|i	|o	|[28][S]\darr	|y	|*	|a	|g	|e	|s	|u	|*	|.
|n	|*	|*	|ł	|a	|[29][S]\drarr	|w	|y	|r	|a	|z	|[][,]{ }	|p	|o	|d	|s	|t	|a	|w	|o	|w	|y	|*	|*	|.
|o	|*	|*	|y	|f	|r	|*	|k	|a	|n	|*	|p	|[30][S]\darr	|l	|d	|ł	|*	|[31][S]\darr	|i	|t	|n	|f	|*	|*	|.
|r	|[32][S]\drarr	|ś	|m	|i	|e	|ć	|*	|w	|c	|*	|o	|e	|o	|a	|u	|[33][S]\darr	|o	|d	|n	|o	|o	|[34][S]\darr	|*	|.
|a	|c	|*	|a	|a	|f	|*	|*	|i	|k	|*	|s	|w	|n	|n	|ż	|e	|r	|ł	|o	|ś	|n	|g	|*	|.
|z	|z	|*	|*	|*	|*	|[35][S]\darr	|[36][S]\darr	|c	|o	|[37][S]\drarr	|p	|o	|c	|i	|ą	|g	|ł	|o	|ś	|ć	|*	|o	|*	|.
|ó	|u	|*	|*	|*	|*	|b	|m	|a	|ś	|o	|o	|l	|z	|e	|c	|r	|o	|*	|ć	|*	|*	|l	|*	|.
|w	|j	|[38][S]\rarr	|u	|z	|i	|o	|m	|*	|ć	|d	|l	|u	|e	|*	|y	|e	|s	|[39][S]\darr	|*	|*	|*	|a	|*	|.
|k	|n	|[40][S]\drarr	|ś	|w	|i	|n	|k	|a	|*	|c	|i	|c	|l	|*	|*	|t	|ę	|c	|*	|[41][S]\darr	|[42][S]\darr	|s	|*	|.
|a	|i	|d	|*	|*	|*	|a	|*	|*	|*	|h	|t	|j	|a	|*	|[43][S]\drarr	|a	|p	|s	|y	|d	|a	|*	|*	|.
|*	|k	|m	|[44][S]\rarr	|b	|o	|n	|g	|o	|*	|ó	|y	|a	|*	|*	|f	|*	|y	|*	|*	|i	|s	|*	|*	|.
|*	|*	|*	|[45][S]\rarr	|p	|r	|z	|e	|p	|ę	|d	|*	|*	|*	|*	|u	|*	|*	|*	|*	|a	|i	|*	|*	|.
|*	|*	|[46][S]\rarr	|w	|i	|w	|a	|t	|*	|*	|*	|*	|*	|*	|*	|s	|*	|*	|*	|*	|k	|r	|*	|*	|.
|*	|*	|*	|*	|*	|*	|*	|*	|*	|*	|*	|*	|*	|*	|*	|*	|*	|*	|*	|*	|*	|*	|*	|*	|.\end{Puzzle}

\newpage

\begin{PuzzleClues}{\textbf{Poziome}\\}\Clue{2}{}{jajko gotowane w wodzie tak długo, aż dojdzie do całkowitego ścięcia żółtka}
\Clue{3}{}{używane w Polsce określenie nauczania prowadzonego w formie nielegalnych zajęć i wykładów organizowanych poza szkołą lub uczelnią w okresie zaborów lub wojny}
\Clue{6}{}{otoczka, obwódka, otok}
\Clue{7}{}{kolegialny organ władzy w Kościele}
\Clue{11}{}{BAKA}
\Clue{13}{}{ARIAN; miejscowość w województwie kieleckim nad Czarną, główny ośrodek Braci Polskich}
\Clue{15}{}{brytyjski pionier lotnictwa (1778-1857); zbudował model śmigłowca}
\Clue{16}{}{nazwa literowa dźwięku, którego częstotliwość dla e1 wynosi 329,6 Hz}
\Clue{17}{}{przestarzała, nie należąca do układu SI jednostka luminancji}
\Clue{19}{}{coś, co potwierdza obecność czegoś w przeszłości lub w teraźniejszości, jest tego namacalnym dowodem}
\Clue{20}{}{pieśń trubadura o bohaterskiej, często satyrycznej treści}
\Clue{23}{}{w chemii: symbol fosforu}
\Clue{29}{}{wyraz, który jest podstawą do stworzenia innego wyrazu w procesie derywacji}
\Clue{32}{}{osoba oceniana jako niezdatna do niczego, człowiek bezużyteczny}
\Clue{37}{}{to, że coś jest pociągłe, podłużne; kształt twarzy lub jej elementów (np. policzków)}
\Clue{38}{}{przewód będący w bezpośredniej styczności z ziemią i przeznaczony do utworzenia z nią połączenia elektrycznego}
\Clue{40}{}{południowoamerykański. gryzoń roślinożerny; zwierzę laboratoryjne i w hodowli amatorskiej}
\Clue{43}{}{ABSYDA}
\Clue{44}{}{rzadki gatunek afrykańskiej antylopy}
\Clue{45}{}{pędzenie bydła, koni z danego miejsca na inne przez jakiś teren}
\Clue{46}{}{rzeczownik oznaczający okrzyk entuzjazmu i uznania; zwykle w liczbie mnogiej}\end{PuzzleClues}

\begin{PuzzleClues}{\textbf{Pionowe}\\}\Clue{1}{}{u starożytnych Greków i Rzymian deska do liczenia, pierwowzór liczydeł}
\Clue{2}{}{nazwa dwóch gatunków zwierząt z rodziny łasicowatych}
\Clue{3}{}{potrawa z mięsa wołowego duszonego z cebulą, podawana w stanie półsurowym}
\Clue{4}{}{dżuma, śmiertelna choroba zakaźna}
\Clue{5}{}{litera oznaczająca wymiar}
\Clue{6}{}{inwazyjna metoda diagnostyczna patologii krążka międzykręgowego}
\Clue{7}{}{Ixodes ricinus - gatunek pajęczaka z rodziny kleszczowatych; żyje głównie w lasach, na paprociach, roślinach leśnych}
\Clue{8}{}{malarz holenderski. (1632-75) kolorysta i luminista; sceny rodzajowe, studia portretowe, pejzaże}
\Clue{9}{}{znany brazylijski klub piłkarski, z którego pochodził między innymi Pele}
\Clue{10}{}{prawica charakteryzująca się skrajnymi poglądami i metodami działania}
\Clue{11}{}{cecha np. kariery, oznaczająca jej początek, świeżość, pierwszy występ, nowość}
\Clue{12}{}{malarz rosyjski (1861-1900) członek pieriedwizników: pejzaże}
\Clue{14}{}{drapieżna ryba Oceanu Atlantyckiego i Spokojnego o długości do 3 m}
\Clue{18}{}{stan nasycenia czegoś wodą mierzony stężeniem wody lub pary wodnej zależnie od sytuacji}
\Clue{19}{}{rzecz jednorazowego użytyku}
\Clue{20}{}{KNUT, BIZUN}
\Clue{21}{}{grupa obozów pracy przymusowej}
\Clue{22}{}{duży drapieżnik z rodziny kotów; poluje na kopytne}
\Clue{23}{}{smyczkowy instr. o 4 strunach strojonych w kwintach, opierany o podłoże}
\Clue{24}{}{zabawka, przedmiot służący do zabawy, zapewniający rozrywkę}
\Clue{25}{}{cecha czegoś, w czym przejawia się to, że ktoś dobrze, stabilnie się czuje w obliczu czegoś}
\Clue{26}{}{cecha kogoś, kto jest oddany - wierny, mocno przywiązany, zdolny do poświęceń}
\Clue{27}{}{zawartość syfonu, grubościennej butli do napojów gazowanych, posiadającej urządzenie zamykające, po otwarciu którego ciśnienie gazu wewnątrz butli wypycha płyn na zewnątrz}
\Clue{28}{}{pracownik (zwykle fizyczny), który pracuje (a bardzo często także mieszka) w domu (zwykle majętnego) pracodawcy}
\Clue{29}{}{poprzeczny rząd krótkich linek przymocowanych do żagla w jego dolnej części}
\Clue{30}{}{trudna do wykonania akrobacja lotniczna}
\Clue{31}{}{Gypaetinae - podrodzina ptaków z rodziny jastrzębiowatych (Accipitridae)}
\Clue{32}{}{zwyczajowa nazwa przyrządów do dokładnego pomiaru długości}
\Clue{33}{}{czapla biała o śnieżnobiałym upierzeniu i długości około 100 cm, w Polsce rzadka, chroniona}
\Clue{34}{}{osoba, która jest biedna, niewiele posiada pod względem majątkowym, biedak}
\Clue{35}{}{żyła złota}
\Clue{36}{}{kod ISO 4217 waluty kyat}
\Clue{37}{}{wyjście, droga wyjściowa}
\Clue{39}{}{w chemii: symbol cezu}
\Clue{40}{}{choroba genetyczna, postać dystrofii (zaniku) mięśni}
\Clue{41}{}{przewodniczący prikazu w Carstwie Rosyjskim}
\Clue{42}{}{kraina na południu Arabii Saudyjskiej, główny ośrodek Abha}
\Clue{43}{}{osad pozostały po zaparzonej herbacie lub kawie}\end{PuzzleClues}\newpage%\section*{Krzyżówka 34}

\noindent\begin{Puzzle}{22}{29}|*	|*	|*	|*	|*	|*	|*	|*	|*	|*	|*	|*	|*	|[1][S]\drarr	|l	|u	|n	|u	|l	|a	|*	|[2][S]\darr	|*	|.
|[3][S]\rarr	|f	|e	|n	|y	|l	|o	|e	|t	|y	|l	|o	|a	|m	|i	|n	|a	|*	|*	|[4][S]\darr	|[5][S]\darr	|a	|*	|.
|*	|[6][S]\darr	|*	|[7][S]\darr	|[8][S]\rarr	|d	|o	|k	|s	|o	|l	|o	|g	|i	|a	|*	|*	|[9][S]\darr	|[10][S]\rarr	|g	|i	|n	|*	|.
|[11][S]\drarr	|r	|y	|b	|y	|[][,]{ }	|w	|ę	|d	|r	|o	|w	|n	|e	|*	|*	|*	|p	|*	|r	|f	|g	|*	|.
|m	|u	|[12][S]\rarr	|e	|*	|*	|*	|*	|*	|[13][S]\rarr	|ż	|u	|b	|r	|ó	|w	|k	|a	|*	|u	|n	|l	|*	|.
|l	|b	|*	|k	|*	|*	|[14][S]\rarr	|k	|o	|b	|i	|e	|r	|z	|e	|c	|*	|r	|*	|d	|i	|i	|*	|.
|e	|a	|[15][S]\darr	|a	|*	|*	|[16][S]\drarr	|d	|a	|l	|m	|i	|n	|e	|*	|*	|[17][S]\darr	|t	|[18][S]\darr	|a	|*	|s	|*	|.
|k	|s	|t	|s	|*	|*	|e	|[19][S]\drarr	|a	|j	|e	|n	|c	|j	|a	|*	|j	|i	|ł	|*	|[20][S]\darr	|t	|*	|.
|o	|z	|r	|*	|*	|*	|s	|k	|*	|*	|*	|*	|[21][S]\darr	|a	|*	|*	|ę	|a	|o	|*	|p	|k	|*	|.
|[][,]{ }	|n	|a	|*	|*	|[22][S]\drarr	|k	|a	|s	|z	|t	|e	|l	|*	|*	|*	|z	|[][,]{ }	|n	|[23][S]\darr	|i	|a	|*	|.
|w	|o	|k	|*	|*	|l	|a	|l	|[24][S]\rarr	|s	|i	|ł	|a	|[][,]{ }	|p	|ł	|y	|w	|o	|w	|a	|*	|*	|.
|[][,]{ }	|ś	|t	|*	|[25][S]\drarr	|e	|d	|e	|*	|*	|*	|[26][S]\rarr	|k	|u	|r	|a	|k	|i	|*	|i	|n	|*	|*	|.
|p	|ć	|a	|*	|p	|n	|r	|j	|[27][S]\rarr	|a	|r	|y	|t	|m	|i	|a	|*	|e	|[28][S]\darr	|c	|o	|*	|*	|.
|r	|*	|t	|*	|r	|o	|a	|d	|[29][S]\darr	|*	|[30][S]\drarr	|p	|o	|d	|g	|a	|r	|d	|l	|e	|*	|*	|*	|.
|o	|*	|[][,]{ }	|*	|a	|n	|*	|o	|k	|*	|h	|[31][S]\rarr	|g	|r	|a	|f	|*	|e	|e	|d	|*	|*	|*	|.
|s	|*	|s	|*	|w	|k	|[32][S]\darr	|s	|o	|*	|a	|[33][S]\rarr	|l	|e	|p	|*	|*	|ń	|k	|y	|*	|*	|*	|.
|z	|*	|o	|*	|o	|i	|t	|k	|m	|[34][S]\darr	|d	|[35][S]\rarr	|o	|m	|a	|m	|*	|s	|t	|r	|*	|[36][S]\darr	|*	|.
|k	|*	|j	|*	|[][,]{ }	|*	|a	|o	|p	|ł	|ż	|[37][S]\rarr	|b	|y	|k	|*	|*	|k	|u	|e	|*	|p	|*	|.
|u	|[38][S]\darr	|u	|*	|g	|[39][S]\darr	|l	|p	|l	|ę	|*	|[40][S]\rarr	|u	|l	|*	|*	|*	|a	|r	|k	|*	|r	|*	|.
|*	|t	|s	|*	|o	|t	|e	|*	|e	|k	|*	|*	|l	|*	|[41][S]\rarr	|c	|t	|*	|a	|t	|*	|z	|*	|.
|[42][S]\drarr	|r	|z	|ę	|s	|o	|r	|e	|k	|[][,]{ }	|m	|n	|i	|e	|j	|s	|z	|y	|*	|o	|*	|e	|*	|.
|a	|y	|n	|[43][S]\drarr	|p	|r	|z	|y	|s	|t	|ę	|p	|n	|o	|ś	|ć	|*	|[44][S]\darr	|*	|r	|[45][S]\darr	|c	|*	|.
|n	|m	|i	|p	|o	|t	|*	|*	|[][,]{ }	|y	|[46][S]\drarr	|p	|a	|n	|t	|a	|l	|o	|n	|*	|p	|i	|*	|.
|a	|o	|c	|e	|d	|e	|*	|*	|ż	|l	|s	|*	|*	|*	|*	|[47][S]\rarr	|s	|s	|p	|*	|e	|w	|*	|.
|m	|w	|z	|p	|a	|l	|*	|*	|y	|n	|p	|*	|*	|[48][S]\rarr	|l	|y	|o	|t	|*	|*	|r	|n	|*	|.
|n	|a	|y	|i	|r	|l	|*	|*	|t	|y	|ł	|*	|[49][S]\rarr	|l	|a	|j	|k	|r	|y	|*	|f	|i	|*	|.
|e	|n	|*	|n	|c	|i	|*	|*	|n	|*	|a	|*	|*	|*	|*	|[50][S]\rarr	|b	|o	|r	|s	|u	|k	|*	|.
|z	|i	|*	|k	|z	|n	|*	|*	|i	|[51][S]\rarr	|t	|e	|l	|e	|o	|l	|o	|g	|i	|z	|m	|*	|*	|.
|a	|e	|*	|a	|e	|i	|*	|*	|*	|*	|a	|[52][S]\rarr	|c	|e	|b	|u	|l	|a	|n	|k	|a	|*	|*	|.
|*	|*	|*	|*	|*	|*	|*	|*	|*	|*	|*	|*	|*	|*	|*	|*	|*	|*	|*	|*	|*	|*	|*	|.\end{Puzzle}

\newpage

\begin{PuzzleClues}{\textbf{Poziome}\\}\Clue{1}{}{wisiorek z metalu w kształcie półksiężyca o ramionach ustawionych w dół}
\Clue{3}{}{organiczny związek chemiczny z grupy amin, pochodna etyloaminy zawierająca grupę fenylową w pozycji 2}
\Clue{8}{}{hymn religijny, który głosi pochwałę Boga}
\Clue{10}{}{DŻYN; JAŁOWCÓWKA}
\Clue{11}{}{ryby dwuśrodowiskowe - ryby zwykle sezonowo i regularnie przemieszczającesię na różne odległości; często jest to związane ze zmianą środowiska (ryby dwuśrodowiskowe)}
\Clue{12}{}{obiektowo zorientowany język programowania przeznaczony dla bezpiecznych obliczeń rozproszonych, stworzony przez Marka S. Millera i innych w Electric Communities w roku 1997}
\Clue{13}{}{TURÓWKA}
\Clue{14}{}{przen. powierzchnia, warstwa czegoś, najczęściej roślin, która przypomina kobierzec}
\Clue{16}{}{miasto we Włoszech (Lombardia); hutnictwo żelaza, przemysł metalowy, maszynowy}
\Clue{19}{}{prowadzenie działalności na zasadzie dzierżawy od określonego przedsiębiorcy}
\Clue{22}{}{zamek lub jego główna część w postaci warownego budynku mieszkalnego, zajmowanego przez kasztelana lub pana feudalnego}
\Clue{24}{}{siła działająca na ciało rozciągłe znajdujące się w polu sił o różnej wartości lub kierunku w różnych punktach ciała}
\Clue{25}{}{miasto w środkowej Holandii, 87,8 tys. mieszkańców (1985 r.), przemysł chemiczny, maszynowy i mleczarski}
\Clue{26}{}{grzebiące, Galliformes - rząd ptaków z podgromady Neornithes; obejmuje gatunki lądowe, grzebiące, zamieszkujące wszystkie strefy klimatyczne na całym świecie, poza niektórymi wyspami}
\Clue{27}{}{zaburzenie rytmu, arytmiczność}
\Clue{30}{}{część ogłowia wędzidłowego, która razem z naczółkiem i nachrapnikiem utrzymuje w określonej pozycji ogłowie na łbie konia}
\Clue{31}{}{w uproszczeniu - zbiór wierzchołków, które mogą być połączone krawędziami w taki sposób, że każda krawędź kończy się i zaczyna w którymś z wierzchołków}
\Clue{33}{}{przenośnie: coś kuszącego i zdradliwego}
\Clue{35}{}{wzrokowy (przywidzenie) lub słuchowy (przysłyszenie) wytwór umysłu}
\Clue{37}{}{domowy samiec bydła domowego, bawołów, jaka, żubra, bizona itp}
\Clue{40}{}{w rzemiośle pszczelarskim, konstrukcja, najczęściej drewniana, używana do hodowli pszczół}
\Clue{41}{}{określenie zawartości złota w stopach (czystości stopu); 1 ct to 1/24 zawartości wagowej złota w stopie, złoto 24-karatowe to złoto czyste}
\Clue{42}{}{Neomys anomalus -  gatunek ssaka z rodziny ryjówkowatych; żyje samotnie nad wodami, występuje w Europie i Azji Zachodniej, w Polsce występuje na Podkarpaciu, w Pieninach, Sudetach oraz w Puszczy Białowieskiej i na Pojezierzu Pomorskim}
\Clue{43}{}{zrozumiałość, brak elementów utrudniających przyswojenie treści}
\Clue{46}{}{PANTALEON}
\Clue{47}{}{kod ISO 4217 funta południowosudańskiego}
\Clue{48}{}{astronom francuski (1879-1952), wynalazca koronografu}
\Clue{49}{}{obcisłe getry (często kolarki) z lycry noszone czasem przez sportowców, obecne także w modzie pozasportowej, zwłaszcza w latach 80. i 90. XX w}
\Clue{50}{}{człowiek skryty, ponurak, samotnik, odludek}
\Clue{51}{}{w filozofii: pogląd, według którego rozwój świata przyrody i rzeczywistości społecznej zmierza do jakiegoś ostatecznego celu}
\Clue{52}{}{zupa cebulowa}\end{PuzzleClues}

\begin{PuzzleClues}{\textbf{Pionowe}\\}\Clue{1}{}{wał piaszczysty zamykający częściowo lub całkowicie zatokę morską; lido}
\Clue{2}{}{nauczycielka, która uczy języka angielskiego}
\Clue{4}{}{sklejona bryła czegoś, często ziemi}
\Clue{5}{}{enklawa w Maroku stanowiąca prowincję zamorską Hiszpanii}
\Clue{6}{}{cecha człowieka: to, że ktoś jest skory do frywolnych żartów}
\Clue{7}{}{ptak z rzędu mew-siewek}
\Clue{9}{}{otwarcie szachowe, które charakteryzuje się posunięciami: 1. e4 e5, 2. Sc3}
\Clue{11}{}{przetwór mleczny uzyskiwany przez odparowanie z pełnego, znormalizowanego lub chudego (odtłuszczonego) mleka krowiego prawie całej wody (pozostałość wody w proszku mlecznym to ok. 2\%)}
\Clue{15}{}{umowa międzynarodowa, w której strony zobowiązują się do bycia nawzajem dla siebie sojusznikami}
\Clue{16}{}{duża jednostka organizacyjna marynarki wojennej}
\Clue{17}{}{sposób porozumiewania się ludzi pewnego środowiska lub zawodu, w tym nie tylko gwara, socjolekt i profesjolekt, ale także kod, będący tzw. sztucznym językiem lub językiem programowania}
\Clue{18}{}{dolna część brzucha zwykle wraz z częściami intymnymi (jest to częsta eufemistyczna nazwa części intymnych)}
\Clue{19}{}{urządzenie optyczne (zabawka), w którym dzięki wielokrotnym odbiciom obrazów różnokolorowych szkiełek, w odpowiednio rozmieszczonych zwierciadłach, obserwuje się różnobarwne, symetryczne figury, zmieniające się przy obracaniu kalejdoskopu}
\Clue{20}{}{wykonanie utworu muzycznego w sposób cichy i delikatny}
\Clue{21}{}{białko zawarte w mleku ssaków, zawierające m.in. ciała odpornościowe}
\Clue{22}{}{wzór okularów o okrągłych oprawkach, których nazwa pochodzi od nazwiska brytyjskiego muzyka Johna Lennona}
\Clue{23}{}{zastępca dyrektora - szefa, kierownika jakiejś instytucji}
\Clue{25}{}{zwyczajowe określenie dziedzin prawa regulujących działalność gospodarczą}
\Clue{28}{}{zajęcia czytania}
\Clue{29}{}{kompleks glebowy obejmujący gleby lekkie, wytworzone z piasków gliniastych mocnych całkowitych lub piasków gliniastych, na których wysiewa się mało wymagające rośliny (np. żyto, kukurydza)}
\Clue{30}{}{pielgrzymka do Mekki, tzw. większa lub pełna}
\Clue{32}{}{antena satelitarna}
\Clue{34}{}{jeden z łęków w stelażu siodła}
\Clue{36}{}{rywal, taki jak w rywalizacji sportowej, konkurent, współzawodnik}
\Clue{38}{}{rozmieszczenie ładunku w ładowniach statku}
\Clue{39}{}{włoskie pierożki z nadzieniem}
\Clue{42}{}{fragment modlitwy eucharystycznej odmawianej w czasie mszy świętej, w którym wspomina się śmierć i zmartwychwstanie Chrystusa}
\Clue{43}{}{GLOGIERÓWKA odporna na mróz odmiana jabłoni}
\Clue{44}{}{ostro zakończone, podługowate twory występujące w budowie morfologicznej roślin; też ostro zakończone, wąskie wyrostki blaszki liściowej występujące u niektórych traw}
\Clue{45}{}{perfumy -  kosmetyki, których jedynym zadaniem jest nadawanie różnym obiektom (zwykle ciału człowieka) przyjemnego i długo utrzymującego się zapachu}
\Clue{46}{}{kwota, którą się płaci, spłaca}\end{PuzzleClues}\newpage%\section*{Krzyżówka 35}

\noindent\begin{Puzzle}{15}{33}|*	|*	|*	|*	|*	|*	|*	|*	|*	|*	|*	|*	|*	|[1][S]\darr	|*	|*	|.
|*	|*	|[2][S]\drarr	|a	|r	|*	|[3][S]\darr	|[4][S]\darr	|*	|*	|[5][S]\darr	|*	|[6][S]\drarr	|f	|g	|*	|.
|*	|*	|m	|*	|*	|*	|c	|m	|*	|[7][S]\darr	|s	|*	|p	|o	|*	|*	|.
|*	|*	|o	|[8][S]\darr	|*	|*	|i	|u	|*	|k	|k	|[9][S]\darr	|r	|r	|*	|*	|.
|*	|*	|n	|c	|[10][S]\darr	|*	|a	|z	|*	|a	|a	|c	|ą	|m	|*	|*	|.
|*	|*	|o	|z	|o	|*	|ł	|a	|*	|t	|f	|z	|d	|a	|*	|*	|.
|*	|[11][S]\rarr	|p	|a	|s	|m	|o	|*	|*	|e	|a	|a	|[][,]{ }	|c	|*	|*	|.
|*	|[12][S]\rarr	|o	|r	|ę	|ż	|*	|[13][S]\darr	|*	|c	|n	|j	|z	|j	|*	|*	|.
|*	|[14][S]\darr	|l	|n	|k	|*	|*	|j	|*	|h	|d	|k	|w	|a	|*	|*	|.
|*	|w	|[][,]{ }	|a	|*	|*	|[15][S]\darr	|a	|*	|e	|e	|a	|a	|*	|*	|*	|.
|*	|y	|n	|[][,]{ }	|*	|*	|p	|j	|*	|z	|r	|[][,]{ }	|r	|*	|*	|*	|.
|[16][S]\drarr	|p	|a	|r	|a	|l	|a	|k	|s	|a	|*	|s	|c	|*	|*	|*	|.
|c	|a	|t	|o	|[17][S]\drarr	|p	|ł	|o	|ć	|*	|*	|a	|i	|*	|[18][S]\darr	|*	|.
|z	|l	|u	|b	|ł	|*	|e	|*	|*	|*	|*	|k	|o	|*	|d	|*	|.
|e	|a	|r	|o	|u	|*	|c	|[19][S]\darr	|*	|[20][S]\darr	|[21][S]\darr	|s	|w	|*	|r	|[22][S]\darr	|.
|r	|r	|a	|t	|s	|*	|z	|r	|[23][S]\darr	|j	|f	|o	|y	|*	|e	|j	|.
|w	|k	|l	|a	|k	|[24][S]\drarr	|k	|a	|m	|i	|e	|ń	|*	|*	|w	|ę	|.
|o	|a	|n	|*	|a	|g	|a	|t	|a	|g	|t	|s	|*	|*	|u	|z	|.
|n	|*	|y	|*	|*	|n	|*	|a	|k	|*	|a	|k	|*	|*	|t	|y	|.
|y	|*	|*	|*	|[25][S]\rarr	|o	|r	|n	|a	|t	|*	|a	|*	|*	|n	|k	|.
|*	|[26][S]\rarr	|a	|k	|c	|j	|a	|*	|r	|*	|[27][S]\darr	|*	|[28][S]\darr	|[29][S]\darr	|i	|[][,]{ }	|.
|*	|*	|*	|*	|[30][S]\darr	|n	|[31][S]\darr	|*	|o	|*	|s	|[32][S]\darr	|e	|r	|a	|n	|.
|[33][S]\rarr	|w	|ł	|ó	|k	|i	|e	|n	|n	|i	|c	|t	|w	|o	|*	|o	|.
|*	|*	|*	|*	|a	|c	|s	|*	|*	|*	|r	|r	|a	|z	|*	|w	|.
|[34][S]\rarr	|t	|e	|t	|r	|a	|p	|o	|d	|y	|*	|e	|n	|p	|*	|o	|.
|[35][S]\drarr	|j	|u	|m	|a	|*	|i	|*	|[36][S]\darr	|[37][S]\darr	|*	|p	|g	|u	|*	|g	|.
|s	|*	|*	|*	|b	|*	|n	|*	|l	|f	|*	|a	|e	|s	|[38][S]\darr	|r	|.
|n	|*	|*	|*	|i	|*	|g	|[39][S]\darr	|e	|r	|*	|k	|l	|t	|p	|e	|.
|i	|*	|*	|*	|n	|*	|o	|k	|j	|a	|[40][S]\darr	|*	|i	|n	|i	|c	|.
|g	|[41][S]\rarr	|c	|h	|i	|l	|l	|i	|*	|n	|b	|*	|a	|i	|u	|k	|.
|g	|[42][S]\drarr	|n	|i	|e	|p	|a	|l	|ą	|c	|y	|*	|r	|c	|s	|i	|.
|a	|t	|*	|*	|r	|*	|*	|*	|[43][S]\rarr	|i	|k	|r	|z	|a	|k	|*	|.
|*	|h	|*	|*	|*	|*	|*	|*	|*	|s	|*	|*	|*	|*	|a	|*	|.
|*	|*	|*	|*	|*	|*	|*	|*	|*	|*	|*	|*	|*	|*	|*	|*	|.\end{Puzzle}

\newpage

\begin{PuzzleClues}{\textbf{Poziome}\\}\Clue{2}{}{w chemii: symbol argonu}
\Clue{6}{}{skrót/symbol franka gwinejskiego}
\Clue{11}{}{następstwo, szereg zdarzeń}
\Clue{12}{}{poroże jelenia, fajki i szable dzika}
\Clue{16}{}{zjawisko pozornej zmiany położenia obiektu na sferze niebieskiej względem dalszych obiektów, wynikające ze zmiany miejsca obserwacji spowodowanej przemieszczeniem się obserwatora}
\Clue{17}{}{PŁOTKA}
\Clue{24}{}{płytka do gry w domino, mahjonga itp}
\Clue{25}{}{wierzchnia szata używana przez księdza podczas mszy haftowana z drogich tkanin}
\Clue{26}{}{atakowanie w czasie rozgrywek sportowych; element gry ofensywnej}
\Clue{33}{}{gałąź przemysłu lekkiego zajmująca się przetwórstwem surowców na włókna, tkaniny, dzianiny itp}
\Clue{34}{}{Tetrapoda - grupa kręgowców w randze nadgromady obejmująca płazy, gady, ptaki i ssaki}
\Clue{35}{}{przedmiot skradziony podczas jumy}
\Clue{41}{}{ogólna nazwa, jaką określa się owoce niektórych odmian, kultywarów i mieszańców papryki o bardzo ostrym smaku}
\Clue{42}{}{człowiek, który nie pali tytoniu}
\Clue{43}{}{samica ryb gotowa do tarła}\end{PuzzleClues}

\begin{PuzzleClues}{\textbf{Pionowe}\\}\Clue{1}{}{grupa osób, które zwłaszcza ze względu na podobny wiek doświadczyły jakiegoś szczególnego wydarzenia lub wydarzeń}
\Clue{2}{}{taka sytuacja na rynku dóbr i usług, w której istnieje tylko jeden przedsiębiorca wytwarzający dane dobro lub usługę, a występowanie więcej niż jednego producenta byłoby utrudnione oraz ekonomicznie nieuzasadnione}
\Clue{3}{}{przenośnie: organ - podmiot, osoba lub grupa osób wyodrębniona z całości ze względu na jej funkcję (np. ciało pedagogiczne)}
\Clue{4}{}{kobieta-inspiratorka, dzięki której mężczyzna (nie tylko artysta) żyje, tworzy, działa}
\Clue{5}{}{kurtka z materiału odpornego na wiatr i deszcz.}
\Clue{6}{}{prąd o wartości przekraczającej dopuszczalne obciążenie instalacji, pojawiający się w obwodzie elektrycznym na skutek wystąpienia zwarcia}
\Clue{7}{}{ustne nauczanie zasad religii chrześcijańskiej}
\Clue{8}{}{ciężka, nudna praca, której jest dużo i której nikt nie ma ochoty robić; to, co każdy odkłada na później, to, co wykonuje się z niechęcią, byle jak}
\Clue{9}{}{rasa gołębia domowego, o białym upierzeniu i kolorowych czółku, łapciach i skrzydłach (z wyjątkiem piór na ramionach)}
\Clue{10}{}{bosak - długi drąg zakończony metalowym hakiem i grotem (szpikulcem)}
\Clue{13}{}{kształt przypominający jajko}
\Clue{14}{}{nagrywarka - urządzenie służące do zapisywania informacji na przeznaczonych do tego celu dyskach optycznych, inaczej płytach}
\Clue{15}{}{zdrobniale: pałka - nóżka drobiowa, mała porcja mięsa (samo udko) z kostką}
\Clue{16}{}{z lekceważeniem, pogardą o komuniście}
\Clue{17}{}{część naboju służąca do umieszczenia w niej ładunku miotającego i spłonki}
\Clue{18}{}{komórka z opałem}
\Clue{19}{}{giętkie drewno pozyskiwane z palmy rotang, używane do produkcji mebli}
\Clue{20}{}{utwór muzyczny stylizowany na taniec jig; często wykorzystywany jest jako część suity klasycznej}
\Clue{21}{}{grecki ser, biały w kolorze, z mieszanki pasteryzowanego mleka owczego z mlekiem kozim, intensywnie słony}
\Clue{22}{}{język indoeuropejski używany współcześnie w Grecji i na Cyprze}
\Clue{23}{}{produkt żywnościowy wytwarzany na bazie ciasta z mąki, wody i niekiedy jaj, który miewa różne rozmiary i kształty, jest dość trwały: można go suszyć i przechowywać}
\Clue{24}{}{deska, służąca jako burta, w wozie do wożenia gnoju}
\Clue{27}{}{kod ISO 4217 rupii seszelskiej}
\Clue{28}{}{księga liturgiczna zawierająca teksty czterech Ewangelii}
\Clue{29}{}{kobieta rozpustna}
\Clue{30}{}{w XVII/XIX w. wyborowy żołnierz kawalerii i piechoty w wielu krajach europejskich}
\Clue{31}{}{krótka ręczna broń palna z podpórką używana w XVI/XVII w. później małokalibrowe działo okrętowe}
\Clue{32}{}{muzyka, do której tańczy się trapaka}
\Clue{35}{}{SZNIKA}
\Clue{36}{}{zagłębienie terenu powstałe wskutek wybuchu bomby, osunięcia się podłoża lub działania wody}
\Clue{37}{}{konstruktor amerykański (1815-92); zbudował turbinę wodną}
\Clue{38}{}{mała, okrągła czapeczka przykrywająca Tonsurę, noszona przez biskupów, kardynałów i papieża}
\Clue{39}{}{stępka}
\Clue{40}{}{błąd, zwłaszcza ortograficzny}
\Clue{42}{}{w chemii: symbol toru}\end{PuzzleClues}\newpage%\section*{Krzyżówka 38}

\noindent\begin{Puzzle}{17}{31}|*	|*	|*	|*	|[1][S]\drarr	|o	|w	|s	|i	|a	|n	|k	|a	|*	|[2][S]\darr	|[3][S]\darr	|*	|[4][S]\darr	|.
|[5][S]\drarr	|p	|a	|s	|k	|a	|l	|*	|*	|*	|*	|*	|*	|*	|l	|n	|*	|s	|.
|m	|[6][S]\drarr	|s	|z	|u	|m	|[][,]{ }	|b	|i	|a	|ł	|y	|*	|*	|a	|a	|*	|k	|.
|o	|s	|*	|[7][S]\rarr	|k	|o	|n	|f	|e	|k	|t	|y	|*	|*	|p	|p	|*	|r	|.
|p	|i	|*	|*	|a	|*	|*	|[8][S]\drarr	|k	|u	|d	|a	|t	|*	|e	|i	|*	|ę	|.
|e	|ł	|*	|*	|w	|[9][S]\rarr	|l	|s	|l	|*	|*	|*	|*	|[10][S]\darr	|k	|ę	|[11][S]\darr	|t	|.
|k	|o	|*	|*	|c	|*	|*	|m	|[12][S]\drarr	|s	|w	|i	|n	|g	|*	|c	|s	|e	|.
|*	|w	|*	|*	|z	|[13][S]\rarr	|r	|o	|m	|n	|y	|*	|[14][S]\darr	|a	|*	|i	|z	|k	|.
|*	|n	|*	|*	|y	|*	|[15][S]\rarr	|l	|a	|t	|o	|p	|i	|s	|i	|e	|c	|*	|.
|*	|i	|*	|*	|k	|*	|*	|e	|t	|*	|*	|*	|n	|l	|[16][S]\darr	|[][,]{ }	|z	|*	|.
|[17][S]\drarr	|k	|u	|c	|[][,]{ }	|m	|e	|r	|e	|n	|s	|*	|w	|i	|n	|r	|e	|*	|.
|d	|[][,]{ }	|*	|[18][S]\darr	|m	|[19][S]\darr	|*	|*	|r	|*	|*	|*	|e	|f	|i	|o	|p	|[20][S]\darr	|.
|z	|h	|[21][S]\drarr	|b	|a	|d	|a	|n	|i	|e	|*	|*	|s	|t	|e	|z	|i	|n	|.
|i	|y	|l	|r	|ł	|a	|[22][S]\drarr	|l	|a	|i	|k	|a	|t	|*	|p	|k	|o	|i	|.
|e	|d	|ą	|a	|y	|h	|a	|*	|ł	|[23][S]\darr	|[24][S]\darr	|*	|y	|*	|o	|ł	|n	|e	|.
|w	|r	|d	|b	|*	|l	|n	|*	|[][,]{ }	|w	|n	|[25][S]\darr	|c	|[26][S]\darr	|p	|a	|k	|s	|.
|c	|a	|e	|a	|*	|*	|k	|*	|p	|j	|e	|w	|j	|m	|u	|d	|a	|y	|.
|z	|u	|k	|n	|[27][S]\drarr	|r	|e	|w	|i	|a	|*	|y	|e	|e	|l	|o	|[][,]{ }	|m	|.
|ę	|l	|*	|s	|s	|*	|r	|*	|r	|z	|[28][S]\darr	|w	|*	|t	|a	|w	|w	|p	|.
|c	|i	|*	|o	|l	|*	|*	|*	|o	|d	|k	|i	|*	|o	|r	|e	|ł	|a	|.
|o	|c	|[29][S]\drarr	|n	|i	|e	|l	|o	|t	|*	|a	|a	|[30][S]\darr	|d	|n	|*	|a	|t	|.
|ś	|z	|n	|a	|p	|*	|*	|*	|e	|*	|c	|d	|p	|y	|o	|*	|s	|y	|.
|ć	|n	|a	|*	|y	|*	|*	|*	|c	|*	|z	|o	|a	|[][,]{ }	|ś	|*	|n	|c	|.
|*	|y	|s	|*	|*	|*	|*	|*	|h	|*	|k	|w	|s	|l	|ć	|[31][S]\darr	|a	|z	|.
|*	|*	|a	|[32][S]\rarr	|c	|e	|m	|e	|n	|t	|a	|c	|j	|a	|*	|t	|*	|n	|.
|[33][S]\drarr	|u	|d	|e	|r	|z	|e	|n	|i	|e	|*	|a	|o	|p	|*	|r	|[34][S]\darr	|o	|.
|n	|[35][S]\rarr	|a	|l	|b	|u	|m	|*	|c	|*	|*	|*	|n	|u	|[36][S]\darr	|ó	|c	|ś	|.
|a	|*	|*	|*	|*	|*	|*	|*	|z	|*	|*	|*	|a	|n	|o	|j	|i	|ć	|.
|z	|*	|*	|*	|*	|*	|[37][S]\rarr	|a	|n	|i	|m	|a	|t	|o	|r	|k	|a	|*	|.
|w	|*	|*	|[38][S]\rarr	|c	|y	|p	|r	|y	|j	|k	|a	|*	|w	|o	|a	|ł	|*	|.
|a	|*	|[39][S]\rarr	|p	|a	|s	|a	|ż	|*	|*	|*	|*	|*	|a	|n	|*	|o	|*	|.
|*	|*	|*	|*	|[40][S]\rarr	|e	|g	|z	|a	|r	|c	|h	|a	|*	|*	|*	|*	|*	|.\end{Puzzle}

\newpage

\begin{PuzzleClues}{\textbf{Poziome}\\}\Clue{1}{}{Leucaspius delineatus, słonecznica pospolita - gatunek małej słodkowodnej ryby z rodziny karpiowatych}
\Clue{5}{}{jednostka ciśnienia (także naprężenia) w układzie SI (Jednostka pochodna układu SI), oznaczana Pa}
\Clue{6}{}{rodzaj szumu akustycznego o całkowicie płaskim widmie}
\Clue{7}{}{słodycze, łakocie}
\Clue{8}{}{miasto w Melanezji w północnej części Borneo}
\Clue{9}{}{kod ISO 4217 waluty loti}
\Clue{12}{}{nieregularny, synkopowy rytm, taki jak w jazzie}
\Clue{13}{}{miasto na Ukrainie nad Sułą}
\Clue{15}{}{kronikarz (głównie ruski)}
\Clue{17}{}{rasa konia z grupy kuców pochodząca z Pirenejów francuskich; początkowo użytkowany jako zwierzę juczne, stał się później koniem rolników, używanym do uprawy wysoko położonych i silnie nachylonych górskich pól, na których traktory byłyby bezużyteczne}
\Clue{21}{}{działanie podjęte przez lekarza na pacjencie w celu kontroli stanu zdrowia pacjenta}
\Clue{22}{}{ogół wiernych świeckich}
\Clue{27}{}{uroczysty przemarsz wojska}
\Clue{29}{}{KIWI}
\Clue{32}{}{cementowanie - jednoczenie, umacnianie}
\Clue{33}{}{silny, zmasowany atak na wroga}
\Clue{35}{}{wydawnictwo płytowe, zamknięta całość, zbiór utworów muzycznych}
\Clue{37}{}{kobieta animator; aktorka, lalkarka}
\Clue{38}{}{mieszkanka Cypru, kobieta pochodzenia cypryjskiego}
\Clue{39}{}{przesuwanie się treści pokarmowych przez przewód pokarmowy}
\Clue{40}{}{zarządca prowincji w Kościele prawosławnym}\end{PuzzleClues}

\begin{PuzzleClues}{\textbf{Pionowe}\\}\Clue{1}{}{Coccycua pumila - gatunek ptaka z rodziny kukułkowatych (Cuculidae), z podrodziny kukułek (Cuculinae)}
\Clue{2}{}{laptop - komputer przenośny}
\Clue{3}{}{najmniejsza wartość napięcia konieczna do rozpoczęcia elektrolizy}
\Clue{4}{}{Funaria - rodzaj mchów z klasy prątników; drobny mech liściasty, jeden z najbardziej pospolitych mchów na kuli ziemskiej}
\Clue{5}{}{Barbastella barbastellus - gatunek ssaka z rzędu nietoperzy, występującego na obszarze południowej, zachodniej i środkowej Europy, aż po Kaukaz, w zachodniej części kontynentu znacznie rzadszego i lokalnie uważanego za gatunek zagrożony; w Polsce zamieszkuje całe terytorium kraju, jednak w niektórych regionach jest względnie częsty (Sudety, wschodnia Polska), w innych lokalnie rzadki lub bardzo rzadki (Pomorze, Tatry)}
\Clue{6}{}{silnik hydrostatyczny o ruchu posuwistym}
\Clue{8}{}{etnograf, publicysta, pisarz górnołużycki (1816-84), autor słownika niemiecko-łużyckiego, zbiór ludowych pieśni}
\Clue{10}{}{wydobywanie ropy naftowej przy zastosowaniu gazu sprężonego}
\Clue{11}{}{szczepionka, która jest przygotowana z substancji pochodzących od martwych bakterii, rozwijających się wcześniej w ognisku zakażenia organizmie tego pacjenta, który ma byc poddany szczepieniu}
\Clue{12}{}{rodzaj materiału wybuchowego, będący zwykle mieszaniną związków chemicznych służących do wytwarzania efektów pirotechnicznych w wyniku bezdetonacyjnej, samopodtrzymującej się reakcji chemicznej}
\Clue{14}{}{wartość dóbr i usług nabytych przez podmioty gospodarcze w celu zwiększenia majątku trwałego}
\Clue{16}{}{cecha człowieka, który nie jest popularny w szerszym gronie, nie jest lubiany, ceniony}
\Clue{17}{}{styl czegoś, co wydaje się pasować do dziewczęcia, sprawia, że ktoś wygląda dziewczęco}
\Clue{18}{}{belgijska pieśń rewolucyjna, obecnie narodowy hymn}
\Clue{19}{}{malarz i grafik norweski (1788-1857) nastrojowe pejzaże, obrazy religijne, portrety}
\Clue{20}{}{to, że coś jest niesympatyczne, jest niemiłe, nieprzyjemne}
\Clue{21}{}{wieś gminna w województwie wielkopolskim, w pow. słupeckim}
\Clue{22}{}{dawna jednostka objętości; w Prusach 34,35 l, w Rosji 36,9 l}
\Clue{23}{}{miejsce, przez które lub którym się wjeżdża}
\Clue{24}{}{w chemii: symbol neonu}
\Clue{25}{}{funkcjonariusz śledczy, pracownik służb śledczych}
\Clue{26}{}{metody, służące do określania stabilności punktu równowagi układu nieliniowego}
\Clue{27}{}{krótkie, męskie spodenki kąpielowe}
\Clue{28}{}{efekt gitarowy nazywany wah-wah (łałą)}
\Clue{29}{}{to na czym coś jest osadzone: rękojeść, trzonek, oprawa}
\Clue{30}{}{osoba, która łatwo wpadający w pasję, nerwus}
\Clue{31}{}{pokój hotelowy dla trzech osób}
\Clue{33}{}{wyraz stanowiący określenie jakiejś rzeczy}
\Clue{34}{}{martwe ciało ludzkie lub (rzadziej) zwierzęce}
\Clue{36}{}{miejscowość w płd. Izraelu, ośrodek wydobycia fosforytów}\end{PuzzleClues}\newpage%\section*{Krzyżówka 41}

\noindent\begin{Puzzle}{21}{26}|*	|*	|*	|*	|*	|*	|*	|*	|*	|*	|*	|*	|*	|*	|*	|[1][S]\drarr	|c	|y	|g	|a	|n	|*	|.
|*	|*	|*	|*	|[2][S]\darr	|*	|*	|[3][S]\rarr	|o	|f	|t	|a	|l	|m	|o	|p	|l	|e	|g	|i	|a	|*	|.
|*	|*	|*	|[4][S]\drarr	|r	|z	|ą	|d	|*	|*	|[5][S]\darr	|[6][S]\rarr	|k	|a	|d	|ł	|u	|b	|*	|[7][S]\darr	|*	|[8][S]\darr	|.
|*	|*	|*	|f	|y	|[9][S]\darr	|*	|*	|*	|[10][S]\drarr	|b	|o	|m	|b	|i	|a	|r	|z	|*	|n	|*	|j	|.
|*	|*	|*	|a	|b	|s	|*	|[11][S]\darr	|[12][S]\rarr	|p	|a	|n	|i	|*	|[13][S]\darr	|s	|[14][S]\darr	|*	|[15][S]\darr	|e	|[16][S]\darr	|o	|.
|*	|*	|*	|k	|i	|t	|*	|p	|[17][S]\darr	|i	|n	|[18][S]\darr	|*	|*	|n	|z	|r	|[19][S]\darr	|w	|r	|w	|n	|.
|*	|*	|*	|c	|[][,]{ }	|r	|[20][S]\drarr	|a	|d	|r	|e	|s	|[][,]{ }	|d	|o	|c	|e	|l	|o	|w	|y	|*	|.
|*	|*	|*	|j	|p	|u	|c	|s	|ę	|i	|r	|ł	|*	|*	|r	|z	|n	|i	|d	|[][,]{ }	|d	|*	|.
|*	|*	|*	|o	|ę	|m	|z	|o	|t	|[][S]-	|[][,]{ }	|u	|*	|*	|n	|y	|n	|b	|a	|u	|r	|[21][S]\darr	|.
|*	|*	|*	|n	|c	|i	|u	|ż	|k	|p	|r	|c	|*	|*	|i	|k	|i	|u	|[][,]{ }	|d	|a	|p	|.
|*	|*	|*	|i	|h	|e	|b	|y	|a	|i	|e	|h	|[22][S]\darr	|[23][S]\darr	|c	|*	|n	|r	|g	|o	|*	|l	|.
|*	|*	|[24][S]\rarr	|s	|e	|n	|a	|t	|*	|r	|k	|y	|n	|g	|a	|*	|a	|n	|u	|w	|*	|a	|.
|*	|[25][S]\darr	|*	|t	|r	|i	|t	|[][,]{ }	|*	|i	|l	|*	|i	|r	|*	|*	|*	|a	|l	|o	|*	|n	|.
|*	|k	|[26][S]\darr	|a	|z	|a	|o	|m	|*	|*	|a	|[27][S]\drarr	|l	|o	|g	|o	|s	|*	|a	|[][S]-	|*	|s	|.
|*	|o	|c	|*	|*	|n	|ś	|o	|*	|*	|m	|c	|s	|n	|*	|*	|*	|*	|r	|g	|[28][S]\darr	|z	|.
|*	|s	|o	|*	|*	|k	|ć	|n	|*	|*	|o	|i	|e	|o	|[29][S]\darr	|*	|*	|*	|d	|o	|h	|a	|.
|[30][S]\rarr	|m	|u	|ł	|ł	|a	|*	|o	|*	|*	|w	|e	|n	|w	|p	|[31][S]\rarr	|p	|i	|o	|l	|a	|*	|.
|*	|o	|t	|*	|*	|*	|*	|k	|*	|*	|y	|ń	|*	|s	|e	|*	|*	|*	|w	|e	|n	|*	|.
|*	|d	|a	|[32][S]\rarr	|m	|i	|k	|s	|e	|r	|*	|*	|*	|k	|r	|[33][S]\rarr	|s	|e	|a	|n	|s	|*	|.
|*	|r	|u	|[34][S]\rarr	|p	|r	|z	|e	|l	|i	|c	|z	|n	|i	|k	|o	|w	|y	|*	|i	|e	|*	|.
|*	|o	|d	|*	|[35][S]\rarr	|g	|o	|n	|i	|a	|t	|y	|t	|*	|i	|[36][S]\rarr	|g	|e	|k	|o	|n	|*	|.
|*	|m	|*	|*	|*	|[37][S]\rarr	|p	|i	|a	|n	|k	|a	|*	|*	|n	|*	|*	|*	|[38][S]\darr	|w	|*	|*	|.
|*	|*	|*	|*	|*	|*	|[39][S]\rarr	|c	|h	|a	|r	|t	|r	|e	|s	|*	|*	|*	|m	|y	|*	|*	|.
|[40][S]\rarr	|o	|d	|w	|i	|e	|d	|z	|a	|l	|n	|o	|ś	|ć	|*	|*	|*	|*	|i	|*	|*	|*	|.
|*	|[41][S]\rarr	|a	|z	|o	|t	|a	|n	|[][,]{ }	|w	|a	|p	|n	|i	|a	|*	|*	|*	|ł	|*	|*	|*	|.
|[42][S]\rarr	|d	|z	|i	|e	|r	|z	|y	|k	|[][,]{ }	|r	|ó	|w	|n	|i	|k	|o	|w	|y	|*	|*	|*	|.
|*	|[43][S]\rarr	|z	|a	|w	|i	|s	|*	|*	|*	|*	|*	|*	|*	|*	|*	|*	|*	|*	|*	|*	|*	|.\end{Puzzle}

\newpage

\begin{PuzzleClues}{\textbf{Poziome}\\}\Clue{1}{}{ktoś, kto ma śniadą cerę i ciemne włosy, ale etnicznie nie jest Cyganem}
\Clue{3}{}{chorobliwe unieruchomienie gałki ocznej, wynikające z porażenia nerwu ruchowego oka}
\Clue{4}{}{władanie (jak wrząd dusz lub wzwiązek rządu)}
\Clue{6}{}{korpus}
\Clue{10}{}{przestępca, który podkłada ładunki wybuchowe lub wszczyna fałszywy alarm z powodu rzekomo podłożonej bomby}
\Clue{12}{}{żartobliwie o partnerce, np. żonie, narzeczonej, kochance}
\Clue{20}{}{w protokole IP liczba nadawana interfejsowi sieciowemu komputera będącego odbiorcą pakietu}
\Clue{24}{}{izba wyższa parlamentu}
\Clue{27}{}{w teologii chrześcijańskiej druga osoba boska, Syn Boży}
\Clue{30}{}{główny sędzia, doktor prawa muzułmańskiego; tytuł muzułmańskich duchownych}
\Clue{31}{}{przyrząd do ręcznego odmięsania skór futerkowych}
\Clue{32}{}{urządzenie do rozdrabniania, mieszania, ubijania składników potraw lub napojów}
\Clue{33}{}{spotkanie odbywające się w uprzednio ustalonym celu}
\Clue{34}{}{wojskowy, który nastawia na przeliczniku artyleryjskim odpowiednie parametry}
\Clue{35}{}{jeden z wymarłych głowonogów z podgromady amonitów}
\Clue{36}{}{niewielka owadożerna jaszczurka z rodziny o tej samej nazwie}
\Clue{37}{}{rodzaj kosmetyku o strukturze piany}
\Clue{39}{}{miasto w środkowej Francji, nad rzeką Eure (dopływ Sekwany)}
\Clue{40}{}{liczba odwiedzin notowana przez miejsce (często w przestrzeni wirtualnej), któremu odwiedziny przynoszą jakiś zysk}
\Clue{41}{}{sól kwasu azotowego i wapnia}
\Clue{42}{}{Laniarius major - gatunek ptaka  z rodziny dzierzbików (Malaconotidae)}
\Clue{43}{}{rodzaj lotu statku powietrznego, podczas którego jest on nieruchomy względem powietrza}\end{PuzzleClues}

\begin{PuzzleClues}{\textbf{Pionowe}\\}\Clue{1}{}{lekkie, krótkie okrycie wierzchnie}
\Clue{2}{}{asymetryczny motyw dekoracyjny o kształcie przypominającym pęcherz pławny ryby}
\Clue{4}{}{członek fakcji - wyróżnionej kategorii w obrębie grupy społecznej}
\Clue{5}{}{wielkoformatowy plakat służący reklamie}
\Clue{7}{}{przyśrodkowa gałąź warstwy głębokiej końcowych gałęzi nerwu udowego i jest najdłuższą gałęzią skórną tego nerwu}
\Clue{8}{}{przedstawiciel plemion protogreckich, które wraz z Achajami i Eolami około 2 tysiąclecia p.n.e. zasiedliły środkową i południową Grecję}
\Clue{9}{}{mieszkanka Strumienia}
\Clue{10}{}{ostra w smaku mała papryczka, owoc rośliny o tej samej nazwie}
\Clue{11}{}{organizm pasożytujący na osobnikach należących do jednego gatunku}
\Clue{13}{}{niewielki gryzoń z rodziny nornikowatych, żyje głównie w europejskich, środkowoazjatyckich i północnoamerykańskich zaroślach i lasach}
\Clue{14}{}{PODPUSZCZKA}
\Clue{15}{}{roztwór wodny octanu ołowiu o silnym działaniu ściągającym i skurczowym, stosowany w XVIII i XIX w. w kosmetyce i medycynie}
\Clue{16}{}{wydra europejska, Lutra lutra - gatunek niewielkiego drapieżnego ssaka z rodziny łasicowatych (Mustelidae), jedyny żyjący w Polsce w stanie naturalnym przedstawiciel rodzaju Lutra}
\Clue{17}{}{komora z elastycznego materiału, wypełniona gazem (najczęściej powietrzem) tłoczonym poprzez wentyl; element ogumienia pojazdów}
\Clue{18}{}{uszy dzika, królika, zająca w gwarze łowieckiej}
\Clue{19}{}{szybki, zwrotny, jednorzędowy okręt wojenny z taranem i jednym masztem}
\Clue{20}{}{to, że coś ma czubaty kształt}
\Clue{21}{}{pole, na którym odbywa się pojedynek szermierczy}
\Clue{22}{}{(1901-29), poeta norweski, twórca K. P. Norwegii}
\Clue{23}{}{(GRÜNEWALD); niemiecki malarz i rysownik (między 1460 a 1480-1528) obrazy religijne}
\Clue{25}{}{wydzielony teren do montażu i startu obiektów kosmicznych}
\Clue{26}{}{rodzina XVII/XVIII w. rzeźbiarzy francuskich, przedstawiciele sztuki dworskiej przełomu baroku i rokoka}
\Clue{27}{}{duch, charakter czegoś, co odczuwalne, odbierane przez ludzi, ma lub miało wielką wagę}
\Clue{28}{}{Gerhard, lekarz norweski (1841-1912); odkrywca zarazków trądu}
\Clue{29}{}{pływak australijski, dwukrotny mistrz olimpijski z Barcelony i Atlanty, na dystansie 1500 m stylem dowolnym}
\Clue{38}{}{ukochany - mężczyzna darzony uczuciem}\end{PuzzleClues}\newpage%\section*{Krzyżówka 43}

\noindent\begin{Puzzle}{21}{23}|*	|*	|[1][S]\drarr	|l	|e	|d	|*	|*	|[2][S]\drarr	|p	|r	|ę	|t	|n	|i	|k	|*	|[3][S]\drarr	|r	|ó	|g	|*	|.
|*	|*	|a	|[4][S]\rarr	|j	|a	|r	|z	|m	|o	|*	|[5][S]\rarr	|v	|i	|v	|i	|e	|r	|*	|*	|*	|*	|.
|*	|[6][S]\darr	|g	|[7][S]\rarr	|s	|p	|i	|r	|y	|t	|u	|s	|[][,]{ }	|e	|t	|y	|l	|o	|w	|y	|*	|*	|.
|[8][S]\drarr	|d	|r	|ą	|ż	|e	|k	|[][,]{ }	|s	|t	|e	|r	|o	|w	|n	|i	|c	|z	|y	|*	|[9][S]\darr	|*	|.
|h	|r	|e	|*	|*	|*	|*	|*	|i	|*	|[10][S]\drarr	|m	|i	|n	|i	|[][S]-	|a	|l	|b	|u	|m	|*	|.
|i	|i	|g	|*	|[11][S]\rarr	|c	|o	|o	|k	|*	|v	|*	|*	|[12][S]\darr	|*	|[13][S]\darr	|[14][S]\darr	|e	|*	|*	|a	|*	|.
|s	|o	|a	|*	|*	|[15][S]\drarr	|b	|u	|r	|d	|a	|*	|*	|s	|[16][S]\darr	|c	|s	|g	|*	|*	|s	|*	|.
|t	|p	|c	|*	|*	|g	|[17][S]\darr	|[18][S]\darr	|ó	|[19][S]\darr	|c	|*	|*	|z	|g	|h	|z	|ł	|*	|*	|t	|*	|.
|o	|i	|j	|*	|*	|r	|w	|p	|l	|t	|*	|*	|*	|t	|a	|l	|e	|o	|[20][S]\darr	|[21][S]\darr	|e	|*	|.
|r	|t	|a	|*	|*	|a	|i	|u	|i	|a	|[22][S]\darr	|*	|*	|u	|l	|u	|s	|ś	|s	|p	|k	|*	|.
|i	|e	|*	|*	|*	|n	|ę	|s	|k	|k	|f	|*	|*	|k	|e	|b	|n	|ć	|p	|a	|t	|*	|.
|a	|k	|[23][S]\darr	|*	|[24][S]\darr	|u	|z	|a	|*	|y	|r	|[25][S]\drarr	|s	|a	|r	|n	|a	|*	|i	|r	|o	|*	|.
|[][,]{ }	|i	|f	|[26][S]\drarr	|b	|l	|i	|n	|d	|r	|e	|j	|a	|*	|i	|o	|s	|*	|e	|a	|m	|*	|.
|s	|*	|i	|p	|o	|a	|e	|*	|*	|*	|j	|a	|[27][S]\darr	|[28][S]\darr	|a	|ś	|t	|*	|k	|g	|i	|*	|.
|z	|*	|n	|a	|r	|t	|ń	|*	|*	|*	|l	|d	|l	|e	|*	|ć	|k	|*	|a	|w	|a	|*	|.
|t	|*	|i	|t	|u	|*	|*	|*	|*	|*	|i	|w	|a	|r	|*	|*	|a	|*	|l	|a	|*	|*	|.
|u	|*	|s	|a	|t	|*	|[29][S]\rarr	|a	|k	|o	|n	|i	|t	|*	|*	|*	|*	|*	|n	|j	|[30][S]\darr	|*	|.
|k	|[31][S]\rarr	|z	|n	|a	|k	|[][,]{ }	|t	|o	|n	|a	|ż	|o	|w	|y	|*	|*	|*	|o	|k	|t	|*	|.
|i	|*	|*	|*	|*	|[32][S]\rarr	|p	|u	|d	|a	|*	|a	|*	|*	|*	|*	|*	|*	|ś	|a	|u	|*	|.
|*	|*	|*	|[33][S]\drarr	|k	|o	|m	|p	|e	|t	|e	|n	|c	|y	|j	|n	|o	|ś	|ć	|*	|k	|*	|.
|*	|[34][S]\rarr	|r	|o	|d	|z	|i	|c	|i	|e	|l	|k	|a	|*	|*	|[35][S]\rarr	|r	|g	|*	|*	|a	|*	|.
|[36][S]\rarr	|k	|a	|s	|e	|t	|a	|*	|*	|[37][S]\rarr	|m	|a	|l	|u	|c	|z	|k	|o	|ś	|ć	|*	|*	|.
|*	|[38][S]\rarr	|z	|a	|k	|r	|y	|s	|t	|i	|a	|*	|*	|*	|*	|*	|*	|*	|*	|*	|*	|*	|.
|*	|*	|*	|*	|*	|*	|*	|*	|*	|*	|*	|*	|*	|*	|*	|*	|*	|*	|*	|*	|*	|*	|.\end{Puzzle}

\newpage

\begin{PuzzleClues}{\textbf{Poziome}\\}\Clue{1}{}{dioda zaliczana do półprzewodnikowych przyrządów optoelektronicznych, emitujących promieniowanie w zakresie światła widzialnego, podczerwieni i ultrafioletu}
\Clue{2}{}{Trichogaster, Colisa - rodzaj ryb słodkowodnych z rzędu okoniokształtnych}
\Clue{3}{}{przedmiot wykonany z rogu zwierzęcego lub przypominający go kształtem, zwłaszcza naczynie, np. róg obfitości, róg z tabaką}
\Clue{4}{}{w elektrotechnice: nieuzwojony element przeznaczony do łączenia rdzeni elektromagnesu, transformatora itp}
\Clue{5}{}{pisarz belgijski tworzący w języku francuskim ur. 1894r, liryki, powieści, eseje, przekłady}
\Clue{7}{}{roztwór alkoholu etylowego o zawartości alkoholu ok. 96\%}
\Clue{8}{}{dźwignia służąca do ręcznego sterowania statkiem powietrznym}
\Clue{10}{}{album muzyczny, który jest zbyt długi, by nazwać go zwykłym singlem czy też - ze względu na zawarty na nim materiał - niekwalifikujący się jako maxi-singel, i który jest jednocześnie zbyt krótki, by uznać go za pełnoprawny album długogrający}
\Clue{11}{}{(1808-92); Anglik, założyciel pierwszego w świecie biura podróży}
\Clue{15}{}{kawał wełnianej tkaniny używany jako płaszcz lub koc przez Arabów}
\Clue{25}{}{ludowa nazwa sarniaka dachówkowatego, grzyba  z rodziny kolcownicowatych}
\Clue{26}{}{reja umocowana na bukszprycie}
\Clue{29}{}{TOJAD}
\Clue{31}{}{oznaczenie statku informujące o dopuszczalnym zanurzeniu statku w wodzie morskiej i słodkiej}
\Clue{32}{}{danie wegetariańskie podobne do omletu}
\Clue{33}{}{profesjonalne obeznanie, biegłość w jakiejś dziedzinie}
\Clue{34}{}{kobieta, która urodziła i (najczęściej) wychowała dziecko (w relacji do tego dziecka)}
\Clue{35}{}{w chemii: symbol pierwiastka roentgen}
\Clue{36}{}{światłoszczelny pojemnik na kliszę , błonę małoobrazkową umożliwiającą zakładanie materiału fotograficznego do aparatu przy świetle}
\Clue{37}{}{określenie człowieka przeciętnego, skromnego, niewyróżniającego się niczym}
\Clue{38}{}{boczne pomieszczenie w świątyni przylegające do prezbiterium}\end{PuzzleClues}

\begin{PuzzleClues}{\textbf{Pionowe}\\}\Clue{1}{}{w szerokim rozumieniu: proces łączenia mniejszych elementów w większą całość, tworzenia całości z czegoś rozproszonego}
\Clue{2}{}{malutki ptak z rzędu wróblowych (Passeriformes), z rodziny mysikrólikowatych (Regulidae)}
\Clue{3}{}{bardzo duży zakres, rozmiar}
\Clue{6}{}{kopalne małpy człekokształtne z miocenu, przodkowie gibbonów}
\Clue{8}{}{przedmiot szkolny, na którym naucza się o sztukach wizualnych w ujęciu historycznym}
\Clue{9}{}{zabieg amputacji piersi}
\Clue{10}{}{miasto na Węgrzech (komitat Peszt) port nad Dunajem}
\Clue{12}{}{dziedzina ludzkiej działalności artystycznej}
\Clue{13}{}{to, że coś jest chlubne - stanowi powód do dumy; na przykład o działaniu}
\Clue{14}{}{liczba 16, numer 16}
\Clue{15}{}{substancja występująca w postaci granulek}
\Clue{16}{}{zgromadzona w jednym miejscu kolekcja wytworów artystycznych, dzieł sztuki; wystawa, stanowiąca samodzielną instytucję albo część większego muzeum}
\Clue{17}{}{ktoś, kto jest więziony, przebywa w więzieniu, jest pozbawiony wolności za karę}
\Clue{18}{}{miasto i główny port Korei Płd. nad Cieśniną Koreańska}
\Clue{19}{}{ilasta kotlina w obszarze pustyń piaszczystych w Azji Środkowej}
\Clue{20}{}{podatność na spiekanie}
\Clue{21}{}{mieszkanka Paragwaju, kobieta pochodzenia paragwajskiego}
\Clue{22}{}{dama dworu cara rosyjskiego}
\Clue{23}{}{ostatni fragment trasy podczas wyścigu, ostatni odcinek do przebycia}
\Clue{24}{}{fikcyjna postać, diabeł zamieszkujący podziemia zamku w Łęczycy}
\Clue{25}{}{siostra zakonna ze Zgromadzenia Sióstr Św. Jadwigi}
\Clue{26}{}{miasto w Nepalu w pobliżu Katmandu}
\Clue{27}{}{jedna z czterech podstawowych pór roku w klimacie umiarkowanym, cechująca się najwyższymi średnimi temperaturami powietrza w skali roku}
\Clue{28}{}{w chemii: symbol erbu}
\Clue{30}{}{ciągnione, czynne narzędzie sieciowe do połowu ryb, podobne do włoka, holowane przez dwa trawlery}
\Clue{33}{}{błonkówka z grupy żądłówek, zwykle drapieżna}\end{PuzzleClues}\newpage%\section*{Krzyżówka 44}

\noindent\begin{Puzzle}{17}{28}|*	|*	|[1][S]\drarr	|t	|r	|a	|u	|m	|a	|t	|y	|c	|z	|n	|o	|ś	|ć	|*	|.
|*	|[2][S]\rarr	|m	|a	|n	|n	|a	|*	|[3][S]\drarr	|g	|r	|o	|t	|e	|s	|k	|a	|*	|.
|*	|*	|l	|*	|*	|[4][S]\darr	|*	|[5][S]\rarr	|d	|o	|k	|*	|*	|*	|[6][S]\darr	|[7][S]\darr	|*	|*	|.
|*	|[8][S]\drarr	|e	|f	|e	|k	|t	|[][,]{ }	|r	|y	|g	|l	|a	|*	|z	|h	|*	|*	|.
|*	|m	|c	|*	|[9][S]\darr	|o	|*	|*	|u	|*	|[10][S]\drarr	|m	|ą	|t	|w	|i	|k	|*	|.
|*	|o	|z	|*	|c	|s	|*	|*	|ż	|*	|l	|*	|*	|*	|i	|s	|[11][S]\darr	|*	|.
|*	|d	|a	|*	|i	|z	|[12][S]\drarr	|w	|y	|l	|e	|w	|*	|*	|e	|t	|t	|[13][S]\darr	|.
|*	|e	|r	|*	|a	|y	|b	|[14][S]\rarr	|n	|a	|p	|ó	|r	|*	|ń	|o	|o	|a	|.
|*	|r	|z	|[15][S]\drarr	|s	|k	|a	|ł	|a	|*	|i	|*	|*	|*	|c	|r	|n	|c	|.
|*	|a	|*	|m	|t	|[][,]{ }	|l	|[16][S]\darr	|*	|[17][S]\drarr	|d	|u	|ń	|c	|z	|y	|k	|*	|.
|*	|t	|[18][S]\darr	|ą	|o	|w	|o	|l	|*	|a	|o	|[19][S]\darr	|[20][S]\darr	|*	|e	|z	|i	|[21][S]\darr	|.
|*	|o	|ś	|ż	|*	|a	|n	|e	|*	|m	|f	|w	|p	|[22][S]\darr	|n	|m	|n	|h	|.
|*	|*	|r	|*	|[23][S]\darr	|l	|*	|w	|*	|y	|i	|e	|o	|d	|i	|*	|*	|a	|.
|*	|*	|e	|[24][S]\rarr	|a	|u	|l	|*	|*	|*	|t	|i	|r	|r	|e	|*	|*	|z	|.
|*	|*	|d	|*	|s	|t	|*	|[25][S]\rarr	|a	|s	|y	|s	|t	|a	|*	|[26][S]\darr	|*	|a	|.
|[27][S]\drarr	|g	|n	|e	|t	|o	|w	|a	|t	|e	|*	|s	|i	|ż	|[28][S]\darr	|n	|*	|r	|.
|o	|*	|i	|*	|e	|w	|*	|*	|*	|*	|*	|*	|n	|e	|g	|a	|*	|d	|.
|s	|[29][S]\darr	|o	|[30][S]\rarr	|r	|y	|j	|ó	|w	|k	|o	|w	|a	|t	|e	|*	|*	|z	|.
|*	|k	|w	|*	|[][,]{ }	|*	|*	|*	|*	|*	|*	|[31][S]\darr	|r	|k	|o	|*	|[32][S]\darr	|i	|.
|*	|a	|i	|[33][S]\rarr	|l	|i	|d	|a	|r	|*	|[34][S]\darr	|h	|i	|a	|m	|[35][S]\darr	|p	|s	|.
|[36][S]\drarr	|r	|e	|z	|o	|n	|a	|n	|s	|*	|k	|e	|*	|*	|a	|h	|ó	|t	|.
|t	|a	|c	|[37][S]\rarr	|w	|r	|z	|e	|c	|i	|o	|n	|o	|*	|n	|u	|ł	|a	|.
|a	|i	|z	|[38][S]\rarr	|r	|e	|j	|o	|n	|*	|m	|n	|*	|[39][S]\rarr	|c	|d	|f	|*	|.
|r	|t	|n	|*	|i	|[40][S]\darr	|*	|*	|*	|*	|p	|a	|*	|*	|j	|s	|i	|*	|.
|p	|a	|i	|[41][S]\drarr	|e	|p	|i	|c	|y	|k	|l	|*	|*	|*	|a	|o	|g	|*	|.
|o	|*	|k	|m	|g	|s	|*	|[42][S]\rarr	|o	|b	|e	|r	|a	|*	|*	|n	|u	|*	|.
|n	|*	|*	|a	|o	|y	|[43][S]\drarr	|p	|ł	|o	|t	|k	|a	|r	|z	|*	|r	|*	|.
|*	|*	|*	|g	|*	|*	|d	|*	|*	|*	|*	|*	|*	|*	|*	|*	|a	|*	|.
|*	|*	|*	|*	|*	|*	|*	|*	|*	|*	|*	|*	|*	|*	|*	|*	|*	|*	|.\end{Puzzle}

\newpage

\begin{PuzzleClues}{\textbf{Poziome}\\}\Clue{1}{}{silny lęk związany z danymi przeżyciami lub sytuacjami}
\Clue{2}{}{odchody pluskwiaka żyjącego na tamaryszku, zastygające na powietrzu w twardą masę}
\Clue{3}{}{utwór muzyczny o elementach komicznie przejaskrawionych, nieprawdopodobnych}
\Clue{5}{}{budowla lub pływające urządzenie w stoczni do budowy lub remontu statków}
\Clue{8}{}{zjawisko ekonomiczne polegające na tym, że pomimo spadku poziomu dochodu gospodarstwa domowego nie zmniejsza się wielkość jego konsumpcji}
\Clue{10}{}{bezkręgowiec pasożytniczy z typu nicieni}
\Clue{12}{}{udar mózgu - zespół objawów klinicznych związanych z nagłym wystąpieniem ogniskowego lub uogólnionego zaburzenia czynności mózgu, powstały w wyniku zaburzenia krążenia mózgowego i utrzymujący się ponad 24 godziny}
\Clue{14}{}{nacisk, silne oddziaływanie czynników (zarówno pochodzących od człowieka, celowych, jak i zewnętrznych, niezależnych, np. czasu) na kogoś}
\Clue{15}{}{duże skupisko minerałów}
\Clue{17}{}{mieszkaniec Danii, człowiek pochodzenia duńskiego}
\Clue{24}{}{Gazella soemmerringii, Nanger soemmerringii - gatunek ssaka parzystokopytnego z rodziny krętorogich; zamieszkuje tereny Afryki Wschodniej (Dżibuti, Erytreę, Etiopię, Kenię, Somalię i Sudan)}
\Clue{25}{}{w sporcie: podanie do zawodnika z tej samej drużyny, po którym zostają zdobyte punkty lub jest zdobyta bramka}
\Clue{27}{}{gniotowate, Gnetaceae - monotypowa rodzina z rzędu gniotowców (Gnetales); rośliny pantropikalne, występujące w wilgotnych lasach równikowych; należą do nagonasiennych, ale ich organy wegetatywne przypominają rośliny dwuliścienne}
\Clue{30}{}{sorki, Soricidae - rodzina ssaków z rzędu owadożernych;  zamieszkują Azję, Europę, Afrykę, Amerykę Północną i Południową}
\Clue{33}{}{urządzenie optyczne podobne do radaru, stosowane przy badaniu chmur}
\Clue{36}{}{reakcja na jakieś zdarzenie}
\Clue{37}{}{precyzyjnie ułożyskowany element obrabiarki w kształcie wału, najczęściej z otworem osiowym}
\Clue{38}{}{umownie wydzielony obszar, względnie jednorodny, różniący się od terenów sąsiednich cechami naturalnymi bądź nabytymi na przestrzeni dziejów}
\Clue{39}{}{kod ISO 4217 franka kongijskiego}
\Clue{41}{}{orbita kolista, po której - w teorii geocentrycznej - poruszać się miały ciała niebieskie, środek tego koła przemieszczał się ruchem jednostajnym wokół Ziemi po wielkim kole orbity zwanej deferentem}
\Clue{42}{}{miasto w Argentynie (Misiones)}
\Clue{43}{}{zawodnik lekkoatletycznej dyscypliny - biegu przez płotki}\end{PuzzleClues}

\begin{PuzzleClues}{\textbf{Pionowe}\\}\Clue{1}{}{specjalista zajmujący się mleczarstwem}
\Clue{3}{}{podstawowa jednostka w strukturze organizacji harcerskich, skupiająca harcerzy i harcerki}
\Clue{4}{}{system wyznaczania kursu waluty danego kraju w oparciu o kompozycję walut krajów obcych}
\Clue{6}{}{efekt pracy, ostateczny, bardzo efektowny i ważny rezultat czyjegoś zaplanowanego działania}
\Clue{7}{}{kierunek w architekturze XIX w polegający na naśladowaniu stylów epok minionych}
\Clue{8}{}{określenie wykonawcze; umiarkowanie}
\Clue{9}{}{substancja, plastyczna masa zazwyczaj z mąki i jakiegoś płynu (czasem też jaj, tłuszczu i innych składników), z której może powstać potrawa, zwykle w wyniku upieczenia lub ugotowania}
\Clue{10}{}{kopalne rośliny drzewiaste z klasy widłaków}
\Clue{11}{}{historyczna nazwa północnego Wietnamu, obejmującego dorzecze Rzeki Czerwonej}
\Clue{12}{}{tyle, ile mieści się w balonie - butli}
\Clue{13}{}{w chemii: symbol aktynu}
\Clue{15}{}{dojrzały płciowo męski osobnik z rodzaju Homo; mężczyzna}
\Clue{16}{}{osoba spod znaku Lwa}
\Clue{17}{}{ur. w 1936 r. kompozytor i dyrygent francuski; utwory orkiestrowe. kameralne, wokalno-instrumentalne}
\Clue{18}{}{mediewista - historyk, badacz okresu średniowiecza}
\Clue{19}{}{Jan, pisarz czeski (1892-1972), psychoanalityczne powieści o tematyce społecznej; „Więzień Zodiaku”}
\Clue{20}{}{brazylijski malarz i grafik (1903-62), monumentalne malowidła ścienne, sceny z życia ludu brazylijskiego}
\Clue{21}{}{ktoś, kto dużo ryzykuje}
\Clue{22}{}{tabletka powlekana otoczką cukrową}
\Clue{23}{}{Aster lowrieanus - gatunek rośliny należący do rodziny astrowatych}
\Clue{26}{}{w chemii: symbol sodu}
\Clue{27}{}{w chemii: symbol osmu}
\Clue{28}{}{wróżenie ze znaków nakreślonych na ziemi, elementów środowiska, topografii}
\Clue{29}{}{wyznawca karaimizmu}
\Clue{31}{}{barwnik roślinny produkowany z liści i pędów rośliny zwanej lawsonią bezbronną}
\Clue{32}{}{połowa figury}
\Clue{34}{}{wszyscy członkowie jakiejś zbiorowości}
\Clue{35}{}{(1550-1611 ) żeglarz angielski; zginął porzucony przez zbuntowaną załogę}
\Clue{36}{}{ryba występująca w tropikalnej strefie Oceanu Atlantyckiego; ceniona w połowach wędkarskich}
\Clue{40}{}{PSOWATE;}
\Clue{41}{}{jeden z biblijnych mędrców, tzw. Trzech Króli}
\Clue{43}{}{litera alfabetu używana w numeracji porządkowej}\end{PuzzleClues}\newpage%\section*{Krzyżówka 45}

\noindent\begin{Puzzle}{22}{33}|*	|*	|*	|*	|*	|*	|*	|*	|*	|*	|*	|*	|*	|*	|*	|*	|*	|*	|[1][S]\darr	|*	|*	|*	|*	|.
|*	|*	|*	|*	|*	|*	|*	|*	|*	|*	|*	|*	|*	|*	|*	|*	|*	|*	|h	|*	|*	|[2][S]\darr	|*	|.
|*	|*	|*	|*	|*	|*	|*	|*	|*	|*	|*	|*	|*	|*	|*	|*	|*	|*	|u	|*	|*	|k	|[3][S]\darr	|.
|*	|*	|*	|*	|*	|*	|*	|*	|*	|*	|*	|*	|*	|*	|*	|*	|*	|*	|m	|*	|*	|u	|k	|.
|*	|*	|*	|*	|*	|*	|*	|*	|*	|[4][S]\drarr	|l	|a	|m	|e	|r	|s	|t	|w	|o	|*	|*	|r	|o	|.
|*	|*	|*	|*	|[5][S]\rarr	|s	|a	|m	|o	|r	|z	|u	|t	|n	|o	|ś	|ć	|*	|r	|[6][S]\darr	|*	|t	|m	|.
|*	|[7][S]\drarr	|s	|t	|e	|r	|o	|w	|i	|e	|c	|[][,]{ }	|m	|i	|ę	|k	|k	|i	|*	|f	|*	|a	|p	|.
|[8][S]\drarr	|a	|n	|t	|y	|m	|o	|n	|e	|k	|[][,]{ }	|g	|a	|l	|u	|*	|*	|[9][S]\darr	|[10][S]\darr	|i	|*	|c	|u	|.
|p	|[][S]0	|*	|[11][S]\rarr	|p	|o	|b	|r	|a	|t	|y	|m	|s	|t	|w	|o	|*	|c	|f	|l	|[12][S]\darr	|z	|t	|.
|o	|*	|[13][S]\rarr	|b	|ą	|k	|[][,]{ }	|z	|w	|y	|c	|z	|a	|j	|n	|y	|*	|z	|u	|m	|b	|e	|e	|.
|j	|[14][S]\rarr	|h	|y	|d	|r	|o	|r	|a	|f	|i	|n	|a	|c	|j	|a	|*	|a	|j	|o	|r	|k	|r	|.
|e	|[15][S]\drarr	|g	|ł	|o	|w	|a	|[][,]{ }	|n	|i	|e	|[][,]{ }	|o	|d	|[][,]{ }	|p	|a	|r	|a	|d	|y	|*	|o	|.
|d	|k	|*	|*	|*	|*	|*	|[16][S]\darr	|[17][S]\rarr	|k	|l	|e	|i	|s	|t	|*	|*	|a	|r	|r	|ł	|[18][S]\darr	|w	|.
|y	|o	|[19][S]\darr	|*	|[20][S]\drarr	|s	|u	|s	|z	|a	|r	|k	|a	|*	|*	|*	|*	|*	|k	|u	|a	|r	|i	|.
|n	|ś	|ż	|*	|k	|*	|[21][S]\drarr	|k	|i	|c	|z	|*	|[22][S]\drarr	|a	|n	|t	|c	|z	|a	|k	|*	|u	|e	|.
|e	|ć	|a	|*	|a	|*	|n	|a	|*	|j	|[23][S]\darr	|*	|z	|*	|*	|[24][S]\darr	|*	|[25][S]\darr	|*	|*	|*	|d	|c	|.
|k	|[][,]{ }	|b	|*	|*	|*	|a	|f	|*	|a	|k	|[26][S]\darr	|m	|[27][S]\darr	|[28][S]\drarr	|b	|a	|s	|k	|i	|n	|a	|*	|.
|[][,]{ }	|s	|a	|*	|[29][S]\drarr	|g	|r	|a	|f	|[][,]{ }	|a	|c	|y	|k	|l	|i	|c	|z	|n	|y	|*	|w	|*	|.
|s	|k	|[][,]{ }	|[30][S]\darr	|ł	|*	|o	|n	|*	|k	|r	|y	|w	|i	|i	|n	|*	|c	|*	|[31][S]\darr	|*	|k	|*	|.
|ą	|r	|c	|p	|a	|[32][S]\darr	|d	|d	|*	|r	|a	|j	|a	|e	|p	|d	|[33][S]\darr	|z	|[34][S]\darr	|s	|*	|a	|*	|.
|d	|o	|z	|r	|ń	|p	|n	|e	|*	|z	|b	|a	|c	|r	|m	|o	|f	|o	|p	|z	|*	|[][,]{ }	|*	|.
|o	|n	|e	|o	|c	|a	|i	|r	|*	|y	|i	|n	|z	|o	|a	|w	|i	|d	|i	|c	|*	|m	|*	|.
|w	|i	|r	|e	|u	|l	|k	|*	|*	|w	|n	|e	|*	|w	|n	|n	|l	|r	|o	|z	|*	|a	|*	|.
|y	|o	|w	|p	|s	|e	|*	|[35][S]\darr	|[36][S]\darr	|e	|[][,]{ }	|l	|*	|n	|n	|i	|o	|z	|t	|e	|*	|l	|*	|.
|*	|w	|o	|i	|z	|c	|*	|f	|b	|j	|m	|l	|*	|i	|*	|c	|l	|e	|r	|k	|*	|a	|*	|.
|[37][S]\drarr	|a	|n	|d	|e	|z	|a	|u	|r	|*	|a	|a	|*	|c	|*	|a	|o	|n	|o	|u	|[38][S]\darr	|j	|*	|.
|w	|*	|o	|e	|k	|n	|[39][S]\darr	|z	|a	|*	|s	|*	|*	|t	|*	|*	|g	|i	|w	|s	|b	|s	|*	|.
|i	|*	|u	|m	|*	|y	|b	|u	|c	|*	|z	|*	|*	|w	|[40][S]\rarr	|w	|i	|e	|s	|z	|a	|k	|*	|.
|e	|*	|c	|i	|*	|*	|a	|l	|h	|*	|y	|*	|*	|o	|*	|*	|a	|c	|k	|k	|t	|a	|*	|.
|n	|*	|h	|c	|[41][S]\rarr	|ś	|m	|i	|e	|t	|n	|i	|k	|*	|*	|*	|*	|*	|i	|i	|y	|*	|*	|.
|i	|*	|a	|z	|*	|*	|*	|*	|t	|*	|o	|[42][S]\rarr	|d	|r	|o	|ż	|d	|ż	|*	|*	|s	|*	|*	|.
|e	|*	|*	|k	|[43][S]\rarr	|w	|a	|ł	|*	|[44][S]\rarr	|w	|ę	|d	|z	|o	|n	|k	|a	|*	|*	|t	|*	|*	|.
|c	|*	|*	|a	|[45][S]\rarr	|b	|i	|e	|r	|n	|y	|[][,]{ }	|z	|a	|w	|o	|d	|o	|w	|o	|*	|*	|*	|.
|*	|*	|*	|*	|*	|[46][S]\rarr	|g	|a	|z	|a	|*	|*	|*	|*	|*	|*	|*	|*	|*	|*	|*	|*	|*	|.\end{Puzzle}

\newpage

\begin{PuzzleClues}{\textbf{Poziome}\\}\Clue{4}{}{brak biegłości, wprawy, wiedzy w jakiejś dziedzinie (bardzo często w zakresie obsługi komputera), a przy tym zachowywanie się niedojrzale, niepoważnie, dziwnie; połączenie niekompetencji z niedojrzałością i nieświadomością własnych braków oraz z brakiem akceptacji przez otoczenie}
\Clue{5}{}{o procesach, zjawiskach itp.: to, że coś się dzieje samo z siebie, bez wpływu czynników zewnętrznych}
\Clue{7}{}{rodzaj sterowca, którego powłoka utrzymuje kształt dzięki nadciśnieniu gazu nośnego}
\Clue{8}{}{związek chemiczny galu i antymonu}
\Clue{11}{}{związek pomiędzy ludźmi oparty o więzy krwi}
\Clue{13}{}{Botaurus stellaris - gatunek dużego wędrownego ptaka wodnego z rodziny czaplowatych (Ardeidae)}
\Clue{14}{}{proces usuwania z paliw uzyskiwanych z ropy naftowej niepożądanych substancji przez działanie wodorem}
\Clue{15}{}{inteligencja, spryt, umiejętności, wiedza - ogół pozytywnie postrzeganych cech związanych z umysłem}
\Clue{17}{}{(1777-1811), pisarz niemiecki, dramatopisarz; „Rozbity dzban”, „Książę Homburgu”, „Jasnowidząca”}
\Clue{20}{}{stojak, na którym umieszcza się najczęściej mokre, wilgotne rzeczy po to, żeby wyschły}
\Clue{21}{}{coś, co zostało zrobione bez polotu, przedmiot charakteryzujący się niską wartością estetyczną, tania tandeta}
\Clue{22}{}{piłkarz ręczny, członek 'brązowej' drużyny z Montrealu}
\Clue{28}{}{falbana w sukni lub bluzce, zaczynająca się w pasie i kończąca w okolicy uda, rozkloszowana; wygląda jak krótka spódniczka nałożona na bluzkę lub sukienkę}
\Clue{29}{}{graf niezawierający cykli (dróg zamkniętych)}
\Clue{37}{}{Andesaurus - rodzaj roślinożernego, czworonożnego zauropoda z rodziny andezaurów; żył w epoce kredy na terenach Ameryki Południowej}
\Clue{40}{}{klamerka lub opaska z haczykiem do zawieszania kabla na lince nośnej}
\Clue{41}{}{miejsce pozbywania się odpadów}
\Clue{42}{}{jednokomórkowy grzyb o małych, elipsoidalnych komórkach}
\Clue{43}{}{część maszyny, najczęściej w kształcie walca, obracająca się wokół własnej osi wraz z zamocowanymi na niej elementami, służąca do przenoszenia momentu obrotowego}
\Clue{44}{}{wędzony przetwór mięsny (zazwyczaj wykonany z jednego kawałka mięsa peklowanego)}
\Clue{45}{}{osoba w wieku powyżej 15 lat, niepracująca i nie będąca bezrobotną}
\Clue{46}{}{chłonny materiał opatrunkowy, produkowany najczęściej z wybielonej bawełny, w postaci rzadkiej tkaniny, następnie zwykle wyjałowiony}\end{PuzzleClues}

\begin{PuzzleClues}{\textbf{Pionowe}\\}\Clue{1}{}{jeden z dojrzałych mechanizmów obronnych, który polega na otwartej ekspresji uczuć bez osobistego dyskomfortu i nieprzyjemnych efektów dla innych}
\Clue{2}{}{leśny ptak o kolorowym upierzeniu i krótkim ogonie; zamieszkuje Australię i Afrykę}
\Clue{3}{}{osoba, która potrafi biegle posługiwać się komputerem, o wysokich umiejętnościach informatycznych, na codzień korzystająca z komputera w stopniu zaawansowanym, hobbysta}
\Clue{4}{}{wykreślanie odcinka, którego długość jest równa długości danej krzywej}
\Clue{6}{}{sitodruk z wykorzystaniem gęstej siatki nałożonej na ramę}
\Clue{7}{}{format zasadniczy arkusza, którego boki mają wymiary 841 na 1189 mm}
\Clue{8}{}{środek dowodowy w średniowiecznym procesie}
\Clue{9}{}{rodzaj ozdobnego kielicha}
\Clue{10}{}{instrument ludowy, piszczałka z kory dębu}
\Clue{12}{}{obiekt, zwykle dość duży; ciało fizyczne o nieregularnej formie, gruby, duży kawał czegoś, gruda}
\Clue{15}{}{parzysta, silnie spneumatyzowana kość wchodząca w skład mózgoczaszki}
\Clue{16}{}{ubiór nurka lub płetwonurka}
\Clue{18}{}{Pteropus vampyrus - gatunek nietoperza z rodziny rudawkowatych (Pteropodidae); występuje w południowo-wschodniej Azji, w Archipelagu Sundajskim, w Brunei, w Kambodży, na Filipinach, w Tajlandii}
\Clue{19}{}{Hylarana erythraea - gatunek płaza bezogonowego z rodziny żabowatych}
\Clue{20}{}{w mitologii egipskiej: duch opiekuńczy człowieka}
\Clue{21}{}{działacz rosyjskiego ruchu demokratycznego lat 70.-90. XIX w}
\Clue{22}{}{substancja używana do zmywania zabrudzenia lub innej niepożądanej substancji z czyszczonej powierzchni}
\Clue{23}{}{samoczynna, zespołowa broń palna strzelająca amunicją karabinową o kalibrze do 20 mm}
\Clue{24}{}{urządzenie biurowe służące do spinania plików papieru za pomocą plastikowej opaski rozdzielonej w części na ciensze paski, które owijają się wokół otworów w kartkach}
\Clue{25}{}{krzewinka lub krzew z rodziny motylkowatych o żółtych kwiatach, uprawiany jako ozdobny}
\Clue{26}{}{jednokomórkowa sinica żyjąca wewnątrz ciała innego organizmu jako endosymbiont}
\Clue{27}{}{kierowanie jakimś przedsięwzięciem lub grupą ludzi}
\Clue{28}{}{biochemik amerykański (1899-1986); wyjaśnił mechanizm fermentacji masłowej i rolę rozszczepiania glikogenu, laureat Nobla}
\Clue{29}{}{zdrobniale: łańcuch - sposób uporządkowania osób lub rzeczy, ustawienie ich obok siebie w linii prostej, szereg obiektów}
\Clue{30}{}{kobieta, która jest przeciwniczką obowiązkowych szczepień, osoba, która nie wierzy w pozytywny wpływ powszechnych szczepień oraz często wierzy w korelację pomiędzy występowaniem niektórych poważnych chorób a szczepieniami}
\Clue{31}{}{OCHOTONY}
\Clue{32}{}{ur. w 1546 r., pianista, laureat Konkursu im. F. Chopina w 1970 r}
\Clue{33}{}{budynek, w którym mieści się wydział filologiczny}
\Clue{34}{}{Stefan, ur. w 1910r. astronom; prace z astrofizyki}
\Clue{35}{}{poeta turecki i azerbejdżański (1495-1556), dywany poezji, utwory religijne, poematy}
\Clue{36}{}{Jean, ur. w 1909 r., cytochemik, cytolog i embriolog belgijski, pierwsze badania nad kwasem rybonukleinowym i jego roli w biosyntezie białek}
\Clue{37}{}{drewniana rama obudowy szybu}
\Clue{38}{}{cienka, przezroczysta tkanina, lniana lub bawełniana o splocie płóciennym}
\Clue{39}{}{kod ISO 4217 marki zamiennej}\end{PuzzleClues}\newpage%\section*{Krzyżówka 46}

\noindent\begin{Puzzle}{22}{33}|*	|*	|*	|*	|[1][S]\darr	|*	|*	|*	|*	|*	|*	|[2][S]\darr	|*	|*	|*	|[3][S]\darr	|*	|[4][S]\darr	|*	|[5][S]\darr	|[6][S]\darr	|*	|*	|.
|*	|*	|*	|*	|m	|*	|*	|*	|*	|*	|*	|c	|*	|*	|[7][S]\darr	|s	|[8][S]\darr	|l	|[9][S]\darr	|u	|k	|*	|*	|.
|*	|*	|*	|[10][S]\darr	|i	|*	|*	|*	|*	|*	|*	|z	|*	|[11][S]\rarr	|d	|e	|d	|e	|r	|k	|o	|*	|*	|.
|*	|*	|*	|m	|s	|*	|*	|*	|*	|*	|*	|ł	|*	|*	|y	|t	|z	|m	|e	|ł	|m	|*	|*	|.
|*	|*	|*	|a	|i	|*	|*	|*	|*	|*	|*	|o	|*	|[12][S]\darr	|s	|n	|i	|a	|p	|a	|u	|*	|*	|.
|*	|*	|*	|n	|ó	|*	|*	|*	|*	|*	|*	|w	|*	|q	|c	|i	|u	|t	|e	|d	|n	|*	|*	|.
|*	|*	|*	|u	|w	|*	|*	|*	|*	|[13][S]\darr	|*	|i	|*	|u	|y	|k	|p	|[][,]{ }	|t	|[][,]{ }	|i	|*	|*	|.
|*	|*	|*	|f	|k	|*	|*	|*	|*	|g	|*	|e	|*	|a	|p	|*	|l	|z	|y	|m	|k	|*	|*	|.
|*	|*	|*	|a	|a	|*	|*	|*	|*	|r	|*	|c	|*	|d	|l	|*	|a	|a	|t	|o	|a	|*	|*	|.
|*	|[14][S]\darr	|*	|k	|[][,]{ }	|*	|*	|[15][S]\drarr	|w	|y	|c	|z	|u	|c	|i	|e	|*	|s	|y	|c	|c	|*	|*	|.
|*	|a	|*	|t	|k	|*	|*	|n	|*	|p	|*	|o	|*	|o	|n	|[16][S]\darr	|*	|s	|w	|z	|j	|*	|*	|.
|*	|r	|*	|u	|a	|*	|*	|e	|*	|k	|*	|ś	|*	|p	|a	|p	|*	|e	|n	|o	|a	|*	|*	|.
|*	|a	|*	|r	|s	|*	|*	|s	|*	|a	|*	|ć	|*	|t	|[][,]{ }	|y	|*	|n	|o	|w	|[][,]{ }	|*	|*	|.
|[17][S]\rarr	|b	|h	|a	|t	|p	|a	|t	|a	|*	|*	|*	|*	|e	|p	|t	|*	|h	|ś	|o	|s	|*	|*	|.
|*	|e	|*	|*	|a	|*	|*	|o	|*	|*	|*	|*	|*	|r	|o	|a	|*	|a	|ć	|[][S]-	|p	|*	|*	|.
|*	|s	|*	|*	|*	|[18][S]\drarr	|k	|r	|u	|p	|i	|e	|c	|*	|k	|n	|*	|u	|*	|p	|o	|*	|*	|.
|*	|k	|*	|*	|*	|b	|*	|*	|*	|*	|*	|*	|*	|*	|a	|i	|*	|s	|*	|ł	|ł	|*	|*	|.
|*	|a	|[19][S]\drarr	|ś	|w	|i	|ę	|t	|o	|p	|i	|e	|t	|r	|z	|e	|*	|a	|*	|c	|e	|*	|*	|.
|*	|*	|u	|*	|*	|e	|*	|[20][S]\darr	|*	|*	|*	|*	|*	|*	|o	|[][,]{ }	|*	|*	|*	|i	|c	|*	|*	|.
|*	|[21][S]\rarr	|p	|i	|e	|l	|g	|r	|z	|a	|n	|*	|*	|*	|w	|r	|*	|*	|*	|o	|z	|*	|*	|.
|*	|*	|i	|[22][S]\rarr	|k	|a	|t	|e	|c	|h	|i	|z	|m	|*	|a	|e	|*	|*	|*	|w	|n	|*	|*	|.
|*	|*	|t	|*	|*	|j	|*	|d	|[23][S]\darr	|*	|*	|[24][S]\darr	|*	|*	|*	|t	|*	|*	|*	|y	|a	|*	|*	|.
|*	|*	|a	|*	|*	|e	|*	|a	|d	|[25][S]\darr	|*	|k	|*	|*	|*	|o	|*	|*	|*	|*	|*	|*	|*	|.
|*	|[26][S]\darr	|*	|*	|*	|w	|*	|*	|i	|b	|*	|a	|*	|*	|*	|r	|*	|*	|*	|*	|*	|*	|*	|.
|[27][S]\rarr	|c	|z	|o	|p	|*	|*	|*	|o	|u	|*	|p	|*	|*	|*	|y	|*	|*	|*	|*	|*	|*	|*	|.
|*	|y	|*	|*	|*	|*	|*	|*	|n	|s	|*	|ł	|*	|*	|*	|c	|*	|*	|*	|*	|*	|*	|*	|.
|[28][S]\rarr	|k	|o	|ł	|o	|w	|a	|c	|i	|z	|n	|a	|*	|*	|*	|z	|*	|*	|*	|*	|*	|*	|*	|.
|*	|l	|*	|*	|*	|*	|*	|*	|z	|c	|*	|n	|*	|*	|*	|n	|*	|*	|*	|*	|*	|*	|*	|.
|[29][S]\rarr	|a	|h	|i	|s	|t	|o	|r	|y	|z	|m	|*	|*	|*	|*	|e	|*	|*	|*	|*	|*	|*	|*	|.
|*	|m	|*	|*	|*	|*	|*	|*	|*	|a	|*	|*	|*	|*	|*	|*	|*	|*	|*	|*	|*	|*	|*	|.
|*	|a	|*	|*	|*	|*	|*	|*	|*	|n	|*	|*	|*	|*	|*	|*	|*	|*	|*	|*	|*	|*	|*	|.
|[30][S]\rarr	|t	|o	|r	|r	|e	|j	|a	|[][,]{ }	|k	|a	|l	|i	|f	|o	|r	|n	|i	|j	|s	|k	|a	|*	|.
|*	|*	|*	|*	|*	|*	|*	|*	|*	|a	|*	|*	|*	|*	|*	|*	|*	|*	|*	|*	|*	|*	|*	|.
|*	|*	|*	|*	|*	|*	|*	|*	|*	|*	|*	|*	|*	|*	|*	|*	|*	|*	|*	|*	|*	|*	|*	|.\end{Puzzle}

\newpage

\begin{PuzzleClues}{\textbf{Poziome}\\}\Clue{11}{}{fotografik, inżynier elektryk (1880-1965); twórca nowoczesnego portretu w fotografice polskiej}
\Clue{15}{}{grzeczne, kulturalne i delikatne zachowywanie się wobec innych osób, taktowne działanie}
\Clue{17}{}{miasto w Indiach (Bengal Zach.)}
\Clue{18}{}{DZIARNINA; skrystalizowany miód pszczeli}
\Clue{19}{}{średniowieczna danina składana na rzecz Stolicy Apostolskiej przez państwa chrześcijańskie}
\Clue{21}{}{bylina z Madagaskaru; woda gromadzona w pochwach liściowych może służyć podróżnym, tzw. drzewo podróżnych}
\Clue{22}{}{podstawy, podstawowa, bazowa część wiedzy}
\Clue{27}{}{klejek - grzyb z rodziny klejówkowatych}
\Clue{28}{}{KRĘCIEK; CENUROZA}
\Clue{29}{}{sposób postrzegania rzeczywistości, pomijający historyczne uwarunkowania składających się na nią zjawisk}
\Clue{30}{}{Torreya californica - gatunek drzew z rodziny cisowatych pochodzący z zachodniego wybrzeża Stanów Zjednoczonych, głównie ze stanu Kalifornia}\end{PuzzleClues}

\begin{PuzzleClues}{\textbf{Pionowe}\\}\Clue{1}{}{Watsonarctia casta - gatunek motyla z rodziny niedźwiedziówkowatych, o brunatnoróżowym ubarwieniu, lata w czerwcu i maju}
\Clue{2}{}{cecha czegoś takiego jak u człowieka}
\Clue{3}{}{niższy oficer, dowódca centurii, najmniejszej jednostki taktycznej legionu}
\Clue{4}{}{techniczny wynik w matematyce, nazwany imieniem Hansa Juliusa Zassenhausa, dotyczący kraty podgrup danej grupy}
\Clue{5}{}{układ w ciele wyższych kręgowców obsługujący funkcje wydalania i rozmnażania}
\Clue{6}{}{proces wytwarzania, przekształcania i przekazywania informacji pomiędzy jednostkami, grupami i organizacjami społecznymi, mający na celu dynamiczne kształtowanie, modyfikację bądź zmianę wiedzy, postaw i zachowań w kierunku zgodnym z wartościami i interesami oddziałujących na nie podmiotów}
\Clue{7}{}{dyscyplina sportu, której zawody są pokazywane na olimpiadzie, ale nie przyznaje się w jej ramach medali}
\Clue{8}{}{nazwa miejsca, w którym dokonywany jest demontaż skradzionych pojazdów (w celu sprzedania ich w częściach), a także - w którym ukrywa się kradzione dobra czy dezaktywuje urządzenia antywłamaniowe}
\Clue{9}{}{to, że jakiś element lub układ elementów zorganizowanej całości (układu, struktury) powtarza się (występuje wielokrotnie) w strukturze tej całości}
\Clue{10}{}{przedsiębiorstwo, w którym wyrabia (wyrabiało) się coś ręcznie}
\Clue{12}{}{model helikoptera - dronu, produktu kolekcjonerskiego lub zabawki, zaopatrzony w cztery wirniki}
\Clue{13}{}{lekka w przebiegu grypa}
\Clue{14}{}{ornament ciągły w formie półwypukłego rzędu perełek ułożonych na przemian z pałeczkami lub krążkami}
\Clue{15}{}{ptak egzotyczny; poszczególne gatunki tego ptaka klasyfikowane są w taksonomii biologicznej w obrębie rodziny nestorów (Strigopidae)}
\Clue{16}{}{figura stylistyczna i retoryczna; pytanie zadane nie dla uzyskania odpowiedzi, lecz w celu skłonienia odbiorcy do przemyśleń na określony temat}
\Clue{18}{}{kosmonauta radziecki na pokładzie Woschoda 2 w 1965 r}
\Clue{19}{}{Upita (lit. Upyte) - wieś, dawne miasteczko na Litwie w rejonie poniewieskim Litwy, siedziba gminy (starostwa)}
\Clue{20}{}{przystań dla statków}
\Clue{23}{}{malarz i grafik (1866-1924) profesor ASP w Krakowie}
\Clue{24}{}{człowiek, który w danej społeczności czy kulturze pełni funkcje kultowe}
\Clue{25}{}{mieszkanka Buska-Zdroju}
\Clue{26}{}{cyklaminian - organiczny związek chemiaczny stosowany jako substancja słodząca}\end{PuzzleClues}\newpage%\section*{Krzyżówka 47}

\noindent\begin{Puzzle}{20}{25}|*	|*	|*	|*	|*	|*	|*	|*	|*	|*	|*	|*	|*	|*	|*	|*	|*	|[1][S]\drarr	|m	|h	|*	|.
|*	|*	|*	|*	|*	|*	|*	|*	|*	|*	|*	|*	|[2][S]\darr	|*	|*	|*	|[3][S]\rarr	|h	|i	|t	|*	|.
|*	|*	|[4][S]\darr	|[5][S]\drarr	|i	|b	|i	|s	|[][,]{ }	|s	|i	|w	|o	|p	|i	|ó	|r	|y	|*	|*	|*	|.
|*	|*	|p	|a	|[6][S]\darr	|[7][S]\drarr	|p	|r	|o	|s	|t	|o	|s	|k	|r	|z	|y	|d	|ł	|y	|*	|.
|*	|*	|u	|n	|i	|p	|*	|[8][S]\rarr	|k	|r	|y	|p	|t	|o	|p	|s	|*	|r	|*	|[9][S]\darr	|*	|.
|*	|*	|n	|t	|m	|o	|[10][S]\rarr	|e	|l	|e	|k	|t	|r	|y	|k	|a	|*	|o	|*	|r	|*	|.
|*	|*	|k	|y	|a	|w	|*	|*	|*	|*	|*	|[11][S]\darr	|ó	|[12][S]\rarr	|r	|a	|d	|a	|*	|z	|*	|.
|*	|*	|t	|l	|d	|o	|*	|*	|*	|*	|*	|g	|g	|[13][S]\drarr	|h	|l	|a	|k	|*	|e	|*	|.
|*	|*	|[][,]{ }	|i	|ł	|ł	|[14][S]\rarr	|s	|i	|t	|a	|r	|*	|k	|*	|*	|*	|u	|[15][S]\darr	|p	|*	|.
|*	|*	|t	|t	|o	|ż	|*	|*	|*	|[16][S]\rarr	|a	|u	|d	|e	|n	|*	|*	|s	|t	|k	|[17][S]\darr	|.
|*	|*	|y	|e	|[][,]{ }	|e	|[18][S]\drarr	|i	|n	|d	|a	|p	|a	|m	|i	|d	|*	|t	|o	|a	|k	|.
|*	|*	|p	|r	|m	|*	|c	|[19][S]\darr	|*	|*	|*	|a	|[20][S]\darr	|i	|[21][S]\drarr	|s	|z	|y	|p	|*	|o	|.
|*	|*	|o	|a	|a	|*	|l	|c	|[22][S]\darr	|*	|*	|[][,]{ }	|g	|*	|ł	|*	|*	|k	|i	|*	|ś	|.
|*	|*	|g	|t	|s	|[23][S]\darr	|*	|h	|f	|[24][S]\rarr	|s	|k	|r	|z	|y	|d	|ł	|a	|*	|*	|ć	|.
|*	|*	|r	|u	|z	|k	|*	|e	|a	|[25][S]\darr	|*	|o	|a	|[26][S]\darr	|s	|[27][S]\darr	|[28][S]\darr	|*	|[29][S]\darr	|*	|[][,]{ }	|.
|*	|*	|a	|r	|y	|o	|*	|r	|l	|k	|*	|ł	|*	|m	|e	|m	|h	|*	|m	|[30][S]\darr	|o	|.
|*	|*	|f	|a	|n	|n	|[31][S]\drarr	|u	|s	|i	|ł	|o	|w	|a	|n	|i	|e	|*	|o	|b	|g	|.
|*	|*	|i	|*	|o	|s	|p	|b	|y	|e	|*	|w	|*	|k	|k	|n	|b	|*	|z	|i	|o	|.
|*	|*	|c	|[32][S]\darr	|w	|u	|l	|i	|f	|l	|[33][S]\darr	|a	|*	|o	|o	|o	|r	|*	|a	|e	|n	|.
|*	|[34][S]\drarr	|z	|i	|e	|l	|e	|n	|i	|c	|e	|*	|*	|w	|*	|g	|a	|[35][S]\darr	|m	|d	|o	|.
|*	|p	|n	|n	|*	|*	|ś	|*	|k	|z	|m	|*	|*	|i	|*	|o	|j	|h	|b	|o	|w	|.
|*	|a	|y	|k	|[36][S]\darr	|*	|n	|*	|a	|a	|p	|*	|*	|a	|*	|w	|s	|u	|i	|t	|a	|.
|*	|p	|*	|a	|n	|[37][S]\darr	|i	|*	|c	|n	|o	|*	|*	|n	|[38][S]\drarr	|a	|k	|c	|j	|a	|*	|.
|*	|u	|*	|s	|i	|l	|a	|*	|j	|k	|r	|*	|*	|k	|a	|t	|i	|u	|k	|*	|*	|.
|[39][S]\rarr	|a	|z	|o	|t	|e	|k	|*	|a	|a	|a	|*	|*	|a	|s	|e	|*	|ł	|a	|*	|*	|.
|*	|*	|*	|*	|*	|*	|*	|*	|*	|*	|*	|*	|*	|*	|*	|*	|*	|*	|*	|*	|*	|.\end{Puzzle}

\newpage

\begin{PuzzleClues}{\textbf{Poziome}\\}\Clue{1}{}{wielokrotność henra (H) - jednostki indukcyjności oraz permeancji (przewodności magnetycznej) w układzie SI; jest równy 10E-3 H}
\Clue{3}{}{przebojowa piosenka, szlagier, przebój}
\Clue{5}{}{Threskiornis melanocephalus - gatunek ptaka z rodziny ibisowatych (Threskiornithidae)}
\Clue{7}{}{owad z rzędu prostoskrzydłych, rozpowszechnionych głównie w strefie klimatu tropikalnego i subtropikalnego}
\Clue{8}{}{Kryptops - rodzaj prymitywnego teropoda z rodziny abelizaurów; żył w okresie wczesnej kredy na terenach dzisiejszego Nigru}
\Clue{10}{}{instalacja elektryczna}
\Clue{12}{}{posiedzenie rady, zebranie ludzi, którzy czymś rządzą, są za coś odpowiedzialni}
\Clue{13}{}{pękate, albo spłaszczone naczynie z szyjką z gliny lub kamionki}
\Clue{14}{}{tradycyjny indyjsko-perski instrument muzyczny; składa się z długiego gryfu, gruszkowato wypukłego pudła rezonansowego, najczęściej sześciu-ośmiu strun melodycznych, szarpanych metalowym plektronem, oraz kilkunastu strun burdonowych}
\Clue{16}{}{anglo-amerykański poeta (1907-73), liryka o tematyce społeczno-moralnej}
\Clue{18}{}{lek moczopędny z grupy sulfonamidów, hamujący resorpcję zwrotną elektrolitów}
\Clue{21}{}{ryba z rodziny jesiotrowatych o długości do 2 m; Morze Czarne, Azowskie i Kaspijskie}
\Clue{24}{}{przenośnie: opieka, nadzór ze strony kogoś lub czegoś}
\Clue{31}{}{próba popełnienia przestępstwa, zagrożona taką samą karą, jak przestępstwo dokonane (wg kodeksu karnego)}
\Clue{34}{}{Chlorophyta - parafiletyczna grupa jednokomórkowych (o strukturze wiciowcowej, kapsalnej i kokoidalnej) lub wielokomórkowych samożywnych roślin występujących w wodach słodkich i słonych, rzadko w środowisku lądowym}
\Clue{38}{}{atakowanie w czasie rozgrywek sportowych; element gry ofensywnej}
\Clue{39}{}{związek azotu, pochodna amoniaku, w którym wszystkie wiązania azotu z atomami wodoru zostały zastąpione wiązaniami z innymi pierwiastkami}\end{PuzzleClues}

\begin{PuzzleClues}{\textbf{Pionowe}\\}\Clue{1}{}{dział akustyki badający rozchodzenie się fal dźwiękowych w środowiskach wodnych dla celów podwodnej lokalizacji i łączności}
\Clue{2}{}{miejsce obwarowania w kształcie czworokąta z narożnymi basztami}
\Clue{4}{}{podstawowa jednostka długości stosowana w systemach miary wielkości czcionek i innych elementach typograficznych}
\Clue{5}{}{twórczość piśmiennicza, w której zostają odrzucone obowiązujące dotychczas zasady, schematy i prawa w literaturze}
\Clue{6}{}{imadło służące do mocowania obrabianych przedmiotów w maszynie}
\Clue{7}{}{obszar w Federacji Rosyjskiej, wzdłuż Wołgi, od ujścia Karny do Morza Kaspijskiego}
\Clue{9}{}{element kończyny zwierząt}
\Clue{11}{}{grupa multiplikatywna wszystkich liczb zespolonych o wartości bezwzględnej 1, np. okrąg jednostkowy na płaszczyźnie zespolonej}
\Clue{13}{}{miasto i port w Finlandii przy ujściu rzeki Kemi od Zatoki Botnickiej, przemysł drzewny i celulozowo-papierniczy}
\Clue{15}{}{sassebi, Damaliscus lunatus - ssak z rodziny krętorogich; występuje w południowej i środkowej Afryce, zasiedlając sawanny, tereny zalewowe i zarośla}
\Clue{17}{}{końcowy odcinek kręgosłupa człowieka, złożony z trzech do pięciu (zwykle czterech) zrośniętych ze sobą kręgów guzicznych (ogonowych, rzekomych), z których największy jest pierwszy}
\Clue{18}{}{w chemii: symbol chloru}
\Clue{19}{}{anioł z najwyższych chórów niebieskich lub przedstawienie owego anioła w sztuce}
\Clue{20}{}{zagranie na instrumencie muzycznym}
\Clue{21}{}{ukraiński kompozytor (1842-1912); pianista, dyrygent i pedagog}
\Clue{22}{}{rozumowanie logiczne, polegające na wykazywaniu fałszywości hipotezy}
\Clue{23}{}{wysoki urzędnik w starożytnym Rzymie}
\Clue{25}{}{mieszkanka Kielc}
\Clue{26}{}{mieszkanka Makowa Mazowieckiego}
\Clue{27}{}{Petromyzontidae - rodzina prymitywnych, anadromicznych lub słodkowodnych zwierząt wodnych zaliczanych do bezżuchwowców (Agnatha); minogowate występują w mulistych wodach słonych i słodkich na obydwu półkulach; w Polsce jest obecnych 5 gatunków}
\Clue{28}{}{przedmiot szkolny lub uczony w ramach kursu, na którym opanowuje się podstawy języka hebrajskiego}
\Clue{29}{}{mieszkanka Mozambiku, kobieta pochodzenia mozambickiego}
\Clue{30}{}{człowiek nieszczęśliwy, biedak}
\Clue{31}{}{inna nazwa chropiatki proponowana w polskim piśmiennictwie mykologicznym; grzyb mikoryzowy, wytwarzający przeważnie ciemnobrązowe, rozetowate, lejkowate lub rozgałęzione owocniki o prawie gładkim hymenoforze, często pokrytym regularnie rozmieszczonymi brodawkami (gr. thele)}
\Clue{32}{}{żądanie zapłaty}
\Clue{33}{}{element architektoniczny występujący najczęściej w kościele w postaci odrębnej kondygnacji o charakterze antresoli, usytuowanej nad kruchtą lub nadwieszonej nad głównym wejściem do świątyni, przeznaczony do umieszczenia tam organów}
\Clue{34}{}{zatoka w południowych wybrzeży Nowej Gwinei}
\Clue{35}{}{koń huculski - rasa konia domowego, górska, ceniona za żywotność, siłę i odporność, używana niegdyś głównie jako zwierzę juczne; obecnie, ze względu na swą łagodność i inteligencję, używane są często w hipoterapii}
\Clue{36}{}{rodzaj pasmanterii, najczęściej służy do ozdabiania przedmiotów i odzieży ze skóry}
\Clue{37}{}{skrót/symbol funta egipskiego}
\Clue{38}{}{symbol arsenu w chemii}\end{PuzzleClues}\newpage%\section*{Krzyżówka 48}

\noindent\begin{Puzzle}{22}{24}|*	|*	|*	|*	|*	|*	|*	|*	|*	|*	|*	|*	|*	|*	|*	|*	|[1][S]\drarr	|o	|g	|i	|e	|ń	|*	|.
|*	|*	|*	|*	|*	|[2][S]\drarr	|b	|r	|*	|[3][S]\darr	|[4][S]\drarr	|s	|a	|l	|b	|u	|t	|a	|m	|o	|l	|*	|[5][S]\darr	|.
|*	|*	|*	|*	|[6][S]\darr	|p	|[7][S]\darr	|*	|*	|d	|w	|*	|*	|*	|*	|[8][S]\drarr	|r	|u	|c	|h	|*	|*	|s	|.
|*	|*	|*	|[9][S]\drarr	|b	|e	|g	|*	|[10][S]\rarr	|w	|i	|c	|h	|r	|*	|ł	|e	|*	|*	|*	|*	|*	|z	|.
|*	|*	|*	|j	|a	|r	|r	|*	|*	|u	|c	|[11][S]\drarr	|c	|h	|ł	|o	|n	|n	|o	|ś	|ć	|*	|y	|.
|*	|*	|*	|ę	|w	|y	|y	|*	|*	|d	|e	|a	|*	|*	|[12][S]\darr	|p	|*	|[13][S]\darr	|*	|[14][S]\darr	|*	|[15][S]\darr	|k	|.
|*	|*	|*	|z	|ó	|f	|z	|*	|[16][S]\darr	|z	|w	|k	|*	|*	|p	|a	|*	|t	|*	|l	|*	|d	|[][,]{ }	|.
|*	|*	|*	|y	|ł	|e	|i	|[17][S]\drarr	|f	|i	|o	|r	|d	|*	|r	|t	|[18][S]\darr	|e	|[19][S]\darr	|e	|[20][S]\darr	|j	|p	|.
|*	|*	|[21][S]\darr	|c	|*	|r	|p	|p	|e	|e	|j	|e	|[22][S]\drarr	|s	|z	|e	|z	|l	|o	|n	|g	|*	|r	|.
|*	|*	|k	|z	|[23][S]\darr	|i	|i	|t	|n	|s	|e	|c	|s	|[24][S]\darr	|e	|c	|ł	|e	|d	|i	|r	|*	|z	|.
|*	|*	|ą	|e	|s	|u	|ó	|a	|*	|t	|w	|j	|w	|d	|r	|z	|o	|s	|m	|w	|a	|*	|e	|.
|*	|*	|t	|k	|t	|m	|r	|s	|*	|o	|o	|a	|o	|y	|ó	|k	|ż	|k	|o	|o	|n	|*	|s	|.
|*	|*	|[][,]{ }	|[][,]{ }	|r	|*	|e	|z	|*	|g	|d	|*	|r	|r	|b	|a	|e	|o	|w	|ś	|a	|*	|t	|.
|*	|*	|ż	|u	|u	|[25][S]\drarr	|k	|e	|b	|r	|a	|c	|z	|o	|*	|*	|*	|p	|a	|ć	|t	|*	|a	|.
|*	|*	|y	|[][,]{ }	|k	|k	|*	|k	|[26][S]\darr	|o	|*	|*	|e	|*	|*	|*	|*	|*	|*	|*	|*	|[27][S]\darr	|w	|.
|*	|*	|l	|w	|t	|e	|[28][S]\darr	|*	|o	|s	|*	|*	|ń	|*	|*	|*	|*	|*	|[29][S]\darr	|*	|*	|e	|n	|.
|*	|[30][S]\drarr	|n	|a	|u	|r	|a	|ń	|c	|z	|y	|k	|*	|[31][S]\drarr	|p	|o	|ł	|y	|s	|k	|*	|k	|y	|.
|*	|w	|y	|g	|r	|k	|l	|*	|e	|ó	|[32][S]\darr	|*	|*	|z	|[33][S]\rarr	|b	|a	|n	|i	|j	|a	|s	|*	|.
|*	|u	|*	|i	|a	|r	|g	|*	|l	|w	|g	|[34][S]\darr	|*	|a	|*	|[35][S]\rarr	|k	|ó	|ł	|k	|o	|*	|*	|.
|*	|*	|*	|*	|*	|a	|a	|*	|*	|k	|e	|k	|[36][S]\rarr	|m	|a	|k	|*	|[37][S]\rarr	|a	|l	|t	|*	|*	|.
|*	|[38][S]\rarr	|n	|e	|r	|d	|*	|[39][S]\rarr	|f	|a	|m	|o	|t	|y	|d	|y	|n	|a	|*	|*	|*	|*	|*	|.
|*	|[40][S]\drarr	|g	|a	|m	|e	|l	|a	|n	|*	|l	|r	|[41][S]\rarr	|s	|t	|u	|d	|n	|i	|a	|*	|*	|*	|.
|[42][S]\rarr	|d	|o	|l	|e	|*	|*	|*	|*	|*	|i	|a	|[43][S]\rarr	|ł	|a	|p	|a	|c	|z	|*	|*	|*	|*	|.
|*	|*	|*	|[44][S]\rarr	|b	|l	|o	|c	|z	|e	|k	|*	|*	|*	|*	|*	|*	|*	|*	|*	|*	|*	|*	|.
|*	|*	|*	|*	|*	|*	|*	|*	|*	|*	|*	|*	|*	|*	|*	|*	|*	|*	|*	|*	|*	|*	|*	|.\end{Puzzle}

\newpage

\begin{PuzzleClues}{\textbf{Poziome}\\}\Clue{1}{}{przenośnie: niesamowita energia}
\Clue{2}{}{skrót/symbol waluty birr}
\Clue{4}{}{organiczny związek chemiczny, wybiórczy agonista receptorów ß2 w mięśniach gładkich oskrzeli}
\Clue{8}{}{zespół czynności, działań, akcji podejmowanych w jakimś celu (taki ruch to nigdy nie jest grupa ludzi, zawsze tylko czynności)}
\Clue{9}{}{BEJ}
\Clue{10}{}{poetycko o wichrze - silnym, gwałtownym wietrze}
\Clue{11}{}{sprawność umysłu; błyskawiczne przyswajanie sobie czegoś, np. wiedzy}
\Clue{17}{}{FIORDYNG: bardzo silny, norweski koń zaprzęgowy}
\Clue{22}{}{kanapa o wydłużonym siedzeniu z jednym lub dwoma oparciami na krótszych bokach}
\Clue{25}{}{określenie drzewa o wyjątkowo twardym drewnie, rosnącego w Ameryce Południowej}
\Clue{30}{}{mieszkaniec Nauru, człowiek pochodzenia naurańskiego}
\Clue{31}{}{efekt odbijania światła przez jakąś powierzchnię; wrażenie optyczne, które powstaje dzięki odbiciu i rozproszeniu światła na powierzchni ciał stałych i cieczy lub bezpośrednio nad ich powierzchnią}
\Clue{33}{}{miasto w zach. Syrii, nad Morzem Śródziemnym, port naftowy, rurociąg z pól naftowych w Iraku}
\Clue{35}{}{małe koło; drobna, okrągła rzecz, np. obrączka, ringo (a także znak, plama)}
\Clue{36}{}{jadalne nasiona rośliny o tej samej nazwie}
\Clue{37}{}{malarz austriacki (1812-1905), malował akwarelowe pejzaże i sceny rodzajowe, przedstawiciel szkoły pejzażowej}
\Clue{38}{}{pogardliwie: zwykle młoda osoba nieprzystosowana do życia w swojej grupie młodzieżowej, introwertyczna, zajmująca się tylko swoimi pasjami (najczęściej komputerami)}
\Clue{39}{}{organiczny związek chemiczny, lek z grupy blokerów H2, hamujący wydzielanie kwasu solnego przez komórki okładzinowe żołądka; stosowany głównie w terapii choroby wrzodowej żołądka i dwunastnicy}
\Clue{40}{}{tradycyjna muzyka grana na Jawie i wyspie Bali}
\Clue{41}{}{pionowy szyb sięgający do wodonośnych warstw gruntu i służący do wydobywania wody}
\Clue{42}{}{miasto we Francji (Burgundia) nad rzeką Doubs}
\Clue{43}{}{agent śledczy, którego zadaniem jest łapanie przestępców}
\Clue{44}{}{mały lub określany zdrobniale blok - foremna, kształtna, regularna bryła czegoś}\end{PuzzleClues}

\begin{PuzzleClues}{\textbf{Pionowe}\\}\Clue{1}{}{utwór liryczny o charakterze żałobnym, wywodzący się ze starogreckiej poezji funeralnej; tren jest poświęcony zmarłej osobie i wyraża żal oraz smutek z powodu jej odejścia, a także przedstawia jej cnoty i zasługi}
\Clue{2}{}{zewnętrzne urządzenie elektroniczne podłączane do komputera, np. mysz, drukarka}
\Clue{3}{}{moneta lub banknot (dawniej) o wartości 20 gr}
\Clue{4}{}{zastępca wojewody - zwierzchnika zespolonej administracji rządowej na terenie województwa}
\Clue{5}{}{ułożenie wyrazów w kolejności innej niż tej, której wymagają ich zależności składniowe}
\Clue{6}{}{człowiek ospały, niemrawy, ociężały, a przy tym najczęściej niezbyt bystry}
\Clue{7}{}{ironiczne określenie osoby zawodowo zajmującej się pisaniem, np. dziennikarza, pisarza}
\Clue{8}{}{zdrobniale o łopateczce - narzędziu kuchennym podobnym do łyżki o płaskiej szerokiej końcówce}
\Clue{9}{}{w wagach starego typu (nieelektronicznych) - drobny element, wypustka, która wskazuje na przechylenie się wagi na jedną lub drugą szalę }
\Clue{11}{}{przyrost obszaru państwa spowodowany działaniem siły Coriolisa; pojęcie prawnicze}
\Clue{12}{}{produkt z przerobionego surowca, np. przerób z mleka}
\Clue{13}{}{przyrząd astronomiczny do obserwacji ciał niebieskich o konstrukcji lunety, lub zaopatrzony w zwierciadło wklęsłe}
\Clue{14}{}{cecha działania lub zachowania - to, że coś jest powolne, niedynamiczne}
\Clue{15}{}{didżej}
\Clue{16}{}{najdrobniejsza jednostka zdawkowa w Chińskiej Republice Ludowej; 1/100 yuana, 1/10 jiao}
\Clue{17}{}{znaczek podobny do litery V, stawiany przy wybieranej odpowiedzi lub po wykonaniu czynności opisanej na kartce}
\Clue{18}{}{skupienie minerałów, skał lub ropy naftowej, powstałe w wyniku różnorodnych procesów geologicznych}
\Clue{19}{}{brak zgody na coś}
\Clue{20}{}{pestkowaty słodko-cierpki w smaku owoc (jagoda) granatowca}
\Clue{21}{}{kąt, który powstaje z połączenia żyły podobojczykowej i żyły szyjnej wewnętrznej, które tworzą razem żyłę ramienno-głowową}
\Clue{22}{}{element konstrukcyjny w postaci wałka lub stożka służący do śrubowego łączenia elementów}
\Clue{23}{}{budowa, np. jakiegoś przedmiotu}
\Clue{24}{}{dyrektor szkoły}
\Clue{25}{}{miasto w Holandii, przy granicy z Niemcami, ważny ośrodek wydobycia węgla kamiennego}
\Clue{26}{}{stalowe narzędzie przydatne do rozniecania ognia}
\Clue{27}{}{były chłopak, partner, mąż; rzeczownik rodzaju męskiego, nieodmienny}
\Clue{28}{}{GLON najprostsza, samożywna roślina plechowa występująca w wodach słodkich i morskich}
\Clue{29}{}{zdolności bojowe wojska}
\Clue{30}{}{jeden z głównych języków chińskich, używany powszechnie w Chinach}
\Clue{31}{}{idea, myśl, pomysł, zarys planu na coś}
\Clue{32}{}{zatoka Morza Marmala u wybrzeży Turcji (Przylądek Kalem)}
\Clue{34}{}{lekka, najczęściej bawełniana tkanina z nietrwałymi wytłaczanymi wzorami}
\Clue{40}{}{cyfra rzymska odpowiadająca 500}\end{PuzzleClues}\newpage%\section*{Krzyżówka 49}

\noindent\begin{Puzzle}{25}{22}|*	|*	|*	|*	|*	|*	|*	|*	|*	|*	|*	|*	|*	|*	|[1][S]\darr	|*	|*	|*	|*	|[2][S]\drarr	|m	|u	|z	|a	|*	|*	|.
|*	|*	|*	|[3][S]\rarr	|w	|i	|e	|r	|t	|ł	|o	|[][,]{ }	|r	|u	|r	|o	|w	|e	|*	|t	|[4][S]\darr	|*	|*	|*	|*	|*	|.
|*	|*	|*	|*	|*	|*	|*	|*	|*	|*	|*	|*	|*	|[5][S]\drarr	|s	|a	|n	|d	|*	|a	|k	|*	|*	|*	|*	|*	|.
|*	|*	|*	|*	|*	|*	|*	|*	|*	|*	|[6][S]\darr	|[7][S]\rarr	|k	|o	|d	|o	|w	|a	|n	|i	|e	|*	|*	|*	|*	|*	|.
|*	|[8][S]\darr	|[9][S]\darr	|[10][S]\rarr	|r	|o	|z	|w	|a	|ł	|k	|a	|*	|p	|*	|*	|*	|[11][S]\darr	|[12][S]\darr	|y	|r	|[13][S]\darr	|[14][S]\darr	|*	|*	|*	|.
|*	|p	|m	|[15][S]\rarr	|w	|a	|t	|t	|e	|a	|u	|*	|*	|u	|[16][S]\darr	|*	|*	|b	|s	|u	|a	|b	|f	|[17][S]\darr	|*	|*	|.
|*	|r	|a	|[18][S]\darr	|*	|[19][S]\darr	|[20][S]\darr	|*	|*	|*	|r	|*	|*	|s	|l	|[21][S]\drarr	|m	|e	|t	|a	|l	|o	|i	|d	|*	|*	|.
|*	|z	|ł	|p	|*	|l	|k	|*	|[22][S]\darr	|*	|t	|*	|*	|z	|i	|a	|*	|s	|a	|n	|c	|ć	|l	|w	|*	|*	|.
|*	|e	|a	|r	|[23][S]\darr	|i	|a	|*	|z	|*	|y	|*	|*	|c	|s	|u	|*	|s	|t	|*	|z	|w	|e	|ó	|*	|*	|.
|*	|c	|[][,]{ }	|y	|ś	|m	|t	|*	|a	|*	|z	|*	|*	|z	|o	|r	|*	|e	|y	|*	|y	|i	|m	|j	|*	|*	|.
|*	|i	|w	|m	|w	|o	|a	|*	|b	|*	|a	|*	|[24][S]\darr	|e	|w	|o	|*	|l	|k	|*	|k	|n	|o	|k	|*	|*	|.
|*	|w	|i	|i	|i	|n	|s	|*	|ó	|*	|c	|*	|a	|n	|c	|r	|*	|*	|*	|*	|*	|a	|n	|a	|*	|*	|.
|[25][S]\drarr	|d	|e	|t	|e	|k	|t	|o	|r	|[][,]{ }	|j	|o	|n	|i	|z	|a	|c	|y	|j	|n	|y	|*	|[][,]{ }	|*	|*	|*	|.
|g	|o	|ś	|y	|t	|a	|r	|*	|*	|*	|a	|*	|a	|e	|y	|*	|*	|*	|[26][S]\rarr	|k	|l	|u	|b	|*	|*	|*	|.
|i	|w	|*	|w	|l	|[][,]{ }	|o	|*	|*	|*	|*	|[27][S]\darr	|d	|*	|k	|*	|[28][S]\rarr	|t	|o	|l	|n	|a	|i	|*	|*	|*	|.
|g	|ó	|*	|n	|i	|k	|f	|*	|[29][S]\darr	|*	|*	|o	|y	|*	|*	|[30][S]\rarr	|j	|e	|r	|e	|m	|i	|a	|k	|*	|*	|.
|a	|d	|*	|o	|k	|a	|i	|[31][S]\rarr	|b	|o	|m	|b	|r	|a	|m	|s	|t	|e	|n	|g	|a	|*	|ł	|*	|*	|*	|.
|n	|*	|*	|ś	|i	|f	|z	|*	|l	|*	|*	|ł	|s	|*	|[32][S]\rarr	|w	|ł	|a	|ś	|c	|i	|w	|o	|ś	|ć	|*	|.
|t	|*	|*	|ć	|*	|f	|m	|*	|u	|*	|*	|o	|k	|*	|*	|*	|[33][S]\rarr	|f	|a	|j	|a	|n	|s	|*	|*	|*	|.
|*	|*	|*	|*	|*	|i	|*	|[34][S]\rarr	|e	|k	|s	|k	|a	|w	|a	|t	|o	|r	|*	|*	|*	|*	|z	|*	|*	|*	|.
|*	|[35][S]\rarr	|a	|f	|e	|r	|z	|y	|s	|t	|a	|*	|*	|*	|[36][S]\rarr	|p	|i	|s	|a	|r	|c	|z	|y	|k	|*	|*	|.
|*	|*	|*	|*	|*	|*	|*	|*	|*	|*	|*	|*	|*	|*	|*	|*	|*	|[37][S]\rarr	|s	|s	|a	|n	|i	|e	|*	|*	|.
|[38][S]\rarr	|s	|k	|l	|e	|p	|i	|e	|n	|i	|e	|[][,]{ }	|k	|l	|a	|s	|z	|t	|o	|r	|n	|e	|*	|*	|*	|*	|.\end{Puzzle}

\newpage

\begin{PuzzleClues}{\textbf{Poziome}\\}\Clue{2}{}{muzyka śpiewana, grana lub tylko odsłuchiwana przez kogoś}
\Clue{3}{}{wiertło w postaci rury, do wiercenia otworów o dużej średnicy}
\Clue{5}{}{właściwie Dudevant - pisarka francuska (1804-76), przyjaciółka Chopina -powieści, dramaty, szkice literackie}
\Clue{7}{}{sposób przekształcenia tekstu na formę cyfrową}
\Clue{10}{}{rozstrzelanie; egzekucja wykonana za pomocą broni palnej}
\Clue{15}{}{malarz holenderski. (1632-75) kolorysta i luminista; sceny rodzajowe, studia portretowe, pejzaże}
\Clue{21}{}{półmetal - pierwiastek chemiczny o właściwościach zarówno metali, jak i niemetali}
\Clue{25}{}{detektor stosowany do oznaczeń stężenia składników mieszanin gazowych, którego działanie polega na pomiarach (z zastosowaniem elektrometru) natężenia prądu płynącego przez gaz w komorze jonizacyjnej}
\Clue{26}{}{lokal rozrywkowy, w którym się tańczy, organizuje koncerty, słucha muzyki, spotyka ze znajomymi}
\Clue{28}{}{(1837-1902), pisarz węgierski, poezje z elementami folkloru, powieści historyczne}
\Clue{30}{}{wierzchnie okrycie rosyjskich chłopów}
\Clue{31}{}{trzecie, licząc od dołu przedłużenie masztu}
\Clue{32}{}{coś, co jest typowe dla kogoś lub czegoś}
\Clue{33}{}{tworzywo do wyrobu ceramiki}
\Clue{34}{}{narzędzie dentystyczne do usuwania tkanki zębowej}
\Clue{35}{}{przestępca zamieszany w jakąś aferę}
\Clue{36}{}{pomocnik pisarza w biurze, urzędzie, sądzie}
\Clue{37}{}{picie płynu z pojemnika poprzez wyciąganie go odpowiednio ułożonymi i poruszanymi ustami}
\Clue{38}{}{sklepienie zbudowane na planie kwadratu z dwóch przenikających się sklepień kolebkowych, z których pozostawiono boczne części sklepień}\end{PuzzleClues}

\begin{PuzzleClues}{\textbf{Pionowe}\\}\Clue{1}{}{kod ISO 4217 dinara serbskiego}
\Clue{2}{}{miasto w Chinach, ośrodek administracyjny prowincji Shanxi; wydobycie węgla kamiennego, hutnictwo żelaza i stali}
\Clue{4}{}{przedstawiciel grupy etnicznej Keralczyków, którzy mieszkają w stanie Kerala w południowo-zachodnich Indiach i posługują się  drawidyjskim językiem malajalam}
\Clue{5}{}{obniżenie ceny}
\Clue{6}{}{przycięcie ogona psu, koniowi albo innemu zwierzęciu}
\Clue{8}{}{w terminologii prawniczej to dowód, który uzasadnia twierdzenie przeciwne twierdzeniu opartemu na innym dowodzie}
\Clue{9}{}{wieś w Polsce położona w województwie mazowieckim, w powiecie grójeckim, w gminie Belsk Duży}
\Clue{11}{}{ur. w 1784r. astronom niemiecki; wyznaczył masy planet i orbitę komety Halleya}
\Clue{12}{}{specjalista w dziedzinie statyki, naukowiec zajmujący się statyką}
\Clue{13}{}{burak liściowy, kapusta rzymska, mangold - warzywo, jeden z kultywarów buraka zwyczajnego; roślina uprawiana zwłaszcza w krajach śródziemnomorskich dla liści jako jednoroczna}
\Clue{14}{}{Philemon citreogularis - gatunek ptaka z rodziny miodojadów (Meliphagidae), który występuje w Australii, Nowej Gwinei, wschodniej Indonezji i Nowej Kaledonii, żywi się nektarem, owadami i innymi bezkręgowcami, kwiatami, owocami i nasionami}
\Clue{16}{}{żołnierz nieregularnej kawalerii polskiej zorganizowanej na początku XVII w}
\Clue{17}{}{dwie osoby, para}
\Clue{18}{}{to, że ktoś jest prymitywny, prostacki; cecha kogoś prymitywnego}
\Clue{19}{}{owoc papedy, mały zielony cytrus o pomarszczonej skórce, która jest niezwykle aromatyczna i znajduje zastosowanie w kuchni Tajlandii, Indonezji i Laosu jako składnik pasty curry}
\Clue{20}{}{teoria z dziedziny geologii i biologii historycznej, według której w historii Ziemi okresy spokojnego funkcjonowania ekosystemów i procesów geologicznych (erozji, akumulacji) rozdzielały gwałtowne przemiany geologiczne, które miały dramatyczny wpływ na biosferę}
\Clue{21}{}{krążownik rosyjski, którego wystrzał był sygnałem szturmu na Pałac Zimowy (25.10.1917 r.)}
\Clue{22}{}{rządy zaborcy sprawowane nad podległym państwem}
\Clue{23}{}{grupa literacko-muzyczna, zespół muzyki alternatywnej wykonujący przede wszystkim piosenki napisane przez lidera Marcina Świetlickiego}
\Clue{24}{}{zatoka w płn-zach części Morza Beringa, u wybrzeży Płw. Czukockiego}
\Clue{25}{}{bardzo duży okaz zwierzęcia; zwierzę, które ma duże rozmiary}
\Clue{27}{}{obserwowane w atmosferze skupiska kondensatów substancji występującej w postaci pary}
\Clue{29}{}{ludowa pieśń Murzynów północnoamerykańskich wyrażająca nastrój tęsknoty i melancholii: podstawowa forma jazzu}\end{PuzzleClues}\newpage%\section*{Krzyżówka 51}

\noindent\begin{Puzzle}{20}{22}|*	|*	|*	|[1][S]\darr	|*	|[2][S]\drarr	|w	|o	|o	|l	|f	|*	|*	|*	|*	|[3][S]\drarr	|t	|r	|u	|p	|*	|.
|*	|*	|*	|b	|[4][S]\drarr	|s	|t	|e	|m	|p	|e	|l	|*	|*	|*	|h	|*	|*	|[5][S]\darr	|*	|*	|.
|*	|[6][S]\drarr	|m	|a	|t	|o	|ł	|*	|[7][S]\drarr	|j	|a	|r	|z	|ą	|b	|e	|k	|*	|t	|*	|[8][S]\darr	|.
|*	|c	|*	|*	|e	|p	|[9][S]\darr	|*	|d	|*	|*	|*	|[10][S]\darr	|[11][S]\rarr	|s	|t	|a	|ł	|a	|*	|b	|.
|*	|i	|*	|*	|r	|l	|p	|*	|ż	|[12][S]\rarr	|o	|d	|s	|e	|t	|e	|k	|*	|r	|[13][S]\darr	|a	|.
|[14][S]\drarr	|e	|f	|e	|m	|e	|r	|y	|d	|y	|*	|*	|k	|*	|*	|r	|*	|[15][S]\darr	|p	|o	|t	|.
|a	|s	|*	|[16][S]\darr	|a	|n	|e	|[17][S]\darr	|ż	|*	|[18][S]\drarr	|z	|ł	|o	|t	|o	|w	|ł	|o	|s	|*	|.
|l	|z	|[19][S]\darr	|k	|*	|i	|[][S]-	|z	|o	|[20][S]\darr	|g	|*	|a	|[21][S]\darr	|*	|d	|*	|a	|n	|t	|[22][S]\darr	|.
|b	|y	|p	|i	|*	|e	|p	|o	|w	|d	|e	|*	|d	|z	|*	|y	|[23][S]\darr	|t	|*	|e	|s	|.
|a	|n	|t	|k	|[24][S]\darr	|c	|a	|r	|n	|i	|r	|*	|*	|a	|[25][S]\darr	|n	|n	|a	|*	|o	|z	|.
|*	|i	|e	|u	|k	|*	|i	|b	|i	|r	|r	|*	|*	|s	|s	|a	|a	|*	|*	|l	|a	|.
|[26][S]\drarr	|a	|r	|t	|u	|r	|d	|a	|k	|t	|y	|l	|*	|t	|m	|*	|r	|*	|*	|i	|b	|.
|s	|n	|o	|n	|p	|*	|*	|*	|*	|*	|m	|*	|*	|ę	|o	|*	|y	|*	|*	|z	|l	|.
|u	|i	|d	|i	|a	|*	|*	|*	|[27][S]\rarr	|k	|a	|n	|a	|p	|k	|a	|*	|*	|*	|a	|o	|.
|r	|n	|a	|c	|*	|[28][S]\rarr	|m	|a	|l	|u	|n	|e	|k	|*	|*	|*	|*	|*	|*	|*	|n	|.
|o	|*	|k	|e	|[29][S]\drarr	|k	|r	|o	|k	|o	|d	|y	|l	|[][,]{ }	|n	|i	|l	|o	|w	|y	|*	|.
|w	|*	|t	|*	|w	|*	|[30][S]\rarr	|p	|r	|z	|e	|n	|i	|k	|l	|i	|w	|o	|ś	|ć	|*	|.
|i	|*	|y	|[31][S]\rarr	|ł	|ę	|g	|*	|*	|[32][S]\rarr	|r	|o	|z	|r	|a	|b	|i	|a	|k	|a	|*	|.
|a	|*	|l	|*	|o	|*	|*	|*	|[33][S]\rarr	|l	|i	|p	|i	|k	|a	|n	|*	|*	|*	|*	|*	|.
|k	|*	|e	|[34][S]\rarr	|c	|z	|e	|c	|z	|e	|n	|i	|e	|c	|*	|*	|*	|*	|*	|*	|*	|.
|*	|*	|*	|*	|h	|*	|*	|*	|[35][S]\rarr	|e	|g	|z	|e	|k	|u	|c	|j	|a	|*	|*	|*	|.
|*	|*	|*	|*	|y	|[36][S]\rarr	|e	|t	|a	|t	|*	|*	|*	|*	|*	|*	|*	|*	|*	|*	|*	|.
|*	|*	|*	|*	|*	|*	|*	|*	|*	|*	|*	|*	|*	|*	|*	|*	|*	|*	|*	|*	|*	|.\end{Puzzle}

\newpage

\begin{PuzzleClues}{\textbf{Poziome}\\}\Clue{2}{}{pisarka angielska (1882-1941), powieści psychologiczne; „Do latami morskiej”, „Pochyła wieża”}
\Clue{3}{}{całkiem wyczerpane, pozbawione energii w baterii urządzenie}
\Clue{4}{}{stojak drewniany będący elementami rusztowania}
\Clue{6}{}{ciężko upośledzona intelektualnie osoba, chora z powodu niedoboru jodu (dawniej określanego jako kretynizm lub matołectwo)}
\Clue{7}{}{ptak leśny z rzędu kuraków o rdzawo-szarym upierzeniu w białe plamy, roślinożerny, łowny}
\Clue{11}{}{w informatyce - fragment kodu źródłowego, który nie może się zmienić}
\Clue{12}{}{procent, nieokreślona część}
\Clue{14}{}{dane, najczęściej tabelaryczne, dotyczące przebiegu przyszłego zjawiska astronomicznego, np. pozorne położenie Słońca, Księżyca i planet na niebie w określonym czasie i w określonym miejscu na Ziemi}
\Clue{18}{}{ASFODEL}
\Clue{26}{}{Arthurdactylus conandoylei - morski pterozaur z rodziny ornitocheirów (Ornithocheiridae); żył w okresie wczesnej kredy (ok. 116 mln lat temu) na terenach Ameryki Południowej}
\Clue{27}{}{zdrobniale: kanapa - tapicerowany mebel przeznaczony do leżenia lub siedzenia, na którym może się zmieścić kilka osób}
\Clue{28}{}{obraz, który nie został namalowany przez wybitnego artystę, lecz przez dziecko lub artystę ludowego}
\Clue{29}{}{Crocodylus niloticus - jeden z trzech gatunków krokodyli występujących w Afryce i drugi pod względem wielkości, zaraz po krokodylu różańcowym, w rodzinie krokodylowatych; występuje prawie w całej Afryce oprócz chłodnych obszarów Afryki północnej oraz pustynnych obszarów Sahary}
\Clue{30}{}{dociekliwość, wnikliwość, uwzględnianie istotnych aspektów lub niuansów zjawiska czy problemu, np. przenikliwość recenzji, wywodu}
\Clue{31}{}{wieś w Polsce położona w województwie śląskim, w powiecie częstochowskim, w gminie Kruszyna}
\Clue{32}{}{dziecko, które rozrabia - psoci}
\Clue{33}{}{szlachetny koń białej maści}
\Clue{34}{}{mieszkaniec Czeczenii, człowiek pochodzenia czeczeńskiego}
\Clue{35}{}{realizacja czegoś (np. egzekucja praw)}
\Clue{36}{}{budżet, przewidywane kwoty dochodów i wydatków na dany okres}\end{PuzzleClues}

\begin{PuzzleClues}{\textbf{Pionowe}\\}\Clue{1}{}{w chemii: symbol baru}
\Clue{2}{}{Uvularia - rodzaj roślin z rodziny zimowitowatych}
\Clue{3}{}{stabilny i zazwyczaj o regulowanej częstotliwości generator drgań elektrycznych stosowany do modulacji (demodulacji) lub zmiany częstotliwości drgań w procesach tzw. heterodynowania (dudnienia/mieszania) elektrycznych przebiegów sinusoidalnych o nieznacznie różnych częstotliwościach}
\Clue{4}{}{gazowy lub elektryczny grzejnik wody}
\Clue{5}{}{Tarpon atlantycki, Megalops atlanticus - gatunek elopsokształtnej ryby amfidromicznej z rodziny tarponowatych (Megalopidae)}
\Clue{6}{}{mieszkaniec Cieszyna}
\Clue{7}{}{SIEWECZKA}
\Clue{8}{}{hebrajska miara objętości płynów ze starożytności, odpowiadająca ok. 21 lub 40 litrom}
\Clue{9}{}{system sprzedaży, w którym kupujący rozlicza się z góry za konkretną ilość toarów lub usług, z których korzysta w późniejszym terminie}
\Clue{10}{}{w rolnictwie: podłużny fragment pola o szerokości sięgającej 30 metrów}
\Clue{13}{}{rozpuszczanie i resorpcja kości występująca w przebiegu wielu chorób}
\Clue{14}{}{długa, biała szata liturgiczna, noszona przez wszystkich duchownych i usługujących wszystkich stopni w czasie liturgii obrządku łacińskiego oraz przez duchownych wielu kościołów protestanckich}
\Clue{15}{}{kawałek naszytego materiału, skóry, który zasłania zniszczone, dziurawe miejsce lub pełni funkcję ozdobną}
\Clue{16}{}{Pycnogonida - gromada niewielkich morskich stawonogów, zwanych z uwagi na wygląd popularniepająkami morskimi; w przeciwieństwie do pozostałych szczękoczułkowców, kikutnice nie zawsze mają 4 pary nóg krocznych; liczba odnóży waha się od 8 do 12 zależnie od gatunku}
\Clue{17}{}{muzyka o charakterystycznym rytmie, do której można tańczyć zorbę}
\Clue{18}{}{nazwa określająca manipulacje dokonywane przy wytyczaniu granic okręgów wyborczych w celu uzyskania korzystnego wyniku przez partię mającą wpływ na kształtowanie ordynacji wyborczej}
\Clue{19}{}{Pterodactyloidea - jeden z dwu podrzędów pterozaurów (Pterosauria); żyły od późnej jury do końca kredy}
\Clue{20}{}{dyscyplina rowerowa, która polega na wykonywaniu skoków i różnych ewolucji w powietrzu na specjalnie przygotowanym torze z zastosowaniem rowerów BMX lub MTB}
\Clue{21}{}{najmniejsza grupa harcerzy}
\Clue{22}{}{wzór, forma, która służy do seryjnego wykonania różnego typu przedmiotów}
\Clue{23}{}{prycza - prymitywne łóżko zbudowane najczęściej z żerdzi lub desek}
\Clue{24}{}{odchody stałe, kał}
\Clue{25}{}{małe dziecko, zazwyczaj chłopiec, które jest aktywne, dużo rozrabia, często też ma duży apetyt}
\Clue{26}{}{w archeologii: rodzaj kamienia, skały twardej podlegający obróbce}
\Clue{29}{}{państwo położone w Europie Południowej, na Półwyspie Apenińskim}\end{PuzzleClues}\newpage%\section*{Krzyżówka 53}

\noindent\begin{Puzzle}{21}{32}|*	|[1][S]\darr	|*	|*	|[2][S]\darr	|*	|*	|[3][S]\darr	|*	|*	|*	|*	|*	|*	|*	|*	|*	|*	|*	|*	|[4][S]\darr	|*	|.
|*	|g	|[5][S]\rarr	|o	|g	|l	|ę	|d	|n	|o	|ś	|ć	|*	|*	|*	|[6][S]\darr	|[7][S]\rarr	|s	|l	|o	|t	|*	|.
|*	|ó	|*	|*	|r	|[8][S]\rarr	|b	|i	|e	|g	|u	|s	|*	|*	|*	|p	|*	|[9][S]\rarr	|t	|o	|r	|*	|.
|*	|r	|[10][S]\rarr	|b	|a	|r	|r	|e	|s	|*	|*	|[11][S]\drarr	|u	|r	|b	|a	|s	|k	|i	|*	|ó	|[12][S]\darr	|.
|*	|a	|*	|*	|n	|*	|[13][S]\darr	|r	|[14][S]\rarr	|r	|a	|k	|o	|w	|a	|t	|o	|ś	|ć	|*	|j	|s	|.
|*	|l	|*	|*	|k	|[15][S]\darr	|l	|a	|*	|*	|[16][S]\darr	|r	|[17][S]\drarr	|c	|h	|e	|m	|e	|x	|*	|l	|z	|.
|*	|e	|[18][S]\drarr	|f	|u	|g	|u	|*	|*	|*	|w	|ę	|t	|*	|*	|n	|*	|*	|*	|*	|i	|k	|.
|*	|k	|k	|[19][S]\darr	|l	|r	|r	|*	|*	|[20][S]\darr	|a	|p	|r	|*	|[21][S]\drarr	|t	|i	|r	|e	|*	|s	|a	|.
|*	|[][,]{ }	|o	|m	|k	|i	|*	|*	|*	|k	|n	|a	|a	|*	|w	|[][,]{ }	|*	|*	|*	|*	|t	|r	|.
|*	|a	|l	|o	|a	|n	|*	|*	|*	|o	|g	|c	|k	|[22][S]\darr	|u	|e	|*	|*	|*	|*	|[][,]{ }	|a	|.
|*	|b	|i	|t	|*	|d	|[23][S]\darr	|*	|*	|n	|a	|z	|t	|z	|j	|u	|*	|*	|*	|*	|o	|d	|.
|*	|i	|m	|o	|[24][S]\darr	|w	|d	|*	|*	|e	|[][,]{ }	|k	|i	|a	|e	|r	|*	|*	|*	|*	|d	|n	|.
|[25][S]\rarr	|s	|a	|r	|n	|a	|*	|*	|*	|s	|m	|i	|e	|p	|c	|o	|*	|*	|*	|*	|g	|o	|.
|*	|y	|c	|*	|y	|l	|*	|[26][S]\darr	|*	|e	|a	|*	|r	|r	|z	|p	|*	|*	|*	|*	|i	|ś	|.
|*	|ń	|j	|[27][S]\darr	|l	|[][,]{ }	|*	|b	|*	|r	|s	|*	|n	|z	|n	|e	|*	|*	|[28][S]\darr	|*	|ę	|ć	|.
|*	|s	|a	|z	|o	|k	|*	|a	|*	|s	|k	|[29][S]\darr	|i	|e	|y	|j	|[30][S]\drarr	|f	|e	|s	|t	|*	|.
|*	|k	|*	|b	|n	|r	|*	|r	|[31][S]\drarr	|t	|o	|j	|a	|d	|[][,]{ }	|s	|m	|u	|k	|ł	|y	|*	|.
|*	|i	|*	|ó	|*	|ó	|*	|a	|m	|w	|w	|a	|*	|a	|b	|k	|i	|*	|s	|*	|*	|*	|.
|*	|*	|*	|j	|*	|t	|*	|n	|e	|o	|a	|ł	|*	|n	|r	|i	|s	|[32][S]\darr	|p	|*	|[33][S]\darr	|[34][S]\darr	|.
|*	|*	|[35][S]\darr	|n	|[36][S]\darr	|k	|*	|e	|d	|*	|*	|o	|*	|i	|a	|*	|a	|c	|u	|[37][S]\darr	|t	|f	|.
|*	|*	|k	|i	|r	|o	|*	|c	|e	|*	|*	|w	|[38][S]\darr	|e	|t	|*	|*	|h	|l	|c	|r	|l	|.
|*	|*	|o	|k	|u	|p	|*	|z	|l	|*	|*	|n	|s	|c	|*	|*	|*	|u	|s	|z	|z	|e	|.
|*	|*	|r	|o	|n	|ł	|*	|e	|l	|*	|*	|i	|p	|*	|*	|*	|[39][S]\darr	|j	|j	|a	|e	|k	|.
|*	|*	|y	|w	|w	|e	|[40][S]\rarr	|k	|i	|e	|ł	|k	|o	|w	|n	|i	|k	|*	|a	|s	|p	|s	|.
|*	|*	|t	|a	|a	|t	|*	|*	|n	|*	|*	|*	|r	|*	|*	|*	|*	|*	|*	|o	|o	|j	|.
|[41][S]\rarr	|m	|o	|t	|y	|w	|i	|k	|*	|[42][S]\rarr	|m	|e	|t	|a	|n	|i	|t	|*	|*	|w	|t	|a	|.
|*	|*	|*	|e	|*	|y	|[43][S]\rarr	|a	|r	|m	|i	|a	|[][,]{ }	|c	|z	|e	|r	|w	|o	|n	|a	|*	|.
|*	|*	|*	|*	|*	|*	|*	|[44][S]\rarr	|c	|z	|e	|r	|w	|o	|n	|i	|e	|c	|*	|i	|n	|*	|.
|*	|[45][S]\rarr	|p	|e	|t	|r	|e	|l	|[][,]{ }	|m	|e	|l	|a	|n	|e	|z	|y	|j	|s	|k	|i	|*	|.
|[46][S]\rarr	|f	|i	|n	|a	|n	|s	|e	|[][,]{ }	|p	|u	|b	|l	|i	|c	|z	|n	|e	|*	|*	|e	|*	|.
|*	|*	|*	|*	|[47][S]\rarr	|t	|r	|ó	|j	|e	|c	|z	|k	|a	|*	|[48][S]\rarr	|s	|k	|o	|k	|*	|*	|.
|*	|*	|*	|*	|*	|*	|*	|*	|*	|*	|[49][S]\rarr	|d	|i	|g	|i	|t	|a	|l	|i	|s	|*	|*	|.
|*	|*	|*	|[50][S]\rarr	|s	|u	|r	|o	|w	|o	|ś	|ć	|*	|*	|*	|*	|*	|*	|*	|*	|*	|*	|.\end{Puzzle}

\newpage

\begin{PuzzleClues}{\textbf{Poziome}\\}\Clue{5}{}{wzgląd na innych, delikatność}
\Clue{7}{}{skrzydełko na przedniej krawędzi skrzydła samolotu}
\Clue{8}{}{ptak z rzędu mew - siewek podobny do brodźca, chroniony}
\Clue{9}{}{dwie szyny podtrzymujące i prowadzące koła pojazdów szynowych, ułożone na podkładach lub wlane w specjalną płytę betonową służą jako droga kolejowa, tramwajowa lub metro, w określonej odległości od siebie}
\Clue{10}{}{(1862-1923), pisarz francuski, rzecznik nacjonalistycznych poglądów politycznych; „Wyrwani z gruntu ojczystego”}
\Clue{11}{}{ugrupowania artystyczne powstałe w Krakowie, zagadnienia koloru i faktury}
\Clue{14}{}{skażenie komórek rakiem, bycie rakowatym}
\Clue{17}{}{dzbanek do parzenia kawy, który jest rodzajem prostego ekspresu}
\Clue{18}{}{takifugu - ryba morska z rodziny rozdymkowatych, której mięso jest bardzo silną trucizną}
\Clue{21}{}{w grze na instrumentach: pociąganie smyczkiem w dół}
\Clue{25}{}{euroazjatycki ssak z rodziny jeleniowatych łowna}
\Clue{30}{}{podniosłe obchody jakiegoś wydarzenia}
\Clue{31}{}{Aconitum variegatum - gatunek rośliny z rodzaju jaskrowatych, występujący w środkowej, południowej i wschodniej Europie oraz w Turcji}
\Clue{40}{}{urządzenie służące do określania zdolności kiełkowania nasion}
\Clue{41}{}{zdrobniale: motyw - w muzyce: charakterystyczny fragment melodii, który się powtarza}
\Clue{42}{}{materiał wybuchowy o właściwościach kruszących}
\Clue{43}{}{siły zbrojne Rosji sowieckiej (RFSRR), następnie wojska lądowe RFSRR w latach 1918-1922 i lądowe siły zbrojne ZSRR w latach 1922-1946}
\Clue{44}{}{Ceratodon purpureus - gatunek mchu liściastego}
\Clue{45}{}{Pseudobulweria becki - gatunek ptaka z rodziny burzykowatych (Procellariidae)}
\Clue{46}{}{dyscyplina naukowa obejmująca procesy związane z gromadzeniem, podziałem i wydatkowaniem finansowych środków publicznych, w oparciu o regulacje prawne, celem finansów publicznych są finansowanie deficytu budżetowego i obsługa długu publicznego}
\Clue{47}{}{zdrobniale o trójce - stopniu szkolnym}
\Clue{48}{}{ruch, który polega na oderwaniu się od podłoża na skutek odpowiedniego odbicia się nogami}
\Clue{49}{}{NAPARSTNICA roślina z trędnikowatych o jajowatych liściach, trująca}
\Clue{50}{}{brak skomplikowania, prostota}\end{PuzzleClues}

\begin{PuzzleClues}{\textbf{Pionowe}\\}\Clue{1}{}{góralek przylądkowy, Procavia capensis - gatunek ssaka z rodziny góralkowatych; zamieszkuje kontynent afrykański, Półwysep Arabski, Bliski Wschód, aż po Turcję}
\Clue{2}{}{mała kula myśliwska, gruby śrut}
\Clue{3}{}{GALERA; starogrecki okręt wojenny o dwóch rzędach wioseł w każdej burcie}
\Clue{4}{}{Trillium recurvatum - gatunek rośliny zielnej z rodziny melantkowatych}
\Clue{6}{}{patent przyznawany przez Europejski Urząd Patentowy i ważny w każdym z państw-członków Europejskiej Organizacji Patentowej wskazanych przez wnioskodawcę we wniosku o udzielenie patentu europejskiego}
\Clue{11}{}{Platysteiridae - rodzina ptaków z rzędu wróblowych (Passeriformes), obejmująca około trzydziestu gatunków ptaków; zamieszkują obszary tropikalne Afryki, są owadożerne, żyją w lasach lub w buszu}
\Clue{12}{}{cecha rzeczy, działania bądź zachowania: to, że coś jest bardzo negatywne, brzydkie i niepożądane}
\Clue{13}{}{prehistoryczny instrument muzyczny o długiej metalowej rurce zakończonej czarą  głosową}
\Clue{15}{}{pilot, Globicephala macrorhynchus - gatunek walenia zaliczany do rodziny delfinowatych, mimo iż swoim zachowaniem bardziej przypomina wieloryba; żyje w stadach liczących od 10 do 30 (lub więcej) okazów}
\Clue{16}{}{Calicalicus rufocarpalis - gatunek ptaka z rodziny wang (Vangidae), który występuje endemicznie na Madagaskarze}
\Clue{17}{}{oberża, jadłodajnia}
\Clue{18}{}{przetwarzanie rozbieżnych wiązek promieniowania na wiązki równoległe}
\Clue{19}{}{jedno- lub dwuśladowy mechaniczny pojazd drogowy bez nadwozia, przeznaczony do przewozu jednej lub dwóch osób; wyposażony jest w silnik spalinowy o masie własnej do 400 kg i o pojemności powyżej 50 cm3}
\Clue{20}{}{ponadprzeciętna znajomość czegoś, często związana z zamiłowaniem, pasją}
\Clue{21}{}{syn wujka}
\Clue{22}{}{osoba, która jest zdrajcą, sprzedawczykiem - dla zysków materialnych działa niezgodnie z zasadami np. moralnymi}
\Clue{23}{}{litera alfabetu używana w numeracji porządkowej}
\Clue{24}{}{tkanina z tego włókna}
\Clue{26}{}{zdrobniale: baranek - biała chmura, zwykle mała, pierzasta lub kłębiasta, konotowana pozytywnie}
\Clue{27}{}{Caenolestidae - rodzina małych, naziemnych torbaczy zaliczanych do skąpoguzkowców; występują w wysokich partiach Andów oraz wzdłuż południowych wybrzeży Chile na terenach trudno dostępnych, gęsto porośniętych i zwykle zimnych, co przyczyniło się do słabego ich poznania}
\Clue{28}{}{wydalenie dyplomaty, uznanego za persona non grata}
\Clue{29}{}{budynek, w którym trzyma się bydło}
\Clue{30}{}{duża miska - naczynie}
\Clue{31}{}{miasto w Kolumbii, w Andach, ośrodek administracyjny departamentu Antioquia, drugie po Bogocie miasto kraju}
\Clue{32}{}{człowiek oceniany bardzo negatywnie (z różnych powodów), nazywany tak, żeby go obrazić}
\Clue{33}{}{BUFFETING}
\Clue{34}{}{odmiana wyrazów}
\Clue{35}{}{podłużne naczynie gospodarskie służące do podawania pożywienia i wody większym zwierzętom hodowlanym}
\Clue{36}{}{betonowy pas startowy na lotnisku}
\Clue{37}{}{klasa leksemów autosemantycznych, których prymarną funkcją jest bycie członem konstytutywnym zdania - orzeczeniem}
\Clue{38}{}{sportowa wersja techniki walki}
\Clue{39}{}{w chemii: symbol potasu}\end{PuzzleClues}\newpage%\section*{Krzyżówka 54}

\noindent\begin{Puzzle}{24}{25}|*	|*	|*	|*	|*	|*	|[1][S]\drarr	|c	|ó	|r	|a	|[][,]{ }	|m	|a	|r	|n	|o	|t	|r	|a	|w	|n	|a	|*	|*	|.
|*	|[2][S]\darr	|[3][S]\darr	|*	|[4][S]\darr	|[5][S]\darr	|b	|*	|*	|[6][S]\drarr	|o	|l	|d	|e	|n	|b	|u	|r	|g	|e	|r	|*	|*	|*	|*	|.
|[7][S]\drarr	|b	|e	|z	|p	|a	|r	|d	|o	|n	|o	|w	|o	|ś	|ć	|*	|*	|*	|*	|*	|*	|*	|[8][S]\darr	|*	|*	|.
|p	|o	|u	|[9][S]\drarr	|s	|t	|e	|p	|n	|i	|a	|r	|k	|a	|[][,]{ }	|p	|a	|s	|k	|o	|w	|a	|n	|a	|*	|.
|o	|l	|k	|p	|i	|e	|y	|*	|*	|e	|*	|*	|[10][S]\drarr	|p	|o	|p	|a	|d	|i	|a	|n	|k	|a	|*	|*	|.
|r	|o	|a	|a	|z	|n	|e	|*	|*	|w	|[11][S]\drarr	|c	|h	|e	|m	|i	|z	|a	|c	|j	|a	|*	|d	|*	|*	|.
|o	|n	|l	|r	|ą	|y	|r	|*	|[12][S]\darr	|ł	|k	|[13][S]\rarr	|o	|g	|n	|i	|o	|m	|i	|s	|t	|r	|z	|*	|*	|.
|z	|i	|i	|ó	|b	|*	|*	|*	|k	|a	|o	|*	|n	|*	|*	|[14][S]\darr	|[15][S]\darr	|[16][S]\darr	|[17][S]\darr	|[18][S]\darr	|*	|*	|ó	|*	|*	|.
|u	|a	|p	|w	|[][,]{ }	|*	|[19][S]\darr	|*	|u	|ś	|p	|[20][S]\darr	|d	|*	|*	|m	|c	|s	|s	|b	|*	|*	|r	|*	|*	|.
|m	|*	|t	|k	|k	|[21][S]\darr	|s	|*	|l	|c	|a	|k	|u	|*	|[22][S]\darr	|i	|o	|z	|ł	|e	|[23][S]\darr	|*	|[][,]{ }	|[24][S]\darr	|*	|.
|i	|*	|u	|a	|a	|t	|p	|[25][S]\drarr	|f	|i	|l	|a	|r	|*	|p	|n	|o	|t	|a	|n	|m	|*	|b	|z	|*	|.
|e	|*	|s	|*	|u	|e	|ł	|w	|o	|w	|i	|p	|a	|*	|i	|i	|l	|u	|w	|t	|e	|*	|a	|a	|*	|.
|n	|*	|*	|*	|k	|k	|y	|e	|n	|o	|n	|u	|s	|*	|e	|s	|e	|r	|n	|a	|l	|*	|n	|p	|*	|.
|i	|*	|*	|*	|a	|t	|w	|l	|*	|ś	|a	|s	|*	|[26][S]\rarr	|s	|t	|r	|w	|o	|l	|a	|t	|k	|a	|*	|.
|e	|*	|*	|*	|s	|o	|[][,]{ }	|l	|*	|ć	|[][,]{ }	|t	|*	|[27][S]\darr	|*	|r	|*	|a	|*	|*	|s	|[28][S]\darr	|o	|r	|*	|.
|[][,]{ }	|*	|*	|[29][S]\darr	|k	|n	|k	|e	|*	|*	|l	|n	|*	|w	|[30][S]\darr	|a	|*	|ł	|*	|*	|*	|a	|w	|z	|*	|.
|p	|*	|*	|t	|i	|i	|a	|r	|*	|*	|e	|i	|*	|y	|k	|n	|*	|*	|*	|*	|*	|w	|y	|a	|*	|.
|ł	|[31][S]\darr	|*	|a	|*	|k	|j	|*	|*	|*	|c	|k	|*	|s	|a	|t	|*	|[32][S]\rarr	|l	|a	|l	|a	|*	|c	|*	|.
|a	|d	|*	|k	|*	|a	|a	|[33][S]\rarr	|g	|a	|z	|*	|*	|y	|m	|u	|[34][S]\rarr	|b	|o	|w	|e	|n	|*	|z	|*	|.
|c	|u	|[35][S]\darr	|e	|*	|[][,]{ }	|k	|*	|[36][S]\rarr	|a	|n	|t	|y	|p	|e	|r	|t	|y	|t	|*	|*	|p	|*	|*	|*	|.
|o	|c	|k	|m	|[37][S]\drarr	|p	|o	|l	|o	|w	|i	|e	|c	|*	|r	|a	|[38][S]\rarr	|e	|d	|a	|f	|o	|n	|*	|*	|.
|w	|h	|r	|i	|s	|ł	|w	|*	|*	|*	|c	|[39][S]\rarr	|h	|o	|t	|*	|[40][S]\rarr	|v	|e	|t	|e	|s	|c	|o	|*	|.
|e	|a	|e	|t	|a	|y	|y	|*	|[41][S]\rarr	|c	|z	|e	|r	|w	|o	|n	|a	|[][,]{ }	|k	|a	|r	|t	|k	|a	|*	|.
|*	|m	|o	|s	|k	|t	|*	|*	|[42][S]\rarr	|k	|a	|r	|b	|o	|n	|y	|l	|e	|k	|*	|*	|*	|*	|*	|*	|.
|[43][S]\rarr	|p	|l	|u	|s	|*	|[44][S]\rarr	|a	|l	|t	|*	|*	|*	|*	|*	|*	|*	|*	|*	|*	|*	|*	|*	|*	|*	|.
|*	|*	|*	|*	|*	|*	|*	|*	|*	|*	|*	|*	|*	|*	|*	|*	|*	|*	|*	|*	|*	|*	|*	|*	|*	|.\end{Puzzle}

\newpage

\begin{PuzzleClues}{\textbf{Poziome}\\}\Clue{1}{}{kobieta, która zbłądziła, ale wróciła na właściwą drogę}
\Clue{6}{}{koń oldenburski, oldenburg - jedna z ras koni, najcięższa spośród niemieckich ras gorącokrwistych, wyhodowana w XVII wieku, głównie dzięki wysiłkom księcia Antona Günther von Oldenburg; nowoczesny typ konia sportowego, przydatnego do skoków przez przeszkody}
\Clue{7}{}{cecha kogoś, kto jest pozbawiony skrupułów, działa bezwzględnie, nie liczy się z innymi}
\Clue{9}{}{Eremias scripta - gatunek gada z rodziny jaszczurkowatych, najmniejszy przedstawiciel rodzaju Eremias; występuje w środkowej Azji, w Kazachstanie i Iranie}
\Clue{10}{}{córka popa}
\Clue{11}{}{stosowanie substancji chemicznych w celu polepszenia jakości upraw, produktów i wyrobów}
\Clue{13}{}{stopień podoficerski w Wojsku Polskim i Państwowej Straży Pożarnej}
\Clue{25}{}{grunt, podstawa, zasadniczy element}
\Clue{26}{}{ryba tropikalnych mórz przypominająca ryby latające o długości do 40 cm}
\Clue{32}{}{figurka przedstawiająca człowieka (często niemowlę), zwierzę lub fikcyjnego humanoida, współczesnie zwykle wykonana z plastiku i tkaniny}
\Clue{33}{}{pedał gazu}
\Clue{34}{}{(1899-1973), pisarka angielska, powieści psychologiczne i nowele z życia mieszczaństwa; „Dziewczynki”}
\Clue{36}{}{rodzaj niejednorodnego zrostu mineralnego, składającego się ze skaleni potasowego i sodowego, w którym przeważa skaleń sodowy}
\Clue{37}{}{niemiecka bryczka myśliwska dla wielu osób, posiadająca specjalną skrzynię na upolowaną zwierzynę}
\Clue{38}{}{ogół organizmów żyjących w glebie}
\Clue{39}{}{styl w jazzie polegający na nadaniu każdemu dźwiękowi subiektywnego wyrazu poprzez stosowanie np. dużych kontrastów}
\Clue{40}{}{Antonio (1778-1833) malarz brazylijski; portrecista}
\Clue{41}{}{sprzeciw wobec czegoś}
\Clue{42}{}{związek koordynacyjny, w którym rolę ligandu pełni tlenek węgla (CO)}
\Clue{43}{}{korzyść osiągana w ostatecznym rozrachunku}
\Clue{44}{}{niski głos kobiecy lub chłopięcy; głos środkowy między sopranem a tenorem}\end{PuzzleClues}

\begin{PuzzleClues}{\textbf{Pionowe}\\}\Clue{1}{}{rzeźbiarz i medalier (1874-1952) portrety, medale, projekty monet, pomniki}
\Clue{2}{}{miasto we Włoszech, stolica Emilii-Romanii, ważny ośrodek przemysłowy, handlowy, kulturalny i turystyczny}
\Clue{3}{}{drewno pozyskane z drzewa o tej samej nazwie}
\Clue{4}{}{Erythronium caucasicum - gatunek roślin z rodziny liliowatych}
\Clue{5}{}{starożytne państwo}
\Clue{6}{}{cecha czegoś, co nie wydaje się właściwe, stosowne, nie jest dobrze oceniane np. pod względem moralnym lub pod względem grzeczności}
\Clue{7}{}{ugoda pomiędzy pracodawcą a grupą pracowników, na podstawie której ustala się wynegocjowane przez strony stawki wynagrodzeń}
\Clue{8}{}{monitorowanie działalności banków komercyjnych przez instytucję nadrzędną; jego podstawowym celem jest zapewnienie stabilności i bezpieczeństwa systemu bankowego oraz tworzenie norm ostrożnościowych}
\Clue{9}{}{żartobliwie: penis, członek}
\Clue{10}{}{państwo w Ameryce Środkowej, pomiędzy Morzem Karaibskim a Oceanem Spokojnym (Fonseca (zatoka)), o łącznej długości wybrzeża 820 km}
\Clue{11}{}{nieprzetworzone surowce mineralne tworzące złoże w skorupie ziemskiej, mające zastosowanie dla celów leczniczych; w szczególności są to wody lecznicze i torfy lecznicze, zwane także borowiną}
\Clue{12}{}{wielki, często brzydki nochal}
\Clue{14}{}{zbiór liturgicznych odpowiedzi ministranta w czasie mszy}
\Clue{15}{}{urządzenie do chłodzenia procesora}
\Clue{16}{}{poziomy wał z nawiniętą liną lub łańcuchem sterowym, połączony z kołem sterowym}
\Clue{17}{}{miasto w północno-zachodniej Polsce, w województwie zachodniopomorskim, w powiecie sławieńskim, położone na Pobrzeżu Koszalińskim, nad rzeką Wieprzą i strugą Moszczenicą, ok. 20 km od wybrzeża Morza Bałtyckiego}
\Clue{18}{}{strefa denna zbiornika wodnego; cechy charakterystyczne tej strefy to bardzo mała ilość światła lub jego brak, mała ilość tlenu, niska temperatura oraz większe ciśnienie}
\Clue{19}{}{wycieczka kajakiempo cieku wodnym}
\Clue{20}{}{popularny motyl wiosenny o białej barwie}
\Clue{21}{}{dominująca współcześnie teoria tłumacząca wielkoskalowe ruchy ziemskiej litosfery}
\Clue{22}{}{łow. samiec lisa, borsuka, jenota}
\Clue{23}{}{(1883-1966), pisarz grecki, sztuki teatralne, powieści historyczno-biograficzne; „Papa Nukoluzos”}
\Clue{24}{}{przyrząd do zaparzania herbaty lub kawy}
\Clue{25}{}{amerykański pediatra i bakteriolog ur. w 1915 r., współtwórca metody hodowli wirusa choroby Heinego-Medina, nagroda Nobla}
\Clue{27}{}{pojawienie się czegoś w dużej ilości lub liczbie, często nagle}
\Clue{28}{}{wysunięta ku nieprzyjacielowi placówka ubezpieczająca}
\Clue{29}{}{ur. w 1930 r., kompozytor japoński, założył pierwsze japońskie Studio Muzyki Elektronicznej}
\Clue{30}{}{dźwięk wzorcowy wedle którego stroi się instrumenty}
\Clue{31}{}{obraz którego plan pierwszy stanowi makieta plastyczna a plan dalszy malowidło bądź fotografia}
\Clue{35}{}{rodowity mieszkaniec Małych i Wielkich Antyli}
\Clue{37}{}{matematyk, profesor Uniwersytetu Lwowskiego (1897-1943); prace z teorii funkcji, zamordowany przez hitlerowców}\end{PuzzleClues}\newpage%\section*{Krzyżówka 55}

\noindent\begin{Puzzle}{24}{30}|*	|*	|*	|*	|*	|*	|*	|[1][S]\drarr	|k	|l	|e	|r	|o	|m	|a	|n	|c	|j	|a	|*	|*	|*	|*	|*	|[2][S]\darr	|.
|*	|*	|*	|*	|*	|[3][S]\rarr	|c	|h	|o	|r	|e	|u	|t	|a	|*	|[4][S]\darr	|*	|[5][S]\drarr	|a	|l	|k	|a	|*	|*	|j	|.
|*	|*	|*	|[6][S]\rarr	|w	|i	|d	|e	|ł	|k	|i	|*	|*	|[7][S]\drarr	|o	|b	|r	|o	|t	|n	|i	|c	|a	|*	|a	|.
|*	|[8][S]\rarr	|p	|u	|s	|t	|o	|r	|o	|ż	|c	|e	|*	|p	|*	|e	|*	|l	|[9][S]\drarr	|b	|u	|t	|*	|[10][S]\darr	|r	|.
|*	|[11][S]\rarr	|d	|ł	|u	|g	|o	|b	|i	|e	|g	|i	|*	|i	|*	|l	|*	|e	|m	|[12][S]\drarr	|s	|b	|*	|j	|z	|.
|[13][S]\drarr	|k	|r	|e	|a	|c	|j	|a	|*	|[14][S]\drarr	|s	|e	|k	|s	|*	|e	|*	|j	|a	|d	|[15][S]\darr	|*	|*	|e	|y	|.
|r	|*	|[16][S]\darr	|*	|*	|*	|*	|t	|[17][S]\rarr	|k	|i	|k	|u	|t	|*	|m	|*	|e	|r	|i	|m	|[18][S]\darr	|[19][S]\darr	|d	|n	|.
|e	|[20][S]\drarr	|b	|u	|f	|o	|n	|a	|d	|a	|*	|[21][S]\rarr	|p	|o	|*	|n	|*	|k	|k	|n	|e	|l	|s	|n	|k	|.
|n	|k	|r	|*	|*	|*	|*	|[][,]{ }	|*	|p	|*	|*	|*	|z	|*	|i	|*	|[][,]{ }	|e	|a	|t	|i	|ł	|o	|a	|.
|*	|o	|z	|*	|*	|*	|[22][S]\darr	|c	|*	|u	|[23][S]\drarr	|f	|r	|a	|c	|t	|o	|s	|t	|r	|a	|t	|u	|s	|*	|.
|[24][S]\drarr	|m	|u	|f	|a	|*	|b	|z	|[25][S]\darr	|c	|ś	|[26][S]\darr	|[27][S]\darr	|u	|*	|*	|[28][S]\darr	|a	|[][,]{ }	|[][,]{ }	|b	|e	|c	|t	|*	|.
|t	|o	|s	|*	|*	|*	|i	|e	|n	|y	|c	|s	|g	|r	|*	|*	|s	|n	|b	|j	|o	|r	|h	|k	|*	|.
|a	|d	|i	|*	|[29][S]\rarr	|m	|a	|r	|o	|n	|i	|t	|a	|*	|[30][S]\darr	|*	|t	|d	|u	|u	|l	|a	|[][,]{ }	|a	|*	|.
|c	|o	|e	|*	|*	|*	|ł	|w	|r	|k	|a	|a	|ś	|*	|f	|*	|a	|a	|d	|g	|i	|t	|m	|[][,]{ }	|*	|.
|a	|r	|c	|*	|*	|*	|e	|o	|m	|a	|n	|w	|n	|[31][S]\darr	|i	|*	|c	|ł	|o	|o	|t	|u	|u	|z	|*	|.
|*	|*	|*	|*	|*	|*	|[][,]{ }	|n	|a	|*	|a	|k	|i	|w	|l	|*	|j	|o	|w	|s	|[][,]{ }	|r	|z	|a	|*	|.
|[32][S]\drarr	|p	|o	|l	|ó	|w	|k	|a	|*	|*	|*	|a	|c	|o	|h	|[33][S]\darr	|a	|w	|l	|ł	|w	|a	|y	|l	|*	|.
|o	|[34][S]\rarr	|s	|i	|o	|ł	|o	|*	|*	|*	|*	|*	|a	|s	|a	|s	|[][,]{ }	|y	|a	|o	|t	|[][,]{ }	|c	|e	|*	|.
|e	|*	|*	|[35][S]\rarr	|f	|a	|ł	|s	|z	|y	|w	|y	|[][,]{ }	|t	|r	|o	|p	|*	|n	|w	|ó	|o	|z	|ż	|*	|.
|*	|[36][S]\rarr	|a	|r	|h	|a	|n	|t	|*	|*	|*	|*	|h	|o	|m	|c	|r	|*	|y	|i	|r	|b	|n	|n	|*	|.
|[37][S]\drarr	|s	|z	|l	|a	|g	|i	|e	|r	|*	|*	|*	|a	|k	|o	|j	|z	|*	|*	|a	|n	|o	|y	|a	|*	|.
|r	|*	|*	|*	|[38][S]\rarr	|d	|e	|l	|f	|z	|i	|j	|l	|*	|n	|o	|e	|*	|[39][S]\darr	|ń	|y	|z	|*	|*	|*	|.
|a	|[40][S]\rarr	|g	|i	|t	|a	|r	|i	|a	|d	|a	|*	|o	|*	|i	|t	|s	|*	|r	|s	|*	|o	|*	|*	|*	|.
|k	|[41][S]\rarr	|w	|a	|r	|d	|z	|a	|n	|k	|a	|*	|n	|*	|k	|e	|y	|*	|a	|k	|[42][S]\darr	|w	|*	|[43][S]\darr	|*	|.
|v	|[44][S]\rarr	|k	|r	|u	|p	|y	|*	|[45][S]\rarr	|r	|e	|s	|o	|r	|*	|c	|ł	|[46][S]\rarr	|f	|i	|t	|a	|*	|a	|*	|.
|e	|[47][S]\rarr	|i	|n	|d	|y	|k	|a	|t	|o	|r	|*	|w	|*	|*	|h	|o	|*	|a	|*	|a	|*	|*	|r	|*	|.
|r	|[48][S]\rarr	|ż	|a	|r	|l	|i	|w	|o	|ś	|ć	|*	|a	|*	|*	|n	|w	|*	|*	|*	|r	|*	|*	|a	|*	|.
|e	|*	|[49][S]\rarr	|ł	|u	|g	|*	|[50][S]\rarr	|s	|i	|t	|h	|*	|*	|*	|i	|a	|*	|[51][S]\rarr	|b	|ł	|y	|s	|k	|*	|.
|*	|*	|*	|*	|*	|*	|[52][S]\rarr	|l	|a	|m	|p	|a	|r	|c	|i	|k	|*	|[53][S]\rarr	|s	|m	|o	|k	|*	|*	|*	|.
|*	|*	|[54][S]\rarr	|b	|r	|ą	|z	|o	|w	|y	|[][,]{ }	|p	|o	|d	|k	|a	|r	|z	|e	|ł	|*	|*	|*	|*	|*	|.
|*	|*	|*	|*	|*	|*	|[55][S]\rarr	|b	|u	|k	|s	|z	|p	|a	|n	|*	|*	|*	|*	|*	|*	|*	|*	|*	|*	|.\end{Puzzle}

\newpage

\begin{PuzzleClues}{\textbf{Poziome}\\}\Clue{1}{}{wróżenie za pomocą kości zwierząt, kości do gry lub nasion bobu}
\Clue{3}{}{jeden z członków chóru w greckim teatrze antycznym}
\Clue{5}{}{ptak wodny z rzędu mew-siewek, czarno-biały, głównie rybożerne; obszar mórz i oceanów w półkuli północnej}
\Clue{6}{}{coś, co ma kształt widełek - rozgałęzienia}
\Clue{7}{}{urządzenie służące do zmiany kierunku pojazdu dokonywanej przez obrócenie platformy, na której ów pojazd stoi}
\Clue{8}{}{krętorogie, pustorogie, wołowate, Bovidae - rodzina ssaków parzystokopytnych z podrzędu przeżuwaczy; zamieszkują wszystkie kontynenty poza Antarktydą i Australią, największą grupę stanowią gatunki afrykańskie, a następnie azjatyckie}
\Clue{9}{}{wierzchnie okrycie stopy z mocną częścią spodnią (podeszwą)}
\Clue{11}{}{Eupetidae - rodzina ptaków z rzędu wróblowych}
\Clue{12}{}{jednostka luminancji układu CGS; 1 sb = 10E+4 nt}
\Clue{13}{}{strój damski noszony na specjalne okazje}
\Clue{14}{}{życie intymne}
\Clue{17}{}{pozostały odstający fragment czegoś, co zostało zniszczone}
\Clue{20}{}{przechwalanie się, zarozumialstwo}
\Clue{21}{}{skrótowa nazwa dawnego przedmiotu realizowanego w szkołach ponadgimnazjalnych i obecnie nauczanego częściowo na studiach, dawniej także w VII i VIII klasach szkół podstawowych; zakres nauczania obejmował szeroko pojętą obronę cywilną, podstawy pierwszej pomocy, metody ochrony przed różnymi zagrożeniami i przygotowanie do postępowania w wypadku katastrof}
\Clue{23}{}{chmura warstwowa, potargana przez wiatr na strzępy}
\Clue{24}{}{(zwornika) w wiertnictwie część zwornika w kształcie stożka, z gwintem wewnętrznym stanowiąca element złącza gwintowego}
\Clue{29}{}{członek chrześcijańskiego Kościoła wschodniego powstałego w VI w}
\Clue{32}{}{składane łóżko o lekkiej konstrukcji}
\Clue{34}{}{wieś, osada, osiedle wiejskie}
\Clue{35}{}{informacja podawana w celu wprowadzenia kogoś w błąd}
\Clue{36}{}{w buddyzmie hinajany jest to ktoś, kto w dążeniu do doskonałości osiągnął najwyższy stopień}
\Clue{37}{}{przebój, hit}
\Clue{38}{}{miasto i port w Holandii nad estuarium rzeki Ems}
\Clue{40}{}{przegląd, którego uczestnicy grają na gitarach}
\Clue{41}{}{Bembix - liczny, występujący prawie wszędzie rodzaj dużego, drapieżnego owada, najczęściej o jasnym ubarwieniu}
\Clue{44}{}{staropolska nazwa kaszy}
\Clue{45}{}{element sprężynujący, łagodzący wstrząsy w czasie jazdy pojazdów}
\Clue{46}{}{narzędzie do pomiaru średnicy drzewa}
\Clue{47}{}{urządzenie, którego zadaniem jest pomiar i zapis przebiegu zmian ciśnienia, które zachodzą w czasie}
\Clue{48}{}{cecha działania - działanie z zapałem, wypełnianie czegoś dokładnie i z ogromnym poświęceniem}
\Clue{49}{}{miejsce, obszar, rejon, który jest podmokły lub położony w pobliżu zbiornika wodnego i który cechuje się właściwą dla dużej wilotności fauną i florą}
\Clue{50}{}{wyznawca Ciemnej Strony Mocy, czerpiący z dorobku i tradycji rasy Sithów}
\Clue{51}{}{zdarzający się zwykle nagle i trwający krótko przejaw czegoś}
\Clue{52}{}{australijski ptak z rodziny czerwonek}
\Clue{53}{}{gumowe zakończenie butelki dla dziecka, umożliwiające mu spożywanie płynu poprzez ssanie}
\Clue{54}{}{obiekt astronomiczny o masie mniejszej niż minimalna masa brązowego karła, który utworzył się w wyniku zagęszczenia obłoku pyłowo-gazowego}
\Clue{55}{}{drewno bukszpanu}\end{PuzzleClues}

\begin{PuzzleClues}{\textbf{Pionowe}\\}\Clue{1}{}{rodzaj herbaty uprawianej w Chinach od ponad 1700 lat}
\Clue{2}{}{gotowy zestaw warzyw do przygotowania zupy sprzedawany w pęczkach}
\Clue{4}{}{głowonóg należący do rzędu belemnitów}
\Clue{5}{}{olejek eteryczny tworzony na bazie wyciągu z drzewa sandałowca}
\Clue{7}{}{Pistosaurus - nazwa rodzajowa morskiego gada, żyjącego w środkowym triasie (ladyn, około 230 milionów lat temu); najprawdopodobniej był formą przejściową między notozaurami a plezjozaurami - do dziś naukowcy nie są zgodni, czy należy go zaliczać do pierwszej, czy drugiej grupy zwierząt}
\Clue{9}{}{duży sklep samoobsługowy sprzedający materiały budowlane}
\Clue{10}{}{spółka handlowa, która kontrolowana jest przez jednostkę dominującą}
\Clue{12}{}{waluta obowiązująca w Jugosławii}
\Clue{13}{}{renifer, Rangifer tarandus - ssak z rodziny jeleniowatych, zamieszkujący arktyczną tundrę i lasotundrę w Eurazji i Ameryce Północnej}
\Clue{14}{}{członkini żeńskiego zakonu o regule św. Franciszka, odłam Klarysek}
\Clue{15}{}{metabolit, który nie jest bezpośrednio niezbędny do wzrostu i rozwoju organizmu}
\Clue{16}{}{skupienie włókien mięśniowych, które tworzy typowe mięśnie szkieletowe wraz ze ścięgnami}
\Clue{18}{}{ogół utworów związanych tematycznie z funkcjonowaniem hitlerowskich obozów koncentracyjnych oraz sowieckich łagrów}
\Clue{19}{}{zdolność różnicowania i powtarzania podstawowych cech dźwięku: wysokości, barwy i siły (głośności)}
\Clue{20}{}{grzecznościowy tytuł najstarszego kapitana statku pasażerskiego}
\Clue{22}{}{osoby pracujące w zawodach niewymagających pracy fizycznej; termin wprowadzony w amerykańskiej socjologii}
\Clue{23}{}{w biologii: u żywych organizmów najbardziej zewnętrzna warstwa jakiegoś organu}
\Clue{24}{}{płaskie naczynie do podawania potraw ale i do zbierania datków}
\Clue{25}{}{normalka, coś normalnego, np. normalna sytuacja}
\Clue{26}{}{przen. cena, jaką trzeba za coś zapłacić}
\Clue{27}{}{gaśnica, w której środkiem gaśniczym są halony}
\Clue{28}{}{podstacja służąca do przesyłania czegoś (najczęściej zasobu)}
\Clue{30}{}{członek orkiestry filharmonicznej}
\Clue{31}{}{seria radzieckich, załogowych statków kosmicznych}
\Clue{32}{}{jednostka natężenia pola magnetycznego w układzie CGS; 1 Oe = 79,577 A/m}
\Clue{33}{}{nauka badająca zachowania społeczne}
\Clue{37}{}{miasto w płn. części Estonii; przemysł spożywczy, drzewny}
\Clue{39}{}{skała podwodna w formie wału albo grzbietu, której górna część znajduje się tuż pod albo tuż nad powierzchnią wody}
\Clue{42}{}{okres godowy ryb, w którym odbywa się składanie jaj i ich zapładnianie}
\Clue{43}{}{miasto w Iranie u podnóża gór Zagros, ważny ośrodek wyrobu dywanów; dawniej: Sultanabad}\end{PuzzleClues}\newpage%\section*{Krzyżówka 57}

\noindent\begin{Puzzle}{20}{27}|*	|[1][S]\drarr	|c	|o	|k	|ó	|ł	|*	|[2][S]\drarr	|c	|i	|e	|t	|r	|z	|e	|w	|*	|*	|*	|*	|.
|*	|g	|*	|*	|*	|[3][S]\darr	|*	|[4][S]\drarr	|p	|e	|l	|i	|s	|a	|*	|[5][S]\darr	|[6][S]\darr	|*	|[7][S]\darr	|*	|*	|.
|[8][S]\drarr	|o	|p	|e	|r	|a	|*	|m	|e	|[9][S]\drarr	|b	|a	|y	|t	|o	|w	|n	|*	|k	|*	|*	|.
|k	|t	|[10][S]\drarr	|m	|i	|g	|d	|a	|ł	|o	|w	|i	|e	|c	|*	|y	|i	|*	|l	|*	|*	|.
|a	|*	|p	|*	|[11][S]\darr	|a	|*	|r	|n	|b	|[12][S]\drarr	|l	|o	|*	|[13][S]\rarr	|s	|e	|z	|a	|m	|*	|.
|r	|[14][S]\darr	|l	|*	|m	|p	|*	|y	|i	|e	|c	|*	|*	|[15][S]\darr	|*	|p	|s	|[16][S]\rarr	|p	|a	|*	|.
|a	|m	|a	|[17][S]\darr	|i	|a	|[18][S]\darr	|n	|a	|d	|h	|*	|*	|c	|*	|i	|z	|*	|s	|[19][S]\darr	|*	|.
|b	|n	|t	|p	|ł	|*	|u	|a	|*	|i	|l	|[20][S]\darr	|*	|z	|*	|a	|c	|*	|*	|w	|*	|.
|e	|i	|f	|r	|o	|[21][S]\darr	|p	|*	|*	|e	|a	|h	|*	|ł	|*	|r	|z	|*	|*	|ą	|*	|.
|l	|e	|o	|e	|s	|p	|i	|*	|*	|n	|d	|u	|[22][S]\drarr	|o	|c	|z	|e	|p	|*	|t	|*	|.
|a	|j	|r	|f	|i	|o	|ę	|[23][S]\darr	|[24][S]\drarr	|c	|e	|m	|e	|n	|t	|*	|r	|*	|[25][S]\darr	|r	|*	|.
|*	|s	|m	|e	|e	|r	|k	|s	|w	|j	|k	|m	|s	|[][,]{ }	|[26][S]\darr	|*	|o	|[27][S]\darr	|n	|ó	|*	|.
|[28][S]\drarr	|z	|a	|k	|r	|y	|s	|t	|i	|a	|*	|u	|k	|p	|b	|[29][S]\darr	|ś	|t	|e	|b	|*	|.
|z	|o	|[][,]{ }	|t	|d	|w	|z	|e	|d	|*	|*	|s	|a	|ę	|r	|m	|ć	|ę	|w	|k	|*	|.
|w	|ś	|w	|u	|z	|c	|e	|r	|z	|*	|*	|*	|r	|d	|o	|i	|*	|s	|r	|a	|*	|.
|i	|ć	|i	|r	|i	|z	|n	|b	|e	|*	|[30][S]\drarr	|b	|i	|o	|e	|t	|y	|k	|a	|*	|*	|.
|ą	|[][,]{ }	|e	|a	|e	|o	|i	|r	|n	|*	|p	|*	|o	|w	|d	|r	|[31][S]\darr	|n	|l	|*	|*	|.
|z	|n	|r	|*	|*	|ś	|e	|a	|i	|*	|a	|*	|l	|y	|e	|a	|k	|i	|g	|*	|*	|.
|a	|a	|t	|*	|*	|ć	|*	|m	|e	|[32][S]\rarr	|s	|i	|*	|*	|r	|*	|u	|c	|i	|*	|*	|.
|n	|r	|n	|*	|*	|*	|*	|s	|[][,]{ }	|*	|*	|[33][S]\rarr	|s	|o	|l	|*	|w	|a	|c	|*	|*	|.
|i	|o	|i	|*	|[34][S]\rarr	|n	|i	|e	|s	|n	|u	|j	|o	|w	|a	|t	|e	|*	|z	|[35][S]\darr	|*	|.
|e	|d	|c	|*	|*	|[36][S]\rarr	|a	|l	|k	|i	|e	|r	|z	|*	|m	|[37][S]\rarr	|t	|a	|n	|k	|*	|.
|[][,]{ }	|o	|z	|[38][S]\rarr	|z	|a	|k	|*	|ó	|*	|*	|*	|*	|*	|*	|*	|a	|[39][S]\darr	|o	|w	|*	|.
|s	|w	|a	|[40][S]\drarr	|g	|a	|l	|a	|r	|d	|a	|*	|*	|*	|*	|*	|*	|f	|ś	|a	|*	|.
|i	|a	|*	|s	|[41][S]\rarr	|t	|r	|a	|n	|s	|w	|e	|s	|t	|y	|t	|a	|*	|ć	|s	|*	|.
|ę	|*	|*	|ą	|*	|*	|*	|[42][S]\rarr	|e	|k	|t	|o	|p	|l	|a	|z	|m	|a	|*	|*	|*	|.
|*	|[43][S]\rarr	|o	|d	|j	|a	|z	|d	|*	|*	|*	|*	|*	|*	|*	|*	|*	|*	|*	|*	|*	|.
|*	|*	|*	|*	|*	|*	|*	|*	|*	|*	|*	|*	|*	|*	|*	|*	|*	|*	|*	|*	|*	|.\end{Puzzle}

\newpage

\begin{PuzzleClues}{\textbf{Poziome}\\}\Clue{1}{}{najniższa, nadziemna część budynku wysunięta lub wsunięta w stosunku do górnych partii muru}
\Clue{2}{}{leśny ptak łowny z rzędu kuraków, u samca lirowaty ogon; charakterystyczne toki; Eurazja}
\Clue{4}{}{zimowy płaszcz damski na futrze}
\Clue{8}{}{przedstawienie teatralno-muzyczne}
\Clue{9}{}{miasto w USA (Teksas) nad Zatoką Meksykańską}
\Clue{10}{}{niewysokie drzewo z różowatych uprawiane w krajach śródziemnomorskich i Kalifornii dla owoców zwanych migdałami}
\Clue{12}{}{skrótowiec odliceum ogólnokszałcące}
\Clue{13}{}{roślina z sezamowatych pochodząca z Afryki, oleista, uprawiana dla nasion do otrzymywania oleju spożywczego}
\Clue{16}{}{symbol paskala - jednostki ciśnienia (także naprężenia) w układzie SI}
\Clue{22}{}{pozioma belka wiążąca górne końce słupów budowy}
\Clue{24}{}{cement stomatologiczny - materiał stosowany w stomatologii jako podkład do wypełnień lub wypełnienie czasowe}
\Clue{28}{}{pomieszczenie w świątyni chrześcijańskiej w którym przechowuje się szaty i sprzęty liturgiczne}
\Clue{30}{}{interdyscyplinarna dziedzina wiedzy teoretycznej i praktycznej wynikająca z refleksji nad etycznym wymiarem działań z zakresu biologii i medycyny}
\Clue{32}{}{w chemii: symbol krzemu}
\Clue{33}{}{sylabowa nazwa dźwięku „g”}
\Clue{34}{}{rodzina błonkówek z podrzędu rośliniarek, larwy roślinożerne, wiele szkodników}
\Clue{36}{}{mały boczny pokój, izba, która pełniła funkcje sypialni}
\Clue{37}{}{w fotografice: głębokie naczynie z przykrywką wykonane z materiału odpornego na chemikalia, przeznaczone do obróbki filmów}
\Clue{38}{}{malarz, rysownik (1884-1926) współzałożyciel grupy Rytm; kompozycje figuralne. portrety}
\Clue{40}{}{FORKASZTEL}
\Clue{41}{}{osoba, która upodabnia się do przedstawicieli płci przeciwnej poprzez strój i zachowanie, celem osiągnięcia satysfakcji emocjonalnej lub seksualnej}
\Clue{42}{}{zewnętrzna warstwa cytoplazmy komórki}
\Clue{43}{}{coś czadowego (odjazdowego) i na topie, także: ocena zjawiska w formie wykrzyknika}\end{PuzzleClues}

\begin{PuzzleClues}{\textbf{Pionowe}\\}\Clue{1}{}{przedstawiciel jednego z plemion wschodniogermańskich}
\Clue{2}{}{stan pełnego rozwoju jakiegoś procesu, szczyt, kulminacja}
\Clue{3}{}{w niektórych środowiskach kościelno-religijnych: wspólna uczta członków jakiejś wspólnoty chrześcijańskiej}
\Clue{4}{}{polski, ludowy, instrument smyczkowy o niskiej skali, przypominający kształtem wiolonczelę; posiada od 2 do 4 strun; basetla}
\Clue{5}{}{określenie mieszkańca Wielkiej Brytanii}
\Clue{6}{}{cecha zachowania, działania: wymuszoność; brak szczerego, prawdziwego zaangażowania w działanie, sytuację, relację itp}
\Clue{7}{}{inaczej gruszka klapsa}
\Clue{8}{}{zbroja z metalowych łusek na podkładzie ze skóry lub grubego płótna}
\Clue{9}{}{W Kościele rzymskokatolickim posłuszeństwo, do jakiego zobowiązani są wierni wobec papieża i biskupów, będących w jedności z papieżem, w sprawach wiary i moralności (duchowa władza Kościoła nad wiernymi)}
\Clue{10}{}{najczęściej konstrukcja pływająca wyposażona w urządzenie wiertnicze, przeznaczona do wykonywania odwiertów pod dnem akwenu}
\Clue{11}{}{dobroć i litość okazywane komuś, współczucie}
\Clue{12}{}{amerykańska kajakarka górska, srebrna medalistka z Atlanty}
\Clue{14}{}{grupa ludzi zamieszkująca obszar danego państwa, odróżniająca się od większości społeczeństwa językiem, kulturą, pochodzeniem etnicznym bądź religią; posiada lub posiadała własne państwo}
\Clue{15}{}{część rośliny, hybryda rozwojowa, pochodząca zarówno ze zawiązków liści, jak i łodygi, według niektórych badaczy prawdopodobnie zatrzymanych częściowo na etapie rozwoju zarodkowego, czasami określana jako plecha}
\Clue{17}{}{jednostka podstawowego podziału administracyjnego w Japonii, Ruandzie, Czadzie i innych państwach}
\Clue{18}{}{to, czego funkcją jest ozdabianie, dodawanie waloru estetycznego}
\Clue{19}{}{potrawa z wątroby zwierzęcej}
\Clue{20}{}{porcja hummusu; określona ilość tego produktu, zazwyczaj plastikowe opakowanie}
\Clue{21}{}{cecha zachowania, które pokazuje, że ktoś jest porwyczy}
\Clue{22}{}{odmiana edywii o dużych podłużnych liściach}
\Clue{23}{}{rodzaj ożaglowania statku żaglowego}
\Clue{24}{}{bezpośredni odbiór fal elektromagnetycznych emitowanych przez badany przedmiot}
\Clue{25}{}{to, że coś jest bardzo ważne, kluczowe, pełni bardzo ważną funkcję}
\Clue{26}{}{malarz włoski (1900-72) malarstwo abstrakcyjne, pejzaże, martwe natury, obrazy figuralne}
\Clue{27}{}{demon słowiański wyobrażany w postaci bladego widma kobiecego, stanowiący personifikację stanu chorobliwego zadumania, rozmarzenia czy utęsknienia}
\Clue{28}{}{złączenie się, zjednoczenie}
\Clue{29}{}{szczególny rodzaj nakrycia głowy władców i arystokracji, będący formą czterodzielnej korony bez metalowej obręczy, obszytej gronostajowym futrem}
\Clue{30}{}{kształt, linia, graficznie wyodrębniony fragment powierzchni}
\Clue{31}{}{płytkie, prostokątne naczynie chemoodporne (najczęściej plastikowe) służące do wypełniania kąpielą fotograficzną lub wodą płucząca, stosowane do ręcznej obróbki materiałów fotograficznych w postaci arkusza (papiery fotograficzne, błony cięte, płyty)}
\Clue{35}{}{nazwa niektórych synsetycznych substancji psychoaktywnych (zwłaszcza najpopularniejszego LSD), zalicznanych przez niektórych do narkotyków, które przyjmuje się doustnie i które mają lekko szczypiący (kwaśny lub metaliczny) smak i które pod względem chemicznym najczęściej są przy okazji kwasami}
\Clue{39}{}{w chemii: symbol fluoru}
\Clue{40}{}{instytucja państwowa powołana do sądzenia, organ wymiaru sprawiedliwości}\end{PuzzleClues}\newpage%\section*{Krzyżówka 58}

\noindent\begin{Puzzle}{25}{25}|*	|[1][S]\darr	|[2][S]\darr	|*	|[3][S]\darr	|[4][S]\darr	|*	|*	|*	|[5][S]\darr	|*	|*	|*	|*	|*	|*	|[6][S]\drarr	|k	|r	|z	|y	|n	|a	|*	|*	|*	|.
|*	|c	|t	|[7][S]\darr	|s	|m	|*	|*	|*	|k	|*	|[8][S]\darr	|*	|[9][S]\drarr	|a	|m	|b	|u	|l	|a	|k	|r	|*	|*	|*	|*	|.
|[10][S]\drarr	|h	|e	|p	|t	|a	|s	|t	|e	|o	|r	|n	|i	|s	|*	|*	|r	|*	|[11][S]\darr	|*	|*	|[12][S]\darr	|*	|[13][S]\darr	|*	|*	|.
|o	|l	|l	|r	|r	|m	|*	|*	|*	|m	|*	|o	|*	|t	|*	|*	|z	|[14][S]\darr	|s	|*	|[15][S]\rarr	|d	|e	|m	|*	|*	|.
|z	|o	|e	|o	|o	|u	|[16][S]\darr	|*	|*	|p	|*	|c	|*	|o	|*	|*	|e	|s	|t	|*	|*	|ź	|*	|o	|*	|*	|.
|i	|r	|o	|t	|f	|t	|r	|*	|*	|l	|*	|n	|*	|p	|*	|*	|z	|z	|a	|*	|*	|w	|*	|b	|*	|*	|.
|ę	|a	|l	|o	|a	|[][,]{ }	|o	|[17][S]\rarr	|l	|e	|n	|i	|w	|e	|[][,]{ }	|p	|i	|e	|r	|o	|g	|i	|*	|i	|*	|*	|.
|b	|n	|o	|k	|n	|w	|ś	|*	|*	|k	|*	|c	|*	|r	|*	|*	|n	|l	|k	|*	|*	|ę	|*	|l	|*	|*	|.
|ł	|[][S](	|g	|ó	|t	|ł	|l	|*	|*	|s	|*	|ó	|*	|*	|*	|*	|a	|m	|*	|*	|*	|k	|*	|i	|*	|*	|.
|o	|v	|i	|ł	|[][,]{ }	|o	|i	|*	|*	|*	|*	|w	|*	|[18][S]\darr	|*	|*	|*	|a	|*	|[19][S]\darr	|*	|*	|*	|z	|*	|*	|.
|ś	|i	|z	|[][,]{ }	|w	|c	|n	|*	|[20][S]\rarr	|b	|u	|k	|s	|z	|p	|a	|n	|*	|[21][S]\rarr	|p	|i	|d	|ż	|a	|k	|*	|.
|ć	|i	|m	|d	|d	|h	|a	|*	|*	|[22][S]\rarr	|k	|a	|t	|a	|s	|t	|r	|o	|f	|i	|z	|m	|*	|c	|*	|*	|.
|[][,]{ }	|[][S])	|*	|w	|z	|a	|[][,]{ }	|*	|*	|*	|*	|*	|*	|p	|*	|*	|*	|*	|*	|p	|*	|*	|*	|j	|*	|*	|.
|p	|[][,]{ }	|*	|o	|i	|t	|p	|[23][S]\drarr	|m	|e	|r	|y	|t	|o	|r	|y	|k	|a	|*	|a	|*	|*	|*	|a	|*	|*	|.
|ł	|a	|*	|r	|ę	|y	|r	|b	|*	|[24][S]\rarr	|e	|m	|e	|r	|y	|t	|u	|r	|a	|*	|*	|*	|*	|*	|*	|*	|.
|c	|m	|*	|s	|c	|*	|z	|y	|*	|[25][S]\rarr	|p	|i	|r	|a	|c	|t	|w	|o	|*	|*	|*	|*	|*	|*	|*	|*	|.
|i	|o	|*	|k	|z	|*	|y	|c	|*	|*	|*	|*	|*	|[][,]{ }	|*	|*	|*	|*	|*	|*	|*	|*	|*	|*	|*	|*	|.
|o	|n	|*	|i	|n	|*	|p	|z	|[26][S]\rarr	|t	|r	|a	|n	|s	|p	|o	|r	|t	|o	|w	|i	|e	|c	|*	|*	|*	|.
|w	|u	|*	|*	|y	|[27][S]\darr	|r	|y	|*	|*	|*	|*	|*	|i	|*	|*	|*	|*	|*	|*	|*	|*	|*	|*	|*	|*	|.
|a	|*	|*	|*	|*	|k	|a	|n	|*	|[28][S]\rarr	|ł	|u	|p	|e	|k	|[][,]{ }	|p	|l	|a	|m	|i	|s	|t	|y	|*	|*	|.
|*	|*	|[29][S]\rarr	|b	|ł	|a	|w	|a	|t	|o	|w	|i	|e	|c	|[][,]{ }	|r	|u	|d	|o	|g	|ł	|o	|w	|y	|*	|*	|.
|*	|*	|*	|*	|*	|c	|o	|*	|*	|*	|*	|*	|*	|i	|*	|*	|*	|*	|*	|*	|*	|*	|*	|*	|*	|*	|.
|*	|*	|*	|*	|*	|z	|w	|*	|*	|*	|*	|*	|*	|o	|*	|*	|*	|*	|*	|*	|*	|*	|*	|*	|*	|*	|.
|[30][S]\rarr	|p	|u	|s	|z	|k	|a	|[][,]{ }	|m	|ó	|z	|g	|o	|w	|a	|*	|*	|*	|*	|*	|*	|*	|*	|*	|*	|*	|.
|*	|*	|*	|*	|*	|a	|*	|*	|*	|*	|*	|*	|[31][S]\rarr	|a	|m	|p	|e	|r	|o	|m	|i	|e	|r	|z	|*	|*	|.
|*	|*	|*	|*	|*	|*	|*	|*	|*	|*	|*	|*	|*	|*	|*	|*	|*	|*	|*	|*	|*	|*	|*	|*	|*	|*	|.\end{Puzzle}

\newpage

\begin{PuzzleClues}{\textbf{Poziome}\\}\Clue{6}{}{odrobina, mała ilość}
\Clue{9}{}{dawny krużganek, zwykle kościelny}
\Clue{10}{}{Heptasteornis - rodzaj dwunożnego, mięsożernego dinozaura, przypuszczalnie z rodziny alwarezaurów, a dokładniej z podrodziny mononykinów}
\Clue{15}{}{podstawowa jednostka administracyjna i terytorialna w starożytnej Grecji, gmina}
\Clue{17}{}{kluski przyrządzane z twarogu, jaj i mąki, gotowane w lekko osolonej wodzie}
\Clue{20}{}{Buxus - rodzaj zimozielonych drzew i krzewów z rodziny bukszpanowatych}
\Clue{21}{}{kurtka marynarska}
\Clue{22}{}{dekadencka postawa wyrażająca przeświadczenie o nieuniknionej, gwałtownej zagładzie obecnej formy świata i cywilizacji}
\Clue{23}{}{wiedza, która dotyczy jakiejś konkretnej dziedziny, zagadnienia}
\Clue{24}{}{okres, w którym osoba po przepracowaniu odpowiedniej liczby lat i osiągnięciu określonego wieku nie pracuje zawodowo, a pobiera wynagrodzenie zwane emeryturą}
\Clue{25}{}{łamanie przepisów ruchu drogowego, powodujące narażenie bezpieczeństwa uczestników ruchu}
\Clue{26}{}{statek przeznaczony do transportu towarów}
\Clue{28}{}{skała metamorficzna powstała w wyniku przeobrażenia łupków ilastych w wyniku kontaktu z intruzją}
\Clue{29}{}{gatunek ptaka z rodziny bławatniowatych (Cotingidae)}
\Clue{30}{}{górna, chrzęstna lub skostniała część czaszki, stykająca się z mózgowiem i osłaniająca mózg, nazywany mózgoczaszką}
\Clue{31}{}{przyrząd do pomiaru natężenia elektrycznego}\end{PuzzleClues}

\begin{PuzzleClues}{\textbf{Pionowe}\\}\Clue{1}{}{nieorganiczny związek chemiczny, sól amonowa kwasu nadchlorowego}
\Clue{2}{}{w filozofii: pogląd, według którego rozwój świata przyrody i rzeczywistości społecznej zmierza do jakiegoś ostatecznego celu}
\Clue{3}{}{Strophanthus gratus - gatunek rośliny z rodziny toinowatych}
\Clue{4}{}{mamut właściwy, Mammuthus primigenius - gatunek wymarłego ssaka z rodziny słoniowatych, z rzędu trąbowców, jedyny gatunek mamuta, który miał gęstą sierść; pojawił się ok. 250 tysięcy lat temu i szybko rozprzestrzenił w Europie i Azji, potem również w Ameryce Północnej}
\Clue{5}{}{zbiór myśli, słów, wyobrażeń silnie skojarzonych z pewną inną ważną i silnie zabarwioną emocjonalnie myślą, która zwykle bywa wyparta ze świadomości, stłumiona}
\Clue{6}{}{drewno brzozy}
\Clue{7}{}{zasady i zwyczaje dotyczące sposobów zachowania obowiązujących na dworze królewskim}
\Clue{8}{}{Rodzaj ćmy z rodziny sówkowatych;}
\Clue{9}{}{sekundomierz}
\Clue{10}{}{obniżenie popędu płciowego}
\Clue{11}{}{fizyk niemiecki (1874-1957); odkrywcze prace z fizyki atomowej; laureat nagrody Nobla}
\Clue{12}{}{element warstwy fonicznej wyrazu}
\Clue{13}{}{powołanie rezerwistów do czynnej służby wojskowej w wypadku zagrożenia bezpieczeństwa państwa}
\Clue{14}{}{ktoś, kto jest zawiadiacki, nieco bezczelny, przy tym żartobliwy, skory do drwin i żartów}
\Clue{16}{}{roślina zawierająca silne substancje aromatyczne, dzięki którym może być stosowana jako przyprawa}
\Clue{18}{}{jeden ze sposobów zabezpieczania sieci i systemów przed intruzami; oprogramowanie blokujące niepowołany dostęp do komputera, na którego straży stoi}
\Clue{19}{}{chiński szarpany instrument strunowy; czasami nazywany chińską lutnią, bo podobnie jak ona ma pudło rezonansowe w gruszkowatym kształcie}
\Clue{23}{}{miasto w Polsce, w województwie opolskim, w powiecie kluczborskim, siedziba gminy miejsko-wiejskiej Byczyna}
\Clue{27}{}{układ samolotu lub szybowca o usterzeniu wysokości umieszczonym przed skrzydłami}\end{PuzzleClues}\newpage%\section*{Krzyżówka 59}

\noindent\begin{Puzzle}{21}{33}|*	|*	|*	|*	|*	|*	|*	|*	|[1][S]\darr	|*	|*	|*	|*	|*	|*	|*	|*	|*	|*	|*	|*	|*	|.
|*	|*	|*	|*	|*	|*	|*	|*	|l	|*	|*	|*	|[2][S]\drarr	|e	|n	|d	|y	|w	|i	|a	|*	|*	|.
|*	|*	|*	|*	|*	|*	|*	|*	|i	|*	|*	|[3][S]\drarr	|p	|e	|r	|i	|l	|l	|u	|s	|*	|*	|.
|*	|*	|*	|*	|*	|*	|*	|[4][S]\darr	|t	|*	|[5][S]\darr	|b	|o	|*	|[6][S]\darr	|*	|[7][S]\darr	|*	|*	|*	|[8][S]\darr	|*	|.
|*	|*	|*	|*	|*	|*	|*	|p	|r	|*	|k	|e	|s	|*	|w	|*	|a	|[9][S]\darr	|*	|*	|w	|*	|.
|*	|[10][S]\rarr	|m	|a	|k	|u	|t	|r	|a	|*	|a	|j	|ą	|[11][S]\darr	|y	|*	|n	|l	|*	|*	|i	|*	|.
|*	|[12][S]\rarr	|a	|r	|i	|u	|s	|z	|*	|*	|r	|r	|g	|w	|s	|*	|t	|i	|*	|*	|e	|*	|.
|*	|[13][S]\drarr	|g	|i	|b	|e	|r	|e	|l	|i	|n	|a	|*	|y	|o	|*	|y	|c	|*	|*	|l	|*	|.
|[14][S]\drarr	|k	|a	|ł	|a	|m	|a	|s	|z	|k	|a	|*	|[15][S]\darr	|c	|k	|*	|s	|e	|*	|*	|k	|*	|.
|t	|u	|[16][S]\rarr	|b	|a	|r	|a	|t	|*	|*	|t	|*	|o	|i	|o	|[17][S]\darr	|e	|n	|*	|*	|o	|*	|.
|u	|d	|*	|*	|[18][S]\rarr	|h	|u	|r	|y	|t	|a	|*	|o	|o	|ś	|w	|p	|c	|[19][S]\darr	|*	|p	|*	|.
|r	|u	|[20][S]\drarr	|ż	|w	|a	|c	|z	|*	|*	|k	|*	|k	|r	|c	|y	|t	|j	|s	|*	|ł	|*	|.
|b	|s	|t	|*	|[21][S]\darr	|*	|*	|e	|[22][S]\drarr	|f	|a	|x	|a	|*	|i	|z	|y	|a	|z	|*	|e	|[23][S]\darr	|.
|i	|*	|e	|[24][S]\darr	|s	|*	|*	|l	|k	|*	|*	|*	|*	|*	|o	|n	|k	|[][,]{ }	|e	|*	|t	|b	|.
|n	|[25][S]\drarr	|k	|u	|c	|h	|n	|i	|a	|[][,]{ }	|g	|a	|z	|o	|w	|a	|*	|p	|r	|*	|w	|a	|.
|a	|t	|s	|ł	|o	|*	|*	|n	|d	|*	|[26][S]\darr	|*	|[27][S]\drarr	|e	|i	|n	|t	|o	|p	|f	|*	|ł	|.
|[][,]{ }	|a	|t	|u	|t	|[28][S]\darr	|[29][S]\darr	|a	|ł	|[30][S]\darr	|s	|[31][S]\darr	|j	|[32][S]\darr	|e	|i	|*	|e	|a	|*	|[33][S]\darr	|a	|.
|w	|r	|*	|s	|t	|m	|j	|*	|u	|o	|k	|s	|i	|j	|c	|e	|*	|t	|*	|[34][S]\darr	|e	|k	|.
|o	|c	|*	|*	|*	|o	|o	|[35][S]\drarr	|b	|r	|u	|z	|d	|a	|*	|*	|*	|y	|[36][S]\darr	|s	|g	|i	|.
|d	|z	|*	|[37][S]\drarr	|f	|r	|y	|s	|*	|k	|l	|a	|i	|b	|*	|[38][S]\darr	|*	|c	|p	|e	|z	|r	|.
|n	|n	|*	|p	|[39][S]\darr	|t	|*	|a	|*	|a	|*	|ł	|s	|ł	|*	|s	|*	|k	|r	|r	|o	|i	|.
|a	|i	|*	|a	|b	|u	|*	|d	|*	|*	|*	|*	|z	|o	|*	|y	|*	|a	|a	|w	|r	|e	|.
|*	|k	|[40][S]\rarr	|s	|o	|s	|n	|o	|w	|a	|t	|e	|*	|ń	|*	|r	|*	|*	|w	|o	|c	|w	|.
|*	|*	|*	|t	|c	|*	|*	|w	|*	|[41][S]\rarr	|ł	|u	|k	|[][,]{ }	|ż	|e	|b	|r	|o	|w	|y	|*	|.
|*	|[42][S]\rarr	|r	|o	|z	|m	|i	|a	|r	|*	|*	|*	|*	|p	|*	|n	|*	|*	|[][,]{ }	|a	|z	|[43][S]\darr	|.
|*	|[44][S]\rarr	|k	|r	|e	|s	|y	|*	|*	|*	|[45][S]\rarr	|s	|m	|u	|ż	|k	|a	|*	|b	|n	|m	|a	|.
|*	|*	|[46][S]\darr	|a	|k	|[47][S]\rarr	|s	|k	|r	|a	|p	|l	|a	|r	|k	|a	|*	|*	|a	|i	|*	|l	|.
|*	|*	|t	|ł	|*	|*	|[48][S]\rarr	|r	|o	|z	|s	|t	|ę	|p	|*	|*	|*	|*	|n	|e	|*	|f	|.
|*	|*	|u	|k	|*	|*	|*	|[49][S]\rarr	|e	|k	|s	|t	|r	|u	|z	|j	|a	|*	|k	|*	|*	|a	|.
|*	|[50][S]\rarr	|n	|a	|b	|ł	|o	|n	|e	|k	|[][,]{ }	|b	|a	|r	|w	|n	|i	|k	|o	|w	|y	|*	|.
|*	|*	|n	|*	|*	|*	|[51][S]\rarr	|t	|e	|s	|t	|[][,]{ }	|k	|o	|m	|e	|t	|o	|w	|y	|*	|*	|.
|*	|*	|*	|*	|*	|[52][S]\rarr	|r	|u	|s	|a	|ł	|k	|o	|w	|a	|t	|e	|*	|e	|*	|*	|*	|.
|*	|[53][S]\rarr	|r	|y	|c	|y	|k	|[][,]{ }	|z	|w	|y	|c	|z	|a	|j	|n	|y	|*	|*	|*	|*	|*	|.
|*	|*	|[54][S]\rarr	|s	|t	|r	|a	|d	|i	|v	|a	|r	|i	|*	|*	|*	|*	|*	|*	|*	|*	|*	|.\end{Puzzle}

\newpage

\begin{PuzzleClues}{\textbf{Poziome}\\}\Clue{2}{}{dwuletnia roślina warzywna z rodziny złożonych, liście spożywane jak sałata}
\Clue{3}{}{północnoameryk. pluskwiak, pasożyt wszystkich stadiów rozwojowych stonki ziemniaczanej}
\Clue{10}{}{gliniane naczynie, które służy do ucierania, zwykle maku (ale też kremu, białek, ciasta)}
\Clue{12}{}{aleksandryjski teolog i filozof (256-335); twórca arianizmu}
\Clue{13}{}{hormon roślinny wpływający na wielkość komórek i przyspieszający kwitnienie}
\Clue{14}{}{KAŁAMAJKA, wąski wózek wybity futrem, bez budki, mniejszy od bryczki}
\Clue{16}{}{stalowy bęben obrotowy do siarczkowania celulozy}
\Clue{18}{}{przedstawiciel historycznego ludu wywodzącego się z Wyżyny Armeńskiej, później zasiedlającego Mezopotamię}
\Clue{20}{}{parzysty mięsień żucia, znajdujący się po obu stronach głowy}
\Clue{22}{}{zatoka Oceanu Atlantyckiego, w zach. wybrzeży Islandii}
\Clue{25}{}{rodzaj kuchenki wykorzystującej gaz ziemny lub mieszaninę propan-butan}
\Clue{27}{}{potrawa jednogarnkowa, popularny posiłek przygotowany w jednym naczyniu, zastępujący cały obiad}
\Clue{35}{}{fałdka, szczelina na mózgu}
\Clue{37}{}{rodzaj flaneli w kratkę, noszonej czasem jako dolna warstwa odzieży}
\Clue{40}{}{Pinaceae - rodzina drzew i krzewów należąca do rzędu sosnowców (Pinales)}
\Clue{41}{}{element anatomiczny, który tworzą chrząstki żeber od dziesiątego do siódmego}
\Clue{42}{}{cecha czegoś, co nie istnieje fizycznie, nie ma łatwych do opisania cech fizycznych, znaczenie, zakres, stopień nasilenia jakiegoś zjawiska lub stanu}
\Clue{44}{}{określenie Kresów Wschodnich, na wschód od Bugu}
\Clue{45}{}{mały gryzoń przesypiający zimę}
\Clue{47}{}{urządzenie do skraplania gazu}
\Clue{48}{}{różnica między największą i najmniejszą wartością cechy statystycznej w zbiorze (lub różnica między najwyższą i najniższą zaobserwowaną wartością zmiennej)}
\Clue{49}{}{w ortodoncji: korekta mająca na celu wydłużenie części naddziąsłowej zęba; wyciskanie zębów z dziąsła, wydłużanie zębów poprzez wyciągnięcie ich części ukrytej w dziąśle}
\Clue{50}{}{struktura anatomiczna występująca w siatkówce i tęczówce oka}
\Clue{51}{}{test służący badaniu uszkodzeń DNA oraz umożliwiający znalezienie sposobu na ich uniknięcie}
\Clue{52}{}{południcowate, południce, perłowce, Nymphalidae - rodzina z grupy motyli dziennych; zawiera 5 700 gatunków rozpowszechnionych na całym świecie (w Polsce 75), m.in. rusałki, przeplatki, dostojki, mieniaki}
\Clue{53}{}{Limosa limosa limosa - podgatunek ptaka wyróżniony w obrębie gatunku rycyk (Limosa limosa)}
\Clue{54}{}{rodzina wybitnych lutników włoskich działających w Cremonie w XVII- XVIII w}\end{PuzzleClues}

\begin{PuzzleClues}{\textbf{Pionowe}\\}\Clue{1}{}{funt ruski w XVI-XVII wieku}
\Clue{2}{}{rzeźbiarz francuski (1535-90) współtwórca typu nagrobka przedstawiającego postać zmarłego w naturalistyczny sposób}
\Clue{3}{}{Dorcatragus megalotis - gatunek antylopy z rodziny krętorogich, jedyny przedstawiciel rodzaju Dorcatragus; zamieszkuje skąpo porośnięte wzgórza Afryki graniczące z Morzem Czerwonym (w Etiopii i Somalii)}
\Clue{4}{}{uszkodzenie czegoś powstałe w wyniku przestrzelenia pociskiem}
\Clue{5}{}{dawniej Majsur; stan w płd Indiach, na wyżynie Dekan, powierzchnia 191,8 tyś. km2, stolica Bangalur}
\Clue{6}{}{wysoki budynek podzielony na piętra}
\Clue{7}{}{środek dezynfekcyjny, środek antyseptyczny - substancja, która niszczy drobnoustroje i ich przetrwalniki}
\Clue{8}{}{słodkowodna ryba okoniokształtnaj z rodziny guramiowatych (Osphronemidae)}
\Clue{9}{}{uzasadniane zamysłem artystycznym odstępstwo od norm lub obyczajów}
\Clue{11}{}{WYSZOR}
\Clue{13}{}{indonezyjskie miasto na Jawie}
\Clue{14}{}{wirnikowy silnik wodny przetwarzający energię mechaniczną płynącej wody}
\Clue{15}{}{pisarz japoński ur. w 1909r - autobiografie, powieści, nowele psychologiczne; „Ognie polne”}
\Clue{17}{}{zbiór podstawowych wierzeń danej religii: takie nazwa odrębnych grup religijnych (np. rzymskokat0lickie, prawosławie, protestanckie)}
\Clue{19}{}{tragarz wykorzystywany przez himalaistów}
\Clue{20}{}{zbiór słów, wypowiedź (często utrwalona graficznie)}
\Clue{21}{}{architekt angielski (1811-78), budowle neogotyckie, budowle rządowe}
\Clue{22}{}{podstawowa część statku złożona ze szkieletu, poszycia burt, grodzi, pokładów i nadbudówek}
\Clue{23}{}{kompozytor rosyjski (1837-1910); symfonie, utwory fortepianowe, poematy symfoniczne; 'Tamara', 'Rus'}
\Clue{24}{}{jednostka organizacyjna, społeczna i terytorialna koczowniczych plemion środkowej Azji i Syberii}
\Clue{25}{}{pluskwiak równoskrzydły z podrzędu czerwców, szkodniki upraw; sadów}
\Clue{26}{}{jednoosobowa łódź półwyścigowa}
\Clue{27}{}{jidysz}
\Clue{28}{}{bieda, ubóstwo, brak środków materialnych do zaspokojenia podstawowych potrzeb jednostki}
\Clue{29}{}{astronom amerykański (1882-1973); wyznaczył prędkość obiegu Słońca wokół środka Galaktyki, odkrył gwiazdy zmienne}
\Clue{30}{}{MIECZNIK; drapieżnik z delfinów o cennym tłuszczu i niejadalnym mięsie}
\Clue{31}{}{szaleństwo, amok - stan psychiczny będący skutkiem silnych: namiętności, gniewu, radości, stan wielkiego podniecenia}
\Clue{32}{}{Malus purpurea - gatunek rośliny z rodziny różowatych, mieszaniec M. atrosanguinea i M. sieversii 'Niedzvetskyana'}
\Clue{33}{}{w Kościele katolickim: obrzęd liturgiczny mający na celu usunięcie wpływu szatana na osobę lub rzecz}
\Clue{34}{}{podawanie jedzenia}
\Clue{35}{}{lekkoatletka ros., wicemistrzyni olimpijska w rzucie dyskiem z Atlanty}
\Clue{36}{}{zespół norm prawnych regulujących powstawanie, funkcjonowanie oraz zasady likwidowania instytucji bankowych (zwłaszcza banków), a także zasady funkcjonowania nadzoru bankowego}
\Clue{37}{}{ludowa, wesoła pieśń polska związana tematycznie z Bożym Narodzeniem, także widowisko sceniczne o takiej tematyce}
\Clue{38}{}{polski samochód osobowy produkowany przez FSO, model Syrena}
\Clue{39}{}{zdrobniale: prawa lub lewa strona ciała ludzkiego lub zwierzęcego}
\Clue{43}{}{Pyrus communis 'Alfa' - odmiana uprawna gruszy pospolitej}
\Clue{46}{}{jezioro w Norwegii, w Górach Skandynawskich}\end{PuzzleClues}\newpage%\section*{Krzyżówka 60}

\noindent\begin{Puzzle}{23}{27}|*	|*	|*	|*	|*	|*	|*	|*	|*	|*	|*	|*	|*	|*	|*	|*	|*	|*	|*	|*	|[1][S]\darr	|*	|[2][S]\darr	|*	|.
|*	|*	|*	|[3][S]\drarr	|k	|o	|n	|s	|t	|r	|u	|k	|t	|y	|w	|n	|o	|ś	|ć	|*	|f	|[4][S]\darr	|t	|*	|.
|*	|*	|[5][S]\darr	|l	|*	|*	|*	|*	|[6][S]\drarr	|a	|r	|*	|[7][S]\rarr	|k	|r	|a	|j	|a	|l	|n	|i	|c	|a	|*	|.
|*	|[8][S]\drarr	|k	|o	|ś	|ć	|[][,]{ }	|k	|u	|l	|s	|z	|o	|w	|a	|*	|*	|*	|*	|*	|n	|z	|t	|*	|.
|*	|p	|o	|t	|*	|[9][S]\drarr	|b	|y	|k	|*	|*	|*	|[10][S]\drarr	|k	|u	|r	|a	|n	|t	|*	|a	|y	|u	|*	|.
|*	|a	|m	|n	|*	|k	|*	|*	|ł	|*	|*	|*	|g	|*	|*	|[11][S]\drarr	|f	|k	|p	|*	|ł	|s	|a	|*	|.
|*	|p	|a	|i	|*	|a	|*	|*	|a	|[12][S]\drarr	|a	|k	|a	|t	|a	|l	|e	|k	|s	|a	|*	|t	|ż	|*	|.
|*	|i	|ń	|a	|[13][S]\drarr	|s	|p	|a	|d	|o	|c	|h	|r	|o	|n	|i	|a	|r	|z	|*	|[14][S]\darr	|e	|[][,]{ }	|*	|.
|*	|e	|c	|r	|t	|z	|*	|*	|[][,]{ }	|f	|*	|*	|b	|*	|*	|w	|*	|[15][S]\darr	|[16][S]\darr	|*	|h	|[][,]{ }	|a	|*	|.
|*	|r	|z	|s	|r	|a	|[17][S]\darr	|[18][S]\darr	|k	|f	|*	|*	|a	|[19][S]\drarr	|l	|e	|n	|e	|k	|*	|a	|r	|m	|*	|.
|*	|n	|a	|t	|z	|l	|e	|n	|a	|t	|*	|*	|t	|a	|[20][S]\darr	|r	|*	|w	|j	|[21][S]\darr	|l	|ę	|a	|*	|.
|*	|i	|n	|w	|e	|o	|n	|i	|l	|o	|*	|[22][S]\darr	|k	|r	|m	|u	|[23][S]\darr	|a	|*	|n	|a	|c	|l	|*	|.
|*	|c	|i	|o	|ź	|t	|i	|e	|k	|p	|[24][S]\drarr	|d	|o	|m	|i	|n	|a	|n	|t	|a	|*	|e	|g	|*	|.
|*	|z	|n	|*	|w	|o	|g	|h	|u	|*	|s	|o	|w	|a	|a	|e	|g	|g	|[25][S]\darr	|t	|[26][S]\darr	|*	|a	|*	|.
|*	|y	|*	|[27][S]\darr	|o	|w	|m	|i	|l	|[28][S]\darr	|t	|m	|a	|t	|u	|k	|e	|e	|o	|u	|f	|*	|m	|*	|.
|*	|*	|*	|n	|ś	|a	|o	|s	|a	|s	|ę	|i	|t	|a	|c	|*	|n	|l	|d	|r	|r	|*	|a	|*	|.
|*	|*	|[29][S]\darr	|a	|ć	|t	|z	|t	|c	|t	|p	|n	|e	|*	|z	|[30][S]\darr	|t	|i	|g	|a	|y	|[31][S]\darr	|t	|*	|.
|[32][S]\drarr	|k	|p	|w	|*	|e	|a	|o	|y	|o	|a	|i	|*	|*	|e	|h	|u	|k	|r	|l	|w	|m	|o	|*	|.
|p	|*	|o	|i	|*	|*	|u	|r	|j	|p	|k	|k	|*	|*	|k	|u	|r	|a	|y	|i	|o	|i	|w	|*	|.
|a	|[33][S]\drarr	|p	|e	|l	|e	|r	|y	|n	|a	|*	|a	|*	|*	|[][,]{ }	|s	|a	|l	|w	|z	|l	|e	|y	|*	|.
|p	|h	|ł	|r	|*	|*	|*	|c	|y	|*	|*	|n	|*	|*	|z	|l	|*	|i	|k	|m	|n	|d	|*	|*	|.
|i	|e	|o	|z	|*	|*	|*	|z	|*	|*	|*	|k	|[34][S]\rarr	|g	|i	|e	|m	|z	|a	|*	|o	|n	|[35][S]\darr	|*	|.
|r	|r	|c	|c	|*	|*	|*	|n	|[36][S]\rarr	|g	|ł	|a	|d	|c	|e	|*	|*	|m	|*	|*	|ś	|i	|l	|*	|.
|u	|t	|h	|h	|[37][S]\drarr	|a	|p	|o	|l	|l	|o	|*	|[38][S]\rarr	|f	|l	|e	|t	|*	|*	|*	|ć	|c	|o	|*	|.
|s	|z	|*	|n	|e	|*	|*	|ś	|*	|*	|*	|*	|[39][S]\rarr	|p	|o	|d	|k	|ł	|a	|d	|*	|z	|r	|*	|.
|*	|*	|*	|i	|*	|*	|*	|ć	|[40][S]\rarr	|k	|r	|y	|m	|i	|n	|a	|l	|i	|s	|t	|y	|k	|a	|*	|.
|*	|[41][S]\rarr	|k	|a	|l	|k	|a	|*	|*	|[42][S]\rarr	|b	|a	|ł	|t	|y	|s	|t	|y	|k	|a	|*	|a	|*	|*	|.
|*	|*	|*	|*	|*	|[43][S]\rarr	|t	|k	|l	|i	|w	|o	|ś	|ć	|*	|*	|*	|*	|*	|*	|*	|*	|*	|*	|.\end{Puzzle}

\newpage

\begin{PuzzleClues}{\textbf{Poziome}\\}\Clue{3}{}{to, że coś jest skuteczne i pouczające; np. konstruktywność krytyki}
\Clue{6}{}{w chemii: symbol argonu}
\Clue{7}{}{urządzenie służące do krojenia produktów spożywczych}
\Clue{8}{}{część obręczy miedniczej, kość składająca się z trzonu i gałęzi}
\Clue{9}{}{familiarny, żartobliwy zwrot do znajomego}
\Clue{10}{}{melodia wygrywana przez mechanizmy pozytywkowe znajdujące się w zegarach lub innych przedmiotach (np. w tabakierze)}
\Clue{11}{}{kod ISO 4217 funta falklandzkiego}
\Clue{12}{}{W strofice: jeżeli ostatnia stopa wersu lub członu wersowego jest pełna, mówi się o akataleksie}
\Clue{13}{}{sportowiec zajmujący się spadochroniarstwem zawodowo lub hobbystycznie}
\Clue{19}{}{jednoroczna roślina zielna z lnowatych}
\Clue{24}{}{rzeźbiarz włoski (1380-1466) jeden z najwybitniejszych artystów wczesnego renesansu, przedstawiciel szkoły florenckiej}
\Clue{32}{}{kod ISO 4217 wona północnokoreańskiego}
\Clue{33}{}{szerokie okrycie wierzchnie bez rękawów zapinane z przodu}
\Clue{34}{}{miękka koźla skóra, używana zwykle na wierzchy luksusowego obuwia}
\Clue{36}{}{Homalozoa - jeden z podtypów, w jakich grupowane są szkarłupnie obejmujący wyłącznie organizmy kopalne}
\Clue{37}{}{piękny, przystojny, młody, wysportowany i silny młodzieniec}
\Clue{38}{}{XVIII-wieczny kielich do wina}
\Clue{39}{}{element nawierzchni kolejowej w postaci ułożonej pod szynami belki}
\Clue{40}{}{jednostka policyjna zajmująca się podczas śledztwa badaniami prowadzącymi do wykrycia przestępcy}
\Clue{41}{}{cienki papier nasączony atramentem}
\Clue{42}{}{kierunek studiów, który swoim zakresem obejmuje naukę o kulturze, literaturze i językach bałtyckich}
\Clue{43}{}{smutek i czułość w wyrazie}\end{PuzzleClues}

\begin{PuzzleClues}{\textbf{Pionowe}\\}\Clue{1}{}{koniec, zakończenie czegoś}
\Clue{2}{}{przebarwienie na błonie śluzowej jamy ustnej powstające pod wpływem przenikania składników wypełnień amalgatowych (rtęci i srebra) wgłąb tkanek miękkich i kości}
\Clue{3}{}{dziedzina sportu, latanie na lotniach}
\Clue{4}{}{niedopuszczenie się oszustwa; uczciwość w konkretnej sytuacji}
\Clue{5}{}{mieszkaniec Komańczy}
\Clue{6}{}{klasyfikacja kosztów prowadzenia działalności gospodarczej w tzw. pozycje kalkulacyjne niezbędne do obliczenia kosztu wytworzenia produktu: koszty pośrednie i koszty bezpośrednie}
\Clue{8}{}{sklep, w którym są sprzedawane przybory szkolne, artykuły papiernicze i biurowe}
\Clue{9}{}{Physeteridae - rodzina waleni z podrzędu zębowców}
\Clue{10}{}{Notodontidae - rodzina motyli, należąca do motyli nocnych, obejmująca 2,5-3,5 tys. gatunków, głównie tropikalnych; w Polsce występuje ponad 30}
\Clue{11}{}{dawniej: przymusowa dostawa żywności dla wojska}
\Clue{12}{}{temat, który jest poruszany jako dodatkowy, jest mniej istotny na tle całokształtu komunikacji; komentarz, dygresja, uwaga nie na temat}
\Clue{13}{}{przytomność, możliwość skoncentrowania się, brak senności}
\Clue{14}{}{obszerne pomieszczenie fabrycznie}
\Clue{15}{}{nurt pobożnościowy w protestantyzmie, oparty na specyficznej duchowości}
\Clue{16}{}{tysiąc dżuli}
\Clue{17}{}{Enigmosaurus - rodzaj wszystkożernego teropoda z rodziny terizinozaurów; żył w epoce późnej kredy na terenach centralnej Azji}
\Clue{18}{}{nieistnienie w historii, niewystępowanie w historii}
\Clue{19}{}{działo o bliskim płaskiemu torze lotu pocisku, służące do ostrzeliwania celów będących na linii pola widzenia}
\Clue{20}{}{Ailuroedus crassirostris - gatunek ptaka z rodziny altanników (Ptilonorhynchidae)}
\Clue{21}{}{w sztuce: m.in. literaturze, filmie: sposób przedstawiania, charakterystyczny dla realizmu}
\Clue{22}{}{mieszkanka Dominikany, kobieta pochodzenia dominikańskiego}
\Clue{23}{}{placówka szpiegowska}
\Clue{24}{}{ciężki, zimnokrwisty koń, używany do prac wykonywanych w wolnym tempie}
\Clue{25}{}{odegranie się na kimś}
\Clue{26}{}{cecha czynności, zachowań, wytworów ludzkich: swoboda obyczajowa; to, że coś jest nieprzyzwoite, ma nieprzyzwoitą wmowę, nieprzyzwoity ton}
\Clue{27}{}{twarda wierzchnia warstwa drogi lub ulicy}
\Clue{28}{}{końcowa część nogi człowieka}
\Clue{29}{}{Onopordum - rodzaj roślin należący do rodziny astrowatych}
\Clue{30}{}{instrument smyczkowy popularny w Polsce w okresie renesansu}
\Clue{31}{}{zawartość miedniczki, miski, która służy do mycia się, prania, zmywania}
\Clue{32}{}{Cyperus papyrus - gatunek rośliny bagiennej z rodziny ciborowatych, którą w starożytności uprawiano w celu pozyskania włókna używanego do wyrobu materiału pisarskiego}
\Clue{33}{}{fizyk niemiecki (1887-1975); przyczynił się do wyznaczania poziomów energetycznych atomów, laureat nagrody Nobla}
\Clue{35}{}{papuga o smuklej budowie i brązowym, połyskliwym upierzeniu; żywi się nektarem kwiatów; zamieszkuje lasy Australii i Polinezji}
\Clue{37}{}{nazwa literowa dźwięku, którego częstotliwość dla e1 wynosi 329,6 Hz}\end{PuzzleClues}\newpage%\section*{Krzyżówka 61}

\noindent\begin{Puzzle}{24}{28}|*	|*	|*	|[1][S]\darr	|*	|*	|[2][S]\drarr	|l	|i	|*	|[3][S]\drarr	|c	|z	|y	|n	|o	|w	|n	|i	|k	|*	|*	|*	|*	|[4][S]\darr	|.
|*	|*	|[5][S]\rarr	|d	|u	|l	|k	|a	|*	|[6][S]\rarr	|s	|y	|n	|a	|p	|s	|y	|d	|y	|*	|*	|*	|[7][S]\darr	|*	|e	|.
|*	|*	|[8][S]\rarr	|r	|a	|d	|a	|*	|*	|[9][S]\drarr	|m	|u	|ł	|y	|*	|[10][S]\drarr	|c	|i	|e	|ń	|*	|[11][S]\darr	|t	|[12][S]\darr	|r	|.
|*	|*	|[13][S]\darr	|z	|*	|[14][S]\drarr	|p	|r	|o	|m	|o	|t	|o	|r	|*	|w	|[15][S]\darr	|[16][S]\drarr	|p	|c	|*	|b	|w	|w	|b	|.
|*	|*	|q	|e	|*	|p	|u	|*	|*	|r	|l	|[17][S]\darr	|[18][S]\rarr	|g	|i	|a	|c	|o	|b	|i	|n	|i	|d	|y	|*	|.
|*	|[19][S]\darr	|u	|w	|*	|i	|s	|*	|*	|o	|e	|w	|*	|[20][S]\drarr	|p	|r	|a	|k	|t	|y	|k	|a	|*	|d	|*	|.
|*	|k	|a	|o	|*	|l	|t	|*	|[21][S]\darr	|*	|ń	|o	|*	|s	|*	|r	|n	|t	|[22][S]\drarr	|w	|e	|ł	|n	|a	|*	|.
|*	|o	|d	|ł	|[23][S]\darr	|n	|a	|[24][S]\darr	|b	|*	|s	|l	|*	|ł	|*	|e	|o	|e	|b	|[25][S]\darr	|*	|o	|[26][S]\darr	|t	|*	|.
|[27][S]\drarr	|p	|r	|a	|w	|o	|[][,]{ }	|h	|o	|o	|k	|e	|[][S]’	|a	|*	|n	|*	|t	|a	|c	|[28][S]\darr	|r	|g	|k	|*	|.
|t	|a	|i	|z	|y	|w	|w	|i	|ś	|*	|*	|*	|*	|w	|*	|*	|*	|*	|r	|u	|t	|u	|r	|i	|*	|.
|k	|r	|v	|[][,]{ }	|d	|a	|a	|t	|n	|[29][S]\rarr	|m	|e	|l	|o	|d	|y	|s	|t	|a	|*	|s	|s	|u	|[][,]{ }	|*	|.
|a	|k	|i	|j	|a	|c	|r	|*	|i	|[30][S]\rarr	|ż	|e	|r	|n	|i	|k	|i	|*	|n	|*	|u	|z	|p	|s	|*	|.
|n	|a	|u	|a	|t	|z	|z	|*	|a	|*	|[31][S]\drarr	|h	|e	|i	|d	|e	|g	|g	|e	|r	|*	|c	|a	|o	|*	|.
|i	|[][,]{ }	|m	|s	|e	|*	|y	|*	|c	|*	|g	|[32][S]\rarr	|t	|a	|k	|t	|*	|[33][S]\darr	|c	|[34][S]\darr	|*	|z	|[][,]{ }	|c	|*	|.
|n	|p	|*	|k	|k	|*	|w	|[35][S]\drarr	|k	|o	|l	|o	|s	|*	|[36][S]\darr	|*	|*	|ś	|z	|e	|[37][S]\darr	|y	|k	|j	|*	|.
|a	|o	|*	|r	|[][,]{ }	|*	|n	|s	|o	|*	|o	|[38][S]\rarr	|n	|i	|p	|*	|*	|w	|e	|r	|s	|z	|e	|a	|*	|.
|[][,]{ }	|d	|*	|a	|s	|*	|a	|h	|ś	|[39][S]\rarr	|b	|r	|o	|d	|a	|w	|n	|i	|k	|*	|e	|n	|t	|l	|*	|.
|s	|p	|*	|w	|o	|*	|[][,]{ }	|o	|ć	|*	|u	|[40][S]\rarr	|e	|i	|s	|l	|e	|r	|*	|*	|n	|a	|o	|n	|*	|.
|z	|o	|*	|o	|c	|*	|p	|u	|*	|[41][S]\rarr	|s	|r	|*	|*	|y	|[42][S]\rarr	|n	|o	|m	|o	|s	|*	|n	|e	|*	|.
|t	|z	|*	|u	|j	|*	|a	|j	|*	|*	|*	|*	|*	|*	|w	|[43][S]\rarr	|i	|n	|a	|*	|a	|*	|o	|*	|*	|.
|u	|i	|*	|d	|a	|*	|s	|o	|[44][S]\rarr	|k	|a	|r	|b	|o	|n	|*	|*	|e	|*	|*	|c	|[45][S]\darr	|w	|*	|*	|.
|c	|o	|*	|y	|l	|[46][S]\drarr	|t	|a	|ś	|m	|a	|[][,]{ }	|p	|r	|o	|d	|u	|k	|c	|y	|j	|n	|a	|*	|*	|.
|z	|m	|*	|*	|n	|ś	|e	|i	|*	|*	|*	|*	|[47][S]\darr	|*	|ś	|*	|*	|*	|*	|*	|a	|a	|*	|[48][S]\darr	|*	|.
|n	|o	|*	|*	|y	|n	|w	|*	|*	|[49][S]\rarr	|d	|y	|l	|*	|ć	|*	|*	|*	|*	|*	|*	|g	|*	|y	|*	|.
|a	|w	|*	|*	|*	|i	|n	|[50][S]\drarr	|d	|ż	|e	|m	|i	|k	|*	|*	|[51][S]\rarr	|l	|e	|h	|a	|r	|*	|b	|*	|.
|*	|a	|*	|[52][S]\rarr	|l	|e	|a	|s	|i	|n	|g	|[][,]{ }	|p	|r	|a	|c	|o	|w	|n	|i	|c	|z	|y	|*	|*	|.
|*	|*	|[53][S]\rarr	|z	|u	|g	|*	|d	|*	|*	|[54][S]\rarr	|c	|a	|p	|s	|t	|r	|z	|y	|k	|*	|e	|*	|*	|*	|.
|*	|*	|*	|*	|*	|*	|*	|g	|*	|*	|*	|*	|*	|*	|[55][S]\rarr	|k	|s	|i	|ę	|g	|o	|w	|y	|*	|*	|.
|*	|*	|*	|*	|*	|*	|*	|*	|*	|*	|*	|*	|*	|*	|*	|*	|*	|*	|*	|*	|*	|*	|*	|*	|*	|.\end{Puzzle}

\newpage

\begin{PuzzleClues}{\textbf{Poziome}\\}\Clue{2}{}{w chemii: symbol litu}
\Clue{3}{}{urzędnik państwowy w carskiej Rosji}
\Clue{5}{}{pierścień lub widełki służące do opierania wioseł w czasie wiosłowania}
\Clue{6}{}{Synapsida - grupa owodniowców powstałych w pensylwanie (ok. 318 - 299 mln lat temu) pod koniec karbonu; dominująca grupa kręgowców lądowych w permie i na początku triasu}
\Clue{8}{}{grupa ludzi, wybrana, żeby radzić, rządzić (np. rada starszych)}
\Clue{9}{}{mięśnie widoczne u osoby dobrze zbudowanej (najczęściej dwugłowe ramienia)}
\Clue{10}{}{duch, charakter czegoś, co odczuwalne, odbierane przez ludzi, ma lub miało wielką wagę}
\Clue{14}{}{substancja, która po dodaniu do katalizatora znacznie przyspiesza szybkość reakcji}
\Clue{16}{}{parsek - jednostka odległości używana w astronomii, odległość, dla której paralaksa roczna wynosi 1 sekundę łuku}
\Clue{18}{}{DRAKONIDY}
\Clue{20}{}{praca, zwłaszcza lekarza lub prawnika}
\Clue{22}{}{przędza, włóczka wełniana}
\Clue{27}{}{prawo mechaniki określające zależność odkształcenia od naprężenia, które głosi, że odkształcenie ciała pod wpływem działającej na nie siły jest proporcjonalne do tej siły}
\Clue{29}{}{kompozytor uwydatniający melodię w utworze muzycznym}
\Clue{30}{}{dzielnica Gliwic}
\Clue{31}{}{twórczość Heideggera, zbiór jego myśli i poglądów}
\Clue{32}{}{w muzyce - jednostka podziału metrycznego}
\Clue{35}{}{olbrzymi posąg nadnaturalnej wielkości, np. Kolos Rodyjski}
\Clue{38}{}{numer identyfikacji podatkowej, dziesięciocyfrowy kod służący do identyfikacji podatników w Polsce}
\Clue{39}{}{madagaskarski ptak z wróblowatych}
\Clue{40}{}{kompozytor niemiecki (1898-1962); autor muzyki do hymnu państwowego byłej NRD, współpracownik Brechta}
\Clue{41}{}{symbol steradiana - jednostki uzupełniającej układu SI określającej wartość kąta bryłowego}
\Clue{42}{}{tradycyjna melodia w starogreckiej muzyce}
\Clue{43}{}{rzeka polska}
\Clue{44}{}{materiał oparty na włóknach węglowych, włóknach karbonizowanych}
\Clue{46}{}{taśma, długi pasek na rolkach, przy którym odbywa się produkcja na skalę przemysłową}
\Clue{49}{}{gruba deska, także element betonowy lub gipsowy jako element ścienny stropowy}
\Clue{50}{}{porcja (najczęściej: słoik) dżemu}
\Clue{51}{}{kompozytor węgierski (1870-1948); przedstawiciel operetki wiedeńskiej 'Wesoła wdówka', 'Kraina uśmiechu'}
\Clue{52}{}{wynajem pracowników, z którymi został zawarty stosunek pracy na rzecz innych podmiotów gospodarczych}
\Clue{53}{}{jezioro w Szwajcarii w dorzeczu rzeki Reuss, powierzchnia 38 km2}
\Clue{54}{}{wieczorny przemarsz wojska ulicami miasta}
\Clue{55}{}{pracownik działu księgowości, osoba zajmująca się wszelkimi czynnościami związanymi z prowadzeniem ksiąg rachunkowych podmiotów gospodarczych}\end{PuzzleClues}

\begin{PuzzleClues}{\textbf{Pionowe}\\}\Clue{1}{}{Allobates femoralis syn. Epipedobates femoralis - płaz z rodziny Aromobatidae, zamieszkujący subtropikalne lub tropikalne wilgotne lasy deszczowe i bagna o świeżej wodzie w Boliwii, Brazylii, Kolumbii, Ekwadorze, Gujanie Francuskiej, Gujanie, Peru i Surinamie}
\Clue{2}{}{Brassica oleracea var. medullosa - odmiana kapusty warzywnej; jest to roślina dwuletnia należąca do rodziny kapustowatych}
\Clue{3}{}{miasto w Rosji, stolica obwodu smoleńskiego}
\Clue{4}{}{niemiecki internista (1840-1921); metoda elektrodiagnostyki}
\Clue{7}{}{kod ISO 4217 dolara tajwańskiego}
\Clue{9}{}{kod ISO 4217 waluty ugija}
\Clue{10}{}{(1905-89), amerykański pisarz i krytyk literacki, powieści o problematyce społecznej; „Gubernator”, „Nocny jeździec”}
\Clue{11}{}{stosowane rzadko określenie zapożyczeń (wyrazów, składni, zwrotów) z języka białoruskiego}
\Clue{12}{}{wydatki przeznaczane na świadczenia społeczne: ubezpieczenia, pomoc społeczną, świadczenia rodzinne i wydatki związane z polityką rynku pracy; także wydatki na funkcjonowanie instytucji publicznych zarządzających tymi świadczeniami}
\Clue{13}{}{grupa przedmiotów wykładanych w szkołach średniowiecznych; cztery spośród siedmiu sztuk wyzwolonych: geometria, arytmetyka, muzyka i astronomia}
\Clue{14}{}{człowiek, który zajmuje się pilnowaniem, doglądaniem, dozorowaniem czegoś}
\Clue{15}{}{(1460-1526) żeglarz hiszpański, uczestnik w wyprawie Magellana}
\Clue{16}{}{zgrupowanie ośmiu cząstek elementarnych}
\Clue{17}{}{powiększenie tarczycy}
\Clue{19}{}{koparka przystosowana do pracy poniżej poziomu, na którym jest ustawiona}
\Clue{20}{}{kraina historyczna w Chorwacji, między Drawią a Sawą}
\Clue{21}{}{zespół cech czegoś lub kogoś takiego jak w Bośni i Hercegowinie, także: stereotypowe cechy uznawane za właściwe Bośniakom}
\Clue{22}{}{zdrobniale: baranek - biała chmura, zwykle mała, pierzasta lub kłębiasta, konotowana pozytywnie}
\Clue{23}{}{kwota pieniężna przeznaczane na świadczenia społeczne: ubezpieczenia, pomoc społeczną, świadczenia rodzinne i wydatki związane z polityką rynku pracy; także pieniądze przeznaczane na funkcjonowanie instytucji publicznych zarządzających tymi świadczeniami}
\Clue{24}{}{przebojowa piosenka, szlagier, przebój}
\Clue{25}{}{w chemii: symbol miedzi}
\Clue{26}{}{grupa funkcyjna występująca w wielu typach związków organicznych, składająca się z atomu węgla połączonego wiązaniem podwójnym z atomem tlenu}
\Clue{27}{}{tkanina wykonana z włókien syntetycznych}
\Clue{28}{}{miasto w Japonii (środkowe Honsiu), ośrodek administracyjny prefektury Mie, nad Oceanem Spokojnym}
\Clue{31}{}{kulisty model Ziemi, sfery niebieskiej albo też innego ciała niebieskiego}
\Clue{33}{}{w dawnych folwarkach - mały spichlerz}
\Clue{34}{}{w chemii: symbol erbu}
\Clue{35}{}{anime o zabarwieniu romantycznym, w którym bohaterkami są młode dziewczęta}
\Clue{36}{}{cecha działania, zachowania: to, że coś pozbawione jest aktywności, energii}
\Clue{37}{}{wiadomość, która jest zaskakująca, niespodziewana, wywołuje wielkie zainteresowanie}
\Clue{45}{}{osiąganie wysokiej temperatury, nagrzewanie się}
\Clue{46}{}{substancja, która jest wynikiem opadów śniegu}
\Clue{47}{}{kiedy plan nie wypali lub kiedy jest nudno}
\Clue{48}{}{jednostka informacji w systemie dwójkowym, oznaczająca 2\textasciicircum80 bajtów}
\Clue{50}{}{kod ISO 4217 funta sudańskiego}\end{PuzzleClues}\newpage%\section*{Krzyżówka 62}

\noindent\begin{Puzzle}{22}{31}|*	|*	|*	|*	|*	|*	|*	|*	|*	|*	|*	|*	|*	|*	|*	|[1][S]\drarr	|s	|h	|a	|u	|l	|a	|*	|.
|*	|*	|*	|[2][S]\darr	|*	|*	|*	|[3][S]\rarr	|c	|z	|a	|r	|n	|y	|[][,]{ }	|p	|i	|o	|t	|r	|u	|ś	|*	|.
|*	|*	|[4][S]\rarr	|f	|r	|y	|z	|e	|*	|*	|*	|[5][S]\drarr	|h	|u	|m	|o	|r	|*	|*	|*	|*	|*	|*	|.
|*	|*	|*	|u	|[6][S]\darr	|[7][S]\darr	|[8][S]\darr	|[9][S]\darr	|*	|*	|*	|u	|[10][S]\rarr	|a	|m	|b	|r	|a	|z	|u	|r	|a	|*	|.
|*	|*	|[11][S]\drarr	|n	|a	|m	|o	|r	|d	|n	|i	|k	|*	|[12][S]\rarr	|m	|o	|s	|p	|a	|n	|*	|[13][S]\darr	|*	|.
|*	|*	|g	|k	|k	|a	|w	|o	|[14][S]\rarr	|s	|i	|ł	|a	|*	|*	|ż	|[15][S]\drarr	|z	|a	|m	|e	|k	|*	|.
|*	|*	|a	|c	|s	|l	|o	|z	|[16][S]\drarr	|p	|r	|a	|w	|o	|z	|n	|a	|w	|s	|t	|w	|o	|*	|.
|*	|*	|w	|j	|j	|i	|d	|k	|r	|[17][S]\darr	|*	|d	|[18][S]\darr	|*	|[19][S]\darr	|o	|m	|[20][S]\darr	|*	|[21][S]\darr	|*	|m	|*	|.
|*	|*	|r	|a	|o	|n	|n	|o	|f	|h	|[22][S]\darr	|[][,]{ }	|l	|[23][S]\darr	|b	|ś	|o	|s	|[24][S]\darr	|s	|[25][S]\darr	|p	|*	|.
|*	|*	|o	|[][,]{ }	|m	|a	|i	|l	|*	|a	|t	|s	|u	|s	|r	|ć	|n	|k	|s	|y	|k	|i	|*	|.
|*	|*	|n	|p	|a	|*	|o	|e	|*	|r	|y	|ł	|f	|e	|z	|*	|i	|r	|a	|g	|o	|l	|*	|.
|*	|*	|*	|r	|t	|*	|w	|c	|[26][S]\rarr	|p	|r	|o	|t	|r	|y	|p	|t	|y	|l	|i	|n	|a	|*	|.
|*	|*	|[27][S]\darr	|z	|[][,]{ }	|*	|c	|*	|*	|i	|s	|n	|*	|[][,]{ }	|d	|[28][S]\darr	|*	|b	|d	|e	|t	|t	|*	|.
|*	|[29][S]\rarr	|d	|e	|p	|r	|e	|s	|j	|a	|*	|e	|*	|s	|a	|c	|*	|a	|o	|t	|r	|o	|*	|.
|*	|[30][S]\darr	|r	|s	|a	|*	|*	|*	|*	|*	|*	|c	|[31][S]\darr	|z	|l	|h	|[32][S]\darr	|*	|[][,]{ }	|y	|o	|r	|*	|.
|*	|c	|z	|t	|s	|*	|*	|*	|*	|*	|*	|z	|k	|w	|*	|l	|p	|*	|c	|ń	|l	|*	|*	|.
|*	|o	|e	|ę	|c	|[33][S]\rarr	|k	|o	|n	|t	|y	|n	|u	|a	|t	|o	|r	|*	|y	|s	|a	|*	|*	|.
|*	|m	|w	|p	|h	|*	|*	|*	|*	|*	|*	|y	|n	|j	|*	|r	|e	|[34][S]\darr	|k	|k	|[][,]{ }	|*	|*	|.
|*	|p	|o	|n	|a	|*	|[35][S]\drarr	|p	|u	|l	|a	|*	|a	|c	|[36][S]\darr	|e	|w	|c	|l	|i	|p	|*	|*	|.
|*	|t	|[][,]{ }	|a	|*	|[37][S]\rarr	|ż	|ą	|d	|z	|a	|*	|*	|a	|m	|l	|e	|y	|i	|*	|a	|*	|*	|.
|*	|o	|p	|*	|*	|[38][S]\drarr	|w	|o	|l	|e	|*	|*	|*	|r	|a	|l	|n	|n	|c	|*	|r	|*	|*	|.
|[39][S]\drarr	|n	|i	|e	|b	|i	|a	|n	|i	|n	|*	|*	|*	|s	|j	|a	|c	|i	|z	|*	|z	|*	|*	|.
|m	|*	|e	|[40][S]\darr	|*	|s	|w	|*	|*	|*	|*	|*	|*	|k	|s	|*	|j	|a	|n	|*	|y	|[41][S]\darr	|*	|.
|a	|[42][S]\drarr	|c	|z	|e	|k	|o	|l	|a	|d	|a	|[][,]{ }	|p	|i	|t	|n	|a	|*	|e	|*	|s	|l	|*	|.
|l	|l	|z	|a	|*	|r	|ś	|[43][S]\rarr	|k	|a	|t	|o	|n	|*	|e	|*	|*	|*	|*	|[44][S]\rarr	|t	|i	|*	|.
|e	|e	|ę	|k	|*	|ó	|ć	|*	|[45][S]\drarr	|s	|k	|u	|m	|b	|r	|i	|a	|*	|*	|*	|o	|n	|*	|.
|ń	|n	|c	|o	|*	|w	|*	|[46][S]\rarr	|c	|m	|o	|k	|n	|o	|n	|s	|e	|n	|s	|*	|ś	|*	|*	|.
|s	|d	|i	|n	|*	|k	|*	|*	|z	|*	|*	|*	|*	|[47][S]\rarr	|i	|d	|e	|a	|*	|*	|c	|*	|*	|.
|t	|l	|o	|*	|*	|a	|[48][S]\rarr	|z	|a	|t	|r	|o	|s	|k	|a	|n	|i	|e	|*	|*	|i	|*	|*	|.
|w	|e	|w	|*	|*	|*	|[49][S]\rarr	|z	|r	|o	|ś	|l	|a	|k	|*	|[50][S]\rarr	|i	|r	|o	|n	|*	|*	|*	|.
|o	|r	|e	|*	|*	|[51][S]\rarr	|k	|w	|a	|s	|[][,]{ }	|h	|u	|m	|u	|s	|o	|w	|y	|*	|*	|*	|*	|.
|*	|*	|*	|*	|[52][S]\rarr	|w	|a	|z	|*	|*	|*	|*	|*	|*	|*	|*	|*	|*	|*	|*	|*	|*	|*	|.\end{Puzzle}

\newpage

\begin{PuzzleClues}{\textbf{Poziome}\\}\Clue{1}{}{jedna z gwiazd w gwiazdozbiorze Skorpiona}
\Clue{3}{}{czynnik przyczyniający się do przegranej, przynoszący pecha w rozgrywce; najczęściej w sporcie}
\Clue{4}{}{elektrotechnik (1885-1964); prace z teorii obwodów elektrycznych}
\Clue{5}{}{jeden z dojrzałych mechanizmów obronnych, który polega na otwartej ekspresji uczuć bez osobistego dyskomfortu i nieprzyjemnych efektów dla innych}
\Clue{10}{}{futryna, framuga okienna lub drzwiowa}
\Clue{11}{}{druciana lub skórzana plecionka zakładana zwierzętom (zwłaszcza psom) na pysk, aby uniemożliwić im kąsanie}
\Clue{12}{}{używany poufale tytuł grzecznościowy odpowiadający dzisiejszemupan, skrót od mości pan}
\Clue{14}{}{pewien parametr, natężenie jakiegoś zjawiska}
\Clue{15}{}{ruchoma część odtylcowej broni palnej}
\Clue{16}{}{prawo jako nauka, nauka o prawie}
\Clue{26}{}{trójpierścieniowy lek przeciwdepresyjny, amina II-rzędowa, silny inhibitor wychwytu zwrotnego noradrenaliny; stosowany w leczeniu depresji z zahamowaniem psychoruchowym bez cech lęku i niepokoju}
\Clue{29}{}{zaburzenie charakteryzujące się obniżonym nastrojem, niemożnością odczuwania przyjemności, uczuciem zmęczenia, bezsilności; stanowi samodzielne zaburzenie lub część choroby psychicznej}
\Clue{33}{}{ten, kto prowadzi pracę rozpoczętą przez poprzedników}
\Clue{35}{}{waluta Botswany}
\Clue{37}{}{silne odczucie fizjologiczne i psychiczne, związane z niezaspokojeniem jakiejś potrzeby, popychające często do nieracjonalnych czy nieodpowiedzialnych działań}
\Clue{38}{}{rozszerzenie przełyku niektórych ptaków służące do magazynowania pokarmu, rozmiękczania go i wstępnego trawienia}
\Clue{39}{}{szczęśliwy mieszkaniec nieba starożytnych mitologii, bóg bądź deifikowany heros}
\Clue{42}{}{napój powstały z mieszanki śmietanki, mleka i - w wersji klasycznej - czekolady gorzkiej, spożywany przeważnie na gorąco}
\Clue{43}{}{człowiek przypominający charakterem Katona - bardzo surowy, odznaczający się bezkompromisowością, wymagający od innych i od siebie, wierny prawu}
\Clue{44}{}{w chemii: symbol tytanu}
\Clue{45}{}{inna nazwa makreli}
\Clue{46}{}{pocałunek w rękę - pogardliwie}
\Clue{47}{}{głębsza myśl, rozbudowany pomysł, zazwyczaj dotyczący jakiejś ważnej sprawy}
\Clue{48}{}{troska, niepokój o coś lub o kogoś}
\Clue{49}{}{bliźniak urodzony w parze bliźniąt nierozdzielonych}
\Clue{50}{}{rodzaj kija golfowego służącego do wbijania piłki do dołka}
\Clue{51}{}{związek organiczny, będący częścią próchnicy glebowej i roztworów wód naturalnych, o różnym składzie, uzależnionym od składu materii, z której powstał}
\Clue{52}{}{gad łuskonośny beznogi, drapieżny, szeroko rozpowszechniony poza strefą polarną}\end{PuzzleClues}

\begin{PuzzleClues}{\textbf{Pionowe}\\}\Clue{1}{}{praktyki religijne, sposób oddawania czci religijnej}
\Clue{2}{}{funkcja, która nie jest funkcją algebraiczną}
\Clue{5}{}{układ planetarny, podobny do naszego, w którym ośrodkiem jest centralna gwiazda}
\Clue{6}{}{założenie, że prosta na płaszczyźnie, która nie przechodzi przez żaden z wierzchołków trójkąta i przecina jeden jego bok, przecina jeszcze drugi}
\Clue{7}{}{grupa gatunków roślin z rodzaju Rubus wyróżniana zwyczajowo ze względu na czerwoną barwę owoców i łatwość ich odpadania od dna kwiatowego}
\Clue{8}{}{Amniota - klad obejmujący kręgowce mające zdolność rozwoju zarodkowego na lądzie (gady, ptaki i ssaki); uzyskały ją dzięki wytworzeniu błon płodowych, które tworzą środowisko dla właściwego rozwoju zarodka}
\Clue{9}{}{ślimak ciepłych mórz o muszli z kolcem; z wydzieliny płaszcza otrzymywano purpurę}
\Clue{11}{}{GAPA; ptak z rodziny krukowatych o czarnym upierzeniu, wszystkożerny, długości około 50 cm}
\Clue{13}{}{program służący do automatycznego tłumaczenia kodu napisanego w jednym języku (języku źródłowym) na równoważny kod w innym języku (języku wynikowym)}
\Clue{15}{}{głowonóg z podgromady amonitów}
\Clue{16}{}{symbol oznaczający rutherford}
\Clue{17}{}{kobieta okrutna, dręczycielka, potwór}
\Clue{18}{}{przewód w piecu, w kominie, przez który był odprowadzany dym; dzisiaj we frazeologizmiedo luftu}
\Clue{19}{}{ktoś brzydki lub coś brzydkiego}
\Clue{20}{}{żartobliwie lub ironicznie o człowieku, który trudni się pisaniem (określenie podrzędnego literata)}
\Clue{21}{}{kompozytor (1896-1955); twórca Państwowego Zespołu Pieśni i Tańca 'Mazowsze'; opery-balety; 'Karczma na rozdrożu'}
\Clue{22}{}{ornament w formie laski oplecionej liśćmi}
\Clue{23}{}{odmiana sera żółtego}
\Clue{24}{}{różnica między wydatkami i dochodami państwa, realizowanymi w warunkach, gdy gospodarka nie funkcjonuje przy pełnym wykorzystaniu mocy produkcyjnych}
\Clue{25}{}{metoda wykrywania przełamań w transmitowanych wiadomościach, polegająca na dodaniu do wiadomości bitu kontrolnego}
\Clue{27}{}{Sigillaria - zwyczajowa nazwa sygilarii}
\Clue{28}{}{jednokomórkowy glon żyjący w planktonie słodkowodnym}
\Clue{30}{}{fizyk amerykański (1892-1962); prace z dziedziny fizyki jądrowej i atomowej, laureat nagrody Nobla}
\Clue{31}{}{waluta Chorwacji}
\Clue{32}{}{zespół działań mających zapobiec niepożądanym skutkom}
\Clue{34}{}{JAKOBINKA, CYNKA roślina roczna lub podkrzew z rodziny złożonych}
\Clue{35}{}{cecha człowieka przejawiająca się w działaniu; energiczność, żywość}
\Clue{36}{}{warsztat rzemieślniczy}
\Clue{38}{}{depesza radiotelegraficzna}
\Clue{39}{}{małe, malutkie dziecko}
\Clue{40}{}{zgromadzenie religijne, którego członkowie związani są z nim ślubami uroczystymi, powstałe najpóźniej w 1535 roku}
\Clue{41}{}{ryba z rodziny karpiowatych hodowana w stawach o długości do 45 cm}
\Clue{42}{}{utwór do tańczenia lendlera}
\Clue{45}{}{tyle, ile mieści się w czarze - kielichu}\end{PuzzleClues}\newpage%\section*{Krzyżówka 63}

\noindent\begin{Puzzle}{18}{31}|*	|[1][S]\darr	|*	|*	|*	|*	|*	|*	|*	|[2][S]\darr	|*	|*	|*	|[3][S]\darr	|*	|[4][S]\darr	|*	|*	|*	|.
|*	|s	|[5][S]\drarr	|w	|o	|ł	|g	|o	|g	|r	|a	|d	|*	|p	|*	|p	|*	|*	|[6][S]\darr	|.
|*	|z	|v	|[7][S]\rarr	|d	|i	|o	|p	|t	|e	|r	|*	|*	|r	|*	|e	|[8][S]\darr	|*	|t	|.
|[9][S]\drarr	|n	|i	|p	|i	|g	|o	|n	|*	|c	|[10][S]\darr	|*	|*	|a	|[11][S]\darr	|r	|m	|*	|t	|.
|a	|u	|p	|[12][S]\rarr	|r	|o	|s	|o	|ł	|e	|k	|*	|[13][S]\darr	|w	|i	|m	|o	|*	|d	|.
|t	|r	|*	|[14][S]\drarr	|r	|z	|e	|c	|z	|p	|o	|s	|p	|o	|l	|i	|t	|a	|*	|.
|*	|*	|[15][S]\darr	|m	|[16][S]\darr	|*	|*	|*	|*	|t	|ń	|*	|i	|[][,]{ }	|o	|s	|o	|[17][S]\darr	|*	|.
|[18][S]\drarr	|q	|u	|i	|n	|e	|t	|*	|*	|a	|[][,]{ }	|*	|c	|p	|c	|y	|r	|a	|*	|.
|w	|*	|k	|r	|i	|[19][S]\rarr	|c	|y	|f	|r	|a	|*	|o	|r	|z	|w	|*	|g	|[20][S]\darr	|.
|y	|*	|ł	|*	|m	|*	|*	|*	|[21][S]\darr	|i	|n	|*	|n	|y	|y	|n	|[22][S]\darr	|r	|k	|.
|t	|*	|a	|*	|f	|*	|*	|*	|ś	|u	|g	|*	|*	|w	|n	|o	|ś	|o	|o	|.
|r	|*	|d	|*	|a	|*	|*	|*	|w	|s	|l	|*	|*	|a	|[][,]{ }	|ś	|w	|w	|m	|.
|z	|*	|[][,]{ }	|*	|*	|[23][S]\rarr	|a	|v	|i	|z	|o	|*	|[24][S]\darr	|t	|t	|ć	|i	|ł	|e	|.
|e	|*	|p	|*	|*	|*	|[25][S]\darr	|[26][S]\darr	|a	|*	|a	|*	|a	|n	|e	|*	|s	|ó	|r	|.
|s	|*	|i	|*	|*	|*	|p	|d	|t	|*	|r	|[27][S]\darr	|w	|e	|n	|*	|t	|k	|a	|.
|z	|*	|a	|*	|*	|[28][S]\darr	|a	|y	|*	|*	|a	|d	|i	|*	|s	|[29][S]\darr	|e	|n	|ż	|.
|c	|*	|s	|*	|*	|s	|n	|s	|[30][S]\darr	|[31][S]\rarr	|b	|e	|z	|l	|o	|t	|k	|i	|*	|.
|z	|*	|e	|*	|*	|n	|t	|p	|k	|*	|s	|k	|o	|*	|r	|r	|*	|n	|*	|.
|k	|*	|c	|*	|*	|e	|e	|l	|a	|*	|k	|o	|*	|*	|o	|a	|*	|a	|*	|.
|a	|*	|k	|[32][S]\drarr	|s	|e	|r	|a	|p	|h	|i	|n	|e	|*	|w	|c	|*	|*	|[33][S]\darr	|.
|[][,]{ }	|*	|i	|m	|[34][S]\drarr	|m	|a	|z	|u	|r	|*	|c	|*	|*	|y	|z	|*	|*	|c	|.
|t	|*	|e	|a	|k	|*	|[][,]{ }	|j	|c	|[35][S]\rarr	|r	|e	|s	|t	|*	|k	|*	|[36][S]\darr	|u	|.
|a	|*	|g	|t	|u	|[37][S]\darr	|ś	|a	|y	|[38][S]\rarr	|k	|n	|o	|t	|*	|a	|*	|c	|d	|.
|r	|*	|o	|e	|r	|r	|n	|*	|n	|[39][S]\rarr	|e	|t	|k	|i	|n	|*	|*	|l	|o	|.
|c	|*	|*	|*	|a	|o	|i	|*	|*	|[40][S]\rarr	|f	|r	|a	|z	|e	|r	|*	|a	|t	|.
|z	|*	|*	|[41][S]\rarr	|ś	|l	|e	|p	|y	|[][,]{ }	|z	|a	|u	|ł	|e	|k	|*	|r	|w	|.
|ó	|*	|*	|*	|*	|m	|ż	|*	|[42][S]\rarr	|w	|s	|c	|h	|ó	|d	|*	|*	|k	|ó	|.
|w	|[43][S]\rarr	|b	|a	|l	|o	|n	|*	|[44][S]\drarr	|r	|a	|j	|d	|ó	|w	|k	|a	|*	|r	|.
|k	|*	|*	|*	|*	|p	|a	|[45][S]\rarr	|r	|i	|t	|a	|r	|d	|a	|n	|d	|o	|*	|.
|a	|*	|[46][S]\rarr	|p	|a	|s	|*	|*	|ó	|*	|*	|*	|*	|*	|*	|*	|*	|*	|*	|.
|*	|*	|*	|*	|*	|*	|*	|[47][S]\rarr	|g	|u	|a	|n	|a	|b	|a	|n	|a	|*	|*	|.
|*	|*	|*	|*	|*	|*	|*	|*	|*	|*	|*	|*	|*	|*	|*	|*	|*	|*	|*	|.\end{Puzzle}

\newpage

\begin{PuzzleClues}{\textbf{Poziome}\\}\Clue{5}{}{miasto obwodowe w Rosji, nad dolną Wołgą}
\Clue{7}{}{przyrząd stosowany w strzelectwie zwiększający precyzję strzelania}
\Clue{9}{}{jezioro w Kanadzie, powierzchnia 4,8 tyś. km2, głębokość do 165 m, rzeką Nipign połączone z Jeziorem Górnym}
\Clue{12}{}{zdrobniale: rosół - niezagęszczana (klarowna) zupa na wywarze mięsno-warzywnym}
\Clue{14}{}{staropolskie określenie państwa o republikańskim ustroju politycznym}
\Clue{18}{}{(1803-75), francuski pisarz, historyk, polityk „.Napoleon”}
\Clue{19}{}{inicjał - pierwsza litera imienia lub nazwiska}
\Clue{23}{}{zawiadomienie o nadejściu przesyłki pocztowej, której nie można doręczyć adresatowi bezpośrednio, informujące adresata o możliwości odbioru przesyłki we wskazanym urzędzie pocztowym (lub innej instytucji) oraz do kiedy będzie to możliwe; jest to niepoprawny ortograficznie zapis tego rzeczownika}
\Clue{31}{}{pingwiny, Spheniscidae - rodzina wodnych ptaków nielatających z rzędu pingwinów (Sphenisciformes)}
\Clue{32}{}{inst. klawiszowy, prototyp fisharmonii}
\Clue{34}{}{polski taniec ludowy, w żywym tempie i metrum 3/4 lub 3/8}
\Clue{35}{}{przyrząd bilardowy, który pomaga trafiać w bilę i projektować uderzenie; rodzaj podstawki pod kij bilardowy, wysięgnik z rodzajem celownika, który stanowi miejsce oparcia dla końcówki kija}
\Clue{38}{}{mały chłopiec, brzdąc}
\Clue{39}{}{pianista, laureat Konkursu im F. Chopina w 1927 r}
\Clue{40}{}{angielski etnolog i filolog (1854-1941); przedstawiciel ewolucjonizmu}
\Clue{41}{}{ulica, z której nie ma wyjścia}
\Clue{42}{}{jedna z czterech głównych stron świata; odnosi się do punktu horyzontu, w którym wschodzi Słońce w dniu równonocy}
\Clue{43}{}{aerostat bez napędu silnikowego}
\Clue{44}{}{samochód, który jest przystosowany do jeżdżenia nim w rajdach}
\Clue{45}{}{określenie wykonawcze; zwalniając, opóźniając}
\Clue{46}{}{długi, wąski kawałek skóry, tkaniny lub innego materiału, część garderoby, podtrzymująca w pasie spodnie lub spódnicę lub noszona dla ozdoby}
\Clue{47}{}{Annona muricata, flaszowiec miękkociernisty - gatunek małego flaszowca, który ma jadalne i bardzo smaczne owoce}\end{PuzzleClues}

\begin{PuzzleClues}{\textbf{Pionowe}\\}\Clue{1}{}{przewód izolowany z giętkimi żyłami}
\Clue{2}{}{zbiór zawierający recepty lub przepisy}
\Clue{3}{}{jedna z dwóch podstawowych gałęzi prawa (obok prawa publicznego), skupiająca normy prawne, których zadaniem jest ochrona interesu jednostek i regulacja stosunków pomiędzy nimi}
\Clue{4}{}{postawa przyzwolenia na czyny dewiacyjne czy nawet przestępcze bez wymierzania sankcji lub wymierzania sankcji niewspółmiernie łagodnych; wyróżniamy permisywność społeczną, seksualną, terapeutyczną}
\Clue{5}{}{sława, osobistość, także autorytet lub ktoś na stanowisku, ale wtedy gdy jest zaproszonym gościem i ma uświetnić uroczystość}
\Clue{6}{}{kod ISO 4217 dolara Trynidadu i Tobago}
\Clue{8}{}{osoba (również: grupa), bez której nie może się powieść jakieś przedsięwzięcie, ponieważ jest jego organizatorem, inicjatorem, mobilizuje innych do działania}
\Clue{9}{}{jednostka zdawkowa w Laosie; 1/100 kipa}
\Clue{10}{}{angloarab - jedna z ras koni gorącokrwistych, pochodząca od konia angielskiego skrzyżowanego z koniem arabskim, hodowana głównie we Francji, Wielkiej Brytanii oraz w Polsce; cechuje ją gorący temperament i inteligencja, dzieki czemu odnosi ogromne sukcesy w sporcie jeździeckim}
\Clue{11}{}{przestrzeń wektorowa mająca własność uniwersalnej faktoryzacji}
\Clue{13}{}{(1917-76), francuski eseista i krytyk, prace monograficzne, eseje, antologie}
\Clue{14}{}{radziecka stacja kosmiczna nowej generacji, umożliwia przyłączenie 6 statków}
\Clue{15}{}{układ wirników w śmigłowcu, charakteryzujący się umieszczeniem wirników w układzie tandemowym, poziomo i wzdłużnie, zamontowany jeden przy drugim, najczęściej blisko końców kadłuba. Każdy z wirników porusza się w przeciwnym kierunku}
\Clue{16}{}{ostatnie stadium larwalne owadów przechodzących przeobrażenie niezupełne (hemimetabolia) i niektórych pajęczaków, objawiające się obecnością zawiązków skrzydeł}
\Clue{17}{}{rodzaj tworzywa w postaci grubej włókniny, którego używa się w rolnictwie i ogrodnictwie do osłaniania i ochrony roślin (np. przed temperaturą, szkodnikami) lub gleby i podłoża (np. przed waunkami atmosferycznymi, pojawianiem się chwastów); używana również w budownictwie}
\Clue{18}{}{Astata boops - gatunek owada z rodziny grzebaczowatych}
\Clue{20}{}{potajemny układ, w który się wchodzi w celu osiągnięcia jakiegoś celu; zwykle w lm}
\Clue{21}{}{ogół osób, rzeczy lub zjawisk}
\Clue{22}{}{niewielki kawałek papieru}
\Clue{24}{}{maty okręt wojenny używany do służby patrolowej i pomocniczej}
\Clue{25}{}{gatunek drapieżnego ssaka z rodziny kotowatych, występujący na terenach Azji Środkowej}
\Clue{26}{}{wada wrodzona, polegająca na zaburzeniu pracy komórek}
\Clue{27}{}{proces psychiczny polegający na niemożności utrzymania przy czymś uwagi}
\Clue{28}{}{miasto w Irlandii nad Zatoką Kenmare}
\Clue{29}{}{kobieta, która tarła len, by z łodyg uzyskać miękkie i długie włókna}
\Clue{30}{}{członek zakonu katolickiego o ostrej regule założonego w 1528 r jako odłam franciszkanów}
\Clue{32}{}{naczynie do picia yerba mate}
\Clue{33}{}{dawne określenie potwora, niezwykłego i strasznego stwora}
\Clue{34}{}{(1871-1929), poeta ludowy, zbiory wierszy; „Dzwoń chłopska pieśni”, „Przez ciernie żywota”}
\Clue{36}{}{archeolog brytyjski ur. w 1907 r.; badacz pradziejów Europy i Anglii}
\Clue{37}{}{filet śledziowy z przyprawami zwinięty w rulonik i zamarynowany}
\Clue{44}{}{rożek, który wyrasta na ciele niektórych bezkręgowców}\end{PuzzleClues}\newpage%\section*{Krzyżówka 64}

\noindent\begin{Puzzle}{21}{23}|*	|*	|[1][S]\darr	|*	|*	|*	|*	|*	|*	|*	|[2][S]\darr	|*	|*	|[3][S]\drarr	|s	|k	|i	|p	|[][,]{ }	|a	|*	|[4][S]\darr	|.
|*	|*	|s	|*	|*	|[5][S]\darr	|*	|*	|[6][S]\darr	|*	|ż	|*	|*	|d	|[7][S]\drarr	|s	|p	|l	|o	|t	|*	|h	|.
|*	|[8][S]\darr	|l	|*	|[9][S]\darr	|m	|[10][S]\darr	|*	|k	|*	|ó	|[11][S]\rarr	|s	|z	|p	|a	|l	|e	|r	|*	|*	|[][S]ä	|.
|*	|g	|o	|*	|d	|e	|w	|*	|u	|*	|ł	|*	|*	|i	|r	|*	|*	|*	|*	|*	|[12][S]\darr	|n	|.
|*	|a	|t	|*	|y	|d	|j	|*	|t	|*	|t	|*	|[13][S]\rarr	|k	|o	|c	|i	|o	|ł	|*	|m	|d	|.
|[14][S]\drarr	|w	|y	|ś	|m	|i	|e	|n	|i	|t	|o	|ś	|ć	|*	|p	|*	|*	|*	|[15][S]\darr	|*	|g	|e	|.
|k	|o	|*	|[16][S]\darr	|n	|a	|l	|[17][S]\rarr	|a	|t	|l	|a	|n	|t	|y	|d	|a	|*	|w	|*	|i	|l	|.
|o	|r	|[18][S]\drarr	|s	|i	|n	|a	|n	|*	|*	|i	|[19][S]\rarr	|h	|a	|l	|f	|[][,]{ }	|p	|i	|p	|e	|*	|.
|e	|z	|k	|p	|k	|a	|*	|*	|*	|[20][S]\darr	|c	|*	|[21][S]\darr	|*	|*	|*	|*	|*	|z	|*	|l	|[22][S]\darr	|.
|d	|e	|r	|e	|*	|*	|*	|*	|*	|p	|z	|[23][S]\rarr	|c	|o	|z	|i	|a	|*	|j	|[24][S]\darr	|n	|d	|.
|u	|n	|a	|c	|*	|*	|*	|[25][S]\rarr	|p	|i	|k	|n	|i	|k	|*	|[26][S]\darr	|*	|*	|a	|k	|i	|e	|.
|k	|i	|k	|j	|*	|*	|[27][S]\rarr	|s	|k	|r	|a	|w	|a	|l	|n	|o	|ś	|ć	|*	|i	|c	|j	|.
|a	|e	|o	|a	|[28][S]\darr	|*	|*	|[29][S]\darr	|[30][S]\darr	|a	|*	|[31][S]\darr	|ł	|[32][S]\drarr	|a	|b	|r	|a	|z	|j	|a	|*	|.
|c	|*	|w	|c	|b	|*	|*	|o	|o	|t	|*	|k	|k	|n	|*	|l	|[33][S]\darr	|[34][S]\darr	|[35][S]\darr	|*	|*	|*	|.
|y	|*	|i	|j	|o	|[36][S]\darr	|*	|k	|j	|*	|[37][S]\darr	|a	|o	|a	|[38][S]\darr	|i	|f	|o	|h	|[39][S]\darr	|*	|*	|.
|j	|*	|a	|a	|h	|p	|[40][S]\rarr	|u	|c	|h	|y	|b	|[][,]{ }	|u	|s	|t	|a	|l	|o	|n	|y	|*	|.
|n	|*	|n	|*	|a	|o	|*	|r	|z	|*	|a	|u	|ż	|s	|k	|e	|z	|i	|r	|a	|*	|*	|.
|o	|*	|*	|*	|t	|i	|*	|e	|y	|*	|s	|l	|ó	|z	|a	|r	|z	|w	|r	|s	|*	|*	|.
|ś	|*	|*	|*	|e	|t	|*	|k	|c	|*	|s	|*	|ł	|n	|ł	|a	|a	|k	|o	|t	|*	|*	|.
|ć	|[41][S]\rarr	|t	|e	|r	|e	|n	|*	|*	|*	|*	|*	|t	|i	|a	|c	|n	|a	|r	|u	|*	|*	|.
|*	|*	|*	|*	|*	|v	|[42][S]\rarr	|p	|o	|ż	|y	|t	|e	|k	|*	|j	|*	|*	|*	|l	|*	|*	|.
|*	|*	|[43][S]\rarr	|b	|l	|i	|ź	|n	|i	|a	|k	|i	|*	|*	|*	|a	|*	|*	|*	|a	|*	|*	|.
|*	|*	|*	|[44][S]\rarr	|a	|n	|t	|y	|r	|e	|a	|l	|i	|z	|m	|*	|*	|*	|*	|*	|*	|*	|.
|*	|*	|*	|*	|*	|*	|*	|*	|*	|*	|*	|*	|*	|*	|*	|*	|*	|*	|*	|*	|*	|*	|.\end{Puzzle}

\newpage

\begin{PuzzleClues}{\textbf{Poziome}\\}\Clue{3}{}{w sporcie - bieg z wysokim podnoszeniem kolan - do klatki piersiowej}
\Clue{7}{}{działanie określone dla dwóch funkcji (lub opisywanych przez nie sygnałów) dające w wyniku inną, która może być postrzegana jako zmodyfikowana wersja oryginalnych funkcji}
\Clue{11}{}{aleja obsadzona szpalerami}
\Clue{13}{}{zawartość kotła, dużego głębokiego metalowego naczynia, zwykle zamykanego pokrywą, służącego do gotowania}
\Clue{14}{}{smakowitość - to, że coś jest bardzo dobre w smaku}
\Clue{17}{}{legendarna wyspa, na której prawdopodobnie istniała wysoko rozwinięta cywilizacja}
\Clue{18}{}{architekt turecki (1489-1578), liczne meczety}
\Clue{19}{}{obiekt sportowy przypominający lekko spłaszczoną rurę o znacznej średnicy przeciętą w poziomie}
\Clue{23}{}{rumuńska miejscowość z cennymi zabytkami sakralnymi, cerkwie XIV-XVIII w}
\Clue{25}{}{impreza plenerowa}
\Clue{27}{}{podatność materiału na obróbkę skrawaniem}
\Clue{32}{}{proces zużycia ściernego związany z ubytkiem powierzchniowym elementów maszyn lub urządzeń wskutek oddziaływania na niego luźnych lub umocowanych zarówno suchych, jak i wilgotnych ziarn mineralnych}
\Clue{40}{}{w układzie regulacji: różnica między wartością zadaną sygnału oraz wartością sygnału wyjściowego w stanie ustalonym}
\Clue{41}{}{powierzchnia ziemi o określonym ukształtowaniu, roślinności, zabudowaniach}
\Clue{42}{}{w pszczelarstwie: zebrane przez pszczoły surowce pochodzenia roślinnego, wykorzystywane przez nie do wyrobu miodu i pierzgi}
\Clue{43}{}{dwoje dzieci urodzonych z tej samej ciąży mnogiej}
\Clue{44}{}{kierunek w literaturze i sztuce, który powstał w opozycji do realizmu}\end{PuzzleClues}

\begin{PuzzleClues}{\textbf{Pionowe}\\}\Clue{1}{}{SKRZELA; dodatkowe małe skrzydełka umieszczane na przedniej krawędzi skrzydeł samolotu}
\Clue{2}{}{Ognorhynchus icterotis - gatunek ptaka z rodziny papugowatych (Psittacidae), z podrodziny papug neotropikalnych (Arinae)}
\Clue{3}{}{Sus scrofa - gatunek dużego lądowego ssaka łożyskowego z rzędu parzystokopytnych; jedyny przedstawiciel dziko żyjących świniowatych w Europie, przodek świni domowej}
\Clue{4}{}{niemiecki kompozytor i organista (1685-1759); opery w stylu włoskim ('Julius Cezar') utwory orkiestrowe, organowe, klawesynowe, kameralne; jeden z największych kompozytorów baroku}
\Clue{5}{}{środkowa trójkąta; odcinek, który łączy wierzchołek trójkąta ze środkiem jego przeciwległego boku}
\Clue{6}{}{potrawa wigilijna z gotowanej pszenicy z makiem, miodem i bakaliami}
\Clue{7}{}{grupa funkcyjna, powstała przez oderwanie atomu wodoru od cząsteczki propanu}
\Clue{8}{}{wczesny etap rozwoju mowy po krzyku i głużeniu, pojawiający się ok. 6 miesiąca życia}
\Clue{9}{}{przewód, komin}
\Clue{10}{}{Radyserb - (1822-1907), poeta górnołużycki, wiersze epickie, ballady, opowiadania historyczne, poezje dla dzieci}
\Clue{12}{}{przyrząd służący do zamgławiania roślin na plantacji drzew}
\Clue{14}{}{cecha czegoś, co jest koedukacyjne}
\Clue{15}{}{ogół sygnałów odbieranych za pomocą wzroku, przekazywanych przez nadajnik i odczytywanych przy użyciu odbiornika, np. telewizora}
\Clue{16}{}{proces biologiczny, w wyniku którego powstają nowe gatunki organizmów}
\Clue{18}{}{tablica zastępująca macierz w niektórych obliczeniach}
\Clue{20}{}{osoba, która łamie przepisy ruchu drogowego, narażając bezpieczeństwo swoje i innych osób na drodze}
\Clue{21}{}{przekształcony pęcherzyk Graafa (pęcherzyk jajnikowy), który funkcjonuje jako gruczoł dokrewny}
\Clue{22}{}{tytuł tureckiego namiestnika w Algierii i Tunezji}
\Clue{24}{}{instalacja lub kran, z którego nalewa się beczkowy napój, zwykle alkoholowy, najczęściej piwo}
\Clue{26}{}{zrośnięcie się jam ciała i zamknięcie światła naczyń}
\Clue{28}{}{śmiałek, człowiek mężny; użycie najczęściej ironiczne}
\Clue{29}{}{resztka dopalającego się papierosa}
\Clue{30}{}{prawowity syn własnego ojca}
\Clue{31}{}{rodzaj sosu z musztardy i galaretki porzeczkowej do zimnych mięs}
\Clue{32}{}{część hełmu, której zadaniem jest osłanianie uszu}
\Clue{33}{}{Fezzan - kraina w południowo-zachodniej Libii, na Saharze}
\Clue{34}{}{kosmetyk do nawilżania i natłuszczania skóry, zwłaszcza dziecięcej}
\Clue{35}{}{film, którego pełna napięcia i strachu akcja wywołuje u odbiorców dreszcz emocji i grozy; film zrealizowany w konwencji fantastyki grozy}
\Clue{36}{}{rasa konia domowego zaliczana do koni zimnokrwistych, pochodząca z Poitou we Francji; rasa wywodzi się z krzyżowań między klaczami z Poitou a ciężkimi ogierami importowanymi z Holandii, Norwegii i Danii do drenowania bagien}
\Clue{37}{}{styl w muzyce, łączący elementy współczesnej muzyki improwizowanej, jazzu, punk rocka i folku}
\Clue{38}{}{miasto w województwie małopolskim, w powiecie krakowskim, siedziba gminy miejsko-wiejskiej Skała}
\Clue{39}{}{dżudoka, złoty medalista olimpijski z Atlanty, dwukrotny mistrz świata}\end{PuzzleClues}\newpage%\section*{Krzyżówka 66}

\noindent\begin{Puzzle}{19}{30}|*	|[1][S]\drarr	|p	|r	|o	|b	|l	|e	|m	|a	|t	|y	|c	|z	|n	|o	|ś	|ć	|*	|*	|.
|[2][S]\rarr	|w	|i	|r	|n	|i	|k	|*	|*	|*	|*	|*	|*	|*	|*	|*	|[3][S]\darr	|*	|*	|*	|.
|*	|y	|*	|*	|[4][S]\darr	|*	|*	|*	|*	|*	|*	|*	|*	|*	|[5][S]\drarr	|m	|r	|ó	|z	|*	|.
|*	|w	|*	|*	|n	|[6][S]\drarr	|l	|i	|c	|z	|e	|b	|n	|i	|k	|*	|y	|[7][S]\darr	|*	|[8][S]\darr	|.
|*	|r	|*	|[9][S]\rarr	|o	|s	|t	|r	|o	|n	|ó	|g	|*	|*	|r	|*	|k	|f	|[10][S]\darr	|t	|.
|*	|o	|[11][S]\darr	|*	|n	|a	|*	|[12][S]\drarr	|p	|i	|e	|t	|r	|z	|y	|k	|*	|u	|g	|t	|.
|[13][S]\rarr	|t	|c	|*	|o	|l	|*	|k	|[14][S]\rarr	|g	|e	|n	|i	|u	|s	|z	|*	|n	|t	|d	|.
|*	|k	|h	|[15][S]\darr	|*	|a	|*	|l	|[16][S]\drarr	|p	|l	|a	|n	|e	|t	|o	|i	|d	|a	|*	|.
|*	|a	|e	|l	|[17][S]\darr	|m	|[18][S]\rarr	|e	|k	|s	|p	|o	|r	|t	|a	|c	|j	|a	|*	|*	|.
|*	|*	|m	|y	|l	|a	|*	|i	|o	|*	|[19][S]\darr	|[20][S]\rarr	|e	|k	|l	|e	|r	|*	|*	|*	|.
|*	|*	|i	|g	|o	|n	|*	|k	|t	|*	|g	|*	|[21][S]\drarr	|m	|i	|g	|a	|c	|z	|*	|.
|*	|[22][S]\rarr	|a	|o	|j	|d	|a	|*	|*	|*	|u	|[23][S]\rarr	|k	|o	|z	|a	|*	|[24][S]\darr	|*	|*	|.
|*	|*	|*	|d	|a	|r	|[25][S]\rarr	|w	|o	|d	|z	|a	|r	|k	|a	|*	|*	|w	|[26][S]\darr	|*	|.
|*	|*	|*	|i	|l	|a	|*	|*	|[27][S]\darr	|*	|e	|[28][S]\darr	|y	|*	|c	|*	|[29][S]\darr	|i	|p	|*	|.
|*	|*	|[30][S]\darr	|u	|i	|[][,]{ }	|*	|*	|t	|[31][S]\darr	|k	|b	|t	|[32][S]\darr	|j	|[33][S]\darr	|p	|s	|o	|*	|.
|*	|*	|c	|m	|z	|o	|*	|*	|u	|m	|*	|o	|y	|b	|a	|k	|i	|k	|w	|*	|.
|*	|*	|e	|[][,]{ }	|m	|l	|[34][S]\darr	|*	|r	|o	|[35][S]\darr	|s	|c	|r	|*	|u	|ę	|o	|ó	|*	|.
|*	|*	|g	|p	|*	|b	|d	|*	|n	|r	|r	|s	|z	|o	|*	|r	|k	|z	|z	|*	|.
|*	|*	|ł	|o	|*	|r	|r	|[36][S]\drarr	|i	|z	|o	|a	|n	|e	|m	|o	|n	|a	|*	|*	|.
|*	|*	|a	|g	|*	|z	|a	|o	|k	|y	|z	|[][,]{ }	|o	|d	|*	|r	|o	|*	|[37][S]\darr	|*	|.
|*	|*	|[][,]{ }	|i	|[38][S]\darr	|y	|b	|b	|i	|k	|p	|n	|ś	|e	|*	|t	|s	|*	|k	|*	|.
|*	|*	|s	|ę	|h	|m	|a	|c	|e	|[][,]{ }	|a	|o	|ć	|r	|*	|*	|t	|*	|a	|*	|.
|*	|*	|i	|t	|i	|i	|n	|y	|t	|p	|d	|v	|*	|l	|*	|[39][S]\darr	|k	|*	|l	|*	|.
|*	|[40][S]\rarr	|t	|e	|m	|a	|t	|*	|*	|i	|*	|a	|*	|a	|[41][S]\darr	|m	|a	|[42][S]\darr	|i	|*	|.
|*	|*	|ó	|*	|s	|*	|*	|*	|*	|s	|*	|*	|*	|m	|b	|i	|*	|m	|f	|*	|.
|[43][S]\rarr	|i	|w	|o	|*	|*	|[44][S]\rarr	|s	|i	|k	|k	|i	|m	|*	|o	|o	|*	|o	|a	|*	|.
|*	|[45][S]\rarr	|k	|r	|o	|k	|o	|d	|y	|l	|*	|*	|*	|*	|u	|g	|*	|n	|t	|*	|.
|[46][S]\rarr	|m	|a	|n	|*	|*	|[47][S]\rarr	|c	|h	|i	|[][,]{ }	|k	|w	|a	|d	|r	|a	|t	|*	|*	|.
|*	|*	|*	|*	|*	|*	|[48][S]\rarr	|z	|a	|w	|ł	|o	|t	|n	|i	|a	|*	|a	|*	|*	|.
|*	|[49][S]\rarr	|a	|t	|o	|m	|i	|s	|t	|y	|k	|a	|*	|*	|n	|m	|*	|ż	|*	|*	|.
|[50][S]\rarr	|b	|r	|a	|m	|k	|a	|r	|z	|*	|*	|*	|*	|*	|*	|*	|*	|*	|*	|*	|.\end{Puzzle}

\newpage

\begin{PuzzleClues}{\textbf{Poziome}\\}\Clue{1}{}{bycie problematycznym}
\Clue{2}{}{wirująca część maszyny elektrycznej}
\Clue{5}{}{osad atmosferyczny, tworzący drobne lodowe kryształki w postaci igieł powstających na dowolnym podłożu hydrofilowym}
\Clue{6}{}{wyraz, którego prymarną funkcją jest określanie liczby bądź kolejności bytów czy sytuacji}
\Clue{9}{}{ostronos workowaty, pałanka miodojad, Tarsipes rostratus - jedyny gatunek należący do rodziny ostronogowatych, jeden z nielicznych ssaków odżywiający się nektarem i pyłkiem kwiatowym; zamieszkuje południowo-zachodnią Australię}
\Clue{12}{}{lekkoatleta, srebrny medalista z Montrealu w sztafecie 4x400 m}
\Clue{13}{}{w chemii: symbol technetu}
\Clue{14}{}{osoba posiadająca wybitnie ponadprzeciętne zdolności intelektualne}
\Clue{16}{}{statek kosmiczny wprowadzony na orbitę satelitarną Słońca}
\Clue{18}{}{wyprowadzenie zwłok na miejsce, gdzie pozostają do pogrzebu}
\Clue{20}{}{suwak}
\Clue{21}{}{sygnalizacyjny przyrząd na okręcie}
\Clue{22}{}{śpiewak i poeta w przedhomerowej Grecji opiewający przy wtórze gitary lub formingi bohaterów i bogów}
\Clue{23}{}{mały piecyk}
\Clue{25}{}{dźwignica przemieszczająca ładunek za pośrednictwem dwóch cięgien}
\Clue{36}{}{linia jednakowej średniej siły wiatru}
\Clue{40}{}{jedna z dwóch części wyrazu pochodnego, od której został on utworzony}
\Clue{43}{}{miasto w płd.-zach. Nigerii, ośrodek handlu i rzemiosła}
\Clue{44}{}{stan w płn-wsch. Indiach, w Himalajach, powierzchnia 7,3 tyś. km2, stolica Ganotok}
\Clue{45}{}{gad z rodziny krokodylowatych}
\Clue{46}{}{wyspa należąca do archipelagu Wysp Brytyjskich, położona na Morzu Irlandzkim między Wielką Brytanią a Irlandią}
\Clue{47}{}{zmienna losowa mająca rozkład sumy kwadratów zmiennych losowych o standardpowym rozkładzie normalnym, tj. N(0,1)}
\Clue{48}{}{przedstawiciel glonów (zielenic) należący do rodziny zawłotniowatych (Chlamydomonadaceae), który występuje w wodach słodkich, wilgotnych glebach, na śniegu i lodzie w wysokich górach}
\Clue{49}{}{nauka o atomach oraz o pokojowym wykorzystaniu energii nuklearnej}
\Clue{50}{}{rodzaj biletera, ochroniarza; osoba, która wpuszcza na półzamknięte imprezy, np. do klubu}\end{PuzzleClues}

\begin{PuzzleClues}{\textbf{Pionowe}\\}\Clue{1}{}{przyczepa z ruchomą skrzynią, umożliwiającą szybkie wyładowanie ładunku}
\Clue{3}{}{donośny gardłowy głos zwierzęcia}
\Clue{4}{}{kompozytor włoski (1924-1990); opery, balety, muzyka orkiestrowa}
\Clue{5}{}{proces powstawania fazy krystalicznej z fazy stałej (amorficznej), fazy ciekłej, roztworu lub fazy gazowej}
\Clue{6}{}{płaz ogoniasty z rodziny skrytoskrzelnych}
\Clue{7}{}{to, że ktoś za kogoś płaci, coś funduje}
\Clue{8}{}{kod ISO 4217 dolara Trynidadu i Tobago}
\Clue{10}{}{seria gier komputerowych wydana przez firmę Rockstar Games, jedna z najbardziej kontrowersyjnych serii gier komputerowych - powoduje to fabuła bogata w wątki związane z przemocą, brutalnością i łamaniem prawa}
\Clue{11}{}{ogół zagadnień, procesów i substancji, które znajdują się w zakresie zainteresowania chemii - dziedziny wiedzy}
\Clue{12}{}{zupka, papka z rozgotowanego w wodzie ryżu lub kaszy}
\Clue{15}{}{wężówka pogięta, Lygodium flexuosum - gatunek paproci z rodziny wężówkowatych}
\Clue{16}{}{rekrut}
\Clue{17}{}{podporządkowanie się władzom, posłuszeństwo wobec polityków, rządzących}
\Clue{19}{}{zmiany występujące u chorych na reumatoidalne zapalenie stawów}
\Clue{21}{}{trudność, jednocześnie wielki wpływ na przyszłość}
\Clue{24}{}{włókno wiskozowe; rodzaj włókna powstającego w wyniku chemicznej obróbki celulozy, produkowany w postaci włókna ciągłego (sztuczny jedwab, kord) lub ciętego (zwykle nici są wtedy bawełnopodobne lub wełnopodobne)}
\Clue{26}{}{najczęściej czterokołowy pojazd konny}
\Clue{27}{}{kilkuskrzydłowe drzwi obrotowe}
\Clue{28}{}{styl latynoskiego tańca towarzyskiego w rytmie bossa novy}
\Clue{29}{}{piękny drobiazg służący do ozdoby}
\Clue{30}{}{cegła ceramiczna o dużej liczbie małych otworów prostopadłych do podstawy cegły}
\Clue{31}{}{Synthliboramphus hypoleucus - gatunek ptaka morskiego z rodziny alk (Alcidae)}
\Clue{32}{}{malarz włoski (1900-72) malarstwo abstrakcyjne, pejzaże, martwe natury, obrazy figuralne}
\Clue{33}{}{miejscowość wypoczynkowa}
\Clue{34}{}{żołnierz, najczęściej gwardzista wyższego oficera}
\Clue{35}{}{podział czegoś na części, mniejsze składniki}
\Clue{36}{}{człowiek, który jest nie rozpoznawany jako należący do tej samej społeczności lub inaczej definiowanej grupy, co inni}
\Clue{37}{}{władza i urząd kalifa}
\Clue{38}{}{HOMS; miasto w zach. Syrii, starożytna Emesa, ośrodek adm. muhafazy Hims}
\Clue{39}{}{zapis ilustrujący czynność mięśnia}
\Clue{41}{}{malarz francuski (1829-98) prekursor impresjonizmu}
\Clue{42}{}{film, słuchowisko, przedstawienie, występ (wytwór np. w mediach) złożone z różnych części}\end{PuzzleClues}\newpage%\section*{Krzyżówka 67}

\noindent\begin{Puzzle}{17}{33}|*	|*	|*	|*	|*	|*	|*	|[1][S]\drarr	|a	|k	|t	|*	|*	|[2][S]\drarr	|m	|w	|k	|*	|.
|*	|*	|*	|[3][S]\darr	|*	|[4][S]\darr	|[5][S]\darr	|b	|[6][S]\darr	|*	|*	|*	|[7][S]\darr	|j	|*	|*	|*	|*	|.
|*	|*	|[8][S]\drarr	|p	|ę	|c	|h	|e	|r	|z	|n	|i	|c	|a	|*	|*	|[9][S]\darr	|*	|.
|*	|*	|p	|o	|*	|h	|y	|z	|o	|*	|*	|*	|i	|r	|*	|*	|n	|*	|.
|*	|*	|i	|d	|*	|o	|d	|l	|s	|*	|*	|*	|a	|z	|*	|*	|a	|*	|.
|*	|*	|e	|o	|*	|d	|r	|o	|z	|*	|*	|*	|ł	|ę	|*	|*	|r	|*	|.
|*	|*	|s	|k	|*	|o	|o	|t	|c	|*	|*	|*	|k	|b	|*	|*	|k	|*	|.
|*	|*	|[][,]{ }	|a	|*	|w	|f	|k	|z	|*	|*	|*	|o	|s	|*	|*	|o	|*	|.
|*	|*	|p	|r	|*	|i	|i	|i	|e	|*	|*	|*	|[][,]{ }	|k	|*	|*	|t	|[10][S]\darr	|.
|*	|*	|a	|p	|*	|e	|t	|*	|n	|*	|[11][S]\darr	|*	|m	|i	|*	|*	|y	|t	|.
|*	|*	|s	|o	|*	|c	|*	|*	|i	|*	|k	|*	|r	|*	|*	|[12][S]\darr	|k	|a	|.
|*	|[13][S]\darr	|t	|w	|*	|k	|*	|*	|e	|*	|n	|[14][S]\darr	|ó	|*	|*	|o	|[][,]{ }	|ś	|.
|*	|ł	|e	|a	|*	|i	|*	|*	|*	|*	|e	|m	|w	|*	|[15][S]\darr	|t	|t	|m	|.
|*	|u	|r	|t	|*	|*	|*	|*	|*	|*	|b	|a	|c	|[16][S]\darr	|a	|o	|w	|a	|.
|*	|s	|s	|e	|*	|*	|*	|[17][S]\rarr	|k	|i	|e	|r	|z	|o	|n	|k	|a	|*	|.
|*	|k	|k	|*	|*	|*	|*	|[18][S]\rarr	|p	|o	|l	|l	|e	|r	|t	|*	|r	|*	|.
|*	|o	|i	|*	|*	|*	|[19][S]\drarr	|m	|k	|d	|*	|a	|*	|z	|y	|[20][S]\darr	|d	|*	|.
|*	|s	|*	|*	|*	|[21][S]\rarr	|b	|z	|d	|e	|t	|*	|*	|e	|t	|l	|y	|*	|.
|*	|k	|*	|*	|*	|*	|ł	|*	|*	|*	|*	|*	|*	|s	|r	|u	|*	|*	|.
|*	|r	|[22][S]\rarr	|w	|y	|d	|o	|l	|n	|o	|ś	|ć	|*	|z	|y	|k	|*	|*	|.
|[23][S]\drarr	|z	|u	|c	|h	|e	|n	|k	|a	|*	|[24][S]\darr	|*	|[25][S]\darr	|y	|n	|s	|*	|*	|.
|s	|y	|[26][S]\rarr	|w	|i	|e	|k	|i	|[][,]{ }	|ś	|r	|e	|d	|n	|i	|e	|*	|*	|.
|o	|d	|*	|*	|*	|*	|a	|*	|*	|*	|a	|[27][S]\darr	|r	|a	|t	|m	|*	|*	|.
|k	|ł	|[28][S]\drarr	|d	|o	|l	|*	|[29][S]\drarr	|d	|a	|m	|k	|a	|*	|a	|b	|*	|*	|.
|o	|e	|l	|*	|[30][S]\darr	|[31][S]\darr	|*	|t	|*	|*	|i	|u	|k	|*	|r	|u	|*	|*	|.
|l	|*	|a	|*	|c	|w	|*	|a	|[32][S]\darr	|*	|ę	|r	|o	|*	|z	|r	|*	|*	|.
|s	|*	|n	|*	|a	|z	|*	|r	|g	|*	|*	|w	|n	|*	|e	|s	|*	|*	|.
|t	|*	|c	|*	|r	|g	|*	|u	|ó	|*	|[33][S]\darr	|a	|*	|*	|*	|k	|*	|*	|.
|w	|*	|i	|*	|n	|ó	|[34][S]\darr	|k	|g	|*	|w	|*	|*	|*	|*	|i	|*	|*	|.
|o	|*	|a	|[35][S]\drarr	|o	|r	|t	|a	|l	|i	|o	|n	|*	|*	|*	|*	|*	|*	|.
|*	|*	|*	|z	|t	|z	|o	|*	|e	|*	|l	|*	|*	|*	|*	|*	|*	|*	|.
|*	|*	|*	|ł	|*	|e	|*	|*	|*	|*	|e	|*	|*	|*	|*	|*	|*	|*	|.
|*	|*	|*	|o	|*	|*	|*	|*	|*	|*	|*	|*	|*	|*	|*	|*	|*	|*	|.
|*	|*	|*	|*	|*	|*	|*	|*	|*	|*	|*	|*	|*	|*	|*	|*	|*	|*	|.\end{Puzzle}

\newpage

\begin{PuzzleClues}{\textbf{Poziome}\\}\Clue{1}{}{w filozofii: pojęcie należące do podstawowych terminów metafizyki klasycznej, biorące swój początek w filozofii Arystotelesa, a następnie rozwijane przez średniowieczną myśl scholastyczną, które odnosi do ruchu, a także ogólniej do istnienia, przemian, upływu czasu i celu zdarzeń}
\Clue{2}{}{kod ISO 4217 kwachy malawijskiej}
\Clue{8}{}{Physalis alkekengi - gatunek byliny z rodziny psiankowatych}
\Clue{17}{}{naczynie do wyrobu masła}
\Clue{18}{}{czeski kajakarz górski, wicemistrz olimpijski z Atlanty}
\Clue{19}{}{kod ISO 4217 denara macedońskiego}
\Clue{21}{}{sprawa mało ważna}
\Clue{22}{}{zdolność do ruchu, możność normalnego sprawowania funkcji w organizmie}
\Clue{23}{}{dziewczynka, należąca do formacji zuchów}
\Clue{26}{}{okres w historii od V do XV wieku n.e}
\Clue{28}{}{jednostka natężenia bólu}
\Clue{29}{}{rower z tzw. damską ramą, czyli bez umieszczonej wysoko poprzeczki}
\Clue{35}{}{cienka, nieprzemakalna tkanina z tego włókna używana na płaszcze, kurtki}\end{PuzzleClues}

\begin{PuzzleClues}{\textbf{Pionowe}\\}\Clue{1}{}{pingwiny, Spheniscidae - rodzina wodnych ptaków nielatających z rzędu pingwinów (Sphenisciformes)}
\Clue{2}{}{Adam (1590-1648); kompozytor, skrzypek, poeta}
\Clue{3}{}{zastrzalinowate, Podocarpaceae - rodzina drzew nagonasiennych zaliczana do rzędu araukariowców, występująca przeważnie na półkuli południowej}
\Clue{4}{}{malarz pochodzenia włoskiego (1731-1818) nadworny malarz króla Stanisława Augusta; portrety, kompozycje historyczne, obrazy religijne, alegrafie}
\Clue{5}{}{roślina wodna przytwierdzona do dna lub swobodnie pływająca}
\Clue{6}{}{wypowiedź w stanowczym tonie, której nadawca zdecydowanie chce coś uzyskać}
\Clue{7}{}{obfitująca w tłuszcze i węglowodany powłoka nasion lub owoców niektórych roślin}
\Clue{8}{}{typ psa, który w drodze selekcji dokonywanej przez człowieka, nabrał cech obrońcy stad zwierząt hodowlanych przed drapieżnikami}
\Clue{9}{}{częste określenie narkotyków mocno uzależniających; podziała na narkotyki miękkie i twarde nie jest jasny}
\Clue{10}{}{element przenośnika, ruchomy pas, pod którym umieszczone są rolki, na którym coś się kładzie, by to coś było przesuwane}
\Clue{11}{}{przenośnie o czymś, co ogranicza swobodne wypowiadanie się}
\Clue{12}{}{część czapki (zwłaszcza wojskowej)}
\Clue{13}{}{MOTYLE}
\Clue{14}{}{cienka rzadka tkanina bawełniana; gaza}
\Clue{15}{}{zwolennicy doktryny występującej w wielu różnych wyznaniach chrześcijańskich, nieuznającej nauki o Trójcy Świętej, zapoczątkowanej w I wieku naszej ery}
\Clue{16}{}{wierzchnie zielone łupiny orzechów}
\Clue{19}{}{w biologii termin oznaczający cienką tkankę (często zewnętrzną) o specyficznych właściwościach fizycznych; wyróżnia się wiele rodzajów błonek w zależności od ich budowy, funkcji i procesu powstawania}
\Clue{20}{}{język należący do zachodniej grupy języków germańskich, wywodzi się z frankijskiego dialektu języka niemieckiego; używa go ok. 390 000 ludzi, głównie w Luksemburgu, w części Niemiec, Belgii i Francji}
\Clue{23}{}{towarzystwo sportowe, którego filie (tzw. gniazda) powstawały w krajach słowiańskich, począwszy od drugiej połowy XIX w.; celem tych organizacji było podnoszenie sprawności fizycznej i duchowej oraz rozbudzanie ducha narodowego}
\Clue{24}{}{narząd ruchu lub chwytny u niektórych zwierząt, np. głowonogów lub szkarłupni}
\Clue{25}{}{strzałosmok, Harriotta raleighana - gatunek morskiej ryby chrzęstnoszkieletowej z rodziny drakonowatych (Rhinochimaeridae); drakon żyje na głębokościach 380-2600 m w strefach klimatu umiarkowanego Oceanu Atlantyckiego i Spokojnego oraz w pobliżu południowej Australii}
\Clue{27}{}{o kobiecie wrednej, podłej}
\Clue{28}{}{włoska marka samochodów osobowych założona w 1906 roku w Turynie}
\Clue{29}{}{Hippocamelus antisensis - gatunek ssaka parzystokopytnego z rodziny jeleniowatych, blisko spokrewniony z huemalem; zamieszkuje kraje Ameryki Południowej - Ekwador, Peru, Boliwię, Chile i Argentynę}
\Clue{30}{}{jednostka szybkości zmiany entropii}
\Clue{31}{}{część międzymózgowia znajdująca się pod spoidłem wielkim; przylega do niego podwzgórze}
\Clue{32}{}{wyszukiwarka internetowa stworzona przez amerykańską spółkę Google Inc.; jej celem jest skatalogowanie wszystkich możliwych informacji i udostępnienie ich za pomocą Internetu}
\Clue{33}{}{rozszerzenie lub uchyłek przełyku służące do gromadzenia i przenoszenia pokarmu}
\Clue{34}{}{nazwa zastępująca, ze względu na tabu kulturowe, ogół słownictwa związanego z seksem}
\Clue{35}{}{zły czyn; to, co ludzie robią i co nie jest dobre}\end{PuzzleClues}\newpage%\section*{Krzyżówka 68}

\noindent\begin{Puzzle}{24}{27}|*	|*	|*	|*	|*	|*	|*	|*	|*	|*	|*	|*	|[1][S]\drarr	|t	|o	|w	|a	|r	|z	|y	|s	|t	|w	|o	|*	|.
|*	|*	|*	|*	|[2][S]\darr	|[3][S]\darr	|[4][S]\rarr	|s	|l	|a	|l	|o	|m	|[][,]{ }	|r	|ó	|w	|n	|o	|l	|e	|g	|ł	|y	|*	|.
|*	|*	|*	|[5][S]\drarr	|a	|k	|s	|j	|o	|l	|o	|g	|i	|a	|*	|*	|*	|*	|[6][S]\drarr	|f	|u	|s	|y	|t	|*	|.
|*	|*	|*	|a	|m	|w	|*	|[7][S]\rarr	|f	|e	|r	|m	|e	|n	|t	|o	|r	|*	|g	|*	|*	|*	|*	|*	|*	|.
|*	|*	|*	|k	|e	|i	|[8][S]\drarr	|w	|i	|e	|l	|o	|s	|t	|o	|p	|n	|i	|o	|w	|o	|ś	|ć	|*	|*	|.
|*	|*	|*	|s	|r	|a	|n	|[9][S]\darr	|*	|[10][S]\darr	|*	|*	|z	|[11][S]\drarr	|c	|*	|*	|[12][S]\drarr	|b	|i	|n	|d	|a	|ż	|*	|.
|*	|*	|*	|a	|y	|t	|a	|d	|*	|k	|[13][S]\darr	|*	|a	|ś	|[14][S]\darr	|*	|[15][S]\rarr	|k	|e	|p	|i	|*	|*	|*	|*	|.
|*	|*	|*	|m	|k	|*	|d	|w	|*	|ó	|f	|*	|l	|w	|j	|[16][S]\rarr	|g	|o	|l	|d	|o	|n	|i	|*	|[17][S]\darr	|.
|*	|[18][S]\darr	|[19][S]\darr	|i	|a	|[20][S]\drarr	|z	|a	|w	|ł	|o	|t	|n	|i	|a	|[][,]{ }	|ś	|n	|i	|e	|ż	|n	|a	|*	|b	|.
|*	|t	|e	|t	|n	|a	|i	|[][,]{ }	|*	|k	|r	|[21][S]\darr	|i	|e	|z	|[22][S]\rarr	|k	|o	|n	|s	|u	|l	|*	|*	|i	|.
|*	|i	|r	|n	|k	|n	|e	|ś	|[23][S]\darr	|o	|m	|p	|k	|c	|z	|*	|*	|w	|*	|*	|*	|*	|[24][S]\darr	|[25][S]\darr	|a	|.
|*	|a	|o	|i	|a	|g	|w	|w	|n	|r	|a	|a	|*	|a	|ó	|*	|*	|*	|*	|*	|*	|*	|m	|p	|ł	|.
|*	|z	|z	|k	|*	|l	|k	|i	|a	|o	|c	|p	|*	|[][,]{ }	|w	|*	|[26][S]\drarr	|k	|n	|y	|p	|*	|i	|a	|a	|.
|*	|y	|j	|*	|*	|i	|a	|a	|p	|d	|j	|r	|*	|d	|k	|[27][S]\rarr	|a	|s	|*	|[28][S]\darr	|[29][S]\darr	|*	|l	|u	|[][,]{ }	|.
|[30][S]\drarr	|d	|a	|v	|o	|s	|*	|t	|r	|e	|a	|o	|*	|y	|a	|*	|k	|*	|*	|j	|f	|[31][S]\darr	|i	|p	|n	|.
|s	|*	|[][,]{ }	|*	|*	|t	|*	|y	|ę	|k	|[][,]{ }	|t	|*	|m	|*	|*	|t	|*	|*	|u	|i	|d	|r	|e	|o	|.
|t	|*	|l	|[32][S]\drarr	|c	|a	|p	|*	|ż	|*	|m	|n	|*	|n	|[33][S]\rarr	|l	|u	|d	|*	|c	|l	|z	|a	|r	|c	|.
|e	|*	|o	|g	|[34][S]\darr	|*	|*	|*	|a	|*	|ł	|i	|[35][S]\rarr	|a	|z	|d	|a	|r	|c	|h	|o	|i	|d	|y	|*	|.
|r	|[36][S]\drarr	|d	|u	|b	|l	|i	|ń	|c	|z	|y	|k	|*	|*	|*	|*	|l	|*	|*	|t	|g	|e	|i	|z	|*	|.
|y	|c	|o	|m	|i	|*	|[37][S]\darr	|*	|z	|*	|n	|[][,]{ }	|[38][S]\rarr	|p	|l	|i	|n	|t	|a	|*	|e	|s	|a	|m	|*	|.
|l	|h	|w	|o	|o	|*	|r	|*	|*	|*	|a	|b	|[39][S]\rarr	|b	|u	|k	|o	|w	|i	|a	|n	|i	|n	|*	|*	|.
|i	|a	|c	|l	|l	|[40][S]\rarr	|o	|s	|ę	|k	|*	|r	|*	|*	|*	|*	|ś	|*	|*	|*	|e	|ą	|*	|*	|*	|.
|z	|g	|o	|i	|o	|[41][S]\rarr	|c	|y	|t	|o	|w	|a	|l	|n	|o	|ś	|ć	|*	|*	|*	|t	|t	|*	|*	|*	|.
|a	|a	|w	|t	|g	|*	|z	|[42][S]\drarr	|d	|e	|k	|u	|r	|i	|a	|*	|*	|*	|[43][S]\rarr	|b	|y	|k	|*	|*	|*	|.
|t	|l	|a	|*	|i	|*	|e	|d	|[44][S]\rarr	|g	|i	|n	|*	|[45][S]\rarr	|a	|l	|g	|i	|e	|r	|k	|a	|*	|*	|*	|.
|o	|l	|*	|*	|a	|[46][S]\rarr	|k	|o	|ź	|m	|i	|a	|n	|*	|[47][S]\rarr	|t	|r	|e	|s	|k	|a	|*	|*	|*	|*	|.
|r	|*	|*	|*	|*	|*	|*	|k	|*	|*	|*	|*	|[48][S]\rarr	|t	|r	|z	|o	|n	|e	|k	|*	|*	|*	|*	|*	|.
|*	|[49][S]\rarr	|c	|a	|n	|o	|e	|*	|*	|*	|*	|*	|*	|*	|*	|*	|*	|*	|*	|*	|*	|*	|*	|*	|*	|.\end{Puzzle}

\newpage

\begin{PuzzleClues}{\textbf{Poziome}\\}\Clue{1}{}{to, że ktoś komuś towarzyszy, przebywa w czyjejś obecności}
\Clue{4}{}{jedna z konkurencji narciarstwa alpejskiego, polegająca na tym, że zawodnicy ścigają się po tak samo wyznaczonych trasach}
\Clue{5}{}{przyjmowany system wartości}
\Clue{6}{}{człowiek wybredny, taki, który nie wszystko jada, często krzywi się na jedzenie, odmawia konsumpcji, ma specyficzne przyzwyczajenia żywieniowe}
\Clue{7}{}{zawartość fermentora, naczynia, w którym następuje fermentacja piwa}
\Clue{8}{}{cecha czegoś zbudowanego z wielu stopni}
\Clue{11}{}{symbol kulomba - jednostki ładunku elektrycznego w układzie SI}
\Clue{12}{}{BERSO; chodnik zasklepiony półkolistą kratownicą}
\Clue{15}{}{francuska czapka wojskowa w kształcie ściętego stożka wprowadzona do wojska w XIX w}
\Clue{16}{}{(1707-93) komediopisarz włoski, liczne komedie, libretta do oper, „Sługa dwóch panów”, „Sprytna wdówka”, „Mirandolina”}
\Clue{20}{}{Chlamydomonas nivalis - gatunek glonu z rodziny zawłotniowatych (Chlamydomonadaceae); jednokomórkowa zielenica o postaci wiciowca}
\Clue{22}{}{tytuł władców Francji okresu konsulatu; pierwszym (i jedynym dożywotnim) konsulem był Napoleon}
\Clue{26}{}{szewski nóż}
\Clue{27}{}{mistrz, tuz, znakomitość}
\Clue{30}{}{miasto w Szwajcarii (Gryzonia) w Alpach Retyckich najstarsze uzdrowisko klimatyczne}
\Clue{32}{}{kozioł kozy domowej, zazwyczaj mówi się tak na niego, gdy jest starym zwierzęciem lub gdy jest wykastrowany (choć znany choćby z frazeologii jest fakt rozsiewania przez capy brzydkiej woni - a śmierdzą tylko niewykastrowane dorosłe kozły, rozpłodniki)}
\Clue{33}{}{stan chłopski}
\Clue{35}{}{Azhdarchoidea - nadrodzina pterozaurów z podrzędu pterodaktyli; według najpopularniejszej wśród naukowców klasyfikacji obejmuje cztery rodziny}
\Clue{36}{}{mieszkaniec Dublinu}
\Clue{38}{}{płaska czworoboczna płyta umieszczana pod bazą kolumny lub filaru albo na głowicy doryckiej}
\Clue{39}{}{człowiek, który mieszka w Bukowinie Tatrzańskiej albo pochodzi z tej miejscowości}
\Clue{40}{}{bosak - długi drąg zakończony metalowym hakiem i grotem (szpikulcem)}
\Clue{41}{}{liczba cytatów przypadająca na coś lub na kogoś}
\Clue{42}{}{grupa dziesięciu osób, ustanowiona dla celów organizacyjnych}
\Clue{43}{}{familiarny, żartobliwy zwrot do znajomego}
\Clue{44}{}{DŻYN; JAŁOWCÓWKA}
\Clue{45}{}{mieszkanka Algierii, kobieta pochodzenia algierskiego}
\Clue{46}{}{(1771-1856), poeta, przedstawiciel klasycyzmu postanisłowawskiego, przeciwnik romantyzmu}
\Clue{47}{}{tresa - sztuczne lub prawdziwe włosy przypinane do własnej fryzury w celu jej uzupełnienia}
\Clue{48}{}{rękojeść, uchwyt, rączka jakiegoś narzędzia}
\Clue{49}{}{kanu - tradycyjna łódź wiosłowa Indian północnoamerykańskich}\end{PuzzleClues}

\begin{PuzzleClues}{\textbf{Pionowe}\\}\Clue{1}{}{w starożytnej Grecji naczynie do mieszania wina z wodą}
\Clue{2}{}{w gwarze więziennej: okres próby dla kandydatów na osoby grypsujące}
\Clue{3}{}{łow. ogon zwierzyny płowej i borsuka}
\Clue{5}{}{Clubiona sp. - gatunek pająka z rodziny aksamitnikowatych}
\Clue{6}{}{tkanina dekoracyjna, bardziej pospolita i mniej kosztowna niż arras}
\Clue{8}{}{potrawa, której istotną częścią jest nadzienie}
\Clue{9}{}{o skrajnie różniących się od siebie ludziach, środowiskach}
\Clue{10}{}{Xerula - rodzaj grzybów z rodziny obrzękowcowatych; saprotrofy żyjące na martwym drewnie}
\Clue{11}{}{puszka zawierająca substancję dymotwórczą oraz urządzenie zapłonowe}
\Clue{12}{}{indianista norweski (1867-1848); prace na temat religii Indii}
\Clue{13}{}{w rugby - jedna z dwóch formacji w rugby}
\Clue{14}{}{but o charakterystycznym wyglądzie używany przez tancerzy jazzowych}
\Clue{17}{}{noc nieprzespana, bezsenna}
\Clue{18}{}{lek moczopędny o budowie sulfonamidowej będący pochodną lub analogiem benzotiadiazyny}
\Clue{19}{}{proces mechaniczny zachodzący pod lodowcem, polegający na pobieraniu materiału skalnego z podłoża wskutek jego przymarzania i wciskania w lód, rozkruszaniu, rysowaniu i wygładzaniu skał podłoża}
\Clue{20}{}{ekspert w dziedzinie filologii angielskiej}
\Clue{21}{}{Polystichum braunii - kosmopolityczny gatunek rośliny z rodziny nerecznicowatych; występuje w Azji, Europie i Ameryce Północnej; w Polsce gatunek rzadki, występuje głównie w górach}
\Clue{23}{}{mięsień naprężacz powięzi szerokiej, łac. Musculus tensor fasciae latae - mięsień znajdujący się na kończynie dolnej, zaliczany do grupy tylnej mięśni obręczy kończyny dolnej; jego głównym zadaniem jest napinanie pasma biodrowo-piszczelowego, przez co stabilizuje wyprostowany staw kolanowy}
\Clue{24}{}{mrad - tysięczna część radiana}
\Clue{25}{}{bieda ogarniająca całe warstwy społeczne, zubożenie całych warstw społecznych}
\Clue{26}{}{istotność czegoś w danym momencie}
\Clue{28}{}{wyprawiona skóra bydlęca}
\Clue{29}{}{dział biologii zajmujący się badaniem drogi rozwojowej (filogenezą) organizmów}
\Clue{30}{}{urządzenie służące do sterylizowania - czyszczenia, wyjaławiania czegoś za pomocą wysokiej temperatury, działania jakiejś substancji, promieni itp}
\Clue{31}{}{liczba 10, numer 10}
\Clue{32}{}{wykładzina podłogowa z dwuwarstwowej gumy}
\Clue{34}{}{nauka o przyrodzie}
\Clue{36}{}{malarz francuski (1839-1906) jeden z twórców współczesnego malarstwa: pejzaże, martwe natury kompozycje figuralne, portrety}
\Clue{37}{}{uroczystość organizowana w dniu pierwszych urodzin dziecka}
\Clue{42}{}{budowla lub pływające urządzenie w stoczni do budowy lub remontu statków}\end{PuzzleClues}\newpage%\section*{Krzyżówka 69}

\noindent\begin{Puzzle}{21}{25}|*	|*	|[1][S]\drarr	|f	|a	|s	|o	|l	|a	|[][,]{ }	|a	|d	|z	|u	|k	|i	|*	|*	|[2][S]\darr	|*	|[3][S]\darr	|[4][S]\darr	|.
|*	|*	|p	|[5][S]\drarr	|d	|e	|w	|o	|c	|j	|o	|n	|a	|l	|i	|a	|*	|*	|d	|*	|z	|d	|.
|*	|*	|r	|z	|*	|[6][S]\drarr	|s	|t	|r	|o	|j	|n	|i	|k	|o	|w	|a	|t	|e	|*	|r	|j	|.
|[7][S]\drarr	|c	|z	|a	|g	|r	|a	|[][,]{ }	|b	|r	|ą	|z	|o	|w	|o	|ł	|b	|i	|s	|t	|a	|*	|.
|b	|*	|e	|w	|*	|e	|*	|*	|*	|*	|[8][S]\rarr	|t	|r	|ó	|j	|m	|e	|c	|z	|*	|z	|*	|.
|a	|[9][S]\darr	|r	|ó	|*	|j	|*	|*	|*	|*	|*	|*	|*	|*	|[10][S]\drarr	|m	|u	|n	|c	|h	|*	|*	|.
|r	|c	|z	|j	|[11][S]\rarr	|a	|s	|*	|*	|*	|[12][S]\drarr	|s	|m	|o	|k	|*	|[13][S]\drarr	|r	|z	|u	|t	|*	|.
|*	|h	|u	|*	|*	|*	|*	|*	|[14][S]\rarr	|o	|p	|o	|n	|k	|a	|*	|r	|[15][S]\darr	|ó	|*	|*	|*	|.
|[16][S]\drarr	|i	|t	|a	|l	|i	|a	|ń	|s	|k	|i	|*	|*	|*	|w	|[17][S]\darr	|e	|d	|w	|*	|[18][S]\darr	|*	|.
|t	|a	|*	|[19][S]\rarr	|r	|e	|w	|o	|l	|w	|e	|r	|*	|[20][S]\darr	|a	|k	|g	|ę	|k	|[21][S]\darr	|s	|*	|.
|o	|v	|[22][S]\darr	|[23][S]\drarr	|p	|e	|l	|i	|k	|a	|n	|y	|*	|l	|ł	|r	|e	|b	|a	|e	|i	|*	|.
|r	|a	|d	|m	|*	|[24][S]\drarr	|k	|o	|g	|n	|i	|c	|j	|a	|*	|o	|s	|i	|*	|r	|e	|*	|.
|r	|r	|z	|a	|[25][S]\rarr	|p	|*	|*	|[26][S]\rarr	|j	|e	|d	|e	|n	|a	|s	|t	|k	|a	|*	|d	|*	|.
|e	|i	|a	|y	|*	|ó	|[27][S]\rarr	|r	|y	|ż	|*	|*	|*	|d	|*	|*	|r	|[][,]{ }	|*	|*	|e	|*	|.
|n	|*	|t	|*	|*	|ł	|[28][S]\rarr	|b	|e	|z	|e	|c	|n	|o	|ś	|ć	|*	|d	|*	|*	|m	|*	|.
|t	|[29][S]\rarr	|a	|l	|b	|a	|n	|o	|*	|*	|*	|[30][S]\rarr	|g	|r	|a	|b	|a	|r	|z	|*	|n	|*	|.
|*	|[31][S]\darr	|k	|[32][S]\drarr	|i	|n	|w	|a	|r	|i	|a	|n	|t	|*	|[33][S]\rarr	|d	|ż	|u	|l	|*	|a	|*	|.
|[34][S]\drarr	|m	|i	|s	|k	|a	|[][,]{ }	|s	|o	|c	|z	|e	|w	|i	|c	|y	|*	|m	|*	|*	|s	|*	|.
|t	|u	|*	|a	|[35][S]\rarr	|l	|i	|b	|e	|r	|o	|*	|*	|[36][S]\rarr	|k	|ł	|a	|m	|*	|*	|t	|*	|.
|a	|f	|*	|l	|*	|f	|*	|*	|*	|*	|[37][S]\rarr	|s	|a	|l	|e	|r	|n	|o	|*	|*	|k	|*	|.
|l	|a	|[38][S]\rarr	|t	|r	|a	|n	|s	|i	|t	|*	|[39][S]\rarr	|r	|o	|g	|o	|ź	|n	|i	|c	|a	|*	|.
|a	|*	|[40][S]\rarr	|a	|m	|b	|a	|*	|*	|*	|*	|*	|*	|*	|[41][S]\rarr	|a	|s	|d	|i	|c	|*	|*	|.
|*	|*	|*	|*	|[42][S]\rarr	|e	|s	|a	|*	|*	|*	|[43][S]\rarr	|o	|r	|d	|o	|n	|a	|n	|s	|*	|*	|.
|[44][S]\rarr	|k	|r	|y	|s	|t	|a	|l	|i	|z	|a	|c	|j	|a	|*	|*	|*	|*	|*	|*	|*	|*	|.
|*	|[45][S]\rarr	|r	|y	|b	|a	|[][,]{ }	|p	|o	|[][,]{ }	|g	|r	|e	|c	|k	|u	|*	|*	|*	|*	|*	|*	|.
|[46][S]\rarr	|n	|i	|w	|a	|*	|*	|*	|*	|*	|*	|*	|*	|*	|*	|*	|*	|*	|*	|*	|*	|*	|.\end{Puzzle}

\newpage

\begin{PuzzleClues}{\textbf{Poziome}\\}\Clue{1}{}{warzywo, gatunek fasoli; ziarna rośliny o tej samej nazwie}
\Clue{5}{}{dostępne w sprzedaży niewielkie przedmioty związane z kultem religijnym}
\Clue{6}{}{Lamprididae - rodzina morskich ryb głębinowych z rzędu strojnikokształtnych (Lampridiformes)}
\Clue{7}{}{Tchagra australis - gatunek ptaka  z rodziny dzierzbików (Malaconotidae)}
\Clue{8}{}{zawody rozgrywane między trzema drużynami}
\Clue{10}{}{malarz francuski (1840-1926) jeden z głównych twórców impresjonizmu, autor obrazu 'Impresja', stąd nazwa kierunku zwanego impresjonizmem}
\Clue{11}{}{symbol arsenu w chemii}
\Clue{12}{}{okołobiegunowy gwiazdozbiór w obrębie nieba północnego}
\Clue{13}{}{ruch częścią ciała}
\Clue{14}{}{glon naskalny}
\Clue{16}{}{inna (dowcipna, ironiczna) nazwa języka włoskiego - jednego z języków romańskich}
\Clue{19}{}{wielostrzałowa krótka broń palna z magazynkiem w kształcie bębenka obracającego się za pociągnięciem spustu}
\Clue{23}{}{Pelecanidae - rodzina ptaków z rzędu pelikanowych (Pelecaniformes)}
\Clue{24}{}{czynności prawne sędziego mające na celu ustalenie stanu faktycznego}
\Clue{25}{}{siła, z jaką Ziemia przyciąga masę 1 g w miejscu; 1 gf = 1 p = 0,0980665 N}
\Clue{26}{}{rzut karny w piłce nożnej; wykonuje się go z odległości 11 m od bramki}
\Clue{27}{}{produkt spożywczy otrzymywany z ziarna ryżu siewnego}
\Clue{28}{}{cecha człowieka - brak cnót, podłość, niegodziwość}
\Clue{29}{}{jezioro we Włoszech w Górach Albańskich (Lacjum)}
\Clue{30}{}{padlinożerny chrząszcz; zakopują w ziemi trupy kręgowców, którymi się żywią }
\Clue{32}{}{w językoznawstwie: stały, niezmienny element języka realizowany w formie wariantów}
\Clue{33}{}{DŻAUL; jednostka pracy i energii w układzie SI}
\Clue{34}{}{coś, co ma niewielką wartość, jest niewspółmierne z wartością rzeczy, którą się oddaje w zamian za to, jedynie dla szybkiego, chwilowego zysku}
\Clue{35}{}{środkowy obrońca w siatkówce}
\Clue{36}{}{kłamstwo, fałsz; dziś używane tylko w zwrociezadawać komuś, czemuś kłam}
\Clue{37}{}{zatoka Morza Tyrreńskiego u wybrzeży Włoch (Kampania)}
\Clue{38}{}{system nawigacji satelitarnej}
\Clue{39}{}{wieś w Polsce położona w województwie dolnośląskim, w powiecie świdnickim, w gminie Strzegom}
\Clue{40}{}{przedłużenie bocznych naw w kościele}
\Clue{41}{}{AZDYK; rodzaj hydrolokatora}
\Clue{42}{}{zachodnioeuropejska agencja d/s badań kosmicznych}
\Clue{43}{}{dawniej rozkaz służbowy, polecenie, rozporządzenie władzy}
\Clue{44}{}{proces powstawania fazy krystalicznej z fazy stałej (amorficznej), fazy ciekłej, roztworu lub fazy gazowej}
\Clue{45}{}{podawana na ciepło lub zimno potrawa kuchni polskiej; smażone kawałki ryby lub filety rybne w warzywach (z tartą marchwią, pietruszką i selerem), które wraz z koncentratem pomidorowym tworzą rodzaj gęstego sosu}
\Clue{46}{}{dziedzina, obszar czegoś}\end{PuzzleClues}

\begin{PuzzleClues}{\textbf{Pionowe}\\}\Clue{1}{}{skutek przeniesienia czegoś na drugą stronę, przez coś, co stanowi punkt odniesienia}
\Clue{2}{}{rura kanalizacyjna służąca do odprowadzania wody deszczowej}
\Clue{3}{}{środkowy odcinek pędu z kilkoma pączkami szczepiony na podkładce w celu otrzymania szczepu}
\Clue{4}{}{osoba, która zajmuje się wyborem i miksowaniem muzyki na żywo}
\Clue{5}{}{TURBAN}
\Clue{6}{}{poziome drzewce omasztowania}
\Clue{7}{}{ludzie znajdujący się w barze - lokalu gastronomicznym, w którym konsumuje się zamówione wcześniej dania i napoje alkoholowe, głównie piwa oraz drinki i wódkę}
\Clue{9}{}{miasto we Włoszech (Liguria); ośrodek turystyczno-wypoczynkowy i kąpielisko na Riwierze Włoskiej}
\Clue{10}{}{duża część pewnej, niekoniecznie precyzyjnie określonej, odległości lub spory kawałek obszaru}
\Clue{12}{}{śpiewanie, śpiew}
\Clue{13}{}{rejestr - część skali dźwiękowej o charakterystycznej barwie}
\Clue{15}{}{Dryas drummondii - gatunek rośliny z rodziny różowatych}
\Clue{16}{}{sieć wymiany i dystrybucji plików przez Internet, umożliwiająca odciążenie łączy serwera udostępniającego pliki poprzez przyznanie użytkownikom tych samych uprawnień, tj. do wysyłania i pobierania plików}
\Clue{17}{}{w piłce nożnej lub tenisie: przerzut piłki na przeciwległą część boiska lub kortu}
\Clue{18}{}{liczba 17, numer 17}
\Clue{20}{}{angielski poeta i eseista (1775-1864), powieści poetyckie, dramaty, proza refleksyjna; „Dialogi fikcyjne”}
\Clue{21}{}{w chemii: symbol erbu}
\Clue{22}{}{sanskryckie i palijskie opowieści o poprzednich wcieleniach Buddy; część kanonu buddyzmu}
\Clue{23}{}{pisarz niemiecki (1842-1912), powieści podróżnicze i przygodowe; „Winnetou”}
\Clue{24}{}{pogardliwie o ignorancie, osobie, która nie ma podstawowej wiedzy w jakiejś dziedzinie}
\Clue{31}{}{(kablowa) obudowa złącza dwóch lub więcej odcinków kabla}
\Clue{32}{}{prowincja w płn-zach. Argentynie, głównie w Andach, powierzchnia 154,8 km}
\Clue{34}{}{waluta Samoa}\end{PuzzleClues}\newpage%\section*{Krzyżówka 70}

\noindent\begin{Puzzle}{24}{33}|*	|*	|*	|*	|[1][S]\drarr	|b	|o	|n	|z	|o	|*	|*	|[2][S]\drarr	|s	|u	|b	|i	|t	|o	|*	|[3][S]\darr	|*	|*	|[4][S]\darr	|*	|.
|*	|*	|[5][S]\darr	|[6][S]\darr	|k	|*	|*	|[7][S]\drarr	|s	|t	|u	|d	|n	|i	|c	|k	|i	|*	|[8][S]\darr	|*	|b	|*	|*	|d	|*	|.
|*	|*	|h	|t	|r	|*	|*	|p	|[9][S]\rarr	|m	|ą	|k	|a	|[][,]{ }	|j	|ę	|c	|z	|m	|i	|e	|n	|n	|a	|*	|.
|*	|*	|u	|r	|ę	|[10][S]\rarr	|m	|a	|l	|a	|t	|u	|r	|a	|*	|*	|*	|*	|r	|*	|r	|*	|*	|n	|[11][S]\darr	|.
|*	|*	|b	|z	|p	|*	|*	|z	|*	|*	|*	|*	|ó	|[12][S]\drarr	|c	|i	|s	|t	|a	|*	|e	|[13][S]\darr	|[14][S]\darr	|t	|l	|.
|*	|[15][S]\rarr	|a	|n	|a	|g	|l	|i	|f	|*	|[16][S]\darr	|*	|d	|k	|[17][S]\darr	|*	|*	|*	|w	|[18][S]\darr	|s	|n	|s	|e	|i	|.
|*	|*	|*	|a	|c	|*	|*	|o	|*	|*	|l	|*	|[][,]{ }	|a	|c	|*	|*	|*	|i	|p	|t	|i	|u	|j	|d	|.
|*	|*	|[19][S]\rarr	|d	|z	|i	|e	|w	|o	|j	|a	|*	|w	|m	|a	|*	|*	|*	|ń	|r	|e	|e	|p	|s	|o	|.
|*	|*	|*	|e	|e	|*	|*	|a	|*	|*	|m	|*	|y	|u	|r	|[20][S]\drarr	|p	|y	|s	|z	|c	|z	|e	|k	|*	|.
|*	|*	|*	|l	|k	|*	|*	|t	|*	|*	|p	|*	|b	|s	|v	|a	|*	|*	|k	|o	|z	|d	|r	|a	|*	|.
|*	|*	|*	|*	|[][,]{ }	|*	|*	|e	|[21][S]\darr	|*	|k	|*	|r	|z	|a	|r	|*	|[22][S]\darr	|i	|d	|k	|e	|n	|[][,]{ }	|*	|.
|*	|*	|*	|*	|b	|[23][S]\darr	|*	|*	|j	|*	|a	|*	|a	|n	|l	|m	|[24][S]\darr	|e	|*	|o	|o	|c	|o	|s	|*	|.
|*	|*	|*	|*	|i	|s	|*	|*	|ę	|*	|*	|[25][S]\drarr	|n	|i	|h	|i	|l	|i	|z	|m	|*	|y	|w	|c	|*	|.
|*	|*	|[26][S]\rarr	|m	|a	|t	|*	|*	|z	|*	|*	|ż	|y	|k	|o	|a	|a	|s	|*	|ó	|*	|d	|a	|e	|*	|.
|*	|[27][S]\darr	|*	|*	|ł	|y	|*	|*	|y	|*	|*	|ó	|*	|*	|*	|*	|k	|e	|*	|ż	|*	|o	|[][,]{ }	|n	|*	|.
|*	|m	|*	|*	|o	|m	|*	|*	|k	|[28][S]\rarr	|s	|ł	|o	|ń	|c	|e	|*	|n	|*	|d	|*	|w	|t	|a	|*	|.
|*	|i	|*	|*	|g	|u	|[29][S]\darr	|*	|[][,]{ }	|*	|*	|t	|[30][S]\rarr	|b	|u	|d	|v	|a	|*	|ż	|*	|a	|y	|*	|*	|.
|*	|e	|*	|*	|a	|l	|e	|[31][S]\darr	|c	|*	|*	|n	|*	|*	|*	|*	|*	|c	|*	|e	|*	|n	|p	|[32][S]\darr	|*	|.
|*	|s	|[33][S]\drarr	|p	|r	|a	|p	|ł	|e	|t	|w	|i	|e	|c	|*	|*	|*	|h	|[34][S]\darr	|*	|*	|i	|u	|d	|*	|.
|*	|z	|p	|*	|d	|c	|i	|u	|l	|*	|*	|c	|*	|[35][S]\darr	|*	|*	|*	|*	|r	|*	|*	|e	|[][,]{ }	|e	|*	|.
|*	|c	|r	|*	|ł	|j	|t	|k	|t	|[36][S]\rarr	|z	|a	|m	|k	|n	|i	|ę	|t	|o	|ś	|ć	|*	|i	|c	|*	|.
|*	|z	|o	|[37][S]\darr	|y	|a	|r	|[][,]{ }	|y	|*	|*	|*	|*	|w	|*	|*	|[38][S]\rarr	|r	|z	|u	|t	|*	|a	|e	|*	|.
|*	|a	|t	|d	|*	|[][,]{ }	|y	|w	|c	|[39][S]\rarr	|m	|u	|s	|a	|n	|*	|*	|*	|d	|*	|*	|*	|*	|n	|*	|.
|[40][S]\drarr	|n	|e	|o	|g	|o	|t	|y	|k	|*	|*	|[41][S]\rarr	|e	|s	|s	|e	|ń	|c	|z	|y	|k	|*	|*	|t	|*	|.
|b	|i	|k	|m	|*	|d	|*	|s	|i	|*	|*	|*	|*	|z	|[42][S]\rarr	|k	|a	|l	|i	|n	|a	|*	|*	|r	|*	|.
|a	|n	|t	|*	|*	|w	|*	|p	|*	|*	|*	|*	|[43][S]\rarr	|o	|f	|i	|c	|j	|a	|l	|i	|s	|t	|a	|*	|.
|d	|*	|o	|*	|[44][S]\darr	|i	|*	|o	|*	|[45][S]\rarr	|k	|l	|i	|n	|*	|*	|*	|[46][S]\darr	|l	|*	|*	|*	|*	|c	|*	|.
|a	|*	|r	|*	|p	|e	|[47][S]\drarr	|w	|ę	|g	|l	|ó	|w	|k	|a	|*	|*	|d	|i	|*	|*	|*	|*	|j	|*	|.
|j	|*	|k	|*	|o	|r	|p	|y	|*	|*	|*	|[48][S]\rarr	|w	|a	|r	|m	|i	|a	|k	|*	|*	|*	|*	|a	|*	|.
|o	|*	|a	|*	|r	|t	|a	|*	|[49][S]\rarr	|t	|r	|u	|t	|*	|*	|*	|*	|v	|*	|*	|*	|*	|*	|*	|*	|.
|z	|*	|*	|[50][S]\rarr	|a	|u	|t	|o	|n	|o	|m	|i	|z	|o	|w	|a	|n	|i	|e	|[][,]{ }	|s	|i	|ę	|*	|*	|.
|*	|*	|*	|*	|*	|*	|o	|*	|*	|*	|*	|*	|*	|*	|[51][S]\rarr	|m	|o	|d	|y	|s	|t	|k	|a	|*	|*	|.
|*	|*	|*	|*	|*	|*	|s	|*	|*	|*	|*	|*	|*	|*	|*	|*	|*	|*	|*	|*	|*	|*	|*	|*	|*	|.
|*	|*	|*	|*	|*	|*	|*	|*	|*	|*	|*	|*	|*	|*	|*	|*	|*	|*	|*	|*	|*	|*	|*	|*	|*	|.\end{Puzzle}

\newpage

\begin{PuzzleClues}{\textbf{Poziome}\\}\Clue{1}{}{buddyjski duchowny}
\Clue{2}{}{określenie wykonawcze; nagle, szybko}
\Clue{7}{}{malarz (1906-78) współzałożyciel grupy Pryzmat: postimpresjonistyczne pejzaże, sceny cyrkowe, portrety}
\Clue{9}{}{mąka powstała po zmieleniu jęczmienia; grubo zmielona, biało-beżowa}
\Clue{10}{}{część ornamentyki architektonicznej stanowiąca obraz malowany farbami, zwłaszcza na ścianie lub na drewnie, także na szkle}
\Clue{12}{}{etruskie naczynie do przechowywania pachnideł, kosztowności itp}
\Clue{15}{}{płaskorzeźba zdobiąca naczynie}
\Clue{19}{}{żartobliwie albo ironicznie o młodej dziewczynie}
\Clue{20}{}{twarzyczka, zwłaszcza dziecka lub ukochanej osoby}
\Clue{25}{}{postawa, w której człowiek neguje zasadność i cel wszystkiego, więc też nie poddaje się jakimkolwiek normom}
\Clue{26}{}{najniższy stopień podoficerski w marynarce}
\Clue{28}{}{gwiazda naszego układu gwiezdnego, główne ciało niebieskie Układu Słonecznego}
\Clue{30}{}{miejscowość w Jugosławii (Czarnogóra) słynne kąpielisko}
\Clue{33}{}{Protopterus - rodzaj dwudysznych ryb prapłaźcokształtnych z rodziny prapłetwcowatych (Protopteridae), obejmujący cztery gatunki; ryby zaliczane do tego rodzaju zamieszkują bagniste zbiorniki wodne Afryki tropikalnej}
\Clue{36}{}{cecha grupy ludzi, która jest hermetyczna - zamknięta na nowych członków}
\Clue{38}{}{w matematyce: punkt przecięcia prostej rzutującej z płaszczyzną}
\Clue{39}{}{miasto w KRL-D przy granicy z Chinami, największy w kraju ośrodek wydobycia i wzbogacania rud żelaza}
\Clue{40}{}{POSĄG, RZEŹBA, OBELISK itp. wystawiony ku czci osoby lub na pamiątkę jakiegoś wydarzenia; pomnik}
\Clue{41}{}{członek żydowskiego stronnictwa religijnego łączącego judaizm z irańskim zoroastryzmem (zaratustryzmem) i pitagoreizmem}
\Clue{42}{}{owoc (pestkowiec) kaliny koralowej, jadalny, jeśli zbierany po pierwszym przymrozku}
\Clue{43}{}{osoba zatrudniona przy zarządzaniu majątkiem ziemskim}
\Clue{45}{}{wielościan, którego podstawą jest prostokąt, a ścianami bocznymi dwa trapezy równoramienne i dwa trójkąty równoramienne}
\Clue{47}{}{model RC wykonany z włókna węglowego}
\Clue{48}{}{mieszkaniec Warmii, człowiek pochodzenia warmińskiego}
\Clue{49}{}{samiec pszczoły, truteń}
\Clue{50}{}{uzyskiwanie autonomii przez państwo}
\Clue{51}{}{kobieta, która robi damskie nakrycia głowy}\end{PuzzleClues}

\begin{PuzzleClues}{\textbf{Pionowe}\\}\Clue{1}{}{Platysteira peltata - gatunek ptaka z rodziny krępaczków (Platysteiridae)}
\Clue{2}{}{istniejący w judaizmie i większości wyznań chrześcijańskich pogląd o szczególnym związku Żydów z Bogiem, który jest niezmienny i przekazany w Starym i Nowym Testamencie}
\Clue{3}{}{miasto nad Styrem, na Ukrainie, w obwodzie wołyńskim, w rejonie horochowskim, do 1945 w Polsce, w województwie wołyńskim, w powiecie horochowskim, siedziba gminy Beresteczko}
\Clue{4}{}{zdarzenie nasycone chaosem, mające znamiona groteski, budzące przerażenie; najczęściej w l.mn}
\Clue{5}{}{inaczej borowik}
\Clue{6}{}{ptak z rzędu wróblowatych, owadożerny i ziarnojad, chroniony, żółtawo-rdzawo-brunatne upierzenie}
\Clue{7}{}{rodzina dużych motyli dziennych}
\Clue{8}{}{dyrygent radziecki (1903-1988); dyrektor filharmonii w Leningradzie, propagator muzyki rosyjskiej}
\Clue{11}{}{określenie stosowane głównie w Wielkiej Brytanii oznaczające otwarte baseny kąpielowe wraz z otaczającym je terenem}
\Clue{12}{}{ptak nadwodny z rzędu mew - siewek, owadożerny, w Polsce w czasie przelotów na wybrzeżu, chroniony}
\Clue{13}{}{brak zdecydowania - pewności, inicjatywy w podejmowaniu i realizowaniu decyzji}
\Clue{14}{}{odmiana supernowej, która powstaje w wyniku eksplozji białego karła}
\Clue{16}{}{znicz - szklane, metalowe bądź plastikowe naczynie wypełnione stearyną, w której zatopiony jest knot; zapalane na grobach w celu uczczenia pamięci  zmarłych}
\Clue{17}{}{ur. w 1951 r., brazylijski kompozytor i dyrygent, propagator muzyki współczesnej}
\Clue{18}{}{część mózgu, która jest z przodu}
\Clue{20}{}{lądowe siły zbrojne danego kraju; wojska lądowe}
\Clue{21}{}{język z grupy języków celtyckich, używany przez plemiona celtyckie żyjące w Galii, wyparty przez łacinę po podboju tych ziem przez Rzymian, około VI wieku n.e}
\Clue{22}{}{miasto w Niemczech (Turyngia) u podnóża Lasu Turyńskiego; fabryka samochodów Wartburg}
\Clue{23}{}{czynności prowadzące do zwiększenia przepływu ropy naftowej lub gazu ziemnego ze złoża do odwiertu}
\Clue{24}{}{Cheiranthus - rodzaj roślin z rodziny kapustowatych (krzyżowych)}
\Clue{25}{}{drewno pozyskiwane z drzewa żółtnicy pomarańczowej}
\Clue{27}{}{mieszkaniec miasta}
\Clue{29}{}{w metryce: stopa czterosylabowa}
\Clue{31}{}{łańcuch wysp i wysepek wulkanicznych ściągniętych na kształt łuku, powstający w wyniku intensywnych erupcji law w strefie Benioffa}
\Clue{32}{}{w psychologii - umiejętność wielopłaszczyznowego postrzegania spraw, z kilku perspektyw}
\Clue{33}{}{opiekunka, obrończyni}
\Clue{34}{}{zdrobniale o rozdziale - jednej z części, na które podzielona jest książka, dzieło literackie, naukowe itp}
\Clue{35}{}{kiszone warzywa lub (rzadziej) owoce, rodzaj przetworu spożywczego, dodatek do dań}
\Clue{37}{}{mieszkanie wraz z jego mieszkańcami}
\Clue{40}{}{miasto w Hiszpanii (Estremadura) nad Gwadianą, ośrodek administracyjny prowincji Badajoz}
\Clue{44}{}{czas, w którym coś ma się wydarzyć, ktoś ma coś zrobić}
\Clue{46}{}{malarz (1780-1840) portrety, sceny historyczne i religijne}
\Clue{47}{}{podniosły styl wypowiedzi, zazwyczaj traktującej o sprawach ważnych}\end{PuzzleClues}\newpage%\section*{Krzyżówka 71}

\noindent\begin{Puzzle}{21}{29}|*	|*	|*	|*	|[1][S]\drarr	|p	|o	|j	|a	|z	|d	|[][,]{ }	|s	|p	|e	|c	|j	|a	|l	|n	|y	|*	|.
|*	|*	|[2][S]\drarr	|s	|k	|r	|z	|y	|d	|l	|a	|t	|e	|[][,]{ }	|s	|ł	|o	|w	|o	|*	|[3][S]\darr	|*	|.
|*	|[4][S]\darr	|w	|[5][S]\rarr	|o	|r	|k	|i	|e	|s	|t	|r	|a	|c	|j	|a	|*	|*	|*	|*	|ł	|*	|.
|*	|s	|i	|[6][S]\rarr	|l	|i	|g	|a	|w	|k	|a	|*	|*	|*	|*	|*	|*	|*	|[7][S]\darr	|*	|a	|*	|.
|[8][S]\drarr	|m	|e	|t	|a	|l	|o	|c	|e	|r	|a	|m	|i	|k	|a	|*	|*	|*	|g	|*	|t	|*	|.
|p	|o	|r	|[9][S]\rarr	|r	|z	|e	|k	|o	|t	|k	|a	|[][,]{ }	|m	|i	|s	|t	|k	|o	|w	|a	|*	|.
|r	|k	|z	|*	|s	|*	|[10][S]\rarr	|m	|o	|d	|r	|a	|[][,]{ }	|k	|a	|p	|u	|s	|t	|a	|*	|*	|.
|z	|*	|y	|[11][S]\drarr	|t	|e	|n	|t	|a	|c	|j	|a	|*	|[12][S]\drarr	|p	|i	|c	|t	|o	|r	|*	|*	|.
|e	|*	|c	|k	|w	|*	|*	|*	|[13][S]\drarr	|o	|f	|i	|a	|r	|a	|*	|*	|[14][S]\darr	|w	|*	|*	|*	|.
|s	|*	|i	|r	|o	|[15][S]\rarr	|t	|a	|m	|a	|r	|i	|n	|a	|*	|*	|[16][S]\rarr	|m	|o	|*	|*	|*	|.
|t	|*	|e	|e	|[][,]{ }	|*	|*	|[17][S]\drarr	|a	|b	|s	|u	|r	|d	|a	|l	|n	|o	|ś	|ć	|*	|*	|.
|r	|*	|l	|b	|p	|[18][S]\drarr	|b	|e	|j	|s	|a	|*	|*	|*	|*	|*	|*	|s	|ć	|[19][S]\drarr	|f	|*	|.
|z	|*	|[][,]{ }	|s	|r	|r	|[20][S]\drarr	|k	|o	|k	|o	|r	|n	|a	|k	|*	|[21][S]\darr	|t	|*	|g	|*	|*	|.
|e	|*	|s	|*	|z	|y	|l	|s	|w	|*	|*	|*	|*	|*	|*	|*	|t	|e	|*	|r	|*	|*	|.
|ń	|*	|o	|[22][S]\darr	|e	|n	|i	|t	|i	|*	|[23][S]\drarr	|z	|n	|i	|c	|z	|e	|k	|*	|z	|[24][S]\darr	|*	|.
|[][,]{ }	|*	|l	|n	|ł	|k	|k	|r	|e	|*	|s	|*	|[25][S]\rarr	|ż	|a	|b	|a	|[][,]{ }	|b	|y	|k	|*	|.
|w	|*	|i	|i	|a	|a	|w	|u	|*	|*	|b	|[26][S]\darr	|*	|*	|[27][S]\rarr	|s	|t	|w	|*	|b	|i	|[28][S]\darr	|.
|e	|[29][S]\drarr	|d	|e	|j	|*	|i	|z	|*	|*	|*	|g	|*	|*	|*	|*	|r	|h	|*	|[][,]{ }	|n	|m	|.
|k	|o	|a	|c	|o	|*	|d	|j	|[30][S]\drarr	|k	|o	|l	|u	|m	|n	|a	|*	|e	|*	|ś	|a	|c	|.
|t	|n	|r	|z	|w	|*	|a	|a	|d	|[31][S]\rarr	|m	|o	|ł	|o	|d	|y	|c	|a	|*	|w	|z	|c	|.
|o	|t	|n	|y	|e	|[32][S]\darr	|c	|*	|o	|*	|[33][S]\rarr	|g	|ę	|ś	|*	|*	|*	|t	|*	|i	|a	|h	|.
|r	|o	|y	|n	|*	|w	|j	|*	|b	|*	|*	|i	|[34][S]\rarr	|p	|a	|t	|i	|s	|o	|n	|*	|e	|.
|o	|l	|*	|n	|*	|ł	|a	|[35][S]\drarr	|r	|d	|z	|e	|ń	|*	|*	|*	|*	|t	|*	|i	|*	|t	|.
|w	|o	|*	|o	|*	|o	|*	|m	|y	|[36][S]\rarr	|g	|r	|a	|[][,]{ }	|l	|o	|s	|o	|w	|a	|*	|a	|.
|a	|g	|*	|ś	|*	|s	|*	|m	|*	|*	|[37][S]\darr	|ó	|[38][S]\rarr	|t	|e	|c	|h	|n	|i	|k	|a	|*	|.
|*	|i	|*	|ć	|*	|ó	|*	|*	|*	|*	|t	|w	|*	|[39][S]\rarr	|o	|b	|i	|e	|g	|*	|*	|*	|.
|*	|z	|*	|*	|[40][S]\rarr	|w	|n	|i	|o	|s	|e	|k	|*	|*	|*	|*	|*	|[][S]'	|*	|*	|*	|*	|.
|*	|m	|[41][S]\rarr	|d	|e	|k	|o	|r	|t	|y	|k	|a	|c	|j	|a	|*	|*	|a	|*	|*	|*	|*	|.
|*	|*	|[42][S]\rarr	|g	|u	|a	|y	|a	|m	|a	|*	|*	|[43][S]\rarr	|ś	|l	|u	|z	|*	|*	|*	|*	|*	|.
|*	|*	|*	|*	|*	|*	|*	|*	|*	|*	|*	|*	|*	|*	|*	|*	|*	|*	|*	|*	|*	|*	|.\end{Puzzle}

\newpage

\begin{PuzzleClues}{\textbf{Poziome}\\}\Clue{1}{}{pojazd samochodowy lub przyczepa, przeznaczone do wykonywania specjalnej funkcji, która powoduje konieczność dostosowania nadwozia lub posiadania specj. wyposażenia; w pojeździe tym mogą być przewożone osoby i przedmioty związane z wykonywaniem tej funkcji}
\Clue{2}{}{powszechnie znana i często przytaczana wypowiedź, której autorstwo lub okoliczności powstania da się ustalić}
\Clue{5}{}{instrumentacja}
\Clue{6}{}{polski (mazowiecki) ludowy instrument dęty z grupy aerofonów}
\Clue{8}{}{wyroby wykonane metodą metaloceramiki}
\Clue{9}{}{rzekotka tragarz, Flectonotus goeldii - gatunek płaza bezogonowego z rodziny Hemiphractidae, znany z południowo-wschodniej Brazylii}
\Clue{10}{}{danie wielkopolskie i śląskie w postaci gotowanej czerwonej kapusty z dodatkami}
\Clue{11}{}{kuszenie, nęcenie}
\Clue{12}{}{MALARZ}
\Clue{13}{}{dar dla bóstwa, składany dla wywołania określonego efektu, np. w celu ochrony przed niezbezpieczeństwem}
\Clue{15}{}{tamaryna - nazwa zwyczajowa rodzaju małp szerokonosych}
\Clue{16}{}{w chemii: symbol molibdenu}
\Clue{17}{}{brak sensu, to, że coś jest nielogiczne, absurdalne, pozbawione wewnętrznej logiki, spójności}
\Clue{18}{}{oryks wschodnioafrykański, beisa, Oryx gazella beisa - duża antylopa z rodziny krętorogich; zamieszkuje górzyste tereny Afryki Wschodniej}
\Clue{19}{}{symbol jednostki pojemności elektrycznej w układzie SI}
\Clue{20}{}{pnącze strefy umiarkowanej, bylina o żółtych kwiatach, niektórymi gatunkami obsadza się pergole i altany}
\Clue{23}{}{ptak leśny z rzędu wróblowatych owadożerny, chroniony; Europa, płn. Afryka}
\Clue{25}{}{żaba rycząca, żaba wół, Lithobates catesbeianus, Rana catesbeiana - kosmopolityczny gatunek dużego płaza bezogonowego z rodziny żabowatych, pochodzący z Ameryki Północnej}
\Clue{27}{}{szczególna teoria względności - teoria fizyczna stworzona przez Alberta Einsteina w 1905 roku; zmieniła ona sposób pojmowania czasu i przestrzeni opisanych wcześniej w newtonowskiej mechanice klasycznej}
\Clue{29}{}{miasto w Rumunii, okręg Kluż; przemysł drzewny}
\Clue{30}{}{pionowy układ danych czy elementów w macierzy}
\Clue{31}{}{młoda kobieta, głównie Ukrainka lub Białorusinka}
\Clue{33}{}{ptak łowny z rzędu blaszkodziobych, roślinożerny}
\Clue{34}{}{owoc (nibyjagoda) rośliny o tej samej nazwie}
\Clue{35}{}{włókno; tkanka miękiszowa w łodygach roślin i drzew}
\Clue{36}{}{gra, której wynik nie jest możliwy do przewidzenia i nie istnieją strategie umożliwiające polepszenie swojego wyniku bez złamania zasad gry, np. ruletka, lotto}
\Clue{38}{}{warsztat, umiejętność}
\Clue{39}{}{ruch cieczy lub gazu w układzie termodynamicznie zamkniętym}
\Clue{40}{}{rezultat rozumowania przeprowadzonego na podstawie jakichś przesłanek}
\Clue{41}{}{usuwanie metodą mechaniczną zewnętrznej warstwy z łodyg lub ziaren do zastosowań włókienniczych}
\Clue{42}{}{miasto w płd. Portoryko; przemysł spożywczy}
\Clue{43}{}{lepka i ciągliwa substancja wydzielana przez gruczoły śluzowe}\end{PuzzleClues}

\begin{PuzzleClues}{\textbf{Pionowe}\\}\Clue{1}{}{rodzaj kolarstwa uprawianego na rowerach o konstrukcji zbliżonej do rowerów szosowych, a więc na kołach o rozmiarze 28 cali, lecz rozgrywanego na dość trudnych trasach terenowych obfitujących w liczne przeszkody, które zmuszają zawodnika do schodzenia z roweru i przenoszenia go}
\Clue{2}{}{wierzyciel, który odpowiada za ściągnięcie długu danego dłużnika, również wobec pozostałych jego wierzycieli}
\Clue{3}{}{kawałek naszytego materiału, skóry, który zasłania zniszczone, dziurawe miejsce lub pełni funkcję ozdobną}
\Clue{4}{}{samochód, który jest duży i spala dużo paliwa}
\Clue{7}{}{stan bycia gotowym na coś, co ma nastąpić, odpowiednie przygotowanie do czegoś}
\Clue{8}{}{w matematyce zbiór obiektów (nazywanych ,,wektorami), które mogą być, nieformalnie rzecz ujmując, skalowane i dodawane}
\Clue{11}{}{biochemik niemiecki (1900-81); opisał cykl przemian biochemicznych cukrów, tłuszczów i białek, laureat Nobla}
\Clue{12}{}{jednostka dawki absorbowanej (dawki pochłoniętej) promieniowania jonizującego; dawka absorbowana przez napromieniowane ciało o masie 1 g, jeżeli energia przekazana temu ciału przez cząstki jonizujące równa jest 100 erg czyli 10-5 J}
\Clue{13}{}{grupa ludów indiańskich mówiących językami z rodziny maja, zamieszkujących południowo-wschodni Meksyk (półwysep Jukatan i stan Chiapas), Gwatemalę, Belize i zachodni Honduras}
\Clue{14}{}{mostek pracujący w punkcie równowagi}
\Clue{17}{}{w ortodoncji: korekta mająca na celu wydłużenie części naddziąsłowej zęba; wyciskanie zębów z dziąsła, wydłużanie zębów poprzez wyciągnięcie ich części ukrytej w dziąśle}
\Clue{18}{}{broń spokrewniona ze spisą}
\Clue{19}{}{Suillellus luridus - gatunek grzybów  z rodziny borowikowatych o brunatnym kapeluszu i poduchowatym kształcie}
\Clue{20}{}{pozbycie się czegoś, usunięcie czegoś}
\Clue{21}{}{instytucja, organizacja lub zespół ludzi pracujących na jej rzecz,  które (którzy) podejmują działania związane z przygotowaniem i obsługą widowisk teatralnych}
\Clue{22}{}{cecha czegoś, co nie świadczy usług, jest zamknięte}
\Clue{23}{}{w chemii: symbol antymonu}
\Clue{24}{}{enzym należący do transferaz, katalizujący reakcję przeniesienia grupy fosforanowej ze związku wysokoenergetycznego (np. ATP) na właściwą cząsteczkę docelową}
\Clue{26}{}{PEPINKA odporna na mróz odmiana jabłoni}
\Clue{28}{}{miasto we wsch. części Gruzji, przy ujściu Aragwi do Kury, do VI w stolica Iberii}
\Clue{29}{}{kierunek w naukach filozoficznych, zgodnie z którym Bóg jest wyznacznikiem do formułowania jakiegokolwiek sądu}
\Clue{30}{}{nazwa stopnia szkolnego; ocena pomiędzy bardzo dobrym a dostatecznym}
\Clue{32}{}{Eriochloa - rodzaj rośliny z rodziny wiechlinowatych}
\Clue{35}{}{mila morska - jednostka odległości stosowana w nawigacji morskiej oraz lotnictwie; jednej mili morskiej odpowiadają 1852 metry, czyli uśredniona długość łuku południka ziemskiego odpowiadająca jednej minucie kątowej koła wielkiego}
\Clue{37}{}{drzewo tekowe służące do budowy okrętów}\end{PuzzleClues}\newpage%\section*{Krzyżówka 72}

\noindent\begin{Puzzle}{19}{28}|*	|[1][S]\drarr	|p	|a	|r	|a	|g	|a	|m	|m	|a	|c	|y	|z	|m	|*	|*	|*	|*	|*	|.
|[2][S]\drarr	|s	|t	|a	|r	|a	|*	|[3][S]\drarr	|b	|a	|l	|w	|i	|e	|r	|z	|*	|*	|*	|*	|.
|s	|k	|[4][S]\drarr	|c	|h	|r	|u	|s	|t	|*	|*	|*	|*	|[5][S]\drarr	|g	|a	|z	|i	|k	|*	|.
|u	|a	|t	|*	|*	|*	|*	|o	|[6][S]\drarr	|b	|i	|e	|d	|a	|*	|*	|*	|*	|*	|*	|.
|e	|ł	|r	|*	|[7][S]\rarr	|b	|i	|l	|e	|c	|i	|k	|*	|n	|*	|*	|*	|*	|*	|*	|.
|*	|ó	|z	|*	|*	|*	|*	|a	|l	|[8][S]\rarr	|s	|z	|l	|a	|g	|i	|e	|r	|*	|[9][S]\darr	|.
|[10][S]\drarr	|w	|y	|t	|w	|ó	|r	|n	|i	|a	|[][,]{ }	|f	|i	|l	|m	|o	|w	|a	|*	|k	|.
|s	|k	|m	|[11][S]\rarr	|s	|e	|z	|a	|m	|*	|*	|[12][S]\drarr	|g	|i	|w	|e	|r	|*	|*	|r	|.
|a	|a	|a	|[13][S]\rarr	|m	|o	|l	|*	|i	|*	|*	|f	|*	|z	|[14][S]\darr	|[15][S]\darr	|*	|*	|*	|o	|.
|m	|*	|k	|[16][S]\rarr	|b	|l	|i	|ź	|n	|i	|a	|k	|*	|a	|s	|n	|*	|*	|*	|n	|.
|o	|*	|*	|*	|[17][S]\drarr	|u	|k	|ł	|a	|d	|*	|p	|*	|t	|a	|i	|[18][S]\darr	|*	|*	|i	|.
|g	|*	|*	|[19][S]\drarr	|p	|a	|p	|a	|c	|h	|a	|*	|*	|o	|t	|u	|k	|*	|*	|k	|.
|ł	|[20][S]\drarr	|z	|i	|a	|r	|n	|o	|j	|a	|d	|*	|[21][S]\rarr	|r	|y	|s	|a	|k	|*	|a	|.
|o	|j	|*	|n	|p	|*	|[22][S]\drarr	|t	|e	|l	|o	|m	|*	|[][,]{ }	|r	|*	|ł	|[23][S]\darr	|[24][S]\darr	|r	|.
|s	|a	|*	|t	|i	|*	|t	|*	|*	|*	|[25][S]\rarr	|s	|m	|s	|*	|*	|a	|m	|b	|k	|.
|k	|s	|*	|e	|e	|*	|a	|*	|*	|[26][S]\rarr	|j	|u	|k	|k	|a	|*	|m	|o	|e	|a	|.
|a	|t	|*	|r	|r	|*	|m	|[27][S]\drarr	|p	|r	|z	|e	|d	|ł	|u	|ż	|a	|c	|z	|*	|.
|[][,]{ }	|r	|*	|r	|o	|*	|a	|d	|[28][S]\rarr	|s	|z	|t	|r	|a	|s	|e	|r	|*	|i	|*	|.
|ś	|z	|[29][S]\darr	|e	|ś	|*	|r	|a	|*	|*	|[30][S]\drarr	|l	|i	|d	|o	|*	|z	|[31][S]\darr	|n	|*	|.
|c	|ą	|f	|g	|n	|*	|i	|r	|*	|*	|m	|*	|*	|n	|*	|[32][S]\darr	|*	|h	|w	|*	|.
|i	|b	|o	|n	|i	|*	|l	|t	|*	|[33][S]\rarr	|ł	|u	|p	|i	|s	|k	|ó	|r	|a	|*	|.
|e	|*	|t	|u	|c	|*	|l	|*	|*	|*	|y	|*	|*	|o	|*	|o	|*	|a	|z	|*	|.
|ś	|*	|o	|m	|a	|[34][S]\rarr	|o	|w	|o	|d	|n	|i	|o	|w	|c	|e	|*	|b	|y	|*	|.
|n	|*	|t	|*	|*	|*	|*	|*	|[35][S]\rarr	|d	|o	|k	|*	|y	|*	|n	|*	|a	|j	|*	|.
|i	|*	|y	|[36][S]\rarr	|j	|a	|s	|t	|r	|o	|w	|i	|e	|*	|*	|d	|*	|l	|n	|*	|.
|o	|[37][S]\rarr	|p	|o	|ż	|y	|t	|e	|c	|z	|n	|o	|ś	|ć	|*	|u	|*	|*	|o	|*	|.
|n	|*	|*	|[38][S]\rarr	|i	|p	|o	|d	|[][,]{ }	|v	|i	|d	|e	|o	|*	|*	|*	|*	|ś	|*	|.
|a	|*	|[39][S]\rarr	|s	|t	|e	|r	|e	|o	|t	|a	|k	|s	|j	|a	|*	|*	|*	|ć	|*	|.
|*	|[40][S]\rarr	|c	|y	|t	|a	|d	|e	|l	|a	|*	|*	|[41][S]\rarr	|l	|u	|d	|e	|k	|*	|*	|.\end{Puzzle}

\newpage

\begin{PuzzleClues}{\textbf{Poziome}\\}\Clue{1}{}{wada wymowy charakteryzująca się opuszczaniem głoski g, najczęściej na początku wyrazu (w nagłosie)}
\Clue{2}{}{czyjaś matka, mama}
\Clue{3}{}{cyrulik, golarz}
\Clue{4}{}{suche części drzewa wykorzystywane do rozpalania ognia lub innych celów, np. wyrobu mioteł}
\Clue{5}{}{część opatrunku, w formie gęstej siateczki wykonanej z materiału, jałowej}
\Clue{6}{}{inaczej kłopot, problem}
\Clue{7}{}{zdrobniale bilet}
\Clue{8}{}{przebój, hit}
\Clue{10}{}{przedsiębiorstwo, które tworzy filmy}
\Clue{11}{}{roślina z sezamowatych pochodząca z Afryki, oleista, uprawiana dla nasion do otrzymywania oleju spożywczego}
\Clue{12}{}{kołpak szeregowego żołnierza kawalerii narodowej}
\Clue{13}{}{drobny motyl nocny: gąsienice żerują na grzybach włosach produktach wełnianych}
\Clue{16}{}{budowla, w której jedna ze ścian zewnętrznych budynku przylega do drugiego obiektu, a pozostałe elewacje ustawione są swobodnie}
\Clue{17}{}{w anatomii: zespół narządów współpracujących w wykonywaniu danej funkcji organizmu}
\Clue{19}{}{męska czapka futrzana noszona przez mieszkańców Kaukazu i Kozaków, obecnie też przez wyższych oficerów}
\Clue{20}{}{ptak żywiący się ziarnami}
\Clue{21}{}{koń przyuczony do bardzo szybkiego kłusa}
\Clue{22}{}{końcowa część widlastego rozgałęzienia bezlistnych pędów kopalnych paprotników psylofitów}
\Clue{25}{}{SMS - krótka wiadomość tekstowa, wysyłana przez telefon}
\Clue{26}{}{Yucca - rodzaj roślin z rodziny agawowatych}
\Clue{27}{}{formularz, który może służyć jako dodatkowe miejsce na tekst w dokumentacji urzędowej; dodatkowy arkusz, dzięki któremu można umieścić w dokumentacji urzędowej dłuższy niż przeciętnie tekst}
\Clue{28}{}{rasa gołębia hodowlanego z grupy olbrzymów}
\Clue{30}{}{piaszczysty, wynurzony wał nadbudowany od strony morza przez fale, powoduje powstawanie laguny}
\Clue{33}{}{złodziej, zdzierca}
\Clue{34}{}{Amniota - klad obejmujący kręgowce mające zdolność rozwoju zarodkowego na lądzie (gady, ptaki i ssaki); uzyskały ją dzięki wytworzeniu błon płodowych, które tworzą środowisko dla właściwego rozwoju zarodka}
\Clue{35}{}{budowla lub pływające urządzenie w stoczni przeznaczone do budowy lub remontu jednostek pływających}
\Clue{36}{}{miasto w województwie wielkopolskim, w powiecie złotowskim, siedziba gminy miejsko-wiejskiej Jastrowie}
\Clue{37}{}{to, że coś jest pożyteczne, przynosi korzyści, przydaje się}
\Clue{38}{}{iPod umożliwiający oglądanie filmów}
\Clue{39}{}{stosowana w medycynie metoda, która polega na wprowadzaniu do danych miejsc w mózgu urządzenia (najczęściej elektrody), aby zaktywizować lub uszkodzić ośrodek mózgowy}
\Clue{40}{}{centralna bateria dział znajdująca się na śródokręciu statku, silnie opancerzona}
\Clue{41}{}{niewielka grupa ludzi, mały tłum}\end{PuzzleClues}

\begin{PuzzleClues}{\textbf{Pionowe}\\}\Clue{1}{}{północnoamerykański ptak z rodziny jaskółek}
\Clue{2}{}{(1804-57) pisarz francuski, sensacyjne opowieści z życia ludności Paryża; „Tajemnice Paryża”}
\Clue{3}{}{malarz hiszpański (1885-1945) obraz z życia ubogich dzielnic Madrytu oraz obrazujące dawne zwyczaje obrzędy}
\Clue{4}{}{przyrząd do przytrzymywania przedmiotów w czasie obrabiania lub szlifowania przedmiotów}
\Clue{5}{}{program dokonujący analizy składniowej danych wejściowych w celu określenia ich struktury gramatycznej w związku z określoną gramatyką formalną}
\Clue{6}{}{etap konkursu, w którym eliminuje się słabszych uczestników, pretendentów do czegoś}
\Clue{9}{}{pisarka, historiografka zajmująca się pisaniem kronik}
\Clue{10}{}{samogłoska długa, która na skutek podwyższenia artykulacji zmieniła swoją barwę, upodabniając się do samogłoski krótkiej}
\Clue{12}{}{kod ISO 4217 funta falklandzkiego}
\Clue{14}{}{w mitologii greckiej: demon leśny, bóstwo płodności, wyobrażane jako istota silnej budowy, ze zmierzwionymi włosami, płaskim nosem, spiczastymi uszami, górną połowa ciała ludzką, dolną - zwierzęcą}
\Clue{15}{}{potocznie o newsie, nowinie, świeżej informacji}
\Clue{17}{}{zazwyczaj ozdobne pudełko, etui, pojemnik, który służy do przechowywania papierosów, cygar}
\Clue{18}{}{wymienny zbiornik na tusz w drukarce lub maszynie drukarskiej, zasobnik}
\Clue{19}{}{hist. polit}
\Clue{20}{}{kosmopolityczny ptak drapieżny, poluje na drobne ssaki i ptaki}
\Clue{22}{}{Cyphomandra betacea, cyfomandra grubolistna - gatunek rośliny z rodziny psiankowatych}
\Clue{23}{}{stopień udziału substancji chemicznych w mieszaninach i roztworach}
\Clue{24}{}{w medycynie: cecha działania (diagnozowania, terapii itp.), polegająca na tym, że nie narusza ono struktury narządów wewnętrznych, nie wymaga ingerencji (np. chirurgicznej) w tę strukturę}
\Clue{27}{}{amerykański, przeciwpancerny pocisk rakietowy}
\Clue{29}{}{typ skóry w zależności od ilości zawartej w niej melaniny}
\Clue{30}{}{fabryka nawozów sztucznych}
\Clue{31}{}{pisarz czeski, ur. 1914r, nowele, powieści; „Pociągi pod specjalnym nadzorem” „Postrzyżyny”, „Bar świata”, „Chrzest”}
\Clue{32}{}{kuandu, Coendou prehensilis - gatunek gryzonia z rodziny ursonowatych; zamieszkuje lasy Ameryki Środkowej i Południowej}\end{PuzzleClues}\newpage%\section*{Krzyżówka 73}

\noindent\begin{Puzzle}{22}{31}|*	|*	|*	|*	|*	|*	|*	|*	|*	|[1][S]\drarr	|u	|r	|i	|*	|*	|*	|*	|*	|*	|*	|*	|*	|*	|.
|*	|*	|*	|*	|[2][S]\drarr	|j	|a	|g	|o	|d	|z	|i	|a	|k	|[][,]{ }	|c	|i	|e	|m	|n	|y	|*	|*	|.
|*	|*	|*	|*	|a	|[3][S]\rarr	|o	|b	|r	|z	|y	|d	|l	|i	|w	|o	|ś	|ć	|*	|*	|*	|*	|*	|.
|*	|*	|*	|*	|p	|*	|[4][S]\rarr	|a	|k	|i	|y	|a	|m	|a	|*	|*	|*	|*	|*	|*	|*	|*	|*	|.
|[5][S]\drarr	|f	|i	|g	|a	|*	|*	|*	|[6][S]\rarr	|k	|a	|z	|u	|a	|r	|*	|*	|*	|*	|*	|*	|*	|*	|.
|t	|[7][S]\drarr	|w	|y	|r	|a	|ż	|e	|n	|i	|e	|[][,]{ }	|p	|r	|z	|y	|i	|m	|k	|o	|w	|e	|*	|.
|e	|z	|*	|*	|t	|*	|[8][S]\drarr	|n	|i	|e	|s	|k	|r	|ę	|p	|o	|w	|a	|n	|i	|e	|*	|*	|.
|l	|b	|[9][S]\rarr	|h	|a	|o	|m	|a	|*	|[][,]{ }	|[10][S]\drarr	|b	|i	|n	|a	|r	|n	|o	|ś	|ć	|*	|*	|*	|.
|l	|o	|[11][S]\darr	|*	|m	|[12][S]\darr	|i	|*	|*	|p	|k	|*	|*	|[13][S]\darr	|[14][S]\darr	|[15][S]\darr	|*	|[16][S]\drarr	|p	|r	|u	|s	|*	|.
|u	|r	|z	|*	|e	|c	|e	|*	|*	|o	|w	|*	|*	|w	|d	|g	|[17][S]\rarr	|s	|c	|r	|*	|*	|*	|.
|r	|n	|w	|[18][S]\drarr	|n	|i	|d	|e	|r	|l	|a	|n	|d	|y	|*	|r	|[19][S]\rarr	|z	|a	|n	|i	|k	|*	|.
|e	|o	|i	|e	|t	|a	|ź	|[20][S]\darr	|[21][S]\drarr	|a	|s	|y	|m	|p	|t	|o	|t	|a	|*	|*	|*	|*	|*	|.
|k	|r	|e	|d	|*	|ł	|*	|r	|t	|*	|[][,]{ }	|*	|*	|a	|*	|d	|[22][S]\drarr	|d	|r	|e	|s	|y	|*	|.
|[][,]{ }	|a	|r	|o	|*	|o	|[23][S]\darr	|e	|u	|[24][S]\rarr	|c	|n	|*	|l	|*	|z	|p	|o	|*	|*	|*	|*	|*	|.
|z	|k	|z	|*	|*	|[][,]{ }	|p	|c	|k	|[25][S]\drarr	|h	|a	|l	|a	|*	|a	|i	|k	|*	|*	|*	|*	|*	|.
|ł	|i	|ę	|[26][S]\darr	|*	|p	|ę	|e	|a	|z	|l	|[27][S]\darr	|[28][S]\darr	|r	|*	|*	|e	|*	|[29][S]\darr	|[30][S]\drarr	|l	|x	|*	|.
|o	|*	|[][,]{ }	|j	|*	|o	|t	|p	|n	|l	|o	|r	|a	|k	|*	|[31][S]\darr	|p	|*	|o	|m	|*	|[32][S]\darr	|*	|.
|t	|[33][S]\drarr	|k	|a	|m	|l	|o	|t	|*	|e	|r	|e	|b	|a	|*	|l	|r	|[34][S]\drarr	|p	|u	|c	|h	|*	|.
|a	|b	|o	|r	|[35][S]\drarr	|i	|*	|a	|[36][S]\rarr	|w	|o	|d	|a	|*	|*	|e	|z	|b	|c	|n	|*	|y	|*	|.
|*	|e	|p	|o	|e	|k	|*	|*	|*	|*	|g	|a	|*	|*	|*	|j	|n	|o	|j	|s	|*	|ć	|*	|.
|[37][S]\drarr	|n	|a	|s	|t	|r	|o	|s	|z	|*	|e	|*	|*	|*	|*	|b	|i	|j	|a	|z	|[38][S]\darr	|k	|*	|.
|s	|e	|l	|z	|b	|y	|[39][S]\rarr	|s	|ł	|o	|n	|e	|c	|z	|n	|i	|c	|e	|*	|t	|z	|a	|*	|.
|ł	|s	|n	|*	|*	|s	|[40][S]\rarr	|w	|y	|g	|o	|n	|*	|*	|*	|k	|z	|r	|*	|u	|i	|*	|*	|.
|o	|i	|e	|*	|[41][S]\rarr	|t	|r	|a	|ł	|*	|w	|*	|*	|*	|*	|*	|k	|*	|*	|c	|e	|*	|*	|.
|ń	|ć	|*	|*	|*	|a	|*	|[42][S]\rarr	|a	|r	|y	|s	|t	|o	|k	|r	|a	|t	|y	|z	|m	|*	|*	|.
|*	|*	|[43][S]\rarr	|k	|o	|l	|u	|m	|n	|a	|*	|[44][S]\rarr	|a	|r	|k	|a	|*	|*	|*	|e	|n	|*	|*	|.
|[45][S]\rarr	|r	|ó	|ż	|n	|i	|c	|o	|w	|a	|n	|i	|e	|[][,]{ }	|s	|i	|ę	|*	|*	|k	|i	|*	|*	|.
|[46][S]\rarr	|j	|a	|s	|z	|c	|z	|u	|r	|k	|a	|[][,]{ }	|z	|[][,]{ }	|i	|b	|i	|z	|y	|*	|a	|*	|*	|.
|[47][S]\rarr	|p	|ó	|ł	|d	|z	|i	|e	|w	|i	|c	|a	|*	|[48][S]\rarr	|r	|z	|e	|ź	|n	|i	|k	|*	|*	|.
|[49][S]\rarr	|w	|a	|g	|o	|n	|[][,]{ }	|p	|r	|z	|e	|d	|z	|i	|a	|ł	|o	|w	|y	|*	|*	|*	|*	|.
|*	|[50][S]\rarr	|d	|z	|i	|e	|r	|z	|y	|k	|[][,]{ }	|ż	|ó	|ł	|t	|o	|ł	|b	|i	|s	|t	|y	|*	|.
|[51][S]\rarr	|a	|r	|e	|a	|*	|*	|*	|*	|*	|*	|*	|*	|*	|*	|*	|*	|*	|*	|*	|*	|*	|*	|.\end{Puzzle}

\newpage

\begin{PuzzleClues}{\textbf{Poziome}\\}\Clue{1}{}{kanton w środkowej Szwajcarii w Alpach, stolica Altdore, powierzchnia 1,1 tyś. km2}
\Clue{2}{}{Melanocharis arfakiana - gatunek ptaka z rodziny jagodziaków (Melanocharitidae)}
\Clue{3}{}{skłonność do odczuwania obrzydzenia}
\Clue{4}{}{kosmonauta japoński, odbył lot w ramach programu „Interkosmos” na pokładzie SojuzaTM-11}
\Clue{5}{}{gest zaciśniętej pięści, który pokazuje się, gdy chce się komuś zakomunikować, że nie otrzyma tego, czego się spodziewał}
\Clue{6}{}{duży ptak podobny do strusia, nielatający, o cennym upierzeniu, zamieszkuje lasy Australii, Nowej Gwinei, Tasmanii; 'struś latający'}
\Clue{7}{}{połączenie przyimka z rzeczownikiem, liczebnikiem, przysłówkiem, przymiotnikiem (w użyciu rzeczownym) albo zaimkiem}
\Clue{8}{}{cecha czegoś, co jest nieskrępowane, wolne, nie musi nikomu ani niczemu podlegać, może samo o sobie stanowić}
\Clue{9}{}{napój spożywany podczas rytuałów w religii irańskiej}
\Clue{10}{}{zasada stosowana w językoznawstwie do podziału struktur językowych na dwa tak zwane bezpośrednie składniki, które następnie również można dzielić na dwie części}
\Clue{16}{}{właściwie Głowacki - pisarz i publicysta (1847-1912), humoreski, nowele, powieści; „Placówka”, „Emancypantki”, „Lalka”, „Faraon”, „Antek”, „Kamizelka”, „Katarynka”}
\Clue{17}{}{kod ISO 4217 rupii seszelskiej}
\Clue{18}{}{historyczna nazwa terenów u ujścia Reny, Mozy i Skaldy, obecnie Belgia, Holandia, Luksemburg}
\Clue{19}{}{znikanie czegoś, zamieranie, np. zanik aktywności}
\Clue{21}{}{prosta, której odległość od krzywej maleje w miarę przesuwania się wzdłuż tej prostej}
\Clue{22}{}{spodnie dresowe}
\Clue{24}{}{w chemii: symbol kopernika}
\Clue{25}{}{piętro roślinności górskiej, obszar występowania łąk wysokogórskich zwanych halami}
\Clue{30}{}{jednostka natężenia oświetlenia E w układzie SI (jednostka pochodna układu SI)}
\Clue{33}{}{ciepła tkanina z wielbłądziej wełny używana głównie do szycia okryć wierzchnich}
\Clue{34}{}{lekkie, delikatne pierze, miękkie pióra, posiadające cienką oś, długie i miękkie promienie oraz promyki, które nie łączą się ze sobą}
\Clue{35}{}{w chemii: symbol jodu}
\Clue{36}{}{zbiornik wodny lub ciek występujący naturalnie lub utworzony sztucznie w przyrodzie}
\Clue{37}{}{nazwa motyla z rodziny zawisakowatych}
\Clue{39}{}{Eurypygidae - rodzina ptaków z rzędu słonecznicowych (Eurypygiformes)}
\Clue{40}{}{droga, którą prowadzi się bydło na pastwisko}
\Clue{41}{}{sieć rybacka w kształcie dużego worka ciągniona za lin}
\Clue{42}{}{zachowanie przywodzące na myśl arystokrację - dobre maniery, kultura}
\Clue{43}{}{pionowy układ danych czy elementów w macierzy}
\Clue{44}{}{archaiczne określenie skrzyni, szczególnie w biblii; drewniany lub metalowy pojemnik o kształcie prostopadłościanu z wypukłym wiekiem}
\Clue{45}{}{wyodrębnianie się grup z całości społeczeństwa}
\Clue{46}{}{Podarcis pityusensis - gatunek gada z rodziny jaszczurkowatych, endemit Balearów i pobliskich małych, przybrzeżnych wysp na wybrzeżu Hiszpanii}
\Clue{47}{}{dziewczyna, która ma kontakty seksualne z mężczyznami, ale zachowuje dziewictwo}
\Clue{48}{}{morderca, znęcający się nad ofiarami}
\Clue{49}{}{wagon pasażerski podzielony na przedziały}
\Clue{50}{}{Laniarius mufumbiri - gatunek ptaka  z rodziny dzierzbików (Malaconotidae)}
\Clue{51}{}{w starożytnym Rzymie plac dookoła świątyni lub gmachu publicznego}\end{PuzzleClues}

\begin{PuzzleClues}{\textbf{Pionowe}\\}\Clue{1}{}{staropolska nazwa obecnego Zaporoża}
\Clue{2}{}{luksusowe mieszkanie lub zespół pomieszczeń przeznaczonych dla jednego użytkownika}
\Clue{5}{}{sól o wzorze AuTe2}
\Clue{7}{}{Syncarida - nadrząd skorupiaków z podgromady pancerzowców właściwych}
\Clue{8}{}{rudy kolor, podobny do barwy miedzi - metalu}
\Clue{10}{}{organiczny związek chemiczny z grupy polifenoli, ester kwasu chinowego i kwasu kawowego}
\Clue{11}{}{zwierzę zachowane w głębi ziemi w postaci skamieniałości, pochodzące z minionych epok geologicznych}
\Clue{12}{}{ciało stałe będące zlepkiem wielu monokryształów}
\Clue{13}{}{nagrywarka - urządzenie służące do zapisywania informacji na przeznaczonych do tego celu dyskach optycznych, inaczej płytach}
\Clue{14}{}{litera alfabetu używana w numeracji porządkowej}
\Clue{15}{}{GROBLA; KOFERDAM}
\Clue{16}{}{POMPELA wiecznie zielone drzewo z rodziny rutowatych, o dużych, jadalnych owocach gruszkowatego kształtu}
\Clue{18}{}{dawna stolica Japonii, od 1868 r. zwane Tokio}
\Clue{20}{}{przepis - spisana w krokach procedura wykonania czegoś}
\Clue{21}{}{PIEPRZOJAD; ptak z rzędu dzięciołowatych charakteryzujący się barwnym upierzeniem i bardzo dużym, równie barwnym dziobem, żyje w dziuplach, w tropikalnych lasach Ameryki Płd. i Środkowej}
\Clue{22}{}{małe naczynie do przechowywania pieprzu}
\Clue{23}{}{więzy, krępujące kończyny lub ograniczjące zasięg ruchu jakie zakładano ludziom bądź zwierzętom}
\Clue{25}{}{zlewozmywak - funkcjonalny odpowiednik umywalki (ujęcie wody) zamontowany w kuchni, służący głównie do mycia naczyń oraz produktów spożywczych i usuwania zbędnych płynów, pozostałych z gotowania potraw}
\Clue{26}{}{WEGETARIANIN}
\Clue{27}{}{kaszubskie miasto w północnej Polsce, w województwie pomorskim, w powiecie wejherowskim, położone w Pradolinie Redy-Łeby nad rzeką Redą}
\Clue{28}{}{miasto w południowej Nigerii, ważny ośrodek handlowy: przemysł spożywczy włókienniczy, chemiczny}
\Clue{29}{}{instrument finansowy o niesymetrycznym profilu wypłaty, mogący być przedmiotem obrotu na giełdzie}
\Clue{30}{}{zdrobniale: munsztuk}
\Clue{31}{}{w dawnym wojsku górna część munduru (bluza), na który zakładano spencerek}
\Clue{32}{}{w gwarze wielkopolskiej: sok z owoców czarnego bzu}
\Clue{33}{}{pisarz chorwacki (1883-1957), tłumacz literatury polskiej}
\Clue{34}{}{mały przybrzeżny żaglowiec holenderski}
\Clue{35}{}{kod ISO 4217 waluty birr}
\Clue{37}{}{potężny ssak roślinożerny z chwytną trąbą powstałą ze zrośnięcia nosa z górną wargą}
\Clue{38}{}{warzywo, jadalna bulwa rośliny nazywanej tak samo}\end{PuzzleClues}\newpage%\section*{Krzyżówka 74}

\noindent\begin{Puzzle}{23}{33}|*	|*	|*	|[1][S]\darr	|*	|*	|*	|*	|*	|*	|*	|*	|*	|*	|*	|*	|*	|*	|*	|*	|*	|*	|*	|*	|.
|*	|*	|*	|o	|*	|*	|*	|*	|*	|*	|*	|[2][S]\drarr	|w	|u	|l	|k	|a	|n	|o	|d	|o	|n	|*	|*	|.
|*	|*	|[3][S]\rarr	|b	|i	|o	|l	|o	|g	|i	|c	|z	|n	|y	|[][,]{ }	|o	|j	|c	|i	|e	|c	|*	|*	|*	|.
|*	|*	|*	|r	|*	|[4][S]\rarr	|c	|i	|e	|m	|n	|y	|[][,]{ }	|p	|r	|z	|e	|p	|ł	|y	|w	|*	|[5][S]\darr	|*	|.
|*	|*	|*	|a	|[6][S]\rarr	|t	|a	|m	|w	|o	|r	|s	|y	|*	|[7][S]\drarr	|r	|a	|k	|ó	|w	|*	|*	|z	|*	|.
|*	|*	|*	|z	|*	|*	|*	|*	|*	|[8][S]\darr	|*	|k	|*	|[9][S]\rarr	|p	|i	|s	|u	|a	|r	|*	|*	|d	|*	|.
|*	|*	|*	|*	|*	|[10][S]\drarr	|b	|a	|r	|w	|a	|[][,]{ }	|h	|e	|r	|a	|l	|d	|y	|c	|z	|n	|a	|*	|.
|*	|*	|*	|*	|*	|j	|*	|*	|*	|a	|*	|k	|*	|*	|z	|[11][S]\darr	|[12][S]\drarr	|s	|a	|j	|g	|o	|n	|*	|.
|*	|[13][S]\darr	|*	|[14][S]\rarr	|m	|o	|s	|t	|e	|k	|*	|s	|[15][S]\darr	|*	|e	|s	|p	|*	|*	|*	|*	|*	|i	|*	|.
|*	|ó	|*	|*	|*	|d	|*	|[16][S]\drarr	|m	|a	|l	|i	|n	|ó	|w	|k	|a	|*	|*	|*	|*	|*	|e	|*	|.
|*	|s	|*	|*	|*	|*	|*	|s	|*	|t	|*	|ę	|i	|*	|ó	|u	|ń	|[17][S]\darr	|*	|*	|[18][S]\darr	|*	|[][,]{ }	|*	|.
|*	|m	|*	|*	|*	|*	|*	|z	|*	|*	|*	|g	|e	|*	|d	|b	|s	|g	|*	|*	|f	|*	|e	|*	|.
|*	|a	|[19][S]\rarr	|f	|i	|l	|e	|t	|*	|*	|*	|o	|w	|*	|[][,]{ }	|a	|t	|a	|[20][S]\darr	|[21][S]\darr	|e	|*	|g	|*	|.
|*	|[][,]{ }	|*	|[22][S]\rarr	|o	|w	|e	|r	|o	|l	|*	|w	|ó	|[23][S]\darr	|g	|n	|w	|m	|d	|l	|d	|*	|z	|*	|.
|*	|c	|[24][S]\darr	|[25][S]\drarr	|n	|i	|c	|a	|*	|*	|[26][S]\darr	|y	|d	|t	|r	|k	|o	|b	|r	|u	|e	|*	|a	|*	|.
|*	|z	|w	|d	|[27][S]\darr	|*	|[28][S]\darr	|b	|*	|*	|z	|*	|*	|e	|z	|a	|[][,]{ }	|i	|a	|b	|r	|*	|m	|*	|.
|*	|ę	|i	|z	|z	|*	|p	|a	|[29][S]\darr	|*	|o	|*	|*	|r	|e	|*	|u	|t	|k	|i	|a	|*	|i	|*	|.
|[30][S]\drarr	|ś	|c	|i	|a	|n	|a	|*	|j	|*	|r	|*	|*	|m	|j	|*	|n	|[][,]{ }	|o	|n	|l	|*	|n	|*	|.
|m	|ć	|i	|e	|c	|[31][S]\darr	|l	|*	|a	|*	|t	|*	|*	|i	|n	|*	|i	|p	|n	|i	|i	|*	|u	|*	|.
|o	|*	|o	|s	|i	|w	|i	|[32][S]\rarr	|w	|c	|z	|e	|s	|n	|y	|[][,]{ }	|t	|r	|i	|a	|s	|*	|*	|*	|.
|d	|[33][S]\drarr	|w	|i	|e	|r	|s	|z	|o	|p	|i	|s	|*	|a	|*	|*	|a	|z	|d	|n	|t	|*	|*	|*	|.
|r	|o	|c	|ę	|r	|o	|a	|*	|r	|*	|k	|*	|*	|r	|*	|*	|r	|y	|y	|k	|a	|*	|*	|*	|.
|z	|c	|e	|c	|k	|s	|d	|*	|n	|*	|o	|*	|*	|z	|*	|*	|n	|j	|*	|a	|*	|*	|*	|*	|.
|e	|h	|*	|i	|a	|t	|a	|*	|i	|*	|*	|*	|*	|*	|*	|*	|e	|ę	|*	|*	|*	|*	|*	|*	|.
|w	|ę	|*	|o	|*	|e	|*	|*	|c	|*	|*	|*	|*	|*	|*	|*	|*	|t	|*	|*	|*	|*	|*	|*	|.
|n	|d	|*	|k	|*	|k	|[34][S]\rarr	|a	|k	|u	|c	|z	|i	|[][,]{ }	|r	|u	|d	|y	|*	|*	|*	|*	|*	|*	|.
|i	|ó	|*	|r	|*	|*	|*	|*	|i	|*	|*	|*	|*	|*	|*	|*	|*	|*	|*	|*	|*	|*	|*	|*	|.
|k	|s	|*	|o	|*	|*	|*	|*	|*	|*	|*	|*	|*	|*	|*	|*	|*	|*	|*	|*	|*	|*	|*	|*	|.
|*	|t	|[35][S]\rarr	|t	|w	|i	|e	|r	|d	|z	|e	|n	|i	|e	|[][,]{ }	|p	|e	|t	|t	|i	|s	|a	|*	|*	|.
|[36][S]\rarr	|w	|y	|n	|a	|g	|r	|o	|d	|z	|e	|n	|i	|e	|[][,]{ }	|o	|s	|o	|b	|o	|w	|e	|*	|*	|.
|*	|o	|[37][S]\rarr	|o	|b	|r	|a	|z	|o	|b	|u	|r	|c	|a	|*	|*	|*	|*	|*	|*	|*	|*	|*	|*	|.
|*	|*	|*	|ś	|*	|*	|*	|*	|*	|*	|*	|*	|*	|*	|*	|*	|*	|*	|*	|*	|*	|*	|*	|*	|.
|*	|*	|*	|ć	|*	|*	|*	|*	|*	|*	|*	|*	|*	|*	|*	|*	|*	|*	|*	|*	|*	|*	|*	|*	|.
|*	|*	|*	|*	|*	|*	|*	|*	|*	|*	|*	|*	|*	|*	|*	|*	|*	|*	|*	|*	|*	|*	|*	|*	|.\end{Puzzle}

\newpage

\begin{PuzzleClues}{\textbf{Poziome}\\}\Clue{2}{}{Vulcanodon - rodzaj roślinożernego zauropoda z rodziny wulkanodonów, żył we wczesnej jurze w Zimbabwe}
\Clue{3}{}{ojciec dziecka pod względem biologicznym (niezależnie od relacji z jego matką)}
\Clue{4}{}{hipotetyczny wielkoskalowy ruch materii we Wszechświecie, zaobserwowany we wrześniu 2008 roku przez grupę pod kierunkiem Aleksandra Kashlinsky'ego w danych zebranych przez satelitę WMAP}
\Clue{6}{}{angielska rasa świń typu mięsnego}
\Clue{7}{}{dzielnica Częstochowy położona na południowy wschód od centrum miasta, na lewym brzegu Warty}
\Clue{9}{}{urządzenie sanitarne pozwalające na wygodne oddawanie moczu przez mężczyzn w pozycji stojącej}
\Clue{10}{}{rodzaj barwy używanej w heraldyce}
\Clue{12}{}{bałagan, brak porządku i nagromadzenie wielu rzeczy}
\Clue{14}{}{część rowerowa będąca elementem łączącym ramę i widelec z kierownicą}
\Clue{16}{}{rodzaj bordowego w kolorze jabłka, owoc z drzewa odmiany o tej samej nazwie}
\Clue{19}{}{narzędzie introligatorskie służące do wypalania ozdób}
\Clue{22}{}{ochronny kombinezon roboczy albo sportowy}
\Clue{25}{}{lewa strona tkaniny}
\Clue{30}{}{w geometrii: ściana powierzchni wielościennej albo wielościanu - jeden z wielokątów, które tworzą jej/jego brzeg}
\Clue{32}{}{formacja geologiczna pochodząca z okresu geologicznego o tej samej nazwie}
\Clue{33}{}{poeta, człowiek piszący wiersze}
\Clue{34}{}{akuczi, Myoprocta acouchy - gatunek gryzonia należący do rodziny agutiowatych; występuje w Ameryce Południowej - we wschodnich Andach w południowej Kolumbii, we wschodnim Ekwadorze, północnym Peru i w basenie Amazonki w Brazylii}
\Clue{35}{}{twierdzenie mówiące, że jeżeli A jest podzbiorem grupy topologicznej G, który jest drugiej kategorii i ma własność Baire'a, to zbiór A\textasciicircum-1\}A zawiera otwarte otoczenie elementu neutralnego grupy}
\Clue{36}{}{wynagrodzenie wynikające z umowy do pracy oraz dodatków do pensji dla pracowników zatrudnionych na mocy umowy o pracę}
\Clue{37}{}{IKONOKLASTA, IKONOBURCA}\end{PuzzleClues}

\begin{PuzzleClues}{\textbf{Pionowe}\\}\Clue{1}{}{wizja na ekranie np. telewizora; widok uzyskany metodą projekcji}
\Clue{2}{}{zysk równy wynikowi finansowemu obliczonemu z rachunku zysków i strat przedsiębiorstwa; różnica między przychodami a kosztami jednostki gospodarczej}
\Clue{5}{}{otrzymanie pozytywnego  wyniku z egzaminu}
\Clue{7}{}{przewód, w którym energia elektryczna przetwarzana jest w energię cieplną}
\Clue{8}{}{forma drukująca niezadrukowane (zawierające sam justunek) kolumny na stronach}
\Clue{10}{}{pierwiastek chemiczny z grupy 17 - fluorowców w układzie okresowym}
\Clue{11}{}{przędziwo wykonane z różnorakich odpadów włókienniczych}
\Clue{12}{}{najpowszechniejsza we współczesnym świecie forma państwa, charakteryzująca się wewnętrzną jednolitością; jednostki administracyjne i terytorialne (pozbawione autonomii) podporządkowane są organom centralnym}
\Clue{13}{}{inaczej: jedna ósma, jedna z ośmiu części czegoś podzielnego}
\Clue{15}{}{ciągniona sieć rybacka używana głównie na zamarzniętych jeziorach zakończona tzw. kutlem}
\Clue{16}{}{STRZĘPIĄ}
\Clue{17}{}{gambit hetmański, który powstaje po posunięciach d4 d5, c4 d:c4}
\Clue{18}{}{członek amerykańskiej Partii Federalistycznej}
\Clue{20}{}{GIACOBINIDY; rój meteorów związany z orbitą komety Giacobiniego; Zinnera}
\Clue{21}{}{mieszkanka Lubina}
\Clue{23}{}{kalendarz, który jest dostosowany do robienia w nim notatek pod odpowiednią datą}
\Clue{24}{}{Mastigota, Flagellata, Mastigophora - polifiletyczna grupa pierwotniaków, do której zaliczają się zazwyczaj jednojądrowe haplonty; grupa ta wyróżniana jest na podstawie posiadania wici i jest grupą nienaturalną, dlatego zaliczające się do niej organizmy mogą być przedmiotem badań zarówno zoologii, jak i botaniki}
\Clue{25}{}{właściwość czegoś, co występuje lub jest pomnożone 10 razy}
\Clue{26}{}{pieśń baskijska}
\Clue{27}{}{rodzaj małej, zazwyczaj dość twardej kluski}
\Clue{28}{}{OSTROKÓŁ, CZĘSTOKÓŁ; ściana ze strzelnicami z grubych zaostrzonych pali}
\Clue{29}{}{architekt (1886-1950), gmach YMCA, projekt rozbudowy ratusza w Warszawie}
\Clue{30}{}{Pseudolarix - rodzaj drzew należących do rodziny sosnowatych}
\Clue{31}{}{w językoznawstwie każdy morfem, o ile jest umiejscowiony wewnątrz rdzenia wyrazu (podstawy słowotwórczej)}
\Clue{33}{}{schludność, uporządkowanie, porządek, stan czystości, braku brudu i nieładu}\end{PuzzleClues}\newpage%\section*{Krzyżówka 75}

\noindent\begin{Puzzle}{24}{27}|*	|*	|*	|*	|[1][S]\drarr	|p	|u	|c	|a	|*	|[2][S]\drarr	|k	|a	|d	|z	|i	|e	|l	|n	|i	|c	|z	|k	|a	|*	|.
|*	|*	|[3][S]\darr	|*	|o	|*	|[4][S]\rarr	|i	|b	|i	|s	|[][,]{ }	|c	|z	|c	|z	|o	|n	|y	|*	|*	|[5][S]\darr	|*	|*	|*	|.
|*	|[6][S]\darr	|e	|[7][S]\darr	|d	|[8][S]\drarr	|u	|r	|a	|z	|o	|w	|o	|ś	|ć	|*	|[9][S]\rarr	|b	|ą	|k	|o	|w	|o	|*	|[10][S]\darr	|.
|*	|g	|n	|o	|c	|o	|*	|*	|*	|[11][S]\rarr	|l	|o	|n	|g	|l	|e	|y	|*	|[12][S]\darr	|*	|[13][S]\darr	|y	|*	|*	|z	|.
|*	|a	|t	|b	|i	|m	|*	|*	|[14][S]\rarr	|p	|u	|c	|c	|i	|n	|i	|*	|*	|n	|*	|l	|ż	|*	|*	|i	|.
|[15][S]\drarr	|s	|o	|s	|n	|a	|[][,]{ }	|d	|r	|o	|b	|n	|o	|k	|w	|i	|a	|t	|o	|w	|a	|*	|*	|[16][S]\darr	|e	|.
|z	|t	|m	|e	|e	|r	|[17][S]\rarr	|k	|o	|b	|i	|e	|r	|z	|y	|n	|*	|[18][S]\darr	|s	|*	|k	|*	|*	|o	|m	|.
|a	|r	|o	|r	|k	|l	|[19][S]\darr	|*	|*	|*	|l	|[20][S]\darr	|*	|[21][S]\darr	|*	|[22][S]\drarr	|u	|m	|o	|w	|a	|*	|[23][S]\darr	|r	|i	|.
|s	|o	|l	|w	|*	|i	|k	|*	|[24][S]\darr	|[25][S]\darr	|i	|r	|[26][S]\darr	|ś	|[27][S]\darr	|i	|[28][S]\darr	|o	|r	|[29][S]\darr	|t	|*	|e	|y	|o	|.
|i	|e	|o	|a	|[30][S]\darr	|c	|u	|*	|k	|s	|z	|a	|g	|w	|k	|n	|d	|s	|o	|t	|o	|*	|p	|k	|m	|.
|ł	|n	|g	|t	|d	|a	|r	|*	|o	|z	|a	|t	|o	|i	|a	|d	|r	|t	|ż	|a	|s	|*	|s	|s	|o	|.
|k	|t	|i	|o	|y	|*	|w	|[31][S]\rarr	|r	|e	|c	|y	|t	|a	|t	|y	|w	|*	|e	|m	|*	|[32][S]\darr	|i	|[][,]{ }	|r	|.
|o	|e	|a	|r	|m	|*	|i	|[33][S]\darr	|d	|l	|j	|s	|f	|t	|a	|f	|a	|[34][S]\darr	|c	|a	|*	|p	|l	|s	|z	|.
|b	|r	|*	|*	|k	|[35][S]\darr	|s	|r	|o	|m	|a	|z	|r	|*	|j	|e	|l	|l	|[][,]{ }	|r	|[36][S]\darr	|r	|o	|z	|e	|.
|i	|o	|[37][S]\rarr	|m	|a	|s	|z	|y	|n	|a	|[][,]{ }	|c	|y	|b	|e	|r	|n	|e	|t	|y	|c	|z	|n	|a	|*	|.
|o	|l	|*	|[38][S]\darr	|*	|z	|c	|b	|e	|*	|m	|z	|d	|[39][S]\darr	|w	|e	|i	|k	|ę	|n	|e	|y	|[][,]{ }	|b	|*	|.
|r	|o	|*	|k	|*	|c	|z	|o	|k	|*	|i	|*	|*	|b	|*	|n	|k	|t	|p	|a	|n	|z	|m	|l	|*	|.
|c	|g	|[40][S]\drarr	|w	|ę	|z	|e	|ł	|*	|[41][S]\rarr	|c	|f	|*	|u	|[42][S]\darr	|t	|*	|o	|o	|[][,]{ }	|t	|w	|a	|o	|*	|.
|a	|i	|m	|o	|*	|i	|*	|ó	|*	|*	|e	|*	|*	|k	|b	|*	|*	|r	|n	|z	|r	|o	|s	|r	|*	|.
|*	|a	|o	|k	|*	|*	|*	|w	|*	|[43][S]\darr	|l	|*	|[44][S]\darr	|o	|o	|*	|*	|*	|o	|ł	|a	|i	|z	|o	|*	|.
|*	|*	|t	|a	|[45][S]\rarr	|l	|u	|k	|*	|w	|a	|*	|p	|w	|h	|*	|*	|*	|s	|o	|l	|t	|y	|g	|*	|.
|*	|*	|o	|c	|*	|[46][S]\drarr	|l	|i	|d	|a	|r	|*	|a	|i	|r	|*	|*	|*	|y	|t	|a	|o	|n	|i	|*	|.
|*	|*	|k	|z	|*	|w	|*	|*	|[47][S]\drarr	|k	|n	|a	|r	|a	|*	|*	|*	|*	|*	|o	|*	|ś	|o	|*	|*	|.
|*	|*	|r	|*	|*	|u	|*	|*	|p	|a	|a	|*	|ó	|n	|[48][S]\rarr	|t	|i	|m	|e	|r	|*	|ć	|w	|*	|*	|.
|[49][S]\rarr	|b	|o	|b	|*	|r	|*	|*	|ę	|c	|*	|*	|w	|k	|*	|[50][S]\rarr	|i	|m	|i	|ę	|*	|*	|y	|*	|*	|.
|*	|*	|s	|*	|*	|m	|*	|*	|t	|j	|*	|*	|a	|a	|[51][S]\rarr	|s	|m	|u	|ż	|k	|a	|*	|*	|*	|*	|.
|*	|*	|*	|*	|*	|*	|*	|*	|o	|e	|*	|*	|*	|*	|*	|*	|*	|[52][S]\rarr	|m	|a	|l	|o	|r	|y	|*	|.
|[53][S]\rarr	|p	|a	|r	|a	|f	|i	|a	|*	|*	|*	|*	|*	|*	|*	|*	|*	|*	|*	|*	|*	|*	|*	|*	|*	|.\end{Puzzle}

\newpage

\begin{PuzzleClues}{\textbf{Poziome}\\}\Clue{1}{}{człowiek o okrągłej, pucołowatej buzi}
\Clue{2}{}{tyle, ile zmieści się w kadzielniczce - małej kadzielnicy}
\Clue{4}{}{gatunek ptaka z rodziny ibisowatych (Threskiornithidae); zamieszkuje Afrykę na południe od Sahary, Madagaskar i Irak}
\Clue{8}{}{reagowanie urazem, działanie pod wpływem urazu psychicznego}
\Clue{9}{}{wieś kociewska stanowiąca sołectwo, położona w województwie kujawsko-pomorskim, w powiecie świeckim, w gminie Warlubie przy drodze wojewódzkiej nr 391}
\Clue{11}{}{Samuel, ur. w 1834r. heliofizyk amerykański; zbudował bolometr i badał podczerwoną część widma Słońca}
\Clue{14}{}{kompozytor włoski (1858-1924); opery werystyczne; 'Turandot', 'Tosca', 'Madame Butterfly'}
\Clue{15}{}{Pinus parviflora - gatunek drzewa iglastego z rodziny sosnowatych (Pinaceae)}
\Clue{17}{}{o szpitalu psychiatrycznym w Kobierzynie; nazwa potoczna}
\Clue{22}{}{pojęcie prawne, które przeszło do języka ogólnego}
\Clue{31}{}{śpiew zbliżony do deklamacji, także: fragment utworu instrumentalnego o takich cechach}
\Clue{37}{}{maszyna samosterowana do przetwarzania informacji}
\Clue{40}{}{kolanko - zgrubienie pędu (źdźbła) roślin z rodziny wiechlinowatych (traw)}
\Clue{41}{}{skrót/symbol franka komoryjskiego}
\Clue{45}{}{pomieszczenie w samolocie, na statku lub w pojeździe kosmicznym, wykorzystywane najczęściej jako pomieszczenie techniczne (np. bagażownia)}
\Clue{46}{}{urządzenie optyczne podobne do radaru, stosowane przy badaniu chmur}
\Clue{47}{}{knaar, knarr, knorr - statek handlowo-osadniczy Wikingów}
\Clue{48}{}{urządzenie do odliczania czasu, sygnalizujące - zazwyczaj akustycznie, np. dzwonkiem - upływ nastawionej wcześniej liczby minut; wykorzystywane m.in. w gospodarstwie domowym, np. w kuchni, gdzie ułatwia dopilnowanie prawidłowego czasu gotowania potraw}
\Clue{49}{}{kod ISO 4217 waluty boliviano}
\Clue{50}{}{każdy wyraz odmieniający się przez przypadki}
\Clue{51}{}{skórka królika lub zająca w gwarze łowieckiej}
\Clue{52}{}{(1400-71), angielski rycerz, autor legend o królu Arturze}
\Clue{53}{}{jednostka administracyjna zarządzana przez proboszcza}\end{PuzzleClues}

\begin{PuzzleClues}{\textbf{Pionowe}\\}\Clue{1}{}{dziedzina, fragment, wydzielona część działalności}
\Clue{2}{}{jest to proces solubilizacji, polegający na utworzeniu miceli z solubilizatora, będącego substancją powierzchniowo czynną i tworzącego micele, zawierające substancję rozpuszczaną, które tworzą układ koloidalny z cieczą rozpuszczającą}
\Clue{3}{}{dział zoologii zajmujący się badaniem owadów}
\Clue{5}{}{obszar wysokiego ciśnienia atmosferycznego, w którym najwyższe panuje w centrum układu, a prądy powietrza skierowane są na zewnątrz ku obszarom o niższym ciśnieniu}
\Clue{6}{}{oddział w szpitalu specjalizujący się w gastroenterologii - leczeniu schorzeń układu pokarmowego; przełyku, żołądka, jelit, odbytu i gruczołów trawiennych (wątroba, trzustka) oraz dróg żółciowych}
\Clue{7}{}{osoba, która obserwuje, ogląda coś, przygląda się czemuś}
\Clue{8}{}{padlinożerny chrząszcz, w większości pożyteczny}
\Clue{10}{}{fikcyjny świat z cyklu powieści fantasy stworzonego przez Ursulę K. Le Guin}
\Clue{12}{}{nosorożec biały, nosorożec afrykański, nosorożec szerokopyski, Ceratotherium simum - gatunek ssaka nieparzystokopytnego z rodziny nosorożców, największy z żyjących obecnie nosorożców, a także największe - oprócz słoni - współczesne zwierzę lądowe; występuje głównie na terenach Demokratycznej Republiki Konga, Namibii, Republiki Południowej Afryki, Botswany, Zambii i Kenii}
\Clue{13}{}{strzelec amerykański, srebrny medalista olimpijski z Atlanty (Trap)}
\Clue{15}{}{ktoś, kto pobiera zasiłek}
\Clue{16}{}{Oryx dammah - antylopa z rodziny krętorogich, która zamieszkiwała całą Afrykę Północną; obecnie została uznana za gatunek wymarły w stanie dzikim, choć istniały doniesienia o kilku osobnikach żyjących w Republice Nigru i w Czadzie}
\Clue{18}{}{w informatyce: urządzenie do wymiany danych między dwiema lokalnymi sieciami komputerowymi o różnych systemach okablowania}
\Clue{19}{}{kobieta; słowo obraźliwe}
\Clue{20}{}{pika, spisa lub włócznia}
\Clue{21}{}{nieznany, odległy obszar}
\Clue{22}{}{człowiek obojętny, zazwyczaj w sprawach dotyczących religii}
\Clue{23}{}{wartość określająca precyzję obliczeń numerycznych wykonywanych na liczbach zmiennoprzecinkowych}
\Clue{24}{}{skręcona nitka jedwabna lub bawełniana do robót ręcznych}
\Clue{25}{}{ktoś, kto jest zawiadiacki, nieco bezczelny, przy tym żartobliwy, skory do drwin i żartów}
\Clue{26}{}{pięściarz, trzeci zawodnik świata w wadze piórkowej z 1978 r}
\Clue{27}{}{(1897-1986), pisarz rosyjski, komedie, opowiadania, powieści; „Samotny biały żagiel”, „Trawa zapomnienia”, „Czasie naprzód”}
\Clue{28}{}{drobny chrząszcz z rodziny korników, szkodnik drzewa}
\Clue{29}{}{Saguinus midas - gatunek małpy szerokonosej, której nazwa pochodzi od rudawej sierści pokrywającej jej łapy; zamieszkuje zalesione tereny wzdłuż Amazonki w Brazylii, Gujanie, Gujanie Francuskiej, Suriname i Wenezueli}
\Clue{30}{}{mała (młoda) cebulka cebuli zwyczajnej; nowalijka, cebula łagodna w smaku, najczęściej jedzona na surowo}
\Clue{32}{}{cecha człowieka: to, że jest uczciwy, nie oszukuje}
\Clue{33}{}{Anoinae - podrodzina ptaków z rodziny mewowatych}
\Clue{34}{}{chrześcijanin powołany do funkcji czytania Słowa Bożego w zgromadzeniu liturgicznym, mężczyzna, który przyjął posługę lektoratu}
\Clue{35}{}{znany w kuchni rosyjskiej rodzaj zupy z warzyw liściastych, z dodatkiem mięsa, grzybów lub ryb}
\Clue{36}{}{siedziba centrali, główny ośrodek czegoś}
\Clue{38}{}{wodny ptak z siewek o cienkim, wygiętym dziobie}
\Clue{39}{}{kobieta pochodząca z Buka, mieszkanka tej miejscowości}
\Clue{40}{}{motocross}
\Clue{42}{}{matematyk duński (1887-1951 ); rozwinął teorię funkcji ciągłych prawie okresowych}
\Clue{43}{}{urlop, czas wolny od pracy, zwłaszcza latem}
\Clue{44}{}{gorące, parne i duszne warunki atmosferyczne}
\Clue{46}{}{STARNBERG}
\Clue{47}{}{kiełbasa, uformowana na kształt niedomkniętej pętli, często przechowywana wisząca na pręcie lub kołku}\end{PuzzleClues}\newpage%\section*{Krzyżówka 76}

\noindent\begin{Puzzle}{23}{28}|*	|*	|[1][S]\drarr	|u	|k	|ł	|a	|d	|[][,]{ }	|n	|a	|c	|z	|y	|n	|i	|o	|w	|y	|*	|*	|*	|[2][S]\darr	|*	|.
|*	|[3][S]\drarr	|p	|a	|m	|i	|r	|*	|[4][S]\drarr	|c	|z	|a	|p	|l	|a	|[][,]{ }	|m	|a	|s	|k	|o	|w	|a	|*	|.
|*	|g	|ł	|[5][S]\darr	|[6][S]\rarr	|t	|r	|ó	|j	|s	|k	|r	|z	|y	|n	|[][,]{ }	|p	|s	|t	|r	|y	|*	|s	|*	|.
|[7][S]\drarr	|w	|y	|b	|r	|z	|e	|ż	|e	|[][,]{ }	|s	|z	|k	|i	|e	|r	|o	|w	|e	|*	|*	|*	|*	|*	|.
|g	|i	|t	|u	|[8][S]\darr	|*	|*	|[9][S]\rarr	|h	|a	|j	|d	|u	|k	|*	|*	|*	|*	|*	|*	|[10][S]\darr	|*	|[11][S]\darr	|*	|.
|[][S]-	|a	|a	|k	|s	|*	|*	|*	|o	|[12][S]\rarr	|z	|a	|g	|r	|o	|d	|o	|w	|i	|e	|c	|*	|m	|*	|.
|t	|z	|[][,]{ }	|s	|z	|*	|*	|*	|l	|*	|*	|[13][S]\drarr	|ś	|w	|i	|s	|t	|e	|k	|*	|z	|*	|a	|*	|.
|e	|d	|k	|*	|t	|[14][S]\rarr	|s	|h	|o	|t	|g	|u	|n	|*	|*	|[15][S]\rarr	|u	|n	|i	|s	|e	|k	|s	|*	|.
|r	|a	|o	|*	|a	|*	|*	|[16][S]\darr	|p	|[17][S]\rarr	|o	|p	|o	|n	|k	|a	|*	|[18][S]\darr	|*	|*	|r	|[19][S]\darr	|z	|*	|.
|p	|r	|m	|[20][S]\drarr	|b	|a	|s	|z	|t	|a	|[][,]{ }	|ł	|u	|p	|i	|n	|o	|w	|a	|*	|w	|w	|o	|*	|.
|i	|z	|p	|ż	|[][,]{ }	|[21][S]\darr	|*	|a	|e	|[22][S]\rarr	|w	|y	|b	|i	|e	|g	|*	|e	|[23][S]\darr	|[24][S]\darr	|o	|a	|p	|*	|.
|n	|*	|a	|y	|g	|r	|[25][S]\drarr	|k	|r	|ą	|ż	|n	|i	|k	|*	|*	|*	|n	|m	|t	|n	|d	|e	|*	|.
|e	|*	|k	|c	|e	|i	|g	|o	|*	|*	|[26][S]\rarr	|n	|a	|n	|c	|y	|*	|u	|i	|r	|a	|a	|r	|*	|.
|n	|*	|t	|i	|n	|g	|ł	|p	|*	|[27][S]\rarr	|p	|i	|e	|r	|w	|i	|o	|s	|n	|e	|k	|*	|i	|*	|.
|*	|*	|o	|e	|e	|a	|a	|i	|[28][S]\drarr	|c	|z	|e	|l	|a	|d	|k	|a	|*	|i	|ś	|[][,]{ }	|[29][S]\darr	|a	|*	|.
|*	|*	|w	|*	|r	|u	|d	|a	|n	|*	|*	|n	|*	|*	|*	|*	|*	|*	|v	|ć	|c	|t	|*	|*	|.
|*	|[30][S]\drarr	|a	|m	|a	|d	|y	|n	|a	|[][,]{ }	|d	|i	|a	|m	|a	|n	|t	|k	|a	|*	|h	|u	|*	|*	|.
|*	|f	|*	|[31][S]\darr	|l	|o	|s	|k	|w	|*	|[32][S]\darr	|e	|*	|[33][S]\drarr	|b	|r	|a	|u	|n	|*	|i	|t	|*	|*	|.
|*	|a	|*	|m	|n	|n	|z	|a	|r	|*	|d	|*	|[34][S]\rarr	|k	|a	|n	|n	|a	|*	|*	|l	|k	|*	|*	|.
|*	|u	|*	|o	|y	|*	|*	|*	|ó	|*	|e	|*	|*	|u	|[35][S]\rarr	|p	|r	|z	|y	|s	|i	|a	|d	|*	|.
|*	|*	|*	|z	|*	|[36][S]\rarr	|i	|n	|t	|e	|r	|l	|u	|d	|i	|u	|m	|*	|*	|*	|j	|*	|*	|*	|.
|*	|[37][S]\drarr	|h	|a	|m	|l	|e	|t	|*	|*	|o	|*	|[38][S]\darr	|u	|[39][S]\drarr	|k	|a	|l	|i	|p	|s	|o	|*	|*	|.
|*	|a	|*	|r	|[40][S]\rarr	|d	|i	|a	|l	|o	|g	|*	|l	|*	|s	|[41][S]\drarr	|n	|i	|u	|ń	|k	|a	|*	|*	|.
|*	|s	|[42][S]\drarr	|t	|a	|m	|b	|u	|r	|*	|a	|*	|i	|*	|t	|m	|[43][S]\rarr	|b	|o	|e	|i	|n	|g	|*	|.
|*	|e	|w	|*	|[44][S]\drarr	|s	|ł	|o	|n	|i	|c	|k	|a	|*	|o	|e	|*	|*	|*	|*	|*	|*	|*	|*	|.
|*	|b	|ł	|[45][S]\rarr	|p	|r	|o	|f	|e	|s	|j	|o	|n	|a	|l	|n	|o	|ś	|ć	|*	|*	|*	|*	|*	|.
|*	|*	|a	|[46][S]\rarr	|a	|l	|g	|i	|n	|i	|a	|n	|*	|*	|*	|s	|*	|*	|*	|*	|*	|*	|*	|*	|.
|*	|[47][S]\rarr	|m	|i	|s	|t	|r	|z	|*	|*	|*	|*	|*	|[48][S]\rarr	|w	|a	|f	|e	|l	|*	|*	|*	|*	|*	|.
|*	|*	|*	|*	|*	|*	|*	|*	|*	|*	|*	|*	|*	|*	|*	|*	|*	|*	|*	|*	|*	|*	|*	|*	|.\end{Puzzle}

\newpage

\begin{PuzzleClues}{\textbf{Poziome}\\}\Clue{1}{}{u zwierząt i człowieka - układ obsługujący funkcję transportu płynów}
\Clue{3}{}{górska kraina w Azji Środkowej, głównie w Tadżykistanie}
\Clue{4}{}{Ardea humbloti - gatunek ptaka z rodziny czaplowatych (Ardeidae); występuje na północnym i zachodnim wybrzeżu Madagaskaru oraz w rejonie Jeziora Alaotra, zamieszkuje także Komory i Majottę}
\Clue{6}{}{kodieum pstre, kroton, Codiaeum variegatum - gatunek z rodziny wilczomleczowatych; popularna roślina doniczkowa}
\Clue{7}{}{rodzaj wybrzeża morskiego, powstałego w wyniku częściowego zatopienia silnie zmutonizowanego obszaru polodowcowego, czyli obszaru, na którym występują płytkie doliny polodowcowe; charakteryzuje się obecnością setek małych, sterczących wysepek, zwanych szkierami}
\Clue{9}{}{żołnierz formacji zwanej piechotą węgierską (XVI/XVII)}
\Clue{12}{}{ubogi szlachcic, właściciel zagrody}
\Clue{13}{}{niewielki kawałek papieru}
\Clue{14}{}{strzelba}
\Clue{15}{}{idea zakładająca równość płci, uniwersalność}
\Clue{17}{}{glon naskalny}
\Clue{20}{}{konstrukcja obronna wysunięta poza lico muru i otwarta do wnętrza miasta}
\Clue{22}{}{obszar, w którym skoczkowie narciarscy wyhamowują po lądowaniu}
\Clue{25}{}{ciężki walec stanowiący element roboczy kruszarki}
\Clue{26}{}{miasto we Francji, ośrodek administracyjny departamentu Meurthe-et-Moselle, nad Kanałem Marna-Ren}
\Clue{27}{}{dalekowschodni ptak z rodziny pokrzewek}
\Clue{28}{}{gromadka własnych dzieci}
\Clue{30}{}{gatunek małego ptaka z rodziny astryldowatych (Estrildidae)}
\Clue{33}{}{fizyk niemiecki (1850-1918); prace w dziedzinie radiotelegrafii i termoelektryczności, laureat nagrody Nobla}
\Clue{34}{}{eland, Taurotragus oryx - największa współczesna antylopa z rodziny krętorogich; zamieszkuje głównie rezerwaty w Afryce na południe od Sahary}
\Clue{35}{}{figura w tańcu ludowym}
\Clue{36}{}{utwór sceniczny, wykonywany w średniowieczu, między kolejnymi scenami misteriów, zwykle miał charakter komiczny}
\Clue{37}{}{fikcyjna postać z tragedii Williama Szekspira}
\Clue{39}{}{CALYPSO; improwizowana ballada w rytmie amerykańskim, na wyspach Środkowej Ameryki, później: piosenka taneczna}
\Clue{40}{}{utwór literacki składający się z wypowiedzi kilku osób, niebędący jednak dziełem dramatycznym}
\Clue{41}{}{pieszczotliwie o malutkim dziecku, zwykle dziewczynce}
\Clue{42}{}{wschodni instrument perkusyjny}
\Clue{43}{}{typ amerykańskiego samolotu wojskowego, pasażerskiego}
\Clue{44}{}{ur. w 1931 r., śpiewaczka operowa (sopran); solistka Teatru Wielkiego w Warszawie, prof. A.M. w Warszawie}
\Clue{45}{}{cecha kogoś, kto jest profesjonalny}
\Clue{46}{}{sól kwasu alginowego}
\Clue{47}{}{zwycięzca rywalizacji}
\Clue{48}{}{delikatne, łamliwe ciasto w kształcie cienkich, okrągłych lub kwadratowych płatków}\end{PuzzleClues}

\begin{PuzzleClues}{\textbf{Pionowe}\\}\Clue{1}{}{poliwęglanowy krążek z zakodowaną cyfrowo informacją do bezkontaktowego odczytu światłem lasera optycznego}
\Clue{2}{}{w muzyce: dźwięk A obniżony o półton}
\Clue{3}{}{astronom, osoba zajmująca się badaniem gwiazd}
\Clue{4}{}{Jeholopter - mały pterozaur z rodziny Anurognatów (Anurognathidae); mógł żyć w jurze lub we wczesnej kredzie na terenie Chin}
\Clue{5}{}{podparcie końcowych fragmentów osi kół wozu w postaci walca z metalu}
\Clue{7}{}{monoterpen, 4-metylo-1-izopropylo-1,4-cykloheksadien}
\Clue{8}{}{organ dowodzący siłami zbrojnymi w państwie}
\Clue{10}{}{flaming chilijski, Phoenicopterus chilensis - gatunek ptaka z rodziny flamingów (Phoenicopteridae); zamieszkuje Amerykę Południową}
\Clue{11}{}{średniowieczna spółka okrętowa lub rybacka, również kupiecko-szyperska; w Polsce istniały do niedawna}
\Clue{13}{}{wyprzedawanie towaru}
\Clue{16}{}{droga krajowa z Krakowa do Zakopanego, której długość wynosi 102 km, słynąca z dużego poziomu zatłoczenia i dużej liczby wypadków}
\Clue{18}{}{seria radzieckich próbników kosmicznych przeznaczonych do badania planety Wenus}
\Clue{19}{}{negatywna cecha osobowości}
\Clue{20}{}{energia, żywotność, wigor}
\Clue{21}{}{taniec pochodzący z XVII wieku z Prowansji}
\Clue{23}{}{miniwan - samochód, który powstał jako wersja osobowa samochodu dostawczego, przypominająca go wyglądem, choć mniejsza}
\Clue{24}{}{w biologii, medycynie - substancja gromadząca się w narządach}
\Clue{25}{}{człowiek urodziwy, ładny, elegancki, uprzejmy}
\Clue{28}{}{zmiana kierunku ruchu maszyny na przeciwny}
\Clue{29}{}{papierowe opakowanie w kształcie rurki}
\Clue{30}{}{miejscowość w płd.-wsch. Iraku, ważny port naftowy}
\Clue{31}{}{utwór Mozarta odtwarzany przez muzyka lub grupę muzyków}
\Clue{32}{}{uchylenie części normy prawnej i zastąpienie jej nową}
\Clue{33}{}{element zdobniczy w architekturze indyjskiej w formie podkowiastego łuku}
\Clue{37}{}{miasto w Etiopii u wybrzeży Morza Czerwonego}
\Clue{38}{}{LIANG}
\Clue{39}{}{samolot krótkiego startu i lądowania}
\Clue{41}{}{świadczenia kleru parafialnego na utrzymanie biskupa i jego domowników}
\Clue{42}{}{włamanie - fakt, że ktoś usuwa przeszkodę mającą zabezpieczyć coś przed osobami postronnymi}
\Clue{44}{}{część maszyny, która służy do poruszania lub przemieszczania czegoś}\end{PuzzleClues}\newpage%\section*{Krzyżówka 77}

\noindent\begin{Puzzle}{24}{28}|*	|*	|*	|*	|*	|*	|*	|*	|*	|*	|*	|*	|*	|*	|*	|*	|*	|*	|*	|*	|*	|*	|[1][S]\darr	|*	|*	|.
|*	|*	|*	|*	|*	|*	|*	|*	|*	|*	|*	|*	|*	|*	|*	|*	|*	|[2][S]\drarr	|r	|ó	|g	|*	|c	|*	|*	|.
|*	|*	|*	|*	|*	|[3][S]\rarr	|g	|r	|z	|b	|i	|e	|t	|[][,]{ }	|p	|o	|d	|m	|o	|r	|s	|k	|i	|*	|[4][S]\darr	|.
|*	|*	|*	|*	|[5][S]\rarr	|n	|a	|g	|r	|a	|n	|i	|e	|[][,]{ }	|w	|i	|d	|e	|o	|*	|*	|[6][S]\darr	|r	|*	|k	|.
|*	|[7][S]\rarr	|c	|y	|r	|a	|n	|k	|a	|[][,]{ }	|z	|w	|y	|c	|z	|a	|j	|n	|a	|*	|*	|w	|c	|[8][S]\darr	|ó	|.
|*	|*	|[9][S]\rarr	|f	|r	|o	|n	|t	|[][,]{ }	|d	|r	|u	|g	|o	|r	|z	|ę	|d	|n	|y	|*	|e	|i	|h	|ł	|.
|*	|*	|*	|[10][S]\darr	|[11][S]\rarr	|c	|z	|ą	|s	|t	|k	|a	|[][,]{ }	|e	|l	|e	|m	|e	|n	|t	|a	|r	|n	|a	|*	|.
|[12][S]\rarr	|o	|v	|e	|r	|c	|l	|o	|c	|k	|i	|n	|g	|*	|[13][S]\rarr	|s	|a	|l	|u	|t	|*	|t	|u	|r	|*	|.
|*	|[14][S]\rarr	|k	|o	|r	|a	|b	|*	|*	|[15][S]\darr	|*	|*	|[16][S]\drarr	|g	|r	|ą	|d	|*	|*	|*	|*	|e	|s	|e	|[17][S]\darr	|.
|*	|[18][S]\rarr	|i	|m	|i	|t	|a	|c	|j	|a	|*	|*	|ż	|*	|*	|*	|[19][S]\drarr	|b	|a	|c	|i	|k	|*	|m	|b	|.
|*	|[20][S]\rarr	|f	|a	|l	|o	|w	|n	|i	|k	|[][,]{ }	|n	|a	|p	|i	|ę	|c	|i	|a	|*	|*	|s	|*	|*	|i	|.
|*	|*	|[21][S]\darr	|m	|[22][S]\darr	|*	|*	|*	|*	|u	|*	|[23][S]\rarr	|n	|e	|o	|f	|i	|t	|y	|z	|m	|*	|*	|[24][S]\darr	|l	|.
|*	|*	|e	|e	|r	|*	|*	|*	|*	|m	|[25][S]\rarr	|a	|d	|i	|u	|t	|a	|n	|t	|*	|*	|[26][S]\darr	|*	|z	|i	|.
|*	|[27][S]\darr	|l	|n	|e	|[28][S]\rarr	|d	|a	|m	|u	|l	|k	|a	|*	|*	|[29][S]\rarr	|m	|i	|l	|a	|*	|k	|*	|y	|ń	|.
|*	|n	|e	|c	|l	|*	|*	|*	|*	|l	|[30][S]\rarr	|i	|r	|g	|a	|[][,]{ }	|c	|z	|a	|r	|n	|a	|*	|s	|s	|.
|*	|i	|k	|h	|a	|*	|*	|*	|*	|a	|[31][S]\rarr	|s	|m	|*	|*	|*	|i	|*	|*	|*	|[32][S]\darr	|g	|*	|k	|k	|.
|*	|e	|t	|i	|c	|[33][S]\rarr	|v	|i	|o	|t	|t	|i	|*	|*	|[34][S]\rarr	|l	|a	|m	|a	|*	|b	|a	|[35][S]\darr	|[][,]{ }	|a	|.
|[36][S]\drarr	|p	|r	|z	|y	|s	|p	|o	|s	|o	|b	|i	|e	|n	|i	|e	|[][,]{ }	|o	|b	|r	|o	|n	|n	|e	|*	|.
|l	|o	|o	|a	|j	|[37][S]\drarr	|p	|i	|e	|r	|w	|s	|z	|y	|[][,]{ }	|p	|l	|a	|n	|*	|z	|e	|o	|k	|*	|.
|a	|k	|m	|u	|n	|ó	|*	|*	|*	|*	|*	|*	|*	|[38][S]\drarr	|c	|i	|a	|ł	|o	|*	|o	|k	|k	|o	|*	|.
|s	|a	|e	|r	|o	|s	|*	|[39][S]\rarr	|j	|a	|s	|i	|o	|n	|*	|*	|m	|*	|*	|*	|n	|[][,]{ }	|*	|n	|*	|.
|k	|l	|t	|*	|ś	|e	|*	|*	|[40][S]\rarr	|e	|k	|s	|p	|o	|z	|y	|c	|j	|a	|*	|[][,]{ }	|o	|*	|o	|*	|.
|o	|a	|r	|[41][S]\darr	|ć	|m	|[42][S]\rarr	|h	|o	|b	|o	|k	|e	|n	|*	|*	|i	|*	|*	|*	|h	|ś	|*	|m	|*	|.
|w	|n	|i	|w	|*	|k	|*	|*	|*	|*	|[43][S]\rarr	|c	|h	|a	|r	|ł	|a	|k	|*	|*	|i	|w	|*	|i	|*	|.
|i	|e	|a	|r	|[44][S]\rarr	|a	|r	|o	|m	|a	|t	|*	|*	|j	|*	|*	|*	|*	|*	|*	|g	|i	|[45][S]\darr	|c	|*	|.
|e	|k	|*	|ą	|*	|*	|*	|[46][S]\rarr	|p	|i	|e	|l	|g	|r	|z	|y	|m	|k	|a	|*	|g	|a	|l	|z	|*	|.
|c	|*	|[47][S]\rarr	|b	|e	|z	|c	|i	|e	|l	|e	|s	|n	|o	|ś	|ć	|*	|*	|*	|[48][S]\rarr	|s	|t	|e	|n	|*	|.
|*	|*	|*	|*	|*	|*	|[49][S]\rarr	|d	|o	|k	|ł	|a	|d	|n	|o	|ś	|ć	|*	|*	|*	|a	|y	|ń	|y	|*	|.
|[50][S]\rarr	|k	|a	|t	|e	|c	|h	|u	|m	|e	|n	|a	|t	|*	|[51][S]\rarr	|l	|i	|w	|a	|n	|*	|*	|*	|*	|*	|.\end{Puzzle}

\newpage

\begin{PuzzleClues}{\textbf{Poziome}\\}\Clue{2}{}{przedmiot wykonany z rogu zwierzęcego lub przypominający go kształtem, zwłaszcza naczynie, np. róg obfitości, róg z tabaką}
\Clue{3}{}{silnie wydłużona wypukła forma dna oceanicznego o stromych stokach, położona w obrębie dna oceanicznego}
\Clue{5}{}{film, który jest zapisany na kasecie wideo lub nagrany taką techniką}
\Clue{7}{}{cyranka, Anas querquedula - gatunek wędrownego ptaka wodnego średniej wielkości z rodziny kaczkowatych (Anatidae); zamieszkuje niemal całą Eurazję: Europę poza północną częścią Półwyspu Skandynawskiego i Półwyspem Iberyjskim, oraz Azję pasem po Pacyfik i Japonię}
\Clue{9}{}{front atmosferyczny o niewielkiej intensywności}
\Clue{11}{}{w fizyce, cząstka, będąca podstawowym budulcem, czyli najmniejszym i nieposiadającym wewnętrznej struktury}
\Clue{12}{}{zwiększanie szybkości pracy (zwiększeniu taktowania zegara) i wydajności sprzętu komputerowego, np. procesora, karty graficznej czy pamięci RAM za pomocą odpowiedniego oprogramowania lub zmiany pewnych ustawień BIOS-u (w przypadku CPU i RAM-u)}
\Clue{13}{}{pewna liczba salw armatnich oddanych dla uczczenia jakiejś osoby, wydarzenia, rocznicy}
\Clue{14}{}{okręt, statek}
\Clue{16}{}{las liściasty zazwyczaj z przewagą grabu i dębu}
\Clue{18}{}{czynność, zajęcie polegające na upodabnianiu, imitatorstwo}
\Clue{19}{}{antena w najprostszym wariancie, składającym się ze sztywnego lub elastycznego kabla lub prętu; również część takiej anteny}
\Clue{20}{}{falownik zasilany ze źródła napięciowego, na jego wejściu znajduje się kondensator}
\Clue{23}{}{w przenośni: fanatyczne wyznawanie jakiejś ideologii czy doktryny}
\Clue{25}{}{oficer przydzielony do dyspozycji wyższego dowódcy}
\Clue{28}{}{lekceważąco o kobiecie}
\Clue{29}{}{jednostka długości: lądowa 1609,844 m; morska 1852,31m}
\Clue{30}{}{Cotoneaster melanocarpus - gatunek krzewu z rodziny różowatych}
\Clue{31}{}{w chemii: symbol samaru}
\Clue{33}{}{włoski skrzypek, kompozytor (1755-1824); koncerty i sonaty skrzypcowe, kwartety smyczkowe}
\Clue{34}{}{tkanina najczęściej jedwabna, bogato przetykana błyszczącymi, metalowymi nićmi}
\Clue{36}{}{dawny przedmiot realizowany w szkołach ponadgimnazjalnych i obecnie nauczany częściowo na studiach, dawniej także w VII i VIII klasach szkół podstawowych; zakres nauczania obejmował szeroko pojętą obronę cywilną, podstawy pierwszej pomocy, metody ochrony przed różnymi zagrożeniami i przygotowanie do postępowania w wypadku katastrof}
\Clue{37}{}{najwyższe miejsce w hierarchii, któremu nadaje się wartość priorytetową}
\Clue{38}{}{martwe ciało ludzkie lub (rzadziej) zwierzęce}
\Clue{39}{}{drewno pozyskiwane z drzewa jesionu}
\Clue{40}{}{położenie czegoś, to jak coś jest zorientowane wobec stron świata}
\Clue{42}{}{miasto w płn. Belgii nad Skaldą wielka stocznia, rafineria ropy naftowej, huta miedzi}
\Clue{43}{}{człowiek cherlawy, słabowity, chorowity}
\Clue{44}{}{porcja aromatu, olejku aromatycznego używanego jako dodatek do ciast; określona ilość aromatu, np. mała, szklana buteleczka}
\Clue{46}{}{podróż do miejsc świętych, podjęta z pobudek religijnych}
\Clue{47}{}{to, że ktoś lub coś nie ma ciała, nie istnieje w materialnej, fizycznej formie}
\Clue{48}{}{jednostka siły; 1000 niutonów}
\Clue{49}{}{to, że coś (najczęściej urządzenie) działa precyzyjnie}
\Clue{50}{}{przygotowanie dorosłych do przyjęcia chrztu, okres przygotowań do jego przyjęcia}
\Clue{51}{}{monumentalny przedsionek w meczetach z wielkim otworem wyjściowym od strony dziedzińca}\end{PuzzleClues}

\begin{PuzzleClues}{\textbf{Pionowe}\\}\Clue{1}{}{CYRKIEL}
\Clue{2}{}{Grzegorz Mendel - zakonnik, opat zakonu augustianów w Brnie na Morawach, prekursor genetyki}
\Clue{4}{}{pal, drąg, gruby kij zaostrzony na końcu}
\Clue{6}{}{kierunek, w którym gwiazdy wykazują największa wzajemna ruchliwość}
\Clue{8}{}{wiele kobiet związanych z jednym mężczyzną, np. w wielożeństwie}
\Clue{10}{}{Eomamenchisaurus - rodzaj zauropoda żyjący w okresie środkowej jury na terenie Chin}
\Clue{15}{}{rejestr procesora, w którym umieszczane są wyniki operacji jednostki arytmetyczno-logicznej procesora}
\Clue{16}{}{żołnierz policji wojskowej}
\Clue{17}{}{włoski malarz i rzeźbiarz (1882-1916) jedno z głównych przedstawicieli i teoretyk futuryzmu}
\Clue{19}{}{ktoś bardzo niezdarny, słaby}
\Clue{21}{}{pomiar za pomocą metod elektrycznych}
\Clue{22}{}{pozostawanie w pewnej relacji, związku, stosunku do czegoś lub kogoś}
\Clue{24}{}{nadwyżka całkowitego zysku firmy ponad próg rentowności firmy}
\Clue{26}{}{nauczanie, edukacja nieoświeconej społeczności}
\Clue{27}{}{drzewo lub krzew z silnie pachnącymi, lawendowymi kwiatami, które przerwadzają się w drobne, kuliste owoce, dodawane jako ostra przyprawa do mięs; roślina lecznicza}
\Clue{32}{}{bozon, kwant pola Higgsa w Modelu Standardowym}
\Clue{35}{}{wolny koniec poziomych lub ukośnych drzewc omasztowania statku np. bomu, gafla, rei}
\Clue{36}{}{demon żyjący w lasach}
\Clue{37}{}{ósma część całości}
\Clue{38}{}{sztuczny materiał, którego nie trzeba prasować}
\Clue{41}{}{wąska szczelina wycięta w caliźnie celem ułatwienia urabiania kopaliny, np. węgla}
\Clue{45}{}{uczucie, kiedy nic się nie chce}\end{PuzzleClues}\newpage%\section*{Krzyżówka 80}

\noindent\begin{Puzzle}{21}{23}|*	|[1][S]\drarr	|e	|*	|[2][S]\darr	|*	|[3][S]\darr	|*	|[4][S]\drarr	|j	|a	|ź	|w	|i	|e	|c	|*	|[5][S]\drarr	|s	|ę	|p	|*	|.
|[6][S]\drarr	|m	|e	|t	|r	|*	|m	|*	|k	|*	|*	|[7][S]\drarr	|k	|a	|p	|o	|d	|z	|i	|ó	|b	|*	|.
|c	|a	|*	|*	|u	|[8][S]\rarr	|o	|w	|o	|d	|n	|i	|o	|w	|c	|e	|*	|a	|*	|*	|*	|*	|.
|o	|u	|*	|*	|d	|*	|r	|[9][S]\darr	|l	|*	|*	|n	|[10][S]\drarr	|f	|g	|*	|*	|s	|*	|[11][S]\darr	|*	|[12][S]\darr	|.
|x	|r	|*	|[13][S]\darr	|a	|*	|e	|w	|ę	|*	|*	|d	|k	|[14][S]\rarr	|k	|a	|c	|z	|k	|a	|*	|w	|.
|c	|e	|*	|ż	|*	|*	|l	|y	|d	|[15][S]\drarr	|f	|e	|r	|e	|s	|*	|*	|c	|*	|s	|*	|i	|.
|i	|t	|*	|a	|*	|[16][S]\darr	|i	|k	|a	|d	|*	|k	|z	|*	|*	|*	|*	|z	|[17][S]\darr	|z	|[18][S]\darr	|t	|.
|e	|a	|*	|b	|*	|m	|a	|o	|*	|ę	|[19][S]\drarr	|s	|y	|n	|t	|e	|t	|y	|k	|*	|k	|e	|.
|*	|ń	|*	|a	|*	|i	|*	|ń	|*	|t	|d	|[][,]{ }	|w	|*	|*	|*	|*	|t	|i	|*	|r	|ź	|.
|*	|c	|*	|[][,]{ }	|*	|k	|*	|c	|*	|k	|r	|r	|o	|[20][S]\drarr	|a	|g	|e	|n	|c	|j	|a	|*	|.
|*	|z	|*	|p	|[21][S]\darr	|r	|*	|z	|*	|a	|w	|z	|s	|c	|*	|*	|*	|o	|k	|[22][S]\darr	|w	|*	|.
|*	|y	|[23][S]\darr	|i	|r	|o	|*	|e	|[24][S]\darr	|*	|a	|e	|z	|e	|*	|*	|*	|ś	|[][,]{ }	|m	|i	|*	|.
|[25][S]\drarr	|k	|a	|r	|a	|f	|i	|n	|k	|a	|*	|c	|y	|s	|*	|[26][S]\darr	|*	|ć	|b	|a	|e	|[27][S]\darr	|.
|l	|*	|k	|e	|d	|a	|[28][S]\drarr	|i	|n	|f	|u	|z	|j	|a	|*	|d	|[29][S]\darr	|*	|o	|n	|c	|d	|.
|a	|*	|u	|n	|a	|l	|k	|ó	|o	|*	|*	|o	|*	|r	|*	|ę	|p	|*	|k	|i	|t	|a	|.
|i	|*	|m	|e	|[][,]{ }	|a	|l	|w	|t	|*	|*	|w	|*	|s	|[30][S]\darr	|b	|e	|*	|s	|e	|w	|r	|.
|c	|*	|u	|j	|s	|*	|a	|k	|*	|*	|*	|y	|*	|t	|d	|s	|r	|*	|e	|r	|o	|c	|.
|y	|*	|l	|s	|o	|[31][S]\drarr	|p	|a	|n	|e	|w	|*	|*	|w	|a	|k	|e	|*	|r	|z	|*	|h	|.
|z	|*	|a	|k	|ł	|r	|s	|*	|*	|*	|*	|[32][S]\rarr	|m	|o	|r	|i	|c	|z	|*	|y	|[33][S]\darr	|a	|.
|a	|*	|t	|a	|e	|a	|*	|[34][S]\rarr	|m	|a	|n	|e	|ż	|*	|*	|*	|*	|*	|*	|s	|l	|n	|.
|c	|*	|o	|*	|c	|*	|*	|*	|*	|[35][S]\rarr	|m	|i	|n	|i	|m	|a	|l	|i	|s	|t	|a	|*	|.
|j	|*	|r	|[36][S]\rarr	|k	|r	|y	|p	|t	|o	|g	|r	|a	|f	|i	|a	|*	|*	|*	|a	|h	|*	|.
|a	|*	|y	|[37][S]\rarr	|a	|n	|a	|l	|i	|z	|a	|[][,]{ }	|k	|o	|s	|z	|t	|ó	|w	|*	|n	|*	|.
|*	|*	|*	|*	|*	|*	|*	|*	|*	|*	|*	|*	|*	|*	|*	|*	|*	|*	|*	|*	|*	|*	|.\end{Puzzle}

\newpage

\begin{PuzzleClues}{\textbf{Poziome}\\}\Clue{1}{}{liczba niewymierna, będąca podstawą logarytmu naturalnego; można ją definiować na kilka różnych sposobów}
\Clue{4}{}{BORSUK; ssak z rodziny łasicowatych}
\Clue{5}{}{padlinożerny ptak drapieżny z rodziny jastrzębiowatych (Accipitridae), podrodziny sępów (Aegypiinae)}
\Clue{6}{}{nauczyciel tańca, muzyki lub języków obcych}
\Clue{7}{}{Cereopsis novaehollandiae - gatunek dużego ptaka z rodziny kaczkowatych (Anatidae), zamieszkujący wyspy położone u południowych wybrzeży Australii oraz w Cieśninie Bassa}
\Clue{8}{}{kręgowce, których rozwój zarodkowy odbywa się w osłonie błon płodowych: gady, ptaki, ssaki}
\Clue{10}{}{skrót/symbol franka gwinejskiego}
\Clue{14}{}{efekt gitarowy nazywany wah-wah (łałą)}
\Clue{15}{}{miasto w płd. Grecji w pobliżu granicy z Turcją}
\Clue{19}{}{coś, co jest produktem wykonanej przez człowieka syntezy chemicznej}
\Clue{20}{}{instytucja rządowa lub pożytku publicznego działająca jako podmiot gospodarczy}
\Clue{25}{}{z politowaniem o karafce}
\Clue{28}{}{wprowadzenie do organizmu większej ilości płynu w celach leczniczych np. kanałem żylnym lub podskórnie}
\Clue{31}{}{duże, metalowe naczynie używane dziś w warzelniach do warzenia soli}
\Clue{32}{}{(1879-1942), pisarz węgierski; „Krewni”, „Rozbójnik”}
\Clue{34}{}{kierat; urządzenie wykorzystujące siłę pociągową zwierząt (koni lub wołów) do napędu stacjonarnych maszyn rolniczych takich jak sieczkarnia, wialnia czy młocarnia bądź do wydobywania wody}
\Clue{35}{}{filozof zajmujący się poznawaniem empirycznym i zmysłowym}
\Clue{36}{}{sztuka szyfrowania tekstu, wiedza praktyczna}
\Clue{37}{}{analiza, której celem jest dostarczenie informacji o kształtowaniu się kosztów oraz o czynnikach oddziałujących na ich poziom, dynamikę i strukturę}\end{PuzzleClues}

\begin{PuzzleClues}{\textbf{Pionowe}\\}\Clue{1}{}{mieszkaniec Mauritiusa}
\Clue{2}{}{najbardziej wiekowa dzielnica Rudy Śląskiej}
\Clue{3}{}{miasto w Meksyku u podnóży Kordyliery Wulkanicznej, stolica stanu Michoacan}
\Clue{4}{}{KANTYCZKA}
\Clue{5}{}{cecha czegoś, co przynosi zaszczyt}
\Clue{6}{}{malarz niderlandzki (1499-1592) tworzył pod wpływem Rafaela}
\Clue{7}{}{indeks zawierający uporządkowane alfabetycznie terminy, użyte w publikacji lub cyklu publikacji}
\Clue{9}{}{prace wykończeniowe związane z budową jakiegoś budynku, urządzaniem wnętrza, przystosowaniem go do założonej funkcji itd}
\Clue{10}{}{Amblystegium - rodzaj mchów należący do rodziny krzywoszyjowatych}
\Clue{11}{}{pisarz żydowski (1880-1957), trylogia „Przed potopem”}
\Clue{12}{}{motyl z rodziny witezi, czyli paziowatych}
\Clue{13}{}{Pelophylax perezi - gatunek płaza bezogonowego z rodziny żabowatych}
\Clue{15}{}{komora z elastycznego materiału, wypełniona gazem (najczęściej powietrzem) tłoczonym poprzez wentyl; element ogumienia pojazdów}
\Clue{16}{}{niewielkie urządzenie do przyrządzania potraw przy urzyciu promieni mikrofalowych}
\Clue{17}{}{osoba uprawiająca kick boxing}
\Clue{18}{}{dziedzina rzemiosła i przemysłu zajmująca się szyciem odzieży i bielizny z tkanin i dzianin}
\Clue{19}{}{drewno w polanach przeznaczone do palenia ognia}
\Clue{20}{}{kraj, którym rządzi cesarz}
\Clue{21}{}{ciało doradcze wspomagające działalność sołtysa}
\Clue{22}{}{artysta, który tworzy według maniery, naśladuje}
\Clue{23}{}{metaforycznie: energia w człowieku, która powoduje, że on ma siłę, jest wypoczęty, dobrze się czuje}
\Clue{24}{}{miara prędkości statku lub wiatru nad zbiornikiem wodnym}
\Clue{25}{}{nadanie charakteru świeckiego; usunięcie wpływu, władzy duchowieństwa}
\Clue{26}{}{ur. w 1953 r., kompozytor i skrzypek jazzowy; utwory instrumentalne, wokalno-instrumentalne; 'Missa brevis'}
\Clue{27}{}{miasto w płn. Mongolii; w pobliżu wydobycie węgla brunatnego}
\Clue{28}{}{w filmie - scena, kolejne ujęcie}
\Clue{29}{}{pisarz żydowski (1852-1915), klasyk literatury jidysz w Polsce, poezje, nowele; „Kpiarz”}
\Clue{30}{}{w sensie niematerialnym; np.dar życia}
\Clue{31}{}{w chemii: symbol radu}
\Clue{33}{}{miasto w Niemczech nad rzeką Lahn}\end{PuzzleClues}\newpage%\section*{Krzyżówka 81}

\noindent\begin{Puzzle}{22}{30}|*	|*	|[1][S]\darr	|*	|*	|[2][S]\drarr	|b	|e	|k	|a	|*	|*	|*	|[3][S]\drarr	|d	|r	|u	|m	|f	|i	|l	|l	|*	|.
|*	|*	|k	|*	|[4][S]\darr	|f	|*	|*	|[5][S]\darr	|[6][S]\darr	|*	|*	|*	|e	|[7][S]\darr	|[8][S]\drarr	|f	|r	|i	|s	|c	|h	|*	|.
|*	|*	|a	|*	|s	|a	|*	|*	|k	|f	|*	|*	|*	|m	|i	|c	|*	|[9][S]\darr	|*	|*	|*	|*	|*	|.
|*	|[10][S]\darr	|k	|*	|w	|k	|[11][S]\drarr	|g	|l	|u	|t	|e	|n	|i	|n	|a	|*	|d	|*	|*	|[12][S]\darr	|*	|*	|.
|[13][S]\drarr	|k	|a	|p	|i	|t	|a	|n	|a	|t	|*	|[14][S]\drarr	|o	|s	|t	|r	|o	|ż	|n	|o	|ś	|ć	|*	|.
|p	|o	|o	|*	|j	|u	|p	|*	|r	|e	|*	|n	|[15][S]\darr	|j	|e	|l	|*	|a	|*	|*	|l	|*	|[16][S]\darr	|.
|a	|m	|*	|*	|a	|r	|t	|*	|e	|r	|*	|o	|c	|a	|r	|i	|*	|u	|*	|[17][S]\darr	|i	|*	|r	|.
|n	|ó	|*	|[18][S]\darr	|ż	|a	|e	|*	|t	|k	|*	|g	|h	|[][,]{ }	|e	|s	|*	|l	|*	|p	|m	|[19][S]\darr	|a	|.
|c	|r	|[20][S]\darr	|ś	|s	|[][,]{ }	|k	|*	|y	|o	|*	|a	|o	|t	|s	|l	|*	|*	|*	|a	|a	|z	|f	|.
|e	|k	|p	|w	|k	|k	|a	|[21][S]\darr	|n	|*	|*	|l	|r	|e	|*	|e	|*	|[22][S]\darr	|*	|s	|c	|a	|a	|.
|r	|a	|r	|i	|*	|o	|r	|r	|*	|[23][S]\rarr	|m	|o	|o	|r	|e	|*	|*	|t	|[24][S]\darr	|c	|z	|ś	|e	|.
|z	|[][,]{ }	|o	|a	|*	|r	|z	|y	|*	|*	|*	|w	|b	|m	|*	|*	|*	|a	|p	|i	|e	|l	|l	|.
|o	|i	|p	|t	|*	|y	|ó	|n	|*	|*	|[25][S]\darr	|a	|a	|o	|[26][S]\rarr	|h	|u	|r	|o	|n	|k	|a	|*	|.
|w	|n	|a	|ł	|*	|g	|w	|e	|*	|*	|d	|t	|[][,]{ }	|e	|*	|*	|[27][S]\darr	|n	|s	|*	|*	|z	|*	|.
|c	|i	|g	|o	|*	|u	|n	|k	|*	|[28][S]\drarr	|r	|e	|a	|l	|p	|o	|l	|i	|t	|i	|k	|*	|*	|.
|e	|c	|a	|[][,]{ }	|[29][S]\darr	|j	|a	|[][,]{ }	|*	|p	|ę	|*	|n	|e	|[30][S]\drarr	|s	|e	|n	|i	|o	|r	|*	|*	|.
|[][,]{ }	|j	|c	|c	|o	|ą	|*	|k	|[31][S]\darr	|a	|t	|*	|g	|k	|n	|*	|m	|a	|n	|[32][S]\darr	|*	|*	|*	|.
|w	|a	|j	|z	|n	|c	|*	|o	|w	|s	|w	|[33][S]\darr	|i	|t	|i	|[34][S]\darr	|e	|*	|g	|k	|*	|[35][S]\darr	|*	|.
|ł	|l	|a	|e	|e	|a	|*	|n	|o	|i	|a	|d	|e	|r	|ż	|k	|r	|[36][S]\darr	|*	|e	|*	|b	|*	|.
|a	|n	|[][,]{ }	|r	|[][S]-	|*	|[37][S]\darr	|t	|j	|e	|[][,]{ }	|z	|l	|o	|*	|o	|c	|d	|*	|s	|*	|a	|*	|.
|ś	|a	|b	|w	|s	|*	|t	|e	|s	|c	|b	|i	|s	|n	|[38][S]\rarr	|b	|i	|e	|g	|*	|*	|n	|*	|.
|c	|*	|ł	|o	|t	|[39][S]\darr	|a	|s	|k	|z	|r	|a	|k	|o	|[40][S]\darr	|i	|e	|s	|[41][S]\darr	|*	|[42][S]\darr	|k	|*	|.
|i	|*	|ę	|n	|e	|k	|f	|t	|o	|k	|u	|d	|a	|w	|t	|t	|r	|k	|t	|*	|l	|o	|*	|.
|w	|[43][S]\rarr	|d	|e	|p	|u	|t	|o	|w	|a	|n	|y	|*	|a	|u	|a	|*	|a	|i	|*	|a	|w	|*	|.
|e	|*	|u	|*	|*	|m	|a	|w	|y	|*	|a	|*	|*	|*	|b	|*	|*	|*	|t	|*	|b	|i	|*	|.
|*	|*	|*	|*	|*	|*	|*	|a	|*	|[44][S]\rarr	|t	|o	|s	|k	|a	|ń	|s	|k	|i	|*	|i	|e	|*	|.
|[45][S]\drarr	|b	|e	|z	|w	|z	|g	|l	|ę	|d	|n	|o	|ś	|ć	|*	|*	|*	|*	|*	|*	|r	|c	|*	|.
|g	|[46][S]\rarr	|d	|z	|i	|e	|n	|n	|i	|k	|a	|r	|z	|[][,]{ }	|ś	|l	|e	|d	|c	|z	|y	|*	|*	|.
|u	|*	|*	|[47][S]\rarr	|h	|u	|s	|y	|t	|a	|*	|*	|*	|*	|*	|[48][S]\rarr	|h	|u	|m	|a	|n	|*	|*	|.
|z	|[49][S]\rarr	|o	|c	|z	|a	|r	|*	|*	|*	|*	|*	|[50][S]\rarr	|a	|f	|e	|r	|z	|y	|s	|t	|a	|*	|.
|*	|*	|*	|*	|*	|*	|*	|*	|*	|[51][S]\rarr	|c	|a	|r	|a	|v	|a	|g	|g	|i	|o	|*	|*	|*	|.\end{Puzzle}

\newpage

\begin{PuzzleClues}{\textbf{Poziome}\\}\Clue{2}{}{zawartość beki, dużej beczki}
\Clue{3}{}{monitor odsłuchowy dla perkusistów, charakteryzujący się m.in. sprawnym przenoszeniem pasma niskiego}
\Clue{8}{}{pisarz szwajcarski ur. 1911r, tworzący w języku niemieckim; „Homo Faber”, „Stiller”, „Biedermann i podpalacze”}
\Clue{11}{}{białko roślinne, składnik glutenu}
\Clue{13}{}{komisja organizująca zawody sprotów wodnych}
\Clue{14}{}{cecha człowieka rozważnie podejmującego działanie, nieskorego do szybkich czynów}
\Clue{23}{}{biochemik amerykański (1913-82); prace z chemii aminokwasów i struktury molekularnej białek, laureat nagrody Nobla}
\Clue{26}{}{członkini plemienia Indian Ameryki Północnej mieszkającego obecnie głównie w Oklahomie (USA) i Ontario (Kanada)}
\Clue{28}{}{Polityka uzwględniająca rzeczywisty stosunek sił działających podmiotów}
\Clue{30}{}{najstarszy, najbardziej doświadczony, powszechnie szanowany członek jakiejś grupy lub społeczności}
\Clue{38}{}{tryb działania urządzenia, także: pozycja przekładni w mechanizmie}
\Clue{43}{}{członek deputacji, wybieralny przedstawiciel reprezentujący jakąś zbiorowość, wysłannik upoważniony do prowadzenia jakiejś sprawy w czyimś imieniu}
\Clue{44}{}{ważny dialekt dawnych Włoch (sprzed zjednoczenia), który stał się podstawą dla współczesnego ogólnego języka włoskiego}
\Clue{45}{}{działanie, które musi zostać zrealizowane w sposób idealny, zgodny z planem}
\Clue{46}{}{dziennikarz zajmujący się ujawnianiem informacji istotnych dla opinii publicznej, a jednocześnie trudno dostępnych dla badaczy, oraz opisujący i wyjaśniający sprawy, które najczęściej powinny być przedmiotem dochodzenia organów do tego celu powołanych (prokuratura, policja, sądy)}
\Clue{47}{}{zwolennik nauczania Jana Husa, zwłaszcza uczestnik jego ruchu religijnego i społecznego z XV wieku}
\Clue{48}{}{przedmiot humanistyczny, zwłaszcza na studiach technicznych}
\Clue{49}{}{ozdobny krzew lub drzewko z Azji Wsch. i Ameryki Północnej, liście owłosione, kwiaty żółte}
\Clue{50}{}{osoba nastawiona na poszukiwanie sensacji i afer}
\Clue{51}{}{plastyczne przedstawienie małego aniołka}\end{PuzzleClues}

\begin{PuzzleClues}{\textbf{Pionowe}\\}\Clue{1}{}{napój z proszku kakaowego przyrządzany na wodzie lub mleku}
\Clue{2}{}{faktura wystawiona poprawnie, przekazana klientowi zamiast faktury wystawionej z błędem}
\Clue{3}{}{emisja elektronów przez rozgrzane ciała, w wyniku cieplnego pobudzenia elektronów}
\Clue{4}{}{wieś w Rosji, w Tatarstanie}
\Clue{5}{}{zakonnik należący do Zgromadzenia Misjonarzy Klaretynów}
\Clue{6}{}{zdrobniale: futro - okrycie wierzchnie (płaszcz, krótsza kurtka) zrobione z futra zwierząt}
\Clue{7}{}{zakład np. handlowy, przedsiębiorstwo}
\Clue{8}{}{miasto w Anglii, nad rzeką Eden, ośrodek administracyjny hrabstwa Cumbria; przemysł gumowy, maszynowy, metalowy}
\Clue{9}{}{DŻUL}
\Clue{10}{}{występująca u roślin komórka pozostająca stale zdolna do podziału, stale zachowuje charakter merystemtyczny}
\Clue{11}{}{córka aptekarza (sprzedawcy w aptece albo właściciela apteki)}
\Clue{12}{}{zdrobniale: ślimak - ornament w kształcie muszli ślimaka}
\Clue{13}{}{Eumalacostraca - najliczniejsza podgromada skorupiaków z gromady pancerzowców}
\Clue{14}{}{nogale, Megapodiidae - rodzina ptaków z rzędu grzebiących (Galliformes)}
\Clue{15}{}{choroba występująca u dzieci, związana z zaburzeniami gospodarki wapniowo-fosforowej, spowodowana najczęściej niedoborem witaminy D; powoduje zmiany w układzie kostnym i zaburzenia rozwojowe}
\Clue{16}{}{SANTI; włoski malarz i architekt (1483-1520) jeden z największych artystów epoki renesansu, kierował budową bazyliki św. Piotra w Rzymie, obrazy ' Piękna ogrodniczka' 'Madonna Sykstyńska'}
\Clue{17}{}{francuski malarz i grafik (1885-1930); portrety, sceny religijne, kompozycje figuralne}
\Clue{18}{}{element sygnalizacji świetlnej, którego zapalenie się oznacza zakaz przechodzenia lub przejeżdżania przez drogę badź skrzyżowanie}
\Clue{19}{}{roślina zielna ze ślazowatych uprawiana w Azji na grube włókna i jako roślina lecznicza}
\Clue{20}{}{statystyczne zjawisko występujące w operacjach dokonywanych na wartościach obarczonych błędem}
\Clue{21}{}{rynek oligopolistyczny, na którym sprzedawcy zachowują się podobnie jak w warunkach konkurencyjnych}
\Clue{22}{}{owoc rośliny o tej samej nazwie, mały pestkowiec o cierpkogorzkawym smaku}
\Clue{24}{}{to, co ktoś napisał, opublikował; wiadomość lub wiadomości udostępnione najczęściej w formie postów w sieci, na przykład na forach internetowych}
\Clue{25}{}{Torpedo nobiliana - gatunek morskiej ryby chrzęstnoszkieletowej z rodziny drętwowatych (Torpedinidae); drętwa brunatna żyje w Atlantyku (na wschód od Angoli do południowej Afryki) oraz w Morzu Śródziemnym na głębokości od 10 do 350 m; osiąga długość 1,8 m i masę 70 kg}
\Clue{27}{}{architekt francuski (1585-1654), reprezentant klasycyzmu}
\Clue{28}{}{zdrobniale o pasiece}
\Clue{29}{}{amerykański taniec towarzyski, w takcie 2/4, w stylu ragtime, szybszy od fokstrota}
\Clue{30}{}{niskie ciśnienie atmosferyczne}
\Clue{31}{}{osoba, która pełni służbę w siłach zbrojnych danego kraju i jest zobowiązana do obrony jego granic}
\Clue{32}{}{kod ISO 4217 szylinga kenijskiego}
\Clue{33}{}{dawna uroczystość ku czci zmarłych na Białorusi i Litwie; uczta obrzędowa i wywoływanie dusz zmarłych}
\Clue{34}{}{partnerka życiowa, żona}
\Clue{35}{}{specjalista od bankowości}
\Clue{36}{}{dziewczyna z bardzo małym biustem}
\Clue{37}{}{gęsta nieco sztywna tkanina jedwabna o delikatnym połysku, szeleszcząca przy poruszaniu, niemnąca się, używana głównie na suknie wieczorowe}
\Clue{39}{}{ojciec chrzestny czyjegoś dziecka lub ojciec dziecka, które jest czyimś chrześniakiem}
\Clue{40}{}{kształt tuby, przypominający rulon, rurę, walec lub ścięty stożek pusty w środku}
\Clue{41}{}{amerykańska małpa z rodziny płaksowatych}
\Clue{42}{}{skomplikowamy układ, plątanina, np. uliczek}
\Clue{45}{}{zgrubienie pochodzenia naturalnego lub wywołane uderzeniem}\end{PuzzleClues}\newpage%\section*{Krzyżówka 82}

\noindent\begin{Puzzle}{21}{24}|*	|*	|*	|*	|*	|*	|*	|*	|*	|*	|[1][S]\darr	|*	|*	|*	|*	|*	|*	|[2][S]\darr	|*	|[3][S]\darr	|*	|*	|.
|*	|*	|*	|*	|[4][S]\rarr	|t	|ł	|u	|m	|a	|c	|z	|e	|n	|i	|e	|*	|n	|*	|n	|*	|*	|.
|*	|*	|*	|*	|*	|*	|*	|[5][S]\rarr	|s	|t	|y	|l	|i	|z	|a	|c	|j	|a	|*	|a	|*	|*	|.
|*	|*	|*	|*	|*	|*	|[6][S]\rarr	|s	|o	|n	|g	|n	|i	|m	|*	|[7][S]\darr	|*	|p	|*	|j	|*	|*	|.
|*	|*	|[8][S]\drarr	|m	|i	|o	|d	|ó	|w	|k	|a	|[][,]{ }	|c	|z	|a	|r	|n	|a	|*	|e	|*	|*	|.
|*	|*	|w	|[9][S]\darr	|[10][S]\drarr	|k	|o	|m	|p	|e	|n	|s	|a	|c	|j	|a	|*	|s	|[11][S]\darr	|ź	|*	|*	|.
|*	|*	|i	|p	|c	|[12][S]\rarr	|b	|u	|d	|ż	|e	|t	|*	|*	|*	|m	|*	|t	|b	|d	|*	|*	|.
|*	|*	|t	|e	|z	|*	|*	|[13][S]\rarr	|l	|a	|k	|i	|e	|r	|*	|u	|*	|n	|r	|ź	|*	|*	|.
|*	|*	|a	|c	|y	|*	|[14][S]\drarr	|k	|o	|t	|*	|*	|[15][S]\darr	|*	|[16][S]\darr	|z	|*	|i	|a	|c	|*	|*	|.
|*	|[17][S]\drarr	|m	|o	|n	|o	|s	|t	|y	|c	|h	|*	|m	|[18][S]\darr	|l	|*	|*	|c	|h	|a	|*	|*	|.
|*	|p	|i	|r	|s	|[19][S]\drarr	|p	|o	|s	|t	|s	|y	|m	|b	|o	|l	|i	|z	|m	|*	|*	|[20][S]\darr	|.
|*	|r	|n	|i	|z	|l	|r	|*	|*	|*	|*	|*	|*	|o	|g	|[21][S]\darr	|[22][S]\darr	|k	|s	|[23][S]\darr	|*	|p	|.
|*	|z	|k	|n	|[][,]{ }	|u	|e	|[24][S]\rarr	|s	|ą	|d	|e	|c	|z	|a	|n	|k	|a	|*	|b	|*	|i	|.
|*	|e	|a	|o	|w	|ź	|j	|[25][S]\drarr	|o	|m	|a	|m	|*	|o	|t	|i	|l	|*	|*	|u	|*	|r	|.
|*	|d	|*	|*	|o	|n	|*	|b	|[26][S]\darr	|*	|[27][S]\drarr	|p	|a	|n	|o	|r	|a	|m	|a	|*	|*	|o	|.
|[28][S]\rarr	|s	|b	|*	|l	|o	|*	|i	|p	|*	|g	|*	|*	|*	|m	|w	|s	|*	|*	|*	|*	|e	|.
|*	|z	|*	|*	|n	|ś	|[29][S]\rarr	|b	|u	|t	|e	|l	|k	|a	|*	|a	|a	|*	|[30][S]\darr	|[31][S]\darr	|*	|l	|.
|[32][S]\rarr	|k	|u	|*	|y	|ć	|*	|l	|k	|[33][S]\rarr	|t	|a	|g	|u	|a	|n	|*	|*	|u	|p	|*	|e	|.
|*	|o	|*	|*	|*	|*	|*	|i	|*	|*	|z	|[34][S]\rarr	|b	|i	|p	|a	|n	|*	|s	|a	|*	|k	|.
|[35][S]\drarr	|l	|u	|z	|a	|c	|k	|o	|ś	|ć	|*	|[36][S]\rarr	|c	|u	|c	|*	|*	|*	|k	|j	|*	|t	|.
|c	|e	|[37][S]\rarr	|p	|u	|n	|k	|t	|[][,]{ }	|p	|r	|o	|g	|r	|a	|m	|u	|*	|o	|d	|*	|r	|.
|*	|*	|*	|*	|*	|[38][S]\rarr	|d	|e	|r	|m	|a	|t	|o	|g	|l	|i	|f	|i	|k	|a	|*	|y	|.
|[39][S]\rarr	|c	|h	|l	|e	|b	|e	|k	|[][,]{ }	|p	|s	|z	|c	|z	|e	|l	|i	|*	|*	|*	|*	|k	|.
|[40][S]\rarr	|t	|u	|r	|z	|y	|c	|a	|[][,]{ }	|o	|d	|l	|e	|g	|ł	|o	|k	|ł	|o	|s	|a	|*	|.
|[41][S]\rarr	|k	|w	|i	|n	|t	|a	|*	|*	|*	|*	|*	|*	|*	|*	|*	|*	|*	|*	|*	|*	|*	|.\end{Puzzle}

\newpage

\begin{PuzzleClues}{\textbf{Poziome}\\}\Clue{4}{}{tekst, który został przetłumaczony z jednego języka na inny; przekład}
\Clue{5}{}{stylizowanie, przekształcanie formy czegoś, w celu nadania walorów dekoracyjnych oraz po to, by przywołać skojarzenie z czymś innym - z inną tematyką, obszarem zainteresowań, znanym dziełem itp}
\Clue{6}{}{miasto w KRL-D nad rzeką Tedong-gang; hutnictwo żelaza, przemysł chemiczny, wydobycie rud żelaza}
\Clue{8}{}{Myzomela nigrita - gatunek ptaka z rodziny miodojadów (Meliphagidae)}
\Clue{10}{}{zastąpienie czegoś występującego w ilości niewystarczającej lub nadmiernej (substancji, cechy, konkretnego działania, zachowania) czymś innym; skorygowanie czegoś}
\Clue{12}{}{budżet państwa}
\Clue{13}{}{roztwór (lub zawiesina) środków powłokotwórczych}
\Clue{14}{}{łow. samiec zająca}
\Clue{17}{}{jednowierszowy utwór, który przypomina wiersz}
\Clue{19}{}{RELIEF}
\Clue{24}{}{planowana od wielu lat (obecnie obiecana na 2022 rok) droga szybkiego ruchu łącząca autostradę A4 z Nowym Sączem, Brzeskiem i granicą Polski}
\Clue{25}{}{wzrokowy (przywidzenie) lub słuchowy (przysłyszenie) wytwór umysłu}
\Clue{27}{}{malowidło znajdujące się na wewnętrznej stronie płótna w kształcie cylindra, zwykle dużych rozmiarów}
\Clue{28}{}{w chemii: symbol antymonu}
\Clue{29}{}{butelka do karmienia niemowlęcia}
\Clue{32}{}{heterodimer białka Ku70 (masa 69 kDa) i Ku86 (masa 83 kDA, zwane również Ku80), będący ważnym czynnikiem łączącym końce nici DNA w procesie niehomologicznego scalania końców DNA}
\Clue{33}{}{nocny gryzoń nadrzewny rodziny wiewiórek; po bokach ciała fałd skórny umożliwiający lot spadochronowy - płd. wsch. Azja}
\Clue{34}{}{płyta wiórowa wykonana z odpadów fornirów}
\Clue{35}{}{cecha zdarzenia, sytuacji, którym towarzyszy luźna, nieskrępowana, swobodna atmosfera}
\Clue{36}{}{kod ISO 4217 peso kubańskiego wymienialnego}
\Clue{37}{}{element wydarzenia powtarzalny, oczekiwany i przewidywany}
\Clue{38}{}{nauka zajmująca się liniami papilarnymi człowieka, wykorzystywana np. w kryminalistyce}
\Clue{39}{}{pierzga - pokarm pszczół, który powstaje w wyniku fermentacji pyłku roślin owadopylnych (entomofilnych) albo wiatropylnych zbieranego przez pszczoły robotnice różnych gatunków pszczół}
\Clue{40}{}{Carex remota - gatunek rośliny z rodziny ciborowatych}
\Clue{41}{}{odległość między dwoma dźwiękami, z których wyższa jest w stosunku do niższego piątym stopniem w podstawowym szeregu dźwięków}\end{PuzzleClues}

\begin{PuzzleClues}{\textbf{Pionowe}\\}\Clue{1}{}{młody Cygan}
\Clue{2}{}{zawodniczka w piłce nożnej, która gra na pozycji najbliższej do przeciwnej bramki i dlatego w drużynie głównie to ona jest odpowiedzialna za zdobywanie bramek}
\Clue{3}{}{osoba, która najechała na innych ludzi, często nie sama}
\Clue{7}{}{(1878-1947), pisarz szwajcarski tworzący w języku francuskim, poezje, powieści, opowiadania; „Pastwisko na Derborence”}
\Clue{8}{}{organiczny związek chemiczny, niezbędny do prawidłowego funkcjonowania organizmu żywego; może być pochodzenia naturalnego lub otrzymywana syntetycznie}
\Clue{9}{}{stary biały szczep winorośli, spotykany jedynie w środkowych Włoszech - w regionach Abruzja (prowincje Chieti, L'Aquila i Teramo), Lacjum (prowincja Rieti), Marche (prowincje Ancona, Ascoli Piceno i Macerata) oraz Umbria}
\Clue{10}{}{opłaty uiszczane za wynajem lokalu na wolnym rynku, które nie są zależne od obowiązujących zasad, zwyczajów i przepisów}
\Clue{11}{}{utwór Brahmsa odtwarzany przez muzyka lub grupę muzyków}
\Clue{14}{}{płyn znajdujący się, wraz z gazem nośnym, w pojemniku pod dużym ciśnieniem i przeznaczony do rozpylania}
\Clue{15}{}{mila morska - jednostka odległości stosowana w nawigacji morskiej oraz lotnictwie; jednej mili morskiej odpowiadają 1852 metry, czyli uśredniona długość łuku południka ziemskiego odpowiadająca jednej minucie kątowej koła wielkiego}
\Clue{16}{}{sztuczny wyraz nie mający znaczenia, używany w badaniach psychologicznych i logopedycznych}
\Clue{17}{}{przenośnie: specjalne, łatwiejsze warunki stworzone dla kogoś lub czegoś będącego we wstępnej fazie rozwoju lub na początku jakiejś drogi}
\Clue{18}{}{cząstka posiadająca spin całkowity}
\Clue{19}{}{rzadkość konsystencji}
\Clue{20}{}{materiał, który ma zdolność generowania siły elektromotorycznej pod wpływem zmian temperatury}
\Clue{21}{}{w filozofii i religii buddyjskiej; stan który jest wyzwoleniem ostatecznym od bytu indywidualnego, bólu, pragnień i namiętności; osiągnięcie najwyższej szczęśliwości spokoju i prawdy absolutnej}
\Clue{22}{}{kolejny stopień w systemie rozgrywek ligowych}
\Clue{23}{}{buszel - miara objętości (pojemności) materiałów sypkich stosowaną w krajach anglosaskich}
\Clue{25}{}{budynek, będący siedzibą instytucji o tej samej nazwie}
\Clue{26}{}{wykrzyknik naśladujący odgłos pukania, używany zwykle w formie powtórzenia: 'puk, puk'}
\Clue{27}{}{amerykański saksofonista tenorowy (1927-1991); przedstawiciel modern jazzu}
\Clue{30}{}{struktura skalna powstała w wyniku pęknięcia i przesunięcia się skał względem siebie}
\Clue{31}{}{duża kromka chleba}
\Clue{35}{}{symbol Lorenca - jednostki prędkości}\end{PuzzleClues}\newpage%\section*{Krzyżówka 86}

\noindent\begin{Puzzle}{19}{24}|*	|*	|*	|*	|*	|*	|*	|*	|*	|*	|*	|*	|*	|*	|[1][S]\darr	|[2][S]\darr	|*	|[3][S]\darr	|*	|*	|.
|*	|*	|[4][S]\drarr	|d	|r	|u	|ż	|y	|n	|n	|i	|k	|*	|*	|t	|w	|*	|b	|*	|[5][S]\darr	|.
|*	|[6][S]\drarr	|o	|g	|ó	|r	|*	|[7][S]\drarr	|l	|e	|*	|*	|*	|*	|r	|ś	|*	|a	|[8][S]\darr	|n	|.
|*	|v	|b	|*	|*	|[9][S]\rarr	|r	|o	|g	|a	|t	|k	|a	|*	|a	|c	|*	|l	|o	|p	|.
|*	|*	|s	|*	|*	|*	|[10][S]\rarr	|p	|r	|o	|p	|l	|i	|o	|p	|i	|t	|e	|k	|*	|.
|*	|[11][S]\drarr	|e	|k	|s	|p	|l	|o	|z	|j	|a	|*	|*	|*	|e	|e	|[12][S]\darr	|t	|u	|*	|.
|*	|k	|r	|[13][S]\rarr	|b	|u	|r	|s	|z	|t	|y	|n	|i	|a	|r	|k	|a	|*	|l	|*	|.
|[14][S]\drarr	|o	|w	|a	|d	|*	|*	|i	|[15][S]\drarr	|k	|a	|z	|u	|s	|*	|l	|n	|*	|a	|*	|.
|k	|r	|a	|*	|*	|*	|*	|k	|d	|*	|*	|[16][S]\drarr	|h	|a	|l	|i	|t	|*	|r	|*	|.
|a	|u	|c	|[17][S]\darr	|*	|*	|*	|[][,]{ }	|o	|*	|*	|c	|[18][S]\darr	|[19][S]\darr	|*	|c	|y	|*	|y	|*	|.
|s	|n	|j	|p	|[20][S]\darr	|*	|*	|b	|n	|*	|*	|h	|d	|ż	|*	|a	|p	|*	|[][,]{ }	|*	|.
|z	|d	|a	|o	|b	|*	|[21][S]\rarr	|r	|a	|j	|t	|u	|z	|y	|*	|[][,]{ }	|s	|*	|t	|*	|.
|u	|[][,]{ }	|*	|z	|r	|*	|*	|a	|t	|[22][S]\darr	|*	|j	|w	|c	|*	|d	|y	|*	|r	|*	|.
|b	|s	|*	|o	|u	|[23][S]\darr	|*	|z	|i	|h	|[24][S]\darr	|*	|o	|i	|*	|o	|c	|*	|ó	|*	|.
|s	|y	|[25][S]\drarr	|s	|t	|a	|r	|y	|*	|o	|b	|[26][S]\darr	|n	|o	|*	|r	|h	|*	|j	|*	|.
|k	|n	|g	|t	|a	|n	|[27][S]\drarr	|l	|e	|p	|i	|k	|*	|d	|*	|o	|o	|*	|w	|*	|.
|o	|t	|l	|a	|l	|t	|s	|i	|[28][S]\darr	|l	|z	|a	|*	|a	|*	|d	|l	|*	|y	|*	|.
|ś	|e	|o	|ł	|n	|y	|z	|j	|g	|i	|n	|r	|*	|w	|[29][S]\darr	|n	|o	|*	|m	|*	|.
|ć	|t	|b	|o	|o	|f	|a	|s	|o	|t	|e	|a	|*	|c	|b	|a	|g	|*	|i	|*	|.
|*	|y	|u	|ś	|ś	|o	|n	|k	|n	|o	|s	|z	|*	|a	|a	|*	|i	|*	|a	|*	|.
|*	|c	|s	|ć	|ć	|n	|t	|i	|t	|w	|*	|j	|*	|*	|j	|*	|z	|*	|r	|*	|.
|*	|z	|*	|*	|*	|a	|a	|*	|y	|c	|*	|a	|*	|[30][S]\drarr	|d	|e	|m	|*	|o	|*	|.
|*	|n	|*	|*	|*	|r	|*	|*	|n	|e	|*	|*	|*	|d	|a	|*	|*	|*	|w	|*	|.
|*	|y	|*	|[31][S]\rarr	|s	|z	|k	|ł	|a	|*	|*	|*	|*	|z	|*	|*	|*	|*	|e	|*	|.
|*	|*	|[32][S]\rarr	|k	|a	|*	|*	|*	|*	|*	|*	|*	|*	|*	|*	|*	|*	|*	|*	|*	|.\end{Puzzle}

\newpage

\begin{PuzzleClues}{\textbf{Poziome}\\}\Clue{4}{}{wojownik będący członkiem drużyny książęcej}
\Clue{6}{}{o człowieku (często sportowcu, zawodniku), który jest nie najlepszy w tym, co robi, który słabo sobie radzi i nie uchodzi za wybitnego, dobrego, zdolnego; słabeusz}
\Clue{7}{}{skrót/symbol funta egipskiego}
\Clue{9}{}{budynek na granicy miasta, u wylotu głównej arterii, gdzie pobierano opłat; wjazdowe}
\Clue{10}{}{kopalna małpa z oligocenu}
\Clue{11}{}{szybkie rozwarcie narządów mowy, które występuje przy artykulacji spółgłosek zwarto-wybuchowych}
\Clue{13}{}{jubilerka}
\Clue{14}{}{INSEKT, przedstawiciel stawonogów}
\Clue{15}{}{niespodziewane zdarzenie, zazwyczaj nieprzyjemne, którego nie dało się przewidzieć}
\Clue{16}{}{minerał, którego głównym składnikiem jest chlorek sodu}
\Clue{21}{}{grube rajstopy dziecięce (wynonane z bawełny)}
\Clue{25}{}{mężczyzna, który gdzieś pełni funkcję jakiegoś rodzaju nadzorcy, np. szefa, strażnika}
\Clue{27}{}{lepka substancja stosowana w budownictwie jako klej do przyklejania papy, klepek podłogowych, jako materiał izolacyjny}
\Clue{30}{}{w biologii: populacja lokalna, populacja genetyczna, populacja geograficzna - grupa osobników jednego gatunku zasiedlająca jednolity obszar}
\Clue{31}{}{przyrząd optyczny, zbudowany z pary szkieł i oprawy, umożliwiającej umocowanie szkieł przed oczami, najczęściej za pomocą zauszników, służący, z reguły, do poprawiania ostrości widzenia, osłabionej przez chorobę lub uraz oka albo przez wiek człowieka}
\Clue{32}{}{w mitologii egipskiej: duch opiekuńczy człowieka}\end{PuzzleClues}

\begin{PuzzleClues}{\textbf{Pionowe}\\}\Clue{1}{}{myśliwy, zazwyczaj z Ameryki Północnej (USA, Kanada), który za pomocą pułapek poluje na zwierzęta futerkowe (lisy, norki, gronostaje itp.)}
\Clue{2}{}{Manica rubida - gatunek mrówki z podrodziny wścieklic; kolonię zakłada w sposób klasztorny; środowiskiem tej mrówki są dobrze nasłonecznione otwarte tereny górskie z ubogą roślinnością}
\Clue{3}{}{scena baletowa}
\Clue{4}{}{proces psychiczny polegający na przyglądaniu się swojemu zachowaniu, emocjom, psychice}
\Clue{5}{}{w chemii: symbol neptunu}
\Clue{6}{}{w chemii: symbol wanadu}
\Clue{7}{}{Gracilinanus emiliae, Marmosa emiliae - gatunek torbacza z rodziny dydelfowatych, znany z kilku stanowisk w północnej części Ameryki Południowej}
\Clue{8}{}{rodzaj okularów, które dzięki specjalnej konstrukcji szkieł umożliwiają oglądanie obrazu w kinie, w telewizorze, w innym przystosowanym do tego urządzeniu wideo w trzech wymiarach, dają wrażenie głębi}
\Clue{11}{}{syntetycznie wytworzony trójtlenek glinu, wzbogacony domieszkami innych tlenków, stosowany jako materiał ścierny}
\Clue{12}{}{przeciwstawny dla psychologizmu kierunek fiolozoficzny, w myśl którego czynniki psychologiczne nie mają determinującego i najważniejszego charakteru, nie stanowią osnowy zjawisk}
\Clue{14}{}{kaszubska tożsamość, poczucie przynależności do Kaszubów, identyfikowanie się z Kaszubami}
\Clue{15}{}{astronom włoski (1826-73), pionierskie prace ze spektroskopii astronomicznej}
\Clue{16}{}{męski narząd płciowy}
\Clue{17}{}{to, co zostaje z większej całości}
\Clue{18}{}{sygnalizacyjne urządzenie dźwiękowe o kształcie odwróconego kielicha}
\Clue{19}{}{osoba, która zapewnia byt, zwłaszcza materialny}
\Clue{20}{}{to, że coś jest bardzo szczere, często obrazowe}
\Clue{22}{}{Hoplocarida - podgromada skorupiaków z gromady pancerzowców}
\Clue{23}{}{księga zawierająca śpiewy liturgiczne}
\Clue{24}{}{korzyść, przewidywany zysk z jakiejś działalności}
\Clue{25}{}{model kulisty Ziemi lub innego ciała niebieskiego}
\Clue{26}{}{rodzaj grubego sukna angielskiego}
\Clue{27}{}{wieloletnia roślina zielna z rodziny wargowatych}
\Clue{28}{}{KĄCINA, KONTYNA; świątynia słowiańska na Pomorzu}
\Clue{29}{}{baśń, jeden z fantastycznych gatunków epickich, zazwyczaj niewielkich rozmiarów, odwołujący się zwykle do folkloru}
\Clue{30}{}{bezpokładowy jacht mieczowy wiosłowo-żaglowy (dziesięciowiosłowy - stąd nazwa) o ożaglowaniu kecza gaflowego, kadłub drewniany lub z tworzywa sztucznego}\end{PuzzleClues}\newpage%\section*{Krzyżówka 87}

\noindent\begin{Puzzle}{21}{26}|*	|*	|*	|[1][S]\drarr	|j	|a	|j	|a	|*	|[2][S]\drarr	|p	|c	|h	|l	|i	|[][,]{ }	|t	|a	|r	|g	|*	|[3][S]\darr	|.
|[4][S]\drarr	|p	|o	|s	|t	|o	|ł	|*	|[5][S]\rarr	|m	|ł	|o	|t	|n	|i	|k	|*	|*	|*	|*	|*	|i	|.
|m	|[6][S]\drarr	|c	|u	|d	|z	|o	|z	|i	|e	|m	|s	|z	|c	|z	|y	|z	|n	|a	|*	|*	|g	|.
|a	|z	|*	|b	|*	|[7][S]\rarr	|c	|e	|f	|t	|r	|i	|a	|k	|s	|o	|n	|*	|*	|*	|*	|l	|.
|n	|a	|*	|d	|[8][S]\drarr	|s	|ł	|o	|n	|e	|c	|z	|n	|i	|c	|o	|w	|e	|*	|*	|*	|i	|.
|g	|ś	|*	|e	|p	|[9][S]\darr	|*	|*	|*	|o	|*	|[10][S]\drarr	|f	|l	|*	|[11][S]\drarr	|k	|i	|j	|*	|*	|c	|.
|a	|p	|*	|p	|s	|p	|*	|[12][S]\drarr	|k	|r	|e	|p	|a	|*	|*	|k	|*	|*	|*	|[13][S]\darr	|[14][S]\darr	|a	|.
|b	|i	|[15][S]\darr	|r	|z	|i	|[16][S]\rarr	|n	|i	|*	|*	|o	|[17][S]\drarr	|p	|a	|l	|c	|ó	|w	|k	|a	|*	|.
|a	|e	|o	|e	|c	|r	|*	|i	|[18][S]\drarr	|r	|ą	|c	|z	|ę	|t	|a	|*	|*	|*	|o	|w	|*	|.
|[][,]{ }	|w	|d	|s	|z	|a	|*	|m	|p	|*	|[19][S]\darr	|z	|n	|[20][S]\darr	|[21][S]\darr	|k	|*	|*	|*	|z	|a	|*	|.
|c	|*	|k	|j	|o	|m	|[22][S]\drarr	|b	|u	|ł	|a	|t	|*	|s	|z	|s	|[23][S]\darr	|*	|[24][S]\darr	|a	|n	|*	|.
|z	|*	|r	|a	|ł	|i	|p	|o	|b	|[25][S]\drarr	|p	|a	|r	|a	|w	|o	|d	|ó	|r	|*	|t	|*	|.
|a	|*	|y	|*	|a	|d	|a	|s	|*	|l	|u	|[][,]{ }	|*	|g	|a	|n	|e	|*	|y	|*	|u	|*	|.
|r	|*	|t	|*	|[][,]{ }	|a	|r	|t	|*	|e	|l	|e	|*	|a	|ł	|*	|m	|*	|w	|*	|r	|*	|.
|n	|*	|y	|*	|m	|[][,]{ }	|a	|r	|[26][S]\rarr	|g	|i	|l	|a	|n	|*	|*	|*	|[27][S]\darr	|a	|*	|o	|*	|.
|a	|*	|[][,]{ }	|*	|a	|f	|m	|a	|*	|i	|a	|e	|*	|*	|*	|*	|*	|l	|l	|[28][S]\darr	|w	|*	|.
|*	|*	|s	|*	|r	|i	|e	|t	|*	|a	|*	|k	|*	|*	|*	|[29][S]\darr	|*	|o	|k	|p	|a	|*	|.
|*	|[30][S]\darr	|z	|*	|o	|n	|t	|u	|*	|*	|*	|t	|[31][S]\rarr	|m	|ó	|j	|*	|g	|a	|a	|n	|*	|.
|*	|t	|a	|[32][S]\darr	|k	|a	|r	|s	|[33][S]\rarr	|p	|a	|r	|z	|e	|n	|i	|c	|a	|*	|n	|i	|*	|.
|*	|u	|c	|c	|a	|n	|y	|*	|[34][S]\rarr	|s	|c	|o	|t	|t	|*	|v	|*	|n	|*	|t	|e	|*	|.
|[35][S]\drarr	|c	|h	|i	|ń	|s	|z	|c	|z	|y	|z	|n	|a	|*	|*	|e	|*	|i	|*	|e	|[][,]{ }	|*	|.
|d	|u	|*	|o	|s	|o	|a	|*	|[36][S]\rarr	|b	|r	|i	|e	|n	|z	|*	|*	|o	|*	|r	|s	|*	|.
|u	|m	|*	|ł	|k	|w	|c	|*	|*	|*	|*	|c	|[37][S]\rarr	|ż	|a	|r	|o	|w	|*	|k	|i	|*	|.
|p	|a	|*	|*	|a	|a	|j	|*	|[38][S]\rarr	|p	|r	|z	|e	|b	|ó	|j	|*	|a	|*	|a	|ę	|*	|.
|r	|n	|[39][S]\rarr	|d	|*	|*	|a	|[40][S]\rarr	|s	|t	|o	|n	|o	|g	|o	|w	|a	|t	|e	|*	|*	|*	|.
|e	|*	|[41][S]\rarr	|w	|i	|ć	|*	|*	|*	|*	|[42][S]\rarr	|a	|p	|o	|l	|o	|g	|e	|t	|a	|*	|*	|.
|*	|*	|*	|*	|[43][S]\rarr	|k	|r	|a	|t	|k	|a	|*	|[44][S]\rarr	|c	|z	|o	|p	|*	|*	|*	|*	|*	|.\end{Puzzle}

\newpage

\begin{PuzzleClues}{\textbf{Poziome}\\}\Clue{1}{}{cecha człowieka twardego, podziwianego, o mocnej osobowości}
\Clue{2}{}{targ, na którym prowadzi się handel starymi rzeczami}
\Clue{4}{}{obuwie z kory, łyka lub skóry noszone dawniej przez słowiańskich chłopów; najczęściej l.mn}
\Clue{5}{}{robotnik, który zajmuje się obróbką metalu za pomocą młota (młotów)}
\Clue{6}{}{to, co obce, cudzoziemskie}
\Clue{7}{}{organiczny związek chemiczny, antybiotyk będący cefalosporyną III generacji o działaniu przeciwbakteryjnym, oporny na działanie klasycznych ß-laktamaz}
\Clue{8}{}{Eurypygiformes -  rząd ptaków z podgromady Neornithes; obejmuje dwie monotypowe, blisko ze sobą spokrewnione rodziny}
\Clue{10}{}{w chemii: symbol pierwiastka flerow}
\Clue{11}{}{przyrząd sportowy, wydłużony przedmiot, który ma określony kształt i który w jakimś sporcie służy do odbijania, przemieszczania piłki, podpierania się itp}
\Clue{12}{}{tkania jedwabna, wełniana lub bawełniana z silnie skręconej przędzy, matowa, używana jako oznaka żałoby}
\Clue{16}{}{w chemii: symbol niklu}
\Clue{17}{}{tradycyjna kiełbasa wieprzowa w cienkim jelicie, z mięsa surowego, podsuszana}
\Clue{18}{}{pieszczotliwie o małych rękach, dłoniach, przede wszystkim dziecięcych}
\Clue{22}{}{rodzaj szabli orientalnej, wykonanej z bardzo twardej i sprężystej stali; oprócz materiału (stali zwanej bułatową lub damasceńską) broń ta wyróżniała się także krótką, krzywą głownią o szerokim piórze, rozszerzającym się ku końcowi}
\Clue{25}{}{cząsteczka wodoru, w której spiny tworzących ją atomów wodoru są skierowane przeciwnie}
\Clue{26}{}{stan w Iranie, ośrodek administracyjny Reszt, powierzchnia 15 tyś. km2}
\Clue{31}{}{o mężu; określenie, którego użyć może tylko żona}
\Clue{33}{}{regionalny (góralski) półmiękki, niedojrzewający, półtłusty ser parzony produkowany z pasteryzowanego mleka krowiego (oryginalnie z mleka owczego)}
\Clue{34}{}{astronauta amerykański na pokładzie Gemini 8}
\Clue{35}{}{chiński język}
\Clue{36}{}{jezioro w Szwecji! i w Alpach Berneńskich, powierzchnia 30 km2, głębokość do 261 m, przez Brienz przepływa Aare}
\Clue{37}{}{poeta rosyjski ur. 1904r, wiersze rewolucyjne, poematy}
\Clue{38}{}{zaskakujące wydarzenie, coś, co robi wrażenie (pozytywne lub negatywne)}
\Clue{39}{}{nazwa literowa drugiego dźwięku w gamie, także od niej bierze oznaczenie tonacja, której toniką jest d}
\Clue{40}{}{Oniscidae - rodzina skorupiaków z rzędu równonogów}
\Clue{41}{}{giętka, cienka, długa gałązka drzewa}
\Clue{42}{}{głosiciel jakichś poglądów, skłonny do ich aktywnej obrony}
\Clue{43}{}{zdrobniale: krata - element najczęściej jakiejś konstrukcji; krzyżujące się pręty lub inne podłużne elementy}
\Clue{44}{}{występ pięty masztu}\end{PuzzleClues}

\begin{PuzzleClues}{\textbf{Pionowe}\\}\Clue{1}{}{łagodna depresja}
\Clue{2}{}{ciało niebieskie obiegające Słońce, rozżarza się w atmosferze ziemskiej co czyni go podobnym do spadającej gwiazdy}
\Clue{3}{}{zwieńczenie wieży w kształcie smukłego ostrosłupa lub stożka, typowa dla gotyku}
\Clue{4}{}{Lophocebus aterrimus - małpa z rodziny makakowatych, największy przedstawiciel mangab; zamieszkuje lasy Afryki Centralnej}
\Clue{6}{}{sposób mówienia i akcentowania wyrazów, charakterystyczny dla specyficznego miejsca lub regionu}
\Clue{8}{}{Apis mellifera major - podgatunek pszczoły miodnej pochodzący z łańcucha górskiego Rif w północno-zachodnim Maroku}
\Clue{9}{}{struktura finansowa, w której zysk konkretnego uczestnika jest bezpośrednio uzależniony od wpłat późniejszych uczestników, stojących niejako niżej w tej strukturze}
\Clue{10}{}{usługa internetowa, w nomenklaturze prawnej określana zwrotem świadczenie usług drogą elektroniczną, służąca do przesyłania wiadomości tekstowych, jak i multimedialnych, tzw. listów elektronicznych - stąd zwyczajowa nazwa tej usługi}
\Clue{11}{}{sygnalizator akustyczny przy samochodzie lub motocyklu}
\Clue{12}{}{deszczowa chmura warstwowa}
\Clue{13}{}{SIUTA bezroga samica sarny}
\Clue{14}{}{urządzanie głośnych i burzliwych kłótni niekiedy połączonych z rękoczynami i niszczeniem przedmiotów}
\Clue{15}{}{forma odkrytego ataku, którego celem jest król}
\Clue{17}{}{w chemii: symbol cynku}
\Clue{18}{}{w polskich warunkach: miejsce, gdzie można się czegoś napić; zazwyczaj piwiarnia, lokal otwierany zwykle w godzinach popołudniowych}
\Clue{19}{}{PUGLIA- region w płn. Włoszech nad Morzem Adriatyckim, stolica Bari, powierzchnia 19,3 tyś. km2}
\Clue{20}{}{ur. 1935r, pisarka francuska, współczesne powieści psychologiczne; „Witaj smutku”, „Zamek w Szwecji”, „Pewien uśmiech”, nowele, sztuki}
\Clue{21}{}{stos czegoś, duża ilość czegoś zwalonego}
\Clue{22}{}{w robotyce: sposób przekształcania macierzy obrotu o wymiarach 3x3 na wektor o wymiarach 1x3}
\Clue{23}{}{kod ISO 4217 marki niemieckiej}
\Clue{24}{}{kobieta, która z kimś rywalizuje}
\Clue{25}{}{starorzymski oddział liczący od 4000 do 6000 żołnierzy LEGION}
\Clue{27}{}{połatowate, Loganiaceae - rodzina roślin należąca do rzędu goryczkowców; obejmuje 13 rodzajów z około 420 gatunkami roślin rocznych, krzewów i lian, występują one w strefie umiarkowanej i subtropikalnej, zwłaszcza w Australii i na Nowej Kaledonii}
\Clue{28}{}{Chromileptes altivelis - gatunek ryby okoniokształtnej z rodziny strzępielowatych}
\Clue{29}{}{amerykański taniec towarzyski z kanonów tańców latynoamerykańskich, powstały po 1910 roku}
\Clue{30}{}{miasto w Argentynie u podnóży Andów, stolica prowincji, Tucuman, ośrodek handlowy największego w kraju regionu uprawy trzciny cukrowej}
\Clue{32}{}{ktoś głupi, ociężały umysłowo, głupek}
\Clue{35}{}{francuski malarz i grafik (1870-1943) współzałożyciel grupy nabistów; kompozycje religijne, portrety, pejzaże, malowidła ścienne}\end{PuzzleClues}\newpage%\section*{Krzyżówka 91}

\noindent\begin{Puzzle}{20}{31}|*	|*	|*	|*	|*	|*	|*	|*	|*	|[1][S]\drarr	|o	|f	|i	|c	|e	|r	|*	|[2][S]\drarr	|k	|*	|*	|.
|*	|*	|[3][S]\darr	|*	|*	|[4][S]\rarr	|s	|z	|y	|b	|o	|l	|e	|t	|*	|*	|[5][S]\darr	|f	|[6][S]\darr	|*	|*	|.
|*	|[7][S]\drarr	|j	|e	|ż	|[][,]{ }	|u	|s	|z	|a	|t	|y	|*	|*	|*	|*	|p	|l	|l	|*	|*	|.
|*	|v	|e	|*	|[8][S]\rarr	|p	|i	|e	|t	|r	|z	|y	|k	|*	|[9][S]\darr	|*	|r	|i	|o	|*	|[10][S]\darr	|.
|*	|o	|d	|[11][S]\drarr	|z	|b	|i	|ó	|r	|k	|a	|*	|[12][S]\rarr	|l	|o	|g	|o	|s	|*	|*	|o	|.
|*	|l	|n	|f	|[13][S]\drarr	|b	|u	|r	|g	|*	|*	|*	|*	|*	|p	|[14][S]\darr	|t	|a	|[15][S]\darr	|*	|s	|.
|*	|v	|o	|b	|g	|*	|*	|*	|*	|*	|*	|*	|*	|*	|i	|l	|o	|k	|c	|*	|n	|.
|*	|o	|s	|*	|r	|*	|*	|*	|[16][S]\drarr	|n	|a	|g	|ó	|r	|n	|i	|k	|*	|z	|[17][S]\darr	|u	|.
|*	|*	|t	|*	|o	|[18][S]\rarr	|m	|i	|k	|s	|e	|r	|*	|*	|i	|n	|ó	|*	|a	|z	|w	|.
|*	|[19][S]\darr	|r	|[20][S]\drarr	|m	|e	|d	|i	|o	|l	|a	|n	|*	|*	|a	|d	|ł	|*	|j	|a	|i	|.
|*	|s	|o	|l	|a	|*	|[21][S]\rarr	|e	|m	|b	|a	|r	|g	|o	|*	|b	|*	|[22][S]\darr	|n	|m	|k	|.
|*	|k	|n	|o	|d	|[23][S]\rarr	|z	|j	|a	|z	|d	|*	|[24][S]\rarr	|s	|z	|l	|e	|m	|i	|k	|*	|.
|[25][S]\drarr	|a	|n	|t	|a	|ł	|e	|k	|*	|*	|*	|*	|*	|[26][S]\darr	|*	|a	|*	|u	|c	|n	|*	|.
|l	|ł	|o	|[][,]{ }	|[][,]{ }	|*	|*	|*	|*	|*	|*	|*	|[27][S]\rarr	|b	|s	|d	|*	|t	|z	|i	|*	|.
|a	|a	|ś	|r	|z	|*	|[28][S]\rarr	|w	|y	|w	|r	|o	|t	|e	|k	|*	|*	|a	|e	|ę	|*	|.
|m	|[][,]{ }	|ć	|o	|u	|*	|*	|[29][S]\drarr	|p	|i	|e	|p	|r	|z	|y	|k	|*	|c	|k	|c	|*	|.
|p	|o	|*	|z	|c	|*	|[30][S]\darr	|m	|[31][S]\darr	|*	|[32][S]\rarr	|d	|u	|p	|a	|*	|*	|j	|*	|i	|*	|.
|k	|r	|*	|p	|h	|[33][S]\drarr	|k	|o	|n	|c	|e	|n	|t	|r	|a	|c	|j	|a	|*	|e	|*	|.
|a	|g	|*	|o	|o	|s	|o	|ł	|i	|*	|*	|*	|*	|z	|*	|[34][S]\darr	|*	|*	|*	|[][,]{ }	|*	|.
|[][,]{ }	|a	|*	|z	|w	|z	|n	|o	|z	|[35][S]\rarr	|i	|z	|o	|e	|n	|z	|y	|m	|*	|c	|*	|.
|w	|n	|*	|n	|a	|c	|i	|j	|a	|[36][S]\drarr	|k	|l	|i	|s	|z	|a	|*	|*	|*	|e	|*	|.
|i	|o	|*	|a	|*	|z	|c	|e	|m	|l	|*	|*	|*	|t	|[37][S]\rarr	|k	|r	|e	|o	|l	|*	|.
|e	|g	|*	|w	|*	|e	|z	|c	|i	|o	|[38][S]\rarr	|d	|a	|r	|v	|a	|s	|*	|*	|n	|*	|.
|c	|e	|*	|c	|*	|p	|e	|*	|*	|r	|*	|[39][S]\darr	|*	|z	|*	|ź	|*	|[40][S]\darr	|*	|e	|*	|.
|z	|n	|*	|z	|*	|i	|k	|[41][S]\drarr	|n	|i	|e	|w	|i	|e	|r	|n	|o	|ś	|ć	|*	|*	|.
|y	|i	|*	|y	|*	|c	|*	|b	|[42][S]\darr	|a	|[43][S]\darr	|l	|*	|n	|*	|o	|*	|w	|*	|*	|*	|.
|s	|c	|*	|*	|[44][S]\darr	|i	|[45][S]\darr	|r	|s	|*	|ż	|e	|*	|n	|*	|ś	|*	|i	|*	|*	|*	|.
|t	|z	|*	|[46][S]\drarr	|r	|e	|c	|e	|p	|t	|y	|w	|n	|o	|ś	|ć	|*	|t	|*	|*	|*	|.
|a	|n	|*	|f	|o	|l	|u	|s	|l	|*	|ł	|*	|*	|ś	|*	|*	|*	|e	|*	|*	|*	|.
|*	|a	|*	|l	|s	|*	|d	|t	|i	|*	|k	|*	|*	|ć	|*	|*	|*	|ź	|*	|*	|*	|.
|*	|*	|*	|*	|s	|*	|a	|*	|t	|*	|a	|*	|*	|*	|*	|*	|*	|*	|*	|*	|*	|.
|*	|*	|*	|*	|*	|*	|*	|*	|*	|*	|*	|*	|*	|*	|*	|*	|*	|*	|*	|*	|*	|.\end{Puzzle}

\newpage

\begin{PuzzleClues}{\textbf{Poziome}\\}\Clue{1}{}{adiutant morskiego dowódcy: oficer nawigacyjny, oficer wachtowy}
\Clue{2}{}{kelwin - jednostka temperatury w układzie SI równa 1/273,16 temperatury termodynamicznej punktu potrójnego wody, oznaczana K}
\Clue{4}{}{znak rozpoznawczy, często wyraz lub hasło zawierające charakterystyczną dla danego języka głoskę, której cudzoziemiec, obcy lub wróg nie potrafi wymówić}
\Clue{7}{}{Hemiechinus auritus - ssak z rodziny jeżowatych, o dosyć długich uszach, ogonku dochodzącym do 5 cm i długości ciała 12-27 cm; żyje na stepach i pustyniach Egiptu, Mongolii, Azji Mniejszej, południowej Rosji, części Indii i Chin}
\Clue{8}{}{lekkoatleta, srebrny medalista z Montrealu w sztafecie 4x400 m}
\Clue{11}{}{dłuższe zebranie, które służy odbyciu jakichś czynności; najczęściej w harcerstwie}
\Clue{12}{}{w teologii chrześcijańskiej druga osoba boska, Syn Boży}
\Clue{13}{}{miasto w Niemczech, port nad kanałem Łaba; Hawela (Saksonia-Anhalt)}
\Clue{16}{}{chroniony w Polsce ptak z rzędu wróblowatych; góry Eurazji, Afryka}
\Clue{18}{}{mikser kuchenny - urządzenie służące do rozcierania, urabiania, ubijania ciast lub innych produktów spożywczych}
\Clue{20}{}{miasto i gmina w północnych Włoszech, stolica prowincji Mediolan i regionu Lombardia}
\Clue{21}{}{zakaz opuszczania przez obce statki portów lub wód terytorialnych danego państwa}
\Clue{23}{}{jazda z góry na dół}
\Clue{24}{}{wzięcie dwunastu lew przez gracza karcianego}
\Clue{25}{}{BARYŁKA; niewielka beczułka na wino tub piwo}
\Clue{27}{}{kod ISO 4217 dolara bahamskiego}
\Clue{28}{}{but szyty na lewej stronie, a noszony po wywróceniu na prawą}
\Clue{29}{}{inaczej pikanteria}
\Clue{32}{}{srom; żeńskie zewnętrzne narządy płciowe}
\Clue{33}{}{zjawisko polegające na skupieniu, ześrodkowaniu uwagi (w domyśle: świadomości), skierowaniu jej na określoną myśl, przedmiot, zagadnienie, wydarzenie, sytuację czy zjawisko i utrzymywaniu w czasie}
\Clue{35}{}{izozym - homologiczny enzym w obrębie danego organizmu, katalizujący tę samą reakcję co inny izoenzym, ale różniący się od niego budową i właściwościami fizycznymi}
\Clue{36}{}{płyta drukarska z której otrzymuje się reprodukcje jakiegoś obrazu lub tekstu}
\Clue{37}{}{francuskojęzyczny mieszkaniec Luizjany lub każdy mieszkaniec tego stanu, niezależnie od pochodzenia i koloru skóry, nazywany tak (stereotypowo, ironicznie) przez mieszkańców pozostałych części USA}
\Clue{38}{}{węgierski pisarz i działacz polityczny (1912-73), członek Światowej Rady Pokoju; „Czarny chleb”, powieści, dramaty}
\Clue{41}{}{brak zgodności z oryginałem lub tekstem źródłowym}
\Clue{46}{}{nastawienie na odbiór, przyjmowanie informacji}\end{PuzzleClues}

\begin{PuzzleClues}{\textbf{Pionowe}\\}\Clue{1}{}{pas barkowy, obręcz barkowa; zespół kości}
\Clue{2}{}{osoba zawodowo zajmująca się przewozem turystów na rzece Dunajec}
\Clue{3}{}{ograniczenie intelektualne, uparte trzymanie się jakichś poglądów}
\Clue{5}{}{oficjalne, pisane na bieżąco sprawozdanie przebiegu rozmaitych posiedzeń, zebrań, obrad, wyborów itp}
\Clue{6}{}{skrótowiec odliceum ogólnokszałcące}
\Clue{7}{}{samochód marki Volvo}
\Clue{9}{}{dokument ekspercki, w którym zawarta jest analiza problemu}
\Clue{10}{}{Linyphia - rodzaj pająka z rodziny osnuwikowatych}
\Clue{11}{}{Facebook - popularny portal społecznościowy}
\Clue{13}{}{jednostka w strukturze organizacji harcerskiej, skupiająca jej członków zwyczajnych}
\Clue{14}{}{astronom szwedzki (1895-1965), podał teorię powstania ramion spiralnych Galaktyki}
\Clue{15}{}{zawartość czajniczka, naczynia, które służy do gotowania wody}
\Clue{16}{}{KOMAT; wada wielu układów optycznych}
\Clue{17}{}{plomba, pieczęć lub inny znak urzędowy nakładany przez organ celny lub inne upoważnione do tego jednostki organizacyjne na towary, pomieszczenia, składy celne, magazyny celne, środki przewozowe lub ich części}
\Clue{19}{}{biolit - skała osadowa powstała na skutek nagromadzenia szczątków organicznych lub wytrącenia substancji chemicznych powstałych w efekcie przemian fizjologicznych organizmów}
\Clue{20}{}{lot, najczęściej na użytek wojskowy, mający na celu rozpoznanie terenu i rozkład obiektów i skupisk osób na nim}
\Clue{22}{}{zmiana brzmienia głosu u ludzi na skutek zmian fizjologicznych w obrębie narządów głosu, zachodząca zarówno u chłopców, jak i (w mniejszym stopniu) u dziewcząt}
\Clue{25}{}{nigdy nie gaszona lampka stawiana przed tabernakulum}
\Clue{26}{}{cecha czegoś, co nie może być zlokalizowane w obrębie przestrzeni, co nie ma własności przestrzennych; jedna z cech bytu według tez niektórych kierunków filozoficznych}
\Clue{29}{}{młodzieniec, młody chłopak}
\Clue{30}{}{pieszczotliwie o koniu - zwierzęciu}
\Clue{31}{}{poeta perski (1141-ok.1210), prekursor analizy psychologicznej w literaturze perskiej; „Chamsa”}
\Clue{33}{}{ten, kto zaszczepia jakieś poglądy, idee, krzewiciel}
\Clue{34}{}{bycie zakaźnym, zdolność do wywołania zakażenia}
\Clue{36}{}{fizyk (1883-1958); odkrywca rozpraszania światła przez ośrodki gazowe; prace o promieniotwórczości}
\Clue{39}{}{wprowadzenie płynu do zbiornika}
\Clue{40}{}{jezioro na Białorusi, w pobliżu Nowogródka, powierzchnia 1,5 km}
\Clue{41}{}{na jednostce pływającej jest to lina cumownicza działająca prostopadle do osi jednostki}
\Clue{42}{}{miasto w Chorwacji nad Morzem Adriatyckim w Dalmacji, duży ośrodek turystyczny}
\Clue{43}{}{ozdoba, pasek o szer. 2-5 mm, stosowany w rzemiośle artystycznym jako motyw zdobniczy, najczęściej jako element intarsji lub inkrustacji}
\Clue{44}{}{socjolog amerykański (1866-1951 ); przedstawiciel psychologizmu w socjologii}
\Clue{45}{}{coś dziwnego, niezwykłego, co zaskakuje, czasami śmieszy}
\Clue{46}{}{w chemii: symbol pierwiastka flerow}\end{PuzzleClues}\newpage%\section*{Krzyżówka 92}

\noindent\begin{Puzzle}{22}{25}|*	|*	|*	|*	|*	|*	|[1][S]\darr	|*	|*	|*	|*	|*	|*	|*	|*	|*	|*	|*	|[2][S]\darr	|*	|[3][S]\darr	|*	|[4][S]\darr	|.
|*	|*	|[5][S]\darr	|*	|*	|*	|o	|[6][S]\darr	|*	|*	|[7][S]\rarr	|ł	|a	|d	|o	|w	|n	|i	|k	|*	|w	|[8][S]\darr	|u	|.
|*	|*	|ś	|[9][S]\darr	|*	|*	|d	|b	|*	|*	|*	|*	|*	|*	|*	|[10][S]\darr	|*	|[11][S]\darr	|l	|*	|u	|m	|s	|.
|*	|*	|w	|n	|*	|[12][S]\rarr	|n	|o	|r	|w	|i	|c	|h	|*	|*	|c	|[13][S]\darr	|b	|a	|*	|j	|e	|t	|.
|*	|[14][S]\darr	|i	|a	|[15][S]\rarr	|z	|a	|j	|a	|z	|d	|*	|*	|[16][S]\rarr	|p	|o	|k	|u	|t	|a	|*	|d	|a	|.
|*	|ż	|ę	|p	|*	|[17][S]\darr	|w	|o	|*	|[18][S]\drarr	|d	|u	|p	|a	|[][,]{ }	|b	|i	|s	|k	|u	|p	|a	|*	|.
|*	|a	|t	|i	|*	|m	|i	|w	|[19][S]\rarr	|c	|h	|o	|d	|ż	|a	|*	|n	|z	|a	|*	|*	|l	|*	|.
|*	|b	|y	|ę	|*	|i	|a	|n	|*	|z	|*	|*	|*	|[20][S]\drarr	|i	|*	|d	|e	|*	|*	|*	|i	|*	|.
|*	|a	|[][,]{ }	|c	|[21][S]\darr	|a	|n	|i	|*	|ą	|[22][S]\rarr	|f	|o	|r	|t	|*	|ż	|l	|*	|*	|[23][S]\darr	|o	|*	|.
|*	|[][,]{ }	|m	|i	|a	|m	|i	|k	|[24][S]\rarr	|b	|e	|l	|u	|a	|r	|d	|a	|*	|[25][S]\darr	|*	|d	|n	|[26][S]\darr	|.
|[27][S]\rarr	|p	|i	|e	|n	|i	|e	|*	|*	|e	|*	|*	|[28][S]\drarr	|n	|a	|k	|ł	|a	|d	|k	|a	|*	|e	|.
|*	|o	|k	|[][,]{ }	|a	|*	|*	|*	|*	|r	|*	|[29][S]\darr	|k	|d	|*	|[30][S]\darr	|*	|*	|z	|[31][S]\darr	|c	|*	|r	|.
|[32][S]\drarr	|m	|o	|d	|r	|z	|e	|l	|e	|*	|*	|s	|o	|k	|*	|g	|[33][S]\darr	|*	|i	|a	|h	|*	|g	|.
|n	|i	|ł	|o	|c	|*	|*	|*	|[34][S]\rarr	|s	|p	|i	|n	|a	|k	|e	|r	|*	|a	|r	|[][,]{ }	|*	|a	|.
|a	|d	|a	|t	|h	|*	|*	|*	|*	|*	|*	|e	|t	|[][,]{ }	|*	|o	|y	|*	|d	|b	|p	|[35][S]\darr	|t	|.
|t	|o	|j	|y	|*	|*	|*	|*	|*	|*	|*	|n	|y	|r	|*	|c	|s	|*	|u	|i	|u	|i	|y	|.
|u	|r	|*	|k	|*	|[36][S]\rarr	|l	|u	|s	|i	|t	|a	|n	|o	|*	|e	|a	|*	|l	|t	|l	|n	|w	|.
|r	|o	|[37][S]\rarr	|o	|p	|a	|s	|ł	|o	|ś	|ć	|*	|a	|z	|*	|n	|k	|*	|e	|r	|p	|s	|n	|.
|a	|w	|[38][S]\rarr	|w	|y	|p	|a	|d	|*	|[39][S]\rarr	|z	|b	|*	|b	|*	|t	|*	|[40][S]\darr	|k	|a	|i	|y	|o	|.
|l	|a	|[41][S]\rarr	|e	|l	|i	|k	|s	|i	|r	|*	|[42][S]\rarr	|g	|i	|e	|r	|e	|k	|*	|l	|t	|n	|ś	|.
|n	|*	|*	|*	|*	|*	|[43][S]\rarr	|s	|z	|t	|y	|b	|l	|e	|t	|y	|*	|n	|[44][S]\darr	|n	|o	|u	|ć	|.
|o	|*	|*	|*	|*	|*	|[45][S]\rarr	|r	|z	|e	|k	|a	|[][,]{ }	|r	|o	|z	|t	|o	|k	|o	|w	|a	|*	|.
|ś	|*	|[46][S]\rarr	|i	|m	|p	|e	|d	|y	|m	|e	|n	|t	|a	|*	|m	|*	|r	|u	|ś	|y	|t	|*	|.
|ć	|*	|*	|*	|*	|[47][S]\rarr	|l	|e	|s	|z	|c	|z	|y	|n	|y	|*	|*	|r	|r	|ć	|*	|o	|*	|.
|*	|*	|*	|*	|*	|*	|*	|[48][S]\rarr	|n	|a	|w	|a	|l	|a	|n	|k	|a	|*	|s	|*	|*	|r	|*	|.
|*	|*	|*	|*	|[49][S]\rarr	|p	|r	|o	|n	|a	|c	|j	|a	|*	|*	|*	|*	|*	|*	|*	|*	|*	|*	|.\end{Puzzle}

\newpage

\begin{PuzzleClues}{\textbf{Poziome}\\}\Clue{7}{}{łączący kilka naboi element służący przyspieszeniu ładowania broni}
\Clue{12}{}{miasto w USA (Connectiicut); przemysł metalowy, włókienniczy, chemiczny}
\Clue{15}{}{KARCZMA, OBERŻA}
\Clue{16}{}{kara za złe czyny, zazwyczaj nakładana w kontekście religinym, przenośnie używa się tego słowa również szerzej}
\Clue{18}{}{towarzyska gra karciana o luźnych zasadach}
\Clue{19}{}{tytuł grzecznościowy funkcjonujący w kulturze muzułmańskiej}
\Clue{20}{}{w chemii: symbol jodu}
\Clue{22}{}{ziemna budowla obronna, przystosowana do długotrwałej obrony}
\Clue{24}{}{obszerna baszta; w XVI wieku w fortyfikacjach o narysie bastionowym, niewielki, mało rozwinięty bastion}
\Clue{27}{}{śpiewanie, śpiew}
\Clue{28}{}{część złącza bala lub belki powstała przez ich nacięcie i nałożenie na siebie}
\Clue{32}{}{zgrubienie skórne występujące u niektórych zwierząt, np. siedzeniowe u wielu małp}
\Clue{34}{}{trójkątny żagiel stawiany dodatkowo na jachcie przy żegludze z wiatrem motyl}
\Clue{36}{}{koń luzytański - jedna z gorącokrwistych ras konia domowego pochodząca z terenów Ribatejo i Alentejo w Portugalii; powstała na skutek inwazji Maurów na Hiszpanię, gdy kuce Sorraia połączyły się z krwią koni berberyjskich dosiadanych przez zbrojnych z Afryki Północnej}
\Clue{37}{}{o książce lub innej materialnej formie tekstu: duża objętość, grubość}
\Clue{38}{}{podróż, zazwyczaj krótka i w celach turystycznych}
\Clue{39}{}{odpowiednik zettabajta w systemie dwójkowym, równy 2\textasciicircum70 bajtów}
\Clue{41}{}{przen. lekarstwo, środek zaradczy}
\Clue{42}{}{okres w historii Polski (PRL), kiedy pierwszym sekretarzem KC PZPR był Edward Gierek}
\Clue{43}{}{kamasze męskie z nie rozciętymi cholewkami i wszytymi po bokach kawałkami elastycznej gumy}
\Clue{45}{}{rzeka, której koryta rozdzielone są licznymi wyspami lub raczej odsypami korytowymi i mieliznami}
\Clue{46}{}{przeszkody prawne, uniemożliwiające ważne zawarcie związku małżeńskiego lub unieważniające już zawarte małżeństwo}
\Clue{47}{}{wieś w Polsce położona w województwie świętokrzyskim}
\Clue{48}{}{bijatyka, bójka, sytuacja, w której ludzie się nawalają ze sobą - biją się}
\Clue{49}{}{inaczej nawracanie, rotacja wewnętrzna - obrót dystalnej części kończyny do wewnątrz w stosunku do jej długiej osi, tj. od równoległego do skrętoległego położenia kości promieniowej i łokciowej względem siebie}\end{PuzzleClues}

\begin{PuzzleClues}{\textbf{Pionowe}\\}\Clue{1}{}{renowacja, odświeżenie czegoś już istniejącego}
\Clue{2}{}{nieduże, ciasne pomieszczenie, klitka}
\Clue{3}{}{brat matki}
\Clue{4}{}{wargi okalające otwór gębowy}
\Clue{5}{}{dzień św. Mikołaja (nazwa często wyrażana w dopełniaczu), który przypada 6 grudnia}
\Clue{6}{}{BATALION; ptak z rzędu mew-siewek; szata godowa samców strojna z wielobarwną kryzą, łowny; Eurazja}
\Clue{8}{}{płaskorzeźba albo malowidło w okrągłym obramowaniu}
\Clue{9}{}{napięcie między dwoma punktami jednocześnie dostępnymi nie należącymi do obwodu elektrycznego, które może dotknąć dwiema częściami ciała, np. dwiema rękami lub ręką i stopą}
\Clue{10}{}{brytyjski typ użytkowy konia o przeznaczeniu zaprzęgowym; użytkowane nie tylko jako konie zaprzęgowe, ale irodzinne wierzchowce dla mniej wprawnych jeźdźców, z uwagi na swój spokojny, zrównoważony charakter i niezbyt duże rozmiary}
\Clue{11}{}{jednostka objętości ciał sypkich w angielskich i amerykańskich układach miar 36,3687 dm3, 35,2391 dm3}
\Clue{13}{}{HANDŻAŁ, PUGINAŁ}
\Clue{14}{}{Dyscophus guineti - gatunek płaza bezogonowego z rodziny wąskopyskowatych}
\Clue{17}{}{miasto w USA (Floryda) nad Oceanem Atlantyckim, światowej sławy kąpielisko i miejscowość wypoczynkowa}
\Clue{18}{}{jednoroczna roślina z wargowatych uprawiana głównie dla przyprawowego ziela}
\Clue{20}{}{spotkanie dwóch osób mające na celu relację seksualną}
\Clue{21}{}{żartobliwie albo pogardliwie, zgrubiale: uczestnik ruchu anarchistycznego}
\Clue{23}{}{dach jednospadowy (o jednej połaci dachowej)}
\Clue{25}{}{pieszczotlwie o staruszku}
\Clue{26}{}{wyrażenie podmiotu w formie przypadka zależnego}
\Clue{28}{}{GONTYNA, KĄCINA}
\Clue{29}{}{miasto we Włoszech (Toskania) w Rudawach Toskańskich; znany w świecie ośrodek turystyczny}
\Clue{30}{}{cecha ułożenia ciał niebieskich taka, że Ziemia jest w środku układu}
\Clue{31}{}{cecha tego, co arbitralne: niepodlegające dyskusji, odgórnie ustalone, uchodzące za takie}
\Clue{32}{}{cecha czegoś, co wynika z praw natury}
\Clue{33}{}{kłusak: koń o bardzo szybkim kłusie}
\Clue{35}{}{ktoś, kto rzuca na kogoś oszczerstwa, insynuuje coś}
\Clue{40}{}{chemik niemiecki (1859-1921); zsyntetyzował wiele związków organicznych m in. pirazol i jego pochodne}
\Clue{44}{}{rok nauki z określonym programem nauczania}\end{PuzzleClues}\newpage%\section*{Krzyżówka 93}

\noindent\begin{Puzzle}{25}{21}|*	|*	|*	|*	|*	|[1][S]\darr	|*	|[2][S]\drarr	|s	|i	|e	|r	|p	|i	|e	|ń	|*	|*	|[3][S]\darr	|*	|*	|*	|*	|[4][S]\darr	|*	|*	|.
|*	|*	|*	|[5][S]\darr	|[6][S]\rarr	|p	|u	|s	|z	|k	|a	|r	|z	|*	|*	|[7][S]\darr	|*	|*	|j	|*	|*	|*	|*	|b	|*	|*	|.
|*	|*	|*	|k	|*	|i	|*	|i	|*	|*	|*	|*	|*	|*	|[8][S]\rarr	|k	|w	|i	|a	|t	|*	|*	|*	|y	|*	|*	|.
|[9][S]\drarr	|d	|z	|i	|e	|r	|g	|a	|c	|z	|[][,]{ }	|ł	|u	|s	|k	|o	|w	|a	|n	|y	|*	|*	|*	|s	|*	|*	|.
|w	|*	|*	|k	|*	|a	|*	|d	|*	|[10][S]\rarr	|t	|w	|ó	|r	|[][,]{ }	|s	|i	|a	|t	|k	|o	|w	|a	|t	|y	|*	|.
|i	|*	|*	|o	|*	|n	|*	|[][,]{ }	|*	|*	|*	|*	|*	|*	|[11][S]\rarr	|z	|a	|s	|a	|d	|a	|*	|*	|r	|[12][S]\darr	|*	|.
|l	|*	|*	|n	|*	|o	|*	|p	|[13][S]\rarr	|e	|l	|e	|m	|e	|n	|t	|a	|r	|z	|*	|*	|*	|*	|z	|u	|*	|.
|k	|*	|*	|g	|[14][S]\rarr	|m	|a	|r	|g	|r	|a	|f	|*	|*	|*	|[][,]{ }	|*	|*	|a	|*	|*	|*	|*	|y	|l	|*	|.
|[][,]{ }	|*	|*	|o	|*	|e	|*	|o	|*	|[15][S]\rarr	|p	|e	|r	|i	|z	|o	|n	|i	|u	|m	|*	|*	|*	|k	|o	|*	|.
|p	|*	|*	|*	|*	|t	|*	|s	|*	|*	|*	|*	|*	|*	|*	|c	|[16][S]\drarr	|g	|r	|a	|f	|a	|*	|[][,]{ }	|t	|*	|.
|s	|*	|*	|*	|[17][S]\drarr	|r	|a	|t	|l	|e	|r	|k	|a	|*	|*	|e	|l	|*	|*	|*	|*	|*	|*	|r	|n	|*	|.
|z	|*	|*	|*	|s	|*	|*	|y	|[18][S]\rarr	|b	|i	|a	|ł	|a	|[][,]{ }	|n	|o	|c	|*	|*	|*	|*	|*	|ó	|o	|*	|.
|c	|*	|*	|*	|u	|*	|*	|*	|*	|[19][S]\darr	|*	|*	|*	|*	|*	|y	|r	|*	|*	|*	|*	|*	|*	|ż	|ś	|*	|.
|z	|*	|*	|*	|k	|*	|*	|*	|*	|p	|*	|*	|*	|*	|*	|[][,]{ }	|c	|*	|*	|*	|*	|[20][S]\darr	|*	|o	|ć	|*	|.
|e	|*	|*	|*	|c	|*	|*	|*	|[21][S]\rarr	|i	|n	|w	|e	|r	|s	|j	|a	|[][,]{ }	|o	|p	|a	|d	|ó	|w	|*	|*	|.
|l	|*	|[22][S]\rarr	|c	|e	|g	|ł	|a	|[][,]{ }	|s	|i	|t	|ó	|w	|k	|a	|*	|*	|*	|*	|*	|e	|*	|y	|*	|*	|.
|i	|*	|*	|*	|s	|*	|*	|*	|*	|e	|*	|*	|[23][S]\rarr	|o	|r	|k	|i	|s	|z	|*	|*	|s	|*	|*	|*	|*	|.
|*	|*	|*	|*	|i	|*	|*	|*	|*	|m	|*	|[24][S]\rarr	|p	|r	|z	|o	|d	|o	|w	|n	|i	|k	|*	|*	|*	|*	|.
|*	|*	|*	|*	|k	|*	|*	|*	|*	|k	|*	|*	|*	|*	|*	|ś	|*	|[25][S]\rarr	|u	|r	|n	|a	|*	|*	|*	|*	|.
|*	|*	|*	|*	|*	|*	|*	|*	|*	|o	|*	|*	|*	|*	|*	|c	|[26][S]\rarr	|w	|r	|ó	|g	|*	|*	|*	|*	|*	|.
|*	|*	|*	|*	|*	|*	|*	|*	|*	|*	|*	|*	|[27][S]\rarr	|a	|d	|i	|a	|f	|o	|r	|a	|*	|*	|*	|*	|*	|.
|*	|*	|*	|*	|*	|*	|*	|*	|*	|[28][S]\rarr	|o	|m	|l	|e	|t	|*	|*	|*	|*	|*	|*	|*	|*	|*	|*	|*	|.\end{Puzzle}

\newpage

\begin{PuzzleClues}{\textbf{Poziome}\\}\Clue{2}{}{fala strajków w sierpniu 1980 roku, zakończona podpisaniem porozumień sierpniowych i powstaniem Solidarności}
\Clue{6}{}{rzemieślinik zajmujący się wyrobem broni palnej}
\Clue{8}{}{łow. ogon zwierzyny płowej i borsuka}
\Clue{9}{}{Plocepasser donaldsoni - gatunek ptaka z rodziny wróblowatych (Passeridae) występujący w Afryce}
\Clue{10}{}{struktura pnia mózgu, odpowiedzialna za odbieranie sygnałów z różnych części ciała i utrzymywania struktur mózgu w ciągłej gotowości}
\Clue{11}{}{umowa, że coś uznajemy za coś innego; coś będzie od tej pory funkcjonowało podobnie do tego drugiego}
\Clue{13}{}{podstawy, elementarne wiadomości, ABC}
\Clue{14}{}{zwierzchnik marchii}
\Clue{15}{}{drapowana opaska na biodrach Chrystusa}
\Clue{16}{}{sposób, w jaki w komputerze i w wytworach komputerowych wyświetla się obraz; jakość obrazu rozpatrywana zarówno pod względem technicznym jak i artystycznym}
\Clue{17}{}{suczka ratlerka}
\Clue{18}{}{noc nieprzespana, bezsenna}
\Clue{21}{}{inwersja opadowa - zmniejszenie ilości opadów atmosferycznych w górach wraz ze znacznym wzrostem wysokości}
\Clue{22}{}{cegła ceramiczna o dużej liczbie małych otworów prostopadłych do podstawy cegły}
\Clue{23}{}{jęczmień dwurzędowy, Hordeum distichon - odmiana jęczmienia}
\Clue{24}{}{przedsiębiorstwo lub osoba, które przoduje w czymś}
\Clue{25}{}{naczynie służące do przechowywania prochów zmarłego}
\Clue{26}{}{ktoś, z kim trzeba się zmierzyć, kogo/co trzeba zwalczyć, pokonać, ktoś lub coś, z czym się walczy, bije, zwłaszcza nieprzyjaciel polityczno-geograficzny (wrogie wojsko, państwo)}
\Clue{27}{}{w religii - zwyczaj, rytuał, który jest dopuszczony, ale nie nakazany}
\Clue{28}{}{potrawa sporządzana ze starannie rozmąconych jaj z innymi dodatkami, które wylewa się na patelnię i nie miesza ich w czasie smażenia}\end{PuzzleClues}

\begin{PuzzleClues}{\textbf{Pionowe}\\}\Clue{1}{}{przyrząd do pomiaru rozproszonego promieniowania słonecznego}
\Clue{2}{}{siad, w którym plecy są wyprosotwane, a nogi skierowane wprzód, wyprostowane w kolanach i złączone}
\Clue{3}{}{Ianthasaurus - nazwa rodzajowa synapsyda, zaliczanego do rodziny edafozaurów, żyjącego pod koniec karbonu (późny pensylwan, około 300 milionów lat temu); jego szczątki odkryto na terenie obecnego stanu Kansas}
\Clue{4}{}{Hyphessobrycon rosaceus - gatunek ryby z rodziny kąsaczowatych (Characidae)}
\Clue{5}{}{jeden z języków afrykańskich}
\Clue{7}{}{koszty mające na celu kontrolę stopnia zbieżności jakości wymaganej danego towaru z jakością uzyskaną}
\Clue{9}{}{taszczyn pszczeli, Philanthus triangulum - owad zaliczany do rodziny grzebaczowatych z rzędu błonkówek; wilk pszczeli ma ubarwienie czarne w żółte trójkątne plamki i paski; zamieszkuje piaszczyste lasy, żywi się pszczołami miodnymi, którym odbiera zebrany nektar, paraliżując je przy użyciu jadu, a następnie zaciąga je do wygrzebanej w ziemi norki}
\Clue{12}{}{krótkotrwałość, np. uczuć}
\Clue{16}{}{miasto w Hiszpanii (Murcja) nad rzeką Sangonerą}
\Clue{17}{}{spełnienie w sferze życiowych dążeń (zdobycie powodzenia, sławy, majątku itp.)}
\Clue{19}{}{list z jakiegoś urzędu lub instytucji}
\Clue{20}{}{narta - podłużny, wąski przyrząd sportowy do jazdy po śniegu, wodzie lub innym podłożu}\end{PuzzleClues}\newpage%\section*{Krzyżówka 94}

\noindent\begin{Puzzle}{18}{29}|*	|*	|*	|*	|*	|*	|*	|*	|*	|*	|*	|*	|*	|[1][S]\drarr	|r	|u	|r	|a	|*	|.
|*	|*	|*	|*	|*	|*	|*	|*	|[2][S]\rarr	|d	|z	|i	|a	|b	|k	|a	|*	|[3][S]\darr	|*	|.
|*	|*	|*	|*	|*	|*	|*	|*	|*	|*	|*	|*	|*	|r	|[4][S]\drarr	|ł	|ó	|j	|*	|.
|*	|*	|[5][S]\darr	|*	|[6][S]\rarr	|v	|e	|r	|r	|o	|c	|c	|h	|i	|o	|*	|[7][S]\darr	|o	|[8][S]\darr	|.
|*	|[9][S]\rarr	|e	|k	|s	|p	|o	|z	|y	|c	|j	|a	|*	|e	|d	|[10][S]\darr	|b	|g	|p	|.
|*	|[11][S]\rarr	|p	|r	|z	|e	|t	|y	|k	|a	|n	|i	|e	|*	|c	|d	|r	|i	|o	|.
|*	|*	|o	|*	|[12][S]\rarr	|l	|e	|w	|[][,]{ }	|g	|ó	|r	|s	|k	|i	|*	|a	|n	|ś	|.
|*	|*	|k	|*	|*	|[13][S]\rarr	|g	|r	|a	|d	|s	|t	|e	|i	|n	|*	|z	|i	|l	|.
|*	|[14][S]\darr	|a	|[15][S]\darr	|[16][S]\drarr	|o	|p	|a	|k	|o	|w	|a	|n	|i	|e	|*	|y	|*	|u	|.
|*	|k	|[][,]{ }	|s	|d	|[17][S]\drarr	|h	|a	|l	|o	|g	|e	|n	|e	|k	|*	|l	|[18][S]\darr	|b	|.
|*	|o	|l	|z	|o	|p	|*	|*	|*	|[19][S]\drarr	|n	|y	|s	|a	|*	|*	|i	|e	|i	|.
|*	|p	|o	|t	|b	|d	|[20][S]\darr	|[21][S]\rarr	|a	|p	|o	|r	|i	|a	|*	|*	|j	|b	|n	|.
|*	|a	|d	|u	|r	|*	|d	|[22][S]\drarr	|u	|ł	|a	|n	|i	|*	|*	|*	|k	|e	|y	|.
|[23][S]\drarr	|n	|o	|t	|a	|c	|j	|a	|*	|o	|*	|*	|[24][S]\rarr	|o	|c	|z	|a	|r	|*	|.
|w	|i	|w	|o	|[][,]{ }	|[25][S]\darr	|f	|r	|[26][S]\rarr	|c	|z	|u	|b	|a	|c	|z	|*	|t	|*	|.
|a	|c	|a	|w	|f	|k	|*	|g	|*	|h	|[27][S]\darr	|[28][S]\darr	|[29][S]\darr	|*	|[30][S]\darr	|[31][S]\darr	|[32][S]\darr	|h	|*	|.
|z	|z	|*	|i	|i	|ą	|[33][S]\darr	|y	|[34][S]\drarr	|l	|o	|d	|o	|z	|w	|a	|ł	|*	|*	|.
|e	|a	|*	|a	|n	|t	|k	|r	|a	|i	|r	|o	|b	|*	|a	|m	|u	|*	|*	|.
|l	|n	|[35][S]\darr	|n	|a	|[][,]{ }	|o	|o	|u	|w	|z	|n	|r	|*	|t	|n	|k	|*	|*	|.
|i	|k	|w	|i	|l	|p	|r	|z	|g	|o	|e	|i	|u	|[36][S]\darr	|e	|i	|[][,]{ }	|*	|*	|.
|n	|a	|y	|n	|n	|ł	|k	|a	|e	|ś	|s	|e	|s	|n	|r	|o	|p	|*	|*	|.
|a	|*	|d	|*	|e	|a	|o	|u	|r	|ć	|z	|s	|*	|u	|s	|c	|l	|*	|*	|.
|*	|*	|o	|*	|*	|s	|c	|r	|u	|*	|n	|i	|*	|r	|a	|e	|a	|*	|*	|.
|[37][S]\drarr	|k	|l	|o	|c	|k	|i	|*	|t	|*	|i	|e	|*	|z	|l	|n	|z	|*	|*	|.
|s	|[38][S]\drarr	|n	|a	|c	|i	|ą	|g	|*	|*	|c	|n	|[39][S]\darr	|a	|i	|t	|m	|*	|*	|.
|i	|k	|o	|*	|*	|*	|g	|*	|*	|*	|a	|i	|d	|n	|n	|e	|o	|*	|*	|.
|ł	|r	|ś	|*	|*	|*	|*	|*	|*	|*	|*	|e	|o	|i	|g	|z	|w	|*	|*	|.
|a	|i	|ć	|[40][S]\rarr	|p	|r	|a	|k	|r	|y	|t	|*	|s	|e	|*	|a	|y	|*	|*	|.
|*	|s	|*	|*	|[41][S]\rarr	|r	|o	|z	|w	|i	|d	|l	|a	|c	|z	|*	|*	|*	|*	|.
|*	|*	|*	|*	|*	|*	|*	|*	|*	|*	|*	|*	|*	|*	|*	|*	|*	|*	|*	|.\end{Puzzle}

\newpage

\begin{PuzzleClues}{\textbf{Poziome}\\}\Clue{1}{}{element konstrukcyjny o pierścieniowym przekroju poprzecznym i znacznej długości, stosowane są do prowadzenia cieczy i gazów lub jako elementy budowy maszyn i urządzeń}
\Clue{2}{}{czekan do wspinaczki w lodzie lub drytoolingu}
\Clue{4}{}{naturalny lubrykant skóry ssaków wydzielany przez gruczoły łojowe}
\Clue{6}{}{grafik (1914-54) jeden z głównych twórców współczesnego plakatu polskiego}
\Clue{9}{}{ilość światła padającego na film (lub na sensor elektroniczny w przypadku aparatu cyfrowego) konieczna dla prawidłowego zrobienia zdjęcia fotograficznego}
\Clue{11}{}{wtykać wiele czegoś w różnych miejscach, przybierać coś czymś}
\Clue{12}{}{drapieżny ssak z rodziny kotowatych o stosunkowo małej głowie, smukłym i gibkim tułowiu}
\Clue{13}{}{kompozytor (1904-1954); utwory orkiestrowe, kameralne, fortepianowe, kantaty, pieśni}
\Clue{16}{}{rzecz, która służy jako osłona czegoś, czasem jako pojemnik na coś}
\Clue{17}{}{ogólna nazwa soli kwasu halogenowodorowego: fluorowodorowego, chlorowodorowego, bromowodorowego lub jodowodorowego}
\Clue{19}{}{marka samochodu (mikrobusa) polskiej produkcji (produkowany w latach 1958-1994); wytwór Zakładu Samochodów Dostawczych w Nysie}
\Clue{21}{}{trudność w logicznym rozumowaniu, pozornie niemożliwa do przezwyciężenia, skutkująca sprzecznymi albo przeciwstawnymi rozwiązaniami}
\Clue{22}{}{żołnierze lekkiej kawalerii, utworzonej w Polsce w XVIII w}
\Clue{23}{}{oznaczanie czegoś określonymi znakami umownymi}
\Clue{24}{}{ozdobny krzew lub drzewko z Azji Wsch. i Ameryki Północnej, liście owłosione, kwiaty żółte}
\Clue{26}{}{duży amerykański ptak leśny z rzędu kuraków, głównie owocożerny, łowny}
\Clue{34}{}{zwał brył lodowych}
\Clue{37}{}{zabawka dziecięca w formie różnokolorowych wielościanów, wykonana zazwyczaj z tworzywa sztucznego (najczęściej plastiku) bądź drewna}
\Clue{38}{}{urządzenie w mechanizmie zegarowym pośredniczące w przekazywaniu energii z zewnątrz do elementu napędowego}
\Clue{40}{}{język z grupy prakrytów, które są językami indoaryjskimi}
\Clue{41}{}{przyrząd, który rozwidla}\end{PuzzleClues}

\begin{PuzzleClues}{\textbf{Pionowe}\\}\Clue{1}{}{miękki ser podpuszczkowy o ostrym smaku z warstewką pleśni na powierzchni}
\Clue{3}{}{adeptka systemu rozwoju duchowego w hinduizmie i buddyzmie}
\Clue{4}{}{kwit, odcięta część dokumentu}
\Clue{5}{}{okres zlodowaceń w Europie odpowiadający plejstocenowi, zaproponowany w 1838 przez niemieckiego botanika K. Schimpera jako alternatywa terminu dyluwium}
\Clue{7}{}{mieszkanka Brazylii, kobieta pochodzenia brazylijskiego}
\Clue{8}{}{wstąpienie w związek małżeński}
\Clue{10}{}{litera oznaczająca wymiar}
\Clue{14}{}{mieszkanka Kopanicy}
\Clue{15}{}{mieszkaniec Sztutowa, człowiek pochodzący ze Sztutowa - wsi na terenie Żuław Wiślanych}
\Clue{16}{}{dobra nabyte przez ostatecznego użytkownika}
\Clue{17}{}{w chemii: symbol palladu}
\Clue{18}{}{anatomopatolog niemiecki (1835-1926); odkrył w 1880 r. pałeczkę duru brzusznego}
\Clue{19}{}{cecha kogoś, kto łatwo się płoszy}
\Clue{20}{}{kod ISO 4217 franka dżibutyjskiego}
\Clue{22}{}{Argyrosaurus - rodzaj zauropoda z grupy tytanozaurów; żył w epoce późnej kredy na terenach Ameryki Południowej}
\Clue{23}{}{mieszanina alkanów z pogranicza stanu ciekłego i stałego, tworząca mazistą, niskotopliwą, niewysychającą, bezwonną i dość rzadką substancję o dużej przezierności, bezbarwną albo koloru od żółtego do brązowego}
\Clue{25}{}{część płaszczyzny ograniczona dwoma półprostymi (ramionami kąta) wychodzącymi z jednego punktu (wierzchołka kąta)}
\Clue{27}{}{Muscardinus avellanarius - gatunek niewielkiego rudego gryzonia z rodziny popielicowatych, jedynego przedstawiciela rodzaju Muscardinus, mieszkającego w lasach; występuje od Wielkiej Brytanii poprzez Francję aż do zachodniej Rosji, na północy zamieszkuje tylko południową część Szwecji}
\Clue{28}{}{zadenuncjować, poinformować o czymś alarmującym, naskarżyć}
\Clue{29}{}{tkanina lub dzianina używana do przykrycia stołu, zwykle na czas posiłków, pełniąca funkcję dekoracyjną i ochronną}
\Clue{30}{}{BLINDGAFEL}
\Clue{31}{}{inwazyjne badanie prenatalne, polegające na punkcji jamy owodni przez powłoki jamy brzusznej ciężarnej, pod kontrolą ultrasonograficzną, celem aspiracji płynu owodniowego zawierającego złuszczone komórki płodu}
\Clue{32}{}{strumień plazmy wydzielany z palnika plazmowego, służący m.in. do cięcia metalu}
\Clue{33}{}{autorotacyjna figura akrobacji lotniczej, odbywająca się na pozakrytycznych kątach natarcia po linii śrubowej o małym promieniu, przy częściowej lub całkowitej utracie sterowności}
\Clue{34}{}{wielka oaza w płn. Saharze, należy do Algierii}
\Clue{35}{}{cecha sprawnego funkcjonowania przynoszącego efekty, zyski}
\Clue{36}{}{Vallisneria - rodzaj roślin wodnych z rodziny żabiściekowatych}
\Clue{37}{}{energia, którą dysponuje człowiek, witalność, tyle zdrowia i zapału, ile ma w danej chwili}
\Clue{38}{}{sztylet lub puginał malajski, zwykle o wężowatej klindze z bruzdami i często małą haczykowatą wypustką}
\Clue{39}{}{potrawa indyjska, cienki placek z soczewicy i ryżu}\end{PuzzleClues}\newpage%\section*{Krzyżówka 96}

\noindent\begin{Puzzle}{17}{26}|*	|*	|*	|[1][S]\drarr	|t	|r	|o	|m	|b	|o	|n	|i	|s	|t	|a	|*	|*	|*	|.
|*	|[2][S]\rarr	|l	|e	|w	|i	|s	|*	|[3][S]\darr	|*	|[4][S]\darr	|*	|*	|*	|[5][S]\darr	|*	|*	|*	|.
|*	|*	|*	|l	|*	|*	|*	|[6][S]\darr	|d	|[7][S]\drarr	|s	|u	|m	|a	|k	|*	|*	|[8][S]\darr	|.
|*	|*	|[9][S]\darr	|e	|*	|*	|*	|s	|z	|i	|c	|*	|[10][S]\darr	|*	|a	|[11][S]\darr	|*	|p	|.
|*	|*	|g	|k	|[12][S]\darr	|*	|[13][S]\darr	|k	|i	|m	|h	|*	|b	|*	|s	|g	|*	|o	|.
|*	|*	|a	|t	|k	|*	|ł	|l	|ę	|p	|i	|[14][S]\darr	|a	|*	|a	|u	|*	|z	|.
|*	|*	|n	|r	|a	|*	|a	|e	|c	|a	|p	|b	|l	|*	|c	|s	|*	|a	|.
|*	|*	|z	|o	|m	|[15][S]\darr	|j	|j	|i	|s	|a	|a	|o	|*	|j	|i	|*	|b	|.
|*	|*	|h	|r	|e	|g	|d	|a	|o	|*	|*	|a	|t	|*	|a	|e	|*	|i	|.
|*	|*	|o	|e	|r	|w	|a	|c	|ł	|[16][S]\drarr	|a	|d	|a	|m	|*	|w	|*	|b	|.
|[17][S]\drarr	|p	|u	|c	|h	|a	|c	|z	|[][,]{ }	|ś	|r	|e	|d	|n	|i	|*	|*	|l	|.
|b	|*	|*	|e	|e	|n	|t	|*	|s	|w	|*	|*	|a	|[18][S]\darr	|*	|*	|*	|i	|.
|ą	|[19][S]\drarr	|s	|p	|r	|a	|w	|d	|z	|i	|a	|n	|*	|c	|*	|*	|[20][S]\darr	|j	|.
|k	|l	|*	|c	|*	|k	|o	|*	|a	|s	|*	|*	|[21][S]\darr	|z	|*	|[22][S]\darr	|g	|n	|.
|o	|w	|*	|j	|[23][S]\darr	|o	|*	|*	|r	|t	|[24][S]\darr	|*	|k	|e	|*	|ż	|e	|o	|.
|j	|i	|*	|a	|s	|*	|*	|*	|y	|u	|i	|*	|o	|r	|[25][S]\darr	|ó	|n	|ś	|.
|a	|a	|*	|*	|z	|*	|[26][S]\darr	|*	|*	|n	|ł	|*	|d	|s	|p	|ł	|i	|ć	|.
|d	|*	|[27][S]\drarr	|r	|ó	|ż	|a	|[][,]{ }	|s	|k	|o	|c	|z	|k	|o	|w	|a	|*	|.
|*	|[28][S]\darr	|p	|[29][S]\darr	|s	|[30][S]\darr	|l	|[31][S]\rarr	|n	|a	|w	|o	|i	|*	|n	|[][,]{ }	|l	|*	|.
|[32][S]\drarr	|h	|a	|p	|t	|o	|d	|u	|s	|*	|i	|*	|k	|*	|o	|b	|n	|*	|.
|w	|a	|w	|i	|k	|k	|i	|[33][S]\rarr	|t	|r	|e	|p	|*	|*	|r	|ł	|o	|*	|.
|y	|n	|ł	|n	|a	|o	|s	|[34][S]\rarr	|z	|a	|c	|h	|ó	|d	|*	|o	|ś	|*	|.
|w	|z	|o	|i	|*	|*	|*	|*	|*	|*	|*	|*	|*	|*	|*	|t	|ć	|*	|.
|ó	|a	|w	|o	|[35][S]\rarr	|c	|z	|c	|h	|o	|w	|i	|a	|n	|i	|n	|*	|*	|.
|d	|*	|*	|l	|*	|*	|*	|*	|*	|*	|*	|[36][S]\rarr	|d	|u	|d	|y	|*	|*	|.
|*	|[37][S]\rarr	|z	|a	|b	|o	|r	|o	|w	|o	|*	|*	|*	|*	|*	|*	|*	|*	|.
|*	|*	|*	|*	|*	|*	|*	|*	|*	|*	|*	|*	|*	|*	|*	|*	|*	|*	|.\end{Puzzle}

\newpage

\begin{PuzzleClues}{\textbf{Poziome}\\}\Clue{1}{}{osoba grająca na trombonie; puzonista}
\Clue{2}{}{angielski, lekki karabin maszynowy, kaliber 7,7 mm}
\Clue{7}{}{barwnik pozyskiwany z liści i gałęzi sumaku garbarskiego używany w garbarstwie oraz w farbiarstwie do barwienia na czarno}
\Clue{16}{}{każdy człowiek noszący imię Adam}
\Clue{17}{}{Bubo bubo interpositus - podgatunek ptaka drapieżnego wyróżniony w obrębie gatunku puchacz zwyczajny (Bubo bubo); występuje na Ukrainie i Kaukazie}
\Clue{19}{}{przyrząd pomiarowy, przy pomocy którego stwierdza się, czy dany wymiar przedmiotu jest prawidłowy i nie przekracza wartości granicznej (dolnej i górnej)}
\Clue{27}{}{taka pozycja skoczka, z której może on wykonać ruch w każdą stronę}
\Clue{31}{}{miasto w Uzbekistanie w dolinie Zerawszanu; duża elektrownia cieplna}
\Clue{32}{}{Haptodus - nazwa rodzajowa zwierzęcia uznawanego za jednego z najstarszych przedstawicieli synapsydów; żył w okresie późnego karbonu i wczesnego permu 300-280 mln lat temu, zamieszkując tereny Pangei}
\Clue{33}{}{stary, sfatygowany but, a także: duży, ciężki i solidny but (np. wojskowy, albo górski)}
\Clue{34}{}{wysiłek w zabieganiu o coś}
\Clue{35}{}{mieszkaniec Czchowa}
\Clue{36}{}{dęty instrument muzyczny - GAJDY, KOBZA,}
\Clue{37}{}{dawne miasto, obecnie dzielnica Leszna (województwo wielkopolskie)}\end{PuzzleClues}

\begin{PuzzleClues}{\textbf{Pionowe}\\}\Clue{1}{}{zdolność odbierania bodźców elektrycznych z otoczenia, wykrywanie obecności oraz zmian pola elektrycznego za pomocą elektroreceptorów}
\Clue{3}{}{Dendropicos elachus - gatunek ptaka z rodziny dzięciołowatych (Picidae), z podrodziny dzięciołów (Picinae)}
\Clue{4}{}{włoski śpiewak; tenor liryczny (1889-1965); solista czołowych teatrów operowych}
\Clue{5}{}{nadzwyczajny środek zaskarżenia od prawomocnego wyroku sądu odwoławczego}
\Clue{6}{}{osoba zajmująca się sklejaniem potłuczonych przedmiotów}
\Clue{7}{}{sytaucja, w której nic się nie dzieje, nie ma dążenia do zmiany}
\Clue{8}{}{w religioznawstwie - cecha czegoś, co pochodzi z innego źródła niż Biblia, ale jest np. przedmiotem teologicznej, religijnej refleksji; np. pozabiblijność tekstu źródłowego}
\Clue{9}{}{miasto w Chinach nad rzeką Gian Jiang, ośrodek wydobycia rudy wolframu}
\Clue{10}{}{rodzaj podskoku konia na miejscu na tylnych nogach bez wierzgnięcia}
\Clue{11}{}{(1909-44), rosyjski poeta, dramatopisarz, scenarzysta}
\Clue{12}{}{szambelan nadworny}
\Clue{13}{}{podły czyn, nieuczciwość, zachowanie charakterystyczne dla łajdaka}
\Clue{14}{}{Walter, ur. 1893r. astrofizyk amerykański; odkrywca populacji gwiazdowych, dokonał rewizji skali odległości galaktyk}
\Clue{15}{}{GUANAKO}
\Clue{16}{}{chroniony ptak leśny z rzędu wróblowatych, wydaje świszczący głos, zimuje w Afryce}
\Clue{17}{}{drobny, owadożerny ptak afrykański; towarzyszy dużym ssakom wyjadając im pasożyty skóry}
\Clue{18}{}{miasto w południowej części województwa pomorskiego, w powiecie chojnickim, siedziba gminy miejsko-wiejskiej Czersk}
\Clue{19}{}{zatoka Morza Śródziemnego, u wybrzeży Francji, głębokość do 1000 m}
\Clue{20}{}{cecha kogoś genialnego, wybitnie uzdolnionego w jakiejś dziedzinie, talent}
\Clue{21}{}{kod źródłowy}
\Clue{22}{}{Emys orbicularis - gatunek żółwia z rodziny żółwi błotnych, z podrzędu żółwi skrytoszyjnych; jedyny żyjący naturalnie w Polsce, w środowisku naturalnym żyje ponad 100 lat}
\Clue{23}{}{ocena w szkole (w polskiej szkole obecnie najlepsza)}
\Clue{24}{}{niezłupkowacona skała osadowej w postaci skonsolidowanego iłu}
\Clue{25}{}{miejsce wpływu wód pod powierzchnię terenów na obszarach krasowych}
\Clue{26}{}{rodzaj przenośnej lampy sygnalizacyjnej umożliwiającej nadawanie sygnałów}
\Clue{27}{}{teoria Pawłowa}
\Clue{28}{}{w średniowieczu: związek kupców lub miast}
\Clue{29}{}{jadalny orzeszek sosny pinii. bogaty w tłuszcz, długości do 2 cm}
\Clue{30}{}{to, co jest kształtem lub funkcją podobne do oka}
\Clue{32}{}{dzieło poświęcone wykrywaniu początku czegoś}\end{PuzzleClues}\newpage%\section*{Krzyżówka 98}

\noindent\begin{Puzzle}{21}{31}|*	|*	|[1][S]\darr	|*	|*	|*	|*	|[2][S]\drarr	|c	|e	|z	|a	|*	|[3][S]\drarr	|l	|i	|c	|z	|i	|*	|[4][S]\darr	|*	|.
|*	|[5][S]\darr	|l	|[6][S]\darr	|*	|[7][S]\darr	|[8][S]\darr	|n	|[9][S]\darr	|*	|*	|*	|*	|r	|[10][S]\drarr	|m	|e	|n	|i	|s	|k	|*	|.
|*	|b	|u	|k	|*	|u	|d	|i	|z	|[11][S]\darr	|[12][S]\rarr	|k	|r	|e	|s	|c	|y	|t	|y	|w	|a	|*	|.
|[13][S]\drarr	|u	|p	|o	|m	|n	|i	|e	|n	|i	|e	|*	|[14][S]\rarr	|n	|e	|m	|e	|t	|h	|*	|l	|*	|.
|c	|ł	|a	|r	|[15][S]\darr	|i	|a	|z	|*	|n	|[16][S]\darr	|*	|*	|t	|l	|*	|*	|*	|*	|*	|o	|[17][S]\darr	|.
|h	|g	|n	|a	|b	|w	|k	|n	|[18][S]\darr	|k	|s	|*	|[19][S]\darr	|a	|e	|*	|*	|*	|*	|*	|r	|h	|.
|e	|a	|*	|l	|u	|e	|*	|a	|w	|u	|ó	|*	|p	|[][,]{ }	|n	|*	|[20][S]\darr	|*	|*	|[21][S]\darr	|i	|o	|.
|m	|r	|*	|*	|k	|r	|[22][S]\darr	|j	|i	|b	|l	|*	|r	|p	|o	|*	|i	|*	|*	|c	|a	|d	|.
|i	|y	|[23][S]\drarr	|j	|a	|s	|n	|o	|t	|a	|*	|*	|o	|l	|n	|*	|p	|*	|*	|i	|*	|o	|.
|a	|z	|a	|*	|c	|a	|i	|m	|a	|t	|*	|*	|k	|a	|a	|*	|e	|*	|*	|ś	|[24][S]\darr	|w	|.
|[][,]{ }	|a	|k	|*	|i	|ł	|e	|y	|c	|o	|*	|[25][S]\darr	|a	|n	|u	|*	|r	|*	|*	|n	|r	|l	|.
|f	|c	|r	|*	|a	|*	|w	|*	|z	|r	|[26][S]\darr	|k	|r	|i	|t	|[27][S]\darr	|y	|*	|*	|i	|ó	|a	|.
|i	|j	|y	|*	|r	|*	|ó	|[28][S]\darr	|*	|[][,]{ }	|m	|o	|b	|s	|y	|m	|t	|*	|*	|e	|w	|[][,]{ }	|.
|z	|a	|l	|*	|n	|*	|d	|z	|[29][S]\drarr	|p	|a	|ł	|a	|t	|k	|a	|*	|*	|*	|n	|n	|z	|.
|j	|*	|o	|[30][S]\darr	|i	|*	|*	|n	|f	|r	|l	|o	|z	|y	|a	|r	|[31][S]\darr	|*	|*	|i	|i	|w	|.
|o	|*	|n	|b	|a	|*	|*	|a	|u	|z	|m	|m	|y	|c	|*	|t	|o	|*	|*	|e	|k	|i	|.
|l	|*	|i	|r	|*	|*	|*	|c	|t	|e	|o	|y	|n	|z	|*	|w	|b	|*	|*	|[][,]{ }	|o	|e	|.
|o	|[32][S]\rarr	|t	|e	|j	|u	|*	|z	|r	|d	|*	|j	|a	|n	|[33][S]\darr	|i	|e	|[34][S]\darr	|*	|o	|w	|r	|.
|g	|*	|r	|n	|*	|*	|*	|e	|ó	|s	|[35][S]\darr	|k	|*	|a	|p	|c	|r	|t	|*	|s	|y	|z	|.
|i	|*	|y	|n	|*	|[36][S]\drarr	|a	|k	|w	|i	|t	|a	|n	|*	|ł	|a	|ż	|a	|*	|m	|[][,]{ }	|ą	|.
|c	|[37][S]\drarr	|l	|a	|r	|w	|a	|*	|k	|ę	|r	|*	|*	|[38][S]\darr	|o	|[][,]{ }	|y	|p	|*	|o	|p	|t	|.
|z	|s	|*	|*	|[39][S]\drarr	|s	|y	|l	|a	|b	|a	|[][,]{ }	|z	|a	|m	|k	|n	|i	|ę	|t	|a	|*	|.
|n	|z	|*	|*	|h	|k	|*	|*	|*	|i	|p	|*	|*	|k	|i	|o	|a	|r	|[40][S]\darr	|y	|s	|*	|.
|a	|l	|*	|*	|i	|a	|*	|*	|*	|o	|e	|*	|*	|t	|e	|r	|*	|[][,]{ }	|w	|c	|[][,]{ }	|*	|.
|*	|a	|*	|*	|p	|z	|[41][S]\rarr	|s	|p	|r	|z	|ę	|t	|*	|ń	|k	|[42][S]\darr	|a	|o	|z	|c	|*	|.
|*	|r	|*	|*	|s	|ó	|*	|*	|*	|c	|o	|*	|*	|*	|*	|o	|f	|n	|l	|n	|i	|*	|.
|*	|n	|*	|*	|t	|w	|[43][S]\rarr	|e	|l	|z	|e	|w	|i	|r	|*	|w	|a	|t	|f	|e	|s	|*	|.
|[44][S]\drarr	|i	|n	|d	|e	|k	|s	|*	|*	|o	|d	|*	|*	|[45][S]\rarr	|t	|a	|l	|a	|*	|*	|z	|*	|.
|i	|k	|[46][S]\rarr	|t	|r	|a	|m	|p	|*	|ś	|r	|*	|*	|*	|*	|*	|a	|*	|*	|*	|y	|*	|.
|w	|*	|*	|*	|*	|*	|*	|*	|*	|c	|*	|*	|*	|*	|*	|*	|*	|*	|*	|*	|*	|*	|.
|a	|*	|*	|*	|*	|*	|[47][S]\rarr	|s	|p	|i	|c	|h	|r	|z	|*	|*	|*	|*	|*	|*	|*	|*	|.
|*	|[48][S]\rarr	|a	|m	|a	|d	|e	|u	|s	|*	|*	|*	|*	|*	|*	|*	|*	|*	|*	|*	|*	|*	|.\end{Puzzle}

\newpage

\begin{PuzzleClues}{\textbf{Poziome}\\}\Clue{2}{}{sieć, niewód do połowu ryb dennych}
\Clue{3}{}{niskie drzewo pochodzące z Chin uprawiane ze względu na smaczne, czerwone owoce - śliwka chińska}
\Clue{10}{}{znajdująca się między powierzchniami stawowymi kości tarczka z chrząstki włóknistej w niektórych stawach, np. żuchwowym, kolanowym}
\Clue{12}{}{wzrost zamożności, znaczenia, poprawa bytu}
\Clue{13}{}{pouczenie o tym, że zachowanie, postępowanie było niewłaściwe}
\Clue{14}{}{(1901-75), pisarz węgierski, powieści psychologiczne, dramaty historyczne, eseje; „Litość”, „Odraza”, „Galilea”}
\Clue{23}{}{eurazjatycka roślina zielna z wargowatych tzw. 'głucha pokrzywa' o leczniczych kwiatach}
\Clue{29}{}{brezentowa płachta używana przez wojsko i harcerstwo do sporządzania namiotów i jako okrycie wierzchnie}
\Clue{32}{}{TEŻU; południowoamerykańska. jaszczurka o plamistym ubarwieniu łowiona ze względu na smaczne mięso}
\Clue{36}{}{najstarszy wiek epoki miocenu w erze kenozoicznej, trwający 2,5 miliona lat}
\Clue{37}{}{postać w rozwoju zwierząt różniąca się od osobnika dojrzałego wyglądem, budową, trybem życia}
\Clue{39}{}{sylaba, która kończy się spółgłoską, np. dom, kak-tus}
\Clue{41}{}{żartobliwie lub ironicznie o męskim przyrodzeniu}
\Clue{43}{}{książka wydana w wydawnictwie Elzewirów, o charakterystycznym formacie i niskiej cenie}
\Clue{44}{}{pomocniczy, uporządkowany spis obiektów jakiejś kategorii (np. indeks autorów, indeks terminów), często z odsyłaczami do miejsc, w których znajduje się rozwinięcie lub omówienie tych obiektów}
\Clue{45}{}{miasto w Egipcie na płd. od Tanty}
\Clue{46}{}{statek towarowy odbywający nieregularne rejsy do różnych portów}
\Clue{47}{}{budynek gospodarczy przeznaczony do przechowywania zboża}
\Clue{48}{}{bezodpływowe słone jezioro w środkowej Australii}\end{PuzzleClues}

\begin{PuzzleClues}{\textbf{Pionowe}\\}\Clue{1}{}{pisarz mołdawski ur. 1912r, poezje, sztuki teatralne, satyra, krytyka literacka}
\Clue{2}{}{człowiek, którego się nie zna}
\Clue{3}{}{forma daniny publicznej płaconej jednorazowo przez właściciela nieruchomości na rzecz gminy}
\Clue{4}{}{pozaukładowa jednostka energii wynosząca 4,1868 J (dżuli)}
\Clue{5}{}{przyswajanie kultury bułgarskiej pod przymusem}
\Clue{6}{}{kolor czerwony, przypominający kolor naturalnych korali}
\Clue{7}{}{odezwa hetmanów, wojskowych, wojewodów}
\Clue{8}{}{pisarz w kancelarii królewskiej na Rusi}
\Clue{9}{}{w chemii: symbol cynku}
\Clue{10}{}{dział kosmonautyki dotyczący lotów na Księżyc}
\Clue{11}{}{instytucja przeznaczona do wspierania przedsiębiorczości przez udostępnianie lokali, środków trwałych i oferowanie szkoleń oraz pomocy w zdobyciu dofinansowania dla firm}
\Clue{13}{}{nauka zajmująca się chemią w organizmach żywych, a w szczególności biosyntezą, strukturą, stężeniem, funkcjami (w tym skutkami niedoboru oraz nadmiaru) i przemianami substancji chemicznych w organizmach (z uwzględnieniem aspektów energetycznych)}
\Clue{15}{}{budynek gospodarczy, w którym przechowuje się bydło rzeźne}
\Clue{16}{}{w kryptografii: dane dodawane podczas szyfrowania lub przygotowywania skrótu w celu powiększenia rozmiaru danych będących argumentem funkcji skrótu bez potrzeby rzeczywistego zwiększenia rozmiarów danych}
\Clue{17}{}{zespół zabiegów zmierzających do poprawienia założeń dziedzicznych (genotypu) zwierząt gospodarskich, w zakres których wchodzi ocena wartości użytkowej i hodowlanej zwierząt gospodarskich, selekcja i dobór osobników do kojarzenia prowadzony w warunkach prawidłowego chowu}
\Clue{18}{}{znak ustawiany przy wjeździe do miejscowości lub regionu na powitanie przyjeżdżającym osobom}
\Clue{19}{}{lek przeciwnowotworowy, inhibitor replikacji DNA; blokuje wytwarzanie białek i DNA przez komórki nowotworu poprzez degradację matrycy DNA, co uniemożliwia im wzrost i namnażanie się}
\Clue{20}{}{gaz musztardowy, bojowy środek użyty podczas I wojny światowej przez Niemców pod Ypres}
\Clue{21}{}{ciśnienie, którym należy działać na roztwór, aby powstrzymać przepływ rozpuszczalnika przez półprzepuszczalną membranę, która rozdziela roztwory o różnym stężeniu}
\Clue{22}{}{ciągniona sieć rybacka używana głównie na zamarzniętych jeziorach zakończona tzw. kutlem}
\Clue{23}{}{organiczny związek chemiczny z grupy nitryli stosowany głównie do syntezy poliakrylonitrylu}
\Clue{24}{}{obszar zbieżności dwóch systemów pasatowych}
\Clue{25}{}{ukraiński taniec ludowy podobny do krakowiaka}
\Clue{26}{}{(MALMÖ) miasto i port szwedzki nad Sundem, ośrodek administracyjny hrabstwa Malmöhus}
\Clue{27}{}{pokłady martwego łyka poprzedzielane korkiem i ewentualnie przekształconą fellodermą kolejno tworzonych peryderm}
\Clue{28}{}{rodzaj broszki, odznaka, najczęściej metalowa, noszona dla ozdoby albo dla zamanifestowania, ujawnienia przynależności do jakiejś organizacji lub grupy społecznej}
\Clue{29}{}{warstwa skóry lub tkaniny wyściełająca wnętrze obuwia}
\Clue{30}{}{włoski malarz, dekorator i architekt (1747-1820) dekoracje malarskie pałacu w Natolinie i zamku w Łańcucie}
\Clue{31}{}{BAKŁAŻAN}
\Clue{33}{}{język ognia; emitujący światło gaz, w którym zachodzą reakcje pirolizy, utleniania i spalania}
\Clue{34}{}{Tapirus terrestris - gatunek ssaka nieparzystokopytnego z rodziny tapirowatych; zamieszkuje tropikalne i subtropikalne rejony Ameryki Południowej, na wschód od Andów, od północnej Kolumbii do północnej Argentyny, liczny również w amazońskich lasach deszczowych, poza tym obszarem rozproszony}
\Clue{35}{}{wielościan o ścianach czworokątnych parami nierównych i nierównoległych}
\Clue{36}{}{informacje na jakiś temat, które mogą być do czegoś wykorzystane}
\Clue{37}{}{ptak z rzędu wróblowatych; Afryka, Australia, Japonia}
\Clue{38}{}{część opery stanowiąca pewną tematyczną całość}
\Clue{39}{}{przedstawiciel współczesnej subkultury, funkcjonującej zarówno wśród nastolatków, jak i ludzi po 20. i 30. roku życia i starszych; wyznacznikiem stylu życia hipsterów jest deklarowana niezależność wobec głównego nurtu kultury masowej (tzw.mainstreamu) i ironiczny stosunek do niego oraz przesadne akcentowanie swojej oryginalności i indywidualności}
\Clue{40}{}{Rudolf, ur. w 1816r. astronom szwajcarski, dyrektor obserwatorium w Berlinie i Zurychu; ustalił średnią długość cyklu aktywności słonecznej}
\Clue{42}{}{pasmo włosów, które się skręca}
\Clue{44}{}{eurazjatycki gatunek wierzby, w Polsce pospolita, srebrzyste bazie, pędy na obręcze do beczek}\end{PuzzleClues}\newpage%\section*{Krzyżówka 99}

\noindent\begin{Puzzle}{18}{25}|*	|*	|*	|*	|*	|*	|*	|*	|*	|[1][S]\drarr	|r	|a	|k	|*	|[2][S]\drarr	|t	|b	|*	|*	|.
|[3][S]\rarr	|g	|r	|u	|c	|z	|o	|ł	|[][,]{ }	|p	|r	|z	|ę	|d	|n	|y	|*	|*	|*	|.
|[4][S]\drarr	|s	|t	|a	|c	|j	|a	|[][,]{ }	|r	|o	|b	|o	|c	|z	|a	|*	|[5][S]\darr	|*	|*	|.
|o	|*	|*	|*	|[6][S]\darr	|*	|*	|[7][S]\darr	|[8][S]\darr	|s	|[9][S]\drarr	|e	|t	|e	|r	|*	|s	|*	|*	|.
|r	|*	|[10][S]\rarr	|t	|o	|r	|*	|u	|w	|k	|n	|*	|[11][S]\drarr	|p	|c	|*	|i	|[12][S]\darr	|*	|.
|d	|*	|*	|*	|ż	|*	|*	|l	|e	|o	|u	|*	|w	|*	|y	|*	|e	|w	|*	|.
|y	|*	|[13][S]\rarr	|k	|a	|r	|p	|*	|t	|k	|t	|*	|l	|*	|z	|[14][S]\darr	|c	|y	|*	|.
|n	|[15][S]\drarr	|g	|a	|r	|ś	|ć	|*	|e	|*	|a	|*	|e	|[16][S]\darr	|[][,]{ }	|u	|z	|ś	|*	|.
|a	|c	|[17][S]\rarr	|t	|e	|m	|p	|e	|r	|a	|*	|*	|w	|l	|n	|w	|k	|c	|*	|.
|t	|*	|[18][S]\darr	|*	|k	|*	|[19][S]\rarr	|m	|y	|l	|a	|r	|*	|o	|i	|a	|a	|i	|*	|.
|*	|*	|n	|*	|*	|[20][S]\rarr	|b	|o	|n	|i	|f	|r	|a	|t	|e	|r	|*	|g	|*	|.
|[21][S]\rarr	|p	|e	|r	|f	|o	|r	|m	|a	|n	|s	|*	|*	|k	|z	|u	|*	|[][,]{ }	|*	|.
|*	|[22][S]\drarr	|k	|w	|o	|k	|a	|*	|r	|*	|*	|[23][S]\rarr	|p	|a	|r	|n	|i	|k	|*	|.
|[24][S]\rarr	|p	|r	|z	|y	|z	|w	|o	|i	|t	|o	|ś	|ć	|*	|ó	|k	|*	|o	|*	|.
|*	|i	|o	|*	|[25][S]\drarr	|z	|a	|p	|a	|ś	|n	|i	|c	|t	|w	|o	|*	|n	|*	|.
|[26][S]\drarr	|s	|m	|o	|c	|z	|e	|k	|*	|*	|*	|*	|*	|*	|n	|w	|[27][S]\darr	|n	|*	|.
|a	|e	|a	|[28][S]\rarr	|z	|a	|k	|r	|ę	|t	|*	|*	|*	|*	|a	|a	|g	|y	|*	|.
|l	|m	|n	|[29][S]\rarr	|w	|ó	|z	|e	|k	|*	|*	|[30][S]\rarr	|m	|a	|n	|n	|a	|*	|*	|.
|d	|k	|c	|[31][S]\rarr	|o	|f	|i	|a	|k	|o	|d	|o	|n	|*	|y	|i	|l	|*	|*	|.
|e	|o	|j	|[32][S]\rarr	|r	|u	|f	|ó	|w	|k	|a	|*	|[33][S]\darr	|*	|*	|e	|i	|*	|*	|.
|h	|*	|a	|[34][S]\rarr	|o	|m	|e	|g	|a	|*	|[35][S]\drarr	|a	|b	|a	|t	|*	|a	|*	|*	|.
|y	|*	|*	|[36][S]\rarr	|n	|i	|k	|a	|r	|a	|g	|u	|a	|n	|k	|a	|*	|*	|*	|.
|d	|*	|[37][S]\rarr	|g	|ó	|r	|s	|k	|i	|*	|r	|[38][S]\rarr	|z	|a	|w	|ó	|j	|*	|*	|.
|*	|*	|*	|[39][S]\rarr	|g	|r	|u	|s	|z	|k	|a	|*	|a	|*	|*	|*	|*	|*	|*	|.
|[40][S]\rarr	|r	|y	|k	|*	|*	|*	|*	|*	|*	|d	|*	|*	|*	|*	|*	|*	|*	|*	|.
|*	|*	|*	|*	|*	|*	|*	|*	|*	|*	|*	|*	|*	|*	|*	|*	|*	|*	|*	|.\end{Puzzle}

\newpage

\begin{PuzzleClues}{\textbf{Poziome}\\}\Clue{1}{}{nowotwór złośliwy (zazwyczaj powiemy tak o chorobie wywodzącej się z tkanki nabłonkowej)}
\Clue{2}{}{w chemii: symbol terbu}
\Clue{3}{}{występujący u wielu stawonogów narząd wydzielający płynną, lepką substancję, krzepnącą pod wpływem enzymów w zetknięciu z powietrzem do postaci jedwabistej nici przędnej}
\Clue{4}{}{wysokiej klasy komputer o wyższej wydajności niż komputer osobisty, szczególnie ze względu na możliwość przetwarzania grafiki komputerowej, moc obliczeniową i wielowątkowość}
\Clue{9}{}{domyślne określenie eteru dietylowego}
\Clue{10}{}{wytyczona trasa, po której należy się poruszać}
\Clue{11}{}{komputer osobisty, ang. personal computer}
\Clue{13}{}{najważniejsza hodowlana ryba słodkowodna o długości do 1 m}
\Clue{15}{}{tyle, ile się mieści w garści (np. garść zboża)}
\Clue{17}{}{odmiana malarstwa niegeometrycznego, polega na tworzeniu ze swobodnych palm barwnych uzyskiwanych np. przez rozlewanie bądź rozpryskiwanie farb}
\Clue{19}{}{folia stosowana jako izolacja elektryczna i podkład taśm magnetycznych}
\Clue{20}{}{członek Zakonu Bonifratrów}
\Clue{21}{}{pojęcie z zakresu performatyki: zachowanie w określonym typie sytuacji - w pewnym sensie rytualne, które jestodgrywane z towarzyszeniem różnych sposobów kodowania kulturowego}
\Clue{22}{}{kura domowa, która wysiaduje jajka i ma młode}
\Clue{23}{}{kocioł, w którym poddaje się różne produkty parowaniu pod ciśnieniem}
\Clue{24}{}{to, że coś jest odpowiednie, w należytej ilości}
\Clue{25}{}{zapasy - sport walki, polegający na fizycznym zmaganiu dwóch zawodników, których walka odbywa się wręcz przez stosowanie chwytów; początki tego sportu sięgają czasów starożytnych}
\Clue{26}{}{okrągły narząd gębowy minoga}
\Clue{28}{}{świr, szalona głowa, osoba, o której można powiedzieć, że jest zakręcona (zazwyczaj: na jakimś punkcie)}
\Clue{29}{}{w maszynach do pisania podzespołów się poziomo po prowadnicy, wyposażony w wałek i inne urządzenia}
\Clue{30}{}{mannianka, pospolita, wieloletnia trawa, dawniej używana na kaszę}
\Clue{31}{}{Ophiacodon - rodzaj drapieżnego pelykozaura, żyjącego w późnym karbonie i wczesnym permie, najlepiej poznany przedstawiciel rodziny Ophiacodontidae z powodu powszechnego występowania jego skamieniałości na terenie obecnej Ameryki Północnej; jego szczątki odkryto między innymi w kanadyjskiej Nowej Szkocji oraz stanach Ohio i Teksas}
\Clue{32}{}{nadbudówka na rufie}
\Clue{34}{}{jasna mgławica gazowa na pograniczu gwiazdozbiorów Strzelca, Tarczy i Węża}
\Clue{35}{}{przełożony klasztoru, Opat}
\Clue{36}{}{mieszkanka Nikaragui, kobieta pochodzenia nikaraguańskiego}
\Clue{37}{}{krytyk literacki, pisarz (1870-1959); „Monsalwat”, Rzecz o Adamie Mickiewiczu” - autor manifestu Młoda Polska}
\Clue{38}{}{TURBAN}
\Clue{39}{}{popularny owoc o kulistym, zwężającym się ku górze kształcie; owoc (wielopestkowiec) drzewa nazywanego tak samo}
\Clue{40}{}{donośny gardłowy głos zwierzęcia}\end{PuzzleClues}

\begin{PuzzleClues}{\textbf{Pionowe}\\}\Clue{1}{}{w uzwojeniach maszyn elektrycznych odległość roboczych boków tego samego zezwoju mierzona liczbą żłobków, np. wirnika albo wycinków komutatora}
\Clue{2}{}{Narcissus incomparabilis - gatunek roślin należący do rodziny amarylkowatych}
\Clue{4}{}{osoba, w ręce której został powierzony majątek do rodzielenia w ramach spadku na mocy fideikomisu (ordynacji rodowej)}
\Clue{5}{}{małe paciorki, które mają formę pociętej słomki (stąd ich nazwa)}
\Clue{6}{}{ŻACHWA; morskie zwierzę z osłonic}
\Clue{7}{}{w rzemiośle pszczelarskim, konstrukcja, najczęściej drewniana, używana do hodowli pszczół}
\Clue{8}{}{kierunek studiów, na którym kształci się weterynarzy}
\Clue{9}{}{potocznie: melodia}
\Clue{11}{}{wprowadzanie, w celach leczniczych, płynu do organizmu}
\Clue{12}{}{gonitwa na koniach, pojedynczy wyścig, podczas którego rywalizują ze sobą pary jeździeckie}
\Clue{14}{}{warunek, składnik mający na coś wpływ}
\Clue{15}{}{litera alfabetu oznaczająca w numeracji porządkowej: trzeci}
\Clue{16}{}{ruchoma część skrzydła samolotu stanowiąca ster poprzeczny}
\Clue{18}{}{forma praktyk magicznych, w której czarujący (nekromanta) przyzywa cienie zmarłych}
\Clue{22}{}{czasopismo dla dzieci}
\Clue{25}{}{zwierzę domowe mające cztery nogi, zwłaszcza pies lub kot}
\Clue{26}{}{związek organiczny posiadający grupę aldehydową, połączoną z jednym lub dwoma atomami wodoru}
\Clue{27}{}{kraina historyczna w Europie Zachodniej, obecnie tereny Francji, Belgii, Szwajcarii i północno-zachodnich Włoch, zamieszkana przez plemiona celtyckie}
\Clue{33}{}{zasoby i urządzenia jednej gałęzi gospodarki będące podstawą rozwoju innych gałęzi}
\Clue{35}{}{substancja, jaka spada podczas opadów gradu}\end{PuzzleClues}\newpage%\section*{Krzyżówka 101}

\noindent\begin{Puzzle}{24}{29}|*	|*	|*	|*	|*	|*	|*	|*	|*	|*	|*	|*	|*	|*	|*	|*	|*	|*	|[1][S]\drarr	|c	|e	|w	|a	|*	|*	|.
|*	|*	|*	|*	|*	|*	|[2][S]\darr	|*	|*	|[3][S]\rarr	|k	|o	|n	|g	|r	|e	|g	|a	|c	|j	|a	|*	|[4][S]\darr	|[5][S]\darr	|*	|.
|*	|*	|[6][S]\rarr	|s	|i	|t	|n	|i	|k	|[][,]{ }	|p	|ł	|y	|w	|a	|j	|ą	|c	|y	|*	|*	|*	|s	|r	|*	|.
|[7][S]\rarr	|z	|l	|e	|w	|n	|i	|a	|*	|[8][S]\drarr	|p	|o	|d	|k	|a	|s	|t	|i	|n	|g	|*	|*	|z	|o	|*	|.
|*	|*	|[9][S]\rarr	|d	|r	|z	|e	|w	|o	|r	|y	|t	|n	|i	|a	|*	|[10][S]\rarr	|w	|a	|r	|k	|o	|c	|z	|*	|.
|*	|*	|[11][S]\darr	|*	|*	|*	|d	|[12][S]\rarr	|r	|o	|s	|t	|r	|a	|*	|*	|*	|[13][S]\darr	|m	|*	|[14][S]\darr	|*	|z	|ł	|[15][S]\darr	|.
|*	|*	|p	|*	|*	|*	|ź	|[16][S]\rarr	|i	|t	|a	|l	|i	|a	|*	|[17][S]\darr	|*	|b	|o	|*	|b	|*	|u	|ą	|b	|.
|[18][S]\drarr	|p	|r	|z	|y	|z	|w	|o	|i	|t	|o	|ś	|ć	|*	|*	|k	|[19][S]\darr	|o	|n	|[20][S]\darr	|u	|[21][S]\darr	|p	|k	|u	|.
|m	|*	|z	|*	|*	|[22][S]\darr	|i	|*	|*	|w	|[23][S]\darr	|*	|[24][S]\darr	|[25][S]\darr	|*	|r	|s	|w	|[][,]{ }	|k	|t	|d	|ł	|a	|k	|.
|a	|*	|e	|*	|*	|m	|e	|*	|*	|e	|d	|*	|s	|p	|[26][S]\darr	|a	|m	|e	|m	|a	|[][,]{ }	|r	|o	|*	|s	|.
|l	|*	|ś	|*	|[27][S]\darr	|o	|d	|[28][S]\drarr	|n	|i	|e	|s	|t	|a	|c	|j	|o	|n	|a	|r	|n	|o	|ś	|ć	|*	|.
|u	|[29][S]\darr	|c	|*	|p	|c	|z	|o	|*	|l	|f	|*	|a	|n	|z	|e	|ł	|*	|g	|ł	|a	|m	|ć	|[30][S]\darr	|*	|.
|c	|e	|i	|*	|o	|z	|i	|s	|*	|e	|i	|*	|l	|[][,]{ }	|a	|[][,]{ }	|a	|*	|e	|o	|r	|o	|*	|d	|[31][S]\darr	|.
|h	|k	|e	|[32][S]\darr	|e	|a	|ó	|t	|*	|r	|c	|[33][S]\darr	|l	|d	|j	|r	|[][,]{ }	|*	|l	|w	|c	|m	|*	|z	|k	|.
|*	|s	|r	|z	|m	|r	|w	|r	|[34][S]\darr	|k	|y	|j	|a	|o	|k	|o	|w	|*	|l	|a	|i	|e	|[35][S]\darr	|i	|ł	|.
|[36][S]\drarr	|p	|a	|ł	|a	|n	|k	|o	|w	|a	|t	|e	|*	|m	|a	|z	|ę	|*	|a	|t	|a	|r	|n	|e	|ę	|.
|j	|l	|d	|o	|t	|i	|a	|w	|e	|*	|*	|z	|[37][S]\darr	|u	|*	|w	|g	|*	|ń	|o	|r	|o	|u	|w	|b	|.
|a	|o	|ł	|t	|[][,]{ }	|k	|[][,]{ }	|c	|r	|*	|*	|u	|ś	|*	|*	|i	|l	|*	|s	|ś	|s	|n	|r	|i	|e	|.
|m	|a	|o	|a	|o	|[][,]{ }	|p	|z	|n	|*	|*	|i	|w	|*	|*	|j	|o	|*	|k	|ć	|k	|*	|z	|a	|k	|.
|a	|t	|[][,]{ }	|[][,]{ }	|p	|w	|u	|a	|i	|*	|*	|t	|i	|*	|*	|a	|w	|*	|i	|*	|i	|*	|a	|r	|[][,]{ }	|.
|[][,]{ }	|a	|k	|r	|i	|y	|r	|n	|s	|[38][S]\rarr	|k	|a	|r	|a	|z	|j	|a	|*	|*	|*	|*	|*	|n	|k	|n	|.
|p	|t	|ą	|e	|s	|g	|p	|i	|a	|*	|*	|*	|*	|*	|*	|ą	|*	|*	|*	|*	|*	|*	|i	|a	|e	|.
|a	|o	|p	|n	|o	|i	|u	|n	|ż	|[39][S]\rarr	|s	|t	|r	|z	|e	|c	|h	|w	|o	|w	|a	|t	|e	|*	|r	|.
|c	|r	|i	|e	|w	|ę	|r	|*	|*	|*	|*	|*	|*	|[40][S]\rarr	|n	|e	|r	|k	|o	|w	|i	|e	|c	|*	|w	|.
|h	|k	|e	|t	|y	|t	|k	|[41][S]\rarr	|c	|i	|a	|m	|c	|i	|a	|[][,]{ }	|l	|a	|m	|c	|i	|a	|*	|[42][S]\darr	|ó	|.
|o	|a	|l	|a	|*	|y	|a	|*	|[43][S]\rarr	|ż	|ą	|d	|z	|a	|[][,]{ }	|s	|u	|k	|c	|e	|s	|u	|*	|c	|w	|.
|w	|*	|o	|*	|*	|*	|*	|*	|[44][S]\rarr	|s	|z	|p	|a	|t	|[][,]{ }	|i	|s	|l	|a	|n	|d	|z	|k	|i	|*	|.
|a	|[45][S]\rarr	|w	|o	|d	|o	|w	|s	|t	|r	|ę	|t	|*	|[46][S]\rarr	|b	|ę	|b	|e	|n	|*	|*	|*	|*	|o	|*	|.
|*	|*	|e	|[47][S]\rarr	|a	|s	|t	|r	|o	|m	|e	|t	|r	|i	|a	|*	|*	|*	|[48][S]\rarr	|t	|o	|r	|o	|s	|*	|.
|*	|*	|*	|*	|*	|*	|*	|*	|*	|*	|*	|*	|*	|*	|*	|*	|*	|*	|*	|*	|*	|*	|*	|*	|*	|.\end{Puzzle}

\newpage

\begin{PuzzleClues}{\textbf{Poziome}\\}\Clue{1}{}{część składowa maszyny wyciągowej; bęben, na którym zwoje liny nawijają się na siebie}
\Clue{3}{}{zjazd duchownych}
\Clue{6}{}{Isolepis fluitans, Scirpus fluitans - rodzaj roślin z rodziny ciborowatych}
\Clue{7}{}{miejsce skupu mleka}
\Clue{8}{}{forma internetowej publikacji dźwiękowej lub filmowej, najczęściej w postaci regularnych odcinków, z zastosowaniem technologii RSS}
\Clue{9}{}{pracownia, w której tworzone są drzeworyty}
\Clue{10}{}{uczesanie z trzech (lub więcej) pasm długich, splecionych włosów, zabezpieczone przed rozplątaniem}
\Clue{12}{}{ozdoba architektoniczna w kształcie dzioba starożytnego okrętu}
\Clue{16}{}{inna nazwa Włoch}
\Clue{18}{}{cecha człowieka: porządność, brak wulgarności; to, że ktoś zachowuje się kulturalnie i obyczajnie}
\Clue{28}{}{cecha czegoś, co robi się zaocznie, nie na miejscu}
\Clue{36}{}{oposy australijskie, Phalangeridae - rodzina torbaczy z rzędu dwuprzodozębowców; obejmuje gatunki zamieszkujące środowiska leśne w Australii, Nowej Gwinei i kilku mniejszych wysp Indonezji}
\Clue{38}{}{sukmana krakowska; KIEREZJA}
\Clue{39}{}{Grimmiaceae - rodzina mchów z rzędu strzechwowców}
\Clue{40}{}{Dioctophyma renale - pasożyt zwierzęcy, nicień, który jest częstą przyczyną zapalenia miedniczek nerkowych u psów (rzadziej u innych zwierząt i ludzi)}
\Clue{41}{}{ktoś bardzo niezdarny, słaby}
\Clue{43}{}{intensywna chęć osiągnięcia powodzenia, bez względu na koszty i warunki}
\Clue{44}{}{tradycyjna nazwa minerału stanowiącego chemicznie czystą, przezroczystą wielokrystaliczną odmianę kalcytu, wykształconą w postaci romboedrów}
\Clue{45}{}{wstręt do wody}
\Clue{46}{}{część maszyny w kształcie obrotowego walca}
\Clue{47}{}{najstarszy dział astronomii, obejmuje metody określania położenia i ruchów ciał niebieskich}
\Clue{48}{}{zwał lodu powstały wskutek ściśnięcia się pokrywy lodowej, może mieć wysokość nawet 10 metrów}\end{PuzzleClues}

\begin{PuzzleClues}{\textbf{Pionowe}\\}\Clue{1}{}{kora zacierpu Wintera stosowana jako przyprawa}
\Clue{2}{}{Rhyparia purpurata - gatunek motyla z rodziny niedźwiedziówkowatych; występuje w Europie i Azji}
\Clue{4}{}{niewielka ilość}
\Clue{5}{}{rozstanie, oddalenie od kogoś, przerwanie znajomości z kimś}
\Clue{8}{}{suka rasy rottweiler}
\Clue{11}{}{rodzaj dużego ręcznika, używanego po kąpieli, w saunie, na plaży lub w podobnych okolicznościach, służący do okrycia całego ciała}
\Clue{13}{}{(1899-1973), pisarka angielska, powieści psychologiczne i nowele z życia mieszczaństwa; „Dziewczynki”}
\Clue{14}{}{but, za pomocą którego można przymocować nartę do nogi w sposób umożliwiający jazdę lub bieg na nartach}
\Clue{15}{}{podparcie końcowych fragmentów osi kół wozu w postaci walca z metalu}
\Clue{17}{}{kraje Ameryki Południowej, Afryki, Azji, Oceanii o niskim poziomie dóbr materialnych}
\Clue{18}{}{dziecko, brzdąc}
\Clue{19}{}{czarna, mazista ciecz, powstająca jako efekt uboczny odgazowywania węgla kamiennego, która po poddaniu obróbce mechanicznej i chemicznej (zwłaszcza destylacji), znajduje zastosowanie w budownictwie, kosmetologii i farmacji}
\Clue{20}{}{w przenośni: cecha czegoś, co nie posiada wielkiej wartości, jest marne i nijakie}
\Clue{21}{}{Dromomeron - rodzaj archozaura z kladu Dinosauromorpha żyjącego w noryku, jednen z najprymitywniejszych przedstawicieli tego kladu; znany z nielicznych skamieniałości odkrytych na stanowisku Hayden Quarry, na terenie Ghost Ranch w Nowym Meksyku, a być może pewne skamieniałości z Arizony i Teksas również należą do tego zwierzęcia}
\Clue{22}{}{Hygrohypnum eugyrium - gatunek mchu należący do rodziny krzywoszyjowatych}
\Clue{23}{}{brak, niedobór, zbyt mała ilość czegoś, która nie może sprostać potrzebom}
\Clue{24}{}{drewniana lub kamienna ławka ustawiona w prezbiterium; zgodnie z normą tylko w liczbie mnogiej}
\Clue{25}{}{mężczyzna stojący na czele domu, rodziny, gospodarstwa}
\Clue{26}{}{chroniony ptak błotno-łąkowy z rzędu mew siewek, na głowie czubek}
\Clue{27}{}{utwór wierszowany, którego podstawową formą podawczą jest opis}
\Clue{28}{}{mieszkaniec Ostrowca Świętokrzyskiego}
\Clue{29}{}{kobieta użytkująca bogactwa naturalne}
\Clue{30}{}{maszyna włókiennicza}
\Clue{31}{}{osoba znerwicowana, niekontrolująca swoich nerwów}
\Clue{32}{}{złota reneta, angielska odmiana jabłek uprawiana od XVII w.; daje owoce duże lub średniej wielkości, o cienkiej, złotożółtej skórce z czerwonym zabarwieniem}
\Clue{33}{}{zakonnik z zakonu założonego przez Ignacego Loyolę}
\Clue{34}{}{francuski malarz i grafik (1868-1940) członek grupy nabistów, portrety, martwe natury, malowidła ścienne, wyroby rzemiosła}
\Clue{35}{}{Vallisneria - rodzaj roślin wodnych z rodziny żabiściekowatych}
\Clue{36}{}{jama znajdująca się u nasady kończyny górnej człowieka}
\Clue{37}{}{wariat, osoba zachowująca się z dozą szaleństwa}
\Clue{42}{}{blok kamienny w kształcie graniastosłupa, zwykle taki, którego przynajmniej jeden bok jest ozdobny}\end{PuzzleClues}\newpage%\section*{Krzyżówka 102}

\noindent\begin{Puzzle}{23}{25}|*	|*	|[1][S]\drarr	|p	|i	|e	|r	|w	|o	|t	|n	|i	|a	|k	|*	|*	|*	|[2][S]\drarr	|m	|a	|t	|e	|*	|*	|.
|*	|[3][S]\rarr	|p	|r	|e	|f	|e	|k	|t	|*	|[4][S]\rarr	|j	|e	|d	|e	|n	|a	|s	|t	|k	|a	|*	|*	|*	|.
|[5][S]\rarr	|t	|r	|z	|e	|c	|i	|a	|[][,]{ }	|n	|e	|r	|k	|a	|*	|[6][S]\rarr	|p	|o	|n	|u	|r	|*	|*	|*	|.
|[7][S]\drarr	|i	|z	|b	|a	|[][,]{ }	|a	|d	|w	|o	|k	|a	|c	|k	|a	|*	|[8][S]\rarr	|l	|i	|d	|a	|r	|*	|*	|.
|ć	|*	|y	|[9][S]\darr	|*	|*	|*	|*	|[10][S]\rarr	|f	|u	|t	|u	|r	|e	|[][,]{ }	|s	|i	|m	|p	|l	|e	|*	|*	|.
|m	|[11][S]\rarr	|w	|s	|z	|e	|t	|e	|c	|z	|n	|o	|ś	|ć	|*	|*	|[12][S]\rarr	|d	|r	|o	|p	|s	|*	|*	|.
|y	|[13][S]\darr	|r	|t	|*	|[14][S]\rarr	|k	|i	|e	|ł	|ż	|[][,]{ }	|z	|w	|y	|c	|z	|a	|j	|n	|y	|*	|*	|*	|.
|*	|k	|o	|a	|[15][S]\rarr	|w	|s	|z	|e	|c	|h	|b	|o	|h	|a	|t	|e	|r	|s	|t	|w	|o	|*	|*	|.
|[16][S]\drarr	|a	|t	|r	|a	|k	|t	|a	|n	|t	|*	|*	|*	|[17][S]\darr	|*	|*	|*	|n	|[18][S]\darr	|[19][S]\darr	|*	|[20][S]\darr	|*	|*	|.
|k	|z	|n	|a	|*	|*	|[21][S]\darr	|*	|*	|[22][S]\drarr	|p	|a	|*	|l	|[23][S]\darr	|*	|[24][S]\darr	|o	|k	|c	|*	|b	|*	|*	|.
|o	|a	|i	|*	|*	|*	|b	|*	|[25][S]\darr	|k	|*	|[26][S]\darr	|*	|e	|t	|*	|e	|ś	|i	|i	|*	|e	|[27][S]\darr	|*	|.
|z	|r	|k	|*	|*	|*	|o	|*	|h	|u	|[28][S]\darr	|b	|[29][S]\darr	|m	|o	|[30][S]\darr	|n	|ć	|n	|e	|[31][S]\darr	|ł	|o	|*	|.
|a	|k	|[][,]{ }	|*	|*	|*	|k	|*	|a	|c	|a	|y	|k	|u	|r	|z	|c	|[][,]{ }	|e	|n	|p	|t	|g	|*	|.
|*	|a	|b	|*	|[32][S]\drarr	|ż	|ó	|ł	|w	|[][,]{ }	|f	|l	|o	|r	|y	|d	|y	|j	|s	|k	|i	|*	|o	|*	|.
|*	|[][,]{ }	|o	|*	|k	|*	|w	|*	|a	|p	|e	|i	|r	|[][,]{ }	|z	|e	|k	|a	|k	|i	|e	|[33][S]\darr	|r	|*	|.
|*	|s	|g	|*	|l	|*	|k	|*	|j	|o	|r	|c	|s	|w	|m	|n	|l	|j	|o	|[][,]{ }	|s	|b	|z	|*	|.
|*	|z	|u	|*	|u	|*	|a	|*	|e	|t	|k	|a	|a	|a	|*	|e	|o	|n	|p	|p	|*	|l	|e	|*	|.
|*	|a	|m	|*	|c	|*	|*	|*	|*	|t	|a	|[][,]{ }	|r	|r	|[34][S]\darr	|r	|p	|i	|*	|a	|*	|o	|l	|*	|.
|*	|r	|i	|[35][S]\drarr	|z	|i	|o	|b	|r	|o	|*	|s	|z	|i	|g	|w	|e	|k	|*	|l	|*	|k	|i	|*	|.
|*	|o	|ł	|z	|*	|[36][S]\rarr	|d	|y	|s	|k	|*	|z	|*	|*	|a	|o	|d	|ó	|[37][S]\drarr	|u	|r	|a	|n	|*	|.
|[38][S]\drarr	|g	|a	|r	|d	|z	|i	|e	|l	|*	|*	|t	|*	|*	|t	|w	|y	|w	|c	|s	|*	|d	|a	|*	|.
|z	|ł	|*	|y	|[39][S]\rarr	|z	|a	|s	|t	|r	|z	|y	|k	|*	|k	|a	|z	|*	|n	|z	|*	|a	|*	|*	|.
|w	|o	|*	|w	|*	|*	|[40][S]\rarr	|z	|a	|b	|a	|w	|a	|*	|i	|n	|m	|*	|o	|e	|*	|*	|*	|*	|.
|ó	|w	|*	|k	|*	|*	|*	|[41][S]\rarr	|u	|s	|t	|n	|i	|k	|*	|i	|*	|*	|t	|k	|*	|*	|*	|*	|.
|d	|a	|*	|a	|[42][S]\rarr	|r	|y	|n	|n	|i	|c	|a	|*	|[43][S]\rarr	|d	|e	|b	|r	|a	|*	|*	|*	|*	|*	|.
|*	|*	|*	|*	|[44][S]\rarr	|h	|u	|s	|a	|r	|z	|*	|*	|*	|*	|*	|*	|*	|*	|*	|*	|*	|*	|*	|.\end{Puzzle}

\newpage

\begin{PuzzleClues}{\textbf{Poziome}\\}\Clue{1}{}{drobny jednokomórkowy organizm eukariotyczny, przedstawiciel protistów zwierzęcych}
\Clue{2}{}{naczynie do picia yerba mate}
\Clue{3}{}{wysoki rangą katecheta}
\Clue{4}{}{całość, która składa się z jedenastu elementów}
\Clue{5}{}{bardzo rzadka wada rozwojowa polegająca na występowaniu trzeciej nerki, którą należy chirurgicznie usunąć}
\Clue{6}{}{podwodna część budowli wodnych}
\Clue{7}{}{ogół adwokatów i aplikantów adwokackich, mających siedzibę na terenie izby, której zasięg terytorialny określa Naczelna Rada Adwokacka, biorąc pod uwagę głównie podział terytorialny administracji sądowej}
\Clue{8}{}{urządzenie optyczne stosowane do badania chmur, mgły itd}
\Clue{10}{}{czas w języku angielskim wyrażający spontanicznie podjęte decyzje i przewidywane wydarzenia lub zamiary}
\Clue{11}{}{cecha tego, co jest wszeteczne - nieobyczajne, rozwiązłe}
\Clue{12}{}{twardy, drobny cukierek o różnym smaku, pakowany zwykle w ruloniki}
\Clue{14}{}{Gammarus inaequicauda - morski gatunek skorupiaka z rzędu obunogów}
\Clue{15}{}{ogromne bohaterstwo, odwaga w stawianiu czołą każdej możliwej sytuacji}
\Clue{16}{}{czynnik wabiący - dowolnej natury czynnik (dźwięk, światło, związki chemiczne, kształty lub kombinacje takich sygnałów) o działaniu wabiącym, zwiększający atrakcyjność seksualną, przyciągający gatunki zapylające kwiaty, rozsiewające nasiona, itp}
\Clue{22}{}{w chemii: symbol protaktynu}
\Clue{32}{}{Pseudemys floridana - gatunek gada z rodziny żółwi błotnych}
\Clue{35}{}{pilot szybowcowy, wicemistrz świata w klasie otwartej z 1976 r}
\Clue{36}{}{sprzęt lekkoatletyczny używany podczas rzutu dyskiem}
\Clue{37}{}{siódma wg oddalenia od Słońca planeta}
\Clue{38}{}{w anatomii: trójkątna przestrzeń zawarta pomiędzy łukami podniebienno-językowymi a podniebienno-gardłowymi}
\Clue{39}{}{wprowadzenie roztworu (najczęściej leku, lecz także - narkotyku) do tkanki za pomocą strzykawki z igłą}
\Clue{40}{}{spotkanie towarzyskie, na którym główną aktywnością jest taniec}
\Clue{41}{}{część papierosa, fajki, którą przykłada się lub częściowo wkłada do ust}
\Clue{42}{}{mikroskopijnej; wielkości chrząszcz z rodziny stonek}
\Clue{43}{}{forma ukształtowania powierzchni ziemi: wąska, płytka, V-kształtna dolina sucha o znacznym i niewyrównanym spadku}
\Clue{44}{}{żołnierz ciężkozbrojnej jazdy polskiej (II połowa XVI w)}\end{PuzzleClues}

\begin{PuzzleClues}{\textbf{Pionowe}\\}\Clue{1}{}{Alchemilla bogumili - gatunek rośliny z rodziny różowatych}
\Clue{2}{}{zgodność kobiet oraz wzajemne stawanie po swojej stronie podczas dyskutowania o mężczyznach lub spierania się z poglądami mężczyzn}
\Clue{7}{}{motyle nocne}
\Clue{9}{}{familiarna forma używana przy zwracaniu się do dobrej znajomej}
\Clue{13}{}{Tadorna cana - gatunek ptaka z rodziny kaczkowatych (Anatidae)}
\Clue{16}{}{udomowione zwierzę pochodzące od kozy bezoarowej i markura, hodowana dla mleka, wełny, mięsa}
\Clue{17}{}{Varecia variegata - gatunek małpiatki z rodziny lemurowatych, której jest największym przedstawicielem; prawdopodobnie jest to jedyny gatunek naczelnych budujący gniazda wyłącznie na czas porodu i pierwszych tygodni opieki nad młodymi; zamieszkuje wschodni Madagaskar, żyjąc na terenach zalesionych, od brzegu morza do wysokości 1 350 m n.p.m}
\Clue{18}{}{rodzaj lampy obrazowej wykorzystującej magnetyczne odchylanie elektronów}
\Clue{19}{}{bateria w kształcie walca o długości ok. 44,5 mm, średnicy ok. 10,5 mm i wadze ok. 11,5 grama}
\Clue{20}{}{tanie wino, często owocowe}
\Clue{21}{}{Pleurocybella - rodzaj grzybów należący do rodziny twardzioszkowatych; w Polsce występuje jeden tylko gatunek}
\Clue{22}{}{kuc baskijski, pottok - rasa konia z grupy kuców; dziś nadal żyje w stanie dzikim w Pirenejach, regionie częściowo należącym do Francji, a częściowo do Hiszpanii}
\Clue{23}{}{konserwatyzm brytyjski; poglądy przedstawicieli stronnictwa torysów}
\Clue{24}{}{metoda nauczania, w której największe znaczenie ma wiedza encyklopedyczna}
\Clue{25}{}{najmłodszy z 50. stanów USA (dołączony 21 sierpnia 1959 roku)}
\Clue{26}{}{Artemisia rigida, Artemisia trifida - gatunek roślin z rodziny astrowatych}
\Clue{27}{}{ogorzałość, opalenizna}
\Clue{28}{}{drobne przestępstwo, często dotyczące wyższych sfer lub osób zaufania publicznego, które ma potencjał medialny; niewielka afera}
\Clue{29}{}{uzbrojony, prywatny statek handlowy upoważniony do prowadzenia wojny morskiej}
\Clue{30}{}{stres, dynamiczna relacja adaptacyjna pomiędzy możliwościami jednostki, a wymogami sytuacji, charakteryzująca się brakiem równowagi psychicznej i fizycznej; zaburzenie homeostazy spowodowane czynnikiem fizycznym lub psychologicznym}
\Clue{31}{}{człowiek, który za czymś przepada, łasy na coś, taki, który chętnie by się na coś rzucił w każdej chwili}
\Clue{32}{}{przedmiot służący do otwierania zamków lub kłódek}
\Clue{33}{}{automatyczne zabezpieczenie urządzenia od skutków niepożądanych procesów lub manipulacji przy nich}
\Clue{34}{}{zdrobniale o gaciach, majtkach}
\Clue{35}{}{reklamówka, worek foliowy, najczęściej z szeleszczącej, prześwitującej folii, nazywany tak od sposobu, w jaki odczepia się go z całego pliku worków - poprzez zerwanie}
\Clue{37}{}{zaleta, zwłaszcza człowieka, czasem też zwierzęcia}
\Clue{38}{}{taktyka oszukiwania kogoś, zwodzenie}\end{PuzzleClues}\newpage%\section*{Krzyżówka 103}

\noindent\begin{Puzzle}{23}{30}|*	|*	|*	|*	|[1][S]\drarr	|m	|i	|ę	|k	|i	|s	|z	|[][,]{ }	|p	|a	|l	|i	|s	|a	|d	|o	|w	|y	|*	|.
|*	|*	|*	|[2][S]\darr	|p	|*	|*	|*	|*	|*	|*	|*	|[3][S]\drarr	|p	|r	|z	|e	|s	|i	|ą	|k	|r	|a	|*	|.
|*	|[4][S]\darr	|*	|n	|r	|*	|[5][S]\drarr	|r	|ó	|ż	|d	|ż	|k	|a	|*	|*	|*	|*	|*	|*	|*	|*	|*	|*	|.
|*	|k	|[6][S]\darr	|u	|z	|*	|m	|*	|*	|[7][S]\rarr	|d	|m	|u	|c	|h	|*	|*	|[8][S]\drarr	|n	|u	|t	|a	|*	|*	|.
|[9][S]\drarr	|r	|o	|m	|e	|r	|o	|*	|[10][S]\rarr	|ż	|u	|b	|r	|*	|[11][S]\rarr	|s	|t	|o	|p	|a	|*	|*	|*	|[12][S]\darr	|.
|o	|a	|b	|e	|t	|*	|n	|*	|[13][S]\darr	|[14][S]\rarr	|r	|a	|k	|ó	|w	|*	|[15][S]\rarr	|s	|z	|o	|p	|*	|*	|t	|.
|c	|w	|i	|r	|a	|[16][S]\rarr	|i	|n	|s	|t	|y	|t	|u	|c	|j	|a	|*	|t	|*	|*	|*	|*	|*	|a	|.
|h	|a	|e	|[][,]{ }	|r	|*	|t	|[17][S]\drarr	|a	|p	|t	|a	|m	|e	|r	|*	|[18][S]\rarr	|a	|u	|d	|*	|*	|[19][S]\darr	|r	|.
|r	|t	|k	|b	|g	|*	|o	|t	|ł	|*	|*	|*	|a	|[20][S]\rarr	|d	|a	|r	|n	|i	|a	|k	|*	|s	|a	|.
|o	|k	|t	|u	|[][,]{ }	|*	|r	|w	|o	|*	|*	|[21][S]\darr	|[][,]{ }	|[22][S]\rarr	|m	|e	|k	|i	|n	|t	|o	|s	|z	|*	|.
|n	|a	|[][,]{ }	|r	|n	|*	|[][,]{ }	|i	|*	|*	|[23][S]\drarr	|z	|w	|o	|l	|n	|i	|e	|n	|i	|e	|*	|c	|*	|.
|a	|*	|f	|t	|i	|*	|h	|e	|*	|[24][S]\darr	|m	|e	|ą	|[25][S]\rarr	|g	|u	|c	|c	|i	|*	|[26][S]\darr	|*	|z	|*	|.
|[][,]{ }	|*	|o	|o	|e	|[27][S]\darr	|o	|r	|*	|l	|o	|f	|s	|[28][S]\darr	|[29][S]\drarr	|d	|*	|[][,]{ }	|*	|*	|p	|*	|ą	|*	|.
|ś	|*	|r	|w	|o	|p	|l	|d	|*	|i	|s	|i	|k	|g	|m	|*	|[30][S]\rarr	|d	|o	|h	|a	|*	|t	|*	|.
|r	|*	|t	|y	|g	|o	|t	|z	|[31][S]\drarr	|s	|t	|r	|o	|n	|a	|*	|[32][S]\darr	|e	|*	|[33][S]\darr	|j	|[34][S]\darr	|k	|*	|.
|o	|*	|y	|*	|r	|d	|e	|e	|k	|t	|*	|*	|l	|i	|s	|[35][S]\darr	|d	|f	|[36][S]\darr	|p	|a	|k	|i	|*	|.
|d	|*	|f	|*	|a	|r	|r	|n	|a	|[][,]{ }	|[37][S]\drarr	|p	|i	|e	|k	|i	|e	|l	|n	|i	|c	|a	|*	|*	|.
|o	|*	|i	|*	|n	|ó	|o	|i	|r	|k	|a	|[38][S]\darr	|s	|w	|a	|n	|m	|a	|a	|e	|*	|s	|*	|*	|.
|w	|*	|k	|[39][S]\darr	|i	|ż	|w	|e	|l	|r	|r	|j	|t	|o	|*	|f	|o	|c	|c	|g	|*	|p	|*	|*	|.
|i	|[40][S]\drarr	|a	|r	|c	|u	|s	|[][,]{ }	|s	|e	|c	|a	|n	|s	|*	|a	|k	|y	|a	|u	|*	|a	|*	|*	|.
|s	|s	|c	|a	|z	|j	|k	|p	|t	|d	|h	|g	|a	|z	|*	|n	|r	|j	|i	|s	|*	|r	|*	|*	|.
|k	|p	|y	|m	|o	|ą	|i	|e	|a	|y	|e	|m	|*	|*	|*	|t	|a	|n	|r	|e	|*	|o	|[41][S]\darr	|*	|.
|a	|o	|j	|a	|n	|c	|[][,]{ }	|t	|d	|t	|o	|i	|*	|*	|*	|e	|t	|y	|e	|k	|*	|w	|g	|*	|.
|*	|c	|n	|*	|y	|y	|e	|t	|*	|o	|w	|n	|*	|*	|*	|r	|a	|*	|*	|*	|*	|*	|o	|*	|.
|*	|z	|y	|*	|*	|*	|k	|i	|[42][S]\darr	|w	|i	|*	|*	|*	|*	|z	|*	|*	|[43][S]\rarr	|w	|i	|j	|e	|*	|.
|*	|y	|*	|*	|*	|*	|g	|s	|k	|y	|e	|*	|[44][S]\rarr	|t	|r	|y	|u	|m	|w	|i	|r	|a	|t	|*	|.
|*	|n	|[45][S]\rarr	|r	|y	|k	|*	|a	|o	|*	|c	|*	|*	|*	|[46][S]\rarr	|s	|k	|i	|p	|[][,]{ }	|c	|*	|e	|*	|.
|*	|e	|[47][S]\rarr	|r	|ę	|k	|a	|*	|l	|*	|*	|[48][S]\rarr	|h	|a	|f	|t	|a	|r	|n	|i	|a	|*	|l	|*	|.
|*	|k	|[49][S]\rarr	|n	|i	|e	|p	|r	|o	|f	|e	|s	|j	|o	|n	|a	|l	|n	|o	|ś	|ć	|*	|*	|*	|.
|*	|*	|*	|*	|[50][S]\rarr	|k	|l	|a	|r	|o	|w	|a	|n	|i	|e	|*	|*	|*	|*	|*	|*	|*	|*	|*	|.
|*	|*	|*	|*	|*	|*	|*	|*	|*	|*	|*	|*	|*	|*	|*	|*	|*	|*	|*	|*	|*	|*	|*	|*	|.\end{Puzzle}

\newpage

\begin{PuzzleClues}{\textbf{Poziome}\\}\Clue{1}{}{odmiana chlorenchymy u roślin okrytonasiennych dwuliściennych}
\Clue{3}{}{Hydrilla - rodzaj rośliny z rodziny żabiściekowatych}
\Clue{5}{}{atrybut postaci bajkowych: wróżek, czrodziejów}
\Clue{7}{}{powietrze doprowadzane do pieca pod ciśnieniem}
\Clue{8}{}{nuta, składowa część dźwięku}
\Clue{9}{}{Maikre; bokser kubański, mistrz olimpijski z Atlanty w wadze do 51 kg}
\Clue{10}{}{przeżuwacz z rodziny krętorogich objęty ewidencją rodowodową, w Polsce głównie w Puszczy Białowieskiej}
\Clue{11}{}{część instrumentu muzycznego, mechanizm umożliwiający grę na centrali w perkusji i hi-hacie}
\Clue{14}{}{wieś w Polsce położona w województwie świętokrzyskim, w powiecie kieleckim, w gminie Raków}
\Clue{15}{}{rodzaj zakładu produkcyjnego podczas okupacji hitlerowskiej, obecny na terenie getta}
\Clue{16}{}{organizacja o charakterze publicznym realizująca określone działania}
\Clue{17}{}{cząsteczka oligonukleotydów (krótkie fragmenty DNA lub RNA) lub peptydów, które wiążą się specyficznie z określoną cząsteczką}
\Clue{18}{}{kod ISO 4217 dolara autralijskiego}
\Clue{20}{}{kozioł sarny o słabo rozwiniętych porostkach}
\Clue{22}{}{rodzaj czerwonego w kolorze jabłka, owoc z drzewa odmiany o tej samej nazwie}
\Clue{23}{}{nieobecność pracownika, ucznia lub studenta w pracy lub na zajęciach w szkole (na uczelni)}
\Clue{25}{}{ZAPRAWA}
\Clue{29}{}{litera oznaczająca fonem}
\Clue{30}{}{stolica i największe miasto Kataru położone na wschodnim wybrzeżu półwyspu Katar, w Zatoce Perskiej}
\Clue{31}{}{kierunek, punkt lub oś, względem których się orientujemy}
\Clue{37}{}{SZWEJA}
\Clue{40}{}{funkcja odwrotna do funkcji secans rozpatrywanej na przedziale 0,?>}
\Clue{43}{}{gromada stawonogów lądowych o walcowatym ciele}
\Clue{44}{}{sojusz polityczny trzech mężów stanu w starożytnym Rzymie}
\Clue{45}{}{gardłowy, tubalny krzyk wydawany przez człowieka}
\Clue{46}{}{w sporcie - bieg z energicznymwyrzucaniem nóg do tyłu i w górę, z uderzaniem piętami o pośladki}
\Clue{47}{}{wykroczenie w piłce nożnej, niedozwolone dotknięcie piłki ręką}
\Clue{48}{}{pracownia rzemieślnicza, w której się haftowało}
\Clue{49}{}{dyletanctwo, działanie w oderwaniu od specjalistycznej wiedzy, niezgodne z  zasadami panującymi w danej branży, dziedzinie}
\Clue{50}{}{porządkowanie osprzętu statku}\end{PuzzleClues}

\begin{PuzzleClues}{\textbf{Pionowe}\\}\Clue{1}{}{forma przetargu charakteryzująca się zaproszeniem nieograniczonego kręgu osób do składania ofert kupna, zwykle poprzez publiczne ogłoszenie}
\Clue{2}{}{numer taktyczny dobrze widoczny, wymalowany na burcie okrętu lub nadbudówce, służy do idenfikacji okrętu na duże odległości}
\Clue{3}{}{Curcuma angustifolia - gatunek byliny z rodziny strelicjowatych}
\Clue{4}{}{w Krakowie i okolicach: krawat}
\Clue{5}{}{urządzenie rejestrujące pracę serca (EKG), w sposób ciągły, przez 24 godziny na dobę w celu późniejszej, szczegółowej, często komputerowej analizy}
\Clue{6}{}{budowla inżynieryjna zapewniająca efektywne prowadzenie ognia, walki i obserwacji oraz osłonę wojsk, sprzętu bojowego i urządzeń tyłowych przed środkami rażenia nieprzyjaciela}
\Clue{8}{}{izolowany pagórek o stromych stokach i płaskim szczycie, często porośnięty roślinnością lub pokryty odporniejszym materiałem}
\Clue{9}{}{podjęcie lub zaniechanie działań wpływających na środowisko, a w szczególności racjonalne gospodarowanie zasobami środowiska, przeciwdziałanie zanieczyszczeniom i przywracanie elementów przyrody do stanu właściwego}
\Clue{12}{}{waga opakowania, różnica między masą brutto a masą netto}
\Clue{13}{}{charakterystyczna dla kuchni ukraińskiej solona słonina, która jest spożywana jako przekąska}
\Clue{17}{}{twierdzenie mówiące, że jeżeli A jest podzbiorem grupy topologicznej G, który jest drugiej kategorii i ma własność Baire'a, to zbiór A\textasciicircum-1\}A zawiera otwarte otoczenie elementu neutralnego grupy}
\Clue{19}{}{zwłoki, kościotrup, prochy}
\Clue{21}{}{bawełniana cienka i gęsta tkanina o gładkiej powierzchni, która zwykle posiada jakiś efekt tkacki (fakturę, widoczny wzór, jeśli wykonana z użyciem kolorowych nici), stosowana na koszule męskie i bluzki damskie}
\Clue{23}{}{część ośrodkowego układu nerwowego należąca do pnia mózgu}
\Clue{24}{}{list polecający, którym bank uwiarygodnia swojego klienta, aby umożliwić mu lub innej wymienionej osobie pobranie gotówki w oddziale tego banku}
\Clue{26}{}{lalka w kolorowym stroju przedstawiająca istotę humanoidalną płci męskiej (na przykład chłopca, klowna, trefnisia)}
\Clue{27}{}{ten, który podróżuje, człowiek w podróży}
\Clue{28}{}{MIEDZIANKA; niejadowity, chroniony w Polsce wąż}
\Clue{29}{}{(silnika) część nadwozia samochodu w postaci pokrywy osłaniającej silnik}
\Clue{31}{}{miasto w Szwecji, port nad jeziorem Wener, ośrodek administracyjny hrabstwa Varmland}
\Clue{32}{}{wyborca Partii Demokratycznej w USA}
\Clue{33}{}{zdrobniale: piegus - osoba piegowata}
\Clue{34}{}{szachista gruziński, mistrz świata od 1985 r}
\Clue{35}{}{dawny piechur}
\Clue{36}{}{bębenek arabski}
\Clue{37}{}{drobny jednokomórkowiec zaliczany do prokariotów}
\Clue{38}{}{rzeźbiarz i ceramik (1875-1961 ); ceramiczne rzeźby o secesyjnych formach}
\Clue{39}{}{w budownictwie: prętowy układ konstrukcyjny, płaski lub przestrzenny, obciążony siłami skupionymi, momentami lub obciążeniem rozłożonym (równomiernie bądź też nierównomiernie)}
\Clue{40}{}{odpoczynek potrzebny do odzyskania sił}
\Clue{41}{}{geolog, badacz Tatr (1889-1972); współtwórca parków narodowych w Tatrach, Pieninach i na Babiej Górze}
\Clue{42}{}{w pokerze: pięć kart jednego koloru na ręce}\end{PuzzleClues}\newpage%\section*{Krzyżówka 104}

\noindent\begin{Puzzle}{14}{33}|*	|*	|[1][S]\darr	|*	|*	|*	|[2][S]\drarr	|p	|r	|o	|t	|e	|z	|a	|*	|.
|*	|[3][S]\drarr	|l	|i	|s	|i	|c	|z	|k	|a	|*	|*	|*	|*	|*	|.
|*	|s	|i	|[4][S]\rarr	|e	|g	|z	|e	|m	|p	|l	|a	|r	|z	|*	|.
|*	|t	|s	|*	|*	|[5][S]\drarr	|e	|k	|s	|p	|e	|r	|t	|*	|*	|.
|*	|a	|z	|*	|*	|d	|r	|[6][S]\darr	|*	|*	|*	|*	|[7][S]\darr	|*	|*	|.
|*	|r	|k	|*	|*	|y	|p	|p	|*	|*	|*	|[8][S]\darr	|a	|*	|*	|.
|*	|y	|o	|*	|*	|w	|a	|o	|*	|*	|*	|b	|r	|*	|*	|.
|*	|[][,]{ }	|j	|*	|*	|a	|l	|w	|*	|[9][S]\drarr	|k	|l	|i	|f	|*	|.
|*	|b	|a	|*	|[10][S]\darr	|n	|n	|i	|*	|b	|*	|u	|m	|*	|[11][S]\darr	|.
|*	|y	|d	|*	|m	|o	|i	|e	|*	|o	|*	|e	|a	|[12][S]\darr	|ż	|.
|*	|k	|y	|*	|a	|k	|a	|ś	|[13][S]\darr	|n	|[14][S]\darr	|s	|*	|f	|ó	|.
|*	|*	|*	|*	|r	|s	|[][,]{ }	|ć	|k	|n	|a	|m	|*	|o	|ł	|.
|*	|*	|*	|*	|k	|z	|p	|[][,]{ }	|i	|*	|l	|a	|*	|n	|w	|.
|*	|*	|*	|*	|i	|t	|a	|r	|e	|[15][S]\darr	|i	|n	|*	|d	|[][,]{ }	|.
|*	|*	|*	|[16][S]\drarr	|z	|a	|p	|a	|ł	|k	|a	|*	|[17][S]\darr	|u	|ś	|.
|*	|*	|*	|r	|e	|ł	|i	|d	|ż	|u	|n	|[18][S]\darr	|w	|e	|w	|.
|*	|*	|*	|ó	|t	|t	|e	|i	|[][,]{ }	|l	|s	|i	|a	|[][,]{ }	|i	|.
|*	|*	|*	|ż	|a	|n	|r	|o	|j	|a	|[][,]{ }	|n	|n	|c	|ą	|.
|*	|*	|*	|a	|*	|e	|u	|w	|e	|*	|s	|w	|g	|z	|t	|.
|*	|*	|*	|[][,]{ }	|*	|*	|*	|a	|z	|*	|t	|e	|a	|e	|y	|.
|*	|[19][S]\rarr	|o	|p	|o	|k	|a	|*	|i	|*	|r	|s	|[][,]{ }	|k	|n	|.
|[20][S]\drarr	|c	|h	|u	|j	|*	|*	|*	|o	|*	|a	|t	|h	|o	|n	|.
|k	|*	|*	|s	|*	|[21][S]\darr	|*	|*	|r	|*	|t	|y	|a	|l	|y	|.
|o	|*	|*	|t	|*	|s	|*	|[22][S]\drarr	|n	|i	|e	|c	|k	|a	|*	|.
|n	|*	|*	|y	|*	|e	|*	|j	|y	|[23][S]\darr	|g	|j	|o	|d	|*	|.
|k	|*	|*	|n	|*	|r	|*	|u	|*	|k	|i	|a	|d	|o	|*	|.
|u	|[24][S]\rarr	|p	|i	|z	|a	|*	|r	|*	|r	|c	|*	|z	|w	|*	|.
|r	|*	|*	|*	|*	|j	|*	|o	|*	|z	|z	|*	|i	|e	|*	|.
|e	|*	|*	|*	|*	|*	|*	|r	|*	|y	|n	|*	|o	|*	|*	|.
|n	|*	|*	|*	|*	|*	|*	|*	|*	|ż	|y	|*	|b	|*	|*	|.
|t	|[25][S]\rarr	|z	|j	|a	|w	|i	|s	|k	|o	|*	|*	|a	|*	|*	|.
|*	|*	|*	|[26][S]\rarr	|w	|a	|r	|g	|o	|w	|c	|e	|*	|*	|*	|.
|*	|*	|*	|*	|*	|*	|*	|*	|*	|e	|*	|*	|*	|*	|*	|.
|*	|*	|*	|*	|*	|*	|*	|*	|*	|*	|*	|*	|*	|*	|*	|.\end{Puzzle}

\newpage

\begin{PuzzleClues}{\textbf{Poziome}\\}\Clue{2}{}{substytut, coś, czym zastępuje się coś innego, co wynagradza, ukrywa brak czegoś}
\Clue{3}{}{inna nazwa pieprznika jadalnego}
\Clue{4}{}{sztuka czegoś}
\Clue{5}{}{starszy mason, który przeprowadza inicjację młodych,brat starszy}
\Clue{9}{}{FALEZA; urwisty brzeg morski}
\Clue{16}{}{fryzura, w której głowę goli się na łyso}
\Clue{19}{}{skała mieszana, zbudowana z organogenicznej krzemionki (opal, chalcedon, przeważnie pochodzące z krzemionkowych gąbek) i węglanu wapnia}
\Clue{20}{}{męski narząd płciowy}
\Clue{22}{}{tyle, ile mieści się w niecce}
\Clue{24}{}{miasto we Włoszech (Toskania) nad rzeką Arno, międzynarodowe znaczenie turystyczne, tzw. krzywa wieża}
\Clue{25}{}{coś bardzo dziwnego, często budzącego zachwyt}
\Clue{26}{}{jasnotowce, Lamiales - rząd roślin okrytonasiennych z kladu astrowych}\end{PuzzleClues}

\begin{PuzzleClues}{\textbf{Pionowe}\\}\Clue{1}{}{gąsieniczniki, Campephagidae - rodzina ptaków z rzędu wróblowych (Passeriformes) obejmująca około osiemdziesięciu gatunków ptaków występujących w strefie tropikalnej i subtropikalnej Afryki, Azji i Australazji}
\Clue{2}{}{zakład produkujący ręcznie czerpany papier}
\Clue{3}{}{dorosły mężczyzna, starszy w porównaniu z kimś (np. grupą innych ludzi lub sobą samym z wcześniejszego okresu)}
\Clue{5}{}{rekiny dywanowe, Orectolobiformes - rząd ryb chrzęstnoszkieletowych (Chondrichthyes); największym przedstawicielem rekinów dywanowych i jednocześnie największym z rekinów jest rekin wielorybi}
\Clue{6}{}{powieść tworzona dla radia i czytana w odcinkach w radiu}
\Clue{7}{}{miasto w Trynidadzie i Tobago na wyspie Trynindad}
\Clue{8}{}{miłośnik muzyki bluesowej}
\Clue{9}{}{miasto w Niemczech, port nad Renem (Nadrenia Płn. Westfalia) miejsce urodzin Beethovena}
\Clue{10}{}{przezroczysta, cienka i lekka, dość ekskluzywna tkanina bawełniana, używana głównie na firanki i zasłony}
\Clue{11}{}{Hieremys annandalii - gatunek gada z rodziny batagurowatych, zamieszkujący Półwysep Indochiński; według buddyjskich wierzeń człowiek, który uratuje od śmierci tego żółwia, może się spodziewać lepszego losu w swoim przyszłym życiu, dlatego ludność uwalnia z sieci i innych pułapek żółwie świątynne (często też inne gatunki) i wypuszcza je na wolność w pobliżu tzw.świątyń żółwi}
\Clue{12}{}{potrawa typu fondue, w której masę serową zastępuje się roztopioną czekoladą kilku gatunków, a pieczywo owocami lub biszkoptami}
\Clue{13}{}{Gammarus lacustris - słodkowodny gatunek skorupiaka z rzędu obunogów}
\Clue{14}{}{połączenie współpracy i konkurencji w pewnej grupie przedsiębiorstw dostarczających gamę produktów częściowo komplementarnych; umowa między konkurentami, na zasadach partnerstwa}
\Clue{15}{}{nabój wystrzeliwany z broni palnej}
\Clue{16}{}{róża gipsowa, tworząca się na obszarach pustynnych w warunkach klimatu suchego i gorącego}
\Clue{17}{}{Vanga curvirostris - gatunek ptaka z rodziny wang (Vangidae)}
\Clue{18}{}{efekt zainwestowanych środków}
\Clue{20}{}{rywal, przeciwnik we współzawodnictwie różnego rodzaju}
\Clue{21}{}{zamknięta część domu, przeznaczona wyłącznie dla kobiet z rodziny męża i żony}
\Clue{22}{}{arbiter, który nie musi być ekspertem, powołany do oceny uczestników konkursu}
\Clue{23}{}{rodzina roślin zielnych z klasy dwuliściennych}\end{PuzzleClues}\newpage%\section*{Krzyżówka 105}

\noindent\begin{Puzzle}{17}{31}|*	|*	|*	|[1][S]\drarr	|m	|a	|r	|c	|o	|n	|i	|*	|*	|*	|*	|*	|*	|*	|.
|*	|*	|[2][S]\darr	|p	|*	|[3][S]\darr	|*	|*	|[4][S]\darr	|*	|*	|[5][S]\drarr	|h	|e	|b	|e	|i	|*	|.
|*	|*	|r	|ó	|[6][S]\drarr	|t	|r	|ó	|j	|g	|u	|z	|k	|o	|w	|c	|e	|*	|.
|*	|*	|y	|ł	|c	|a	|*	|[7][S]\darr	|o	|[8][S]\darr	|[9][S]\drarr	|d	|u	|m	|a	|s	|*	|*	|.
|*	|*	|b	|j	|m	|p	|*	|a	|r	|b	|l	|o	|[10][S]\drarr	|ł	|a	|z	|y	|*	|.
|*	|*	|i	|a	|*	|i	|*	|b	|a	|a	|a	|b	|p	|[11][S]\darr	|*	|*	|[12][S]\darr	|*	|.
|*	|[13][S]\rarr	|k	|w	|a	|r	|t	|a	|*	|l	|j	|i	|o	|i	|*	|*	|c	|[14][S]\darr	|.
|*	|*	|*	|a	|[15][S]\darr	|*	|*	|*	|*	|u	|t	|k	|l	|n	|*	|*	|z	|h	|.
|*	|*	|*	|*	|t	|*	|[16][S]\drarr	|j	|a	|t	|a	|*	|s	|i	|[17][S]\darr	|*	|a	|o	|.
|*	|*	|*	|*	|ł	|[18][S]\darr	|h	|*	|*	|*	|*	|*	|k	|c	|b	|*	|s	|l	|.
|*	|*	|[19][S]\darr	|*	|u	|n	|e	|*	|[20][S]\rarr	|n	|a	|w	|i	|j	|a	|c	|z	|*	|.
|*	|*	|k	|[21][S]\drarr	|c	|a	|l	|b	|a	|y	|o	|g	|*	|a	|l	|[22][S]\darr	|a	|[23][S]\darr	|.
|*	|*	|u	|ś	|z	|r	|m	|[24][S]\darr	|[25][S]\darr	|[26][S]\darr	|[27][S]\drarr	|k	|a	|t	|o	|n	|*	|t	|.
|*	|[28][S]\darr	|c	|w	|e	|c	|a	|ś	|n	|b	|r	|*	|*	|y	|w	|a	|*	|o	|.
|*	|z	|h	|i	|n	|i	|n	|w	|a	|o	|y	|*	|*	|w	|i	|d	|*	|w	|.
|*	|w	|e	|s	|i	|a	|*	|i	|r	|c	|j	|*	|*	|a	|c	|z	|*	|a	|.
|[29][S]\drarr	|i	|n	|t	|e	|r	|m	|e	|z	|z	|o	|*	|*	|[][,]{ }	|z	|ó	|*	|r	|.
|h	|ą	|k	|u	|c	|s	|*	|r	|ą	|e	|s	|*	|*	|u	|k	|r	|*	|z	|.
|u	|z	|a	|n	|*	|t	|*	|k	|d	|ń	|k	|*	|*	|c	|a	|[][,]{ }	|*	|y	|.
|b	|e	|[][,]{ }	|*	|*	|w	|[30][S]\darr	|l	|[][,]{ }	|[][,]{ }	|o	|*	|*	|h	|*	|p	|*	|s	|.
|a	|k	|e	|*	|*	|o	|s	|a	|l	|n	|c	|*	|*	|w	|*	|e	|*	|z	|.
|[][,]{ }	|[][,]{ }	|l	|*	|[31][S]\darr	|[][,]{ }	|y	|n	|e	|a	|z	|*	|*	|a	|*	|d	|*	|[][,]{ }	|.
|p	|s	|e	|*	|a	|n	|n	|i	|m	|s	|k	|*	|*	|ł	|[32][S]\darr	|a	|*	|n	|.
|a	|p	|k	|*	|g	|o	|g	|e	|i	|t	|o	|*	|*	|o	|d	|g	|*	|i	|.
|c	|o	|t	|*	|u	|r	|i	|c	|e	|r	|w	|*	|*	|d	|y	|o	|*	|e	|.
|h	|r	|r	|*	|l	|w	|e	|*	|s	|o	|a	|*	|*	|a	|s	|g	|[33][S]\darr	|d	|.
|n	|t	|y	|[34][S]\rarr	|h	|e	|l	|*	|z	|s	|t	|*	|[35][S]\darr	|w	|k	|i	|p	|o	|.
|ą	|o	|c	|*	|a	|s	|*	|*	|o	|z	|e	|*	|b	|c	|[][,]{ }	|c	|i	|l	|.
|c	|w	|z	|*	|s	|k	|*	|*	|w	|o	|*	|*	|o	|z	|z	|z	|t	|i	|.
|a	|y	|n	|*	|*	|i	|*	|*	|y	|n	|*	|[36][S]\rarr	|m	|a	|i	|n	|e	|*	|.
|*	|*	|a	|*	|*	|e	|*	|*	|*	|y	|*	|*	|a	|*	|p	|y	|a	|*	|.
|*	|*	|*	|*	|*	|*	|*	|*	|*	|*	|*	|*	|*	|*	|*	|*	|*	|*	|.\end{Puzzle}

\newpage

\begin{PuzzleClues}{\textbf{Poziome}\\}\Clue{1}{}{Leandro (1834-1919), syn Henryka, architekt, przedstawiciel historyzmu}
\Clue{5}{}{HOPEJ - prowincja we wsch. Chinach, powierzchnia 202,7 tyś. km2', ważny region przemysłowy, 54,9 min mieszkańców}
\Clue{6}{}{Trituberculata) - całkowicie wymarłe drobne zwierzęta ssakokształtne. Ich szczątki znaleziono w mezozoicznych pokładach pochodzących z 215-85 mln lat}
\Clue{9}{}{chemik francuski (1800-84); metoda oznaczania azotu w związkach organicznych}
\Clue{10}{}{Łazy - liczne wsie w całym kraju}
\Clue{13}{}{dawna jednostka objętości ciał sypkich a także naczynie o pojemności około 1 litra}
\Clue{16}{}{górnicza szopa do przechowywania rudy}
\Clue{20}{}{wokalista reggae i dancehall, śpiewający w charakterystyczny sposób przypominający recytację}
\Clue{21}{}{miasto w Filipinach, na wyspie Samar port wywozu manili; przemysł rybny}
\Clue{27}{}{człowiek przypominający charakterem Katona - bardzo surowy, odznaczający się bezkompromisowością, wymagający od innych i od siebie, wierny prawu}
\Clue{29}{}{XIX w miniatura fortepianowa}
\Clue{34}{}{aparat telegraficzny HELLA; aparat telegrafii alfabetowej}
\Clue{36}{}{kraina historyczna w płn-zach. Francji}\end{PuzzleClues}

\begin{PuzzleClues}{\textbf{Pionowe}\\}\Clue{1}{}{stan częściowej świadomości}
\Clue{2}{}{owad bezskrzydłowy okryty metalicznie lśniącymi łuskami, szkodnik w magazynach i spiżarniach}
\Clue{3}{}{Tapirus - rodzaj ssaków z rzędu nieparzystokopytnych, występujący w tropikalnych dżunglach Ameryki Południowej i Azji}
\Clue{4}{}{rumuński kompozytor i pedagog (1891-1971); balety, suity orkiestrowe, utwory kameralne, pieśni}
\Clue{5}{}{chrząszcz z tutkarzy atakuje topolę, brzozę i inne}
\Clue{6}{}{w chemii: symbol kiuru}
\Clue{7}{}{ABAJA}
\Clue{8}{}{tradycyjna potrawa kuchni południowo-wschodnioazjatyckich, gotowane kurze lub kacze jajko, wewnątrz którego znajduje się w pełni uformowany zarodek ptaka}
\Clue{9}{}{architekt węgierski (1875-1920), modernista}
\Clue{10}{}{przedmiot szkolny lub uczony w ramach kursu, na którym opanowuje się podstawy języka polskiego}
\Clue{11}{}{uprawnienie do przedkładania władzy samorządowej projektów uchwał}
\Clue{12}{}{część kuli, która jest odcięta płaszczyzną}
\Clue{14}{}{HALL; mieszkalna sień w angielskim domu, także duży przedpokój, poczekalnie}
\Clue{15}{}{ciastko z tłuczonych migdałów}
\Clue{16}{}{ur. w 1937 r., muzykolog, profesor Uniwersytetu Warszawskiego}
\Clue{17}{}{bywalczyni imprez}
\Clue{18}{}{narciarstwo, które wykształciło się w dużej mierze w krajach skandynawskich, w oparciu o tradycyjne narty „z wolną piętą” - z wiązaniem przytrzymującym jedynie przednią część buta}
\Clue{19}{}{urządzenie kuchenne zamieniająca energię elektryczną na ciepło używane do gotowania, smażenia i pieczenia}
\Clue{21}{}{gatunek kaczki, roślinożerna, w Polsce nieliczna, łowna}
\Clue{22}{}{nadzór sprawowany nad działalnością dydaktyczną, wychowawczą i opiekuńczą szkół, placówek i nauczycieli; termin prawny}
\Clue{23}{}{osoba, która współcierpi z kimś}
\Clue{24}{}{wieś w Polsce położona w województwie śląskim, w powiecie tarnogórskim, w gminie Świerklaniec, siedziba gminy Świerklaniec}
\Clue{25}{}{występujący u wielu zwierząt, chemoreceptywny, parzysty narząd zmysłu wykrywający szereg substancji chemicznych}
\Clue{26}{}{Pleurochaete squarrosa - gatunek mchu należący do rodziny płoniwowatych (Pottiaceae); występuje w suchych obszarach Europy Południowej, w Polsce bardzo rzadki, objęty ścisłą ochroną gatunkową}
\Clue{27}{}{ryjoskoczki, Macroscelididae - jedyna rodzina w obrębie rzędu ryjkonosów, należy do niej kilkanaście współcześnie żyjących oraz kilka wymarłych gatunków niewielkich afrykańskich zwierząt, wcześniej zaliczanych do owadożernych}
\Clue{28}{}{stowarzyszenie, a także związek stowarzyszeń, który zrzesza co najmniej 3 kluby sportowe}
\Clue{29}{}{Gloeophyllum odoratum - grzyb z rodziny niszczycowatych, który ciemniej z wiekiem}
\Clue{30}{}{SINGEL}
\Clue{31}{}{ciepły prąd morski na oceanie Indyjskim}
\Clue{32}{}{nośnik danych o pojemności do 750 mb używany głównie do tworzenia kopii zapasowej danych i archiwizacji plików}
\Clue{33}{}{miasto i port w Szwecji w pobliżu ujścia rzeki Pite do Zatoki Botnickiej, przemysł drzewny i celulozowo-papierniczy}
\Clue{35}{}{miasto w Zairze, port nad rzeką Kongo}\end{PuzzleClues}\newpage%\section*{Krzyżówka 106}

\noindent\begin{Puzzle}{22}{23}|*	|*	|[1][S]\darr	|*	|[2][S]\darr	|[3][S]\drarr	|k	|a	|c	|e	|r	|z	|*	|[4][S]\drarr	|h	|u	|r	|y	|c	|k	|i	|*	|*	|.
|*	|*	|w	|[5][S]\rarr	|t	|r	|ą	|b	|k	|a	|*	|[6][S]\darr	|[7][S]\drarr	|s	|y	|c	|z	|e	|k	|*	|[8][S]\darr	|*	|*	|.
|*	|*	|i	|[9][S]\darr	|a	|a	|[10][S]\drarr	|s	|y	|l	|w	|e	|t	|k	|a	|*	|[11][S]\drarr	|m	|e	|n	|s	|a	|*	|.
|*	|*	|a	|i	|g	|n	|k	|[12][S]\drarr	|e	|p	|i	|k	|u	|r	|e	|i	|z	|m	|*	|[13][S]\darr	|z	|*	|*	|.
|*	|*	|t	|d	|*	|w	|o	|w	|[14][S]\darr	|[15][S]\darr	|*	|s	|j	|a	|[16][S]\darr	|[17][S]\drarr	|a	|u	|t	|k	|o	|*	|*	|.
|*	|[18][S]\drarr	|r	|o	|ż	|e	|n	|i	|e	|c	|*	|t	|a	|p	|p	|c	|r	|*	|*	|a	|s	|*	|*	|.
|*	|b	|ó	|l	|*	|r	|i	|r	|l	|e	|*	|r	|*	|l	|o	|u	|a	|*	|*	|l	|z	|*	|*	|.
|*	|a	|w	|*	|*	|s	|k	|t	|e	|r	|*	|a	|[19][S]\darr	|a	|r	|k	|n	|[20][S]\drarr	|s	|m	|o	|k	|*	|.
|*	|g	|k	|[21][S]\darr	|*	|y	|*	|u	|a	|e	|*	|k	|t	|r	|z	|i	|i	|w	|*	|a	|ń	|*	|*	|.
|*	|d	|a	|a	|*	|*	|*	|o	|r	|s	|*	|c	|u	|k	|ą	|e	|e	|i	|[22][S]\darr	|r	|s	|*	|*	|.
|*	|e	|*	|g	|[23][S]\drarr	|g	|u	|z	|*	|*	|*	|j	|r	|a	|d	|r	|*	|e	|g	|*	|k	|*	|*	|.
|*	|*	|*	|e	|r	|[24][S]\rarr	|p	|e	|n	|d	|ż	|a	|b	|*	|n	|n	|*	|l	|ł	|*	|i	|[25][S]\darr	|*	|.
|*	|*	|[26][S]\rarr	|n	|a	|d	|p	|r	|o	|ż	|e	|*	|i	|*	|o	|i	|[27][S]\drarr	|k	|o	|k	|*	|m	|*	|.
|*	|[28][S]\rarr	|t	|e	|s	|t	|[][,]{ }	|s	|p	|r	|a	|w	|n	|o	|ś	|c	|i	|o	|w	|y	|*	|ł	|*	|.
|*	|[29][S]\drarr	|s	|z	|a	|m	|o	|t	|a	|n	|i	|n	|a	|*	|ć	|a	|b	|ś	|a	|[30][S]\rarr	|l	|o	|*	|.
|*	|b	|*	|j	|m	|*	|*	|w	|[31][S]\drarr	|a	|l	|t	|*	|*	|*	|*	|i	|ć	|*	|*	|*	|c	|*	|.
|*	|r	|*	|a	|*	|*	|*	|o	|b	|*	|[32][S]\rarr	|d	|e	|i	|m	|o	|s	|*	|*	|[33][S]\rarr	|c	|a	|*	|.
|*	|u	|*	|[][,]{ }	|*	|[34][S]\rarr	|d	|*	|i	|*	|*	|*	|*	|[35][S]\rarr	|g	|n	|o	|z	|a	|*	|*	|r	|*	|.
|*	|n	|*	|n	|*	|[36][S]\rarr	|r	|o	|s	|z	|c	|z	|e	|n	|i	|o	|w	|o	|ś	|ć	|*	|z	|*	|.
|[37][S]\rarr	|d	|z	|e	|t	|a	|*	|*	|*	|*	|*	|*	|[38][S]\rarr	|n	|a	|p	|i	|n	|a	|c	|z	|*	|*	|.
|[39][S]\rarr	|a	|c	|r	|e	|*	|*	|*	|*	|*	|[40][S]\rarr	|n	|a	|g	|r	|z	|e	|w	|*	|*	|*	|*	|*	|.
|*	|g	|*	|k	|[41][S]\drarr	|p	|r	|ą	|d	|[][,]{ }	|f	|a	|r	|a	|d	|y	|c	|z	|n	|y	|*	|*	|*	|.
|*	|e	|*	|i	|w	|[42][S]\rarr	|c	|z	|y	|ś	|c	|i	|c	|i	|e	|l	|*	|*	|*	|*	|*	|*	|*	|.
|*	|*	|*	|*	|*	|*	|*	|*	|*	|*	|*	|*	|*	|*	|*	|*	|*	|*	|*	|*	|*	|*	|*	|.\end{Puzzle}

\newpage

\begin{PuzzleClues}{\textbf{Poziome}\\}\Clue{3}{}{sieć rybacka do połowu zarówno ryb i raków}
\Clue{4}{}{wymarły język używany przez Hurytów na obszarach północnej Mezopotamii, nad jeziorem Wan, wschodniej Anatolii i Syrii północnej już w połowie III tysiąclecia p.n.e}
\Clue{5}{}{haustellum, proboscis - wieloznaczny termin stosowany w zoologii dla określenia wydłużonego, nieparzystego przydatku na głowie niektórych zwierząt, głównie bezkręgowych, zazwyczaj połączonego z otworem gębowym i służącego do pobierania pokarmu}
\Clue{7}{}{gatunek sowy, w Polsce rzadki, chroniony, długość do 20 cm}
\Clue{10}{}{budowa ciała}
\Clue{11}{}{świadczenia kleru parafialnego na utrzymanie biskupa i jego domowników}
\Clue{12}{}{kierunek filozoficzny, według którego u podstaw szczęścia leży racjonalne korzystanie z przyjemności}
\Clue{17}{}{samochodzik-zabawka}
\Clue{18}{}{rożeniec zwyczajny, Anas acuta - gatunek dużego, wędrownego ptaka wodnego z rodziny kaczkowatych (Anatidae)}
\Clue{20}{}{kosz ssawny}
\Clue{23}{}{każde nietypowe uwypuklenie, opór przy palpacji, odgraniczone ognisko w badaniu obrazowym, itp. o nieznanym (tzn. niezdiagnozowanym) pochodzeniu}
\Clue{24}{}{stan w płn-zach. Indiach na Nizinie Hindostańskiej, obszar 50,4 tyś. km2, stolica Czandigarh}
\Clue{26}{}{poziomy, płaski lub sklepiony (przesklepienie) element konstrukcyjny w formie belki umieszczanej ponad otworami w ścianie}
\Clue{27}{}{kucharz okrętowy - osoba gotująca dla załogi na jednostce pływającej, legitymująca się świadectwem wydanym przez urząd morski}
\Clue{28}{}{sprawdzian sprawności fizycznej, podczas którego wykonuje się określone ćwiczenia, a wyniki stają się podstawą jakieś kwalifikacji}
\Clue{29}{}{sytuacja, w której ktoś z kimś się szarpie}
\Clue{30}{}{skrótowiec odliceum ogólnokszałcące}
\Clue{31}{}{malarz austriacki (1812-1905), malował akwarelowe pejzaże i sceny rodzajowe, przedstawiciel szkoły pejzażowej}
\Clue{32}{}{jeden z satelitów Marsa}
\Clue{33}{}{w chemii: symbol wapnia}
\Clue{34}{}{litera oznaczająca wymiar}
\Clue{35}{}{synkretyczne nurty religijne hellenistycznego i wczesnochrześcijańskiego okresu}
\Clue{36}{}{postawa przejawiająca się w nadmiernych i nieuzasadnionych żądaniach}
\Clue{37}{}{szósta litera alfabetu greckiego, w greckim systemie liczbowym oznacza liczbę 7}
\Clue{38}{}{mięsień napinacz podniebienia miękkiego, łac. musculus tensor veli palatini — parzysty mięsień poprzecznie prążkowany biorący udział w czynności podniebienia miękkiego i trąbki słuchowej}
\Clue{39}{}{stan w północno-zach. Brazylii, przy granicy z Boliwią i Peru, stolicą Rio Brance, powierzchnia 152,6 tyś. km2}
\Clue{40}{}{osiąganie wysokiej temperatury, nagrzewanie się}
\Clue{41}{}{asymetryczny prąd indukcyjny o częstotliwości od 50 do 100Hz, który otrzymuje się z induktora}
\Clue{42}{}{osoba czyszcząca, sprzątająca}\end{PuzzleClues}

\begin{PuzzleClues}{\textbf{Pionowe}\\}\Clue{1}{}{rodzaj cienkiej kurtki, która ma za zadanie chronić przed opadami i wiatrem}
\Clue{2}{}{unikatowy podpis twórcy, artysty pod pracą graficzną}
\Clue{3}{}{wywinięte wyłogi w żakiecie lub bluzce damskiej, stanowiące ozdobne wykończenie}
\Clue{4}{}{urządzenie do skraplania gazu}
\Clue{6}{}{usunięcie (najczęściej zęba, niekiedy siatkówki oka)}
\Clue{7}{}{drzewo iglaste z rodziny cyprysowatych; żywotnik}
\Clue{8}{}{odmiana języka, której używają Szoszoni}
\Clue{9}{}{osoba uwielbiana, popularna, sławna}
\Clue{10}{}{mały koń}
\Clue{11}{}{początek czegoś, inicjalna faza jakiegoś procesu}
\Clue{12}{}{cecha czegoś, co jest wspaniałe, mistrzowskie, godne wirtuoza - osoby, która stała się mistrzem w jakiejś dziedzinie, robi coś z lekkością, finezją, wspaniale; przejaw czyjejś wirtuozerii}
\Clue{13}{}{KAŁAMARNICA; dziesięciornica o torpedowatym ciele z dwoma płetwami  i smacznym mięsie}
\Clue{14}{}{XVI-XVII w ochotnik, harcownik, żołnierz nieregularnej jazdy lekkiej}
\Clue{15}{}{największa planetoida odkryta w 1801 r}
\Clue{16}{}{cecha rezultatu jakiegoś działania: to, że coś jest dobrze zrobione (i zarazem - pozytywnie oceniane)}
\Clue{17}{}{naczynie do cukru}
\Clue{18}{}{rodzaj identyfikatora, znaczka z hasłem, przypinki mocowanej do ubrania (lemat pochodzi z języka angielskiego)}
\Clue{19}{}{silnik przepływowy przetwarzający energię wewnętrzną sprężonego gazu, pary, cieczy na energię mechaniczna ruchu obrotowego wirnika}
\Clue{20}{}{cecha tego, co jest wielkie (pod względem rozmiaru fizycznego)}
\Clue{21}{}{wada wrodzona polegająca na niewykształceniu się nerki}
\Clue{22}{}{ścięgno początkowe mięśnia}
\Clue{23}{}{południowoindyjska zupa na bazie soku z owoców tamaryndowca z dodatkiem soczewicy i przypraw}
\Clue{25}{}{osoba młócąca zboże, pracująca przy młócce}
\Clue{27}{}{azjatycki ptak z rodziny szczudłaków}
\Clue{29}{}{amerykański działacz sportowy, przewodniczący Międzynarodowego Komitetu Olimpijskiego 1952-72, rzecznik przestrzegania amatorstwa w sporcie}
\Clue{31}{}{dodatkowe wykonanie utworu muzycznego na koniec występu}
\Clue{41}{}{węzeł - jednostka miary, równa jednej mili morskiej na godzinę; stosowana do określania prędkości morskich jednostek pływających, a w części państw i w ruchu międzynarodowym także statków powietrznych (samolotów, śmigłowców, szybowców, balonów)}\end{PuzzleClues}\newpage%\section*{Krzyżówka 107}

\noindent\begin{Puzzle}{22}{22}|*	|*	|*	|[1][S]\darr	|*	|[2][S]\drarr	|d	|y	|n	|k	|s	|*	|*	|[3][S]\drarr	|n	|u	|r	|o	|*	|[4][S]\darr	|*	|*	|[5][S]\darr	|.
|*	|*	|*	|h	|[6][S]\rarr	|b	|r	|o	|d	|z	|i	|k	|*	|s	|*	|*	|*	|*	|[7][S]\darr	|p	|[8][S]\darr	|*	|b	|.
|*	|*	|*	|i	|*	|i	|*	|[9][S]\drarr	|o	|g	|o	|r	|z	|a	|ł	|k	|a	|*	|g	|o	|m	|*	|a	|.
|*	|[10][S]\darr	|*	|s	|*	|e	|[11][S]\drarr	|k	|a	|p	|t	|u	|r	|n	|i	|k	|*	|*	|a	|m	|o	|*	|t	|.
|*	|k	|*	|t	|[12][S]\drarr	|b	|r	|a	|d	|l	|e	|y	|*	|d	|*	|*	|*	|*	|r	|i	|y	|*	|a	|.
|*	|u	|*	|o	|l	|r	|ó	|p	|*	|[13][S]\drarr	|k	|ó	|ł	|e	|c	|z	|k	|o	|*	|n	|z	|*	|k	|.
|*	|s	|*	|n	|e	|z	|ż	|i	|*	|p	|*	|[14][S]\drarr	|p	|r	|z	|y	|l	|e	|p	|i	|e	|c	|*	|.
|*	|a	|*	|[][,]{ }	|w	|a	|a	|t	|*	|a	|[15][S]\drarr	|k	|o	|s	|m	|o	|s	|*	|*	|ę	|s	|[16][S]\darr	|*	|.
|*	|c	|*	|ł	|*	|*	|*	|a	|[17][S]\darr	|n	|k	|o	|*	|o	|[18][S]\rarr	|p	|o	|m	|o	|c	|*	|a	|*	|.
|*	|z	|[19][S]\darr	|ą	|[20][S]\darr	|*	|[21][S]\darr	|n	|p	|d	|b	|ł	|[22][S]\rarr	|n	|i	|c	|o	|l	|a	|i	|*	|s	|*	|.
|*	|[][,]{ }	|m	|c	|b	|*	|w	|a	|r	|o	|*	|o	|*	|i	|[23][S]\rarr	|s	|a	|r	|g	|e	|n	|t	|*	|.
|[24][S]\drarr	|r	|e	|z	|o	|n	|a	|t	|o	|r	|[][,]{ }	|k	|w	|a	|r	|c	|o	|w	|y	|*	|*	|r	|*	|.
|f	|d	|d	|n	|g	|*	|r	|*	|d	|a	|[25][S]\darr	|o	|[26][S]\darr	|*	|*	|*	|*	|*	|*	|[27][S]\darr	|[28][S]\darr	|o	|*	|.
|u	|z	|r	|i	|n	|*	|i	|[29][S]\darr	|u	|*	|t	|l	|m	|[30][S]\darr	|*	|*	|[31][S]\darr	|*	|*	|t	|l	|f	|*	|.
|l	|a	|e	|k	|a	|[32][S]\drarr	|a	|k	|c	|j	|o	|n	|a	|r	|i	|u	|s	|z	|*	|y	|h	|o	|*	|.
|a	|w	|s	|o	|r	|p	|c	|i	|e	|*	|w	|i	|r	|w	|[33][S]\rarr	|i	|n	|f	|o	|m	|a	|t	|*	|.
|n	|y	|a	|w	|*	|l	|t	|m	|n	|*	|a	|a	|e	|d	|*	|*	|i	|*	|*	|p	|s	|o	|*	|.
|i	|*	|*	|y	|*	|e	|w	|a	|t	|*	|r	|*	|n	|*	|*	|*	|f	|*	|*	|a	|a	|m	|*	|.
|*	|[34][S]\rarr	|d	|*	|*	|j	|o	|*	|*	|*	|*	|*	|z	|[35][S]\rarr	|n	|a	|t	|r	|o	|n	|*	|e	|*	|.
|[36][S]\rarr	|ł	|u	|s	|k	|a	|*	|*	|*	|*	|[37][S]\rarr	|l	|i	|ś	|c	|i	|e	|c	|*	|*	|*	|t	|*	|.
|[38][S]\rarr	|s	|k	|ł	|a	|d	|*	|*	|[39][S]\rarr	|ż	|y	|w	|o	|ś	|ć	|*	|r	|[40][S]\rarr	|a	|x	|a	|r	|*	|.
|[41][S]\rarr	|e	|u	|c	|h	|a	|r	|y	|s	|t	|i	|a	|*	|[42][S]\rarr	|z	|r	|*	|*	|*	|*	|*	|*	|*	|.
|[43][S]\rarr	|t	|e	|m	|a	|*	|*	|*	|*	|*	|*	|*	|*	|*	|*	|*	|*	|*	|*	|*	|*	|*	|*	|.\end{Puzzle}

\newpage

\begin{PuzzleClues}{\textbf{Poziome}\\}\Clue{2}{}{wihajster, który stanowi sprytne rozwiązanie techniczne}
\Clue{3}{}{miasto we Włoszech (Sardynia), ośrodek turystyczny}
\Clue{6}{}{płytki basen przeznaczony dla dzieci do zabaw lub dla osób, które nie umieją pływać}
\Clue{9}{}{ogorzałka zwyczajna, Aythya marila - gatunek średniej wielkości wędrownego ptaka wodnego z rodziny kaczkowatych (Anatidae)}
\Clue{11}{}{tracz amerykański, Lophodytes cucullatus - gatunek ptaka z rodziny kaczkowatych (Anatidae), jedyny przedstawiciel rodzaju Lophodytes; gniazduje w pobliżu słodkich wód w Ameryce Północnej}
\Clue{12}{}{astronom angielski (1693-1762), odkrył aberrację światła i nutację osi Ziemi}
\Clue{13}{}{małe kółko - kształt}
\Clue{14}{}{kawałek materiału, pokryty substancją samoprzylepną, stosowany na rany}
\Clue{15}{}{szerokie perspektywy, bardzo duży wybór czegoś}
\Clue{18}{}{działanie dla dobra kogoś lub czegoś, zazwyczaj znajdujących się w trudnej sytuacji}
\Clue{22}{}{kompozytor niemiecki (1810-1849); opera komiczna 'Wesołe kumoszki z Windsoru'}
\Clue{23}{}{dyrygent amerykański (1895-1967); kierował m.in. orkiestrą symfoniczną BBC}
\Clue{24}{}{element elektroniczny, którego zasada działania oparta jest na zjawisku piezoelektrycznym w krysztale kwarcu}
\Clue{32}{}{wspólnik spółki akcyjnej lub spółki komandytowo-akcyjnej będący posiadaczem akcji wyemitowanych przez taką spółkę}
\Clue{33}{}{rodzaj urządzenia, najczęściej komputera, który ma służyć prezentowaniu lub wyszukiwaniu informacji}
\Clue{34}{}{cyfra rzymska odpowiadająca 500}
\Clue{35}{}{jezioro w Taplanii na Wyżynie Wschodnio -Afrykańskiej}
\Clue{36}{}{element struktury, cienka warstwa stałej substancji często podobna do łuski zwierzęcej - kostnej lub rogowej płytki, która wraz z innymi łuskami tworzy pokrywę ciała}
\Clue{37}{}{zielony lub brunatny owad tropikalny z rzędu patyczaków, roślinożerny}
\Clue{38}{}{miejsce, w którym odbywa się detaliczna sprzedaż towarów}
\Clue{39}{}{cecha koloru, który jest nazywany żywym}
\Clue{40}{}{zatoka Oceanu Atlantyckiego o północnych wybrzeży Islandii}
\Clue{41}{}{msza święta, uroczysta celebracja Ostatniej Wieczerzy, uczty paschalnej, którą według wierzeń chrześcijan Jezus spożył przed swoją zbawczą męką, śmiercią i zmartwychwstaniem}
\Clue{42}{}{w chemii: symbol cyrkonu}
\Clue{43}{}{miasto w Ghanie, główny port morski kraju nad Zatoką Gwinejską, ważny ośrodek przemysłowy kraju}\end{PuzzleClues}

\begin{PuzzleClues}{\textbf{Pionowe}\\}\Clue{1}{}{histon przynależący do rodziny małych, zasadowych, białek histonowych, których podstawową rolą jest utrzymanie zwartej struktury chromatyny z zachowaniem odpowiednich odległości między poszczególnymi nukleosomami}
\Clue{2}{}{rzeka}
\Clue{3}{}{Sandersonia - rodzaj rośliny z rodziny zimowitowatych}
\Clue{4}{}{nieświadome przeoczenie czegoś, zapomnienie o czymś, niezauważenie czegoś}
\Clue{5}{}{jezioro w Bułgarii, w pobliżu Plovdiv}
\Clue{7}{}{zawartość dużego garnka; tyle, ile się mieści w garze}
\Clue{8}{}{słowacki kompozytor i pedagog (1906-1984); utwory symfoniczne, kameralne, kantaty, pieśni}
\Clue{9}{}{komisja organizująca zawody sprotów wodnych}
\Clue{10}{}{Crypturellus brevirostris - gatunek ptaka z rodziny kusaczy (Tinamidae)}
\Clue{11}{}{owoc (z botanicznego punktu widzenia jest to owoc szupinkowy swoistego typu) dzikiej róży}
\Clue{12}{}{jeden ze znaków zodiaku (tzw. tropikalnego, astrologicznego)}
\Clue{13}{}{trójstrunowy, dawny instrument muzyczny, rodzaj lutni o strunach szarpanych kostką używany już w starożytnej Grecji}
\Clue{14}{}{dzwonnica wybudowana przy cerkwi}
\Clue{15}{}{kilobajt - jednostka używana w informatyce do określenia ilości informacji lub wielkości pamięci}
\Clue{16}{}{używany do pomiaru światła ciał niebieskich}
\Clue{17}{}{w branży filmowej: główny inwestor produkcji filmowej zatrudniający wszystkie potrzebne osoby do jego produkcji i zapewniający sprzęt oraz środki materialne}
\Clue{19}{}{muzułmańska szkoła, która kiedyś mieściła się w meczecie, dziś funkcjonuje samodzielnie}
\Clue{20}{}{mechanik węgierski (1909-1987); prace z metrologii wielkich częstotliwości}
\Clue{21}{}{czyn szalony, wychodzący poza normy postępowania}
\Clue{24}{}{język afrykański}
\Clue{25}{}{wytwór człowieka, często będący przedmiotem oferty handlowej; najczęściej występuje jako rzeczownik zbiorowy}
\Clue{26}{}{kompozytor włoski (1533-1599); wybitny twórca madrygałów}
\Clue{27}{}{element architektoniczny; wewnętrzne trójkątne pole frontonu, gładkie lub wypełnione rzeźbami, stanowiące charakterystyczny element monumentalnych budowli w różnych stylach (zwykle kojarzony ze stylem klasycznym, ale też np. z gotykiem)}
\Clue{28}{}{miasto w Chinach, stolica autonomicznego regionu Tybet}
\Clue{29}{}{sen, spanie, drzemka}
\Clue{30}{}{międzywojenny, turystyczno-szkolny, polski samolot}
\Clue{31}{}{rodzaj kieliszka do degustacji, który ma zwężające się ku górze ścianki, które pomagają dobrze poznać aromat trunku}
\Clue{32}{}{grupa nuklidów (izotopów) danego pierwiastka}\end{PuzzleClues}\newpage%\section*{Krzyżówka 108}

\noindent\begin{Puzzle}{16}{28}|*	|*	|*	|*	|[1][S]\drarr	|b	|u	|r	|z	|y	|k	|[][,]{ }	|m	|a	|ł	|y	|*	|.
|*	|[2][S]\drarr	|p	|o	|m	|u	|r	|n	|i	|k	|*	|[3][S]\darr	|*	|*	|[4][S]\darr	|*	|*	|.
|*	|t	|[5][S]\rarr	|h	|a	|r	|a	|s	|*	|[6][S]\darr	|[7][S]\darr	|k	|*	|*	|w	|*	|*	|.
|*	|e	|*	|*	|r	|*	|[8][S]\darr	|[9][S]\darr	|*	|t	|c	|a	|[10][S]\rarr	|c	|e	|s	|*	|.
|*	|r	|[11][S]\darr	|[12][S]\drarr	|c	|h	|e	|b	|z	|a	|u	|r	|*	|*	|s	|*	|*	|.
|*	|m	|w	|c	|z	|[13][S]\darr	|r	|i	|*	|g	|k	|o	|*	|*	|e	|*	|*	|.
|*	|o	|e	|o	|a	|a	|d	|e	|*	|*	|r	|t	|[14][S]\darr	|*	|l	|*	|*	|.
|*	|l	|l	|n	|k	|m	|e	|l	|*	|[15][S]\drarr	|z	|e	|k	|s	|*	|*	|*	|.
|*	|o	|u	|*	|*	|b	|k	|i	|*	|w	|y	|n	|a	|[16][S]\darr	|[17][S]\darr	|[18][S]\darr	|[19][S]\darr	|.
|[20][S]\rarr	|k	|r	|e	|p	|a	|*	|k	|*	|i	|c	|*	|g	|a	|c	|p	|ś	|.
|*	|a	|*	|*	|*	|*	|*	|*	|*	|e	|a	|[21][S]\darr	|e	|l	|h	|a	|m	|.
|[22][S]\drarr	|c	|h	|r	|u	|ś	|c	|i	|e	|l	|*	|s	|l	|l	|o	|z	|i	|.
|p	|j	|[23][S]\drarr	|p	|r	|z	|y	|s	|ł	|o	|n	|a	|*	|e	|r	|u	|e	|.
|e	|a	|w	|*	|[24][S]\rarr	|l	|i	|m	|a	|k	|*	|l	|*	|l	|o	|r	|c	|.
|r	|*	|i	|*	|[25][S]\darr	|*	|*	|[26][S]\rarr	|p	|r	|z	|e	|g	|u	|b	|*	|h	|.
|s	|[27][S]\darr	|e	|[28][S]\rarr	|k	|r	|o	|k	|*	|o	|*	|p	|*	|j	|a	|[29][S]\darr	|[][,]{ }	|.
|*	|t	|l	|[30][S]\rarr	|o	|f	|f	|s	|e	|t	|*	|*	|[31][S]\darr	|a	|[][,]{ }	|k	|p	|.
|[32][S]\drarr	|u	|k	|ł	|a	|d	|[][,]{ }	|h	|e	|n	|o	|n	|a	|*	|g	|s	|r	|.
|g	|a	|o	|*	|t	|*	|[33][S]\darr	|*	|*	|o	|*	|*	|d	|*	|e	|e	|z	|.
|l	|l	|l	|*	|i	|*	|o	|*	|*	|ś	|*	|[34][S]\darr	|a	|*	|n	|n	|e	|.
|i	|e	|u	|*	|*	|*	|k	|*	|*	|ć	|*	|s	|m	|[35][S]\darr	|e	|o	|z	|.
|s	|t	|d	|[36][S]\rarr	|d	|e	|l	|t	|a	|*	|*	|z	|ó	|a	|t	|f	|[][,]{ }	|.
|t	|a	|*	|[37][S]\rarr	|b	|i	|e	|r	|u	|t	|*	|ó	|w	|u	|y	|i	|ł	|.
|a	|*	|*	|*	|*	|[38][S]\rarr	|p	|h	|o	|b	|o	|s	|*	|r	|c	|l	|z	|.
|*	|*	|*	|*	|[39][S]\rarr	|b	|i	|s	|k	|w	|i	|t	|*	|e	|z	|i	|y	|.
|*	|*	|*	|[40][S]\rarr	|s	|f	|e	|r	|a	|[][,]{ }	|d	|y	|s	|o	|n	|a	|*	|.
|*	|[41][S]\rarr	|k	|r	|a	|j	|c	|a	|r	|*	|*	|*	|*	|l	|a	|*	|*	|.
|[42][S]\rarr	|i	|n	|d	|y	|k	|*	|*	|*	|*	|*	|*	|*	|a	|*	|*	|*	|.
|*	|*	|*	|*	|*	|*	|*	|*	|*	|*	|*	|*	|*	|*	|*	|*	|*	|.\end{Puzzle}

\newpage

\begin{PuzzleClues}{\textbf{Poziome}\\}\Clue{1}{}{Puffinus assimilis - gatunek ptaka z rodziny burzykowatych (Procellariidae)}
\Clue{2}{}{Parietaria - rodzaj jednorocznych roślin zielnych lub bylin z rodziny pokrzywowatych (Urticaceae Juss.)}
\Clue{5}{}{rodzaj dawnej, lekkiej i cienkiej (używanej do szycia np. letnich okryć wierzchnich) wełnianej tkaniny, której nazwa pochodziła od miejscowości Arras we Francji}
\Clue{10}{}{dźwięk „c” obniżony o pół tonu}
\Clue{12}{}{Chebsaurus - rodzaj zauropoda żyjący w epoce środkowej jury na terenach północnej Afryki, opisany na podstawie skamieniałości młodego osobnika}
\Clue{15}{}{okrzyk wartownika oznaczający alarm}
\Clue{20}{}{kauczuk naturalny sprzedawany jako półprodukt, w formie arkuszy}
\Clue{22}{}{DERKACZ, ŁYSKA, KOKOSZKA WODNA; przedstawiciel ptaków wodno-błotnych; 132 gatunki}
\Clue{23}{}{to, co zakrywa, najczęściej jakieś zjawisko, np. przesłona z chmur, z dymu}
\Clue{24}{}{wiąz szypułkowy; liściaste drzewo o wysokości do 30 m}
\Clue{26}{}{połączenie dwóch elementów umożliwiające ich względny obrót}
\Clue{28}{}{krocze, część ciała pomiędzy nogami}
\Clue{30}{}{maszyna drukująca techniką offset}
\Clue{32}{}{układ dwóch równań różnicowych nieliniowych przedstawionych przez francuskiego astronoma Michela Hénona}
\Clue{36}{}{jedna z gwiazd w gwiazdozbiorze Cefeusza (8)}
\Clue{37}{}{okres w historii Polski, kiedy rządy, jako prezydent RP, przewodniczący KC PZPR i prezes Rady Ministrów PRL, sprawował Bolesław Bierut}
\Clue{38}{}{FOBOS}
\Clue{39}{}{wstępnie spieczona masa ceramiczna}
\Clue{40}{}{hipotetyczna megastruktura, opisana po raz pierwszy w 1959 roku przez amerykańskiego fizyka i futurologa Freemana Dysona, która całkowicie otaczałaby gwiazdę, umożliwiając wykorzystanie energii wypromieniowywanej przez gwiazdę na użytek cywilizacji}
\Clue{41}{}{grajcar; pręt do wykręcania przedmiotów uwięzionych np. w rurach}
\Clue{42}{}{dość duży ptak hodowlany lub łowny z rodzaju Meleagris, ceniony ze względu na delikatne mięso}\end{PuzzleClues}

\begin{PuzzleClues}{\textbf{Pionowe}\\}\Clue{1}{}{zając z wiosennego miotu}
\Clue{2}{}{wykrywanie obiektów i określanie ich położenia przy użyciu promieniowania cieplnego (podczerwonego)}
\Clue{3}{}{organiczny związek chemiczny, rozbudowany przestrzennie węglowodór nienasycony zawierający 40 atomów węgla; pomarańczowożółty organiczny barwnik roślinny, występujący między innymi w wielu warzywach i owocach}
\Clue{4}{}{miasto w Niemczech (Nadrenia Płn., Westfalia), port przy ujściu rzeki Lippe i zbiegu kanału Wesel Datteln z Renem}
\Clue{6}{}{w informatyce - znak lub słowo, które jest przyporządkowane jakiejś informacji}
\Clue{7}{}{termin z dziedziny cukrownictwa, oznaczający gęstą mieszaninę kryształów cukru i syropu międzykryształowego}
\Clue{8}{}{zatoka Morza Marmara u wybrzeży tureckim}
\Clue{9}{}{BIRKUT; drapieżny ptak chroniony o długości około 1 m, upierzenie brązowo-płowe, ogon biały; pobrzeża wód Eurazji, strefy północnej i umiarkowanej}
\Clue{11}{}{welurowy but}
\Clue{12}{}{określenie wykonawcze}
\Clue{13}{}{przedłużenie bocznych naw w kościele}
\Clue{14}{}{argentyński kompozytor i dyrygent ur. w 1931r., utwory kameralne wokalno-instrumentalne, sceniczne, elektroniczne}
\Clue{15}{}{kategoria gramatyczna czasownika, informująca o wielokrotnym przebiegu działania}
\Clue{16}{}{przejęte z judaizmu wezwanie do wiernych o złożenie hołdu Bogu wyraz radości i triumfu zaczynający lub kończący psalmy i pieśni chrześcijańskie}
\Clue{17}{}{choroba wrodzona powstała na skutek mutacji w obrębie genu lub genów}
\Clue{18}{}{zaostrzony koniec kotwicy}
\Clue{19}{}{udawanie szczęścia, wmawianie sobie i innym, że wszystko jest w porządku, kiedy nie jest}
\Clue{21}{}{proszek salepowy - sproszkowane bulwy storczyka męskiego; rodzaj mączki, proszku, z którego można przygotować napój}
\Clue{22}{}{mieszkaniec Persji, od 1935 roku przemianowanej na Iran}
\Clue{23}{}{bardzo wysoki człowiek; słowo żartobliwe}
\Clue{25}{}{ostronos rudy - południowoamerykański ssak z rodziny szopów, wszystkożerny}
\Clue{27}{}{określenie czynności higienicznej, której częścią jest mycie się i przebieranie}
\Clue{29}{}{dewiacja seksualna polegająca na kontaktach seksualnych z nieznajomymi, bez chęci nawiązania z nimi relacji emocjonalnych}
\Clue{31}{}{wieś w Polsce położona w województwie lubelskim, w powiecie zamojskim, w gminie Adamów}
\Clue{32}{}{nicień pasożytujący często w jelicie człowieka}
\Clue{33}{}{żelazne sidła zastawiane na grubą zwierzynę przez kłusowników}
\Clue{34}{}{szósty dzień (najczęściej bieżącego lub przyszłego) miesiąca}
\Clue{35}{}{poświata}\end{PuzzleClues}\newpage%\section*{Krzyżówka 109}

\noindent\begin{Puzzle}{19}{25}|*	|*	|*	|*	|*	|*	|*	|*	|*	|[1][S]\drarr	|t	|r	|u	|m	|n	|i	|a	|k	|*	|*	|.
|*	|*	|*	|[2][S]\darr	|*	|[3][S]\rarr	|h	|e	|k	|s	|a	|g	|r	|a	|m	|*	|[4][S]\darr	|*	|*	|*	|.
|[5][S]\rarr	|c	|u	|g	|*	|[6][S]\rarr	|d	|e	|k	|o	|m	|u	|n	|i	|z	|a	|c	|j	|a	|*	|.
|*	|[7][S]\rarr	|p	|r	|z	|e	|c	|i	|w	|s	|t	|o	|k	|*	|*	|[8][S]\darr	|h	|*	|[9][S]\darr	|*	|.
|*	|*	|*	|*	|*	|*	|*	|[10][S]\rarr	|i	|n	|k	|u	|b	|a	|c	|j	|a	|*	|p	|*	|.
|[11][S]\rarr	|s	|o	|l	|i	|d	|a	|r	|n	|o	|ś	|ć	|*	|[12][S]\darr	|*	|a	|o	|*	|o	|*	|.
|*	|[13][S]\darr	|*	|*	|*	|[14][S]\rarr	|s	|c	|h	|w	|a	|b	|*	|p	|[15][S]\darr	|j	|s	|*	|s	|*	|.
|[16][S]\drarr	|d	|z	|i	|e	|j	|o	|p	|i	|s	|*	|*	|[17][S]\darr	|o	|w	|e	|*	|*	|ł	|*	|.
|a	|y	|*	|*	|*	|[18][S]\darr	|[19][S]\drarr	|p	|i	|k	|*	|*	|s	|d	|r	|c	|*	|[20][S]\darr	|u	|*	|.
|w	|s	|*	|*	|[21][S]\drarr	|p	|i	|l	|n	|i	|k	|*	|a	|a	|z	|z	|[22][S]\darr	|m	|c	|*	|.
|a	|c	|*	|[23][S]\darr	|n	|i	|n	|[24][S]\darr	|[25][S]\darr	|*	|*	|*	|r	|t	|a	|n	|w	|a	|h	|*	|.
|n	|y	|[26][S]\drarr	|m	|i	|o	|d	|o	|w	|y	|[][,]{ }	|m	|i	|e	|s	|i	|ą	|c	|*	|*	|.
|g	|p	|t	|e	|e	|ł	|i	|d	|o	|*	|*	|[27][S]\darr	|*	|k	|k	|c	|ż	|e	|*	|*	|.
|a	|l	|r	|d	|m	|u	|e	|n	|l	|*	|*	|l	|[28][S]\darr	|[][,]{ }	|l	|a	|*	|d	|*	|*	|.
|r	|i	|u	|g	|o	|n	|[][,]{ }	|a	|a	|*	|*	|i	|s	|d	|i	|*	|*	|o	|*	|*	|.
|d	|n	|m	|i	|ż	|*	|w	|w	|*	|*	|*	|n	|c	|r	|w	|*	|*	|ń	|*	|*	|.
|a	|a	|n	|d	|n	|[29][S]\darr	|s	|i	|[30][S]\darr	|*	|[31][S]\darr	|e	|e	|o	|o	|*	|*	|c	|*	|*	|.
|*	|*	|a	|i	|o	|d	|c	|a	|p	|*	|d	|a	|n	|g	|ś	|*	|[32][S]\darr	|z	|*	|*	|.
|*	|*	|*	|a	|ś	|r	|h	|c	|e	|*	|i	|r	|i	|o	|ć	|*	|k	|y	|*	|*	|.
|*	|*	|*	|*	|ć	|a	|o	|z	|r	|*	|o	|n	|c	|w	|*	|*	|l	|k	|*	|*	|.
|[33][S]\rarr	|h	|a	|k	|*	|b	|d	|*	|ć	|[34][S]\rarr	|p	|o	|z	|y	|c	|j	|a	|*	|*	|*	|.
|[35][S]\drarr	|f	|l	|a	|m	|i	|n	|g	|*	|*	|t	|ś	|n	|*	|*	|*	|s	|*	|*	|*	|.
|g	|[36][S]\rarr	|z	|d	|a	|n	|i	|e	|*	|*	|e	|ć	|o	|*	|[37][S]\rarr	|m	|a	|*	|*	|*	|.
|r	|*	|*	|*	|*	|a	|e	|[38][S]\rarr	|ż	|u	|r	|*	|ś	|*	|*	|*	|*	|*	|*	|*	|.
|a	|[39][S]\rarr	|n	|i	|t	|*	|*	|*	|*	|*	|*	|*	|ć	|*	|*	|*	|*	|*	|*	|*	|.
|*	|*	|*	|*	|*	|*	|*	|*	|*	|*	|*	|*	|*	|*	|*	|*	|*	|*	|*	|*	|.\end{Puzzle}

\newpage

\begin{PuzzleClues}{\textbf{Poziome}\\}\Clue{1}{}{złej jakości pantofel damski}
\Clue{3}{}{sześcioramienna gwiazda złożona z dwóch zachodzących na siebie pionowo trójkątów równoramiennych (najczęściej równobocznych) odwróconych względem siebie}
\Clue{5}{}{zaprzęg konny składających się z sześciu (rzadziej czterech) koni dobranych maścią, wzrostem lub innymi cechami}
\Clue{6}{}{proces społeczno-polityczny polegający na likwidacji struktur komunistycznych w krajach byłego ZSRR}
\Clue{7}{}{stok przeciwległy innemu stokowi}
\Clue{10}{}{w hodowli kurcząt okres sztucznego ich wylęgu w inkubatorze}
\Clue{11}{}{to, że ktoś jest solidarny w ramach jakiejś grupy, wobec kogoś; zgoda i wzajemne wsparcie w gronie osób, które współdziałają, bo łączą je wspólne interesy, poglądy, wartości, dążenia}
\Clue{14}{}{(1792-1850), późnoromantyczny poeta niemiecki, popularyzował mity starożytne i ludowe podania}
\Clue{16}{}{kronika, chronograf}
\Clue{19}{}{karta koloru pik (¦)}
\Clue{21}{}{narzędzie służące do piłowania, czyli skrawania z obrabianej powierzchni cienkiej warstwy o grubości od 0,01 do 1 mm}
\Clue{26}{}{pierwszy miesiąc po ślubie spędzany we dwoje}
\Clue{33}{}{metalowy klin, który po wbiciu w szczelinę skalną, może być używany jako punkt asekuracyjny dla wspinacza lub w speleologii}
\Clue{34}{}{miejsce na liście, jeden z punktów spisu}
\Clue{35}{}{ptak wodny; poszczególne gatunki tego ptaka klasyfikowane są w taksonomii biologicznej w obrębie rodziny flamingów (Phoenicopteridae)}
\Clue{36}{}{w logice wypowiedź, która stwierdza określony stan rzeczy}
\Clue{37}{}{jednostka prędkości odnosząca się do obiektów poruszających się w płynie lub poruszających się płynów}
\Clue{38}{}{żurek - zakwaszona zupa mączna}
\Clue{39}{}{przestarzała, nie należąca do układu SI jednostka luminancji}\end{PuzzleClues}

\begin{PuzzleClues}{\textbf{Pionowe}\\}\Clue{1}{}{architekt (1880-1939), profesor Politechniki Warszawskiej, kościół Św. Rocha w Białymstoku}
\Clue{2}{}{skrót nazwy gran - jednostki masy}
\Clue{4}{}{materia, z której ukonstytuował się kosmos}
\Clue{8}{}{smażona potrawa z jaj, najczęściej kurzych, podawana z dodatkiem boczku lub bekonu}
\Clue{9}{}{cecha człowieka, który jest słuchany, poważany; także w związku wyrazowymmieć posłuch (mieć poważanie, autorytet)}
\Clue{12}{}{podatek, którego przedmiotem opodatkowania są środki transportu}
\Clue{13}{}{rodzaj sportu}
\Clue{15}{}{cecha czegoś, co jest wrzaskliwe - głośne (np. ulicy, atmosfery)}
\Clue{16}{}{środowisko artystów, grupa osób, których twórczość jest nowatorska, nietradycyjna i któe przez to wyznaczają nowy kierunek artystycznych działań}
\Clue{17}{}{właściwie Jadwiga Schayer; światowej sławy śpiewaczka (1886-1968); sopran koloraturowy, występy m.in. w La Scali}
\Clue{18}{}{eurazjatycka bylina używana jako lek, przyprawa - bylica piołunowa}
\Clue{19}{}{wschodnia część Indii}
\Clue{20}{}{mieszkaniec starożytnej Macedonii, człowiek pochodzący z antycznej Macedonii}
\Clue{21}{}{zauważalny brak działania lub działanie w jakiś (określony) sposób niepełne}
\Clue{22}{}{gad, który żyje w wodzie lub na lądzie i ma wydłużone ciało, które przemieszcza pełzając}
\Clue{23}{}{miasto w Rumunii (okręg Konstanca); przemysł maszynowy, materiałów budowlanych, spożywczy}
\Clue{24}{}{osoba, która coś odnawia, odświeża}
\Clue{25}{}{poglądy, uczucia człowieka, które są podstawą do podejmowania decyzji}
\Clue{26}{}{trapezoidalne lub prostokątne pole na boisku do koszykówki, na którym zawodnik może przebywać nie dłużej niż trzy sekundy w czasie gry}
\Clue{27}{}{to, że coś jest linearne - przedstawione poprzez linie}
\Clue{28}{}{to, że coś jest charakterystyczne dla występów na scenie}
\Clue{29}{}{konstrukcja z równolegle ustawionych żerdzi stanowiąca boczną część wozu drabiniastego}
\Clue{30}{}{stroma górska ścieżka}
\Clue{31}{}{przeziernik; element celownika umożliwiający kontrolę wzrokową położenia przyrządu względem obserwowanego obiektu}
\Clue{32}{}{dział w szkole artystycznej, przedmiot poświęcony danej specjalizacji}
\Clue{35}{}{współistnienie, zwykle: pełne napięć}\end{PuzzleClues}\newpage%\section*{Krzyżówka 110}

\noindent\begin{Puzzle}{19}{21}|*	|*	|*	|*	|*	|*	|[1][S]\drarr	|b	|r	|y	|g	|a	|d	|i	|e	|r	|*	|*	|[2][S]\darr	|*	|.
|*	|*	|[3][S]\drarr	|f	|o	|r	|s	|y	|c	|j	|a	|*	|[4][S]\darr	|*	|[5][S]\drarr	|k	|u	|n	|a	|*	|.
|*	|*	|m	|*	|[6][S]\drarr	|s	|z	|k	|r	|a	|b	|i	|k	|*	|r	|[7][S]\darr	|*	|[8][S]\darr	|z	|*	|.
|*	|*	|o	|*	|b	|*	|k	|[9][S]\drarr	|t	|u	|n	|e	|l	|*	|a	|z	|*	|d	|a	|*	|.
|*	|*	|t	|[10][S]\darr	|o	|*	|ł	|m	|*	|[11][S]\darr	|[12][S]\drarr	|b	|a	|j	|k	|a	|*	|o	|r	|*	|.
|*	|*	|y	|g	|n	|[13][S]\darr	|o	|i	|*	|l	|b	|*	|r	|*	|*	|w	|*	|n	|a	|*	|.
|*	|*	|l	|o	|g	|r	|*	|l	|*	|a	|l	|[14][S]\drarr	|o	|s	|t	|a	|d	|e	|*	|*	|.
|*	|*	|e	|s	|o	|e	|[15][S]\darr	|l	|*	|k	|o	|k	|w	|*	|*	|ł	|[16][S]\drarr	|g	|w	|*	|.
|*	|*	|k	|z	|*	|n	|g	|e	|*	|t	|m	|u	|n	|*	|*	|a	|l	|a	|*	|*	|.
|*	|*	|*	|y	|[17][S]\rarr	|o	|b	|r	|z	|y	|d	|ł	|o	|ś	|ć	|*	|a	|l	|[18][S]\darr	|*	|.
|*	|*	|*	|z	|*	|m	|p	|*	|[19][S]\darr	|d	|a	|a	|ś	|*	|*	|*	|n	|*	|d	|*	|.
|*	|*	|[20][S]\rarr	|m	|p	|a	|*	|*	|ś	|*	|h	|n	|ć	|*	|*	|[21][S]\darr	|d	|*	|o	|*	|.
|*	|*	|*	|*	|*	|*	|*	|*	|n	|*	|l	|*	|*	|*	|*	|k	|i	|*	|p	|*	|.
|*	|*	|*	|*	|*	|[22][S]\rarr	|d	|z	|i	|k	|*	|[23][S]\rarr	|r	|o	|u	|e	|n	|*	|ł	|*	|.
|*	|*	|*	|[24][S]\drarr	|p	|a	|p	|i	|e	|r	|[][,]{ }	|g	|a	|z	|e	|t	|o	|w	|y	|*	|.
|*	|[25][S]\rarr	|e	|m	|b	|r	|i	|o	|g	|e	|n	|e	|z	|a	|*	|*	|*	|*	|w	|*	|.
|*	|[26][S]\rarr	|w	|o	|d	|a	|[][,]{ }	|g	|u	|l	|a	|r	|d	|o	|w	|a	|*	|*	|*	|*	|.
|*	|*	|*	|s	|*	|[27][S]\rarr	|ż	|ó	|ł	|w	|[][,]{ }	|l	|a	|m	|p	|a	|r	|c	|i	|*	|.
|*	|*	|*	|p	|*	|[28][S]\rarr	|p	|i	|a	|s	|k	|o	|w	|i	|e	|c	|*	|*	|*	|*	|.
|*	|[29][S]\rarr	|n	|a	|s	|a	|d	|a	|*	|*	|*	|*	|*	|*	|*	|*	|*	|*	|*	|*	|.
|[30][S]\rarr	|k	|a	|n	|i	|a	|[][,]{ }	|j	|a	|s	|n	|o	|b	|a	|r	|k	|o	|w	|a	|*	|.
|*	|*	|*	|*	|*	|*	|*	|*	|*	|*	|*	|*	|*	|*	|*	|*	|*	|*	|*	|*	|.\end{Puzzle}

\newpage

\begin{PuzzleClues}{\textbf{Poziome}\\}\Clue{1}{}{stopień wojskowy pomiędzy pułkownikiem a generałem}
\Clue{3}{}{parkowy krzew ozdobny z Dalekiego Wschodu o złocistożółtych kwiatach rozwijających się wczesną wiosną}
\Clue{5}{}{wspólna nazwa kilku gatunków niewielkich, drapieżnych zwierząt z rodzaju Martes z rodziny łasicowatych, m.in.: kuny domowej i kuny leśnej}
\Clue{6}{}{zdrobniale, pieszczotliwie o dziecku}
\Clue{9}{}{wyrobisko górnicze}
\Clue{12}{}{historyjka dla dzieci (także: sceniariusz filmu), opowieść zmyślona lub oparta na przekazach ustnych}
\Clue{14}{}{Adriaen (1610-85) malarz holenderski., uczeń Halsa; sceny z życia chłopów}
\Clue{16}{}{jednostka mocy równa 1000000000 watów}
\Clue{17}{}{wstręt, odraza, odczucie obrzydzenia}
\Clue{20}{}{jednostka ciśnienia, która jest równa milionowi paskali}
\Clue{22}{}{Sus scrofa - gatunek dużego lądowego ssaka łożyskowego z rzędu parzystokopytnych; jedyny przedstawiciel dziko żyjących świniowatych w Europie, przodek świni domowej}
\Clue{23}{}{miasto we Francji (Normandia) nad Sekwaną ośrodek administracyjny departamentu Seine-Maritime}
\Clue{24}{}{rodzaj papieru przeznaczonego do druku typograficznego jednobarwnych gazet, paragonów, biletów i innych druków jednorazowych, na których nie pisze się atramentem}
\Clue{25}{}{rozwój zarodka, który rozpoczyna się w chwili zapłodnienia aż do momentu wydostania się zarodka z osłon jajowych (u jajorodnych) lub do opuszczenia układu rozrodczego samicy (urodzenie u żyworodnych)}
\Clue{26}{}{roztwór wodny octanu ołowiu o silnym działaniu ściągającym i skurczowym, stosowany w XVIII i XIX w. w kosmetyce i medycynie}
\Clue{27}{}{żółw plamisty, Psammobates pardalis - gatunek gada z rodziny żółwi lądowych, występujący w Afryce}
\Clue{28}{}{Arenaria - rodzaj roślin należący do rodziny goździkowatych obejmujący od ok. 160 do ponad 270 gatunków}
\Clue{29}{}{zespół organów mowy ponad krtanią}
\Clue{30}{}{Milvus migrans affinis - podgatunek ptaka wyróżniony w obrębie gatunku kania czarna (Milvus migrans)}\end{PuzzleClues}

\begin{PuzzleClues}{\textbf{Pionowe}\\}\Clue{1}{}{szklane naczynie (czasem o innym naczyniopodobnym wyrobie ze szkła, np. o kloszu od lampy)}
\Clue{2}{}{gatunek południoamerykańskiego lisa}
\Clue{3}{}{przyrząd sportowo-rekreacyjny, rodzaj dmuchanego rękawka, który zakłada się na każde z ramion i który pomaga w utrzymaniu się na powierzchni wody}
\Clue{4}{}{czystość, przezroczystość}
\Clue{5}{}{jeden ze znaków zodiaku}
\Clue{6}{}{rzadki gatunek afrykańskiej antylopy}
\Clue{7}{}{zapora przeciwpancerna budowana z drzew o średnicy ponad 25 cm, ściętych na wysokości 60-150 cm, zwalonych na krzyż i oplecionych drutem kolczastym, wzmocniona minami i fugasami}
\Clue{8}{}{tkanina wełniana z grubej przędzy zgrzebnej, z jasną osnową i ciemnym wątkiem używana głównie na płaszcze i ubrania sportowe}
\Clue{9}{}{Henry (1891-1980), pisarz amerykański, powieści, opowiadania autobiograficzne; „Kolos z Maroussi”}
\Clue{10}{}{ideologia, która powstała w krajach zachodnich w latach 60. XX wieku}
\Clue{11}{}{heterocykliczny związek organiczny z grupy estrów}
\Clue{12}{}{kompozytor szwedzki (1916-1968); symfonie, koncerty, utwory kameralne, opera}
\Clue{13}{}{pochlebna opinia na temat kogoś lub czegoś, dobra reputacja}
\Clue{14}{}{ssak z rodziny koniowatych, zamieszkuje pustynne obszary środkowej Azji}
\Clue{15}{}{kod ISO 4217 funta szterlinga}
\Clue{16}{}{włoski kompozytor i organista (1325-1397); najwybitniejszy przedstawiciel włoskiego ars nova; ballady, caccie, madrygały}
\Clue{18}{}{rzeka lub mniejszy ciek, który nie uchodzi bezpośrednio do zbiornika wodnego (morza, jeziora), ale do innego cieku}
\Clue{19}{}{biało-brązowy ptak z rzędu wróblowatych, do Polski zalatuje zimą; chroniona}
\Clue{21}{}{jednostka pływająca o napędzie żaglowym, posiadająca tylko jeden maszt (grot), oraz tylko jeden żagiel (również grot) - dowolnego typu ożaglowania skośnego}
\Clue{24}{}{używany poufale tytuł grzecznościowy odpowiadający dzisiejszemupan, skrót od mości pan}\end{PuzzleClues}


\newpage%\section*{Krzyżówka 1}

\noindent\begin{Puzzle}{22}{24}|*	|*	|*	|[1][S]\darr	|[2][S]\darr	|*	|*	|*	|*	|*	|*	|[3][S]\darr	|*	|*	|*	|*	|*	|[4][S]\darr	|*	|*	|[5][S]\darr	|*	|*	|.
|*	|*	|*	|m	|h	|[6][S]\darr	|*	|[7][S]\rarr	|b	|o	|b	|o	|w	|i	|a	|n	|k	|a	|*	|*	|h	|*	|*	|.
|*	|*	|*	|a	|o	|n	|[8][S]\drarr	|a	|z	|o	|t	|a	|n	|[][,]{ }	|g	|l	|i	|n	|u	|*	|u	|*	|[9][S]\darr	|.
|*	|[10][S]\darr	|[11][S]\rarr	|k	|r	|u	|p	|y	|*	|*	|*	|z	|[12][S]\darr	|[13][S]\rarr	|b	|a	|ł	|a	|m	|u	|t	|*	|b	|.
|*	|p	|[14][S]\darr	|a	|d	|r	|i	|*	|*	|[15][S]\drarr	|ł	|a	|s	|k	|u	|n	|*	|s	|*	|*	|n	|*	|ą	|.
|*	|ł	|m	|r	|a	|o	|ł	|*	|*	|z	|[16][S]\darr	|*	|z	|*	|*	|*	|*	|t	|*	|*	|i	|*	|k	|.
|*	|a	|i	|t	|*	|*	|k	|*	|*	|o	|k	|[17][S]\rarr	|p	|i	|ł	|k	|a	|r	|z	|y	|k	|i	|*	|.
|*	|s	|l	|*	|[18][S]\rarr	|n	|a	|d	|z	|o	|r	|c	|a	|[][,]{ }	|s	|ą	|d	|o	|w	|y	|*	|*	|*	|.
|[19][S]\drarr	|k	|a	|g	|e	|l	|*	|*	|*	|*	|y	|*	|d	|*	|[20][S]\rarr	|m	|u	|f	|k	|a	|*	|*	|*	|.
|g	|o	|[][,]{ }	|[21][S]\darr	|*	|[22][S]\drarr	|f	|r	|u	|s	|t	|r	|a	|c	|j	|a	|*	|a	|*	|*	|*	|*	|*	|.
|ł	|s	|n	|h	|[23][S]\drarr	|p	|o	|l	|o	|n	|e	|z	|*	|[24][S]\rarr	|s	|e	|r	|*	|*	|*	|*	|*	|*	|.
|u	|z	|a	|o	|i	|u	|*	|*	|*	|[25][S]\rarr	|r	|u	|m	|i	|a	|n	|e	|k	|*	|*	|*	|*	|[26][S]\darr	|.
|p	|*	|[][,]{ }	|m	|n	|c	|*	|*	|[27][S]\rarr	|p	|i	|ą	|t	|a	|[][,]{ }	|c	|z	|ę	|ś	|ć	|*	|*	|p	|.
|k	|*	|g	|e	|t	|e	|[28][S]\darr	|*	|[29][S]\rarr	|b	|u	|r	|g	|e	|r	|*	|*	|*	|*	|*	|*	|*	|r	|.
|o	|*	|o	|r	|e	|k	|s	|*	|[30][S]\rarr	|s	|m	|u	|k	|w	|o	|w	|a	|t	|e	|*	|[31][S]\darr	|*	|z	|.
|w	|*	|d	|y	|r	|*	|o	|*	|*	|*	|[][,]{ }	|*	|*	|[32][S]\drarr	|r	|o	|m	|f	|o	|r	|d	|*	|y	|.
|a	|[33][S]\rarr	|z	|d	|r	|a	|d	|z	|i	|e	|c	|t	|w	|o	|*	|*	|*	|*	|*	|*	|o	|[34][S]\darr	|c	|.
|t	|*	|i	|a	|e	|[35][S]\darr	|a	|[36][S]\drarr	|z	|j	|a	|w	|i	|s	|k	|o	|*	|*	|*	|*	|m	|r	|h	|.
|o	|*	|n	|*	|g	|h	|*	|m	|*	|*	|ł	|*	|[37][S]\rarr	|ł	|u	|s	|k	|a	|*	|*	|o	|o	|a	|.
|ś	|*	|ę	|[38][S]\rarr	|n	|o	|g	|a	|*	|*	|k	|*	|[39][S]\rarr	|o	|k	|r	|e	|s	|*	|*	|w	|d	|c	|.
|ć	|*	|*	|*	|u	|n	|[40][S]\rarr	|d	|y	|s	|o	|n	|a	|n	|s	|*	|*	|*	|*	|*	|i	|r	|i	|.
|*	|*	|*	|*	|m	|g	|*	|d	|*	|*	|w	|*	|[41][S]\rarr	|k	|a	|ł	|k	|a	|n	|*	|n	|i	|e	|.
|*	|*	|*	|*	|*	|z	|*	|e	|*	|*	|e	|[42][S]\rarr	|w	|a	|t	|e	|r	|s	|z	|t	|a	|g	|*	|.
|*	|[43][S]\rarr	|d	|z	|i	|e	|l	|n	|i	|k	|*	|*	|*	|*	|*	|*	|*	|*	|*	|*	|*	|o	|*	|.
|[44][S]\rarr	|b	|a	|e	|z	|*	|*	|*	|*	|*	|*	|*	|*	|*	|*	|*	|*	|*	|*	|*	|*	|*	|*	|.\end{Puzzle}

\newpage

\begin{PuzzleClues}{\textbf{Poziome}\\}\Clue{7}{}{mieszkanka Bobowej}
\Clue{8}{}{sól kwasu azotowego i glinu na III stopniu utlenienia}
\Clue{11}{}{staropolska nazwa kaszy}
\Clue{13}{}{uwodziciel - mężczyzna, który uwodzi kobiety}
\Clue{15}{}{nadrzewny lub naziemny ssak z rodziny łasz}
\Clue{17}{}{gra w piłkarzyki - przedmiot}
\Clue{18}{}{organ postępowania upadłościowego z możliwością zawarcia układu, w którym ustanowiono zarząd własny upadłego lub postępowania naprawczego, ustanawiany przez sąd; funkcjonariusz publiczny}
\Clue{19}{}{argentyński kompozytor i dyrygent ur. w 1931r., utwory kameralne wokalno-instrumentalne, sceniczne, elektroniczne}
\Clue{20}{}{mała mufa; łącznik rur}
\Clue{22}{}{zespół przykrych emocji związanych z niemożliwością realizacji potrzeby lub osiągnięcia określonego celu}
\Clue{23}{}{model samochodu osobowego produkowany przez Fabrykę Samochodów Osobowych w Warszawie od 3 maja 1978 roku do 22 kwietnia 2002 roku}
\Clue{24}{}{porcja sera, zazwyczaj kostka, ale też opakowanie; określona ilość sera}
\Clue{25}{}{roślina zielna ze złożonych, aromatyczny chwast polny, uprawiany także jako roślina lecznicza}
\Clue{27}{}{inaczej: jedna piąta, jedna z pięciu części czegoś podzielnego}
\Clue{29}{}{kanapka z miękkiej bułki z wieloma dodatkami w środku, z których to dodatków najbardziej charakterystycznym dla tej potrawy jest gruby kotlet o takiej samej nazwie}
\Clue{30}{}{Scoliidae - rodzina owadów z grupy żądłówek}
\Clue{32}{}{miasto w Anglii w hrabstwie Essex na płn.-wsch. od Londynu}
\Clue{33}{}{cecha człowieka, który jest zdradziecki}
\Clue{36}{}{widzenie senne lub mara}
\Clue{37}{}{element struktury, cienka warstwa stałej substancji często podobna do łuski zwierzęcej - kostnej lub rogowej płytki, która wraz z innymi łuskami tworzy pokrywę ciała}
\Clue{38}{}{część jakiegoś sprzętu, która służy jako wspornik, podstawa}
\Clue{39}{}{miesiączka}
\Clue{40}{}{rym niedokładny, nie obejmuje przedostatniej akcentowanej samogłoski wersu}
\Clue{41}{}{lekka, okrągła, silnie wypukła tarcza, typowa dla krajów wschodnich, takich jakich Persja bądź Turcja, znana także w Polsce oraz na Węgrzech}
\Clue{42}{}{lina łącząca nok bukszprytu z dziobem statku, usztywnia bukszpryt}
\Clue{43}{}{łącznik, dywiz, tiret - drukarski znak graficzny, krótka pozioma kreska umieszczona ponad podstawową linią pisma; służy do oznaczenia łączenia wyrazów złożonych, nazw podwójnych lub przeniesienia części wyrazu do następnego wiersza}
\Clue{44}{}{ur. w 1941 r., piosenkarka amerykańska; występuje przeciw wojnie, rasizmowi}\end{PuzzleClues}

\begin{PuzzleClues}{\textbf{Pionowe}\\}\Clue{1}{}{malarz austriacki (1840-84) kompozycje historyczne i alegoryczne o charakterze eklektycznym}
\Clue{2}{}{u Turków, Tatarów: wojsko, obóz wojskowy, orda}
\Clue{3}{}{teren o bardzo bujnej roślinności na obszarze pustyń i półpustyń}
\Clue{4}{}{rodzaj inwersji, przestawienie wyrazów w związku frazeologicznym}
\Clue{5}{}{pracownik huty szkła lub metalu, specjalista w dziedzinie hutnictwa}
\Clue{6}{}{miasto we Włoszech (Sardynia), ośrodek turystyczny}
\Clue{8}{}{przedmiot służący do gry i zabawy; sprężysta kula różnej wielkości. }
\Clue{9}{}{dziecko, zwłaszcza małe}
\Clue{10}{}{Ramaria Bonord. - rodzaj wielkoowocnikowych grzybów podstawkowych należący do rodziny siatkolistowatych (Gomphaceae), którego gatunkiem typowym jest koralówka czerwonowierzchołkowa (Ramaria botrytis)}
\Clue{12}{}{sportowa broń kolna, posiada klingę o trójkątnym przekroju i duży kosz, osłaniający dłoń zawodnika}
\Clue{14}{}{stosowana w krajach anglosaskich jednostka prędkości, oznaczana mph}
\Clue{15}{}{teren udostępniony odwiedzającym, na którym hodowane są zwierzęta, najczęściej pochodzące z różnych obszarów geograficznych}
\Clue{16}{}{metoda sprawdzania, czy nieskończony szereg liczbowy o nieujemnych wyrazach jest zbieżny}
\Clue{19}{}{pobłażliwie, żartobiliwie lub z ironią o czyjejś głupocie; cecha kogoś, kto postrzegany jest jako głupkowaty}
\Clue{21}{}{rapsod rozpowszechniający poezję Homera}
\Clue{22}{}{pyzaty mężczyzna, chłopiec}
\Clue{23}{}{hist. polit}
\Clue{26}{}{miejsce przy chacie, w najbliższym otoczeniu domu, obszar otaczający chatę}
\Clue{28}{}{wodorowęglan sodu, soda oczyszczona (NaHCO3) - nieorganiczny związek chemiczny z grupy wodorowęglanów, wodorosól kwasu węglowego i sodu, która stosowana jest głównie jako jeden ze składników proszku do pieczenia i dodatek do żywności regulujący pH}
\Clue{31}{}{trumna - specjalna skrzynia, w której spoczywa ciała zmarłego}
\Clue{32}{}{struktura rozwijająca się początkowo u nasady zalążka roślin nasiennych i w miarę rozwoju stopniowo go obrastająca}
\Clue{34}{}{ur. w 1902 r., kompozytor hiszpański; w twórczości nawiązuje do rodzinnego folkloru; koncerty, utwory kameralne, fortepianowe, gitarowe}
\Clue{35}{}{jezioro w Chinach na Nizinie Chińskiej}
\Clue{36}{}{jezioro w Panamie, z jeziora wypływa rzeka Chagres}\end{PuzzleClues}\newpage%\section*{Krzyżówka 2}

\noindent\begin{Puzzle}{23}{33}|*	|*	|*	|*	|*	|*	|*	|*	|*	|*	|*	|*	|*	|*	|*	|*	|*	|*	|*	|*	|*	|*	|[1][S]\darr	|*	|.
|*	|*	|*	|*	|*	|*	|*	|*	|*	|*	|*	|*	|*	|*	|*	|*	|[2][S]\darr	|*	|*	|[3][S]\darr	|*	|*	|w	|*	|.
|*	|*	|*	|*	|*	|*	|*	|*	|*	|*	|*	|*	|*	|*	|*	|*	|p	|*	|*	|p	|*	|*	|e	|*	|.
|*	|*	|*	|*	|*	|*	|*	|*	|*	|*	|[4][S]\rarr	|c	|z	|e	|c	|h	|o	|s	|ł	|o	|w	|a	|k	|*	|.
|*	|*	|*	|*	|*	|[5][S]\darr	|*	|*	|*	|*	|*	|*	|*	|*	|*	|*	|b	|[6][S]\darr	|*	|t	|*	|*	|t	|*	|.
|*	|*	|*	|*	|*	|t	|*	|*	|*	|*	|*	|*	|*	|*	|*	|*	|o	|s	|[7][S]\darr	|a	|*	|*	|o	|*	|.
|*	|[8][S]\darr	|*	|*	|*	|ł	|*	|*	|*	|*	|*	|*	|*	|*	|*	|*	|ż	|k	|r	|ż	|*	|*	|r	|*	|.
|*	|a	|*	|*	|*	|u	|*	|*	|*	|*	|*	|*	|*	|*	|*	|*	|n	|r	|e	|[][,]{ }	|*	|*	|[][,]{ }	|*	|.
|*	|n	|*	|*	|*	|s	|*	|[9][S]\darr	|*	|*	|*	|*	|*	|*	|*	|*	|o	|z	|z	|ż	|*	|*	|k	|*	|.
|*	|a	|*	|*	|*	|t	|*	|k	|*	|*	|*	|*	|*	|*	|*	|*	|ś	|y	|y	|r	|*	|*	|o	|*	|.
|*	|t	|*	|*	|*	|o	|*	|r	|*	|*	|[10][S]\darr	|*	|*	|*	|[11][S]\darr	|*	|ć	|p	|d	|ą	|*	|*	|l	|*	|.
|*	|o	|*	|*	|*	|g	|*	|o	|*	|[12][S]\rarr	|a	|m	|b	|o	|n	|a	|*	|i	|e	|c	|*	|*	|u	|*	|.
|*	|m	|*	|*	|*	|o	|*	|k	|*	|*	|n	|*	|*	|[13][S]\darr	|i	|*	|*	|c	|n	|y	|*	|*	|m	|*	|.
|*	|i	|*	|*	|*	|n	|*	|o	|*	|*	|d	|*	|[14][S]\rarr	|s	|e	|r	|v	|e	|t	|*	|*	|*	|n	|*	|.
|*	|a	|*	|*	|*	|[][,]{ }	|*	|d	|*	|*	|o	|*	|*	|y	|c	|*	|*	|*	|u	|*	|*	|*	|o	|*	|.
|*	|[][,]{ }	|*	|*	|*	|a	|*	|y	|*	|*	|r	|*	|*	|l	|h	|[15][S]\darr	|*	|*	|r	|*	|*	|*	|w	|*	|.
|*	|r	|*	|*	|[16][S]\darr	|f	|[17][S]\darr	|l	|*	|*	|k	|*	|*	|w	|l	|p	|*	|[18][S]\darr	|a	|*	|*	|*	|y	|*	|.
|*	|a	|*	|*	|p	|r	|u	|[][,]{ }	|*	|*	|a	|*	|*	|e	|u	|r	|*	|n	|*	|*	|*	|*	|*	|*	|.
|*	|d	|*	|*	|r	|y	|n	|b	|*	|*	|*	|*	|*	|t	|j	|e	|*	|a	|*	|*	|*	|*	|*	|*	|.
|*	|i	|*	|*	|ą	|k	|i	|ł	|*	|[19][S]\drarr	|b	|e	|z	|a	|n	|m	|a	|s	|z	|t	|*	|*	|*	|*	|.
|*	|o	|*	|*	|d	|a	|w	|o	|[20][S]\darr	|k	|*	|*	|*	|*	|o	|i	|*	|z	|*	|*	|*	|*	|*	|*	|.
|*	|l	|*	|*	|[][,]{ }	|ń	|e	|t	|k	|o	|*	|*	|*	|*	|ś	|a	|*	|[][,]{ }	|*	|*	|*	|*	|*	|*	|.
|*	|o	|*	|*	|g	|s	|r	|n	|a	|z	|*	|*	|*	|*	|ć	|[][,]{ }	|*	|c	|*	|*	|*	|*	|*	|*	|.
|*	|g	|[21][S]\rarr	|m	|a	|k	|s	|y	|m	|a	|l	|i	|z	|m	|*	|o	|*	|z	|*	|*	|*	|*	|*	|*	|.
|*	|i	|*	|*	|l	|i	|y	|*	|y	|*	|*	|*	|*	|*	|*	|p	|*	|ł	|*	|*	|*	|*	|*	|*	|.
|*	|c	|*	|*	|w	|*	|t	|*	|k	|*	|*	|[22][S]\rarr	|p	|s	|z	|c	|z	|o	|l	|i	|n	|k	|i	|*	|.
|[23][S]\rarr	|z	|ę	|b	|a	|t	|e	|k	|*	|*	|*	|*	|*	|*	|*	|y	|*	|w	|*	|*	|*	|*	|*	|*	|.
|*	|n	|*	|*	|n	|*	|c	|*	|*	|*	|*	|*	|*	|*	|*	|j	|*	|i	|*	|*	|*	|*	|*	|*	|.
|*	|a	|*	|*	|i	|*	|k	|*	|*	|*	|*	|*	|*	|*	|*	|n	|*	|e	|*	|*	|*	|*	|*	|*	|.
|*	|*	|*	|*	|c	|*	|o	|*	|*	|*	|*	|*	|*	|*	|*	|a	|*	|k	|*	|*	|*	|*	|*	|*	|.
|*	|*	|*	|*	|z	|*	|ś	|*	|*	|*	|*	|*	|*	|*	|*	|*	|*	|*	|*	|*	|*	|*	|*	|*	|.
|*	|*	|*	|*	|n	|*	|ć	|*	|*	|*	|*	|*	|*	|*	|*	|*	|*	|*	|*	|*	|*	|*	|*	|*	|.
|[24][S]\rarr	|l	|i	|r	|y	|k	|*	|*	|*	|*	|*	|*	|*	|*	|*	|*	|*	|*	|*	|*	|*	|*	|*	|*	|.
|[25][S]\rarr	|b	|ó	|b	|*	|*	|*	|*	|*	|*	|*	|*	|*	|*	|*	|*	|*	|*	|*	|*	|*	|*	|*	|*	|.\end{Puzzle}

\newpage

\begin{PuzzleClues}{\textbf{Poziome}\\}\Clue{4}{}{mieszkaniec Czechosłowacji, państwa w Europie (1945-1990)}
\Clue{12}{}{KAZALNICA}
\Clue{14}{}{hiszpański teolog i lekarz (1511-53); odkrywca małego (płucnego) krążenia krwi, za kwestionowanie dogmatu Trójcy Św. spalony na stosie}
\Clue{19}{}{nazwa tylnego żagla na jednostce żaglowej trzy- lub więcej-masztowej; jeżeli statek ma 2 maszty, to ostatni może być nazwany bezanmasztem tylko wtedy, gdy pierwszy to grotmaszt}
\Clue{21}{}{koncepcja filozoficzna opierająca się na dążeniu do wyjaśnienia całej rzeczywistości, zmierzająca do systematycznego rozwinięcia wszystkich działów filozofii}
\Clue{22}{}{Andreninae - podrodzina błonkówek występujących prawie na całym świecie przy czym największą różnorodność prezentuje jeden rodzaj - pszczolinka (Andrena), która zawiera 1300 gatunków}
\Clue{23}{}{australijski ptak z rzędu wróblowatych, rodzina altanników}
\Clue{24}{}{poeta, twórca poezji, wierszy}
\Clue{25}{}{Vicia faba - gatunek jednorocznej rośliny uprawnej z rodziny bobowatych}\end{PuzzleClues}

\begin{PuzzleClues}{\textbf{Pionowe}\\}\Clue{1}{}{macierz o wymiarze mx1}
\Clue{2}{}{cecha człowieka, który żyje pobożnie, jest wierzący i spełnia praktyki religijne}
\Clue{3}{}{mieszanina substacji, o dominującej zawartości wodorotlenku potasu, powstająca w procesie kaustyfikacji z potażu}
\Clue{5}{}{Pachyuromys duprasi - gatunek niewielkiego gryzonia z podrodziny suwaków, jedyny przedstawiciel rodzaju Pachyuromys; występuje w Afryce Północnej w północnej części pustyni leżącej na zachód od delty Nilu w Egipcie, a literatura opisuje rownież jego występowanie na terenach Algierii, Libii, Tunezji}
\Clue{6}{}{instrument ludowy, prototyp skrzypiec}
\Clue{7}{}{forma doskonalenia zawodowego umożliwiająca zdobycie określonej specjalizacji lekarskiej finansowana przez Ministerstwo Zdrowia}
\Clue{8}{}{dział anatomii, pierwotnie opisujący organizm ludzki przy użyciu zdjęć rentgenowskich układu kostnego}
\Clue{9}{}{Crocodylus palustris - gatunek gada z rodziny krokodylowatych podobny do krokodyla nilowego, występujący w Iranie, Pakistanie, Indiach, Nepalu, Bangladeszu i Sri Lance}
\Clue{10}{}{mieszkanka Andory, kobieta pochodzenia andorskiego}
\Clue{11}{}{to, że coś wygląda niechlujnie, jest zrobione, przeprowadzone niechlujnie - bez dbałości o poprawność, wygląd, szczegóły; to, że coś świadczy o czyjejś niechlujności w wykonywaniu czegoś}
\Clue{13}{}{masywna figura człowieka lub zwierzęcia, niezbyt dobrze widoczna}
\Clue{15}{}{kwota, jaką dostaje sprzedawca opcji od jej kupca}
\Clue{16}{}{stały prąd charakteryzujący się niskim napięciem i małym natężeniem; nie wywołuje skurczu mięśni, znajduje zastosowanie w medycynie}
\Clue{17}{}{to, że coś jest uniwersytetem, cecha uczelni wyższej wyrażająca się w używaniu przez nią nazwyUniwersytet, wynikająca z charakteru instytucji jako jednostki prowadzącej działalność naukową, badawczą, mającej uprawnienia do nadawania tytułów naukowych}
\Clue{18}{}{osoba, która należy do naszego środowiska lub ugrupowania, która stoi po naszej stronie}
\Clue{19}{}{dawny pogański obrzęd, który w tradycji ludowej przetrwał do połowy XX wieku}
\Clue{20}{}{orzech ziemny w kolorowej polewie}\end{PuzzleClues}\newpage%\section*{Krzyżówka 3}

\noindent\begin{Puzzle}{22}{31}|*	|*	|*	|*	|*	|*	|*	|*	|[1][S]\darr	|[2][S]\drarr	|s	|p	|o	|t	|*	|*	|*	|*	|[3][S]\drarr	|l	|u	|z	|*	|.
|*	|*	|*	|*	|*	|*	|*	|[4][S]\darr	|f	|r	|*	|*	|*	|*	|*	|[5][S]\drarr	|ł	|a	|s	|k	|u	|n	|*	|.
|*	|*	|*	|*	|*	|[6][S]\darr	|*	|k	|e	|a	|*	|*	|*	|*	|*	|p	|[7][S]\darr	|*	|z	|*	|*	|[8][S]\darr	|*	|.
|*	|*	|*	|[9][S]\darr	|*	|s	|*	|a	|l	|c	|*	|*	|*	|*	|*	|o	|b	|*	|o	|*	|*	|s	|*	|.
|*	|*	|*	|k	|*	|a	|*	|w	|d	|h	|*	|*	|*	|*	|*	|l	|o	|[10][S]\darr	|p	|*	|*	|z	|*	|.
|*	|*	|*	|r	|*	|ł	|*	|a	|j	|u	|[11][S]\darr	|*	|*	|[12][S]\drarr	|m	|i	|r	|i	|*	|*	|*	|k	|*	|.
|*	|*	|*	|a	|*	|a	|*	|l	|e	|n	|ł	|[13][S]\darr	|*	|p	|[14][S]\darr	|c	|o	|s	|[15][S]\darr	|*	|*	|a	|*	|.
|*	|*	|*	|ś	|*	|t	|*	|e	|g	|k	|u	|w	|*	|u	|f	|j	|w	|e	|c	|*	|*	|r	|*	|.
|*	|*	|*	|n	|*	|a	|[16][S]\rarr	|r	|e	|i	|k	|i	|*	|c	|r	|a	|c	|o	|w	|*	|*	|ł	|*	|.
|*	|*	|*	|i	|*	|[][,]{ }	|*	|*	|r	|*	|*	|w	|[17][S]\darr	|h	|o	|n	|e	|*	|a	|*	|*	|a	|*	|.
|*	|*	|*	|k	|*	|m	|*	|*	|*	|[18][S]\rarr	|d	|e	|s	|a	|n	|t	|*	|*	|n	|*	|[19][S]\darr	|t	|*	|.
|[20][S]\drarr	|f	|u	|[][,]{ }	|z	|o	|n	|g	|*	|*	|*	|r	|p	|c	|t	|*	|*	|*	|o	|*	|w	|k	|*	|.
|d	|[21][S]\drarr	|o	|p	|ó	|r	|*	|[22][S]\rarr	|o	|ł	|t	|a	|r	|z	|*	|*	|*	|[23][S]\darr	|ś	|*	|a	|a	|*	|.
|z	|p	|*	|i	|*	|s	|*	|[24][S]\rarr	|d	|e	|j	|*	|z	|y	|*	|*	|*	|f	|ć	|[25][S]\darr	|r	|[][,]{ }	|*	|.
|i	|r	|*	|ę	|[26][S]\rarr	|k	|u	|j	|a	|[][,]{ }	|b	|ł	|ę	|k	|i	|t	|n	|a	|*	|n	|t	|k	|*	|.
|e	|e	|*	|c	|*	|a	|*	|*	|*	|*	|*	|*	|t	|[][,]{ }	|*	|*	|*	|z	|*	|e	|o	|r	|*	|.
|r	|c	|*	|i	|*	|*	|[27][S]\rarr	|k	|l	|a	|n	|g	|*	|g	|*	|*	|*	|a	|[28][S]\darr	|g	|ś	|ó	|*	|.
|z	|j	|*	|o	|[29][S]\rarr	|h	|i	|p	|o	|t	|e	|l	|o	|r	|y	|z	|m	|*	|s	|a	|c	|l	|*	|.
|b	|o	|[30][S]\drarr	|p	|y	|ł	|k	|o	|j	|a	|d	|[][,]{ }	|s	|z	|a	|r	|y	|*	|k	|c	|i	|e	|*	|.
|i	|z	|w	|l	|*	|*	|*	|[31][S]\rarr	|s	|z	|a	|c	|h	|y	|*	|*	|*	|*	|a	|y	|o	|w	|*	|.
|k	|a	|a	|a	|*	|[32][S]\rarr	|m	|i	|e	|j	|s	|c	|o	|w	|y	|*	|*	|[33][S]\darr	|k	|j	|w	|s	|*	|.
|[][,]{ }	|*	|c	|m	|*	|*	|*	|*	|*	|*	|*	|[34][S]\rarr	|v	|i	|p	|*	|*	|p	|u	|n	|o	|k	|*	|.
|z	|[35][S]\drarr	|h	|e	|l	|l	|e	|n	|i	|s	|t	|y	|k	|a	|*	|[36][S]\darr	|*	|e	|n	|o	|ś	|a	|*	|.
|b	|m	|l	|k	|*	|*	|*	|*	|[37][S]\darr	|*	|[38][S]\rarr	|d	|o	|s	|t	|o	|j	|n	|o	|ś	|ć	|*	|*	|.
|r	|a	|a	|*	|[39][S]\drarr	|s	|e	|w	|a	|n	|*	|*	|*	|t	|*	|b	|*	|i	|w	|ć	|*	|*	|*	|.
|o	|g	|r	|*	|b	|*	|*	|*	|s	|[40][S]\rarr	|p	|a	|s	|y	|w	|i	|s	|t	|a	|*	|*	|*	|*	|.
|c	|g	|z	|*	|u	|*	|*	|[41][S]\rarr	|p	|o	|p	|o	|w	|*	|*	|*	|*	|e	|t	|*	|*	|*	|*	|.
|z	|i	|*	|*	|s	|*	|*	|[42][S]\rarr	|a	|n	|d	|r	|o	|g	|e	|n	|*	|n	|e	|*	|*	|*	|*	|.
|n	|o	|[43][S]\rarr	|l	|a	|s	|[][,]{ }	|d	|z	|i	|e	|w	|i	|c	|z	|y	|*	|t	|*	|*	|*	|*	|*	|.
|y	|r	|*	|*	|*	|*	|*	|*	|j	|*	|*	|*	|*	|*	|*	|*	|*	|*	|*	|*	|*	|*	|*	|.
|*	|e	|[44][S]\rarr	|k	|o	|ń	|[][,]{ }	|k	|a	|b	|a	|r	|d	|y	|ń	|s	|k	|i	|*	|*	|*	|*	|*	|.
|*	|*	|[45][S]\rarr	|k	|l	|i	|p	|s	|*	|*	|*	|*	|*	|*	|*	|*	|*	|*	|*	|*	|*	|*	|*	|.\end{Puzzle}

\newpage

\begin{PuzzleClues}{\textbf{Poziome}\\}\Clue{2}{}{lampa, która daje oświetlenie w postaci snopu światła zbliżonego w kształcie do stożka}
\Clue{3}{}{różnica między wymiarem otworu a wymiarem wałka przednich połączeniem}
\Clue{5}{}{ssak z podrodziny łaskunów, niewielkich ssaków drapieżnych w rodzinie łaszowatych; występuje w południowej i południowo-wschodniej Azji}
\Clue{12}{}{miasto w Malezji na płn. wybrzeżu Borneo (Sarawak), ważny port naftowy}
\Clue{16}{}{duchowy wpływ, energia duchowa}
\Clue{18}{}{wojska biorące udział w takiej operacji}
\Clue{20}{}{ur. w 1934 r., pianista chiński; laureat Konkursu im. F. Chopina w 1955 r}
\Clue{21}{}{siła działająca na poruszające się ciało fizyczne, która przeciwdziała poruszaniu się tego ciała}
\Clue{22}{}{w wolnomularstwie: stół z istotnymi dla masonerii przedmiotami, symbolami, który stoi w loży masońskiej}
\Clue{24}{}{miasto w Rumunii, okręg Kluż; przemysł drzewny}
\Clue{26}{}{Coua caerulea - gatunek ptaka z rodziny kukułkowatych (Cuculidae), z podrodziny kuji (Couinae)}
\Clue{27}{}{uderzenie w dzwon okrętowy}
\Clue{29}{}{wada wrodzona, objawiająca się zmniejszeniem rozstawu gałek ocznych}
\Clue{30}{}{Conopophila whitei - gatunek ptaka z rodziny miodojadów (Meliphagidae) występujący w Australii i na Nowej Gwinei}
\Clue{31}{}{gra prowadzona na szachownicy przez dwóch przeciwników, z których każdy dysponuje 16 figurami}
\Clue{32}{}{człowiek, który mieszka w tym miejscu, człowiek, który jest stąd}
\Clue{34}{}{peptydowy hormon składający się z 28 reszt aminokwasowych; u człowieka produkowany w jelitach (komórki D1), trzustce i niektórych strukturach mózgu}
\Clue{35}{}{nauka o kulturze i języku Grecji starożytnej i współczesnej}
\Clue{38}{}{powaga, elegancja, wyraz czyjegoś dostojeństwa}
\Clue{39}{}{jezioro w Armenii, na wysokości 1900 m, powierzchnia 1200 km2 z jeziora wypływa rzeka Razdan}
\Clue{40}{}{zwolennik pasywizmu tj. kierunku politycznego popularnego w czasie I wojny światowej i okresie międzywojennym w Polsce}
\Clue{41}{}{fizyk rosyjski (1859-1906); pionier radiotechniki, zbudował radiotelegraf i uzyskał łączność radiową na odległość 5 km}
\Clue{42}{}{hormon płciowy o budowie steroidowej o działaniu maskulinizującym, fizjologicznie występujący u mężczyzn; w małych stężeniach występuje również u kobiet}
\Clue{43}{}{las, w którego kształtowaniu człowiek nie brał udziału}
\Clue{44}{}{stara rasa koni wywodząca się z Kaukazu, charakteryzująca się wytrzymałością ,doskonałą równowagą i stabilnym chodem w trudnym terenie, zdolnością do pokonywania długich dystansów, odwagą i inteligencją; wytworzyła się w wyniku mieszania wielu ras koni stepowych (nogajskie, kałmuckie, baszkirskie, donieckie) i szlachetnych (karabachskie, perskie, achałtekińskie, arabskie) z rasami mongolskimi}
\Clue{45}{}{biżuteria damska, przyczepiana do płatków uszu uchwytem w formie klamerki}\end{PuzzleClues}

\begin{PuzzleClues}{\textbf{Pionowe}\\}\Clue{1}{}{XIX w. żołnierz batalionu strzelców w armii austro-węgierskiej i w Rosji, pełniący służbę żandarma}
\Clue{2}{}{lekcje matematyki, na których uczy się liczenia, arytmetyki}
\Clue{3}{}{rodzaj ssaków z rodziny szopowatych}
\Clue{4}{}{tytuł przysługujący odznaczonym niektórymi orderami}
\Clue{5}{}{osoba należąca do służb policji}
\Clue{6}{}{błonica sałatowa, ulwa sałatowa, Ulva lactuca - gatunek roślin z gromady zielenic (Chlorophyta)}
\Clue{7}{}{Nyctalus - rodzaj nietoperzy z rodziny mroczkowatych}
\Clue{8}{}{Alisterus scapularis - gatunek ptaka z rodziny papugowatych (Psittacidae), z podrodziny papug wschodnich (Psittaculinae)}
\Clue{9}{}{Zygaena trifolii - gatunek motyla z rodziny kraśnikowatych; występuje na większości obszaru Europy; brak go w Norwegii, Finlandii, Irlandii i na Bałkanach}
\Clue{10}{}{jezioro we Włoszech, w Alpach Lombardzkich, powierzchnia 65 km2, głębokość do 251 m, przez Iseo przepływa rzeka Oglio}
\Clue{11}{}{symbol graficzny w notacji muzycznej służący do wydłużania wartości rytmicznych; ligatura}
\Clue{12}{}{Jubula lettii - gatunek ptaka drapieżnego z rodziny puszczykowatych (Strigidae), z podrodziny puszczyków (Striginae)}
\Clue{13}{}{ŁASZA WODNIK TOPIK}
\Clue{14}{}{przednia część budynku, od ulicy}
\Clue{15}{}{cecha czynu, zachowania itp}
\Clue{17}{}{przedmiot użytkowy, np. mebel, kuchenka}
\Clue{19}{}{wysoka wartość pieniężna}
\Clue{20}{}{Malaconotus cruentus - gatunek ptaka  z rodziny dzierzbików (Malaconotidae)}
\Clue{21}{}{kosztowności, które świecą i ładnie wyglądają, świecidełka; określenie dotyczy to najczęściej biżuterii i ozdób}
\Clue{23}{}{obwód elektryczny - układ elementów tworzących drogę zamkniętą dla prądu elektrycznego}
\Clue{25}{}{to, że coś ma charakter (znaczenie) negacji, jest negacyjne}
\Clue{28}{}{Salticidae - kosmopolityczna rodzina pająków, które nie tkają sieci, tylko polują skacząc na swoją ofiarę; w skład rodziny, mającej minimum 65 mln lat, wchodzi ok. 5600 tysięcy gatunków przynależących do blisko 600 rodzajów, co czyni je najliczniejszą rodziną wśród pająków z ok. 13 proc. wszystkich gatunków; w Polsce stwierdzono występowanie 59 gatunków}
\Clue{30}{}{łow. ogon u głuszca}
\Clue{33}{}{w Kościele katolickim: osoba przystępująca do spowiedzi}
\Clue{35}{}{skala majorowa czyli durowa}
\Clue{36}{}{wąski pasek papieru zakładany na obwolutę lub okładkę książki, płyty CD, LP, DVD, Laserdisc, powszechnie stosowany w Japonii; służy on do hasłowego zareklamowania utworu, podania ceny i nazwy wydawnictwa}
\Clue{37}{}{mieszkanka Miletu urodzona około 470 roku p.n.e., która zasłynęła swym związkiem z ateńskim mężem stanu - Peryklesem}
\Clue{39}{}{statek słowiańskich plemion}\end{PuzzleClues}\newpage%\section*{Krzyżówka 4}

\noindent\begin{Puzzle}{20}{30}|*	|*	|*	|*	|*	|*	|*	|*	|*	|*	|*	|[1][S]\drarr	|f	|o	|r	|y	|ś	|*	|[2][S]\darr	|[3][S]\darr	|*	|.
|*	|*	|*	|*	|[4][S]\darr	|*	|*	|*	|[5][S]\rarr	|b	|a	|t	|t	|l	|e	|d	|r	|e	|s	|s	|*	|.
|*	|*	|*	|*	|k	|*	|[6][S]\rarr	|b	|y	|s	|t	|r	|z	|a	|k	|*	|*	|*	|t	|w	|*	|.
|*	|*	|*	|*	|o	|*	|[7][S]\rarr	|k	|a	|z	|u	|i	|s	|t	|y	|k	|a	|*	|o	|i	|*	|.
|*	|*	|*	|*	|n	|[8][S]\rarr	|p	|o	|z	|o	|s	|t	|a	|ł	|o	|ś	|ć	|*	|p	|n	|*	|.
|[9][S]\drarr	|d	|e	|l	|f	|i	|n	|[][,]{ }	|m	|a	|ł	|y	|*	|*	|*	|*	|*	|*	|n	|g	|*	|.
|l	|*	|*	|*	|e	|[10][S]\drarr	|b	|ł	|ą	|d	|[][,]{ }	|l	|e	|k	|a	|r	|s	|k	|i	|*	|*	|.
|u	|*	|[11][S]\darr	|[12][S]\darr	|s	|k	|[13][S]\darr	|*	|[14][S]\darr	|*	|*	|o	|[15][S]\rarr	|a	|d	|a	|g	|i	|o	|*	|*	|.
|l	|[16][S]\darr	|i	|t	|j	|a	|a	|*	|k	|[17][S]\darr	|*	|d	|*	|*	|*	|*	|*	|*	|w	|*	|*	|.
|o	|c	|g	|e	|a	|r	|g	|[18][S]\drarr	|o	|r	|t	|o	|p	|t	|y	|s	|t	|k	|a	|*	|*	|.
|*	|u	|ł	|o	|*	|t	|a	|p	|ł	|e	|*	|n	|[19][S]\rarr	|h	|e	|n	|g	|e	|l	|o	|*	|.
|[20][S]\drarr	|b	|a	|r	|b	|u	|*	|l	|c	|k	|*	|*	|[21][S]\drarr	|z	|a	|o	|c	|z	|n	|y	|*	|.
|b	|a	|[][,]{ }	|i	|[22][S]\darr	|z	|[23][S]\darr	|a	|h	|i	|*	|*	|m	|*	|*	|[24][S]\darr	|*	|*	|o	|*	|*	|.
|u	|[][,]{ }	|g	|a	|m	|j	|s	|g	|o	|n	|[25][S]\rarr	|f	|i	|l	|l	|e	|r	|*	|ś	|*	|[26][S]\darr	|.
|s	|l	|r	|[][,]{ }	|o	|a	|z	|a	|z	|*	|*	|[27][S]\rarr	|l	|a	|n	|d	|*	|[28][S]\darr	|ć	|*	|c	|.
|z	|i	|a	|h	|m	|*	|u	|*	|*	|*	|[29][S]\drarr	|t	|a	|u	|r	|o	|v	|o	|*	|*	|y	|.
|*	|b	|m	|e	|e	|*	|m	|*	|[30][S]\darr	|*	|s	|*	|[][,]{ }	|*	|*	|*	|*	|r	|*	|*	|f	|.
|[31][S]\drarr	|r	|o	|l	|n	|i	|k	|[][,]{ }	|r	|y	|c	|z	|a	|ł	|t	|o	|w	|y	|*	|[32][S]\darr	|e	|.
|p	|e	|f	|i	|t	|*	|a	|[33][S]\darr	|e	|[34][S]\darr	|y	|*	|n	|*	|*	|*	|*	|k	|*	|w	|r	|.
|r	|*	|o	|o	|[][,]{ }	|*	|*	|ł	|d	|s	|t	|*	|g	|*	|[35][S]\darr	|*	|*	|s	|[36][S]\darr	|ó	|k	|.
|z	|*	|n	|c	|p	|[37][S]\darr	|*	|u	|ł	|t	|*	|*	|i	|*	|t	|*	|*	|*	|p	|z	|a	|.
|e	|*	|o	|e	|ę	|z	|[38][S]\drarr	|b	|o	|o	|g	|i	|e	|[][S]-	|w	|o	|o	|g	|i	|e	|*	|.
|d	|*	|w	|n	|d	|a	|t	|e	|w	|p	|*	|*	|l	|*	|ó	|*	|*	|*	|ż	|k	|*	|.
|p	|*	|a	|t	|u	|c	|y	|k	|o	|a	|[39][S]\rarr	|a	|s	|t	|r	|o	|n	|o	|m	|*	|*	|.
|ł	|*	|*	|r	|*	|h	|p	|*	|*	|*	|*	|*	|k	|*	|*	|*	|*	|*	|a	|*	|*	|.
|a	|*	|*	|y	|*	|y	|*	|[40][S]\rarr	|d	|e	|d	|l	|a	|j	|n	|*	|*	|[41][S]\darr	|k	|*	|*	|.
|t	|*	|*	|c	|[42][S]\rarr	|ł	|ą	|c	|z	|n	|i	|k	|*	|*	|*	|*	|*	|ł	|*	|*	|*	|.
|a	|*	|*	|z	|[43][S]\rarr	|k	|o	|n	|t	|r	|a	|b	|a	|s	|*	|*	|*	|u	|[44][S]\darr	|*	|*	|.
|*	|*	|*	|n	|[45][S]\rarr	|a	|b	|o	|l	|i	|c	|j	|o	|n	|i	|s	|t	|k	|a	|*	|*	|.
|*	|*	|*	|a	|*	|*	|*	|*	|*	|*	|*	|*	|*	|*	|*	|*	|*	|*	|t	|*	|*	|.
|*	|*	|*	|*	|*	|*	|*	|*	|*	|*	|*	|*	|*	|*	|*	|*	|*	|*	|*	|*	|*	|.\end{Puzzle}

\newpage

\begin{PuzzleClues}{\textbf{Poziome}\\}\Clue{1}{}{konny ordynans oficera}
\Clue{5}{}{wojskowa kurtka, rodzaj wiatrówki}
\Clue{6}{}{ktoś, kto jest bystry, inteligentny, szybko kojarzy fakty i wyciąga poprawne wnioski}
\Clue{7}{}{metoda badawcza, metoda prowadzenia obserwacji i rozważań stosowana np. w medycynie, polegająca na omówieniu problemu przez wyliczenie znanych przypadków czy przykładów danego zjawiska}
\Clue{8}{}{to, co powstało w wyniku oddziaływania jakiegoś zjawiska, pozostało i jest trwałym dowodem na zaistnienie tego zjawiska}
\Clue{9}{}{delfin La Platy, Pontoporia blainvillei - gatunek walenia z rodziny Iniidae; żyje w Ameryce Południowej w dorzeczu La Platy}
\Clue{10}{}{nieumyślne działanie, zaniedbanie lub zaniechanie lekarza, lekarza dentysty, pielęgniarki, położnej lub osoby wykonującej inny zawód medyczny powodujące szkodę pacjenta}
\Clue{15}{}{utwór lub fragment melodii w tempie adagio}
\Clue{18}{}{osoba, która zajmuje się zaburzeniami widzenia obuocznego oraz ich leczeniem za pomocą odpowiednio dobranych ćwiczeń}
\Clue{19}{}{miasto we wsch. Holandii; węzeł kolejowy}
\Clue{20}{}{rumuński poeta i matematyk (1895-1961), profesor uniwersytetu w Bukareszcie}
\Clue{21}{}{student, który studiuje w trybie zaocznym}
\Clue{25}{}{jednostka zdawkowa na Węgrzech; 1/100 korony (od 1892), pengő (od 1926) i forinta (od 1946)}
\Clue{27}{}{powierzchnia odczytu w nośnikach optycznych, takich jak CD czy DVD}
\Clue{29}{}{miasto w azjatyckiej części Federacji Rosyjskiej na płn. od Tubolska}
\Clue{31}{}{rolnik dokonujący dostawy produktów rolnych pochodzących z własnej działalności rolniczej lub świadczący usługi rolnicze, korzystając ze zwolnienia od podatku VAT}
\Clue{38}{}{styl gry na fortepianie}
\Clue{39}{}{zajmuje się badaniem ciał niebieskich}
\Clue{40}{}{ostateczny termin wykonania czegoś}
\Clue{42}{}{element przewodu wiertniczego służący do łączenia elementów przewodu o różnych rodzajach gwintów, np. obciążników z rurami płuczkowymi, obciążników ze świdrem itp}
\Clue{43}{}{największy instrument strunowy, mający najniższą skalę tonów, wchodzi w skład orkiestry symfonicznej}
\Clue{45}{}{kobieta, która opowiada się za zniesieniem jakiegoś prawa}\end{PuzzleClues}

\begin{PuzzleClues}{\textbf{Pionowe}\\}\Clue{1}{}{Trityldon - terapsyd z grupy cytodontów, żyjący we wczesnej jurze i prawdopodobnie w późnym triasie (obok dinozaurów); zamieszkiwał południową Afrykę, ale jego skamieniałości znaleziono też na Antarktydzie}
\Clue{2}{}{cecha wyrazu, części mowy: odmiana przez stopnie}
\Clue{3}{}{cios sierpowy}
\Clue{4}{}{dawniej wyznawana religia, wyznanie}
\Clue{9}{}{smaczny owoc (jagoda) psianki lulo}
\Clue{10}{}{klasztor zakonu Kartuzów składający się z domków dla zakonników oraz kościoła klasztornego}
\Clue{11}{}{element przenoszący drgania na wkładkę gramofonową}
\Clue{12}{}{teoria budowy Układu Słonecznego, która mówi, że Słońce jest centrum Wszechświata}
\Clue{13}{}{stopień oficerski w sułtańskiej Turcji}
\Clue{14}{}{formalnie rolnicza spółdzielnia produkcyjna, rodzaj przedsiębiorstwa rolniczego charakterystycznego dla byłego ZSRR}
\Clue{16}{}{koktajl alkoholowy łączący rum, colę i opcjonalnie limonkę}
\Clue{17}{}{ryba spodoustna o długości do 16 m}
\Clue{18}{}{postępujące bardzo szybko i powodujące negatywne skutki zjawisko, na przykład epidemia choroby zakaźnej lub susza}
\Clue{20}{}{formacja roślinna, charakterystyczna dla suchych obszarów podrównikowych}
\Clue{21}{}{pozaukładowa jednostka odległości stosowana w krajach anglosaskich}
\Clue{22}{}{wektorowa wielkość fizyczna opisująca ruch ciała, zwłaszcza jego ruch obrotowy}
\Clue{23}{}{skoczna piosenka ukraińska}
\Clue{24}{}{dawna stolica Japonii, od 1868 r. zwane Tokio}
\Clue{26}{}{zdrobniale o cyfrze - inicjale}
\Clue{28}{}{Oryx - rodzaj antylop z podrodziny antylop końskich (Hippotraginae) w rodzinie krętorogich; występuje w kilku gatunkach w Afryce Wschodniej, Południowej i Północnej}
\Clue{29}{}{przedstawiciel jednego z koczowniczych ludów irańskich wywodzących się z obszarów pomiędzy Ałtajem a dolną Wołgą, zamieszkujących od schyłku VIII lub od VII wieku p.n.e. północne okolice Morza Czarnego}
\Clue{30}{}{Redłowo - nadmorska dzielnica mieszkalna Gdyni, granicząca z następującymi dzielnicami tegoż miasta: Wzgórze Św. Maksymiliana (od północy), Działki Leśne, Mały Kack (obie od zachodu), Orłowo (od południa), a od wschodu także z Morzem Bałtyckim}
\Clue{31}{}{rodzaj rozliczenia transakcji, w którym kupujący pokrywa całą należność z góry}
\Clue{32}{}{część biegowa szynowego złożona z dwu lub więcej zestawów kołowych}
\Clue{33}{}{płyta stanowiąca element złączany przy łączeniu różnorakich konstrukcji na styk}
\Clue{34}{}{(zęba) część zęba koła zębatego zawarta między powierzchnią podziałową i powierzchnią podstaw}
\Clue{35}{}{stworzenie, istota żywa, zwłaszcza człowiek albo zwierzę}
\Clue{36}{}{PIŻMOSZCZUR; północnoamerykański. gryzoń ziemnowodny o cennym futrze}
\Clue{37}{}{cienistka, Gymnocarpium - rodzaj paproci należących do rodziny rozrzutkowatych; występuje powszechnie na półkuli północnej; wykształca podłużne, pierzaste liście; gatunkiem typowym jest Gymnocarpium dryopteris}
\Clue{38}{}{przypuszczenie, informacja (cynk), że jakieś przedsięwzięcie zakończy się właśnie w dany sposób, zazwyczaj gdy mowa o zakładach}
\Clue{41}{}{OSTROŁUK}
\Clue{44}{}{w chemii: symbol astatu}\end{PuzzleClues}\newpage%\section*{Krzyżówka 5}

\noindent\begin{Puzzle}{22}{32}|*	|*	|*	|*	|*	|*	|*	|*	|*	|*	|*	|*	|*	|*	|*	|*	|[1][S]\darr	|*	|*	|*	|*	|[2][S]\darr	|*	|.
|*	|*	|*	|*	|*	|*	|*	|*	|[3][S]\drarr	|f	|r	|a	|j	|e	|r	|s	|t	|w	|o	|*	|[4][S]\darr	|k	|*	|.
|*	|*	|*	|*	|*	|[5][S]\rarr	|k	|a	|s	|z	|t	|a	|n	|e	|k	|*	|r	|*	|*	|*	|h	|i	|*	|.
|*	|*	|[6][S]\darr	|[7][S]\darr	|*	|[8][S]\drarr	|j	|ę	|t	|k	|a	|*	|[9][S]\darr	|*	|*	|*	|y	|[10][S]\darr	|*	|*	|e	|l	|*	|.
|*	|*	|p	|n	|*	|o	|*	|*	|ę	|*	|*	|*	|n	|*	|*	|*	|m	|p	|*	|*	|r	|i	|*	|.
|*	|*	|o	|i	|[11][S]\darr	|m	|*	|[12][S]\rarr	|p	|i	|n	|t	|a	|*	|[13][S]\darr	|*	|o	|ł	|*	|*	|s	|m	|*	|.
|*	|*	|w	|e	|p	|a	|[14][S]\rarr	|n	|i	|e	|s	|k	|r	|ę	|p	|o	|w	|a	|n	|i	|e	|*	|*	|.
|*	|[15][S]\rarr	|i	|b	|i	|s	|[][,]{ }	|k	|a	|s	|z	|t	|a	|n	|o	|w	|a	|t	|y	|*	|y	|*	|*	|.
|*	|*	|e	|e	|e	|t	|*	|*	|*	|*	|*	|*	|*	|*	|d	|*	|n	|e	|*	|[16][S]\darr	|*	|*	|[17][S]\darr	|.
|*	|*	|ś	|z	|k	|a	|*	|[18][S]\darr	|*	|*	|[19][S]\darr	|*	|*	|*	|ł	|*	|i	|k	|*	|k	|[20][S]\darr	|*	|p	|.
|*	|*	|ć	|p	|a	|*	|*	|ś	|[21][S]\darr	|*	|c	|*	|*	|*	|o	|[22][S]\darr	|e	|*	|*	|o	|j	|*	|o	|.
|*	|*	|[][,]{ }	|i	|r	|*	|[23][S]\darr	|n	|p	|[24][S]\darr	|y	|*	|*	|*	|ż	|p	|*	|*	|[25][S]\darr	|d	|a	|*	|t	|.
|*	|*	|a	|e	|n	|*	|p	|i	|o	|p	|k	|[26][S]\darr	|[27][S]\rarr	|b	|e	|l	|w	|e	|d	|e	|r	|*	|w	|.
|*	|*	|u	|c	|i	|[28][S]\drarr	|r	|e	|w	|o	|l	|w	|e	|r	|*	|e	|*	|[29][S]\darr	|y	|k	|m	|*	|a	|.
|*	|*	|t	|z	|a	|a	|z	|ż	|o	|d	|[][,]{ }	|i	|*	|*	|*	|b	|*	|z	|s	|s	|a	|*	|l	|.
|*	|*	|o	|e	|*	|b	|e	|k	|ź	|a	|l	|ś	|[30][S]\darr	|[31][S]\rarr	|k	|i	|j	|e	|k	|*	|r	|*	|[][,]{ }	|.
|*	|*	|b	|ń	|[32][S]\drarr	|s	|z	|a	|n	|t	|u	|n	|g	|*	|*	|s	|*	|s	|*	|*	|k	|*	|o	|.
|*	|*	|i	|s	|b	|o	|i	|[][,]{ }	|i	|n	|n	|i	|e	|*	|*	|c	|*	|t	|*	|*	|o	|[33][S]\darr	|l	|.
|*	|*	|o	|t	|o	|l	|e	|c	|k	|o	|a	|e	|o	|*	|*	|y	|*	|r	|[34][S]\darr	|*	|w	|b	|b	|.
|*	|*	|g	|w	|l	|u	|r	|z	|*	|ś	|r	|w	|g	|*	|*	|t	|*	|z	|m	|*	|i	|u	|r	|.
|*	|*	|r	|o	|e	|c	|n	|a	|*	|ć	|n	|*	|r	|*	|*	|*	|*	|a	|a	|*	|c	|r	|o	|.
|*	|*	|a	|*	|s	|j	|i	|r	|*	|*	|y	|*	|a	|*	|*	|*	|*	|ł	|u	|*	|z	|a	|t	|.
|*	|*	|f	|*	|ł	|a	|k	|n	|*	|*	|*	|*	|f	|*	|*	|*	|*	|*	|r	|*	|*	|c	|o	|.
|*	|*	|i	|[35][S]\darr	|a	|*	|*	|o	|*	|*	|*	|*	|*	|[36][S]\rarr	|d	|e	|r	|b	|y	|*	|*	|t	|w	|.
|*	|*	|c	|a	|w	|[37][S]\rarr	|w	|s	|t	|ę	|ż	|n	|i	|c	|e	|*	|*	|*	|t	|*	|*	|w	|i	|.
|*	|*	|z	|p	|c	|[38][S]\rarr	|e	|k	|l	|e	|k	|t	|y	|z	|m	|*	|*	|*	|y	|*	|*	|o	|e	|.
|*	|*	|n	|a	|z	|*	|[39][S]\rarr	|r	|ó	|w	|[][,]{ }	|i	|r	|y	|g	|a	|c	|y	|j	|n	|y	|*	|c	|.
|*	|[40][S]\rarr	|a	|n	|a	|l	|i	|z	|a	|t	|o	|r	|[][,]{ }	|r	|ó	|ż	|n	|i	|c	|o	|w	|y	|*	|.
|*	|*	|*	|a	|n	|*	|*	|y	|[41][S]\rarr	|g	|w	|a	|t	|e	|m	|a	|l	|c	|z	|y	|k	|*	|*	|.
|*	|*	|*	|ż	|k	|[42][S]\rarr	|o	|d	|n	|a	|w	|i	|a	|n	|i	|e	|*	|*	|y	|*	|*	|*	|*	|.
|*	|*	|*	|*	|a	|*	|*	|ł	|*	|*	|*	|*	|[43][S]\rarr	|j	|u	|d	|a	|i	|k	|a	|*	|*	|*	|.
|*	|*	|*	|*	|*	|[44][S]\rarr	|z	|a	|d	|z	|w	|o	|n	|i	|e	|n	|i	|e	|*	|*	|*	|*	|*	|.
|[45][S]\rarr	|w	|y	|k	|u	|s	|z	|*	|*	|*	|*	|*	|*	|*	|*	|*	|*	|*	|*	|*	|*	|*	|*	|.\end{Puzzle}

\newpage

\begin{PuzzleClues}{\textbf{Poziome}\\}\Clue{3}{}{naiwność (cecha człowieka)}
\Clue{5}{}{mały kasztan - owoc kasztanowca (drzewa)}
\Clue{8}{}{belka podtrzymująca krokwie dachu}
\Clue{12}{}{stosowana w dawnej Francji miara objętości cieczy; około 0,9 litra}
\Clue{14}{}{niezależność, swoboda, brak ograniczeń}
\Clue{15}{}{Plegadis falcinellus - gatunek ptaka z rodziny ibisowatych (Threskiornithidae)}
\Clue{27}{}{warszawski pałac, reprezentacyjna rezydencja państwowa}
\Clue{28}{}{wielostrzałowa nieautomatyczna krótka broń powtarzalna, w której funkcję magazynka niewymiennego spełnia obrotowy bęben z kilkoma komorami nabojowymi (zwykle sześcioma)}
\Clue{31}{}{zdrobniale: kij - kawałek drewna o cylindrycznym kształcie, o różnej długości i grubości, posiadający dwa lub więcej końców (rozwidlenia), który został ułamany lub obcięty z drzewa, krzewu, trzciny lub trawy (bambus), ewentualnie wystrugany z drewna}
\Clue{32}{}{SHANDONG}
\Clue{36}{}{miasto w Anglii nad rzeką Dermrent; przemysł lotniczy, maszynowy, samochodowy, chemiczny}
\Clue{37}{}{Nemertea - typ zwierząt bezkręgowych o bardzo długim, robakokształtnym, obłym lub płaskim, nieczłonowanym ciele, w większości wodnych, przeważnie morskich; charakterystyczną cechą (synapomorficzną) wstężnic jest obecność długiego, wysuwanego, chwytnego ryjka (proboscis), oddzielonego od jelita, uzbrojonego lub nie}
\Clue{38}{}{nurt polegający na twórczości kompilacyjnej, łączącej różne elementy i treści z różnych stylów, epok i kierunków artystycznych, nieprowadzący do nowej syntezy, nieoryginalny}
\Clue{39}{}{ręcznie lub mechanicznie wykonane podłużne zagłębienie w ziemi służące do regulowania gospodarki wodnej upraw rolnych}
\Clue{40}{}{analogowy komputer do rozwiązywania równań różniczkowych}
\Clue{41}{}{mieszkaniec Gwatemali, człowiek pochodzenia gwatemalskiego}
\Clue{42}{}{renowacja, odświeżenie czegoś już istniejącego}
\Clue{43}{}{dokumenty, druki, przedmioty dotyczące Żydów, ich religii i kultury}
\Clue{44}{}{dźwięczny i metaliczny odgłos}
\Clue{45}{}{forma architektoniczna wzorowana na budownictwie Bliskiego Wschodu (architektura islamu) stanowiąca wystający z lica elewacji fragment budynku poszerzający przylegające wnętrze, nadwieszony powyżej pierwszego piętra na wysokości jednej lub kilku kondygnacji}\end{PuzzleClues}

\begin{PuzzleClues}{\textbf{Pionowe}\\}\Clue{1}{}{nadawanie żaglowi, kadłubowi jachtu lub linie prawidłowego kształtu czy długości w celu zapewnienia najkorzystniejszych warunków żeglugi}
\Clue{2}{}{dwustronna tkanina dekoracyjna na ścianę lub podłogę, wełniana, której wzór tworzy wątek pokrywający osnowę}
\Clue{3}{}{powolny, miarowy chód zwierząt, zwłaszcza koni}
\Clue{4}{}{amerykański pisarz i publicysta, ur. 1914r, „Hiroszima”, „Sprzysiężenie”}
\Clue{6}{}{typ powieści, której fabuła oparta jest na faktach z życia autora, często przeplatających się z wydarzeniami fikcyjnymi}
\Clue{7}{}{stan, sytuacja, zdarzenie, położenie grożące czymś złym, zagrażające komuś}
\Clue{8}{}{OKRASA}
\Clue{9}{}{miasto w Japonii (środkowe Honsiu), ośrodek administracyjny prefektury Nara w pobliżu Osaki, dawna stolica cesarstwa}
\Clue{10}{}{kryształek lodu lub śniegu, mała drobinka zamarzniętej wody}
\Clue{11}{}{pomieszczenie służące do wypieku pieczywa}
\Clue{13}{}{powód czegoś, czynnik coś wywołujący, będący czegoś przyczyną}
\Clue{16}{}{zbiór przepisów prawnych}
\Clue{17}{}{kaszalot, potwal, Physeter macrocephalus - gatunek walenia z rodziny kaszalotowatych, największy przedstawiciel podrzędu zębowców; występuje we wszystkich oceanach}
\Clue{18}{}{Montifringilla adamsi - gatunek ptaków z rodziny wróblowatych (Passeridae) występujący w Eurazji}
\Clue{19}{}{19 lat, po upływie których te same fazy Księżyca odpowiadają tym samym dniom słonecznej rachuby czasu}
\Clue{20}{}{bywalec jarmarków}
\Clue{21}{}{człowiek zajmujący się zawodowo produkcją powozów}
\Clue{22}{}{głosowanie członków społeczności na określonym terytorium (np. kraju, województwa, gminy) w różnych sprawach mających związek z ich miejscem zamieszkania}
\Clue{23}{}{motyl z rodziny przeziernikowatych, podobny do osy, o czarnym odwłoku w żółte lub czerwone pierścienie}
\Clue{24}{}{cecha przedmiotu lub osoby wskazująca na możliwość ich kształtowania przez czynniki zewnętrzne lub na skłonność do ulegania ich wpływowi}
\Clue{25}{}{twardy dysk, pomocnicza pamięć w komputerze}
\Clue{26}{}{wieś w Polsce położona w województwie mazowieckim, w powiecie siedleckim, w gminie Wiśniew}
\Clue{28}{}{rozgrzeszenie, uwolnienie od kar kościelnych}
\Clue{29}{}{poprzeczna podpórka skrzydeł w samolocie}
\Clue{30}{}{nauczyciel uczący w szkole geografii}
\Clue{32}{}{mieszkanka Bolesławca}
\Clue{33}{}{grupa zachowujących się niekulturalnie, chamsko, prostacko ludzi}
\Clue{34}{}{mieszkaniec Mauritiusa, człowiek pochodzenia maurytyjskiego}
\Clue{35}{}{wynagrodzenie, renta lub inna forma dochodu dla monarchów i ich rodzin, wysokich urzędników państwowych, byłych głów państw itp}\end{PuzzleClues}\newpage%\section*{Krzyżówka 6}

\noindent\begin{Puzzle}{18}{29}|*	|*	|*	|*	|*	|*	|*	|*	|*	|*	|*	|*	|*	|*	|[1][S]\darr	|[2][S]\darr	|*	|*	|*	|.
|*	|*	|*	|*	|[3][S]\drarr	|w	|a	|n	|i	|e	|n	|k	|a	|*	|g	|s	|*	|*	|[4][S]\darr	|.
|*	|*	|*	|[5][S]\rarr	|m	|o	|t	|o	|w	|ą	|ż	|*	|*	|[6][S]\darr	|e	|k	|[7][S]\darr	|[8][S]\darr	|l	|.
|*	|*	|[9][S]\rarr	|j	|a	|m	|r	|a	|j	|e	|*	|*	|*	|k	|t	|r	|g	|w	|e	|.
|[10][S]\drarr	|ż	|u	|r	|n	|a	|l	|*	|*	|*	|*	|*	|*	|u	|t	|y	|ł	|e	|w	|.
|r	|[11][S]\rarr	|h	|e	|n	|r	|y	|k	|[][,]{ }	|i	|v	|[][,]{ }	|p	|r	|o	|b	|u	|s	|*	|.
|y	|[12][S]\darr	|*	|*	|*	|*	|*	|*	|*	|*	|*	|*	|*	|i	|*	|a	|c	|z	|[13][S]\darr	|.
|t	|n	|*	|[14][S]\darr	|[15][S]\rarr	|k	|r	|e	|t	|s	|c	|h	|m	|e	|r	|*	|h	|[][,]{ }	|k	|.
|e	|i	|[16][S]\rarr	|c	|y	|l	|i	|n	|d	|e	|r	|*	|*	|r	|[17][S]\darr	|*	|ó	|m	|o	|.
|l	|e	|[18][S]\rarr	|z	|a	|k	|ł	|a	|d	|z	|i	|n	|y	|*	|s	|[19][S]\darr	|w	|o	|r	|.
|*	|s	|[20][S]\darr	|a	|*	|*	|*	|*	|[21][S]\rarr	|n	|a	|r	|y	|b	|e	|k	|*	|r	|d	|.
|*	|y	|k	|p	|*	|*	|*	|*	|*	|*	|[22][S]\drarr	|c	|f	|*	|n	|a	|*	|s	|o	|.
|[23][S]\drarr	|m	|u	|l	|t	|y	|p	|l	|i	|k	|a	|c	|j	|a	|*	|m	|[24][S]\darr	|k	|n	|.
|a	|p	|k	|a	|*	|*	|*	|*	|[25][S]\rarr	|ł	|u	|g	|o	|w	|n	|i	|c	|a	|*	|.
|k	|a	|l	|[][,]{ }	|*	|*	|*	|*	|*	|*	|d	|[26][S]\darr	|*	|*	|*	|e	|z	|*	|*	|.
|c	|t	|i	|z	|*	|*	|[27][S]\rarr	|c	|a	|r	|y	|z	|m	|*	|*	|n	|e	|[28][S]\darr	|*	|.
|e	|y	|k	|i	|[29][S]\rarr	|k	|o	|p	|i	|s	|t	|a	|*	|*	|*	|i	|r	|k	|*	|.
|n	|c	|[][,]{ }	|e	|[30][S]\drarr	|e	|s	|k	|i	|m	|o	|s	|k	|a	|*	|u	|w	|a	|*	|.
|t	|z	|c	|l	|s	|*	|[31][S]\rarr	|m	|o	|d	|r	|z	|e	|w	|*	|s	|o	|r	|*	|.
|u	|n	|h	|o	|p	|*	|[32][S]\drarr	|b	|o	|y	|*	|c	|*	|*	|*	|z	|n	|t	|*	|.
|a	|o	|i	|n	|a	|*	|l	|[33][S]\drarr	|ś	|l	|i	|z	|g	|*	|[34][S]\darr	|k	|e	|a	|*	|.
|c	|ś	|l	|a	|ł	|*	|u	|h	|*	|*	|*	|y	|*	|*	|b	|a	|[][,]{ }	|g	|*	|.
|j	|ć	|i	|w	|a	|*	|z	|u	|[35][S]\rarr	|p	|o	|t	|a	|l	|a	|*	|m	|i	|*	|.
|a	|*	|j	|a	|*	|*	|*	|n	|*	|*	|*	|n	|[36][S]\darr	|*	|r	|[37][S]\darr	|i	|ń	|*	|.
|*	|*	|s	|*	|[38][S]\rarr	|m	|e	|t	|y	|l	|d	|o	|p	|a	|*	|d	|ę	|c	|*	|.
|*	|[39][S]\rarr	|k	|l	|u	|c	|z	|*	|*	|*	|*	|ś	|a	|[40][S]\rarr	|m	|y	|s	|z	|*	|.
|*	|[41][S]\rarr	|i	|l	|u	|m	|i	|n	|a	|t	|*	|ć	|d	|*	|*	|n	|o	|y	|*	|.
|*	|*	|*	|*	|*	|*	|*	|*	|*	|*	|*	|*	|ó	|*	|*	|i	|*	|k	|*	|.
|*	|*	|*	|*	|*	|*	|*	|*	|*	|[42][S]\rarr	|s	|i	|ł	|a	|*	|a	|*	|*	|*	|.
|*	|*	|*	|*	|*	|*	|*	|*	|*	|*	|*	|*	|*	|*	|*	|*	|*	|*	|*	|.\end{Puzzle}

\newpage

\begin{PuzzleClues}{\textbf{Poziome}\\}\Clue{3}{}{zawartość wanienki, laboratoryjnego lub kuchennego naczynia, które jest nieduże i dość płytkie}
\Clue{5}{}{sieć służąca do połowu węgorzy}
\Clue{9}{}{jamrajokształtne, Peramelemorphia - rząd torbaczy obejmujący niewielkie ssaki o wydłużonym pysku, długim ogonie oraz ze zrośniętym 2. i 3. palcem stopy}
\Clue{10}{}{czasopismo poświęcone modzie, obficie ilustrowane modelami wzorów}
\Clue{11}{}{książę wrocławski w latach 1270-1290, książę krakowski w latach 1288-1290}
\Clue{15}{}{twórczość, poglądy i dorobek naukowy Ernsta Kretschmera, niemieckiego lekarza psychiatry i twórcy teorii konstytucjonalnej}
\Clue{16}{}{sztywny męski kapelusz jedwabny z wysoką główką w kształcie walca}
\Clue{18}{}{obyczaj, związany z rozpoczęciem budowy nowego domu, wypełniany, aby zapewnić domostwu i rodzinie powodzenie i spokój}
\Clue{21}{}{młode, małe ryby (do ukończenia pierwszego roku życia), które po zużyciu zawartości woreczka żółtkowego rozpoczęły samodzielne zdobywanie pokarmu}
\Clue{22}{}{skrót/symbol franka komoryjskiego}
\Clue{23}{}{zwielokrotnienie}
\Clue{25}{}{rodzaj naczynia laboratoryjnego, które służy do ługowania}
\Clue{27}{}{monarchiczny ustrój państwa z carem na czele; też: system takich rządów}
\Clue{29}{}{osoba, która przepisywała księgi}
\Clue{30}{}{przedstawicielka grupy rdzennych ludów obszarów arktycznych i subarktycznych Grenlandii, Kanady, Alaski i Syberii}
\Clue{31}{}{długowieczne drzewo iglaste o igłach opadających na zimę i bardzo cennym drewnie}
\Clue{32}{}{pseudonim Tadeusza Kamila Marcjana Żeleńskiego - polskiego tłumacza literatury francuskiej, krytyka literackiego i teatralnego, pisarza, poety-satyryka, kronikarza, eseisty, działacza społecznego, z wykształcenia lekarza}
\Clue{33}{}{pojedynczy zjazd w saneczkarstwie sportowym, wykonany na specjalnie przygotowanym torze}
\Clue{35}{}{siedziba Dalajlamy w Tybecie}
\Clue{38}{}{lek hipotensyjny, który wskutek podobieństwa do dihydroksyfenyloalaniny (DOPA) blokuje powstawanie dopaminy, która jest prekursorem noradrenaliny}
\Clue{39}{}{charakterystyczna formacja tworzona podczas wędrówki przez niektóre migrujące ptaki}
\Clue{40}{}{wszystkożerny, szkodliwy gryzoń długości do 10 cm}
\Clue{41}{}{laminat świecący stosowany do wystroju wnętrz}
\Clue{42}{}{przemoc, sięganie po siłę mięśni lub broni, uciekanie się do nacisku fizycznego}\end{PuzzleClues}

\begin{PuzzleClues}{\textbf{Pionowe}\\}\Clue{1}{}{grupa społeczna izolująca się od otoczenia, posiadająca własne prawa i zasady, według których funkcjonuje}
\Clue{2}{}{dawniej: osoba, która przepisywała teksty, nie tylko zawodowo (lub w związku z powołaniem) oraz spisywała teksty np. ze słuchu, pod dyktando}
\Clue{3}{}{Heinrich (1871-1956), pisarz niemiecki, antyfaszysta; „Profesor Unrat”, „Cesarstwo”, nowele, eseje}
\Clue{4}{}{jednostka walutowa w Bułgarii}
\Clue{6}{}{pracownik firmy kurierskiej, dostarczający przesyłki}
\Clue{7}{}{wieś w Polsce położona w województwie łódzkim, w powiecie skierniewickim, w gminie Głuchów}
\Clue{8}{}{Saduria entomon - drapieżny morski gatunek skorupiaka z rzędu równonogów (Isopoda), dochodzący do 8 cm długości; pancerz koloru żółto-szarego składa się ze względnie małej głowy, 7 segmentów tułowia (z 7 parami odnóży) oraz 4 segmentów odwłoka, zakończonego charakterystyczną trójkątną (klinowatą) płytką ogonową}
\Clue{10}{}{duża wieś borowiacka w Polsce położona w województwie pomorskim, w powiecie chojnickim, w gminie Czersk}
\Clue{12}{}{to, że coś jest odbierane jako niesympatyczne, niemiłe, uciążliwe}
\Clue{13}{}{system posterunków granicznych}
\Clue{14}{}{Butorides striata - gatunek ptaka z rodziny czaplowatych (Ardeidae)}
\Clue{17}{}{marzenie senne - to, co się człowiekowi czasem śni, gdy śpi; serie obrazów, dźwięków, emocji, myśli i innych wrażeń zmysłowych pojawiające się podczas snu}
\Clue{19}{}{Histrionicus histrionicus - gatunek średniego ptaka wodnego z rodziny kaczkowatych (Anatidae), zamieszkujący Syberię na wschód od Bajkału, Aleuty, Alaskę, północną i środkową Kanadę, Grenlandię i Islandię}
\Clue{20}{}{Geum chiloense - gatunek rośliny z rodziny różowatych}
\Clue{22}{}{członek sądu kościelnego (może być świecki) przygotowujący materiał procesowy}
\Clue{23}{}{w psychologii proces poznawczy akcentowania różnic pomiędzy członkami grupy własnej a członkami grupy obcej przy jednoczesnej tendencji do ujednolicania grupy obcej, co prowadzi do powstawania stereotypów}
\Clue{24}{}{mięso, które ma więcej tkanki tłuszczowej i mioglobuliny (białka biorącego udział w magazynowaniu tlenu), niż mięso białe}
\Clue{26}{}{cecha czegoś, co przynosi zaszczyt}
\Clue{28}{}{mieszkaniec Kartaginy}
\Clue{30}{}{część powierzchni pnia drzewa (głównie sosny) poddawana żywicowaniu}
\Clue{32}{}{dużo miejsca, przestrzeni, przestronne miejsce}
\Clue{33}{}{angielski malarz i grafik (1817-1910), obrazy o tematyce religijnej i literackiej}
\Clue{34}{}{pierwiastek z grupy II głównej układu okresowego pierwiastków, z okresu szóstego}
\Clue{36}{}{świat, Ziemia}
\Clue{37}{}{Cucurbita - rodzaj roślin jednorocznych z rodziny dyniowatych obejmujący około 20 gatunków}\end{PuzzleClues}\newpage%\section*{Krzyżówka 7}

\noindent\begin{Puzzle}{21}{21}|*	|*	|*	|*	|[1][S]\drarr	|o	|r	|g	|a	|n	|[][,]{ }	|r	|e	|n	|t	|o	|w	|y	|*	|*	|*	|[2][S]\darr	|.
|*	|*	|*	|[3][S]\drarr	|c	|y	|n	|k	|*	|*	|*	|*	|*	|*	|*	|*	|*	|*	|*	|*	|[4][S]\darr	|t	|.
|[5][S]\drarr	|r	|a	|s	|o	|w	|o	|ś	|ć	|*	|*	|*	|*	|*	|*	|*	|*	|*	|*	|[6][S]\darr	|p	|a	|.
|k	|*	|*	|c	|r	|[7][S]\darr	|*	|[8][S]\drarr	|m	|a	|m	|m	|o	|l	|o	|g	|i	|a	|*	|d	|e	|b	|.
|r	|*	|[9][S]\darr	|h	|r	|o	|[10][S]\drarr	|s	|n	|a	|j	|p	|e	|r	|k	|a	|*	|*	|[11][S]\darr	|i	|r	|u	|.
|ą	|[12][S]\drarr	|a	|w	|i	|c	|e	|n	|i	|a	|*	|[13][S]\darr	|*	|[14][S]\darr	|[15][S]\darr	|*	|[16][S]\darr	|[17][S]\darr	|t	|a	|g	|n	|.
|ż	|r	|l	|y	|b	|z	|d	|i	|*	|*	|*	|w	|[18][S]\darr	|r	|s	|[19][S]\drarr	|k	|l	|a	|p	|a	|*	|.
|e	|o	|d	|z	|*	|a	|y	|f	|*	|*	|*	|k	|l	|a	|z	|a	|r	|e	|r	|s	|m	|*	|.
|n	|ś	|r	|*	|[20][S]\darr	|r	|t	|f	|*	|*	|*	|ł	|e	|k	|t	|t	|z	|g	|c	|y	|i	|[21][S]\darr	|.
|i	|l	|i	|*	|y	|*	|o	|e	|*	|*	|*	|a	|w	|i	|u	|a	|t	|i	|z	|d	|n	|p	|.
|e	|i	|n	|[22][S]\drarr	|b	|a	|r	|r	|i	|g	|u	|d	|o	|*	|r	|k	|a	|t	|k	|y	|*	|ł	|.
|[][,]{ }	|n	|*	|s	|*	|[23][S]\darr	|k	|*	|*	|*	|*	|*	|*	|*	|w	|[][,]{ }	|*	|k	|a	|*	|[24][S]\darr	|o	|.
|m	|a	|*	|z	|[25][S]\rarr	|s	|a	|l	|a	|[][,]{ }	|p	|l	|e	|n	|a	|r	|n	|a	|*	|*	|s	|t	|.
|ó	|[][,]{ }	|*	|y	|*	|u	|*	|*	|*	|*	|[26][S]\rarr	|b	|e	|z	|ł	|a	|d	|*	|*	|*	|e	|e	|.
|z	|u	|*	|b	|[27][S]\rarr	|p	|r	|z	|y	|d	|a	|w	|k	|a	|*	|k	|*	|*	|*	|*	|n	|k	|.
|g	|p	|*	|*	|*	|o	|*	|[28][S]\rarr	|n	|e	|u	|t	|r	|a	|l	|i	|z	|a	|c	|j	|a	|*	|.
|o	|r	|*	|[29][S]\rarr	|k	|r	|u	|p	|i	|e	|c	|*	|*	|[30][S]\rarr	|b	|e	|n	|n	|e	|t	|t	|*	|.
|w	|a	|[31][S]\rarr	|h	|u	|t	|y	|r	|a	|*	|*	|*	|[32][S]\rarr	|z	|a	|t	|o	|p	|e	|k	|*	|*	|.
|e	|w	|*	|*	|*	|*	|[33][S]\rarr	|m	|a	|r	|t	|w	|a	|[][,]{ }	|w	|o	|d	|a	|*	|*	|*	|*	|.
|*	|n	|*	|*	|*	|*	|*	|[34][S]\rarr	|t	|o	|m	|a	|s	|z	|e	|w	|s	|k	|i	|*	|*	|*	|.
|*	|a	|*	|[35][S]\rarr	|r	|a	|d	|c	|a	|[][,]{ }	|p	|r	|a	|w	|n	|y	|*	|*	|*	|*	|*	|*	|.
|*	|*	|[36][S]\rarr	|j	|e	|r	|a	|[][,]{ }	|w	|i	|ę	|k	|s	|z	|a	|*	|*	|*	|*	|*	|*	|*	|.\end{Puzzle}

\newpage

\begin{PuzzleClues}{\textbf{Poziome}\\}\Clue{1}{}{instytucja odpowiadająca za wypłatę emerytur, rent i innych świadczeń oraz pobieranie składek od ubezpieczonych osób}
\Clue{3}{}{tajna informacja, ostrzeżenie, wskazówka}
\Clue{5}{}{posługiwanie się rasą jako kategorią klasyfikującą i wartościującą ludzi}
\Clue{8}{}{TERIOLOGIA; dział zoologii zajmujący się badaniem ssaków}
\Clue{10}{}{karabin przeznaczony dla strzelca wyborowego w celu prowadzenia ognia pojedynczego na dużą odległość, z dużą precyzją, używany do likwidowania pojedynczych, istotnych dla przeciwnika celów (głównie osób), w odległości do kilkuset metrów}
\Clue{12}{}{ROZCIĘŻA krzew morskich wybrzeży strefy międzyzwrotnikowej}
\Clue{19}{}{pokrywa służąca do zamykania otworów w instrumencie muzycznym}
\Clue{22}{}{wełniak szary i największa z amerykańskich kapucynek, owocożerna}
\Clue{25}{}{sala do obrad plenarnych odbywających się w gmachu ważnej instytucji}
\Clue{26}{}{zamęt, zamieszanie z jakiegoś powodu; bałagan}
\Clue{27}{}{część zdania określająca rzeczownik lub zaimek rzeczowny}
\Clue{28}{}{likwidacja jakiegoś czynnika, którego wpływ oceniany jest jako negatywny, szkodliwy}
\Clue{29}{}{DZIARNINA; skrystalizowany miód pszczeli}
\Clue{30}{}{James, dziennikarz amerykański, inicjator międzynarodowych zawodników balonowych}
\Clue{31}{}{węgierski lekarz weterynarii (1860-1934); profesor Wyższej Szkoły Weterynaryjnej w Budapeszcie}
\Clue{32}{}{lekkoatleta czechosłowacki, długodystansowiec, zwany 'czeską lokomotywą' czterokrotny mistrz olimpijski z Londynu i Helsinek, 18-krotny rekordzista świata}
\Clue{33}{}{akwen, w którym fale wewnętrzne uniemożliwiają ruch jednostkom pływającym}
\Clue{34}{}{muzykolog i wydawca ur. w 1921 r., profesor A. M. w Krakowie}
\Clue{35}{}{prawnik świadczący pomoc prawną podmiotom gospodarczym, jednostkom organizacyjnym oraz osobom fizycznym, która polega przede wszystkim na udzielaniu porad prawnych, sporządzaniu opinii prawnych, opracowywaniu projektów aktów prawnych oraz występowaniu przed sądami i urzędami}
\Clue{36}{}{Jaera albifrons - gatunek skorupiaka z rzędu równonogów}\end{PuzzleClues}

\begin{PuzzleClues}{\textbf{Pionowe}\\}\Clue{1}{}{jezioro w zachodniej Irlandii, powierzchnia 176 km2, głębokość do 46 m}
\Clue{2}{}{dawniej; koń stepowy}
\Clue{3}{}{kanton w środkowej Szwajcarii, obszar 908 km2}
\Clue{4}{}{potocznie: papier pergaminowy}
\Clue{5}{}{przepływ krwi przez naczynia krwionośne mózgowia, zapewniający czynność ośrodkowego układu nerwowego}
\Clue{6}{}{Diapsida - grupa owodniowców z gromady zauropsydów (Sauropsida), w których czaszce rozwinęły się dwie pary otworów skroniowych (po dwa - górny i dolny - za każdym okiem), przez które przewleczone są silne mięśnie szczęk; dzięki temu czaszka stała się lżejsza a szczęki bardziej ruchliwe}
\Clue{7}{}{ozdobny krzew lub drzewko z Azji Wsch. i Ameryki Północnej, liście owłosione, kwiaty żółte}
\Clue{8}{}{program komputerowy lub urządzenie, którego zadaniem jest przechwytywanie i ewentualnie analizowanie danych przepływających w sieci}
\Clue{9}{}{Edwin, ur. w 1930r. astronauta amerykański, uczestniczył w pierwszej wyprawie załogowej na Księżyc}
\Clue{10}{}{wydawczyni}
\Clue{11}{}{niewielkich rozmiarów tarcza - obracająca się cześć telefonu, służąca do wybierania numeru}
\Clue{12}{}{roślina użytkowa nieeksploatowana ze stanowisk naturalnych, lecz z upraw stworzonych i pielęgnowanych przez człowieka}
\Clue{13}{}{pieniądze, które zostały złożone w banku, np. oszczędności}
\Clue{14}{}{drzewołazy}
\Clue{15}{}{poziomy wał z nawiniętą liną lub łańcuchem sterowym, połączony z kołem sterowym}
\Clue{16}{}{odrobina, kapka, ociupina, mała ilość}
\Clue{17}{}{dokument potwierdzający to, że ktoś ma jakiś status, przynależy gdzieś}
\Clue{18}{}{lewa strona}
\Clue{19}{}{manewr zbliżania samolotu do obiektu przeciwnika w celu zajęcia pozycji i wystrzelenia rakiety lub zrzucenia bomby}
\Clue{20}{}{jednostka informacji w systemie dwójkowym, oznaczająca 2\textasciicircum80 bajtów}
\Clue{21}{}{przeszkoda, przez którą przeskakują lekkoatleci podczas biegów}
\Clue{22}{}{w budownictwie - pionowy kanał w konstrukcji budynku}
\Clue{23}{}{część układu napędowego roweru - ułożyskowany wałek, wkręcany do tulei, zwanej mufą suportu, która znajduje się w miejscu łączenia rury podsiodłowej i rury dolnej ramy}
\Clue{24}{}{budynek rządowy, w którym odbywają się posiedzenia senatu}\end{PuzzleClues}\newpage%\section*{Krzyżówka 8}

\noindent\begin{Puzzle}{22}{22}|*	|*	|*	|*	|*	|*	|*	|*	|*	|*	|*	|*	|*	|*	|*	|*	|[1][S]\drarr	|g	|o	|ł	|ą	|b	|*	|.
|*	|*	|[2][S]\darr	|*	|[3][S]\darr	|[4][S]\darr	|*	|*	|[5][S]\rarr	|s	|i	|e	|d	|m	|i	|o	|b	|ó	|j	|*	|*	|*	|*	|.
|*	|*	|p	|*	|g	|ś	|*	|*	|*	|[6][S]\rarr	|k	|l	|o	|n	|i	|n	|a	|*	|*	|*	|*	|*	|*	|.
|*	|[7][S]\darr	|o	|*	|o	|w	|*	|*	|[8][S]\darr	|[9][S]\drarr	|g	|a	|r	|*	|[10][S]\rarr	|t	|r	|u	|c	|h	|ł	|o	|*	|.
|*	|n	|w	|*	|m	|i	|*	|*	|s	|m	|*	|[11][S]\rarr	|t	|r	|z	|c	|i	|n	|i	|a	|k	|*	|*	|.
|*	|o	|i	|*	|ó	|ę	|*	|[12][S]\darr	|e	|i	|[13][S]\darr	|*	|*	|[14][S]\rarr	|l	|a	|b	|i	|r	|y	|n	|t	|*	|.
|*	|c	|e	|*	|ł	|c	|*	|d	|r	|g	|p	|[15][S]\drarr	|a	|n	|i	|m	|a	|l	|n	|o	|ś	|ć	|*	|.
|*	|e	|ś	|[16][S]\rarr	|k	|o	|ł	|o	|w	|r	|o	|t	|e	|k	|*	|[17][S]\rarr	|l	|o	|t	|i	|*	|*	|*	|.
|*	|k	|ć	|*	|a	|n	|[18][S]\drarr	|d	|i	|a	|l	|e	|k	|t	|y	|k	|*	|*	|[19][S]\darr	|*	|*	|[20][S]\darr	|*	|.
|*	|[][,]{ }	|[][,]{ }	|*	|*	|e	|m	|a	|s	|c	|o	|l	|*	|*	|*	|[21][S]\darr	|*	|[22][S]\darr	|r	|[23][S]\darr	|[24][S]\darr	|t	|*	|.
|*	|b	|e	|*	|*	|*	|o	|t	|*	|j	|*	|e	|*	|*	|*	|s	|*	|p	|o	|k	|a	|a	|*	|.
|[25][S]\rarr	|e	|p	|i	|g	|e	|n	|e	|z	|a	|*	|f	|[26][S]\rarr	|p	|o	|m	|p	|a	|d	|o	|u	|r	|*	|.
|*	|c	|i	|*	|*	|*	|o	|k	|*	|[][,]{ }	|*	|o	|[27][S]\darr	|*	|*	|o	|*	|p	|n	|ń	|t	|c	|*	|.
|*	|h	|s	|*	|*	|*	|*	|*	|*	|z	|*	|n	|o	|*	|[28][S]\darr	|l	|*	|a	|i	|[][,]{ }	|o	|z	|*	|.
|*	|s	|t	|[29][S]\rarr	|s	|ł	|o	|m	|i	|a	|k	|*	|b	|*	|l	|u	|*	|t	|a	|d	|m	|o	|*	|.
|*	|t	|o	|*	|*	|*	|[30][S]\rarr	|e	|k	|s	|p	|o	|r	|t	|a	|c	|j	|a	|*	|u	|a	|w	|*	|.
|*	|e	|l	|[31][S]\rarr	|f	|l	|e	|t	|n	|i	|a	|*	|a	|[32][S]\darr	|m	|h	|*	|c	|*	|n	|t	|n	|*	|.
|*	|i	|a	|[33][S]\rarr	|s	|z	|y	|p	|u	|ł	|a	|*	|z	|n	|a	|*	|*	|z	|*	|a	|*	|i	|*	|.
|*	|n	|r	|*	|*	|*	|[34][S]\rarr	|s	|a	|k	|l	|a	|*	|a	|*	|*	|*	|*	|*	|j	|*	|c	|*	|.
|*	|a	|n	|[35][S]\rarr	|m	|i	|a	|s	|t	|o	|[][,]{ }	|o	|t	|w	|a	|r	|t	|e	|*	|s	|*	|a	|*	|.
|*	|*	|a	|*	|*	|*	|*	|*	|[36][S]\rarr	|w	|e	|b	|m	|a	|s	|t	|e	|r	|*	|k	|*	|*	|*	|.
|*	|*	|*	|*	|[37][S]\rarr	|ł	|y	|c	|z	|a	|k	|ó	|w	|*	|*	|[38][S]\rarr	|c	|u	|k	|i	|e	|r	|*	|.
|[39][S]\rarr	|w	|e	|l	|o	|ń	|s	|k	|i	|*	|*	|*	|*	|*	|*	|*	|*	|*	|*	|*	|*	|*	|*	|.\end{Puzzle}

\newpage

\begin{PuzzleClues}{\textbf{Poziome}\\}\Clue{1}{}{ptak; poszczególne gatunki tego ptaka w taksonomii biologicznej klasyfikowane są w obrębie rodziny gołębiowatych (Columbidae), w podrodzinie gołębi (Columbinae)}
\Clue{5}{}{dyscyplina sportowa składająca się z siedmiu konkurencji}
\Clue{6}{}{klony - drzewa}
\Clue{9}{}{garnek pokaźnych rozmiarów}
\Clue{10}{}{pogardliwie o starszym człowieku}
\Clue{11}{}{ptak nawodny z rzędu wróblowatych; buduje koszyczkowe gniazdo zawieszone na trzcinach; Eurazja, płn. Afryka; chroniony}
\Clue{14}{}{gąszcz, plątanina; w znaczeniu przenośnym o skomplikowanej sytuacji, w której trudno się zorientować}
\Clue{15}{}{zwierzęcość; bycie jak zwierzę (czyli istota nieuduchowiona)}
\Clue{16}{}{walec obracający się pod wpływem ruchów umieszczonego wewnątrz stworzenia}
\Clue{17}{}{właściwie Viaud - pisarz francuski (1850-1923), powieści z życia marynarzy i rybaków; „Rybak islandzki”}
\Clue{18}{}{filozof uprawiający dialektykę}
\Clue{25}{}{teoria biologiczna, według której rozwój embrionalny przebiega z niezróżnicowanych początkowo komórek zygoty, które różnicują się dopiero na dalszych etapach wzrostu embrionu, prowadząc do utworzenia narządów i ich układów}
\Clue{26}{}{karmazynowy róż}
\Clue{29}{}{ul słomiany}
\Clue{30}{}{wyprowadzenie zwłok na miejsce, gdzie pozostają do pogrzebu}
\Clue{31}{}{prosty instrument muzyczny składający się z szeregu złączonych razem piszczałek różnej długości}
\Clue{33}{}{część organu, przez którą przechodzi wiązka nerwowa lub naczyniowa}
\Clue{34}{}{dom mieszkalny kaukaskich górali z kamienia, gliny lub drewna}
\Clue{35}{}{miasto, które na czas konfliktu zbrojnego zostało ogłoszone przez organy sprawujące nad nim władzę miastem niebronionym oraz takie miasto, które zgodnie z prawem międzynarodowym nie może zostać zbombardowane}
\Clue{36}{}{osobę zajmującą się projektowaniem, kodowaniem, szatą graficzną oraz aktualizacją witryny internetowej}
\Clue{37}{}{Łyczaków - dzielnica Lwowa}
\Clue{38}{}{słodka w smaku przyprawa, zwykle w postaci krystalizowanej sacharozy}
\Clue{39}{}{włoski malarz i rzeźbiarz (1509-66) obrazy i freski religijne}\end{PuzzleClues}

\begin{PuzzleClues}{\textbf{Pionowe}\\}\Clue{1}{}{niedźwiedź z lasów Ameryki Płd}
\Clue{2}{}{powieść skonstruowana w formie listów (niekiedy przeplatających się z fragmentami pamiętnika), wymienianych między sobą przez bohaterów}
\Clue{3}{}{mała szybka z ołowianym łączeniem; kiedyś szybki takie stosowane powszechnie w oknach, dziś - do celów dekoracyjnych (głównie w witrażach)}
\Clue{4}{}{nazwa pokarmów (głównie mięsa, jaj, chrzanu, chleba etc.) święconych w Wielką Sobotę w kościołach katolickich Polski oraz w graniczących ze Słowenią austriackich regionach Styrii, Karyntii, południowego Tyrolu oraz Bawarii}
\Clue{7}{}{Myotis bechsteinii - gatunek ssaka z rzędu nietoperzy z rodziny mroczkowatych; w Polsce jego zasięg ogranicza się do południowej części kraju, najdalej na północ sięga do Cedyńskiego Parku Krajobrazowego, Wielkopolski, okolic Tomaszowa Mazowieckiego i Polesia Lubelskiego}
\Clue{8}{}{punkt usługowy}
\Clue{9}{}{rodzaj migracji będący społecznym obciążeniem dla krajów przyjmujących imigrantów, ponieważ ponoszą one dodatkowe koszty zapewnienia tym osobom zakwaterowania, pomocy prawnej i usług służby zdrowia}
\Clue{12}{}{akcesorium w postaci np. biżuterii, szala, krawatu, będące ozdobą, dopełnieniem ubioru}
\Clue{13}{}{volkswagen z modelu Polo}
\Clue{15}{}{telefonia, podłączenie do sieci telefonicznej}
\Clue{18}{}{jezioro w USA u podnóża gór Sierra Nevada}
\Clue{19}{}{żeński organ rozmnażania się mszaków, paprotników i roślin nagozalążkowych}
\Clue{20}{}{ZŁOTOROST}
\Clue{21}{}{człowiek o śniadym lub ciemnym kolorze skóry, rozpoznawany jako nienależący do kultury białych ludzi}
\Clue{22}{}{włoskie ciastko z ciasta drożdżowego z rodzynkami i cynamonem}
\Clue{23}{}{jedna z ras konia, ukształtowana dopiero na początku XX wieku, poprzez skrzyżowanie oryginalnego ogiera noniusa z klaczami gidran; używany jako wierzchowiec i jako silny koń do lekkich zaprzęgów}
\Clue{24}{}{maszyna wykonująca cały cykl swej pracy bez udziału człowieka}
\Clue{27}{}{w matematyce - zbiór wszystkich wartości (należących do przeciwdziedziny) przyjmowanych przez funkcję dla każdego elementu danego podzbioru jej dziedziny oraz zbiór wszystkich elementów dziedziny, które są odwzorowywane na elementy danego podzbioru przeciwdziedziny}
\Clue{28}{}{duchowny buddyjski w Tybecie i Mongolii}
\Clue{32}{}{składowa część kościoła położona pomiędzy prezbiterium a kruchtą, przeznaczona dla wiernych}\end{PuzzleClues}\newpage%\section*{Krzyżówka 9}

\noindent\begin{Puzzle}{24}{31}|*	|*	|*	|*	|*	|*	|*	|*	|*	|*	|*	|*	|*	|[1][S]\drarr	|m	|ó	|z	|g	|*	|[2][S]\darr	|*	|*	|[3][S]\darr	|[4][S]\darr	|*	|.
|*	|*	|*	|*	|[5][S]\rarr	|f	|o	|t	|o	|c	|h	|r	|o	|m	|y	|*	|*	|[6][S]\darr	|*	|f	|*	|[7][S]\darr	|k	|r	|[8][S]\darr	|.
|*	|*	|*	|*	|*	|*	|*	|*	|*	|*	|[9][S]\darr	|*	|*	|u	|*	|*	|*	|s	|*	|i	|*	|t	|o	|e	|c	|.
|*	|*	|*	|*	|*	|*	|*	|*	|*	|*	|p	|[10][S]\rarr	|s	|s	|a	|k	|*	|z	|[11][S]\darr	|l	|*	|o	|p	|p	|o	|.
|*	|*	|*	|*	|*	|*	|[12][S]\rarr	|k	|o	|m	|i	|n	|*	|z	|[13][S]\darr	|*	|[14][S]\darr	|k	|p	|m	|[15][S]\darr	|l	|u	|r	|l	|.
|*	|*	|*	|*	|*	|*	|*	|*	|[16][S]\drarr	|w	|e	|k	|*	|k	|e	|*	|n	|o	|r	|[][,]{ }	|s	|e	|ł	|e	|u	|.
|*	|*	|*	|[17][S]\rarr	|p	|r	|z	|y	|p	|o	|r	|a	|*	|i	|p	|*	|a	|l	|o	|m	|t	|r	|a	|z	|m	|.
|*	|[18][S]\rarr	|s	|z	|t	|y	|l	|p	|a	|*	|c	|*	|*	|e	|i	|*	|d	|a	|k	|u	|a	|a	|[][,]{ }	|e	|b	|.
|*	|*	|[19][S]\rarr	|k	|o	|s	|o	|d	|r	|z	|e	|w	|*	|t	|p	|*	|ś	|r	|u	|z	|r	|n	|p	|n	|u	|.
|*	|*	|[20][S]\drarr	|l	|o	|d	|ó	|w	|k	|a	|*	|*	|*	|e	|s	|[21][S]\darr	|w	|s	|r	|y	|z	|c	|a	|t	|s	|.
|*	|*	|a	|[22][S]\darr	|*	|*	|[23][S]\drarr	|m	|a	|s	|a	|[][,]{ }	|g	|r	|a	|w	|i	|t	|a	|c	|y	|j	|n	|a	|*	|.
|[24][S]\rarr	|p	|l	|a	|m	|i	|c	|a	|*	|*	|*	|*	|*	|*	|m	|a	|e	|w	|t	|z	|*	|a	|c	|c	|*	|.
|*	|*	|m	|n	|*	|[25][S]\darr	|i	|[26][S]\drarr	|k	|a	|c	|y	|k	|*	|m	|d	|t	|o	|o	|n	|*	|*	|e	|j	|*	|.
|*	|*	|a	|t	|[27][S]\drarr	|g	|a	|l	|o	|i	|s	|*	|*	|*	|o	|l	|l	|*	|r	|y	|[28][S]\darr	|*	|r	|a	|*	|.
|*	|[29][S]\darr	|[][,]{ }	|i	|a	|o	|ł	|i	|*	|[30][S]\drarr	|c	|u	|g	|*	|n	|i	|n	|*	|[][,]{ }	|*	|o	|*	|n	|*	|*	|.
|*	|k	|m	|g	|m	|u	|o	|t	|[31][S]\drarr	|s	|w	|i	|n	|g	|*	|w	|a	|[32][S]\rarr	|g	|i	|k	|*	|a	|[33][S]\darr	|*	|.
|*	|a	|a	|u	|b	|d	|[][,]{ }	|e	|s	|y	|*	|[34][S]\drarr	|ż	|a	|k	|o	|*	|*	|e	|*	|n	|*	|*	|h	|*	|.
|*	|s	|t	|a	|a	|a	|n	|r	|t	|g	|*	|m	|[35][S]\darr	|*	|*	|ś	|*	|*	|n	|*	|o	|*	|[36][S]\darr	|e	|*	|.
|*	|z	|e	|*	|*	|*	|i	|a	|a	|n	|[37][S]\darr	|o	|d	|[38][S]\darr	|*	|ć	|*	|*	|e	|*	|*	|[39][S]\darr	|k	|b	|*	|.
|*	|t	|r	|*	|[40][S]\drarr	|t	|e	|t	|r	|a	|s	|t	|y	|c	|h	|*	|*	|*	|r	|*	|*	|b	|l	|a	|*	|.
|*	|e	|*	|*	|s	|*	|b	|u	|a	|ł	|e	|y	|m	|z	|*	|*	|[41][S]\drarr	|g	|a	|u	|g	|u	|i	|n	|*	|.
|*	|l	|*	|[42][S]\darr	|i	|*	|i	|r	|*	|[][,]{ }	|r	|l	|i	|a	|*	|*	|c	|*	|l	|*	|*	|d	|n	|o	|*	|.
|*	|a	|[43][S]\darr	|o	|e	|*	|e	|a	|*	|c	|p	|i	|s	|s	|*	|*	|h	|*	|n	|*	|*	|y	|*	|w	|*	|.
|*	|ń	|k	|t	|s	|[44][S]\darr	|s	|[][,]{ }	|*	|y	|e	|c	|j	|z	|[45][S]\rarr	|f	|o	|r	|y	|ś	|*	|ń	|*	|i	|*	|.
|*	|s	|a	|t	|i	|t	|k	|l	|[46][S]\darr	|f	|n	|a	|a	|a	|*	|*	|m	|*	|*	|*	|*	|*	|*	|e	|*	|.
|*	|t	|n	|o	|e	|e	|i	|a	|p	|r	|t	|*	|*	|*	|*	|*	|i	|*	|*	|*	|*	|*	|*	|c	|*	|.
|*	|w	|o	|n	|n	|m	|e	|g	|e	|o	|y	|*	|*	|[47][S]\rarr	|p	|e	|c	|h	|s	|t	|e	|i	|n	|*	|*	|.
|[48][S]\rarr	|o	|p	|o	|k	|a	|*	|r	|r	|w	|n	|*	|[49][S]\rarr	|n	|i	|e	|z	|g	|o	|d	|n	|o	|ś	|ć	|*	|.
|*	|*	|a	|w	|i	|t	|*	|o	|r	|y	|*	|*	|[50][S]\rarr	|c	|h	|l	|e	|b	|o	|d	|a	|w	|c	|a	|*	|.
|*	|*	|*	|i	|*	|*	|*	|w	|e	|*	|*	|*	|*	|*	|[51][S]\rarr	|s	|k	|a	|j	|l	|a	|j	|t	|*	|*	|.
|[52][S]\rarr	|n	|i	|e	|o	|s	|t	|a	|t	|e	|c	|z	|n	|o	|ś	|ć	|*	|*	|*	|*	|*	|*	|*	|*	|*	|.
|*	|*	|*	|*	|*	|*	|*	|*	|*	|*	|*	|*	|*	|*	|*	|*	|*	|*	|*	|*	|*	|*	|*	|*	|*	|.\end{Puzzle}

\newpage

\begin{PuzzleClues}{\textbf{Poziome}\\}\Clue{1}{}{osoba o nieprzeciętnym umyśle}
\Clue{5}{}{przyrząd optyczny, którego soczewki zmieniają kolor wpływem światła}
\Clue{10}{}{ssąca głowica pogłębiarki}
\Clue{12}{}{piec kuchenny}
\Clue{16}{}{słoik, który służy do przechowywania przetworów}
\Clue{17}{}{rusztowanie, które ma zabezpieczać ściany przed zawaleniem}
\Clue{18}{}{ochronny mankiet zakładany na rękaw np. munduru}
\Clue{19}{}{zwyczajowa nazwa sosny górskiej}
\Clue{20}{}{gatunek kaczki o biało-brunatno-czarnym upierzeniu; w Polsce na przelotach, zamieszkuje obszary mórz dalekiej północy; łowna}
\Clue{23}{}{masa, która wynika z oddziaływania grawitacyjnego}
\Clue{24}{}{zakaźna choroba larw jedwabników}
\Clue{26}{}{środkowoamerykański ptak z rzędu wróblowatych}
\Clue{27}{}{matematyk francuski (1811-32); twórca nowoczesnej teorii równań algebraicznych oraz teorii gry}
\Clue{30}{}{dość intensywny ruch powietrza}
\Clue{31}{}{jeden ze stylów jazzowych o nieregularnym rytmie}
\Clue{32}{}{bom na małych jachtach śródlądowych}
\Clue{34}{}{afrykański ptak z rzędu papug o szarym upierzeniu i różowym ogonie, umie naśladować mowę - papuga popielata}
\Clue{40}{}{stofa licząca 4 wersy}
\Clue{41}{}{samodzielne muzeum malarstwa, rzeźby lub salon wystawowy połączony ze sprzedażą dzieł sztuki}
\Clue{45}{}{w dawnym wojsku: konny ordynans oficera}
\Clue{47}{}{niemiecki malarz i grafik (1881-1955) reprezentant ekspresjonizmu; kompozycje figuralne. pejzaże, portrety}
\Clue{48}{}{skała mieszana, zbudowana z organogenicznej krzemionki (opal, chalcedon, przeważnie pochodzące z krzemionkowych gąbek) i węglanu wapnia}
\Clue{49}{}{brak zgodności z czymś, sprzeczność}
\Clue{50}{}{pracodawca; osoba zatrudniająca pracownika}
\Clue{51}{}{oszklone okienko w pokładzie statku lub dachu kabiny, które przepuszcza światło z zewnątrz}
\Clue{52}{}{cecha decyzji, wyroku, który można zmienić}\end{PuzzleClues}

\begin{PuzzleClues}{\textbf{Pionowe}\\}\Clue{1}{}{żołnierz, którego podstawowym uzbrojeniem był muszkiet oraz rapier; formacje złożone z muszkieterów występowały w XVI-XVII wieku w niemal wszystkich armiach europejskich i nie tylko}
\Clue{2}{}{gatunek filmowy, w którym przeplatają się sceny mówione, śpiewane i taneczne}
\Clue{3}{}{stała, nieruchoma pancerna czasza, najczęściej wykonana jako odlew ze staliwa lub ze stalowych płyt połączonych nitami, służąca do celów obserwacyjnych lub jako stanowisko ogniowe dział lub karabinów maszynowych}
\Clue{4}{}{grupa sportowców, którzy w rozgrywkach reprezentują dany kraj lub klub}
\Clue{6}{}{opieranie się w swych sądach, decyzjach, postępowaniu na szkolnych, uproszczonych formułach}
\Clue{7}{}{zdolność do znoszenia rzeczy ocenianych jako negatywne, umiejętność lub chęć do przymykania oka na czyny, zachowania, sprawy naganne}
\Clue{8}{}{miasto w USA (Georgia) nad rzeką Chattahooche; przemysł lotniczy, samochodowy, elektroniczny, maszynowy}
\Clue{9}{}{amerykański filozof i logik (1839-1914); jeden z twórców pragmatyzmu}
\Clue{11}{}{naczelny organ prokuratury}
\Clue{13}{}{zespół organizmów żyjących na powierzchni ziarenek piasku dna zbiornika wodnego}
\Clue{14}{}{prędkość większa od prędkości światła}
\Clue{15}{}{PLUR. czyiś rodzice}
\Clue{16}{}{gruba, dłuższa kurtka wypełniona naturalnym lub syntetycznym puchem, z wyściełanym futrem kapturem}
\Clue{20}{}{podniosłe określenie szkoły wyższej, najczęściej uniwersytetu}
\Clue{21}{}{o przedmiotach, ludziach, zwierzętach, zjawiskach naturalnych - niepełność, niekompletność, choroba, także uszkodzenie}
\Clue{22}{}{miasto w płd. Gwatemali; ośrodek handlowy i turystyczny}
\Clue{23}{}{każdy naturalny obiekt fizyczny oraz układ powiązanych ze sobą obiektów lub ich struktur występujący w przestrzeni kosmicznej poza granicą atmosfery ziemskiej}
\Clue{25}{}{gatunek sera twardego, podpuszczkowego, dojrzewającego, produkowanego z mleka krowiego}
\Clue{26}{}{ogół utworów związanych tematycznie z funkcjonowaniem hitlerowskich obozów koncentracyjnych (lagrów) w czasie II wojny światowej}
\Clue{27}{}{trudne położenie, sytuacja bez wyjścia}
\Clue{28}{}{luka lub prześwit w jakiejś ciemniejszej materii, pozwalająca na spojrzenie na drugą stronę}
\Clue{29}{}{urząd lokalny w średniowiecznej Polsce}
\Clue{30}{}{sygnał, którego dziedzina i zbiór wartości są dyskretne w czasie}
\Clue{31}{}{czyjaś małżonka, żona}
\Clue{33}{}{PERSYMONA, KAKI}
\Clue{34}{}{BARCIAK}
\Clue{35}{}{rezygnacja z zajmowanego stanowiska lub urzędu}
\Clue{36}{}{porcja alkoholu, którą pije się rano, gdy ma się kaca w nadziei, że przejdzie}
\Clue{37}{}{minerał zaliczany do krzemianów warstwowych}
\Clue{38}{}{rozpraszacz lub odbłyśnik umieszczony pod lampą}
\Clue{39}{}{porcja budyniu, tj. proszku, z którego po dodaniu mleka otrzymuje się budyń}
\Clue{40}{}{wielkopolski ludowy instrument muzyczny}
\Clue{41}{}{chomik}
\Clue{42}{}{okres panowania Ottonów w Świętym Cesarstwie Rzymskim}
\Clue{43}{}{egipska lub etruska urna do przechowywania wnętrzności zmarłego}
\Clue{44}{}{zagadnienie omawiane w szkole, zazwyczaj w ramach jednej jednostki lekcyjnej}
\Clue{46}{}{architekt francuski (1874-1954), zwany też ojcem żelbetu - ratusz w Hawrze}\end{PuzzleClues}\newpage%\section*{Krzyżówka 10}

\noindent\begin{Puzzle}{24}{31}|*	|*	|*	|[1][S]\darr	|*	|*	|*	|*	|*	|*	|*	|*	|*	|*	|*	|*	|*	|*	|[2][S]\darr	|*	|*	|*	|*	|*	|[3][S]\darr	|.
|*	|[4][S]\rarr	|p	|i	|e	|ś	|n	|i	|a	|r	|k	|a	|*	|*	|[5][S]\darr	|*	|*	|*	|b	|[6][S]\darr	|*	|[7][S]\darr	|*	|*	|w	|.
|[8][S]\drarr	|j	|u	|n	|i	|o	|r	|k	|a	|[][,]{ }	|m	|ł	|o	|d	|s	|z	|a	|*	|l	|p	|*	|h	|[9][S]\darr	|*	|a	|.
|o	|*	|[10][S]\darr	|t	|*	|[11][S]\drarr	|p	|r	|u	|s	|k	|i	|[][,]{ }	|d	|r	|y	|l	|*	|e	|r	|*	|a	|n	|*	|r	|.
|g	|*	|l	|e	|[12][S]\rarr	|s	|p	|ó	|ł	|g	|ł	|o	|s	|k	|a	|[][,]{ }	|c	|i	|s	|z	|ą	|c	|a	|*	|i	|.
|r	|*	|e	|r	|*	|ł	|*	|*	|*	|[13][S]\darr	|*	|[14][S]\rarr	|o	|d	|l	|e	|w	|*	|b	|e	|*	|k	|l	|*	|a	|.
|ó	|[15][S]\darr	|m	|e	|[16][S]\darr	|o	|*	|[17][S]\rarr	|m	|o	|u	|n	|d	|o	|u	|*	|*	|[18][S]\darr	|o	|g	|*	|n	|e	|*	|n	|.
|d	|s	|n	|s	|c	|w	|*	|*	|*	|b	|*	|*	|[19][S]\darr	|*	|c	|*	|*	|s	|k	|i	|[20][S]\darr	|e	|ż	|*	|t	|.
|e	|t	|i	|*	|i	|i	|*	|*	|*	|n	|*	|[21][S]\darr	|r	|*	|h	|*	|*	|u	|*	|ę	|p	|y	|n	|*	|y	|.
|k	|y	|s	|*	|s	|k	|[22][S]\darr	|*	|[23][S]\darr	|i	|*	|i	|e	|*	|*	|*	|*	|b	|*	|c	|a	|*	|o	|*	|w	|.
|[][,]{ }	|l	|k	|*	|o	|*	|r	|*	|c	|ż	|*	|r	|f	|*	|*	|[24][S]\darr	|*	|s	|[25][S]\darr	|i	|s	|[26][S]\darr	|ś	|*	|n	|.
|l	|[][,]{ }	|a	|[27][S]\darr	|l	|*	|a	|*	|y	|o	|[28][S]\darr	|a	|o	|*	|*	|k	|*	|k	|k	|e	|k	|p	|ć	|[29][S]\darr	|o	|.
|e	|z	|t	|t	|i	|*	|p	|*	|r	|w	|h	|n	|r	|*	|*	|a	|*	|r	|o	|*	|o	|u	|[][,]{ }	|i	|ś	|.
|t	|a	|a	|e	|s	|*	|t	|*	|a	|a	|e	|i	|m	|*	|*	|s	|*	|y	|p	|[30][S]\darr	|w	|d	|p	|m	|ć	|.
|n	|k	|[][,]{ }	|o	|t	|*	|u	|[31][S]\darr	|n	|t	|l	|s	|a	|*	|*	|a	|[32][S]\rarr	|p	|a	|s	|i	|e	|r	|b	|*	|.
|i	|o	|b	|r	|[][,]{ }	|*	|l	|n	|e	|e	|i	|t	|c	|*	|*	|c	|*	|c	|n	|y	|k	|ł	|z	|i	|*	|.
|*	|p	|e	|i	|g	|*	|a	|i	|c	|*	|n	|a	|j	|*	|*	|j	|*	|j	|i	|t	|[][,]{ }	|e	|y	|r	|*	|.
|*	|i	|r	|a	|ę	|[33][S]\drarr	|r	|e	|z	|y	|g	|n	|a	|c	|j	|a	|*	|a	|c	|n	|s	|c	|w	|[][,]{ }	|*	|.
|*	|a	|n	|[][,]{ }	|s	|d	|z	|k	|k	|*	|*	|*	|*	|*	|*	|*	|*	|*	|a	|i	|z	|z	|o	|m	|*	|.
|*	|ń	|o	|e	|t	|ż	|*	|l	|a	|*	|[34][S]\rarr	|c	|o	|m	|p	|a	|c	|t	|*	|k	|a	|n	|z	|a	|*	|.
|[35][S]\drarr	|s	|u	|w	|n	|i	|c	|a	|[][,]{ }	|o	|d	|l	|e	|w	|n	|i	|c	|z	|a	|*	|r	|i	|o	|l	|*	|.
|p	|k	|l	|o	|o	|n	|[36][S]\drarr	|r	|z	|a	|d	|z	|i	|z	|n	|a	|*	|*	|*	|*	|o	|k	|w	|a	|*	|.
|e	|i	|l	|l	|l	|s	|s	|o	|w	|*	|[37][S]\darr	|*	|*	|*	|*	|*	|[38][S]\darr	|*	|*	|*	|g	|[][,]{ }	|a	|j	|*	|.
|t	|*	|i	|u	|i	|y	|t	|w	|y	|*	|k	|*	|*	|[39][S]\rarr	|m	|e	|t	|a	|l	|*	|ł	|l	|*	|s	|*	|.
|y	|*	|e	|c	|s	|*	|y	|n	|c	|*	|i	|*	|*	|[40][S]\rarr	|z	|a	|r	|k	|a	|*	|o	|i	|[41][S]\darr	|k	|*	|.
|h	|*	|g	|j	|t	|[42][S]\rarr	|k	|o	|z	|i	|b	|r	|ó	|d	|*	|*	|a	|*	|*	|*	|w	|s	|b	|i	|*	|.
|o	|*	|o	|i	|n	|*	|a	|ś	|a	|*	|i	|*	|*	|[43][S]\rarr	|c	|y	|p	|*	|*	|*	|y	|t	|u	|*	|*	|.
|r	|*	|*	|*	|y	|*	|*	|ć	|j	|*	|t	|*	|*	|*	|[44][S]\rarr	|l	|o	|g	|*	|*	|*	|w	|l	|[45][S]\darr	|*	|.
|z	|*	|*	|*	|*	|*	|*	|*	|n	|[46][S]\rarr	|k	|a	|r	|t	|a	|[][,]{ }	|w	|i	|z	|y	|t	|o	|w	|a	|*	|.
|e	|[47][S]\rarr	|p	|r	|z	|y	|s	|t	|a	|w	|a	|n	|i	|e	|*	|*	|y	|*	|*	|*	|*	|w	|a	|n	|*	|.
|c	|*	|[48][S]\rarr	|g	|w	|a	|ł	|t	|*	|*	|*	|*	|*	|*	|*	|*	|*	|*	|*	|*	|*	|y	|*	|a	|*	|.
|*	|[49][S]\rarr	|s	|k	|u	|b	|u	|n	|[][,]{ }	|k	|l	|e	|s	|z	|c	|z	|o	|w	|n	|i	|k	|*	|*	|*	|*	|.\end{Puzzle}

\newpage

\begin{PuzzleClues}{\textbf{Poziome}\\}\Clue{4}{}{poetka}
\Clue{8}{}{zawodniczka sportowa, która rywalizuje w najmłodszej grupie wiekowej}
\Clue{11}{}{powszechna nazwa, jaką obdarza się nową metodę szkolenia i musztry żołnierzy, jaką wprowadził w latach dwudziestych i trzydziestych XVIII w. w Prusach Leopold von Anhalt-Dessau}
\Clue{12}{}{spółgłoska szczelinowa lub zwarto-szczelinowa artykułowana przez wysklepienie języka w kierunku przedniej części podniebienia przy jednoczesnym kontakcie czubka języka z dziąsłami - ś, ź, ć, dź, ń; określenie używane odnośnie do polskich spółgłosek środkowojęzykowych}
\Clue{14}{}{produkt otrzymywany przez wypełnienie formy tworzywem w stanie ciekłym}
\Clue{17}{}{miasto w płd. Czadzie, nad rzeką Logone, ośrodek handlowy regionu uprawy bawełny}
\Clue{32}{}{syn męża lub żony z poprzedniego związku}
\Clue{33}{}{decyja wydana przez kogoś, w której (na piśmie lub ustnie) zostaje wyrażona chęć zrezygnowania z czegoś, zrzeczenia się czegoś}
\Clue{34}{}{płyta CD}
\Clue{35}{}{suwnica pomostowa przeznaczona do pracy w odlewniach stali}
\Clue{36}{}{to, że coś jest bardzo rzadkie, niegęste}
\Clue{39}{}{podgatunek muzyki rockowej powstały na przełomie lat 60. i 70. XX wieku w Wielkiej Brytanii i Stanach Zjednoczonych}
\Clue{40}{}{miasto w płn. Jordanii na płn.-wsch od Ammanu; rafineria ropy naftowej}
\Clue{42}{}{SALSEFIA roślina zielna ze złożonych, kozibród lekarski zwany jest salsefią}
\Clue{43}{}{kod ISO 4217 funta cypryjskiego}
\Clue{44}{}{urządzenie do pomiaru prędkości statku wodnego względem wody}
\Clue{46}{}{cecha, sposób zachowania, charakterystyczny element stanowiący znak rozpoznawczy jakiejś osoby lub grupy osób; coś, co decyduje o tym, jak ktoś jest postrzegany przez otoczenie}
\Clue{47}{}{w geometrii relacja równoważności figur zdefiniowana przez izometrię, rozumianą intuicyjnie jako identyczność kształtu i wielkości figury}
\Clue{48}{}{zaprzeczenie jakiemuś ustalonemu stanowi, normie}
\Clue{49}{}{Ischyropsalis helwigii - gatunek pajęczaka z rzędu kosarzy, z rodziny Ischyropsalididae}\end{PuzzleClues}

\begin{PuzzleClues}{\textbf{Pionowe}\\}\Clue{1}{}{coś do załatwienia, do zrobienia}
\Clue{2}{}{Damaliscus pygargus phillipsi - podgatunek bonteboka, ssaka parzystokopytnego z rodziny krętorogich; zasiedla tereny wschodniej i środkowej części Afryki Południowej}
\Clue{3}{}{to, że coś ma kilka wariantów, istnieje w kilku wariantach, odmianach, wersjach}
\Clue{5}{}{lekceważące i pogardliwe najczęściej zdradzające negatywne nastawienie (ale też dowcipne, może być użyte pieszczotliwie) określenie dziecka}
\Clue{6}{}{zdecydowana, absolutna przesada}
\Clue{7}{}{kuc Hackney - rasa prawdziwych kuców z typowym dla nich charakterem, dzieło hodowcy Christophera Wilsona z Kirkby Lonsdale w Kumbrii; dzięki ciekawemu i bardzo charakterystycznemu sposobowi ruchu można je obecnie często spotkać na różnego rodzaju pokazach}
\Clue{8}{}{zewnętrzna część lokalu gastronomicznego ustawiana na zewnątrz lokalu w okresie wiosenno-letnim}
\Clue{9}{}{cło lub podobna opłata o równoważnym skutku należna przy wwozie towarów do danego kraju}
\Clue{10}{}{krzywa będąca zbiorem punktów, dla której iloczyn odległości od dwóćh ognisk: F(-a, 0) i F (a, 0) wynosi a do potęgi drugiej}
\Clue{11}{}{chrząszcz z rodziny ryjkowców}
\Clue{13}{}{Liocranidae - rodzina pająków z podrzędu Opisthothelae}
\Clue{15}{}{styl architektoniczny wprowadzony przez Stanisława Witkiewicza w latach 90. XIX wzorowany na tradycyjnym budownictwie górali podhalańskich i wzbogacający je elementami secesji}
\Clue{16}{}{Taxiphyllum densifolium - gatunek mchu z rodziny rokietowatych}
\Clue{18}{}{forma tzw. zwykłego podwyższenia kapitału zakładowego w spółce akcyjnej (w odróżnieniu od podwyższenia warunkowego i celowego) - zarząd tej spółki oferuje w drodze ogłoszenia akcje, co do których służy akcjonariuszom prawo poboru}
\Clue{19}{}{ruch religijno-polityczno-społeczny zapoczątkowany przez Marcina Lutra w XVI wieku, mający na celu odnowę chrześcijaństwa}
\Clue{20}{}{Ptiloprora perstriata - gatunek ptaka z rodziny miodojadów (Meliphagidae)}
\Clue{21}{}{fikcyjna kraina stworzona przez Roberta E. Howarda w cyklu powieści fantasy, których bohaterem jest Conan Barbarzyńca i które kontynuowane były przez innych autorów}
\Clue{22}{}{rodzaj księgi parafialnej, w której proboszczowie zapisywali dane o zdarzeniach podlegających rejestracji w chwili ich zgłaszania lub bezpośrednio po nich}
\Clue{23}{}{cyraneczka, Anas crecca - gatunek średniego, wędrownego ptaka wodnego z rodziny kaczkowatych (Anatidae)}
\Clue{24}{}{likwidacja czegoś, unieważnienie}
\Clue{25}{}{wieś w Polsce położona w województwie wielkopolskim, w powiecie wolsztyńskim, w gminie Siedlec}
\Clue{26}{}{Pyxidea mouhotii - gatunek żółwia z rodziny batagurowatych}
\Clue{27}{}{w potocznym rozumieniu - darwinizm}
\Clue{28}{}{stanowisko budowy małych statków wodnych, szybowców itp}
\Clue{29}{}{Zingiber malaysianum - gatunek rośliny należący do rodziny imbirowatych}
\Clue{30}{}{saturator}
\Clue{31}{}{nieprzejrzystość, niejasność czegoś, to, że coś budzi wątpliwości, nie jest jednoznaczne}
\Clue{33}{}{spodnie wzorowane na spodniach farmerskich, płócienne lub drelichowe nabijane metalowymi nitami}
\Clue{35}{}{pancerny żołnierz na Litwie w XVI/XVIII w}
\Clue{36}{}{malarz i poeta ludowy (1849-1922) pochodził z Podhala}
\Clue{37}{}{rodzaj namiotu koczowniczego ludów z Azji}
\Clue{38}{}{marynarz pełniący służbę przy trapie}
\Clue{41}{}{gula, zgrubienie}
\Clue{45}{}{miasto w Mezopotamii nad Eufratem}\end{PuzzleClues}\newpage%\section*{Krzyżówka 11}

\noindent\begin{Puzzle}{21}{31}|*	|[1][S]\darr	|*	|[2][S]\darr	|*	|*	|*	|*	|*	|*	|*	|*	|*	|*	|*	|*	|*	|*	|*	|[3][S]\darr	|*	|*	|.
|*	|i	|[4][S]\drarr	|k	|u	|c	|h	|n	|i	|a	|*	|*	|*	|[5][S]\drarr	|d	|e	|k	|i	|e	|l	|*	|*	|.
|[6][S]\drarr	|s	|p	|o	|w	|i	|e	|d	|ź	|[][,]{ }	|p	|o	|w	|s	|z	|e	|c	|h	|n	|a	|*	|[7][S]\darr	|.
|a	|n	|o	|r	|*	|*	|[8][S]\rarr	|h	|u	|b	|a	|[][,]{ }	|m	|a	|ś	|l	|a	|k	|*	|p	|*	|p	|.
|n	|a	|s	|a	|*	|[9][S]\rarr	|g	|ł	|o	|w	|i	|e	|n	|k	|a	|*	|[10][S]\darr	|*	|*	|a	|[11][S]\darr	|r	|.
|c	|*	|t	|n	|*	|*	|[12][S]\rarr	|l	|e	|n	|o	|n	|k	|i	|*	|*	|k	|*	|*	|r	|p	|z	|.
|y	|*	|o	|*	|*	|*	|*	|[13][S]\drarr	|s	|z	|p	|a	|l	|e	|r	|*	|l	|*	|*	|e	|o	|e	|.
|m	|[14][S]\darr	|z	|[15][S]\rarr	|w	|r	|ó	|b	|l	|o	|w	|e	|*	|w	|*	|*	|u	|*	|*	|n	|s	|c	|.
|o	|a	|u	|[16][S]\rarr	|o	|p	|ł	|o	|t	|k	|i	|*	|*	|n	|*	|*	|c	|*	|*	|t	|t	|i	|.
|n	|r	|c	|[17][S]\rarr	|b	|ł	|y	|s	|z	|c	|z	|[][,]{ }	|m	|i	|e	|d	|z	|i	|*	|o	|ę	|w	|.
|*	|r	|h	|*	|*	|*	|[18][S]\rarr	|s	|i	|ł	|a	|*	|*	|k	|*	|*	|*	|[19][S]\darr	|*	|z	|p	|n	|.
|*	|h	|*	|*	|*	|[20][S]\rarr	|c	|u	|r	|w	|o	|o	|d	|*	|*	|[21][S]\darr	|*	|l	|*	|a	|o	|i	|.
|[22][S]\drarr	|e	|o	|t	|r	|i	|c	|e	|r	|a	|t	|o	|p	|s	|*	|b	|*	|u	|*	|u	|w	|k	|.
|b	|n	|*	|[23][S]\rarr	|e	|l	|i	|t	|y	|z	|m	|*	|*	|[24][S]\rarr	|l	|a	|n	|d	|a	|r	|a	|*	|.
|r	|i	|*	|*	|*	|*	|*	|*	|*	|*	|*	|*	|*	|[25][S]\rarr	|u	|s	|m	|a	|ń	|*	|n	|*	|.
|*	|u	|*	|[26][S]\rarr	|t	|u	|l	|i	|ł	|e	|z	|k	|o	|w	|a	|t	|e	|*	|*	|*	|i	|*	|.
|[27][S]\drarr	|s	|y	|s	|t	|e	|m	|[][,]{ }	|i	|n	|s	|t	|a	|n	|c	|y	|j	|n	|y	|*	|e	|*	|.
|a	|*	|*	|[28][S]\rarr	|c	|e	|n	|a	|[][,]{ }	|n	|o	|m	|i	|n	|a	|l	|n	|a	|*	|*	|[][,]{ }	|*	|.
|s	|*	|*	|*	|[29][S]\rarr	|p	|r	|z	|e	|k	|ł	|a	|m	|a	|n	|i	|e	|*	|*	|*	|c	|[30][S]\darr	|.
|t	|*	|*	|*	|[31][S]\drarr	|z	|ł	|o	|t	|o	|p	|i	|ó	|r	|k	|a	|*	|*	|*	|*	|y	|c	|.
|r	|[32][S]\drarr	|ż	|ó	|ł	|w	|[][,]{ }	|m	|a	|l	|o	|w	|a	|n	|y	|*	|*	|*	|*	|[33][S]\darr	|w	|e	|.
|o	|p	|[34][S]\rarr	|n	|u	|m	|e	|r	|y	|c	|z	|n	|o	|ś	|ć	|*	|*	|*	|*	|j	|i	|d	|.
|n	|ó	|[35][S]\rarr	|o	|p	|o	|l	|a	|n	|i	|n	|*	|*	|*	|*	|*	|*	|[36][S]\darr	|*	|a	|l	|e	|.
|a	|ł	|*	|[37][S]\rarr	|n	|i	|e	|s	|p	|o	|r	|c	|z	|a	|k	|i	|*	|w	|[38][S]\darr	|n	|n	|t	|.
|w	|p	|*	|*	|i	|*	|*	|*	|[39][S]\darr	|*	|*	|*	|*	|*	|*	|[40][S]\drarr	|b	|o	|c	|c	|e	|*	|.
|i	|i	|*	|*	|a	|*	|*	|*	|s	|[41][S]\rarr	|ł	|u	|c	|z	|e	|k	|*	|j	|y	|a	|*	|*	|.
|g	|ę	|*	|*	|k	|*	|[42][S]\rarr	|b	|u	|n	|t	|*	|[43][S]\rarr	|d	|n	|o	|*	|s	|d	|r	|*	|*	|.
|a	|t	|*	|*	|*	|*	|*	|[44][S]\rarr	|m	|i	|l	|u	|*	|*	|*	|l	|*	|k	|o	|z	|*	|*	|.
|c	|r	|*	|*	|*	|*	|*	|[45][S]\rarr	|k	|r	|a	|s	|p	|e	|d	|o	|d	|o	|n	|*	|*	|*	|.
|j	|o	|*	|*	|[46][S]\rarr	|b	|a	|h	|a	|r	|i	|a	|z	|a	|u	|r	|*	|*	|i	|*	|*	|*	|.
|a	|*	|*	|*	|*	|*	|*	|*	|*	|*	|*	|*	|*	|*	|*	|*	|*	|*	|a	|*	|*	|*	|.
|*	|[47][S]\rarr	|r	|y	|t	|m	|[][,]{ }	|b	|i	|o	|l	|o	|g	|i	|c	|z	|n	|y	|*	|*	|*	|*	|.\end{Puzzle}

\newpage

\begin{PuzzleClues}{\textbf{Poziome}\\}\Clue{4}{}{przen. tajniki, warsztat przygotowania czegoś}
\Clue{5}{}{pokrywka, przykrywka, zamknięcie czegoś}
\Clue{6}{}{dawniej - spowiedź całego zgromadzenia, mająca charakter sakramentu (odpuszczenia wszystkich grzechów)}
\Clue{8}{}{Suillus luteus - gatunek grzyba z rodziny maślakowatych; jest szeroko rozprzestrzeniony, występuje na całej półkuli północnej na obszarach o klimacie umiarkowanym}
\Clue{9}{}{głowienka zwyczajna, kaczka rdzawogłowa, Aythya ferina - gatunek ptaka z rodziny kaczkowatych (Anatidae); zamieszkuje środkowe szerokości geograficzne Eurazji - Wyspy Brytyjskie, Europę Środkową i Wschodnią i pas w Azji Środkowej po Mandżurię i północną Japonię, poza tym izolowana populacja występuje w Azji Mniejszej}
\Clue{12}{}{wzór okularów o okrągłych oprawkach, których nazwa pochodzi od nazwiska brytyjskiego muzyka Johna Lennona}
\Clue{13}{}{przejście utworzone z dwóch kolumn ustawionyc obok siebie ludzi}
\Clue{15}{}{Passeriformes - rząd ptaków z podgromady Neornithes}
\Clue{16}{}{obszar koło domu czy chałupy, otoczony płotkiem}
\Clue{17}{}{minerał z grupy siarczków}
\Clue{18}{}{energia, którą dysponuje człowiek, witalność, tyle zdrowia i zapału, ile ma w danej chwili}
\Clue{20}{}{(1878-1927), pisarz amerykański, popularne powieści z życia Indian i traperów; „Szara wilczyca”, „Władca skalny”}
\Clue{22}{}{Eotriceratops - rodzaj roślinożernego dinozaura z rodziny ceratopsów; żył w okresie późnej kredy na terenach Ameryki Północnej, mógł dorastać do 12 metrów i ważył 13 ton}
\Clue{23}{}{pogląd społeczno-polityczny, zakładający wyodrębnienie ze społeczeństwa warstwy wyższej - elity}
\Clue{24}{}{duże, wystawne auto}
\Clue{25}{}{miasto w europejskiej części Federacji Rosyjskiej na płn.-wsch od Woroweża}
\Clue{26}{}{Roridulaceae - rodzina roślin z rzędu wrzosowców}
\Clue{27}{}{prowadzenie spraw sądowych przez kilka etapów postępowania w różnych sądach}
\Clue{28}{}{oficjalna wersja ceny}
\Clue{29}{}{stwierdzenie w części prawdziwe}
\Clue{31}{}{Leptosittaca branickii - gatunek ptaka z rodziny papugowatych (Psittacidae), z podrodziny papug neotropikalnych (Arinae)}
\Clue{32}{}{Chrysemys picta - gatunek gada z podrzędu żółwi skrytoszyjnych z rodziny żółwi błotnych, jedyny przedstawiciel rodzaju Chrysemys, charakteryzujący się pancerzem barwy zielono-oliwkowej lub oliwkowej w różnych odcieniach (aż do czarnego) i występowaniem na tym tle na karapaksie i głowie żółtych lub pomarańczowych plam; żyje na obszarze od południowej Kanady do północnego Meksyku}
\Clue{34}{}{liczebność; liczbowość}
\Clue{35}{}{mieszkaniec Opola}
\Clue{37}{}{maleńkie bezkręgowce o nie ustalonej przynależności, zaliczane zwykle do stawonogów}
\Clue{40}{}{gra sportowa ściśle związana z pétanque, bowls i grą prowansalską, która polega na umieszczeniu własnych kul jak najbliżej małej kulki}
\Clue{41}{}{stosowany w zapisie nutowym znak graficzny  łączący kilka nut}
\Clue{42}{}{protest przeciwko czemuś, postawa wyrażająca - najczęściej w sposób spontaniczny i gwałtowny - przeciwny stosunek do czegoś}
\Clue{43}{}{pogardliwie: kompletny ignorant, zero; stosowane także w odniesieniu do osoby zgniłej moralnie}
\Clue{44}{}{wsch. azjatycki ssak z jeleniowatych- obecnie tylko w zoo}
\Clue{45}{}{Craspedodon - rodzaj dinozaura znany jedynie na podstawie trzech zębów datowanych na górną kredę, znalezionych w okolicach wsi Lonzée w Belgii}
\Clue{46}{}{Bahariasaurus - rodzaj dinozaura z grupy teropodów; żył w okresie kredy na terenach Afryki}
\Clue{47}{}{zjawisko polegające na występowaniu w organizmach żywych pewnych przemian o charakterze cyklicznym, związanych z działaniem swoistych oscylatorów, zwanych zegarami biologicznymi}\end{PuzzleClues}

\begin{PuzzleClues}{\textbf{Pionowe}\\}\Clue{1}{}{egipskie miasto nad Nilem}
\Clue{2}{}{egzemplarz Koranu}
\Clue{3}{}{Lapparentosaurus - rodzaj zauropoda o niepewnej pozycji filogenetycznej, dawniej uznawany za przedstawiciela cetiozaurów, brachiozaurów lub tytanokształtnych}
\Clue{4}{}{Postosuchus - rodzaj drapieżnego archozaura żyjącego od środkowego karniku do noryku (późny trias) na terenie obecnej Ameryki Północnej}
\Clue{5}{}{Origma solitaria - gatunek małego ptaka z rodziny buszówkowatych (Acanthizidae); endemit, występuje jedynie na terenie Australii, w Nowej Południowej Walii; jego środowiskiem występowania są lasy klimatu umiarkowanego, zarośla oraz obszary skalne (klify, szczyty górskie)}
\Clue{6}{}{agregat, model, utrapienie, nicpoń}
\Clue{7}{}{rywal, taki jak w rywalizacji sportowej, konkurent, współzawodnik}
\Clue{10}{}{narzędzie do nakręcania mechanizmu sprężynowego}
\Clue{11}{}{postępowanie, w którym rozpatruje się sprawy z zakresu prawa cywilnego, rodzinnego i opiekuńczego, prawa pracy, sprawy dotyczące ubezpieczeń społecznych oraz inne sprawy, do których przepisy kodeksu postępowania cywilnego stosuje się z mocy ustaw szczególnych; termin prawny}
\Clue{13}{}{(1627-1704), francuski pisarz, teolog i historyk; „Uwagi nad historią powszechną”}
\Clue{14}{}{astrofizyk i fizykochemik szwedzki (1839-1927), nagroda Nobla w 1903 r}
\Clue{19}{}{zespół miejski w Chinach, obejmuje miasto Dalian i Lushuin}
\Clue{21}{}{paryska twierdza z XIVw, później więzienie}
\Clue{22}{}{skrót/symbol waluty birr}
\Clue{27}{}{nawigacja opierająca się na pozycjach ciał niebieskich, względem których określa się położenie statku}
\Clue{30}{}{potoczna nazwa największego w Warszawie domu towarowego (Centralny Dom Towarowy, czyli stołeczny Powszechny Dom Towarowy) stosowana obecnie także jako nazwa budynku, w którym ten dom towarowy się mieścił}
\Clue{31}{}{młot kamieniarski z ostrzem do nacinania w kamieniach rysy}
\Clue{32}{}{ANTRESOLA, MEZANIN, MEZZANINO}
\Clue{33}{}{utytułowany żużlowiec, zawodnik Stali Gorzów}
\Clue{36}{}{ARMIA}
\Clue{38}{}{PIGWA}
\Clue{39}{}{określona ilość pieniędzy}
\Clue{40}{}{w pokerze: pięć kart jednego koloru na ręce}\end{PuzzleClues}\newpage%\section*{Krzyżówka 12}

\noindent\begin{Puzzle}{14}{33}|*	|*	|*	|*	|*	|*	|*	|[1][S]\darr	|*	|*	|*	|*	|*	|*	|[2][S]\darr	|.
|*	|[3][S]\rarr	|b	|i	|s	|k	|u	|p	|i	|a	|n	|i	|n	|*	|m	|.
|*	|*	|[4][S]\rarr	|f	|u	|l	|t	|o	|n	|*	|*	|*	|*	|*	|e	|.
|[5][S]\drarr	|w	|i	|e	|l	|k	|o	|ś	|ć	|*	|*	|*	|*	|[6][S]\darr	|l	|.
|o	|*	|*	|[7][S]\rarr	|c	|h	|a	|r	|l	|t	|o	|n	|*	|g	|o	|.
|p	|*	|*	|*	|[8][S]\darr	|*	|*	|e	|*	|*	|*	|*	|*	|e	|d	|.
|r	|*	|[9][S]\darr	|*	|s	|*	|[10][S]\rarr	|d	|y	|n	|i	|a	|*	|r	|r	|.
|e	|*	|k	|*	|k	|*	|*	|n	|*	|*	|[11][S]\darr	|*	|[12][S]\darr	|s	|a	|.
|s	|*	|o	|*	|u	|*	|*	|i	|*	|*	|o	|*	|g	|h	|m	|.
|y	|[13][S]\rarr	|t	|u	|r	|z	|y	|c	|a	|*	|r	|*	|e	|w	|a	|.
|j	|*	|l	|*	|c	|*	|*	|a	|*	|*	|l	|*	|t	|i	|t	|.
|n	|*	|e	|*	|z	|*	|*	|*	|[14][S]\drarr	|b	|a	|ż	|a	|n	|*	|.
|o	|*	|t	|*	|*	|[15][S]\rarr	|g	|a	|m	|a	|*	|*	|*	|*	|[16][S]\darr	|.
|ś	|*	|[][,]{ }	|*	|[17][S]\rarr	|d	|e	|k	|a	|m	|e	|t	|r	|*	|f	|.
|ć	|*	|d	|*	|[18][S]\rarr	|c	|a	|b	|r	|e	|r	|a	|*	|*	|o	|.
|*	|*	|e	|[19][S]\drarr	|s	|p	|i	|n	|k	|s	|*	|[20][S]\darr	|[21][S]\darr	|*	|r	|.
|*	|*	|[][,]{ }	|b	|*	|*	|[22][S]\darr	|*	|e	|*	|*	|o	|j	|*	|m	|.
|*	|[23][S]\drarr	|v	|o	|y	|a	|g	|e	|r	|*	|*	|s	|a	|[24][S]\darr	|u	|.
|[25][S]\drarr	|s	|o	|l	|b	|e	|r	|g	|*	|*	|*	|t	|r	|j	|l	|.
|j	|k	|l	|e	|*	|*	|u	|*	|*	|*	|*	|r	|z	|a	|a	|.
|a	|r	|a	|ń	|*	|*	|s	|*	|*	|[26][S]\darr	|*	|y	|y	|ź	|r	|.
|r	|z	|i	|*	|*	|*	|z	|*	|*	|h	|*	|[][,]{ }	|n	|w	|z	|.
|z	|y	|l	|[27][S]\darr	|[28][S]\rarr	|k	|a	|m	|i	|e	|ń	|s	|k	|i	|*	|.
|ę	|n	|l	|c	|*	|*	|[][,]{ }	|[29][S]\darr	|*	|l	|*	|t	|a	|e	|*	|.
|b	|k	|e	|i	|*	|*	|c	|f	|*	|i	|*	|r	|*	|c	|[30][S]\darr	|.
|i	|a	|*	|e	|*	|*	|h	|i	|*	|n	|*	|z	|*	|*	|k	|.
|n	|*	|[31][S]\drarr	|p	|a	|t	|o	|l	|o	|g	|i	|a	|*	|*	|e	|.
|ó	|*	|l	|l	|*	|*	|j	|a	|*	|*	|*	|ł	|*	|*	|j	|.
|w	|*	|e	|i	|*	|*	|u	|m	|*	|*	|*	|*	|*	|*	|a	|.
|k	|*	|w	|k	|[32][S]\drarr	|k	|r	|e	|m	|o	|w	|o	|ś	|ć	|*	|.
|a	|*	|a	|*	|z	|*	|o	|n	|*	|*	|*	|*	|*	|*	|*	|.
|*	|*	|r	|*	|o	|*	|*	|t	|*	|*	|*	|*	|*	|*	|*	|.
|*	|*	|*	|*	|o	|*	|*	|*	|*	|*	|*	|*	|*	|*	|*	|.
|*	|*	|*	|*	|*	|*	|*	|*	|*	|*	|*	|*	|*	|*	|*	|.\end{Puzzle}

\newpage

\begin{PuzzleClues}{\textbf{Poziome}\\}\Clue{3}{}{przedstawiciel grupy etnograficznej zamieszkującej tereny Biskupizny}
\Clue{4}{}{(1765-1815) amerykański budowniczy statków wodnych}
\Clue{5}{}{cecha tego, co jest wielkie (pod względem rozmiaru fizycznego)}
\Clue{7}{}{Jack, piłkarz brytyjski, złoty medalista mistrzostw świata w Anglii, uznany trener}
\Clue{10}{}{Cucurbita - rodzaj roślin jednorocznych z rodziny dyniowatych obejmujący około 20 gatunków}
\Clue{13}{}{łow. sierść zająca lub królika}
\Clue{14}{}{ur. 1904r, poeta ukraiński, działacz ruchu pokoju, „Mickiewicz w Odessie”, „Rozmowa serc”}
\Clue{15}{}{Vasco da (1460-1524); żeglarz portugalski, odkrył drogę morską do Indii}
\Clue{17}{}{wielokrotność metra, podstawowej jednostki długości w układzie SI; jeden dekametr = 10 metrów}
\Clue{18}{}{ur. 1929r, pisarz kubański „.Odpływająca fala”}
\Clue{19}{}{Michael, bokser amerykański, mistrz olimpijski z Montrealu, zawodowy mistrz świata w kategorii półciężkiej}
\Clue{23}{}{seria 2 amerykańskich próbników badających między innymi Jowisza i Saturna}
\Clue{25}{}{dwuboista norweski, dwukrotny mistrz olimpijski z Grenoble, Sapporo}
\Clue{28}{}{Łucjan, muzykolog i kompozytor (1885-1964); zorganizował pierwsze w Polsce folklorystyczne Archiwum etnograficzne}
\Clue{31}{}{dział medycyny, nauka o chorobach}
\Clue{32}{}{cecha tego, co ma jasnobeżowy kolor, co jest w kolorze białym przełamanym beżem}\end{PuzzleClues}

\begin{PuzzleClues}{\textbf{Pionowe}\\}\Clue{1}{}{pośredniczka, kobieta pośrednik}
\Clue{2}{}{jakieś dramatyczne wydarzenie w życiu, przedstawione w sposób przejaskrawiony}
\Clue{5}{}{ucisk, wykluczanie, marginalizowanie}
\Clue{6}{}{ameryk kompozytor i pianista (1898-1937); muzyka do filmów, rewii, utwory symfoniczne; 'Błękitna rapsodia'}
\Clue{8}{}{zaburzenie pracy mięśni, powodujące ból}
\Clue{9}{}{kotlet panierowany w jajku i bułce tartej, smażony na głębokim oleju sporządzony z roztłuczonej piersi kurczaka nadziewanej masłem i przyprawami}
\Clue{11}{}{rzeka w zachodniej Polsce, największy, prawy dopływ Baryczy o długości 88 km i powierzchni dorzecza 1546 km2}
\Clue{12}{}{drewniane obuwie japońskie przypominające chodaki}
\Clue{14}{}{rekwizyt do gry, substytut broni w paintballu}
\Clue{16}{}{formuła, konwencjonalne sformułowanie, powatrzalny schemat}
\Clue{19}{}{RAP; drapieżna ryba z karpiowatych o długości do 80 cm; rzeki wschodniej Europy}
\Clue{20}{}{strzał, który jest oddany nabojem bojowym}
\Clue{21}{}{przyprawa ze sproszkowanych, suszonych warzyw}
\Clue{22}{}{Pyrus pyriflia 'Chojuro' - odmiana uprawna gruszy chińskiej}
\Clue{23}{}{opakowanie do transportu szklanych butelek standardowych kształtów i rozmiarów (z wódką, winem, piwem, wodą mineralną itp.)}
\Clue{24}{}{BORSUK; ssak z rodziny łasicowatych}
\Clue{25}{}{nalewka domowa robiona z owoców jarzębiny}
\Clue{26}{}{stanowisko budowy małych statków wodnych, szybowców itp}
\Clue{27}{}{ptak z rzędu wróblowatych podobny do trznadla, u samca czoło i podgardle czarne; Eurazja, płn-zach. Afryka}
\Clue{29}{}{włókno białkowe znajdujące się w cytoplazmie, które odpowiada za rozmieszczenie organelli w komórce}
\Clue{30}{}{nadbrzeże uzbrojone w urządzenia cumownicze}
\Clue{31}{}{dźwignik}
\Clue{32}{}{teren udostępniony odwiedzającym, na którym hodowane są zwierzęta, najczęściej pochodzące z różnych obszarów geograficznych}\end{PuzzleClues}\newpage%\section*{Krzyżówka 13}

\noindent\begin{Puzzle}{17}{26}|*	|[1][S]\drarr	|s	|o	|l	|s	|t	|y	|c	|j	|u	|m	|*	|*	|*	|*	|*	|*	|.
|*	|o	|*	|*	|[2][S]\drarr	|t	|e	|n	|d	|e	|r	|*	|*	|*	|[3][S]\darr	|*	|*	|*	|.
|[4][S]\drarr	|s	|e	|r	|p	|e	|n	|t	|*	|[5][S]\drarr	|r	|u	|*	|[6][S]\darr	|p	|*	|[7][S]\darr	|[8][S]\darr	|.
|c	|z	|*	|[9][S]\darr	|o	|[10][S]\drarr	|p	|r	|o	|z	|e	|l	|i	|t	|a	|*	|p	|z	|.
|h	|o	|*	|p	|w	|k	|[11][S]\drarr	|s	|*	|a	|[12][S]\darr	|[13][S]\darr	|[14][S]\darr	|e	|s	|*	|ó	|ł	|.
|a	|ł	|*	|e	|e	|k	|w	|*	|*	|r	|n	|r	|p	|l	|c	|*	|ł	|o	|.
|n	|o	|[15][S]\darr	|r	|r	|*	|y	|*	|[16][S]\darr	|a	|i	|e	|r	|a	|h	|*	|c	|ż	|.
|*	|m	|p	|r	|b	|[17][S]\rarr	|s	|z	|c	|z	|e	|n	|a	|*	|a	|[18][S]\darr	|i	|e	|.
|*	|*	|ę	|a	|a	|[19][S]\darr	|i	|*	|e	|a	|d	|n	|w	|*	|*	|h	|ę	|n	|.
|*	|[20][S]\drarr	|p	|u	|l	|p	|e	|t	|*	|[][,]{ }	|o	|*	|o	|[21][S]\darr	|[22][S]\darr	|o	|ż	|i	|.
|*	|m	|a	|l	|l	|i	|d	|*	|*	|m	|l	|*	|[][,]{ }	|h	|o	|w	|a	|e	|.
|*	|a	|w	|t	|*	|k	|l	|[23][S]\darr	|*	|o	|i	|[24][S]\drarr	|p	|a	|s	|e	|r	|*	|.
|*	|ł	|a	|*	|[25][S]\darr	|*	|o	|k	|[26][S]\darr	|r	|s	|p	|r	|w	|t	|a	|ó	|[27][S]\darr	|.
|*	|a	|[][,]{ }	|*	|m	|[28][S]\darr	|n	|r	|w	|o	|e	|r	|y	|a	|r	|*	|w	|k	|.
|*	|[][,]{ }	|d	|*	|a	|p	|y	|y	|e	|w	|k	|o	|w	|ń	|o	|[29][S]\darr	|k	|a	|.
|[30][S]\drarr	|w	|a	|r	|g	|a	|*	|p	|k	|a	|*	|t	|a	|c	|ż	|w	|a	|w	|.
|s	|i	|c	|*	|l	|t	|*	|t	|t	|*	|*	|a	|t	|z	|e	|o	|*	|a	|.
|z	|e	|h	|*	|o	|o	|*	|o	|o	|[31][S]\darr	|*	|n	|n	|y	|ń	|j	|*	|ł	|.
|t	|ś	|o	|*	|w	|w	|[32][S]\drarr	|p	|r	|z	|ą	|d	|e	|k	|*	|c	|*	|e	|.
|a	|*	|w	|[33][S]\drarr	|n	|o	|k	|s	|*	|e	|*	|r	|*	|*	|*	|i	|*	|k	|.
|n	|[34][S]\darr	|a	|z	|i	|ś	|a	|*	|*	|s	|[35][S]\drarr	|i	|s	|a	|j	|e	|w	|*	|.
|g	|m	|*	|m	|c	|ć	|k	|*	|[36][S]\drarr	|p	|r	|a	|g	|n	|ą	|c	|y	|*	|.
|a	|u	|*	|k	|a	|*	|a	|*	|n	|ó	|u	|*	|*	|*	|*	|h	|*	|*	|.
|*	|z	|*	|*	|*	|*	|*	|*	|o	|ł	|m	|*	|*	|*	|*	|ó	|*	|*	|.
|[37][S]\rarr	|a	|n	|a	|l	|i	|t	|y	|k	|*	|b	|[38][S]\rarr	|t	|y	|k	|w	|a	|*	|.
|*	|*	|*	|*	|*	|*	|*	|*	|*	|[39][S]\rarr	|a	|n	|a	|n	|d	|*	|*	|*	|.
|*	|*	|*	|*	|*	|*	|*	|*	|*	|*	|*	|*	|*	|*	|*	|*	|*	|*	|.\end{Puzzle}

\newpage

\begin{PuzzleClues}{\textbf{Poziome}\\}\Clue{1}{}{przesilenie}
\Clue{2}{}{okręt-baza}
\Clue{4}{}{instrument muzyczny z rodziny cynków; zrobiony z dwóch wydrążonych drewnianych łusek drzewa kasztanowego sklejonych razem i powleczonych skórą, wygięty czterokrotnie, posiada sześć otworów i miedziany ustnik}
\Clue{5}{}{w chemii: symbol rutenu}
\Clue{10}{}{nowo pozyskany, gorliwy wyznawca jakiejś religii}
\Clue{11}{}{w chemii: symbol siarki}
\Clue{17}{}{szczęka; słowo młodzieżowe}
\Clue{20}{}{ktoś dość gruby, pulchny}
\Clue{24}{}{osoba biorąca udział w handlu kradzionymi rzeczami lub je rozprowadzająca}
\Clue{30}{}{zgrubienie na brzegu otworu muszli}
\Clue{32}{}{mężczyzna przędący lub zatrudniony w przędzalni}
\Clue{33}{}{jednostka natężenia oświetlenia}
\Clue{35}{}{Jegor, poeta rosyjski, ur,1926r; „Sąd pamięci”}
\Clue{36}{}{ten, który odczuwa pragnienie}
\Clue{37}{}{specjalista, który coś analizuje na czyjeś zlecenie}
\Clue{38}{}{Lagenaria - rodzaj jednorocznych, pnących roślin tropikalnych z rodziny dyniowatych.}
\Clue{39}{}{ur. 1905r, indyjski pisarz tworzący w języku angielskim i pendżabskim; „Kulis” - Leninowska Nagroda Pokoju}\end{PuzzleClues}

\begin{PuzzleClues}{\textbf{Pionowe}\\}\Clue{1}{}{pogardliwie o osobie owładniętej jakąś myślą, fanatyku religijnym lub ideologicznym}
\Clue{2}{}{zawierający żyroskop przyrząd do treningu i rehabilitacji palców, stawów, mięśni nadgarstka i przedramienia}
\Clue{3}{}{w języku religijnym (z judeo-chrześcijańskiego kręgu kulturowego): ofiara, poświęcenie czegoś na życzenie, w imię Boga (nazwa wywodzi się od zwyczaju, święta znanego w plemionach semickich jeszcze przed Mojżeszem - zabijania na ofiarę jednego z najbardziej okazałych zwierząt jednorocznych w stadzie przed wyruszeniem na wypas i zjadania go na wspólnej kolacji w imię braterstwa pomiędzy ludźmi i przymierza z Bogiem)}
\Clue{4}{}{jedna z najważniejszych szkół chińskiego buddyzmu, należąca do praktycznej i medytacyjnej tradycji buddyzmu}
\Clue{5}{}{ostra bakteryjna choroba zakaźna gryzoni i (rzadziej) innych drobnych ssaków, a także człowieka}
\Clue{6}{}{miasto i port w Hondurasie nad Morzem Karaibskim}
\Clue{7}{}{samochód półciężarowy z nadwoziem typu pikap}
\Clue{8}{}{w językoznawstwie - derywat, który powstał z połączenia conajmniej dwóch rdzeni za pomocą elementu łączącego - międzyrostka}
\Clue{9}{}{architekt francuski (1613-88), przedstawiciel barokowego klasycznego Ludwika XIV}
\Clue{10}{}{akt normatywny stanowiący zbiór przepisów regulujących odpowiedzialność karną obywateli danego państwa, zawierający definicję przestępstwa, zasady odpowiedzialności za przestępstwo, zasady przedawnienia odpowiedzialności karnej oraz spis kar i reguły ich stosowania}
\Clue{11}{}{człowiek, którego wysiedlono, zmuszono do opuszczenia miejsca zamieszkania}
\Clue{12}{}{łow. młody lis}
\Clue{13}{}{pisarz niemiecki, uczestnik hiszpańskiej wojny domowej (1889-1979), „Wojna”, „Murzynek Nobi”}
\Clue{14}{}{jedna z dwóch podstawowych gałęzi prawa (obok prawa publicznego), skupiająca normy prawne, których zadaniem jest ochrona interesu jednostek i regulacja stosunków pomiędzy nimi}
\Clue{15}{}{Crepis tectorum - gatunek rośliny należący do rodziny astrowatych}
\Clue{16}{}{w chemii: symbol ceru}
\Clue{18}{}{KENCJA palma z wysp Oceanu Spokojnego, uprawiana jako ozdobna roślina doniczkowa}
\Clue{19}{}{jeden z czterech kolorów w kartach, oznaczony małym czarnym listkiem}
\Clue{20}{}{wieś w Polsce położona w województwie mazowieckim, w powiecie grójeckim, w gminie Belsk Duży}
\Clue{21}{}{mieszkaniec Hawany}
\Clue{22}{}{roślina zielna strefy umiarkowanej z rodziny jaskrowatych; kwiaty z ostrogą}
\Clue{23}{}{Kryptops - rodzaj prymitywnego teropoda z rodziny abelizaurów; żył w okresie wczesnej kredy na terenach dzisiejszego Nigru}
\Clue{24}{}{zjawisko występujące w kwiatach obupłciowych u niektórych gatunków roślin, u których pręciki dojrzewają szybciej niż słupki}
\Clue{25}{}{maszyna, służąca do maglowania, czyli prasowania przy użyciu systemu walców}
\Clue{26}{}{obiekt matematyczny, który ma moduł, kierunek oraz zwrot (określający orientację wzdłuż danego kierunku)}
\Clue{27}{}{niewielka część czegoś, odcinek, fragment, urywek}
\Clue{28}{}{to, że sytuacja nie ulega zmianie i nie ma widoków na przezwyciężenie trudności i pozytywne rozwiązanie}
\Clue{29}{}{wieś w województwie dolnośląskim, w powiecie lwóweckim}
\Clue{30}{}{sprzęt sportowy służący do wykonywania nim ćwiczeń siłowych}
\Clue{31}{}{grupa muzyków występujących razem}
\Clue{32}{}{nowozelandzki ptak z rodziny papug}
\Clue{33}{}{kod ISO 4217 kwachy zambiijskiej}
\Clue{34}{}{muzyka śpiewana, grana lub tylko odsłuchiwana przez kogoś}
\Clue{35}{}{latynoamerykański taniec towarzyski pochodzący z Kuby}
\Clue{36}{}{kod ISO 4217 korony norweskiej}\end{PuzzleClues}\newpage%\section*{Krzyżówka 15}

\noindent\begin{Puzzle}{22}{29}|*	|*	|*	|*	|*	|*	|*	|*	|*	|*	|*	|[1][S]\drarr	|k	|a	|n	|t	|a	|r	|y	|d	|a	|*	|*	|.
|*	|*	|*	|*	|*	|*	|*	|*	|*	|*	|[2][S]\rarr	|e	|m	|f	|i	|t	|e	|u	|t	|a	|*	|*	|*	|.
|*	|*	|[3][S]\rarr	|p	|o	|d	|r	|z	|e	|ń	|[][,]{ }	|g	|a	|r	|b	|a	|t	|y	|*	|*	|*	|*	|*	|.
|*	|*	|*	|*	|[4][S]\rarr	|k	|ł	|ę	|b	|o	|s	|z	|[][,]{ }	|n	|a	|d	|w	|o	|d	|n	|y	|*	|*	|.
|[5][S]\drarr	|c	|z	|a	|r	|n	|y	|[][,]{ }	|c	|h	|l	|e	|b	|*	|*	|*	|*	|*	|*	|*	|*	|*	|*	|.
|s	|*	|*	|[6][S]\rarr	|s	|t	|a	|t	|u	|s	|[][,]{ }	|m	|a	|t	|e	|r	|i	|a	|l	|n	|y	|*	|*	|.
|t	|[7][S]\rarr	|w	|s	|p	|ó	|ł	|c	|i	|e	|r	|p	|i	|ą	|c	|y	|*	|*	|*	|*	|*	|*	|[8][S]\darr	|.
|ę	|*	|*	|*	|[9][S]\rarr	|s	|t	|a	|s	|z	|e	|l	|*	|*	|*	|*	|*	|*	|*	|*	|*	|*	|k	|.
|ż	|[10][S]\rarr	|a	|s	|f	|o	|d	|e	|l	|o	|w	|a	|[][,]{ }	|ł	|ą	|k	|a	|*	|*	|*	|*	|*	|o	|.
|e	|*	|*	|[11][S]\rarr	|w	|a	|n	|g	|a	|[][,]{ }	|g	|r	|u	|b	|o	|d	|z	|i	|o	|b	|a	|*	|n	|.
|n	|*	|*	|*	|*	|*	|*	|*	|[12][S]\rarr	|p	|r	|z	|e	|g	|l	|ą	|d	|a	|r	|k	|a	|*	|f	|.
|i	|[13][S]\drarr	|t	|o	|r	|f	|o	|w	|i	|e	|c	|[][,]{ }	|c	|i	|e	|m	|n	|y	|*	|[14][S]\darr	|[15][S]\darr	|[16][S]\darr	|e	|.
|e	|m	|*	|*	|[17][S]\rarr	|k	|a	|n	|o	|p	|u	|s	|*	|*	|*	|[18][S]\darr	|[19][S]\darr	|*	|*	|s	|s	|w	|s	|.
|[][,]{ }	|i	|[20][S]\rarr	|k	|u	|r	|o	|b	|r	|o	|d	|y	|*	|*	|*	|p	|r	|*	|[21][S]\darr	|c	|k	|e	|j	|.
|m	|k	|*	|[22][S]\darr	|*	|*	|*	|*	|*	|[23][S]\drarr	|o	|g	|i	|w	|a	|r	|a	|*	|g	|e	|a	|g	|a	|.
|o	|r	|*	|u	|[24][S]\rarr	|k	|a	|p	|e	|l	|a	|n	|*	|[25][S]\rarr	|t	|a	|j	|w	|a	|n	|k	|a	|*	|.
|l	|o	|*	|l	|*	|[26][S]\rarr	|j	|a	|j	|o	|w	|a	|r	|*	|*	|w	|t	|[27][S]\darr	|l	|a	|a	|n	|*	|.
|o	|p	|*	|g	|*	|[28][S]\darr	|[29][S]\rarr	|m	|e	|t	|y	|l	|*	|*	|*	|o	|u	|k	|e	|r	|n	|i	|*	|.
|w	|o	|[30][S]\darr	|a	|*	|b	|*	|*	|*	|n	|*	|n	|*	|*	|*	|[][,]{ }	|z	|r	|a	|i	|k	|n	|*	|.
|e	|w	|w	|[][,]{ }	|*	|a	|*	|*	|*	|i	|*	|y	|*	|*	|*	|n	|y	|ę	|s	|u	|a	|*	|*	|.
|*	|i	|i	|o	|[31][S]\drarr	|l	|o	|f	|t	|k	|i	|*	|*	|*	|*	|i	|*	|t	|*	|s	|*	|*	|*	|.
|*	|e	|r	|d	|e	|e	|*	|*	|*	|*	|[32][S]\rarr	|w	|y	|b	|i	|e	|l	|a	|c	|z	|*	|*	|*	|.
|*	|ś	|o	|s	|l	|t	|[33][S]\rarr	|m	|a	|d	|a	|p	|o	|l	|a	|m	|*	|r	|*	|*	|*	|*	|*	|.
|*	|ć	|p	|e	|a	|*	|*	|*	|*	|[34][S]\rarr	|k	|a	|d	|i	|r	|i	|*	|z	|*	|*	|*	|*	|*	|.
|*	|*	|ł	|t	|s	|*	|*	|*	|*	|*	|[35][S]\rarr	|r	|a	|f	|a	|e	|l	|*	|*	|*	|*	|*	|*	|.
|*	|[36][S]\rarr	|a	|k	|t	|[][,]{ }	|u	|s	|t	|a	|w	|o	|d	|a	|w	|c	|z	|y	|*	|*	|*	|*	|*	|.
|*	|*	|t	|o	|i	|*	|*	|[37][S]\rarr	|u	|r	|z	|ą	|d	|[][,]{ }	|s	|k	|a	|r	|b	|o	|w	|y	|*	|.
|*	|*	|*	|w	|l	|*	|[38][S]\rarr	|g	|a	|r	|d	|e	|r	|o	|b	|i	|a	|n	|a	|*	|*	|*	|*	|.
|*	|*	|*	|a	|*	|*	|*	|*	|*	|*	|*	|*	|[39][S]\rarr	|g	|i	|e	|r	|y	|m	|s	|k	|i	|*	|.
|*	|*	|*	|*	|*	|*	|*	|*	|*	|[40][S]\rarr	|g	|ł	|o	|w	|a	|*	|*	|*	|*	|*	|*	|*	|*	|.\end{Puzzle}

\newpage

\begin{PuzzleClues}{\textbf{Poziome}\\}\Clue{1}{}{MAJKA LEKARSKA; metalicznie zielony chrząszcz}
\Clue{2}{}{czynszownik na prawie emfiteuzy}
\Clue{3}{}{Blechnum gibbum - gatunek paproci należący do rodziny podrzeniowatych (Blechnaceae); pochodzi z wysp Oceanu Spokojnego: Nowej Kaledonii oraz Fidżi}
\Clue{4}{}{Dolomedes plantarius - gatunek dużego europejskiego pająka z rodziny darownikowatych (Pisauridae)}
\Clue{5}{}{gatunek chleba najczęściej żytniego, który powstaje z mąki razowej (grubo mielonej i najsłabiej oczyszczonej)}
\Clue{6}{}{status społeczny osoby lub grupy osób (np. rodziny) określany na podstawie wysokości zarobków i ilości posiadanych dóbr}
\Clue{7}{}{towarzysz cierpień, ten, kto cierpi razem z kimś}
\Clue{9}{}{narciarz, brązowy medalista mistrzostw świata z 1974 r. w biegu na 30 km}
\Clue{10}{}{miejsce, gdzie według mitologii greckiej odbywają przechadzki cienie zmarłych}
\Clue{11}{}{Xenopirostris polleni - gatunek ptaka z rodziny wang (Vangidae)}
\Clue{12}{}{urządzenie optyczne, które służy do oglądania przezroczy}
\Clue{13}{}{torfowiec brunatny, Sphagnum fuscum - gatunek mszaka z rodziny torfowcowatych; rozpowszechniony na półkuli północnej, dość pospolity na terenie Polski}
\Clue{17}{}{gwiazda w gwiazdozbiorze Kilu}
\Clue{20}{}{koralniki, Callaeidae - rodzina ptaków z rzędu wróblowych, obejmująca kilka gatunków ptaków, występujących wyłącznie w Nowej Zelandii}
\Clue{23}{}{narciarz japoński, mistrz olimpijski w kombinacji norweskiej z Albertville i Lillehammer}
\Clue{24}{}{niewielka ryba ze stynkowatych żyjąca w stadach, stanowiąca pokarm drapieżnych ryb}
\Clue{25}{}{mieszkanka Tajwanu, kobieta pochodzenia tajwańskiego}
\Clue{26}{}{urządzenie służące do gotowania jajek}
\Clue{29}{}{grupa funkcyjna, powstała przez oderwanie atomu wodoru od cząsteczki metanu, o wzorze -CH3}
\Clue{31}{}{LOTKI}
\Clue{32}{}{środek chemiczny służący do wybielania}
\Clue{33}{}{cienka, gęsta tkanina bawełniana o splocie płóciennym, podobna do batystu, bielona albo drukowana, używana na bieliznę lub sukienki}
\Clue{34}{}{pisarz uzbecki (1896-1939), pierwsze nowele w literaturze uzbeckiej, powieści, sztuki teatralne}
\Clue{35}{}{włoski malarz i architekt, najmłodszy z trójki genialnych artystów włoskiego renesansu, znany z licznych przedstawień Madonny}
\Clue{36}{}{akt prawny, który został przyjęty w drodze procedury ustawodawczej}
\Clue{37}{}{państwowa jednostka budżetowa obsługująca naczelnika urzędu skarbowego, który jest organem administracji niezespolonej w terenie podlegającym Ministrowi Finansów, a zarazem organem podatkowym pierwszej instancji}
\Clue{38}{}{służąca mająca pieczę nad garderobą i pomagająca pani w ubieraniu się}
\Clue{39}{}{zbiór gipsowych odlewów rzeźb antycznych}
\Clue{40}{}{antropometryczna miara wysokości, stosowana również w przypadku zwierząt}\end{PuzzleClues}

\begin{PuzzleClues}{\textbf{Pionowe}\\}\Clue{1}{}{pierwszy egzemplarz druku, który jest przesyłany przez drukarnię redakcji w celu ostatecznego sprawdzenia i zatwierdzenia przed rozpoczęciem rozpowszechniania nakładu}
\Clue{5}{}{liczba moli danej substancji chemicznej jaką zawiera 1 dm3 roztworu}
\Clue{8}{}{dawniej wyznawana religia, wyznanie}
\Clue{13}{}{powieść o bardzo małej objętości; dłuższe opowiadanie}
\Clue{14}{}{materiał literacki będący podstawą realizacji fabuły filmowej lub teatralnej, zawierający dialogi oraz opis postaci i miejsc}
\Clue{15}{}{przyrząd w gimnastyce artystycznej}
\Clue{16}{}{osoba stosująca dietę wegańską, niespożywająca produktów zwierzęcych}
\Clue{18}{}{średniowieczne prawo miejskie, wzorowane na prawie Magdeburga}
\Clue{19}{}{rodzaj grubych, zwykle wełnianych rajstop bez stóp}
\Clue{21}{}{mały żaglowiec towarowy używany na przełomie XIX i XX w}
\Clue{22}{}{możliwość odliczenia od dochodu odsetek naliczanych z tytułu kredytu zaciągniętego na realizację własnych celów mieszkaniowych}
\Clue{23}{}{PILOT; osoba odpowiedzialna za kierowanie statkiem kosmicznym}
\Clue{27}{}{u owadów człon nogi łączący biodro z udem}
\Clue{28}{}{taniec klasyczny}
\Clue{30}{}{aerodyna zdolna do lotu dzięki powstawaniu siły nośnej na wirujących powierzchniach nośnych}
\Clue{31}{}{dość puszysta przędza uzyskiwana w wyniku modyfikacji stilonu}\end{PuzzleClues}

\newpage%\section*{Krzyżówka 1}

\noindent\begin{Puzzle}{24}{32}|*	|*	|*	|*	|*	|[1][S]\darr	|*	|*	|*	|*	|*	|*	|*	|*	|*	|*	|*	|*	|*	|*	|*	|*	|*	|*	|[2][S]\darr	|.
|*	|*	|*	|*	|[3][S]\darr	|n	|*	|*	|*	|[4][S]\drarr	|k	|a	|r	|c	|z	|o	|c	|h	|*	|[5][S]\darr	|*	|*	|[6][S]\darr	|[7][S]\darr	|u	|.
|*	|*	|[8][S]\darr	|[9][S]\drarr	|k	|a	|d	|ź	|*	|p	|[10][S]\rarr	|a	|n	|t	|y	|s	|z	|a	|c	|h	|y	|*	|h	|h	|g	|.
|*	|*	|t	|r	|r	|g	|[11][S]\drarr	|g	|r	|a	|b	|i	|e	|ż	|c	|a	|*	|*	|[12][S]\darr	|i	|*	|*	|i	|o	|i	|.
|*	|*	|e	|u	|w	|i	|d	|*	|*	|t	|*	|*	|[13][S]\drarr	|n	|p	|r	|*	|*	|h	|p	|[14][S]\darr	|*	|p	|l	|n	|.
|*	|*	|ż	|*	|i	|e	|i	|[15][S]\rarr	|c	|o	|a	|h	|u	|i	|l	|a	|*	|*	|i	|e	|k	|[16][S]\darr	|o	|l	|e	|.
|[17][S]\rarr	|b	|u	|s	|o	|l	|a	|*	|*	|l	|*	|[18][S]\rarr	|k	|o	|s	|z	|t	|[][,]{ }	|p	|r	|o	|s	|t	|y	|*	|.
|*	|[19][S]\darr	|*	|[20][S]\darr	|b	|*	|g	|*	|*	|*	|[21][S]\rarr	|w	|a	|r	|g	|a	|*	|*	|n	|p	|t	|t	|e	|w	|*	|.
|[22][S]\drarr	|d	|e	|b	|i	|l	|n	|o	|ś	|ć	|*	|*	|z	|*	|[23][S]\darr	|*	|*	|*	|o	|o	|e	|r	|z	|o	|*	|.
|m	|i	|[24][S]\drarr	|r	|e	|p	|o	|z	|y	|c	|j	|a	|*	|*	|k	|[25][S]\darr	|*	|*	|t	|p	|w	|a	|a	|o	|*	|.
|a	|n	|t	|z	|g	|*	|s	|*	|*	|*	|[26][S]\darr	|[27][S]\darr	|*	|[28][S]\darr	|o	|k	|*	|*	|y	|r	|k	|t	|[][,]{ }	|d	|*	|.
|n	|a	|e	|ą	|[][,]{ }	|[29][S]\rarr	|t	|e	|r	|*	|j	|g	|*	|p	|c	|u	|*	|*	|k	|a	|o	|y	|p	|*	|*	|.
|i	|r	|x	|k	|p	|*	|y	|[30][S]\darr	|*	|*	|e	|r	|*	|o	|z	|l	|[31][S]\darr	|[32][S]\darr	|*	|w	|w	|f	|e	|*	|*	|.
|f	|[][,]{ }	|a	|n	|ł	|*	|k	|m	|*	|*	|l	|y	|*	|d	|o	|t	|b	|ż	|*	|n	|a	|i	|r	|*	|*	|.
|e	|j	|s	|i	|u	|*	|a	|u	|[33][S]\darr	|*	|e	|z	|[34][S]\darr	|g	|w	|[][,]{ }	|i	|y	|*	|o	|t	|k	|m	|*	|*	|.
|s	|u	|*	|ę	|c	|*	|*	|s	|m	|[35][S]\darr	|n	|i	|u	|r	|n	|c	|a	|w	|[36][S]\darr	|ś	|e	|a	|a	|*	|*	|.
|t	|g	|*	|c	|n	|*	|[37][S]\drarr	|z	|a	|s	|i	|e	|d	|z	|i	|a	|ł	|o	|ś	|ć	|*	|c	|n	|*	|*	|.
|[][,]{ }	|o	|*	|i	|y	|*	|m	|t	|m	|p	|o	|l	|z	|e	|c	|r	|a	|t	|r	|*	|[38][S]\darr	|j	|e	|*	|*	|.
|l	|s	|*	|e	|*	|*	|i	|a	|u	|a	|g	|[][,]{ }	|i	|w	|z	|g	|[][,]{ }	|n	|e	|*	|n	|a	|n	|*	|*	|.
|i	|ł	|*	|*	|*	|[39][S]\darr	|c	|r	|t	|ś	|ó	|t	|a	|a	|k	|o	|d	|i	|d	|*	|a	|*	|c	|*	|*	|.
|t	|o	|*	|*	|[40][S]\darr	|z	|z	|d	|o	|l	|r	|a	|ł	|c	|a	|*	|i	|k	|n	|[41][S]\darr	|r	|*	|j	|*	|*	|.
|e	|w	|*	|*	|b	|g	|u	|ó	|w	|a	|z	|p	|o	|z	|[][,]{ }	|[42][S]\darr	|e	|o	|i	|k	|r	|[43][S]\darr	|i	|*	|*	|.
|r	|i	|*	|*	|a	|l	|r	|w	|i	|k	|a	|e	|w	|*	|c	|m	|t	|w	|a	|w	|a	|p	|*	|*	|*	|.
|a	|a	|*	|*	|j	|i	|i	|k	|e	|*	|n	|t	|i	|*	|z	|i	|a	|i	|*	|a	|c	|l	|*	|*	|*	|.
|c	|ń	|*	|*	|e	|s	|n	|a	|c	|[44][S]\darr	|k	|n	|e	|*	|a	|o	|*	|e	|[45][S]\darr	|d	|y	|e	|*	|*	|*	|.
|k	|s	|*	|[46][S]\darr	|c	|z	|i	|*	|*	|g	|a	|i	|c	|*	|r	|d	|*	|c	|d	|r	|j	|ś	|*	|*	|*	|.
|i	|k	|[47][S]\drarr	|s	|z	|c	|z	|a	|p	|a	|*	|k	|*	|*	|n	|u	|*	|*	|y	|u	|n	|n	|[48][S]\darr	|*	|*	|.
|*	|i	|j	|t	|n	|z	|m	|*	|*	|l	|*	|*	|*	|[49][S]\drarr	|a	|n	|t	|r	|o	|p	|o	|i	|d	|*	|*	|.
|*	|*	|u	|a	|o	|a	|*	|*	|*	|t	|*	|*	|*	|g	|*	|k	|*	|*	|n	|l	|ś	|a	|u	|[50][S]\darr	|*	|.
|*	|*	|n	|t	|ś	|*	|*	|*	|*	|*	|[51][S]\rarr	|p	|a	|r	|w	|a	|n	|a	|*	|a	|ć	|k	|o	|w	|*	|.
|*	|*	|k	|k	|ć	|*	|*	|*	|[52][S]\rarr	|g	|n	|o	|m	|o	|n	|*	|*	|*	|*	|*	|*	|*	|l	|a	|*	|.
|*	|*	|o	|i	|*	|*	|*	|*	|[53][S]\rarr	|b	|u	|r	|a	|s	|e	|k	|*	|*	|[54][S]\rarr	|c	|e	|n	|a	|r	|*	|.
|*	|*	|*	|*	|*	|*	|*	|*	|*	|*	|*	|*	|*	|*	|*	|*	|*	|*	|*	|*	|*	|*	|*	|*	|*	|.\end{Puzzle}

\newpage

\begin{PuzzleClues}{\textbf{Poziome}\\}\Clue{4}{}{warzywna bylina pochodzenia śródziemnomorskiego; jadalne, soczyste dno kwiatostanu z listkami okrywowymi}
\Clue{9}{}{zbiornik na ciecz o dużej pojemności}
\Clue{10}{}{odmiana szachów grana na tej samej szachownicy i tymi samymi bierkami co szachy klasyczne, bicie jest obowiązkiem, króla można bić i nie kończy to gry, wygrywa ta strona, która pierwsza straci wszystkie bierki}
\Clue{11}{}{osoba, która kogoś okradła, niekoniecznie złodziej, także przywłaszczyciel, w wyolbrzymieniu może to być też osoba, któraokrada kogoś zgodnie z prawem, np. nieakceptowany spadkobierca}
\Clue{13}{}{kod ISO 4217 rupii nepalskiej}
\Clue{15}{}{stan w płn. Meksyku, pow. 150 tyś. km2, główne miasta: Torreon, Saltillo}
\Clue{17}{}{kompas}
\Clue{18}{}{koszt zawierający jeden rodzaj kosztów}
\Clue{21}{}{struktura okalająca szparę ust, mająca znaczenie przy spożywaniu posiłku i przy artykulacji dźwięków}
\Clue{22}{}{to, że coś jest debilne, a przez to śmieszne, budzące rozbawienie}
\Clue{24}{}{w medycynie: umieszczenie narządu (przesuniątego wskutek nieprawidłowości, choroby, dolegliwości) w jego pierwotnym, prawidłowym miejscu}
\Clue{29}{}{smar składający się ze smoły, kalafonii i tłuszczu, używany przede wszystkim jako smar ochronny}
\Clue{37}{}{cecha kogoś, kto przybył w jakieś miejsce i zbyt długo w nim (wg miejscowych, gospodarzy) siedzi}
\Clue{47}{}{bardzo chudy człowiek lub chude zwierzę}
\Clue{49}{}{człekokształtna małpa wąskonosa}
\Clue{51}{}{miasto w Estonii, port nad Zatoką Ryską; przemysł rybny, drzewny, uzdrowisko (kąpiele błotne)}
\Clue{52}{}{zegar słoneczny}
\Clue{53}{}{kot o burym umaszczeniu}
\Clue{54}{}{muzyka do tańca cenar}\end{PuzzleClues}

\begin{PuzzleClues}{\textbf{Pionowe}\\}\Clue{1}{}{drewniany lub metalowy kołek do obkładania liny}
\Clue{2}{}{miasto we Francji na wsch. od Lyonu, przemysł metalowy}
\Clue{3}{}{obieg, w którym krew żylna uboga w tlen prowadzona jest z prawej komory serca do płuc, a po oddaniu dwutlenku węgla i pobraniu tlenu wraca do lewego przedsionka}
\Clue{4}{}{człowiek z marginesu społecznego, z wykluczonej, trudnej grupy}
\Clue{5}{}{przesadna poprawność}
\Clue{6}{}{teoria geotektoniczna, która powstała w XIX wieku, postulująca stałe położenie kontynentów na Ziemi od momentu ich powstania}
\Clue{7}{}{najważniejszy ośrodek amerykańskiej kinematografii}
\Clue{8}{}{TEJU}
\Clue{9}{}{w chemii: symbol rutenu}
\Clue{11}{}{czynność służąca rozpoznaniu choroby lub stanu zdrowia, element pracy badawczej lekarza}
\Clue{12}{}{lek z grupy leków, których prymarną funkcją jest leczenie zaburzeń snu}
\Clue{13}{}{z odcieniem żartu lub ironii: nakaz, rozporządzenie, polecenie}
\Clue{14}{}{Trapaceae - monotypowa rodzina roślin wodnych z rzędu mirtowców, wyróżniana w dawniejszych systemach klasyfikacyjnych roślin okrytonasiennych}
\Clue{16}{}{podział społeczeństwa, rozwarstwienie na zależne od siebie i powiązane ze sobą warstwy, których wyróżnianie oparte jest na władzy, pieniądzach, prestiżu, wykształceniu oraz zdrowiu}
\Clue{19}{}{waluta obowiązująca w Jugosławii}
\Clue{20}{}{wydanie dźwięku brząknięcia}
\Clue{22}{}{publikacja prezentująca założenia programowe jakiegoś kierunku, grupy literackiej, nawet pojedynczego pisarza, specyficzna forma publicystyki literackiej}
\Clue{23}{}{Tapinoma erraticum - gatunek mrówki z podrodziny Dolichoderinae}
\Clue{24}{}{polska odmiana dżinsu}
\Clue{25}{}{forma ruchu religijnego z wysp Oceanu Spokojnego, glosząca nadejście nowego porządku: równości wszystkich ludzi, białych i czarnych, oraz powszechnego dobrobytu}
\Clue{26}{}{mieszkanka Jeleniej Góry}
\Clue{27}{}{Atypus piceus - gatunek pająka z rodziny gryzielowatych; jeden z trzech gatunków występujących w Polsce, rzadki; wielkość samca do 9 mm, samicy nawet do 20 mm}
\Clue{28}{}{źródło ciepła; urządzenie, które jest przeznaczone do podgrzewania}
\Clue{30}{}{pojemnik po musztardzie, który służy jako szklanka}
\Clue{31}{}{dieta, którą zaleca dentysta po czyszczeniu lub wybielaniu zębów, żeby uchronić je przed przebarwieniami; w tym czasie unika się kawy, herbaty, papierosów, wina, nie je się np. buraków}
\Clue{32}{}{drzewo iglaste z rodziny cyprysowatych, pochodzi z Japonii, w Polsce sadzony w parkach}
\Clue{33}{}{WELINGTONIA - najpotgżniejsze drzewo iglaste (130 m), ginące i chronione - góry Sierra Newada w Kaliforni}
\Clue{34}{}{współwłaściciel spółki będący posiadaczem wyemitowanych przez nią akcji}
\Clue{35}{}{obraźliwie o kimś otyłym}
\Clue{36}{}{statystyka (funkcja) stosowana jako tzw. miara tendencji centralnej, tzn. wskaźnik pokazujący w jakiś sposóbśrodek rozkładu;środek można zdefiniować na wiele sposobów, istnieje też wiele rodzajów średniej}
\Clue{37}{}{pogląd przypisujący człowiekowi zdolność do niemal dowolnego przeobrażania przyrody, według którego jeśli jeden ze składników w mieszańcu wegetatywnym (szczepieniowym) ma przewagę nad drugim, to przyjmuje rolę mentora, przekształcając mieszańca tak, że przekazuje mu swoje właściwości i cechy, co pozwala na dowolne kształtowanie cech upraw}
\Clue{38}{}{cecha utworu, w którym sposób wypowiedzi ma na celu  przedstawienie zdarzeń w określonym porządku czasowym}
\Clue{39}{}{ruiny, spalone szczątki budowli, miejsce po pożarze}
\Clue{40}{}{cecha kogoś lub czegoś, co wydaje się wspaniałe, doskonałe, niezwykle piękne}
\Clue{41}{}{poczwórna ilość}
\Clue{42}{}{miododajna bylina z szorstkolistnych - Eurazja}
\Clue{43}{}{Mucor - rodzaj grzybów sprzężniaków z rzędu pleśniakowców (Mucorales) obejmujący około 50 gatunków}
\Clue{44}{}{miasto w Kanadzie w prowincji Ontario, ważny węzeł kolejowy}
\Clue{45}{}{jednostka taktyczna złożona z maksymalnie pięciu okrętów tego samego typu}
\Clue{46}{}{gra w statki}
\Clue{47}{}{północnoamerykański ptak z rodziny ziarnojadów}
\Clue{48}{}{figura rytmiczna składająca się z dwóch równych nut}
\Clue{49}{}{malarz francuski (1771-1835) obrazy przedstawiające epopeję napoleońska, portrety}
\Clue{50}{}{ciecz podgrzana do temperatury, w której zaczyna wrzeć; ciecz bardzo gorąca, powodująca oparzenia w kontakcie ze skórą}\end{PuzzleClues}

\end{document}
