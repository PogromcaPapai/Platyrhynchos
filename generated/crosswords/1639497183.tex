
\documentclass[11pt]{article}
\usepackage[utf8]{inputenc}

\usepackage[unboxed]{cwpuzzle}
\usepackage[top=15mm,bottom=15mm,left=15mm,right=3cm,twoside]{geometry}
\usepackage{adjustbox}

\newcommand\drarr{$\rightarrow \!\!\!\!\! \downarrow$}
\newcommand\rarr{$\rightarrow$}
\newcommand\darr{$\downarrow$}

\begin{document}



\newpage\section*{Krzyżówka 1}

\noindent\begin{Puzzle}{24}{32}|*	|*	|*	|*	|*	|[1][S]\darr	|*	|*	|*	|*	|*	|*	|*	|*	|*	|*	|*	|*	|*	|*	|*	|*	|*	|*	|[2][S]\darr	|.
|*	|*	|*	|*	|[3][S]\darr	|n	|*	|*	|*	|[4][S]\drarr	|k	|a	|r	|c	|z	|o	|c	|h	|*	|[5][S]\darr	|*	|*	|[6][S]\darr	|[7][S]\darr	|u	|.
|*	|*	|[8][S]\darr	|[9][S]\drarr	|k	|a	|d	|ź	|*	|p	|[10][S]\rarr	|a	|n	|t	|y	|s	|z	|a	|c	|h	|y	|*	|h	|h	|g	|.
|*	|*	|t	|r	|r	|g	|[11][S]\drarr	|g	|r	|a	|b	|i	|e	|ż	|c	|a	|*	|*	|[12][S]\darr	|i	|*	|*	|i	|o	|i	|.
|*	|*	|e	|u	|w	|i	|d	|*	|*	|t	|*	|*	|[13][S]\drarr	|n	|p	|r	|*	|*	|h	|p	|[14][S]\darr	|*	|p	|l	|n	|.
|*	|*	|ż	|*	|i	|e	|i	|[15][S]\rarr	|c	|o	|a	|h	|u	|i	|l	|a	|*	|*	|i	|e	|k	|[16][S]\darr	|o	|l	|e	|.
|[17][S]\rarr	|b	|u	|s	|o	|l	|a	|*	|*	|l	|*	|[18][S]\rarr	|k	|o	|s	|z	|t	|[][,]{ }	|p	|r	|o	|s	|t	|y	|*	|.
|*	|[19][S]\darr	|*	|[20][S]\darr	|b	|*	|g	|*	|*	|*	|[21][S]\rarr	|w	|a	|r	|g	|a	|*	|*	|n	|p	|t	|t	|e	|w	|*	|.
|[22][S]\drarr	|d	|e	|b	|i	|l	|n	|o	|ś	|ć	|*	|*	|z	|*	|[23][S]\darr	|*	|*	|*	|o	|o	|e	|r	|z	|o	|*	|.
|m	|i	|[24][S]\drarr	|r	|e	|p	|o	|z	|y	|c	|j	|a	|*	|*	|k	|[25][S]\darr	|*	|*	|t	|p	|w	|a	|a	|o	|*	|.
|a	|n	|t	|z	|g	|*	|s	|*	|*	|*	|[26][S]\darr	|[27][S]\darr	|*	|[28][S]\darr	|o	|k	|*	|*	|y	|r	|k	|t	|[][,]{ }	|d	|*	|.
|n	|a	|e	|ą	|[][,]{ }	|[29][S]\rarr	|t	|e	|r	|*	|j	|g	|*	|p	|c	|u	|*	|*	|k	|a	|o	|y	|p	|*	|*	|.
|i	|r	|x	|k	|p	|*	|y	|[30][S]\darr	|*	|*	|e	|r	|*	|o	|z	|l	|[31][S]\darr	|[32][S]\darr	|*	|w	|w	|f	|e	|*	|*	|.
|f	|[][,]{ }	|a	|n	|ł	|*	|k	|m	|*	|*	|l	|y	|*	|d	|o	|t	|b	|ż	|*	|n	|a	|i	|r	|*	|*	|.
|e	|j	|s	|i	|u	|*	|a	|u	|[33][S]\darr	|*	|e	|z	|[34][S]\darr	|g	|w	|[][,]{ }	|i	|y	|*	|o	|t	|k	|m	|*	|*	|.
|s	|u	|*	|ę	|c	|*	|*	|s	|m	|[35][S]\darr	|n	|i	|u	|r	|n	|c	|a	|w	|[36][S]\darr	|ś	|e	|a	|a	|*	|*	|.
|t	|g	|*	|c	|n	|*	|[37][S]\drarr	|z	|a	|s	|i	|e	|d	|z	|i	|a	|ł	|o	|ś	|ć	|*	|c	|n	|*	|*	|.
|[][,]{ }	|o	|*	|i	|y	|*	|m	|t	|m	|p	|o	|l	|z	|e	|c	|r	|a	|t	|r	|*	|[38][S]\darr	|j	|e	|*	|*	|.
|l	|s	|*	|e	|*	|*	|i	|a	|u	|a	|g	|[][,]{ }	|i	|w	|z	|g	|[][,]{ }	|n	|e	|*	|n	|a	|n	|*	|*	|.
|i	|ł	|*	|*	|*	|[39][S]\darr	|c	|r	|t	|ś	|ó	|t	|a	|a	|k	|o	|d	|i	|d	|*	|a	|*	|c	|*	|*	|.
|t	|o	|*	|*	|[40][S]\darr	|z	|z	|d	|o	|l	|r	|a	|ł	|c	|a	|*	|i	|k	|n	|[41][S]\darr	|r	|*	|j	|*	|*	|.
|e	|w	|*	|*	|b	|g	|u	|ó	|w	|a	|z	|p	|o	|z	|[][,]{ }	|[42][S]\darr	|e	|o	|i	|k	|r	|[43][S]\darr	|i	|*	|*	|.
|r	|i	|*	|*	|a	|l	|r	|w	|i	|k	|a	|e	|w	|*	|c	|m	|t	|w	|a	|w	|a	|p	|*	|*	|*	|.
|a	|a	|*	|*	|j	|i	|i	|k	|e	|*	|n	|t	|i	|*	|z	|i	|a	|i	|*	|a	|c	|l	|*	|*	|*	|.
|c	|ń	|*	|*	|e	|s	|n	|a	|c	|[44][S]\darr	|k	|n	|e	|*	|a	|o	|*	|e	|[45][S]\darr	|d	|y	|e	|*	|*	|*	|.
|k	|s	|*	|[46][S]\darr	|c	|z	|i	|*	|*	|g	|a	|i	|c	|*	|r	|d	|*	|c	|d	|r	|j	|ś	|*	|*	|*	|.
|i	|k	|[47][S]\drarr	|s	|z	|c	|z	|a	|p	|a	|*	|k	|*	|*	|n	|u	|*	|*	|y	|u	|n	|n	|[48][S]\darr	|*	|*	|.
|*	|i	|j	|t	|n	|z	|m	|*	|*	|l	|*	|*	|*	|[49][S]\drarr	|a	|n	|t	|r	|o	|p	|o	|i	|d	|*	|*	|.
|*	|*	|u	|a	|o	|a	|*	|*	|*	|t	|*	|*	|*	|g	|*	|k	|*	|*	|n	|l	|ś	|a	|u	|[50][S]\darr	|*	|.
|*	|*	|n	|t	|ś	|*	|*	|*	|*	|*	|[51][S]\rarr	|p	|a	|r	|w	|a	|n	|a	|*	|a	|ć	|k	|o	|w	|*	|.
|*	|*	|k	|k	|ć	|*	|*	|*	|[52][S]\rarr	|g	|n	|o	|m	|o	|n	|*	|*	|*	|*	|*	|*	|*	|l	|a	|*	|.
|*	|*	|o	|i	|*	|*	|*	|*	|[53][S]\rarr	|b	|u	|r	|a	|s	|e	|k	|*	|*	|[54][S]\rarr	|c	|e	|n	|a	|r	|*	|.
|*	|*	|*	|*	|*	|*	|*	|*	|*	|*	|*	|*	|*	|*	|*	|*	|*	|*	|*	|*	|*	|*	|*	|*	|*	|.\end{Puzzle}

\newpage

\begin{PuzzleClues}{\textbf{Poziome}\\}\Clue{4}{}{warzywna bylina pochodzenia śródziemnomorskiego; jadalne, soczyste dno kwiatostanu z listkami okrywowymi}
\Clue{9}{}{zbiornik na ciecz o dużej pojemności}
\Clue{10}{}{odmiana szachów grana na tej samej szachownicy i tymi samymi bierkami co szachy klasyczne, bicie jest obowiązkiem, króla można bić i nie kończy to gry, wygrywa ta strona, która pierwsza straci wszystkie bierki}
\Clue{11}{}{osoba, która kogoś okradła, niekoniecznie złodziej, także przywłaszczyciel, w wyolbrzymieniu może to być też osoba, któraokrada kogoś zgodnie z prawem, np. nieakceptowany spadkobierca}
\Clue{13}{}{kod ISO 4217 rupii nepalskiej}
\Clue{15}{}{stan w płn. Meksyku, pow. 150 tyś. km2, główne miasta: Torreon, Saltillo}
\Clue{17}{}{kompas}
\Clue{18}{}{koszt zawierający jeden rodzaj kosztów}
\Clue{21}{}{struktura okalająca szparę ust, mająca znaczenie przy spożywaniu posiłku i przy artykulacji dźwięków}
\Clue{22}{}{to, że coś jest debilne, a przez to śmieszne, budzące rozbawienie}
\Clue{24}{}{w medycynie: umieszczenie narządu (przesuniątego wskutek nieprawidłowości, choroby, dolegliwości) w jego pierwotnym, prawidłowym miejscu}
\Clue{29}{}{smar składający się ze smoły, kalafonii i tłuszczu, używany przede wszystkim jako smar ochronny}
\Clue{37}{}{cecha kogoś, kto przybył w jakieś miejsce i zbyt długo w nim (wg miejscowych, gospodarzy) siedzi}
\Clue{47}{}{bardzo chudy człowiek lub chude zwierzę}
\Clue{49}{}{człekokształtna małpa wąskonosa}
\Clue{51}{}{miasto w Estonii, port nad Zatoką Ryską; przemysł rybny, drzewny, uzdrowisko (kąpiele błotne)}
\Clue{52}{}{zegar słoneczny}
\Clue{53}{}{kot o burym umaszczeniu}
\Clue{54}{}{muzyka do tańca cenar}\end{PuzzleClues}

\begin{PuzzleClues}{\textbf{Pionowe}\\}\Clue{1}{}{drewniany lub metalowy kołek do obkładania liny}
\Clue{2}{}{miasto we Francji na wsch. od Lyonu, przemysł metalowy}
\Clue{3}{}{obieg, w którym krew żylna uboga w tlen prowadzona jest z prawej komory serca do płuc, a po oddaniu dwutlenku węgla i pobraniu tlenu wraca do lewego przedsionka}
\Clue{4}{}{człowiek z marginesu społecznego, z wykluczonej, trudnej grupy}
\Clue{5}{}{przesadna poprawność}
\Clue{6}{}{teoria geotektoniczna, która powstała w XIX wieku, postulująca stałe położenie kontynentów na Ziemi od momentu ich powstania}
\Clue{7}{}{najważniejszy ośrodek amerykańskiej kinematografii}
\Clue{8}{}{TEJU}
\Clue{9}{}{w chemii: symbol rutenu}
\Clue{11}{}{czynność służąca rozpoznaniu choroby lub stanu zdrowia, element pracy badawczej lekarza}
\Clue{12}{}{lek z grupy leków, których prymarną funkcją jest leczenie zaburzeń snu}
\Clue{13}{}{z odcieniem żartu lub ironii: nakaz, rozporządzenie, polecenie}
\Clue{14}{}{Trapaceae - monotypowa rodzina roślin wodnych z rzędu mirtowców, wyróżniana w dawniejszych systemach klasyfikacyjnych roślin okrytonasiennych}
\Clue{16}{}{podział społeczeństwa, rozwarstwienie na zależne od siebie i powiązane ze sobą warstwy, których wyróżnianie oparte jest na władzy, pieniądzach, prestiżu, wykształceniu oraz zdrowiu}
\Clue{19}{}{waluta obowiązująca w Jugosławii}
\Clue{20}{}{wydanie dźwięku brząknięcia}
\Clue{22}{}{publikacja prezentująca założenia programowe jakiegoś kierunku, grupy literackiej, nawet pojedynczego pisarza, specyficzna forma publicystyki literackiej}
\Clue{23}{}{Tapinoma erraticum - gatunek mrówki z podrodziny Dolichoderinae}
\Clue{24}{}{polska odmiana dżinsu}
\Clue{25}{}{forma ruchu religijnego z wysp Oceanu Spokojnego, glosząca nadejście nowego porządku: równości wszystkich ludzi, białych i czarnych, oraz powszechnego dobrobytu}
\Clue{26}{}{mieszkanka Jeleniej Góry}
\Clue{27}{}{Atypus piceus - gatunek pająka z rodziny gryzielowatych; jeden z trzech gatunków występujących w Polsce, rzadki; wielkość samca do 9 mm, samicy nawet do 20 mm}
\Clue{28}{}{źródło ciepła; urządzenie, które jest przeznaczone do podgrzewania}
\Clue{30}{}{pojemnik po musztardzie, który służy jako szklanka}
\Clue{31}{}{dieta, którą zaleca dentysta po czyszczeniu lub wybielaniu zębów, żeby uchronić je przed przebarwieniami; w tym czasie unika się kawy, herbaty, papierosów, wina, nie je się np. buraków}
\Clue{32}{}{drzewo iglaste z rodziny cyprysowatych, pochodzi z Japonii, w Polsce sadzony w parkach}
\Clue{33}{}{WELINGTONIA - najpotgżniejsze drzewo iglaste (130 m), ginące i chronione - góry Sierra Newada w Kaliforni}
\Clue{34}{}{współwłaściciel spółki będący posiadaczem wyemitowanych przez nią akcji}
\Clue{35}{}{obraźliwie o kimś otyłym}
\Clue{36}{}{statystyka (funkcja) stosowana jako tzw. miara tendencji centralnej, tzn. wskaźnik pokazujący w jakiś sposóbśrodek rozkładu;środek można zdefiniować na wiele sposobów, istnieje też wiele rodzajów średniej}
\Clue{37}{}{pogląd przypisujący człowiekowi zdolność do niemal dowolnego przeobrażania przyrody, według którego jeśli jeden ze składników w mieszańcu wegetatywnym (szczepieniowym) ma przewagę nad drugim, to przyjmuje rolę mentora, przekształcając mieszańca tak, że przekazuje mu swoje właściwości i cechy, co pozwala na dowolne kształtowanie cech upraw}
\Clue{38}{}{cecha utworu, w którym sposób wypowiedzi ma na celu  przedstawienie zdarzeń w określonym porządku czasowym}
\Clue{39}{}{ruiny, spalone szczątki budowli, miejsce po pożarze}
\Clue{40}{}{cecha kogoś lub czegoś, co wydaje się wspaniałe, doskonałe, niezwykle piękne}
\Clue{41}{}{poczwórna ilość}
\Clue{42}{}{miododajna bylina z szorstkolistnych - Eurazja}
\Clue{43}{}{Mucor - rodzaj grzybów sprzężniaków z rzędu pleśniakowców (Mucorales) obejmujący około 50 gatunków}
\Clue{44}{}{miasto w Kanadzie w prowincji Ontario, ważny węzeł kolejowy}
\Clue{45}{}{jednostka taktyczna złożona z maksymalnie pięciu okrętów tego samego typu}
\Clue{46}{}{gra w statki}
\Clue{47}{}{północnoamerykański ptak z rodziny ziarnojadów}
\Clue{48}{}{figura rytmiczna składająca się z dwóch równych nut}
\Clue{49}{}{malarz francuski (1771-1835) obrazy przedstawiające epopeję napoleońska, portrety}
\Clue{50}{}{ciecz podgrzana do temperatury, w której zaczyna wrzeć; ciecz bardzo gorąca, powodująca oparzenia w kontakcie ze skórą}\end{PuzzleClues}\newpage\section*{Krzyżówka 2}

\noindent\begin{Puzzle}{24}{31}|*	|*	|*	|*	|*	|*	|*	|*	|*	|*	|*	|*	|*	|*	|*	|[1][S]\drarr	|g	|r	|a	|f	|f	|i	|t	|i	|*	|.
|*	|*	|*	|*	|*	|*	|*	|*	|*	|*	|*	|*	|*	|*	|[2][S]\rarr	|p	|e	|n	|d	|y	|n	|k	|a	|*	|*	|.
|[3][S]\drarr	|p	|o	|j	|e	|m	|n	|o	|ś	|ć	|[][,]{ }	|a	|d	|s	|o	|r	|p	|c	|y	|j	|n	|a	|*	|*	|*	|.
|a	|*	|[4][S]\drarr	|t	|y	|t	|a	|n	|i	|a	|n	|*	|*	|[5][S]\rarr	|b	|ą	|c	|z	|e	|k	|*	|*	|[6][S]\darr	|*	|*	|.
|u	|[7][S]\darr	|k	|*	|*	|[8][S]\darr	|*	|[9][S]\rarr	|b	|u	|r	|a	|k	|[][,]{ }	|s	|t	|o	|ł	|o	|w	|y	|*	|s	|*	|*	|.
|d	|k	|o	|*	|*	|g	|*	|*	|*	|[10][S]\rarr	|z	|b	|r	|o	|d	|n	|i	|a	|*	|*	|*	|*	|a	|[11][S]\darr	|*	|.
|i	|l	|a	|[12][S]\rarr	|j	|a	|ś	|[][,]{ }	|w	|ę	|d	|r	|o	|w	|n	|i	|c	|z	|e	|k	|*	|*	|n	|b	|*	|.
|*	|a	|d	|*	|*	|ś	|*	|[13][S]\darr	|[14][S]\drarr	|n	|o	|w	|o	|h	|u	|c	|i	|a	|n	|k	|a	|*	|d	|r	|*	|.
|[15][S]\drarr	|m	|i	|e	|r	|n	|i	|k	|o	|w	|i	|e	|c	|*	|[16][S]\rarr	|z	|a	|ś	|l	|a	|z	|*	|w	|y	|*	|.
|s	|r	|u	|*	|*	|i	|*	|a	|p	|*	|[17][S]\rarr	|b	|a	|n	|i	|e	|c	|z	|k	|a	|*	|*	|i	|t	|*	|.
|z	|a	|t	|*	|*	|k	|*	|n	|ł	|*	|*	|*	|*	|*	|[18][S]\rarr	|k	|i	|k	|u	|t	|n	|i	|c	|a	|*	|.
|a	|[][,]{ }	|o	|*	|*	|*	|*	|o	|a	|[19][S]\rarr	|s	|t	|a	|r	|e	|[][,]{ }	|p	|u	|d	|ł	|o	|*	|z	|n	|*	|.
|j	|k	|r	|[20][S]\darr	|*	|[21][S]\darr	|[22][S]\darr	|n	|t	|*	|[23][S]\rarr	|k	|o	|l	|e	|k	|t	|y	|w	|i	|z	|m	|*	|*	|*	|.
|r	|o	|k	|i	|*	|s	|c	|i	|a	|*	|*	|*	|[24][S]\rarr	|w	|y	|r	|ó	|w	|n	|a	|n	|i	|e	|*	|*	|.
|y	|m	|a	|d	|[25][S]\darr	|o	|i	|c	|[][,]{ }	|[26][S]\drarr	|r	|y	|b	|a	|c	|z	|k	|a	|*	|*	|*	|*	|*	|*	|*	|.
|*	|p	|*	|e	|t	|l	|r	|z	|w	|p	|*	|*	|[27][S]\darr	|[28][S]\drarr	|w	|y	|c	|i	|ę	|c	|i	|e	|*	|*	|*	|.
|*	|o	|*	|n	|e	|a	|e	|k	|i	|r	|*	|*	|l	|k	|*	|w	|*	|*	|*	|[29][S]\darr	|*	|[30][S]\darr	|*	|*	|*	|.
|*	|z	|[31][S]\drarr	|t	|u	|r	|b	|i	|n	|a	|[][,]{ }	|p	|e	|l	|t	|o	|n	|a	|*	|d	|*	|c	|*	|*	|*	|.
|*	|y	|t	|y	|t	|z	|o	|*	|i	|k	|*	|[32][S]\darr	|c	|u	|*	|s	|[33][S]\rarr	|t	|r	|i	|n	|i	|a	|*	|*	|.
|*	|c	|o	|f	|o	|*	|n	|*	|e	|t	|*	|t	|*	|c	|*	|z	|*	|*	|*	|n	|[34][S]\darr	|e	|*	|[35][S]\darr	|*	|.
|*	|y	|m	|i	|n	|*	|*	|*	|t	|y	|*	|w	|*	|z	|*	|y	|*	|*	|*	|h	|l	|n	|*	|s	|*	|.
|*	|j	|i	|k	|*	|[36][S]\rarr	|o	|b	|o	|k	|n	|i	|e	|*	|*	|j	|*	|*	|*	|e	|i	|n	|[37][S]\darr	|e	|*	|.
|*	|n	|z	|a	|*	|[38][S]\rarr	|s	|z	|w	|a	|l	|n	|i	|a	|*	|k	|[39][S]\darr	|*	|*	|i	|s	|i	|p	|l	|*	|.
|*	|a	|m	|t	|[40][S]\rarr	|d	|u	|m	|a	|*	|*	|g	|[41][S]\drarr	|d	|i	|o	|p	|t	|e	|r	|*	|k	|a	|e	|*	|.
|*	|*	|*	|o	|*	|[42][S]\rarr	|a	|t	|*	|*	|*	|o	|p	|*	|*	|w	|a	|[43][S]\rarr	|b	|o	|r	|*	|s	|k	|*	|.
|*	|[44][S]\rarr	|b	|r	|y	|t	|f	|a	|n	|n	|a	|*	|o	|*	|*	|y	|p	|*	|*	|z	|*	|*	|z	|t	|*	|.
|*	|*	|*	|*	|*	|*	|*	|*	|*	|[45][S]\rarr	|w	|y	|k	|o	|t	|*	|a	|[46][S]\rarr	|c	|a	|b	|o	|t	|o	|*	|.
|[47][S]\rarr	|s	|z	|u	|m	|[][,]{ }	|n	|a	|d	|m	|i	|a	|r	|o	|w	|y	|*	|[48][S]\rarr	|b	|u	|s	|t	|e	|r	|*	|.
|*	|*	|*	|*	|*	|[49][S]\rarr	|t	|a	|n	|g	|e	|r	|y	|n	|k	|a	|*	|[50][S]\rarr	|v	|r	|ś	|a	|c	|*	|*	|.
|[51][S]\rarr	|u	|r	|z	|ą	|d	|z	|e	|n	|i	|e	|[][,]{ }	|w	|y	|j	|ś	|c	|i	|a	|*	|*	|*	|i	|*	|*	|.
|*	|*	|*	|*	|*	|*	|*	|*	|[52][S]\rarr	|g	|o	|t	|y	|k	|[][,]{ }	|b	|r	|a	|b	|a	|n	|c	|k	|i	|*	|.
|[53][S]\rarr	|e	|k	|w	|a	|d	|o	|r	|c	|z	|y	|k	|*	|*	|*	|*	|*	|*	|*	|*	|*	|*	|*	|*	|*	|.\end{Puzzle}

\newpage

\begin{PuzzleClues}{\textbf{Poziome}\\}\Clue{1}{}{elementy wizualne, np. obrazy, podpisy lub rysunki, które są umieszczane w przestrzeni publicznej lub prywatnej za pomocą różnych technik}
\Clue{2}{}{leiszmanioza wywoływana przez L. brasiliensis, której objawami są zniekształcenia twarzy, uszkodzenia tkanek miękkich, chrząstek i kości nosa}
\Clue{3}{}{wielkość fizyczna, określająca, jaka część powierzchni adsorbentu jest zajęta przez cząsteczki adsorbatu}
\Clue{4}{}{formalna nazwa soli kwasu tytanowego, nieistniejącego w roztworze wodnym; ogólny wzór soli to MTiO3, gdzie M oznacza metal na +II stopniu utlenienia; sole te otrzymuje się przez spiekanie ze sobą tlenku tytanu (TiO2) i tlenku innego metalu}
\Clue{5}{}{mały bąk; zdrobnienie od popularnej, ale nieprawidłowej nazwy rodzaju - bąk}
\Clue{9}{}{duży, jadalny burak o intensywnie czerwonym korzeniu oraz czerwonawych liściach; uprawiany jest na barszcz i jako jarzyna}
\Clue{10}{}{czyn popełniony wbrew przyjętej moralności, zasługujący na potępienie; jest to znaczenie przenośne}
\Clue{12}{}{potoczne w Polsce określenie szkockiej whisky (Johnnie Walker)}
\Clue{14}{}{mieszkanka Nowej Huty}
\Clue{15}{}{typ motyla (rodzina miernikowcowatych); nazwa pochodzi od ruchu gąsienic, które wyglądają tak, jakby odmierzały odległości}
\Clue{16}{}{roślina zielna ze ślazowatych uprawiana w Azji na grube włókna i jako roślina lecznicza}
\Clue{17}{}{tyle, ile zmieści się w małej bańce}
\Clue{18}{}{stawonóg morski pokrojem zbliżony do pająka}
\Clue{19}{}{pogardliwe określenie starej kobiety}
\Clue{23}{}{przeciwstawiany indywidualizmowi pogląd akcentujący rolę wspólnot, grup i zbiorowości}
\Clue{24}{}{suma uzupełniająca niedobory}
\Clue{26}{}{żona mężczyzny zajmującego się rybołóstwem}
\Clue{28}{}{górne wykończenie części ubrania zakrywającej tułów, w odzieży damskiej często wycięte, ozdobione}
\Clue{31}{}{turbina wodna, w której łopatki są ustawione pod kątem 90 stopni do strumienia wody, a protoplastą dla obydwu jest koło młyńskie}
\Clue{33}{}{roślina o białych lub różowych kwiatach, z rodziny baldaszkowatych}
\Clue{36}{}{futryna okienna}
\Clue{38}{}{pracownia bieliźniarska}
\Clue{40}{}{gatunek epicko-liryczny w literaturze ukraińskiej}
\Clue{41}{}{przeziernik; element celownika umożliwiający kontrolę wzrokową położenia przyrządu względem obserwowanego obiektu}
\Clue{42}{}{w chemii: symbol astatu}
\Clue{43}{}{miasto we wschodniej Serbii}
\Clue{44}{}{metalowe naczynie służące do pieczenia mięsa lub ciast}
\Clue{45}{}{narodziny młodych u królika, zająca, sarny, kozicy i muflona}
\Clue{46}{}{żeglarz włoski (1450-89); pierwszy dotarł do Zatoki Hudsona, sporządził mapę świata}
\Clue{47}{}{rodzaj szumu, który wynika z niedoskonałości powierzchni emitujących nośniki i z procesami generacyjno-rekombinacyjnymi na powierzchni półprzewodników, złącz oraz przepływem prądu przez ośrodek nieciągły}
\Clue{48}{}{urządzenie do zwiększania mocy działania maszyny przy znacznym wzroście obciążenia}
\Clue{49}{}{roślina cytrusowa, krzyżówka mandarynki i grejpfruta}
\Clue{50}{}{miasto w Jugosławii (Serbia) w okręgu autonomicznym Wojwodina; węzeł kolejowy}
\Clue{51}{}{urządzenie służące do komunikacji systemu komputerowego z użytkownikiem, umożliwiające odbiór przez użytkownika danych z systemu}
\Clue{52}{}{rodzaj stylu gotyckiego w architekturze nawiązujący do klasycznego gotyku francuskiego, z bogatymi zdobieniami i charakterystycznym wykończeniem okrągłych kolumn głowicą z wyrzeźbionymi liśćmi kapusty}
\Clue{53}{}{mieszkaniec Ekwadoru, człowiek pochodzenia ekwadorskiego}\end{PuzzleClues}

\begin{PuzzleClues}{\textbf{Pionowe}\\}\Clue{1}{}{Microbryum curvicolle - gatunek mchu z rodziny płoniwowatych}
\Clue{3}{}{marka samochodu; niemiecki producent samochodów osobowych należący do koncernu Volkswagen}
\Clue{4}{}{w Kościele katolickim: zakonnica oddana do pomocy przełożonej w zakonie}
\Clue{6}{}{kanapka z dwóch kawałków chleba lub bułki przełożonych wędliną, serem}
\Clue{7}{}{fragmenty na początku i końcu utworu (zdanie, krótsza lub dłuższa wypowiedź), który swoją treścią scala cały utwór}
\Clue{8}{}{przyrząd służący do gaszenia i zapalania świec; ma postać drewnianego kija odpowiedniej długości, na którego szczycie znajduje się stożkowy metalowy kołpak}
\Clue{11}{}{pies obronny}
\Clue{13}{}{zgromadzenie religijne kobiet wiodących wspólne życie według ustalonej reguły}
\Clue{14}{}{opłata za korzystanie z dróg publicznych}
\Clue{15}{}{SHIRES, HETKA, SZKAPA; angielska rasa koni pociągowych}
\Clue{20}{}{urządzenie służące do rozpoznawania zgłaszającego się łącza wejściowego}
\Clue{21}{}{kupiec zajmujący się sprzedażą soli}
\Clue{22}{}{miasto i port w Indonezji na płd.-zach. wybrzeżu Jawy}
\Clue{25}{}{krzyżak - członek Zakonu Krzyżackiego}
\Clue{26}{}{okres szkolenia się}
\Clue{27}{}{(1909-66) poeta i satyryk; „Myśli nieuczesane”, „Kpię i drogę pytam”, „Do Abla i Kaina”}
\Clue{28}{}{środek, sposób na poradzenie sobie z czymś, rozgryzienie, zrozumienie czegoś}
\Clue{29}{}{Dinheirosaurus - rodzaj zauropoda z rodziny diplodoków; żył w epoce późnej jury na terenach obecnej Europy}
\Clue{30}{}{parasol, materiałowa zasłonka, która chroni przed światłem}
\Clue{31}{}{kierunek stanowiący oficjalną filozofię Kościoła katolickiego głoszący tezę o stworzeniu świata przez Boga osobowego, odrzucający wszystko co jest sprzeczne z treścią dogmatów}
\Clue{32}{}{renault z modelu Twingo}
\Clue{34}{}{wyprawione futro z lisa}
\Clue{35}{}{urządzenie, które ma za zadanie rozdzielenie czegoś, np. częstotliwości}
\Clue{37}{}{zdrobniale: pasztet - pasta zazwyczaj mięsna do smarowania chleba}
\Clue{39}{}{tektura nasycona masą smołową stosowana między innymi do pokrycia dachów}
\Clue{41}{}{pierwsza schitinizowana para skrzydeł chrząszczy, owadów i pluskwiaków}\end{PuzzleClues}\newpage\section*{Krzyżówka 3}

\noindent\begin{Puzzle}{21}{33}|*	|[1][S]\darr	|*	|*	|*	|*	|*	|*	|*	|*	|*	|*	|*	|*	|*	|*	|*	|*	|*	|*	|*	|*	|.
|*	|u	|*	|[2][S]\drarr	|m	|o	|d	|*	|*	|[3][S]\drarr	|p	|o	|l	|i	|g	|a	|m	|i	|a	|*	|*	|*	|.
|[4][S]\drarr	|d	|u	|c	|h	|*	|*	|*	|*	|f	|*	|[5][S]\drarr	|s	|e	|k	|r	|e	|c	|j	|a	|*	|[6][S]\darr	|.
|j	|e	|[7][S]\rarr	|e	|v	|a	|n	|s	|*	|o	|*	|n	|[8][S]\darr	|*	|[9][S]\darr	|*	|*	|*	|*	|*	|*	|t	|.
|o	|r	|*	|m	|*	|[10][S]\darr	|*	|*	|[11][S]\drarr	|t	|r	|a	|m	|p	|k	|a	|r	|z	|*	|*	|*	|e	|.
|d	|z	|*	|e	|*	|b	|*	|*	|m	|o	|*	|m	|a	|*	|a	|*	|*	|*	|*	|*	|*	|r	|.
|ł	|e	|*	|n	|*	|r	|*	|*	|o	|e	|*	|i	|r	|[12][S]\darr	|l	|*	|*	|*	|*	|[13][S]\darr	|*	|o	|.
|a	|n	|*	|t	|*	|z	|*	|*	|r	|d	|*	|o	|k	|k	|o	|[14][S]\darr	|*	|*	|*	|p	|*	|c	|.
|[][,]{ }	|i	|*	|[][,]{ }	|*	|a	|*	|*	|ś	|y	|*	|t	|i	|o	|r	|d	|*	|[15][S]\darr	|*	|o	|[16][S]\darr	|e	|.
|s	|e	|[17][S]\darr	|p	|[18][S]\darr	|n	|*	|*	|w	|t	|[19][S]\darr	|n	|z	|ń	|i	|a	|[20][S]\darr	|n	|*	|l	|s	|f	|.
|y	|*	|o	|o	|s	|k	|*	|*	|i	|o	|ż	|i	|a	|[][,]{ }	|a	|n	|n	|i	|*	|i	|z	|a	|.
|c	|*	|l	|r	|k	|a	|*	|*	|n	|r	|a	|k	|*	|t	|*	|e	|e	|e	|[21][S]\darr	|c	|y	|l	|.
|y	|*	|e	|t	|w	|[][,]{ }	|*	|*	|[][,]{ }	|*	|c	|o	|*	|u	|*	|[][,]{ }	|u	|c	|m	|h	|l	|e	|.
|l	|*	|j	|l	|a	|l	|*	|[22][S]\drarr	|s	|c	|h	|w	|a	|r	|z	|s	|c	|h	|i	|l	|d	|*	|.
|i	|*	|[][,]{ }	|a	|p	|a	|*	|z	|z	|*	|w	|a	|[23][S]\darr	|k	|*	|t	|h	|l	|k	|o	|*	|[24][S]\darr	|.
|j	|*	|j	|n	|l	|t	|*	|r	|a	|*	|y	|t	|r	|m	|[25][S]\darr	|a	|a	|u	|r	|r	|*	|s	|.
|s	|[26][S]\rarr	|a	|d	|i	|a	|f	|o	|r	|a	|*	|e	|o	|e	|w	|t	|t	|b	|o	|e	|*	|m	|.
|k	|*	|d	|z	|w	|j	|*	|s	|y	|*	|*	|*	|b	|ń	|i	|y	|e	|n	|t	|k	|*	|o	|.
|a	|*	|a	|k	|o	|ą	|*	|ł	|*	|*	|*	|*	|a	|s	|n	|s	|l	|o	|u	|*	|[27][S]\darr	|c	|.
|*	|[28][S]\rarr	|l	|i	|ś	|c	|i	|o	|n	|o	|s	|y	|*	|k	|k	|t	|*	|ś	|b	|*	|o	|z	|.
|*	|*	|n	|*	|ć	|a	|*	|g	|*	|*	|*	|*	|*	|i	|l	|y	|[29][S]\darr	|ć	|u	|*	|p	|e	|.
|*	|*	|y	|*	|*	|*	|*	|ł	|*	|*	|*	|*	|*	|*	|e	|c	|f	|*	|l	|*	|t	|k	|.
|*	|*	|*	|*	|*	|[30][S]\rarr	|k	|o	|s	|z	|t	|[][,]{ }	|s	|p	|r	|z	|e	|d	|a	|ż	|y	|*	|.
|*	|[31][S]\rarr	|l	|i	|p	|c	|ó	|w	|k	|a	|*	|*	|*	|*	|*	|n	|l	|*	|*	|*	|m	|*	|.
|*	|*	|*	|[32][S]\drarr	|w	|s	|z	|e	|c	|h	|m	|o	|c	|*	|*	|e	|e	|*	|*	|*	|a	|*	|.
|[33][S]\drarr	|p	|e	|l	|o	|t	|a	|*	|*	|*	|*	|*	|*	|*	|*	|*	|r	|*	|*	|*	|l	|*	|.
|m	|[34][S]\rarr	|k	|a	|t	|a	|p	|u	|l	|t	|o	|w	|a	|n	|i	|e	|*	|*	|*	|*	|i	|*	|.
|e	|*	|[35][S]\rarr	|b	|e	|z	|i	|n	|w	|a	|z	|y	|j	|n	|o	|ś	|ć	|*	|*	|[36][S]\darr	|z	|*	|.
|n	|[37][S]\rarr	|n	|i	|e	|o	|g	|r	|a	|n	|i	|c	|z	|o	|n	|o	|ś	|ć	|*	|k	|a	|*	|.
|s	|[38][S]\rarr	|k	|r	|e	|d	|y	|t	|[][,]{ }	|s	|t	|u	|d	|e	|n	|c	|k	|i	|*	|u	|c	|*	|.
|a	|*	|*	|y	|*	|*	|*	|*	|*	|*	|*	|*	|*	|*	|*	|*	|*	|*	|*	|r	|j	|*	|.
|*	|*	|*	|n	|*	|*	|*	|*	|*	|*	|*	|*	|*	|*	|*	|*	|*	|*	|*	|w	|a	|*	|.
|*	|*	|[39][S]\rarr	|t	|u	|r	|b	|i	|n	|a	|[][,]{ }	|z	|a	|c	|z	|e	|p	|o	|w	|a	|*	|*	|.
|*	|*	|*	|*	|*	|*	|*	|*	|*	|*	|*	|*	|*	|*	|*	|*	|*	|*	|*	|*	|*	|*	|.\end{Puzzle}

\newpage

\begin{PuzzleClues}{\textbf{Poziome}\\}\Clue{2}{}{wartość własna macierzy}
\Clue{3}{}{w rozumieniu formalnym: małżeństwo z więcej niż jedną osobą (w tym samym czasie)}
\Clue{4}{}{człowiek o określonych namiętnościach, usposobieniu}
\Clue{5}{}{proces wytwarzania i uwalniania substancji chemicznych z tkanek lub gruczołów w określonym celu}
\Clue{7}{}{amerykański mechanik i wynalazca (1755-1819); skonstruował wysokoprężny, parowy silnik tłokowy}
\Clue{11}{}{piłkarz z drużyny młodzieżowej mający nie więcej niż 15 lat}
\Clue{22}{}{astronom niemiecki (1873-1916), teoria grawitacji}
\Clue{26}{}{w religii - zwyczaj, rytuał, który jest dopuszczony, ale nie nakazany}
\Clue{28}{}{rodzina amer. nietoperzy, w większości owadożerne, niektóre krwiopijne}
\Clue{30}{}{koszty ponoszone przez przedsiębiorcę dotyczące sprzedawanych wyrobów i obejmujące: koszty opakowania, załadunku i wyładunku, ubezpieczenia, reklamy oraz podatek akcyzowy od wyrobów}
\Clue{31}{}{lipcowa wycieczka za miasto}
\Clue{32}{}{właściwość czegoś, np. instytucji, która jest wszechmocna, ma nieograniczoną moc}
\Clue{33}{}{stara gra baskijska (rodzaj tenisa) polegająca na odbijaniu piłki od muru za pomocą długiej rękawicy z wikliny}
\Clue{34}{}{wyrzucanie z samolotu fotela lub kabiny w celu ratowania ludzi lub ładunku}
\Clue{35}{}{to, że jakieś działanie nie wymaga bezpośredniej ingerencji w strukturę materii, z którą działanie to jest związane}
\Clue{37}{}{brak ograniczeń, obwarowania jakimiś warunkami}
\Clue{38}{}{kredyt preferencyjny przeznaczony dla studentów; jest on wypłacany co miesiąc w czasie trwania roku akademickiego, a jego spłatę można rozpocząć najpóźniej dwa lata po ukończeniu studiów}
\Clue{39}{}{wielostopniowa turbina parowa}\end{PuzzleClues}

\begin{PuzzleClues}{\textbf{Pionowe}\\}\Clue{1}{}{wystąpienie lub nasilenie się niekorzystnych warunków atmosferycznych}
\Clue{2}{}{rodzaj cementu otrzymywany ze zmielenia klinkieru cementowego z gipsem w ilości do 5\%}
\Clue{3}{}{człowiek, który zajmuje się edycją, obróbką i wpasowywaniem grafik}
\Clue{4}{}{Abies nebrodensis - gatunek drzewa z rodziny sosnowatych}
\Clue{5}{}{Yponomeutidae - rodzina małych motyli, obejmująca około tysiąca gatunków; budowane przez nie 'namioty' chronią je m.in. przed deszczem}
\Clue{6}{}{Therocephalia - wymarła linia terapsydów z grupy teriodontów, której rozwój przypada na środkowy i późny perm, a także trias; skamieniałości tych zwierząt były licznie znajdowane w basenie Karoo w RPA, choć znajdowano je także w Rosji, Chinach i Antarktyce}
\Clue{8}{}{żona markiza}
\Clue{9}{}{powszechna nazwa klilokalorii - jednostki energii, którą przeciętnie przyswaja ludzki organizm przy spożyciu produktu żywnościowego}
\Clue{10}{}{gatunek ryby z rodziny karpiowatych (Cyprinidae)}
\Clue{11}{}{morświn Burmeistra, morświn czarny, Phocoena spinipinnis - gatunek walenia z rodziny morświnowatych; zamieszkuje przybrzeżne wody Ameryki Południowej od Paita w Peru, wzdłuż zachodniego wybrzeża po południowy kraniec kontynentu i wzdłuż wschodniego wybrzeża na północ po Santa Catarina w Brazylii}
\Clue{12}{}{rasa koni, pochodząca z Turkmenistanu (Afganistan, Iran, część dawnego ZSRR); stara rasa, która była wykorzystywana do wyścigów co najmniej 2000 lat temu}
\Clue{13}{}{polimer, który powstaje w wyniku polimeryzacji monomerów chlorku}
\Clue{14}{}{wskaźnik wpływający na ocenę jakiejś sytuacji, często niemający nic wspólnego z rzetelnym badaniem naukowym; pojęcie używane często w dziennikarstwie}
\Clue{15}{}{cecha działania, postępowania: to, że coś nie jest powodem do dumy, przynosi raczej wstyd niż chlubę}
\Clue{16}{}{rodzaj tablicy z nazwą, którą umieszcza się nad lokalem usługowym}
\Clue{17}{}{ciekły tłuszcz otrzymywany z nasion i owoców niektórych roślin, a także tkanek ssaków i ryb morskich, wykorzystywany w przemyśle spożywczym i lecznictwie jako produkt spożywczy lub środek farmaceutyczny stosowany doustnie}
\Clue{18}{}{cecha człowieka w jakimś działaniu lub gotowego do jakiegoś działania, pełnego zapału, chęci. A1: radość, cieszenie się na; użyteczność, dobro\} + m}
\Clue{19}{}{Ascidiacea - gromada małych zwierząt morskich zaliczanych do osłonic; w stadium dorosłym są osiadłe, mają kształt worka z dużym otworem u góry i mniejszym z boku, żyją pojedynczo lub w koloniach, jako jedyne zwierzęta (w dorosłym stadium rozwoju) mają ścianę komórkową zbudowaną z celulozy - tak jak u roślin}
\Clue{20}{}{kanton w zach. Szwajcarii, przy granicy z Francją, powierzchnia 797 km2, stolica Weuchatel}
\Clue{21}{}{struktura białkowa będąca podstawowym budelcem cytoszkieletu}
\Clue{22}{}{chimery, Holocephali - podgromada ryb chrzęstnoszkieletowych obejmująca blisko 40 współcześnie żyjących gatunków, zamieszkujących głębokie i chłodne wody morskie, oraz wiele taksonów kopalnych}
\Clue{23}{}{wykwintna, strojna suknia do oficjalnych wystąpień, początkowo męska, później tylko kobieca}
\Clue{24}{}{zabawka dla niemowląt, namiastka brodawki sutkowej matki, wkładana dziecku do ust w celu uspokojenia go; dzieci zazwyczaj uspokajają się wtedy, uruchamia się u nich odruch ssania}
\Clue{25}{}{przezroczysta, żywiczna substancja stosowana w malarstwie i grafice np. do zabezpieczania obrazów przed wpływem warunków atmosferycznych}
\Clue{27}{}{w informatyce - poprawa kodu programu (aplikacji), poprawa jakości korzystania z programu (aplikacji)}
\Clue{29}{}{efekt niedoróbki lub uszkodzenia}
\Clue{32}{}{ozdoba ogrodu, zawiły układ ścieżek wyznaczonych ścianami specjalnie zasadzonego żywopłotu}
\Clue{33}{}{Góra Stołowa}
\Clue{36}{}{prostytutka - kobieta odbywająca stosunki płciowe w celach zarobkowych}\end{PuzzleClues}\newpage\section*{Krzyżówka 4}

\noindent\begin{Puzzle}{19}{26}|*	|*	|*	|*	|*	|*	|*	|*	|*	|[1][S]\drarr	|e	|l	|s	|n	|e	|r	|*	|[2][S]\darr	|*	|*	|.
|*	|[3][S]\darr	|*	|*	|*	|*	|*	|[4][S]\drarr	|m	|o	|r	|w	|a	|*	|[5][S]\darr	|*	|[6][S]\darr	|r	|*	|*	|.
|*	|d	|[7][S]\drarr	|o	|f	|e	|n	|s	|y	|w	|a	|*	|*	|*	|p	|*	|r	|z	|*	|*	|.
|*	|e	|m	|*	|[8][S]\rarr	|u	|c	|z	|u	|c	|i	|e	|*	|*	|r	|*	|y	|e	|*	|*	|.
|[9][S]\drarr	|m	|a	|g	|i	|e	|r	|a	|*	|a	|[10][S]\darr	|*	|*	|[11][S]\darr	|z	|*	|n	|k	|*	|*	|.
|b	|*	|r	|[12][S]\darr	|[13][S]\rarr	|k	|o	|r	|a	|*	|s	|[14][S]\darr	|*	|i	|e	|*	|e	|o	|*	|*	|.
|e	|*	|t	|ś	|*	|*	|*	|y	|[15][S]\darr	|*	|k	|ś	|*	|n	|s	|[16][S]\darr	|k	|t	|*	|*	|.
|z	|*	|o	|w	|[17][S]\darr	|*	|*	|t	|m	|[18][S]\darr	|u	|r	|[19][S]\darr	|i	|t	|l	|[][,]{ }	|k	|*	|*	|.
|n	|[20][S]\drarr	|s	|i	|e	|ć	|[][,]{ }	|k	|o	|m	|p	|u	|t	|e	|r	|o	|w	|a	|*	|*	|.
|ó	|m	|*	|s	|n	|*	|*	|a	|d	|a	|i	|b	|r	|k	|z	|r	|e	|[][,]{ }	|*	|*	|.
|ż	|a	|*	|t	|c	|*	|*	|*	|r	|ś	|e	|a	|i	|c	|e	|a	|w	|ż	|*	|*	|.
|k	|ń	|*	|e	|y	|*	|*	|*	|a	|l	|n	|*	|a	|y	|l	|[][,]{ }	|n	|a	|*	|*	|.
|o	|s	|*	|k	|k	|*	|*	|*	|s	|a	|i	|*	|l	|j	|e	|k	|ę	|b	|*	|*	|.
|w	|k	|*	|[][,]{ }	|l	|*	|*	|*	|z	|n	|e	|*	|*	|n	|n	|a	|t	|i	|*	|*	|.
|c	|i	|*	|s	|o	|*	|*	|*	|e	|k	|*	|*	|*	|o	|i	|r	|r	|a	|*	|*	|.
|e	|*	|*	|e	|p	|*	|*	|*	|k	|a	|*	|*	|*	|ś	|e	|m	|z	|*	|*	|*	|.
|*	|*	|*	|l	|e	|*	|*	|*	|[][,]{ }	|*	|*	|*	|*	|ć	|*	|a	|n	|*	|*	|*	|.
|*	|[21][S]\darr	|*	|w	|d	|*	|*	|[22][S]\rarr	|k	|a	|r	|g	|o	|*	|*	|z	|y	|*	|*	|*	|.
|*	|k	|[23][S]\darr	|o	|y	|*	|*	|[24][S]\rarr	|o	|p	|i	|n	|k	|a	|*	|y	|*	|*	|*	|*	|.
|*	|o	|g	|w	|z	|*	|*	|[25][S]\rarr	|r	|e	|g	|u	|l	|a	|r	|n	|o	|ś	|ć	|*	|.
|*	|m	|r	|y	|m	|*	|*	|*	|y	|*	|[26][S]\rarr	|d	|i	|a	|k	|o	|n	|i	|a	|*	|.
|*	|i	|a	|*	|*	|[27][S]\rarr	|r	|a	|d	|ż	|a	|*	|*	|[28][S]\rarr	|t	|w	|ó	|r	|*	|*	|.
|*	|n	|n	|*	|*	|*	|[29][S]\rarr	|k	|o	|n	|k	|l	|u	|z	|j	|a	|*	|*	|*	|*	|.
|*	|o	|a	|*	|*	|*	|*	|[30][S]\rarr	|n	|e	|t	|b	|a	|l	|l	|*	|*	|*	|*	|*	|.
|[31][S]\rarr	|w	|t	|ó	|r	|n	|i	|k	|*	|*	|*	|*	|*	|*	|*	|*	|*	|*	|*	|*	|.
|*	|e	|*	|*	|*	|*	|*	|*	|*	|*	|*	|*	|*	|*	|*	|*	|*	|*	|*	|*	|.
|*	|*	|*	|*	|*	|*	|*	|*	|*	|*	|*	|*	|*	|*	|*	|*	|*	|*	|*	|*	|.\end{Puzzle}

\newpage

\begin{PuzzleClues}{\textbf{Poziome}\\}\Clue{1}{}{kompozytor i pedagog (1769-1854); nauczyciel Chopina; opery symfoniczne, utwory kameralne, rozprawy o muzyce}
\Clue{4}{}{Morus - rodzaj dużych krzewów lub niewielkich drzew liściastych z rodziny morwowatych}
\Clue{7}{}{zorganizowane atakowanie przeciwnika}
\Clue{8}{}{stan psychiczny, który zawiera w sobie stosunek do świata zewnętrznego, zdarzeń, ludzi itp}
\Clue{9}{}{kolarz, drużynowy zwycięzca Wyścigu Pokoju w 1967 i 68, specjalizował się w jeździe na czas}
\Clue{13}{}{zewnętrzna warstwa tkanki korzeni lub łodyg (pni), głównie gdy są zdrewniałe}
\Clue{20}{}{zbiór połączonych ze sobą komputerów}
\Clue{22}{}{transportowany, najczęściej statkiem morskim, ładunek}
\Clue{24}{}{w górnictwie: tama podsadzkowa stanowiąca ocios wyrobiska chodnikowego}
\Clue{25}{}{cecha czegoś, co zostało skonstruowane wg jakichś zasad lub cecha zbioru, którego elementami rządzą jakieś zasady}
\Clue{26}{}{kaplica w Rzymie, przy której rezydowali diakoni}
\Clue{27}{}{historyczny tytuł lokalnego władcy w Indiach, który sprawował władzę sądowniczą i był zwierzchnikiem wojsk}
\Clue{28}{}{stworzenie, istota żywa, zwłaszcza człowiek albo zwierzę}
\Clue{29}{}{wniosek jednej ze stron biorącej udział w procesie}
\Clue{30}{}{gra zespołowa, uznawana za poprzednika koszykówki; rozgrywana w siedmoosobowych drużynach, które grając przeciwko sobie próbują zdobyć punkty umieszczając piłkę w koszu nieposiadającym tablicy}
\Clue{31}{}{wzmacniacz tranzystorowy nie odwracający fazy napięcia wejściowego}\end{PuzzleClues}

\begin{PuzzleClues}{\textbf{Pionowe}\\}\Clue{1}{}{owca domowa, Ovis aries - popularny gatunek hodowlanego zwierzęcia domowego z rodziny krętorogich, trzymany dla skóry, mięsa, mleka i wełny}
\Clue{2}{}{ropuchorzekotka żywiczna, Trachycephalus resinifictrix - gatunek małego płaza bezogonowego z rodziny rzekotkowatych, występujący w Ameryce Południowej}
\Clue{3}{}{podstawowa jednostka administracyjna i terytorialna w starożytnej Grecji, gmina}
\Clue{4}{}{sikorka uboga, Poecile palustris - gatunek małego ptaka z rodziny sikor (Paridae)}
\Clue{5}{}{uszkodzenie powstałe w wyniku przejścia przez coś na wylot wystrzelonego pocisku}
\Clue{6}{}{obszar wspólnoty UE, na którym zostaje zapewniony swobodny przepływ towarów, usług, osób i kapitału}
\Clue{7}{}{rzeźbiarz rosyjski (1754-1835) reprezentant klasycyzmu}
\Clue{9}{}{Eleutherozoa - jeden z podtypów, w jakich grupowane są żyjące współcześnie szkałupnie}
\Clue{10}{}{stan koncentracji, proces psychiczny polegający na ześrodkowaniu uwagi}
\Clue{11}{}{to, w jakim stopniu materiał podatny jest na iniekcję; zdolność wtłaczanej substancji do przepływu w poddanej iniekcji strukturze}
\Clue{12}{}{Hydrolaetare schmidti - gatunek płaza bezogonowego z rodziny świstkowatych}
\Clue{14}{}{złącze, elemnt połączenia śrubowego}
\Clue{15}{}{Polyommatus coridon - gatunek motyla dziennego z rodziny modraszkowatych, z podrodziny modraszków; jest to gatunek europejski, w Polsce rozpowszechniony na niżu}
\Clue{16}{}{Eos bornea - gatunek ptaka z rodziny papugowatych (Psittacidae), z podrodziny papug wschodnich (Psittaculinae)}
\Clue{17}{}{erudycja}
\Clue{18}{}{bułka maślana}
\Clue{19}{}{pierwszy statek wodny zbudowany z blach w 1787 r. przez J. Wilkinsona}
\Clue{20}{}{język z grupy goidelskiej języków celtyckich, który używany jest na Wyspie Man (dependencji Korony brytyjskiej) położonej na Morzu Irlandzkim; pochodzi od języka staroiryjskiego, a zwłaszcza od jego dialektów ulsterskiego i galloway}
\Clue{21}{}{podatek od komina, płacony przez właścicieli budynków z kominem}
\Clue{23}{}{pocisk ręczny lub artyleryjski wypełniony materiałem wybuchowym}\end{PuzzleClues}\newpage\section*{Krzyżówka 5}

\noindent\begin{Puzzle}{18}{26}|*	|*	|[1][S]\drarr	|m	|i	|s	|s	|o	|u	|l	|a	|*	|*	|*	|*	|*	|[2][S]\darr	|*	|*	|.
|*	|[3][S]\rarr	|s	|p	|ó	|ł	|k	|a	|[][S]-	|c	|ó	|r	|k	|a	|*	|[4][S]\darr	|k	|*	|*	|.
|[5][S]\rarr	|b	|a	|r	|b	|i	|e	|r	|i	|*	|*	|*	|*	|[6][S]\drarr	|b	|a	|r	|d	|*	|.
|[7][S]\rarr	|k	|l	|o	|c	|e	|k	|*	|*	|*	|[8][S]\drarr	|r	|y	|c	|y	|n	|a	|*	|*	|.
|[9][S]\rarr	|m	|a	|j	|d	|a	|n	|i	|a	|r	|z	|*	|*	|a	|*	|y	|t	|[10][S]\darr	|*	|.
|*	|*	|m	|[11][S]\drarr	|c	|h	|e	|t	|u	|m	|a	|l	|*	|v	|*	|ż	|k	|p	|*	|.
|[12][S]\rarr	|r	|a	|k	|i	|e	|t	|a	|*	|[13][S]\drarr	|s	|i	|ł	|a	|*	|*	|a	|i	|*	|.
|*	|*	|n	|a	|[14][S]\drarr	|k	|s	|i	|ę	|ż	|y	|k	|*	|l	|*	|*	|*	|t	|*	|.
|*	|[15][S]\darr	|d	|c	|g	|[16][S]\darr	|*	|*	|*	|ó	|p	|*	|[17][S]\drarr	|l	|u	|r	|*	|e	|*	|.
|[18][S]\drarr	|p	|r	|z	|e	|d	|z	|i	|a	|ł	|*	|[19][S]\rarr	|p	|i	|t	|a	|*	|a	|*	|.
|w	|a	|a	|k	|n	|u	|*	|*	|*	|w	|[20][S]\drarr	|b	|e	|n	|d	|e	|l	|*	|*	|.
|k	|s	|[][,]{ }	|a	|e	|t	|*	|*	|*	|[][,]{ }	|d	|[21][S]\drarr	|l	|i	|g	|a	|*	|*	|*	|.
|r	|t	|w	|[][,]{ }	|r	|k	|*	|*	|*	|z	|a	|k	|o	|*	|[22][S]\darr	|*	|*	|*	|*	|.
|ę	|e	|a	|f	|a	|a	|*	|*	|[23][S]\darr	|ł	|g	|a	|b	|*	|z	|*	|*	|*	|*	|.
|t	|r	|l	|i	|ł	|*	|[24][S]\darr	|*	|o	|o	|a	|p	|e	|[25][S]\darr	|ł	|*	|[26][S]\darr	|*	|*	|.
|*	|s	|t	|l	|*	|[27][S]\drarr	|d	|ż	|e	|t	|*	|s	|n	|c	|o	|*	|k	|*	|*	|.
|*	|t	|l	|i	|[28][S]\drarr	|k	|a	|r	|b	|o	|n	|a	|t	|y	|t	|*	|o	|*	|*	|.
|*	|w	|a	|p	|v	|o	|r	|*	|e	|g	|*	|*	|o	|m	|o	|*	|c	|*	|*	|.
|*	|o	|*	|i	|e	|ń	|ń	|*	|n	|ł	|*	|*	|s	|a	|w	|*	|z	|*	|*	|.
|*	|*	|*	|ń	|l	|[][,]{ }	|*	|*	|*	|o	|*	|*	|*	|*	|ł	|*	|o	|*	|*	|.
|*	|*	|*	|s	|d	|ś	|[29][S]\rarr	|l	|e	|w	|a	|c	|t	|w	|o	|*	|w	|*	|*	|.
|*	|*	|*	|k	|e	|l	|*	|*	|*	|y	|[30][S]\rarr	|k	|a	|r	|s	|i	|n	|*	|*	|.
|[31][S]\rarr	|a	|g	|a	|*	|ą	|*	|*	|*	|*	|*	|*	|*	|*	|*	|*	|i	|*	|*	|.
|*	|*	|*	|*	|[32][S]\rarr	|s	|z	|c	|z	|u	|r	|*	|*	|[33][S]\rarr	|h	|e	|c	|a	|*	|.
|[34][S]\rarr	|w	|i	|e	|l	|k	|i	|[][,]{ }	|s	|z	|l	|e	|m	|*	|*	|*	|e	|*	|*	|.
|[35][S]\rarr	|c	|o	|b	|[][,]{ }	|i	|r	|l	|a	|n	|d	|z	|k	|i	|*	|*	|*	|*	|*	|.
|*	|[36][S]\rarr	|m	|u	|r	|*	|*	|*	|*	|*	|*	|*	|*	|*	|*	|*	|*	|*	|*	|.\end{Puzzle}

\newpage

\begin{PuzzleClues}{\textbf{Poziome}\\}\Clue{1}{}{miasto w USA (Montana), ośrodek turystyczny i handlowy, węzeł komunikacyjny}
\Clue{3}{}{spółka handlowa kontrolowana przez jednostkę dominującą}
\Clue{5}{}{światowej sławy śpiewaczka włoska ur. w 1920 r., (mezzosopran), solistka La Scali, Covent Garden, Metropolitan Opera}
\Clue{6}{}{u ludów celtyckich: pieśniarz, poeta dworski opiewający bohaterskie czyny}
\Clue{7}{}{element hamulca ciernego, który podczas hamowania naciska na tarczę hamulcową}
\Clue{8}{}{potoczna nazwa oleju rycynowego, stosowanego w lecznictwie jako środek przeczyszczający}
\Clue{9}{}{rozwoziciel gazet, dostarczający je do sprzedawców ulicznych}
\Clue{11}{}{miasto i port w Meksyku nad Morzem Kaspijskim, stolica stanu Quintana Roo; przemysł papierniczy}
\Clue{12}{}{przyrząd do odbijania piłki w tenisie lub squashu}
\Clue{13}{}{wektorowa wielkość fizyczna będąca miarą oddziaływań fizycznych między ciałami}
\Clue{14}{}{żartobliwie o osobie duchownej}
\Clue{17}{}{prehistoryczny instrument muzyczny o długiej metalowej rurce zakończonej czarą  głosową}
\Clue{18}{}{podstawa jakiegoś podziału lub jego efekt}
\Clue{19}{}{okrągły i płaski pszenny placek, popularny w krajach Bliskiego Wschodu i Maghrebu, jedzony najczęściej z pikantnymi sosami, używany też jako kieszonka, do której można nałożyć składników (mięsa, warzyw, sera itp.) i w ten sposób podać jakieś jedzenie}
\Clue{20}{}{stan w płd-zach Nigerii, główne miasto Benin}
\Clue{21}{}{próba monety, czyli ilość zawartego w niej kruszcu (złota lub srebra)}
\Clue{27}{}{skolimowany strumień plazmowej materii wyrzucany z relatywistycznymi prędkościami z biegunów jądra galaktyki lub gwiazdy}
\Clue{28}{}{jest to skała magmowa, najczęściej głębinowa, rzadziej wylewna lub ultrazasadowa; ponad 50\% składu mineralnego to węglany, głównie dolomit i kalcyt, choć w składzie może byc jeszcze około dwieście innych minerałów}
\Clue{29}{}{radykalne lewicowe poglądy}
\Clue{30}{}{duża wieś kaszubska w Polsce położona w województwie pomorskim, w powiecie kościerskim, w gminie Karsin, na południowych obrzeżach Wdzydzkiego Parku Krajobrazowego, na Ziemi Zaborskiej}
\Clue{31}{}{kururu, olbrzymia ropucha z gruczołami jadowymi w tyle głowy, Ameryka Południowa Środkowa}
\Clue{32}{}{człowiek zgniły moralnie, nikczemny, oślizgły, wstrętny, odpychający, tchórzliwy}
\Clue{33}{}{przedstawienie cyrkowe}
\Clue{34}{}{cykl czterech największych i najbardziej prestiżowych turniejów sportowych w ramach jednej dyscypliny (np. w skokach narciarskich, golfie) w roku kalendarzowym}
\Clue{35}{}{tinker, gypsy vanner - rasa koni ogólnoużytkowych, często o srokatym umaszczeniu; niezawodne konie do użytkowania zaprzęgowego i wierzchowego}
\Clue{36}{}{grupa ludzi lub obiektów, która stoi w zwartym szyku i (najczęściej) odgradza coś od czegoś}\end{PuzzleClues}

\begin{PuzzleClues}{\textbf{Pionowe}\\}\Clue{1}{}{traszka Waltla, żebrowiec Waltla, Pleurodeles waltl - gatunek płaza ogoniastego z rodziny salamandrowatych, największy europejski płaz, bardzo popularny wśród akwarystów ze względu na niewygórowane wymagania; w naturze występuje na Półwyspie Iberyjskim i w Maroku}
\Clue{2}{}{wzór z przecinających się regularnie (najczęściej pod kątem prostym) linii}
\Clue{4}{}{BADIAN roślina zielna z baldaszkowatych, nasiona dostarczają olejku anyżowego;}
\Clue{6}{}{malarz angielski (1776-1837) wybitny pejzażysta, prekursor impresjonizmu}
\Clue{8}{}{czynność, która polega na tym, że coś się sypie, aż się zasypie}
\Clue{10}{}{miasto i port w Szwecji w pobliżu ujścia rzeki Pite do Zatoki Botnickiej, przemysł drzewny i celulozowo-papierniczy}
\Clue{11}{}{Anas luzonica - gatunek ptaka z rodziny kaczkowatych (Anatidae)}
\Clue{13}{}{żółw żółtogłowy, Indotestudo elongata - gatunek gada z rodziny żółwi lądowych; prawdopodobnie pochodzi z Birmy, ale został rozpowszechniony przez człowieka na innych obszarach i teraz występuje w Nepalu, Bangladeszu, Indiach, Birmie, Laosie, Tajlandii, Kambodży, Wietnamie, Malezji i południowych Chinach}
\Clue{14}{}{najwyższy przełożony w niektórych zakonach katolickich; wybierany przez kapitułę i sprawujący władzę nad wszystkimi jednostkami administracyjnymi danego zakonu na świecie}
\Clue{15}{}{rodzaj hodowli zwierząt stadnych}
\Clue{16}{}{używana niegdyś nazwa piszczałki, fujarki}
\Clue{17}{}{bentos związany z dnem mulistym}
\Clue{18}{}{rodzaj śruby o główce mającej wyżłobienie ułatwiające wkręcanie jej}
\Clue{20}{}{broń sieczna rodzaj sztyletu}
\Clue{21}{}{kabza - sakiewka, woreczek na pieniądze}
\Clue{22}{}{Asphodelus - rodzaj rośliny z rodziny złotogłowowych}
\Clue{23}{}{ebenista francuski (1710-1763); nadworny artysta Ludwika XV}
\Clue{24}{}{zwarta okrywa m.in. łąk i pastwisk, składająca się głównie z trawy i roślin motylkowatych lub też z samej trawy i tworząca trawnik; poprzez gęsty system korzeni roślin silnie wiąże się z wierzchnią warstwą gleby}
\Clue{25}{}{profil gzymsu w kształcie litery S - SIMA}
\Clue{26}{}{Nomadinae - podrodzina owadów z rodziny pszczołowatych w podrzędzie trzonkówek}
\Clue{27}{}{rasa koni powstała na bazie koni z terenów Dolnego i Górnego Śląska, odnosząca duże sukcesy jako konie zaprzęgowe w dyscyplinie powożenia; centrum hodowlanym jest Stado Ogierów Książ i działająca przy nim stadnina koni}
\Clue{28}{}{ginekolog holenderski (1873-1937); dyrektor kliniki w Zurychu}\end{PuzzleClues}\newpage\section*{Krzyżówka 6}

\noindent\begin{Puzzle}{18}{33}|*	|*	|*	|*	|*	|[1][S]\darr	|*	|*	|*	|*	|*	|*	|*	|*	|*	|*	|*	|*	|*	|.
|*	|[2][S]\drarr	|m	|a	|*	|p	|[3][S]\drarr	|f	|i	|t	|*	|*	|[4][S]\drarr	|s	|c	|e	|n	|a	|*	|.
|*	|b	|[5][S]\drarr	|d	|i	|a	|b	|e	|ł	|[][,]{ }	|w	|c	|i	|e	|l	|o	|n	|y	|*	|.
|*	|a	|w	|*	|*	|r	|y	|*	|[6][S]\darr	|*	|[7][S]\rarr	|i	|n	|l	|e	|t	|*	|*	|*	|.
|[8][S]\drarr	|c	|y	|t	|w	|a	|r	|*	|s	|[9][S]\darr	|*	|*	|g	|[10][S]\darr	|[11][S]\darr	|*	|*	|[12][S]\darr	|*	|.
|b	|z	|z	|*	|*	|*	|d	|[13][S]\darr	|t	|j	|*	|*	|u	|s	|x	|*	|[14][S]\darr	|d	|*	|.
|r	|k	|n	|*	|*	|*	|*	|l	|y	|e	|[15][S]\darr	|[16][S]\darr	|s	|p	|c	|[17][S]\darr	|b	|e	|*	|.
|y	|i	|a	|*	|*	|*	|*	|e	|l	|d	|r	|k	|z	|r	|d	|p	|a	|s	|*	|.
|g	|*	|n	|[18][S]\darr	|[19][S]\drarr	|l	|e	|n	|i	|n	|ó	|w	|k	|a	|*	|i	|r	|t	|*	|.
|a	|[20][S]\darr	|i	|k	|s	|*	|*	|i	|z	|o	|d	|o	|a	|w	|[21][S]\drarr	|l	|k	|r	|*	|.
|d	|m	|e	|o	|m	|*	|*	|k	|a	|l	|*	|k	|*	|n	|w	|o	|a	|u	|*	|.
|i	|o	|[][,]{ }	|ń	|o	|[22][S]\rarr	|k	|o	|c	|i	|c	|a	|*	|o	|e	|t	|*	|k	|*	|.
|e	|r	|w	|[][,]{ }	|c	|*	|*	|w	|j	|t	|*	|c	|[23][S]\darr	|ś	|ł	|*	|*	|c	|*	|.
|r	|e	|i	|k	|z	|*	|*	|a	|a	|a	|*	|z	|r	|ć	|n	|*	|*	|y	|*	|.
|*	|n	|a	|l	|e	|*	|*	|t	|*	|[][,]{ }	|*	|*	|a	|*	|a	|*	|*	|j	|*	|.
|[24][S]\drarr	|a	|r	|a	|k	|*	|*	|e	|*	|p	|*	|*	|m	|[25][S]\darr	|[][,]{ }	|*	|*	|n	|*	|.
|i	|[][,]{ }	|y	|d	|*	|*	|*	|*	|*	|ł	|*	|*	|s	|d	|m	|[26][S]\darr	|*	|o	|*	|.
|n	|s	|*	|r	|*	|*	|[27][S]\darr	|*	|*	|a	|*	|*	|e	|o	|i	|k	|*	|ś	|*	|.
|s	|p	|*	|u	|*	|*	|m	|[28][S]\drarr	|s	|t	|e	|r	|y	|l	|n	|o	|ś	|ć	|*	|.
|t	|i	|[29][S]\drarr	|b	|a	|w	|i	|a	|l	|n	|i	|a	|*	|l	|e	|l	|*	|*	|*	|.
|a	|ę	|o	|s	|[30][S]\drarr	|p	|r	|a	|w	|o	|[][,]{ }	|m	|o	|o	|r	|e	|[][S]'	|a	|*	|.
|l	|t	|d	|k	|u	|*	|a	|k	|*	|ś	|*	|*	|*	|*	|a	|j	|*	|*	|*	|.
|a	|r	|w	|i	|s	|[31][S]\darr	|ż	|*	|*	|ć	|*	|*	|*	|*	|l	|k	|*	|*	|*	|.
|t	|z	|a	|*	|d	|w	|*	|*	|*	|[][,]{ }	|*	|*	|*	|*	|n	|a	|*	|*	|*	|.
|o	|o	|c	|*	|*	|o	|[32][S]\rarr	|z	|ł	|o	|t	|ó	|w	|k	|a	|*	|*	|*	|*	|.
|r	|n	|h	|*	|*	|r	|*	|*	|[33][S]\rarr	|b	|a	|l	|o	|n	|*	|*	|*	|*	|*	|.
|s	|a	|*	|[34][S]\drarr	|m	|e	|n	|d	|e	|s	|*	|*	|*	|*	|*	|*	|*	|*	|*	|.
|t	|*	|*	|p	|*	|k	|*	|[35][S]\rarr	|s	|z	|a	|b	|a	|ś	|n	|i	|k	|*	|*	|.
|w	|*	|*	|o	|*	|*	|*	|[36][S]\rarr	|g	|a	|l	|i	|l	|e	|u	|s	|z	|*	|*	|.
|o	|*	|*	|l	|*	|*	|[37][S]\rarr	|h	|e	|r	|b	|a	|t	|n	|i	|k	|*	|*	|*	|.
|*	|*	|*	|k	|*	|[38][S]\rarr	|s	|a	|m	|o	|c	|h	|w	|a	|l	|c	|a	|*	|*	|.
|*	|[39][S]\rarr	|i	|a	|t	|a	|*	|[40][S]\rarr	|a	|w	|a	|n	|g	|a	|r	|d	|a	|*	|*	|.
|*	|*	|*	|*	|*	|*	|*	|[41][S]\rarr	|k	|a	|p	|i	|t	|a	|n	|a	|t	|*	|*	|.
|[42][S]\rarr	|c	|a	|m	|e	|r	|a	|t	|a	|*	|*	|*	|*	|*	|*	|*	|*	|*	|*	|.\end{Puzzle}

\newpage

\begin{PuzzleClues}{\textbf{Poziome}\\}\Clue{2}{}{jednostka prędkości odnosząca się do obiektów poruszających się w płynie lub poruszających się płynów}
\Clue{3}{}{okrągły sworzeń służący do prac przy linach}
\Clue{4}{}{działalność teatralna}
\Clue{5}{}{osoba, która jest bardzo ruchliwa i sprawia kłopoty}
\Clue{7}{}{bawełniana tkanina pościelowa, o gęstym splocie używana na wsypy}
\Clue{8}{}{bylica cytwarowa, rupnik, bylica glistnik, Artemisia cina - gatunek rośliny z rodziny astrowatych (złożonych); pochodzi z półpustynnych stepów Azji Środkowej (Turkiestan), Kazachstanu oraz z Chin; w Europie znany od czasów starożytnych, sprowadzony przez Rzymian}
\Clue{19}{}{czapka z daszkiem, podobna do noszonej przez Lenina}
\Clue{21}{}{kod ISO 4217 rupii lankijskiej}
\Clue{22}{}{kobieta-uwodzicielka}
\Clue{24}{}{mocny napój spirytusowy sporządzany w Azji płd. z ryżu lub soku palmowego}
\Clue{28}{}{prostota i surowość czegoś, np. architektury; użycie przenośne}
\Clue{29}{}{SALON}
\Clue{30}{}{prawo empiryczne, wynikające z obserwacji, że ekonomicznie optymalna liczba tranzystorów w układzie scalonym zwiększa się w kolejnych latach zgodnie z trendem wykładniczym}
\Clue{32}{}{potoczna nazwa złotego (używana, chociaż miano to powinno się dziś stosować wyłącznie do jednozłotowej monety)}
\Clue{33}{}{duże, kuliste naczynie szklane z wąską szyjką}
\Clue{34}{}{pisarz francuski, parnasista (1841-1910), poezje, powieści, utwory dramatyczne; „Życie i śmierć klowna”}
\Clue{35}{}{krak. piekarnik pieca kuchennego opalanego drewnem lub węglem}
\Clue{36}{}{włoski astronom, fizyk i filozof (1564-1642), odkrył góry na Księżycu, cztery satelity Jowisza, fazy Wenus, plamy słoneczne}
\Clue{37}{}{przenośnie: alkoholik (osoba, która wchłania alkohol, nasiąka tak łatwo jak herbatnik)}
\Clue{38}{}{osoba, która przechwala się swoimi czynami czy zaletami, często nieprawdziwie}
\Clue{39}{}{Zrzeszenie Międzynarodowego Transportu Lotniczego}
\Clue{40}{}{środowisko artystów, grupa osób, których twórczość jest nowatorska, nietradycyjna i któe przez to wyznaczają nowy kierunek artystycznych działań}
\Clue{41}{}{organ administracyjny wchodzący w skład urzędu morskiego działający w większych portach}
\Clue{42}{}{grupa florenckich muzyków z końca XVI w (PERI, CACCINI)}\end{PuzzleClues}

\begin{PuzzleClues}{\textbf{Pionowe}\\}\Clue{1}{}{waluta Albanii przed 1926 r}
\Clue{2}{}{rodzaj zarostu, który porasta boki twarzy}
\Clue{3}{}{amerykański lotnik i badacz polarny; pierwszy lot nad biegunem}
\Clue{4}{}{mieszkanka Inguszetii lub Osetii Północnej w Federacji Rosyjskiej, kobieta pochodzenia inguszeckiego}
\Clue{5}{}{wyznanie wiary w czasie mszy świętej}
\Clue{6}{}{stylizowanie, przekształcanie formy czegoś, w celu nadania walorów dekoracyjnych oraz po to, by przywołać skojarzenie z czymś innym - z inną tematyką, obszarem zainteresowań, znanym dziełem itp}
\Clue{8}{}{dawny stopień wojskowy, pośredni między pułkownikiem a generałem}
\Clue{9}{}{świadczenie przysługujące rolnikowi, który jest posiadaczem gruntów rolnych, wchodzących w skład gospodarstwa rolnego, o łącznej powierzchni nie mniejszej niż 1 ha}
\Clue{10}{}{cecha objawiająca się w działaniu, gdy coś robione jest sprawnie, umiejętnie}
\Clue{11}{}{kod ISO 4217 dolara wschodniokaraibskiego}
\Clue{12}{}{to, że czyjeś zachowanie lub jego rezultat negatywnie wpływają na innych, powodują rozpad więzi międzyludzkich, demoralizację, zniszczenie}
\Clue{13}{}{Zodariidae - rodzina pająków z podrzędu Opisthothelae}
\Clue{14}{}{rodzaj statku o płaskim dnie, służącego do transportu ładunków w żegludze śródlądowej, najczęściej pozbawiony własnego napędu}
\Clue{15}{}{całe potomstwo jednego samca}
\Clue{16}{}{wodny ptak z siewek o cienkim, wygiętym dziobie}
\Clue{17}{}{grindwal krótkopłetwy, Globicephala macrorhynchus - gatunek walenia zaliczany do rodziny delfinowatych, mimo iż swoim zachowaniem bardziej przypomina wieloryba; żyje w stadach liczących od 10 do 30 (lub więcej) okazów}
\Clue{18}{}{najstarsza czeska rasa koni domowych, wyhodowana w stadninie Kladruby nad Labem, utworzonej w 1572 roku przez cesarza Maksymiliana II na bazie hodowli koni andaluzyjskich; nadaje się do zaprzęgów i do jazdy wierzchem, jednak obecnie rasa traktowana jest jako zabytek narodowy}
\Clue{19}{}{okrągły narząd gębowy minoga}
\Clue{20}{}{morena czołowa uformowana wskutek spiętrzenia osadów na przedpolu nasuwającego się lodowca (lądolodu)}
\Clue{21}{}{materiał izolacyjny pochodzenia mineralnego, używany w budownictwie do izolacji termicznych i akustycznych ścian zewnętrznych i wewnętrznych, stropów, podłóg, dachów, stropodachów oraz ciągów instalacyjnych, a także jako rdzeń izolacyjno-konstrukcyjny budowlanych płyt warstwowych}
\Clue{23}{}{angielski logik i matematyk (1903-30); sformułował uproszczoną teorię typów logicznych}
\Clue{24}{}{instalowanie urządzeń technicznych, przewodów itp}
\Clue{25}{}{paleontolog belgijski (1857-1931); sformułował prawo nieodwracalności ewolucji}
\Clue{26}{}{przedmiot służący do zabawy w postaci pociągu}
\Clue{27}{}{zjawisko najczęściej występujące na dużych otwartych przestrzeniach, np. na pustyni lub otwartym morzu, polegające na specyficznym załamywaniu i odbijaniu się światła}
\Clue{28}{}{dawny żaglowiec pływający po holenderskich kanałach}
\Clue{29}{}{pomieszczenie dla żołnierzy pełniących wartę główną w garnizonie}
\Clue{30}{}{kod ISO 4217 dolara amerykańskiego}
\Clue{31}{}{szerokie pojęcie lub jakaś grupa klasyfikacyjna, do której można wliczyć bardzo wiele różnych rzeczy, spraw}
\Clue{34}{}{pochodzący z Czech, znany w wielu krajach, taniec ludowy, w metrum 2/4, w szybkim tempie}\end{PuzzleClues}\newpage\section*{Krzyżówka 7}

\noindent\begin{Puzzle}{25}{28}|*	|[1][S]\darr	|[2][S]\darr	|[3][S]\darr	|*	|*	|*	|*	|*	|*	|*	|*	|*	|*	|*	|[4][S]\darr	|*	|*	|*	|*	|*	|*	|[5][S]\darr	|*	|*	|*	|.
|*	|f	|n	|s	|*	|*	|*	|[6][S]\rarr	|r	|o	|z	|r	|z	|u	|t	|k	|o	|w	|a	|t	|e	|*	|m	|*	|*	|*	|.
|*	|r	|i	|o	|[7][S]\darr	|[8][S]\drarr	|r	|i	|d	|b	|o	|k	|[][,]{ }	|s	|z	|a	|r	|y	|*	|[9][S]\drarr	|c	|h	|i	|n	|a	|*	|.
|*	|a	|e	|r	|m	|b	|*	|*	|*	|*	|[10][S]\darr	|*	|*	|*	|*	|b	|*	|[11][S]\darr	|[12][S]\darr	|b	|[13][S]\rarr	|m	|e	|v	|*	|*	|.
|[14][S]\drarr	|c	|z	|t	|e	|r	|n	|a	|s	|t	|k	|a	|*	|*	|*	|o	|*	|c	|g	|o	|[15][S]\darr	|*	|s	|*	|[16][S]\darr	|*	|.
|l	|c	|g	|y	|c	|u	|*	|*	|[17][S]\rarr	|g	|o	|g	|l	|e	|*	|t	|*	|z	|u	|l	|s	|*	|z	|*	|d	|*	|.
|i	|e	|o	|m	|h	|z	|*	|*	|*	|*	|m	|[18][S]\drarr	|p	|e	|r	|y	|p	|a	|t	|e	|t	|y	|k	|*	|r	|*	|.
|s	|s	|d	|e	|a	|d	|*	|*	|*	|*	|i	|f	|*	|*	|*	|n	|*	|m	|t	|s	|o	|*	|a	|*	|z	|*	|.
|z	|c	|n	|n	|n	|a	|*	|[19][S]\rarr	|r	|e	|n	|i	|f	|e	|r	|*	|*	|a	|y	|ł	|v	|*	|l	|*	|e	|*	|.
|a	|o	|o	|t	|i	|*	|*	|*	|*	|*	|*	|z	|*	|*	|*	|*	|*	|r	|s	|a	|c	|[20][S]\darr	|n	|*	|w	|*	|.
|j	|*	|ś	|[][,]{ }	|k	|*	|[21][S]\rarr	|m	|e	|d	|g	|y	|e	|s	|*	|*	|*	|k	|o	|w	|k	|o	|i	|*	|o	|*	|.
|[][,]{ }	|*	|ć	|d	|a	|*	|*	|[22][S]\rarr	|e	|t	|a	|k	|r	|y	|d	|y	|n	|a	|*	|i	|h	|s	|c	|*	|[][,]{ }	|*	|.
|r	|*	|*	|r	|[][,]{ }	|*	|*	|*	|*	|[23][S]\drarr	|h	|a	|n	|c	|o	|c	|k	|*	|*	|a	|a	|a	|t	|*	|ś	|*	|.
|u	|*	|[24][S]\darr	|e	|k	|*	|[25][S]\darr	|[26][S]\rarr	|h	|o	|t	|[][,]{ }	|d	|o	|g	|*	|*	|*	|*	|n	|u	|d	|w	|*	|w	|*	|.
|m	|*	|b	|w	|o	|*	|k	|*	|*	|b	|*	|m	|[27][S]\darr	|*	|*	|*	|*	|[28][S]\darr	|*	|k	|s	|n	|o	|[29][S]\darr	|i	|*	|.
|i	|*	|e	|n	|n	|[30][S]\darr	|o	|[31][S]\darr	|*	|e	|[32][S]\darr	|e	|p	|*	|*	|*	|*	|h	|*	|a	|e	|i	|*	|c	|a	|*	|.
|e	|*	|j	|a	|s	|j	|n	|h	|*	|z	|k	|d	|o	|*	|*	|*	|*	|i	|*	|*	|n	|c	|*	|z	|t	|*	|.
|n	|*	|r	|*	|t	|e	|t	|u	|*	|n	|o	|y	|b	|*	|[33][S]\rarr	|w	|i	|s	|ł	|a	|*	|z	|*	|a	|a	|*	|.
|i	|*	|u	|*	|r	|d	|r	|y	|[34][S]\rarr	|a	|r	|c	|u	|s	|[][,]{ }	|c	|o	|t	|a	|n	|g	|e	|n	|s	|*	|*	|.
|o	|*	|t	|*	|u	|n	|a	|g	|*	|n	|s	|z	|d	|*	|[35][S]\rarr	|c	|z	|o	|r	|t	|e	|k	|*	|[][,]{ }	|*	|*	|.
|w	|[36][S]\drarr	|k	|ą	|k	|o	|l	|e	|w	|i	|a	|n	|k	|a	|*	|*	|*	|r	|[37][S]\darr	|[38][S]\darr	|*	|[][,]{ }	|*	|z	|*	|*	|.
|a	|t	|a	|*	|c	|d	|t	|n	|*	|e	|r	|a	|a	|*	|*	|*	|*	|i	|t	|t	|*	|g	|*	|i	|*	|*	|.
|t	|r	|*	|*	|j	|n	|*	|s	|*	|*	|z	|*	|*	|*	|*	|*	|*	|a	|ę	|r	|*	|o	|*	|m	|*	|*	|.
|y	|a	|*	|*	|i	|i	|*	|*	|*	|*	|*	|*	|*	|*	|*	|*	|*	|*	|c	|z	|*	|ł	|*	|o	|*	|*	|.
|*	|p	|*	|*	|*	|ó	|*	|[39][S]\rarr	|r	|e	|a	|k	|t	|o	|r	|[][,]{ }	|g	|a	|z	|o	|w	|y	|*	|w	|*	|*	|.
|*	|*	|[40][S]\rarr	|s	|z	|w	|e	|d	|z	|k	|o	|ś	|ć	|*	|[41][S]\rarr	|ś	|c	|i	|a	|n	|a	|*	|*	|y	|*	|*	|.
|[42][S]\rarr	|j	|ę	|z	|y	|k	|[][,]{ }	|k	|e	|c	|z	|u	|a	|*	|*	|*	|*	|*	|*	|e	|*	|*	|*	|*	|*	|*	|.
|[43][S]\rarr	|s	|o	|c	|j	|a	|l	|i	|s	|t	|a	|*	|*	|*	|[44][S]\rarr	|n	|o	|w	|i	|k	|o	|w	|*	|*	|*	|*	|.
|[45][S]\rarr	|p	|o	|l	|e	|*	|*	|*	|*	|*	|*	|[46][S]\rarr	|c	|a	|b	|e	|z	|o	|n	|*	|*	|*	|*	|*	|*	|*	|.\end{Puzzle}

\newpage

\begin{PuzzleClues}{\textbf{Poziome}\\}\Clue{6}{}{Woodsiaceae - rodzina paproci z rzędu paprotkowców; gatunki paproci należące do niej bywają zaliczane do wietlicowatych; niektóre klasyfikacje nie wyróżniają żadnej z tych rodzin - rodzaj Woodsia jest przez nie włączany do rodziny Dryopteridaceae (nerecznicowate)}
\Clue{8}{}{antylopa sarnia, Pelea capreolus - antylopa z rodziny krętorogich, jedyny przedstawiciel rodzaju Pelea}
\Clue{9}{}{instrument z grupy idiofonów o nieokreślonej wysokości dźwięku, wykorzystywany w zestawie perkusyjnym}
\Clue{13}{}{megaelektronowolt; wielokrotność elektronowolta; 1 MeV = 1 000 000 eV = 10 do potęgi szóstej eV}
\Clue{14}{}{liczba 14, numer 14}
\Clue{17}{}{przyrząd optyczny, przypominający wyglądem okulary, którego zadaniem jest ochrona oczu i części twarzy przed niekorzystnymi czynnikami atmosferycznymi, zwłaszcza opadami i wiatrem, wykorzystywany przede wszystkim w narciarstwie oraz w pracy z bardzo silnym światłem, opiłkami i iskrami}
\Clue{18}{}{każdy arystotelik - zwolennik arystotelizmu w filozofii; nazwa pochodzi od trybu prowadzenia dysput i wykładów w Lykeionie, od spacerów, podczas których Arystoteles przekazywał wiedzę swoim uczniom}
\Clue{19}{}{REN}
\Clue{21}{}{opieka nad nauką bądź sztuką sprawowana przez państwo, instytucję, jednostkę}
\Clue{22}{}{żółty barwnik akrydynowy mający działanie przeciwbakteryjne}
\Clue{23}{}{amerykański pianista i kompozytor jazzowy ur. w 1940 r.; pionier jazzu fusion}
\Clue{26}{}{rodzaj kanapki z gorącą parówką lub kiełbaską z zimnym sosem (majonez, ketchup)}
\Clue{33}{}{najdłuższa rzeka w Polsce}
\Clue{34}{}{arc ctg - funkcja odwrotna do funkcji cotangens rozpatrywanej na przedziale (0,?); oznacza to, że jeśli y = arc ctg(x), to x = ctg(y)}
\Clue{35}{}{pięściarz, wicemistrz Europy z 1939 r., wrzesień, Oświęcimia}
\Clue{36}{}{mieszkanka Kąkolewa - wsi w Wielkopolsce, w powiecie leszczyńskim}
\Clue{39}{}{reaktor energetyczny, który jest chłodzony gazem}
\Clue{40}{}{zespół cech czegoś lub kogoś takiego jak w Szwecji, także: stereotypowe cechy uznawane za właściwe Szwedom}
\Clue{41}{}{w geometrii: ściana powierzchni wielościennej albo wielościanu - jeden z wielokątów, które tworzą jej/jego brzeg}
\Clue{42}{}{aglutynacyjny język Indian Keczua będący językiem urzędowym w Peru i Boliwii}
\Clue{43}{}{zwolennik socjalizmu}
\Clue{44}{}{kompozytor radziecki (1896-1984); autor 'Hymnu młodzieży demokratycznej świata'; pieśni chóralne, masowe}
\Clue{45}{}{możliwość, sposobność jakiegoś działania}
\Clue{46}{}{hiszpański kompozytor i organista (1500-1566); utwory instrumentalne i wokalne}\end{PuzzleClues}

\begin{PuzzleClues}{\textbf{Pionowe}\\}\Clue{1}{}{malarz niemiecki (1774-1840) reprezentant romantyzmu}
\Clue{2}{}{sprzeczność, rozbieżność, niezgoda}
\Clue{3}{}{rodzaj drewna z punktu widzenia spełniania norm jakościowych, wymiarów i przeznaczenia}
\Clue{4}{}{godny pogardy człowiek, który lubi efekciarstwo, idzie na łatwiznę}
\Clue{5}{}{część gospodarki, obejmująca ogół zagadnień związanych z budową mieszkań}
\Clue{7}{}{dział wiedzy związany z inżynierią lądową, skupiający się na analizie i projektowaniu obiektów budowlanych, w taki sposób aby przeciwdziałać siłom działającym na ową konstrukcję a tym samym zapobiegać możliwości jej zniszczenia}
\Clue{8}{}{fałdka, szczelina na mózgu}
\Clue{9}{}{mieszkanka Bolesławia, kobieta pochodząca z Bolesławia}
\Clue{10}{}{w górnictwie - rodzaj wysokiej pionowej wyrwy}
\Clue{11}{}{mała czamara - męskie okrycie wierzchnie pochodzenia węgierskiego, noszone od XVI wieku}
\Clue{12}{}{rzeźbiarz niemiecki (1725-75) polichromowane rzeźby o religijnej tematyce}
\Clue{14}{}{rodzaj choroby autoimunologicznej, która w łagodnej formie pojawia się w formie zaczerwienienia na skórze w nieosłoniętych przed słońcem miejscach i przybiera postać motyla na nosie i po jego obu stronach na policzkach}
\Clue{15}{}{ur. w 1928 r., awangardowy kompozytor niemiecki; utwory orkiestrowe, kameralne, fortepianowe, elektroniczne}
\Clue{16}{}{drzewo obecne w rozmaitych wierzeniach, które ma być filarem świata i symbolem jego równowagi}
\Clue{18}{}{dział fizyki wykorzystujący pojęcia, teorie i metody fizyczne w zastosowaniach medycznych (profilaktyka, diagnostyka, terapia, rehabilitacja)}
\Clue{20}{}{Discelium nudum - gatunek mchu z rodziny osadniczkowatych}
\Clue{23}{}{dosyć dobra znajomość czegoś}
\Clue{24}{}{mieszkanka Bejrutu}
\Clue{25}{}{głos o bardziej rozległej skali niż alt, charakteryzujący się głęboką, ciemną barwą w rejestrze dolnym oraz biegłością (koloraturą), której brak altowi}
\Clue{27}{}{motywacja do zrobienia czegoś}
\Clue{28}{}{nauka o dziejach}
\Clue{29}{}{czas, który zostaje przesunięty na zimę o jedną godzinę w stosunku do czasu strefowego lub urzędowego}
\Clue{30}{}{wydawnictwo okazyjne, aperiodyczne, niezaliczane do prasy}
\Clue{31}{}{fizyk, astronom i matematyk holenderski (1629-95); prekursor falowej teorii światła, twórca teorii rozchodzenia się fal}
\Clue{32}{}{uzbrojony, prywatny statek handlowy upoważniony do prowadzenia wojny morskiej}
\Clue{36}{}{pomost ze statku na ląd lub schodki pomiędzy pokładami}
\Clue{37}{}{łuk zamykający od góry otwór tęczowy w ścianie oddzielającej nawę kościoła od prezbiterium; wyróżniony przez bogate ozdobienie, zmianę w materiale lub w kolorze}
\Clue{38}{}{członek męski, penis}\end{PuzzleClues}\newpage\section*{Krzyżówka 8}

\noindent\begin{Puzzle}{23}{28}|*	|*	|*	|[1][S]\drarr	|a	|w	|a	|n	|s	|*	|[2][S]\drarr	|b	|u	|r	|d	|a	|*	|*	|*	|*	|*	|*	|*	|*	|.
|[3][S]\rarr	|w	|ą	|s	|k	|i	|e	|[][,]{ }	|g	|a	|r	|d	|ł	|o	|*	|*	|*	|[4][S]\drarr	|f	|r	|e	|u	|d	|*	|.
|[5][S]\drarr	|u	|b	|a	|w	|[][,]{ }	|p	|o	|[][,]{ }	|p	|a	|c	|h	|y	|*	|*	|[6][S]\rarr	|m	|o	|l	|t	|o	|n	|*	|.
|p	|*	|*	|c	|[7][S]\drarr	|m	|e	|d	|i	|u	|m	|*	|[8][S]\rarr	|p	|a	|t	|r	|o	|c	|h	|y	|*	|*	|*	|.
|a	|*	|*	|c	|d	|*	|*	|*	|*	|*	|d	|[9][S]\darr	|[10][S]\drarr	|w	|i	|w	|e	|r	|a	|*	|*	|*	|*	|*	|.
|l	|*	|[11][S]\drarr	|h	|y	|r	|a	|d	|*	|*	|y	|c	|p	|*	|*	|*	|*	|f	|*	|*	|[12][S]\darr	|[13][S]\drarr	|m	|*	|.
|c	|[14][S]\rarr	|p	|i	|k	|i	|e	|t	|a	|*	|s	|h	|o	|[15][S]\darr	|*	|*	|*	|e	|[16][S]\darr	|*	|a	|c	|*	|*	|.
|ó	|*	|o	|*	|t	|*	|[17][S]\darr	|*	|*	|[18][S]\drarr	|k	|o	|l	|e	|ś	|*	|*	|m	|k	|*	|n	|h	|*	|*	|.
|w	|*	|l	|*	|a	|*	|n	|*	|*	|j	|*	|m	|e	|p	|*	|*	|*	|[][,]{ }	|o	|*	|g	|o	|*	|*	|.
|k	|*	|e	|*	|t	|*	|a	|*	|*	|a	|*	|ą	|[][,]{ }	|i	|*	|[19][S]\darr	|*	|l	|m	|[20][S]\darr	|l	|c	|[21][S]\darr	|*	|.
|a	|*	|*	|*	|o	|*	|w	|*	|*	|g	|*	|t	|w	|s	|[22][S]\darr	|r	|*	|e	|p	|p	|i	|h	|g	|*	|.
|*	|*	|[23][S]\darr	|*	|r	|*	|l	|[24][S]\drarr	|b	|ł	|ę	|k	|i	|t	|n	|y	|[][,]{ }	|k	|a	|r	|z	|e	|ł	|*	|.
|*	|*	|b	|*	|*	|[25][S]\darr	|e	|g	|*	|a	|*	|o	|r	|o	|i	|j	|[26][S]\darr	|s	|r	|o	|o	|l	|o	|*	|.
|*	|*	|e	|[27][S]\darr	|*	|k	|c	|p	|*	|*	|*	|*	|o	|ł	|e	|e	|i	|y	|a	|m	|w	|k	|s	|*	|.
|[28][S]\drarr	|b	|r	|z	|e	|s	|z	|c	|z	|o	|t	|*	|w	|a	|o	|c	|n	|k	|t	|i	|a	|a	|[][,]{ }	|*	|.
|w	|[29][S]\drarr	|l	|e	|j	|e	|k	|*	|*	|*	|*	|*	|e	|*	|g	|*	|t	|a	|o	|n	|n	|*	|d	|*	|.
|a	|k	|i	|t	|*	|r	|a	|*	|*	|[30][S]\darr	|*	|[31][S]\darr	|*	|*	|r	|[32][S]\darr	|o	|l	|r	|e	|i	|*	|o	|*	|.
|p	|a	|n	|t	|*	|o	|[][,]{ }	|*	|*	|l	|[33][S]\darr	|p	|*	|*	|a	|z	|n	|n	|[][,]{ }	|n	|e	|*	|r	|*	|.
|n	|m	|k	|a	|*	|g	|s	|*	|[34][S]\darr	|a	|e	|o	|*	|*	|n	|i	|a	|y	|b	|t	|*	|[35][S]\darr	|a	|*	|.
|i	|i	|a	|b	|*	|r	|z	|[36][S]\drarr	|s	|k	|u	|b	|a	|n	|i	|e	|c	|*	|ł	|*	|*	|d	|d	|*	|.
|s	|o	|*	|a	|*	|a	|a	|k	|z	|t	|s	|i	|*	|*	|c	|m	|j	|*	|y	|*	|*	|i	|c	|*	|.
|k	|n	|*	|j	|*	|f	|r	|a	|a	|o	|e	|a	|*	|*	|z	|i	|a	|*	|s	|*	|*	|a	|z	|*	|.
|o	|e	|*	|t	|*	|i	|a	|l	|l	|r	|b	|ł	|*	|*	|o	|a	|*	|*	|k	|*	|*	|g	|y	|*	|.
|*	|t	|*	|*	|*	|a	|*	|i	|o	|o	|i	|k	|*	|*	|n	|*	|[37][S]\drarr	|k	|o	|g	|a	|n	|*	|*	|.
|*	|k	|*	|*	|*	|*	|*	|f	|t	|l	|o	|a	|[38][S]\rarr	|p	|o	|d	|k	|o	|w	|a	|*	|o	|*	|*	|.
|*	|a	|[39][S]\rarr	|s	|z	|a	|r	|a	|k	|*	|*	|*	|*	|*	|ś	|*	|u	|*	|y	|*	|*	|z	|*	|*	|.
|*	|*	|[40][S]\rarr	|s	|z	|ó	|s	|t	|a	|[][,]{ }	|c	|z	|ę	|ś	|ć	|*	|l	|*	|*	|*	|*	|a	|*	|*	|.
|*	|*	|*	|*	|*	|*	|*	|*	|*	|*	|*	|*	|*	|*	|*	|*	|a	|*	|*	|*	|*	|*	|*	|*	|.
|*	|*	|*	|*	|*	|*	|*	|*	|*	|*	|*	|*	|*	|*	|*	|*	|*	|*	|*	|*	|*	|*	|*	|*	|.\end{Puzzle}

\newpage

\begin{PuzzleClues}{\textbf{Poziome}\\}\Clue{1}{}{suma, zaliczka}
\Clue{2}{}{kawał wełnianej tkaniny używany jako płaszcz lub koc przez Arabów}
\Clue{3}{}{coś, co sprawia trudność, utrudnia przejście przez jakiś problem, opóźnia coś, powoduje niedopełnienie czegoś}
\Clue{4}{}{austriacki neurolog i psychiatra (1856-1939); twórca psychoanalizy}
\Clue{5}{}{niezły ubaw, dużo śmiechu}
\Clue{6}{}{bielizna stołowa, płat tkaniny kładziony pod obrus, bezpośrednio na blacie stołu}
\Clue{7}{}{osoba, która umie przewidywać przyszłość i odczuwać połączenia telepatyczne}
\Clue{8}{}{termin łowiecki nazywający wnętrzności zwierzyny łownej}
\Clue{10}{}{ŁASZA WODNIK TOPIK}
\Clue{11}{}{polietylen stosowany jako izolacja przewodów i drutów nawojowych}
\Clue{13}{}{skrót, symbol jednostki - metra}
\Clue{14}{}{jedna z najstarszych gier karcianych pochodząca z Francji}
\Clue{18}{}{jakiś mężczyzna lub chłopak, zwłaszcza obcy lub taki, względem którego zachowujemy dystans; facet, gościu}
\Clue{24}{}{gwiazda o typie jasności V i ciągu widmowym O}
\Clue{28}{}{rodzaj ceremonialnego miecza}
\Clue{29}{}{rozszerzona, górna część wlewu głównego ułatwiająca wprowadzenie ciekłego metalu}
\Clue{36}{}{rodzaj ciasta na kruchym spodzie, które zwykle jest przełożone jakąś masą (gł. owocową), na górze też miewa coś chrupiącego}
\Clue{37}{}{skrzypek radziecki (1924-1982); wirtuoz światowej sławy}
\Clue{38}{}{łow. brązowy pas na piersi kuropatwy}
\Clue{39}{}{gatunek łownego zająca}
\Clue{40}{}{jedna szósta, efekt dzielenia na sześć}\end{PuzzleClues}

\begin{PuzzleClues}{\textbf{Pionowe}\\}\Clue{1}{}{wybitny trener piłkarski, prowadził reprezentację Włoch i AC Milan}
\Clue{2}{}{obszar komputerowej pamięci RAM, który jest wykorzystywany jako pamięć dyskowa}
\Clue{4}{}{morfem samodzielny lub związany, obecny w każdym leksemie, o dominującej funkcji semantycznej, odsyła do świata pozajęzykowego, pełniąc funkcję referencjalną}
\Clue{5}{}{polska nazwa niemieckiego dowodu osobistego, wprowadzonego na terenach okupowanej Polski w 1939}
\Clue{7}{}{urzędnik nadzwyczajny w starożytnym Rzymie, posiadający nieograniczoną władzę na okres maksymalnie sześciu miesięcy}
\Clue{9}{}{KAUSZA}
\Clue{10}{}{pole wektorowe, w którym rotacja nie jest równa zero}
\Clue{11}{}{w informatyce: cecha, która opisuje statyczny aspekt obiektu; znaczenie może być tożsame z polem-12, jeśli nastąpi odpowiednie rzutowanie/mapowanie}
\Clue{12}{}{przycinanie ogona i grzywy konia na angielski sposób}
\Clue{13}{}{tyle, ile się mieści w chochelce}
\Clue{15}{}{żartobliwie o rozwlekłym, przydługim i nieciekawym tekście}
\Clue{16}{}{przyrząd astronomiczny służący do wizualnego wykrywania drobnych różnic pomiędzy dwoma obrazami przedstawiającymi wspólny fragment nieba}
\Clue{17}{}{indusium griseum, struktura mózgowa należąca do zewnętrznej części płatu limbicznego}
\Clue{18}{}{kasza jaglana}
\Clue{19}{}{określenie zwierzęcia, które ryje}
\Clue{20}{}{więzień hilerowskiego obozu ubiegający się o względy oprawców}
\Clue{21}{}{w Kodeksie spółek handlowych - głos, który nie upoważnia do bezpośredniego podejmowania decyzji na zebraniach zarządu ani głosowania w jego uchwałach}
\Clue{22}{}{brak ograniczeń, obwarowania jakimiś warunkami}
\Clue{23}{}{dawna srebrna pruska moneta}
\Clue{24}{}{jednostka odległości używana w astronomii pozagalaktycznej i kosmologii; jeden gigaparsek = 10E+9 pc}
\Clue{25}{}{metoda tworzenia kopii dokumentów i dwuwymiarowych elementów graficznych przy użyciu kserokopiarki}
\Clue{26}{}{sposób, w jaki akcentowane są wyrazy i zdania}
\Clue{27}{}{jednostka używana w informatyce oznaczająca (zgodnie z zaleceniami IEC) tryliard bajtów (1 ZB = 10E+21 B)}
\Clue{28}{}{zgrubiale o wapnie}
\Clue{29}{}{półciężarówka}
\Clue{30}{}{napój otrzymywany z pasteryzowanego, odtłuszczonego mleka zakwaszonego maślanką}
\Clue{31}{}{mieszanina kleju i gipsu lub kredy, stosowana jako zaprawa pod malowidło olejne}
\Clue{32}{}{miejsce życia człowieka wierzącego przed pójściem do raju lub śmiercią (w niektórych religiach także piekłem lub czyścem)}
\Clue{33}{}{piłkarz portugalski, zwany 'Czarną Perłą Mozambiku' napastnik Benfiki Lizbona, najlepszy piłkarz Europy w 1965 r}
\Clue{34}{}{rodzaj małej cebuli, warzywa, która jest ceniona w kulinariach dzięki swojemu dość delikatnemu, lekko czosnkowemu smakowi; cebulka rośliny nazywanej tak samo}
\Clue{35}{}{rodzaj sprawdzianu w szkole, który ma dość duże znaczenie}
\Clue{36}{}{muzułmańska organizacja społeczno-polityczna}
\Clue{37}{}{bryła geometryczna w kształcie kuli}\end{PuzzleClues}\newpage\section*{Krzyżówka 9}

\noindent\begin{Puzzle}{23}{27}|*	|[1][S]\darr	|*	|*	|*	|*	|[2][S]\drarr	|r	|y	|b	|i	|t	|w	|a	|[][,]{ }	|b	|i	|a	|ł	|a	|*	|*	|[3][S]\darr	|*	|.
|*	|k	|*	|*	|*	|[4][S]\rarr	|g	|e	|n	|[][,]{ }	|f	|u	|z	|y	|j	|n	|y	|*	|*	|[5][S]\darr	|*	|*	|z	|*	|.
|*	|o	|[6][S]\drarr	|s	|p	|i	|r	|i	|t	|u	|s	|[][,]{ }	|m	|o	|v	|e	|n	|s	|*	|m	|*	|[7][S]\darr	|e	|*	|.
|*	|r	|p	|[8][S]\drarr	|n	|i	|e	|b	|i	|o	|s	|a	|*	|*	|[9][S]\rarr	|b	|a	|m	|y	|a	|*	|g	|g	|*	|.
|[10][S]\drarr	|p	|a	|t	|r	|o	|n	|*	|[11][S]\drarr	|p	|s	|i	|z	|ą	|b	|[][,]{ }	|t	|a	|y	|l	|o	|r	|a	|*	|.
|m	|u	|l	|e	|*	|[12][S]\darr	|l	|*	|z	|*	|[13][S]\darr	|*	|*	|*	|*	|*	|[14][S]\darr	|*	|*	|a	|*	|o	|r	|*	|.
|i	|s	|n	|r	|[15][S]\drarr	|b	|a	|ł	|a	|c	|h	|*	|*	|[16][S]\drarr	|l	|e	|s	|s	|*	|k	|[17][S]\darr	|m	|[][,]{ }	|*	|.
|r	|[][,]{ }	|i	|n	|p	|a	|n	|*	|m	|*	|a	|*	|*	|s	|*	|*	|ł	|*	|*	|k	|d	|o	|z	|*	|.
|u	|u	|k	|e	|i	|s	|d	|*	|e	|*	|e	|*	|*	|ł	|*	|[18][S]\darr	|a	|*	|*	|a	|y	|w	|[][,]{ }	|*	|.
|n	|r	|[][,]{ }	|w	|r	|e	|i	|*	|k	|*	|n	|[19][S]\drarr	|h	|o	|d	|o	|w	|c	|a	|*	|s	|ł	|k	|*	|.
|g	|z	|g	|*	|o	|n	|a	|*	|*	|*	|d	|k	|[20][S]\darr	|w	|[21][S]\darr	|t	|i	|[22][S]\darr	|[23][S]\darr	|*	|z	|a	|u	|*	|.
|a	|ę	|a	|*	|g	|*	|[][,]{ }	|*	|*	|*	|e	|o	|d	|i	|h	|o	|a	|g	|d	|*	|l	|d	|k	|*	|.
|*	|d	|z	|*	|a	|*	|g	|*	|*	|*	|l	|r	|r	|k	|i	|c	|n	|w	|z	|*	|o	|c	|u	|*	|.
|*	|n	|o	|*	|*	|*	|ę	|*	|*	|*	|*	|s	|e	|*	|p	|z	|k	|i	|i	|*	|w	|a	|ł	|*	|.
|*	|i	|w	|*	|*	|[24][S]\rarr	|s	|z	|a	|f	|r	|a	|n	|[][,]{ }	|o	|k	|a	|z	|a	|ł	|y	|*	|k	|*	|.
|*	|c	|y	|*	|*	|[25][S]\darr	|t	|[26][S]\rarr	|z	|e	|b	|r	|a	|*	|p	|a	|*	|d	|d	|*	|*	|*	|ą	|*	|.
|*	|z	|*	|*	|[27][S]\drarr	|r	|a	|p	|*	|*	|*	|z	|ż	|*	|o	|[][,]{ }	|[28][S]\darr	|e	|*	|*	|*	|*	|*	|*	|.
|*	|y	|*	|*	|p	|e	|*	|*	|*	|*	|*	|*	|*	|*	|t	|b	|g	|k	|*	|*	|*	|*	|*	|*	|.
|*	|*	|[29][S]\rarr	|p	|r	|z	|e	|d	|z	|i	|a	|ł	|*	|*	|a	|a	|r	|*	|*	|*	|*	|*	|*	|*	|.
|*	|*	|*	|*	|z	|y	|*	|[30][S]\rarr	|o	|l	|b	|r	|z	|y	|m	|k	|o	|w	|c	|e	|*	|*	|*	|*	|.
|*	|*	|[31][S]\drarr	|ż	|y	|d	|[][,]{ }	|w	|i	|e	|c	|z	|n	|y	|[][,]{ }	|t	|u	|ł	|a	|c	|z	|*	|*	|*	|.
|*	|*	|p	|[32][S]\rarr	|c	|e	|b	|a	|*	|*	|*	|*	|*	|*	|n	|e	|x	|*	|*	|*	|*	|*	|*	|*	|.
|*	|*	|e	|*	|h	|n	|*	|*	|*	|*	|*	|*	|*	|*	|i	|r	|*	|*	|*	|*	|*	|*	|*	|*	|.
|*	|*	|j	|*	|ó	|t	|*	|*	|*	|*	|*	|*	|*	|*	|l	|y	|*	|*	|*	|*	|*	|*	|*	|*	|.
|*	|*	|z	|*	|w	|u	|*	|*	|*	|*	|*	|*	|[33][S]\rarr	|b	|o	|j	|e	|r	|*	|*	|*	|*	|*	|*	|.
|[34][S]\rarr	|b	|a	|t	|e	|r	|i	|a	|[][,]{ }	|k	|o	|k	|s	|o	|w	|n	|i	|c	|z	|a	|*	|*	|*	|*	|.
|*	|*	|ż	|[35][S]\rarr	|k	|a	|p	|ł	|a	|n	|*	|*	|*	|*	|y	|a	|*	|*	|*	|*	|*	|*	|*	|*	|.
|*	|*	|*	|*	|*	|*	|*	|*	|*	|*	|*	|*	|*	|*	|*	|*	|*	|*	|*	|*	|*	|*	|*	|*	|.\end{Puzzle}

\newpage

\begin{PuzzleClues}{\textbf{Poziome}\\}\Clue{2}{}{Gygis alba alba - nominatywny podgatunek ptaka wodnego wyróżniony w obrębie gatunku rybitwa biała (Gygis alba)}
\Clue{4}{}{gen powstający w wyniku translokacji (przeniesienia fragmentu jednego chromosomu na drugi)}
\Clue{6}{}{siła sprawcza jakiegoś działania}
\Clue{8}{}{miejsce, gdzie mieszkają Bóg lub bóstwa, aniołowie, święci i zbawione dusze; raj}
\Clue{9}{}{hinduski bębenek jedno-membranowy}
\Clue{10}{}{prawnik, który jest bezpośrednim opiekunem aplikanta i kontroluje jego pracę}
\Clue{11}{}{Erythronium taylorii - gatunek roślin z rodziny liliowatych}
\Clue{15}{}{gwara lwowskich uliczników}
\Clue{16}{}{pylasta skała osadowa pochodzenia eolicznego}
\Clue{19}{}{osoba, która hoduje zwierzęta lub rośliny}
\Clue{24}{}{Crocus speciosus - gatunek rośliny z rodziny kosaćcowatych}
\Clue{26}{}{ssak z rodzaju Equus (rodzina koniowatych) charakteryzujący się obecnością białych pasów na czarnej sierści; zwierzę, które żyje stadnie na trawiastych terenach Afryki, na południe od Sahary}
\Clue{27}{}{BOLEŃ}
\Clue{29}{}{część skali, zbiór określonych wartości z danej skali}
\Clue{30}{}{Cyatheales - rząd paproci występujący głównie w strefie międzyzwrotnikowej, jak i wilgotnych lasach strefy umiarkowanej, głównie w Nowej Zelandii, Tasmanii i pd. Australii; większość z nich ma wysoką zdrewniałą łodygę, stąd określeniepaprocie drzewiaste}
\Clue{31}{}{legendarna postać Żyda, który miał znieważyć czynnie Chrystusa idącego z krzyżem na Golgotę, za co został ukarany wieczną tułaczką po świecie}
\Clue{32}{}{obciach, zaścianek, wiocha, buractwo}
\Clue{33}{}{ślizg sportowy}
\Clue{34}{}{zespół komór koksowniczych pieca koksowniczego wraz z komorami grzewczymi i regeneratorami}
\Clue{35}{}{człowiek, który w danej społeczności czy kulturze pełni funkcje kultowe}\end{PuzzleClues}

\begin{PuzzleClues}{\textbf{Pionowe}\\}\Clue{1}{}{sztab urzędników pracujących w ramach jakiejś struktury, np. w danym państwie, sektorze}
\Clue{2}{}{Groenlandia densa - gatunek rośliny wodnej z rodziny rdestnicowatcyh}
\Clue{3}{}{zegar typu wahadłowego z wbudowaną mechaniczną kukułką, który wskazując godzinę, wybija dźwięki przypominające jej wołanie}
\Clue{5}{}{MALAKA; stan w Malezji, w płd części Półwyspu Malajskiego, obszar 1,7 tyś.km2, stolica Malakka}
\Clue{6}{}{część kuchni gazowej}
\Clue{7}{}{ten, kto panuje nad piorunami; używane w stosunku do bóstw, przede wszystkim Zeusa-Jowisza}
\Clue{8}{}{ratowniczy pies z Nowej Finlandii}
\Clue{10}{}{słoń morski; największy gatunek z rodziny fok; nos samców wydłużony o krótką trąbkę}
\Clue{11}{}{blokada w informatyce}
\Clue{12}{}{zawartość basenu, naczynia, umożliwiającego osobom obłożnie chorym wypróżnianie się w pozycji leżącej lub półleżącej bez opuszczania łóżka}
\Clue{13}{}{skrzypaczka polska ur. w 1925 r., osiadła w Wielkiej Brytanii}
\Clue{14}{}{mieszkanka Sławy}
\Clue{15}{}{niewielka, płaskodenna łódź, napędzana wiosłem o jednym piórze (pagajem)}
\Clue{16}{}{chrząszcz z rodziny ryjkowców}
\Clue{17}{}{koń jadący przy dyszlu}
\Clue{18}{}{cienka warstwa widoczna w mikroskopie świetlnym otaczająca od zewnątrz ścianę komórkową komórki bakterii}
\Clue{19}{}{armator lub dowódca uzbrojonego statku handlowego, walczący na własny koszt i ryzyko w służbie swego mocodawcy, prowadzącego wojnę na morzu}
\Clue{20}{}{odprowadzanie treści płynnych (fizjologicznych lub patologicznych) z pewnych narządów lub obszarów ciała do innych lub częściej poza ciało chorego}
\Clue{21}{}{hipopotam pospolity, Hippopotamus amphibius - gatunek dużego, przeważnie roślinożernego ssaka, należącego do rodziny hipopotamowatych (Hippopotamidae); zasiedla rzeki i jeziora w Afryce na południe od Sahary do 2000 m n.p.m}
\Clue{22}{}{sygnał dźwiękowy}
\Clue{23}{}{mężczyzna oceniany negatywnie}
\Clue{25}{}{forma doskonalenia zawodowego umożliwiająca zdobycie określonej specjalizacji lekarskiej finansowana przez Ministerstwo Zdrowia}
\Clue{27}{}{wzrost liczby zwierząt hodowlanych lub łownych}
\Clue{28}{}{niemiecki malarz i grafik (1893-1959); twórczość o antywojennej i antyhitlerowskiej wymowie}
\Clue{31}{}{krajobraz przedstawiony w dziele sztuki}\end{PuzzleClues}\newpage\section*{Krzyżówka 10}

\noindent\begin{Puzzle}{16}{30}|*	|*	|*	|*	|*	|*	|*	|*	|*	|*	|*	|*	|*	|*	|*	|*	|[1][S]\darr	|.
|*	|[2][S]\darr	|*	|*	|[3][S]\drarr	|c	|o	|s	|a	|*	|[4][S]\darr	|*	|[5][S]\darr	|*	|*	|[6][S]\darr	|l	|.
|*	|c	|*	|[7][S]\drarr	|f	|o	|t	|o	|t	|y	|p	|*	|s	|[8][S]\darr	|*	|l	|i	|.
|*	|e	|*	|l	|o	|[9][S]\darr	|*	|[10][S]\drarr	|o	|k	|o	|*	|i	|k	|*	|i	|d	|.
|*	|r	|*	|i	|s	|d	|*	|d	|*	|*	|z	|*	|a	|u	|*	|c	|o	|.
|*	|t	|*	|s	|f	|z	|[11][S]\darr	|o	|[12][S]\drarr	|w	|n	|ę	|t	|r	|z	|e	|*	|.
|*	|y	|*	|e	|o	|i	|d	|w	|r	|*	|a	|*	|e	|c	|[13][S]\darr	|u	|*	|.
|*	|f	|[14][S]\darr	|k	|r	|a	|a	|c	|e	|*	|n	|*	|c	|z	|m	|m	|[15][S]\darr	|.
|*	|i	|m	|*	|y	|ł	|l	|i	|d	|*	|i	|*	|z	|l	|u	|[][,]{ }	|p	|.
|*	|k	|i	|[16][S]\drarr	|n	|a	|m	|p	|u	|l	|a	|*	|k	|i	|c	|p	|i	|.
|*	|a	|e	|m	|*	|n	|i	|n	|t	|[17][S]\darr	|n	|*	|a	|w	|h	|l	|k	|.
|[18][S]\drarr	|t	|r	|o	|c	|i	|n	|i	|a	|r	|k	|i	|*	|o	|a	|a	|o	|.
|b	|*	|z	|r	|[19][S]\darr	|e	|e	|ś	|*	|a	|a	|*	|[20][S]\darr	|ś	|[][,]{ }	|s	|m	|.
|*	|[21][S]\darr	|w	|e	|l	|*	|*	|*	|*	|m	|*	|*	|l	|ć	|m	|t	|e	|.
|[22][S]\drarr	|m	|a	|n	|i	|f	|e	|s	|t	|a	|c	|j	|a	|*	|o	|y	|t	|.
|t	|m	|*	|i	|e	|*	|*	|[23][S]\darr	|*	|*	|*	|*	|t	|*	|k	|c	|r	|.
|e	|k	|*	|a	|s	|*	|[24][S]\drarr	|k	|n	|a	|j	|p	|i	|a	|r	|z	|*	|.
|r	|*	|*	|[][,]{ }	|t	|*	|c	|l	|*	|*	|[25][S]\darr	|*	|m	|[26][S]\darr	|a	|n	|*	|.
|m	|[27][S]\darr	|*	|p	|a	|*	|z	|a	|[28][S]\darr	|[29][S]\darr	|f	|*	|e	|ś	|*	|e	|*	|.
|a	|p	|*	|a	|l	|*	|u	|r	|k	|j	|r	|[30][S]\darr	|r	|l	|[31][S]\darr	|*	|*	|.
|*	|r	|*	|w	|*	|*	|b	|*	|o	|i	|e	|w	|i	|i	|w	|*	|*	|.
|*	|e	|*	|i	|*	|*	|*	|*	|s	|v	|d	|i	|a	|n	|i	|*	|*	|.
|[32][S]\rarr	|z	|n	|o	|s	|z	|o	|m	|i	|e	|r	|z	|*	|i	|l	|*	|*	|.
|*	|e	|[33][S]\darr	|o	|[34][S]\rarr	|p	|a	|k	|a	|*	|o	|j	|[35][S]\darr	|a	|g	|*	|*	|.
|[36][S]\drarr	|s	|z	|k	|a	|p	|l	|e	|r	|z	|*	|e	|d	|n	|a	|*	|*	|.
|k	|ó	|n	|a	|*	|*	|[37][S]\rarr	|s	|z	|n	|u	|r	|e	|k	|*	|*	|*	|.
|o	|w	|a	|*	|*	|[38][S]\drarr	|a	|g	|e	|n	|t	|*	|m	|a	|*	|*	|*	|.
|l	|n	|m	|*	|*	|f	|*	|*	|*	|*	|*	|*	|*	|*	|*	|*	|*	|.
|t	|a	|i	|[39][S]\rarr	|k	|l	|a	|s	|a	|*	|*	|*	|*	|*	|*	|*	|*	|.
|*	|*	|ę	|*	|*	|*	|*	|*	|*	|*	|*	|*	|*	|*	|*	|*	|*	|.
|*	|*	|*	|*	|*	|*	|*	|*	|*	|*	|*	|*	|*	|*	|*	|*	|*	|.\end{Puzzle}

\newpage

\begin{PuzzleClues}{\textbf{Poziome}\\}\Clue{3}{}{(1460-1510) żeglarz hiszpański, uczestnik wypraw Kolumba, autor map odkryć Kolumba}
\Clue{7}{}{typ skóry w zależności od ilości zawartej w niej melaniny}
\Clue{10}{}{to, co jest kształtem lub funkcją podobne do oka}
\Clue{12}{}{przestrzeń wewnątrz czegoś}
\Clue{16}{}{miasto w północnym Mozambiku; ośrodek administracyjny prowincji Nampula, przy linii kolejowej z Malawi do miasta Mozambik}
\Clue{18}{}{trociniarkowate, Cossidae, - rodzina motyli, zaliczana do dużych motyli nocnych; obejmuje ok. 670 gatunków; cechą charakterystyczną rodziny jest uwsteczniony narząd gębowy}
\Clue{22}{}{demonstrowanie, prezentowanie, manifestowanie czegoś; pokaz, prezentacja czegoś, ostentacyjne działanie obliczone na efekt}
\Clue{24}{}{osoba, która jest właścicielem knajpy}
\Clue{32}{}{DERYWOMETR}
\Clue{34}{}{południowoamerykański gryzoń z rodziny aguti, roślinożerna, szkodnik upraw}
\Clue{36}{}{kawałek materiału z naszytymi wizerunkami Chrystusa i Matki Bożej, noszony przez osoby świeckie na znak duchowej i cielesnej czystości oraz duchowego związku z Matką Boską; noszenie szkaplerza i stosowanie się do pewnych reguł ma odganiać od noszącego złe moce i zapewnić mu miejsce w niebie}
\Clue{37}{}{cienka, mała struga cieczy}
\Clue{38}{}{tajny funkcjonariusz służb wewnętrznych: UOP-u, ABW, SB, FBI, tajnej policji, także Interpolu, cichociemny, tajniak}
\Clue{39}{}{grupa uczniów, którzy wspólnie uczęszczają na lekcje}\end{PuzzleClues}

\begin{PuzzleClues}{\textbf{Pionowe}\\}\Clue{1}{}{piaszczysty, wynurzony wał nadbudowany od strony morza przez fale, powoduje powstawanie laguny}
\Clue{2}{}{dokument określający przynależność państwową statku, a także stwierdzający przeprowadzenie inspekcji urządzeń technicznych statku}
\Clue{3}{}{sól kwasu fosforawego (fosforowego(III))}
\Clue{4}{}{mieszkanka Poznania}
\Clue{5}{}{zdrobniale: siatka - cienkie linie pokrywające coś i przecinające się}
\Clue{6}{}{typ szkoły artystycznej w Polsce o czteroletnim cyklu kształcenia, dającej wykształcenie w zawodzie plastyk oraz wykształcenie ogólne w zakresie liceum ogólnokształcącego, umożliwiającej uzyskanie świadectwa dojrzałości po zdaniu egzaminu maturalnego}
\Clue{7}{}{zdrobniale: lis - wyprawione futro z lisa}
\Clue{8}{}{skurczliwość - właściwość materiałów oznaczająca ich podatność na zmiany wilgotności i temperatury powodujące zmniejszenie ich wymiarów}
\Clue{9}{}{czyn, akcja, skutek podjęcia czynności}
\Clue{10}{}{człowiek, który lubi żartować i rozśmieszać innych}
\Clue{11}{}{miasto we Włoszech (Lombardia); hutnictwo żelaza, przemysł metalowy, maszynowy}
\Clue{12}{}{bal maskowy, publiczna impreza taneczna}
\Clue{13}{}{rodzaj sztucznej przynęty używanej do amatorskiego połowu ryb metodą muchową; wykonana na haku o jednym lub dwóch ostrzach, przy użyciu różnego rodzaju piór, korków, nici}
\Clue{14}{}{słoma przeznaczona na ściółkę}
\Clue{15}{}{podwielokrotność metra, podstawowej jednostki długości w układzie SI; jeden pikometr równa się 10E-12 m}
\Clue{16}{}{żółw pawiooki, Morenia ocellata - gatunek gada z rodziny batagurowatych, występujący w Birmie}
\Clue{17}{}{pewne założenia stanowiące podstawę jakiegoś przedsięwzięcia}
\Clue{18}{}{bajt - najmniejsza adresowalna jednostka informacji pamięci komputerowej, składająca się z bitów}
\Clue{19}{}{miasto w Szwajcarii, stolica półkantonu Bazylea-okręg, przemysł maszynowy, metalowy}
\Clue{20}{}{jedyny żyjący współcześnie gatunek ryb trzonopłetwych występujący w Oceanie Indyjskim}
\Clue{21}{}{kod ISO 4217 waluty kyat}
\Clue{22}{}{starożytna łaźnia publiczna}
\Clue{23}{}{staranne ułożenie osprzętu pokładowego; porządek na statku}
\Clue{24}{}{rodzaj fryzury, włosy spięte na czubku głowy (kok albo kitka, albo coś jeszcze innego)}
\Clue{25}{}{Andrzej (1620-79), pisarz polityczny, wojewoda podolski}
\Clue{26}{}{gruczoł wytwarzający ślinę}
\Clue{27}{}{córka prezesa}
\Clue{28}{}{nazwa 3gwiazd tworzących tzw. pas w gwiazdozbiorze Oriona}
\Clue{29}{}{muzyka, do której tańczy się jive'a}
\Clue{30}{}{urządzenie montowane w pojazdach bojowych, służące do obserwacji otoczenia oraz do celowania pociskami, np. wizjer czołgu}
\Clue{31}{}{ptak z rzędu wróblowatych długości do 23 cm; samce żółte o czarnych skrzydłach, owadożerna, chroniona}
\Clue{33}{}{część słupka roślin okrytonasiennych przyjmująca ziarna pyłku w trakcie zapylenia (przed zapłodnieniem)}
\Clue{35}{}{podstawowa jednostka administracyjna i terytorialna w starożytnej Grecji, gmina}
\Clue{36}{}{6-strzałowy rewolwer bębenkowy COLT}
\Clue{38}{}{w chemii: symbol pierwiastka flerow}\end{PuzzleClues}\newpage\section*{Krzyżówka 11}

\noindent\begin{Puzzle}{17}{32}|*	|*	|*	|*	|*	|*	|*	|[1][S]\darr	|*	|*	|[2][S]\drarr	|s	|z	|k	|ł	|o	|*	|[3][S]\darr	|.
|*	|*	|[4][S]\darr	|*	|*	|[5][S]\rarr	|i	|n	|d	|e	|k	|s	|*	|*	|*	|*	|*	|p	|.
|*	|[6][S]\darr	|m	|*	|*	|[7][S]\drarr	|b	|o	|m	|b	|a	|s	|t	|*	|*	|*	|*	|a	|.
|*	|d	|a	|*	|*	|z	|*	|r	|*	|*	|p	|*	|*	|*	|*	|*	|*	|p	|.
|*	|o	|i	|[8][S]\darr	|*	|w	|*	|m	|*	|*	|i	|*	|[9][S]\drarr	|z	|b	|*	|*	|i	|.
|*	|r	|t	|g	|*	|i	|*	|a	|*	|*	|t	|*	|k	|*	|[10][S]\darr	|[11][S]\darr	|*	|e	|.
|*	|t	|r	|o	|*	|ą	|[12][S]\drarr	|l	|o	|ż	|a	|*	|o	|*	|k	|z	|*	|r	|.
|*	|m	|e	|ł	|*	|z	|m	|i	|*	|*	|n	|*	|m	|[13][S]\darr	|u	|b	|*	|e	|.
|*	|u	|[][,]{ }	|o	|*	|e	|a	|z	|*	|*	|*	|*	|o	|r	|r	|i	|[14][S]\darr	|k	|.
|*	|n	|d	|b	|[15][S]\drarr	|k	|l	|a	|s	|a	|[][,]{ }	|g	|r	|a	|f	|ó	|w	|*	|.
|*	|d	|[][S]’	|o	|b	|[][,]{ }	|u	|t	|*	|*	|[16][S]\darr	|*	|a	|t	|i	|r	|r	|*	|.
|[17][S]\drarr	|c	|h	|r	|o	|s	|t	|o	|w	|s	|k	|i	|*	|o	|r	|[][,]{ }	|ó	|*	|.
|ś	|z	|o	|z	|a	|i	|i	|r	|*	|[18][S]\darr	|o	|*	|[19][S]\darr	|w	|s	|p	|b	|*	|.
|l	|y	|t	|e	|*	|a	|n	|*	|*	|m	|n	|*	|a	|n	|z	|o	|e	|*	|.
|i	|k	|e	|*	|*	|r	|*	|[20][S]\darr	|*	|l	|t	|*	|l	|i	|t	|t	|l	|*	|.
|s	|*	|l	|*	|[21][S]\rarr	|k	|o	|b	|i	|e	|r	|z	|e	|c	|*	|ę	|[][,]{ }	|*	|.
|k	|*	|*	|[22][S]\rarr	|d	|o	|r	|o	|ż	|k	|a	|*	|r	|t	|[23][S]\darr	|g	|s	|*	|.
|o	|*	|*	|*	|*	|o	|*	|y	|*	|o	|p	|*	|g	|w	|z	|o	|z	|*	|.
|ś	|[24][S]\rarr	|s	|p	|a	|r	|d	|e	|k	|*	|u	|*	|e	|o	|u	|w	|a	|*	|.
|ć	|*	|*	|[25][S]\darr	|*	|g	|[26][S]\darr	|*	|*	|*	|n	|*	|n	|[][,]{ }	|ż	|y	|r	|*	|.
|*	|*	|*	|e	|*	|a	|p	|*	|*	|*	|k	|*	|[][,]{ }	|g	|y	|*	|y	|*	|.
|[27][S]\drarr	|d	|r	|w	|i	|n	|a	|*	|*	|*	|t	|*	|w	|ó	|c	|*	|*	|*	|.
|z	|[28][S]\rarr	|r	|o	|b	|i	|n	|s	|o	|n	|*	|*	|z	|r	|i	|*	|*	|*	|.
|a	|*	|*	|l	|[29][S]\darr	|c	|c	|*	|*	|*	|*	|*	|i	|s	|e	|*	|[30][S]\darr	|*	|.
|p	|*	|[31][S]\darr	|u	|t	|z	|z	|*	|[32][S]\rarr	|r	|o	|ż	|e	|k	|*	|*	|w	|*	|.
|a	|*	|l	|c	|o	|n	|e	|*	|*	|*	|*	|[33][S]\rarr	|w	|i	|d	|ł	|y	|*	|.
|l	|*	|a	|j	|p	|y	|n	|*	|*	|*	|*	|*	|n	|e	|*	|*	|c	|*	|.
|n	|*	|t	|o	|i	|*	|l	|*	|[34][S]\rarr	|g	|ó	|r	|y	|*	|*	|*	|i	|*	|.
|o	|[35][S]\rarr	|a	|n	|e	|l	|a	|c	|e	|*	|*	|*	|*	|*	|*	|*	|ę	|*	|.
|ś	|*	|w	|i	|l	|[36][S]\rarr	|m	|a	|l	|g	|h	|i	|r	|*	|*	|*	|c	|*	|.
|ć	|[37][S]\rarr	|c	|z	|e	|l	|a	|d	|n	|i	|k	|*	|*	|*	|*	|*	|i	|*	|.
|*	|*	|e	|m	|c	|*	|*	|*	|*	|*	|*	|*	|*	|*	|*	|*	|e	|*	|.
|*	|*	|*	|*	|*	|*	|*	|*	|*	|*	|*	|*	|*	|*	|*	|*	|*	|*	|.\end{Puzzle}

\newpage

\begin{PuzzleClues}{\textbf{Poziome}\\}\Clue{2}{}{kawałek szkła uformowany w jakiś sposób, zwykle taki, który ma jakieś zastosowanie}
\Clue{5}{}{książeczka studenta, w której zbiera poświadczenia zdanych egzaminów, zaliczenia itp., rodzaj dokumentu, zaświadczenia}
\Clue{7}{}{dekatyzowana, niskiej jakości tkanina półwełniana, przeznaczona na lżejszą odzież, od XVI wieku wyrabiana w Gdańsku i na Śląsku}
\Clue{9}{}{skrót od zettabitu; jednostka informacji oznaczająca 10\textasciicircum21 bitów}
\Clue{12}{}{miejsce spotkań wolnomularzy}
\Clue{15}{}{klasa zawierająca wszystkie grafy spełniające jakieś warunki}
\Clue{17}{}{grafik (1885-1950) profesor ASP w Warszawie: litografie, drzeworyty, ekslibrisy, plakaty}
\Clue{21}{}{przen. powierzchnia, warstwa czegoś, najczęściej roślin, która przypomina kobierzec}
\Clue{22}{}{lekki pojazd konny  z opuszczaną budą}
\Clue{24}{}{górny, lekki pokład na statku wodnym}
\Clue{27}{}{sytuacja, którą można uznać za żart, często bolesny dla kogoś}
\Clue{28}{}{Glenn, koszykarz Milwaukee Bucks, członek Dream Teamu III, złoty medalista z Atlanty}
\Clue{32}{}{zdrobniale: róg - miejsce, gdzie stykają się linie wyznaczające brzeg czegoś i powstaje kąt}
\Clue{33}{}{miejsce, w któym koryta spotykających się rzek tworzą układ przypominający dwuzębne widły}
\Clue{34}{}{górzysty teren, taki, na którym występują pasma, łańcuchy górskie}
\Clue{35}{}{broń sieczna, krótki sztylet używany w XV w}
\Clue{36}{}{słone jezioro w Algierii, powierzchnia 6700 km2}
\Clue{37}{}{rzemieślnik praktykujący, szkolący się pod kierunkiem majstra}\end{PuzzleClues}

\begin{PuzzleClues}{\textbf{Pionowe}\\}\Clue{1}{}{osoba zajmująca się przygotowaniem norm technicznych}
\Clue{2}{}{dowódca statku powietrznego}
\Clue{3}{}{mały kawałek papieru}
\Clue{4}{}{szef sali restauracyjnej, starszy kelner dyskretnie dbający o gości}
\Clue{6}{}{mieszkaniec Dortmundu}
\Clue{7}{}{związek organiczny zawierający wiązanie węgiel-siarka}
\Clue{8}{}{teren w górach pokryty rumowiskiem skalnym}
\Clue{9}{}{wydzielona, wewnętrzna część maszyny lub urządzenia o rozmaitych zastosowaniach}
\Clue{10}{}{elektor, książę dawnej Rzeszy Niemieckiej}
\Clue{11}{}{dla danego zbioru X zbiór wszystkich jego podzbiorów}
\Clue{12}{}{malarz szwajcarski (1702-89) nastrojowe portrety, sceny rodzajowe, miniatury}
\Clue{13}{}{działania związane z prowadzeniem akcji ratowniczych w terenie górskim, poszukiwanie zaginionych osób, udzielanie pomocy medycznej ofiarom wypadków, transport poszkodowanych do miejsc, gdzie można im udzielić pełnej pomocy medycznej; również działania prewencyjne związane z informowaniem o zagrożeniach, niebezpieczeństwie lawin i spodziewanych załamaniach pogody}
\Clue{14}{}{Passer swainsonii - gatunek ptaka z rodziny wróblowatych (Passeridae)}
\Clue{15}{}{Boinae - podrodzina węży z rodziny dusicieli}
\Clue{16}{}{element przeciwstawny czemuś, uzupełniający coś na zasadzie kontrastu}
\Clue{17}{}{przenośnie cecha czegoś ryzykownego, co może prowadzić do kłopotów}
\Clue{18}{}{zazwyczaj biaława nieprzezroczysta zawiesina}
\Clue{19}{}{antygen zewnątrzpochodny, który po wprowadzeniu do ustroju przez uszkodzony nabłonek oddechowy wywołuje reakcję alergiczną; do alergenów wziewnych zaliczamy kurz i wszystko, co wchodzi w jego skład: szczątki roślin, roztocza, szczątki zwierząt)}
\Clue{20}{}{pisarka szwedzka (1900-41); „Kallocain”}
\Clue{23}{}{tyle, ile się zużyło}
\Clue{25}{}{zbiór teorii opisujących ewolucję społeczeństwa, które przechodzi przez poszczególne etapy podczas rozwoju społecznego}
\Clue{26}{}{najwyższy obok dalajlamy dostojnik duchowny w Tybecie uważany za wcielenie Buddy}
\Clue{27}{}{zapał - chęć do działania}
\Clue{29}{}{popularna czeska zakąska, kiełbasa w marynacie}
\Clue{30}{}{górne wykończenie części ubrania zakrywającej tułów, w odzieży damskiej często wycięte, ozdobione}
\Clue{31}{}{skóroskrzydłe; rząd ssaków z fałdami skórnymi umożliwiającymi lit np. lotokot}\end{PuzzleClues}\newpage\section*{Krzyżówka 12}

\noindent\begin{Puzzle}{19}{33}|*	|*	|*	|*	|*	|*	|*	|*	|*	|*	|[1][S]\drarr	|p	|e	|p	|i	|t	|a	|*	|*	|*	|.
|*	|*	|*	|*	|*	|*	|*	|*	|[2][S]\rarr	|k	|o	|t	|w	|i	|c	|a	|*	|*	|*	|*	|.
|[3][S]\drarr	|h	|e	|r	|b	|a	|t	|a	|[][,]{ }	|e	|k	|s	|p	|r	|e	|s	|o	|w	|a	|*	|.
|w	|*	|*	|*	|[4][S]\drarr	|s	|a	|m	|o	|l	|u	|b	|n	|y	|[][,]{ }	|d	|n	|a	|*	|*	|.
|i	|*	|*	|*	|w	|*	|*	|*	|*	|*	|l	|*	|*	|*	|*	|*	|*	|*	|*	|*	|.
|c	|*	|[5][S]\drarr	|c	|y	|r	|k	|u	|m	|s	|t	|a	|n	|c	|j	|a	|*	|*	|*	|*	|.
|e	|*	|b	|*	|m	|*	|*	|[6][S]\darr	|[7][S]\rarr	|k	|a	|t	|a	|m	|n	|e	|z	|a	|*	|*	|.
|p	|*	|ł	|*	|ó	|*	|*	|k	|*	|*	|c	|*	|*	|*	|[8][S]\darr	|*	|*	|*	|*	|*	|.
|r	|*	|y	|*	|g	|*	|*	|o	|[9][S]\darr	|*	|j	|*	|*	|*	|p	|*	|[10][S]\darr	|*	|*	|*	|.
|e	|*	|s	|*	|*	|*	|*	|n	|g	|[11][S]\drarr	|a	|g	|r	|e	|s	|j	|a	|*	|*	|*	|.
|z	|*	|z	|[12][S]\drarr	|f	|e	|s	|t	|o	|n	|*	|*	|*	|*	|z	|*	|n	|[13][S]\darr	|*	|*	|.
|e	|*	|c	|b	|*	|*	|*	|y	|f	|i	|*	|*	|*	|*	|c	|*	|o	|t	|*	|*	|.
|s	|*	|z	|a	|*	|*	|*	|n	|r	|e	|[14][S]\darr	|*	|*	|*	|z	|[15][S]\darr	|m	|r	|*	|*	|.
|*	|*	|y	|l	|*	|*	|[16][S]\darr	|g	|a	|p	|i	|*	|*	|*	|o	|c	|o	|a	|*	|*	|.
|*	|*	|k	|e	|*	|*	|g	|e	|*	|a	|m	|*	|*	|*	|ł	|y	|d	|w	|*	|*	|.
|*	|*	|[][,]{ }	|t	|*	|*	|r	|n	|*	|m	|p	|*	|*	|*	|a	|n	|o	|n	|*	|*	|.
|*	|*	|p	|*	|*	|*	|a	|t	|*	|i	|a	|*	|*	|*	|[][,]{ }	|a	|n	|i	|*	|*	|.
|*	|*	|r	|*	|*	|*	|n	|*	|*	|ę	|k	|*	|*	|*	|k	|m	|t	|k	|*	|*	|.
|*	|*	|z	|*	|*	|*	|a	|*	|*	|ć	|t	|*	|*	|[17][S]\darr	|a	|o	|y	|o	|*	|*	|.
|*	|*	|e	|*	|*	|*	|d	|*	|*	|[][,]{ }	|o	|*	|*	|a	|r	|n	|*	|w	|*	|*	|.
|*	|*	|z	|*	|*	|*	|i	|*	|*	|w	|w	|*	|*	|n	|ł	|[][,]{ }	|*	|i	|*	|*	|.
|*	|*	|r	|*	|[18][S]\rarr	|e	|l	|o	|p	|s	|o	|p	|o	|d	|o	|b	|n	|e	|*	|*	|.
|*	|*	|o	|*	|*	|*	|l	|*	|*	|t	|ś	|*	|*	|e	|w	|i	|*	|c	|*	|*	|.
|*	|*	|c	|*	|*	|*	|a	|*	|*	|e	|ć	|*	|[19][S]\darr	|r	|a	|a	|*	|[][,]{ }	|*	|*	|.
|*	|*	|z	|*	|*	|*	|*	|*	|*	|c	|*	|*	|b	|s	|t	|ł	|*	|p	|*	|*	|.
|*	|*	|y	|[20][S]\rarr	|ł	|u	|s	|z	|c	|z	|y	|n	|a	|*	|a	|y	|*	|o	|*	|*	|.
|*	|*	|s	|*	|*	|*	|*	|[21][S]\rarr	|a	|n	|t	|e	|n	|a	|*	|*	|*	|s	|*	|*	|.
|*	|*	|t	|*	|*	|[22][S]\rarr	|s	|i	|ł	|a	|[][,]{ }	|n	|o	|ś	|n	|a	|*	|p	|*	|*	|.
|*	|*	|y	|*	|*	|*	|*	|*	|*	|*	|*	|*	|w	|*	|*	|*	|*	|o	|*	|*	|.
|*	|*	|*	|*	|*	|*	|*	|*	|*	|*	|*	|*	|i	|*	|*	|*	|*	|l	|*	|*	|.
|*	|*	|*	|*	|*	|*	|[23][S]\rarr	|t	|r	|o	|c	|i	|n	|i	|a	|r	|k	|i	|*	|*	|.
|*	|*	|*	|*	|*	|*	|*	|*	|*	|*	|*	|*	|a	|*	|*	|*	|*	|t	|*	|*	|.
|*	|*	|*	|*	|*	|*	|*	|*	|*	|*	|*	|*	|*	|*	|*	|*	|*	|y	|*	|*	|.
|*	|*	|*	|*	|*	|*	|*	|*	|*	|*	|*	|*	|*	|*	|*	|*	|*	|*	|*	|*	|.\end{Puzzle}

\newpage

\begin{PuzzleClues}{\textbf{Poziome}\\}\Clue{1}{}{deseń na tkaninie w drobną drukowaną kratkę; tkanina o takim deseniu}
\Clue{2}{}{konstrukcja stalowa do zmniejszenia prędkości bądź utrzymania statku w miejscu; rzucona na linie lub łańcuchu}
\Clue{3}{}{herbata w małych papierowych torebkach, jedna saszetka zazwyczaj pozwala na przygotowanie jednej szklanki naparu}
\Clue{4}{}{jednostka DNA, która stanowiąc część genomu nie przynosi korzyści organizmowi, jak ma to miejsce w przypadku większości genów wpływających na fenotyp}
\Clue{5}{}{warunki, okoliczności jakiejś sytuacji}
\Clue{7}{}{obserwacja chorego po zakończeniu leczenia}
\Clue{11}{}{intencjonalne i ukierunkowane zachowanie dążące do wyrządzenia krzywdy}
\Clue{12}{}{girlanda kwiatów, roślin lub owoców swobodnie zwieszająca się z dwóch punktów zaczepienia}
\Clue{18}{}{Elopomorpha - nadrząd ryb promieniopłetwych (Actinopterygii) z infragromady doskonałokostnych (Teleostei), obejmujący rzędy ryb charakteryzujących się larwą typu leptocefala oraz wydłużonym ciałem osobników dorosłych; są to głównie ryby morskie, w tym głębinowe, lub wpływające do estuariów}
\Clue{20}{}{typ suchego owocu z przegrodą w środku, pęka wzdłuż szwów}
\Clue{21}{}{element urządzenia radiotechnicznego do wypromieniowywania lub odbierania energii elektromagnetycznej w postaci fal radiowych}
\Clue{22}{}{siła działająca na ciało poruszające się w płynie (gazie lub cieczy), prostopadła do kierunku ruchu}
\Clue{23}{}{trociniarkowate, Cossidae, - rodzina motyli, zaliczana do dużych motyli nocnych; obejmuje ok. 670 gatunków; cechą charakterystyczną rodziny jest uwsteczniony narząd gębowy}\end{PuzzleClues}

\begin{PuzzleClues}{\textbf{Pionowe}\\}\Clue{1}{}{Zjawisko zasłaniania gwiazd przez ciała wchodzące w skład Układu Słonecznego}
\Clue{3}{}{zastępca prezesa, szefa, zwierzchnika w instytucji, organizacji, firmie}
\Clue{4}{}{warunek stawiany przy zawieraniu jakiejś umowy}
\Clue{5}{}{Moenkhausia collettii - ryba należąca do kąsaczowatych}
\Clue{6}{}{określona liczba żołnierzy, jaką należy wysłać w dane miejsce}
\Clue{8}{}{pszczoła wschodnioazjatycka, Apis florea - gatunek z rodzaju Apis (pszczoła), występujący w Azji Południowej, w strefie klimatu tropikalnego}
\Clue{9}{}{inna nazwa kory - karbowanego bawełnianego materiału}
\Clue{10}{}{Anomodontia - podrząd gadów ssakokształtnych z rzędu terapsydów; żyły od środkowego permu do późnego triasu na wszystkich kontynentach z wyjątkiem Australii}
\Clue{11}{}{utrata pamięci w odniesieniu do zdarzeń pochodzących sprzed momentu zaistnienia czynnika, który tę utratę spowodował}
\Clue{12}{}{utwór muzyczny napisany specjalnie dla widowiska baletowego}
\Clue{13}{}{Zora spinimana - gatunek pająka z rodziny trawnikowcowatych}
\Clue{14}{}{wywieranie wpływu, osiąganie wymiernego (zwykle marketingowego) celu}
\Clue{15}{}{przyprawa wytwarzana z wysuszonej kory korzybiela białego}
\Clue{16}{}{jadalny owoc (jagoda) męczennicy języczkowatej}
\Clue{17}{}{pisarz niemiecki (1906-70), powieści, opowiadania, poezje; „Huśtawka miłości”, „Pokusa Synezjosa”, „Zaślubiny wrogów”}
\Clue{19}{}{kraina, okręg, teren znajdujący się pod władzą bana}\end{PuzzleClues}\newpage\section*{Krzyżówka 13}

\noindent\begin{Puzzle}{20}{31}|*	|*	|*	|[1][S]\drarr	|c	|h	|l	|e	|b	|o	|w	|i	|e	|c	|*	|*	|*	|*	|*	|*	|*	|.
|*	|*	|*	|o	|*	|*	|*	|[2][S]\drarr	|m	|u	|ł	|*	|[3][S]\drarr	|s	|a	|r	|g	|a	|s	|*	|*	|.
|*	|*	|*	|b	|[4][S]\rarr	|s	|u	|t	|e	|n	|e	|r	|s	|t	|w	|o	|*	|[5][S]\darr	|*	|*	|*	|.
|*	|*	|*	|w	|[6][S]\darr	|[7][S]\rarr	|a	|a	|l	|e	|n	|*	|l	|*	|*	|*	|[8][S]\darr	|c	|[9][S]\darr	|*	|*	|.
|*	|*	|*	|i	|f	|[10][S]\drarr	|p	|r	|z	|e	|s	|ł	|u	|c	|h	|a	|n	|i	|e	|*	|*	|.
|*	|*	|[11][S]\darr	|n	|o	|p	|*	|g	|[12][S]\darr	|[13][S]\rarr	|h	|o	|p	|a	|k	|*	|a	|ą	|n	|*	|*	|.
|*	|*	|e	|i	|r	|i	|*	|i	|k	|*	|*	|*	|[][,]{ }	|[14][S]\darr	|[15][S]\darr	|*	|d	|g	|c	|*	|*	|.
|*	|*	|m	|o	|t	|ó	|*	|*	|a	|*	|*	|*	|w	|s	|c	|*	|ż	|n	|y	|*	|*	|.
|[16][S]\drarr	|z	|a	|n	|u	|r	|z	|e	|n	|i	|e	|*	|o	|r	|y	|*	|d	|i	|k	|*	|*	|.
|k	|[17][S]\rarr	|c	|y	|n	|a	|m	|o	|n	|k	|a	|*	|j	|e	|r	|*	|*	|k	|l	|*	|*	|.
|o	|*	|s	|*	|a	|*	|[18][S]\rarr	|s	|a	|m	|o	|j	|e	|b	|k	|a	|*	|[][,]{ }	|o	|*	|*	|.
|t	|*	|*	|*	|*	|*	|*	|*	|*	|*	|[19][S]\darr	|*	|n	|r	|u	|[20][S]\darr	|[21][S]\darr	|z	|p	|*	|*	|.
|*	|*	|*	|[22][S]\rarr	|c	|h	|ł	|o	|p	|a	|k	|*	|n	|o	|ł	|m	|p	|r	|e	|*	|*	|.
|[23][S]\drarr	|ż	|u	|b	|r	|[][,]{ }	|s	|t	|e	|p	|o	|w	|y	|*	|*	|e	|a	|y	|d	|[24][S]\darr	|*	|.
|i	|*	|*	|*	|[25][S]\drarr	|c	|z	|w	|ó	|r	|k	|a	|*	|[26][S]\darr	|[27][S]\darr	|t	|ł	|w	|y	|c	|*	|.
|g	|[28][S]\drarr	|p	|i	|s	|t	|o	|l	|e	|t	|*	|*	|*	|n	|r	|o	|k	|k	|z	|i	|*	|.
|ł	|ś	|[29][S]\rarr	|c	|u	|c	|l	|i	|n	|*	|*	|[30][S]\darr	|*	|y	|a	|d	|a	|o	|m	|e	|*	|.
|a	|p	|[31][S]\drarr	|z	|ł	|o	|t	|a	|[][,]{ }	|f	|u	|n	|k	|c	|j	|a	|*	|w	|*	|s	|*	|.
|*	|i	|p	|[32][S]\rarr	|k	|u	|g	|u	|a	|r	|*	|e	|*	|z	|a	|[][,]{ }	|*	|y	|*	|z	|*	|.
|[33][S]\drarr	|k	|u	|l	|o	|n	|*	|*	|*	|*	|*	|u	|[34][S]\darr	|*	|[][,]{ }	|k	|*	|*	|[35][S]\darr	|y	|*	|.
|b	|*	|s	|[36][S]\rarr	|w	|i	|l	|k	|ó	|w	|*	|s	|m	|[37][S]\drarr	|m	|a	|j	|s	|a	|n	|*	|.
|o	|[38][S]\rarr	|a	|r	|i	|s	|z	|*	|*	|[39][S]\darr	|[40][S]\drarr	|t	|o	|w	|a	|r	|*	|*	|z	|i	|*	|.
|s	|*	|n	|[41][S]\rarr	|c	|i	|ą	|g	|*	|s	|d	|o	|r	|o	|l	|t	|*	|*	|y	|a	|*	|.
|m	|*	|*	|[42][S]\rarr	|z	|e	|u	|s	|*	|y	|m	|n	|s	|s	|t	|o	|*	|*	|l	|n	|*	|.
|a	|[43][S]\rarr	|p	|r	|a	|o	|c	|e	|a	|n	|*	|*	|z	|c	|a	|g	|*	|*	|*	|k	|*	|.
|n	|[44][S]\rarr	|s	|a	|n	|h	|i	|t	|a	|*	|[45][S]\darr	|*	|c	|h	|ń	|r	|*	|*	|*	|a	|*	|.
|m	|*	|[46][S]\drarr	|k	|i	|l	|o	|g	|r	|a	|m	|*	|z	|o	|s	|a	|*	|*	|*	|*	|*	|.
|a	|[47][S]\drarr	|b	|o	|n	|g	|o	|s	|y	|*	|a	|*	|y	|d	|k	|m	|*	|*	|*	|*	|*	|.
|t	|s	|o	|*	|*	|*	|[48][S]\rarr	|s	|z	|o	|g	|u	|n	|*	|a	|u	|*	|*	|*	|*	|*	|.
|*	|o	|y	|*	|*	|[49][S]\rarr	|f	|u	|l	|t	|o	|n	|*	|*	|*	|*	|*	|*	|*	|*	|*	|.
|*	|k	|*	|*	|*	|*	|*	|[50][S]\rarr	|f	|i	|t	|o	|c	|y	|d	|*	|*	|*	|*	|*	|*	|.
|*	|*	|*	|*	|*	|*	|*	|*	|*	|*	|*	|*	|*	|*	|*	|*	|*	|*	|*	|*	|*	|.\end{Puzzle}

\newpage

\begin{PuzzleClues}{\textbf{Poziome}\\}\Clue{1}{}{spotykana czasem nazwa słabo przyswojonego na polskim rynku owocu chlebowca}
\Clue{2}{}{zwierzę z rodziny koni; pochodzi ze skrzyżowania osła z koniem}
\Clue{3}{}{glon morski z typu brunatnic, występuje w postaci zwartych ławic}
\Clue{4}{}{czerpanie korzyści majątkowych z uprawiania prostytucji przez inną osobę; zwykle jest powiązane ze stręczycielstwem (nakłanianiem do uprawiania prostytucji) i kuplerstwem (ułatwianiem uprawiania prostytucji), a czasami z innymi przestępstwami, jak handel ludźmi oraz stosowanie gróźb i przemocy wobec prostytutek}
\Clue{7}{}{miasto w Niemczech (Badenia-Wirtembergia) węzeł kolejowy}
\Clue{10}{}{wysłuchanie czegoś od początku do końca, zwłaszcza - muzyki, utworu czy koncertu}
\Clue{13}{}{ukraiński taniec ludowy utrzymany w metrum 2/4}
\Clue{16}{}{odległość od powierzchni wody do najgłębiej zanurzonej części kadłuba}
\Clue{17}{}{Anas cyanoptera - gatunek ptaka z rodziny kaczkowatych (Anatidae); zamiekszuje błota Ameryki Północnej, występuje także w środkowej i południowej części Ameryki Południowej}
\Clue{18}{}{bramka samobójcza, strzał, który zawodnik oddaje celnie do bramki własnej drużyny}
\Clue{22}{}{syn}
\Clue{23}{}{żubr pierwotny, prażubr, Bison priscus - wymarły gatunek dużego ssaka krętorogiego zamieszkujący w plejstocenie Europę i Azję Środkową, Beringię i Amerykę Północną}
\Clue{25}{}{w niektórych konkurencjach sportowych: czteroosobowa osada}
\Clue{28}{}{przysiad na jednej nodze, z drugą nogą wyprostowana ku przodowi, np. w jeździe na łyżwach}
\Clue{29}{}{kompozytor rumuński (1885-1978); opery, oratorium, utwory symfoniczne, kameralne, chóralne}
\Clue{31}{}{w matematyce, funkcja zmiennej rzeczywistej, której wykresem w kartezjańskim układzie współrzędnych XY jest górna gałąź określonej hiperboli}
\Clue{32}{}{PUMA}
\Clue{33}{}{ptak z rzędu mew-siewek, żywi się owadami i mięczakami: Eurazja, Afryka; chroniony}
\Clue{36}{}{wieś w Polsce położona w województwie lubelskim, w powiecie opolskim, w gminie Wilków, w Małopolskim Przełomie Wisły}
\Clue{37}{}{muhafaza we wsch. Iraku, główne miasto Amara Majsur-Karnataka}
\Clue{38}{}{miasto w Egipcie na Półwyspie Synaj, port nad Morzem Śródziemnym ośr. Handlowy}
\Clue{40}{}{wytwór człowieka, często będący przedmiotem oferty handlowej; najczęściej występuje jako rzeczownik zbiorowy}
\Clue{41}{}{ciągnięcie}
\Clue{42}{}{drapieżna ryba z jesiotrowatych o silnie spłaszczonym ciele występująca przy europejskich wybrzeżach Atlantyku}
\Clue{43}{}{pierwotny ocean, istniejący na początku dziejów Ziemi}
\Clue{44}{}{podstawowy tekst każdej z Wed (hymny, modlitwy)}
\Clue{46}{}{jednostka masy lub siły równa 1000 gramom}
\Clue{47}{}{instrument kubański składający się z dwóch na stałe połączonych ze sobą bębenków o jednakowej wysokości korpusu i różnych średnicach; korpus oryginalnych bongosów wykonany jest z klepek drewnianych, a jednostronnie naciągniętą membranę stanowi skóra kozła}
\Clue{48}{}{dziedziczny tytuł japońskich naczelnych wodzów sprawujących faktyczną władzę w kraju nominalnie rządzonym przez cesarzy}
\Clue{49}{}{(1765-1815) amerykański budowniczy statków wodnych}
\Clue{50}{}{fitoncyd - substancja roślinna obecna np. w cebuli, czosnku, porze, chrzanie, rzodkwi, wykazująca działanie bakteriobójcze, grzybobójcze lub pierwotniakobójcze  lub hamująca procesy życiowe drobnoustrojów (np. allicyna)}\end{PuzzleClues}

\begin{PuzzleClues}{\textbf{Pionowe}\\}\Clue{1}{}{(wg kodeksu postępowania w sprawach o wykroczenia) osoba, przeciwko której wniesiono wniosek o ukaranie w sprawie o wykroczenie}
\Clue{2}{}{duża impreza, na której firmy prezentują swoją ofertę}
\Clue{3}{}{Slup (slup wojenny; ang. sloop-of-war) - klasa niewielkich okrętów, początkowo żaglowych, później o napędzie śrubowym, obecnie nieistniejąca. W okresie obu wojen światowych slupy stanowiły klasę okrętów eskortowych średniej wielkości}
\Clue{5}{}{ciągnik dostawczy do prac przy zrywce drewna}
\Clue{6}{}{duża ilość dóbr materialnych, czyjaś własność}
\Clue{8}{}{kraina w środkowej części Arabii Saudyjskiej, powierzchnia około 1,4 min km}
\Clue{9}{}{postępowy ruch umysłowy w dawnej Francji dążący m.in. do emancypacji człowieka}
\Clue{10}{}{szczecina dzika w gwarze łowieckiej}
\Clue{11}{}{zaawansowany, udostępniany darmowo edytor tekstu}
\Clue{12}{}{eland, Taurotragus oryx - największa współczesna antylopa z rodziny krętorogich; zamieszkuje głównie rezerwaty w Afryce na południe od Sahary}
\Clue{14}{}{nagroda przyznawana za zdobycie drugiego miejsca w zawodach, zwłaszcza sportowych}
\Clue{15}{}{komisariat policji; nazwa stosowana szczególnie w byłym zaborze rosyjskim}
\Clue{16}{}{żartobliwa nazwa kulki umieszczonej w niektórych klawiaturach pod najniższym rzędem klawiszy (która może też być osobnym urządzeniem) - pełni ona podobne funkcje jak myszka}
\Clue{19}{}{hodowany gołąb o jaskrawym upierzeniu}
\Clue{20}{}{przedstawienie za pomocą barwnego wykresu lub szrafu intensywności zjawiska (np. stopy bezrobocia, przyrostu naturalnego, gęstości zaludnienia) na danym obszarze (w rozbiciu na państwa, województwa, gminy itp.)}
\Clue{21}{}{mięso z nóżki drobiowej, samo udko z kostką}
\Clue{23}{}{iglica - część urządzenia (np. igła cyrkla, igła od wrzeciona)}
\Clue{24}{}{Hacquetia - rodzaj rośliny z rodziny selerowatych}
\Clue{25}{}{mieszkaniec Sułkowic}
\Clue{26}{}{pilot polski; mistrz świata w lataniu precyzyjnym}
\Clue{27}{}{Leucoraja melitensis - gatunek ryby chrzęstnoszkieletowej z rodziny rajowatych (Rajidae); raja maltańska żyje wokół Malty i u wybrzeży Tunezji i Algierii na głębokości 60-250 m}
\Clue{28}{}{uczucie senności, bycie sennym}
\Clue{30}{}{zespół organizmów, których środowiskiem życia jest błonka powierzchniowa wody (granica wody i atmosfery)}
\Clue{31}{}{miasto i główny port Korei Płd. nad Cieśniną Koreańska}
\Clue{33}{}{stopień podoficerski w polskiej marynarce wojennej odpowiadający stopniowi plutonowego}
\Clue{34}{}{FUKUS}
\Clue{35}{}{miejsce, w którym otacza się bezpłatną opieką ludzi w trudnej sytuacji życiowej, bezdomne zwięrzęta}
\Clue{37}{}{seria radzieckich trzyosobowych statków kosmicznych}
\Clue{39}{}{określenie mężczyzny ze względu na stosunku do jego rodziców}
\Clue{40}{}{choroba genetyczna, postać dystrofii (zaniku) mięśni}
\Clue{45}{}{makak magot, Macaca sylvanus - gatunek małpy wąskonosej z rodziny makakowatych; żyje w Afryce Północnej, niewielka kolonia jest sztucznie utrzymywana na Gibraltarze}
\Clue{46}{}{pseudonim Tadeusza Kamila Marcjana Żeleńskiego - polskiego tłumacza literatury francuskiej, krytyka literackiego i teatralnego, pisarza, poety-satyryka, kronikarza, eseisty, działacza społecznego, z wykształcenia lekarza}
\Clue{47}{}{umundurowana i uzbrojona formacja działająca na podstawie ustawy o transporcie kolejowym (Dz. U. z 2003 r. Nr 86, poz. 789)}\end{PuzzleClues}\newpage\section*{Krzyżówka 14}

\noindent\begin{Puzzle}{20}{32}|*	|*	|*	|*	|*	|*	|*	|*	|*	|*	|*	|*	|*	|[1][S]\drarr	|o	|ł	|t	|a	|r	|z	|*	|.
|*	|*	|*	|*	|*	|*	|*	|[2][S]\rarr	|d	|e	|p	|o	|r	|t	|a	|c	|j	|a	|*	|[3][S]\darr	|[4][S]\darr	|.
|*	|*	|*	|*	|*	|*	|*	|*	|*	|*	|*	|[5][S]\drarr	|k	|l	|o	|n	|e	|k	|*	|ł	|c	|.
|*	|*	|[6][S]\darr	|*	|[7][S]\rarr	|b	|ó	|r	|[][,]{ }	|b	|a	|g	|i	|e	|n	|n	|y	|*	|*	|a	|h	|.
|*	|[8][S]\darr	|t	|*	|*	|*	|*	|[9][S]\rarr	|k	|a	|m	|i	|o	|n	|k	|a	|*	|*	|*	|b	|o	|.
|[10][S]\drarr	|t	|u	|r	|z	|y	|c	|a	|*	|*	|*	|g	|*	|e	|*	|*	|*	|*	|*	|ę	|r	|.
|b	|u	|r	|*	|[11][S]\drarr	|b	|a	|b	|i	|c	|z	|a	|n	|k	|a	|*	|*	|*	|*	|d	|o	|.
|r	|z	|z	|[12][S]\rarr	|k	|w	|a	|s	|i	|o	|r	|*	|*	|[][,]{ }	|*	|*	|*	|*	|*	|ź	|b	|.
|z	|i	|y	|[13][S]\drarr	|a	|r	|c	|h	|e	|t	|y	|p	|o	|w	|o	|ś	|ć	|*	|*	|[][,]{ }	|a	|.
|e	|n	|c	|s	|r	|[14][S]\drarr	|w	|s	|z	|e	|c	|h	|p	|o	|l	|a	|k	|*	|*	|c	|[][,]{ }	|.
|g	|*	|a	|a	|ł	|ś	|[15][S]\drarr	|b	|u	|g	|a	|j	|*	|d	|*	|*	|[16][S]\darr	|*	|[17][S]\darr	|z	|k	|.
|o	|*	|[][,]{ }	|k	|o	|w	|c	|*	|[18][S]\rarr	|t	|r	|y	|k	|o	|t	|*	|b	|*	|g	|a	|u	|.
|w	|[19][S]\darr	|p	|i	|w	|i	|ó	|*	|*	|[20][S]\drarr	|s	|t	|a	|r	|c	|i	|e	|*	|a	|r	|f	|.
|c	|c	|r	|*	|a	|e	|r	|*	|*	|e	|*	|[21][S]\darr	|*	|u	|[22][S]\darr	|*	|r	|*	|t	|n	|s	|.
|e	|a	|z	|[23][S]\drarr	|t	|r	|e	|p	|a	|k	|*	|r	|*	|*	|o	|[24][S]\darr	|l	|*	|u	|o	|a	|.
|*	|ł	|y	|p	|o	|s	|c	|[25][S]\rarr	|h	|o	|m	|o	|[][,]{ }	|s	|a	|p	|i	|e	|n	|s	|*	|.
|*	|k	|d	|r	|ś	|z	|z	|*	|*	|n	|*	|b	|*	|*	|z	|r	|n	|*	|e	|z	|[26][S]\darr	|.
|*	|a	|y	|a	|ć	|c	|k	|[27][S]\drarr	|m	|o	|g	|o	|ł	|*	|a	|z	|k	|[28][S]\darr	|k	|y	|z	|.
|*	|[][,]{ }	|m	|w	|*	|z	|a	|t	|[29][S]\drarr	|m	|e	|t	|k	|a	|*	|e	|a	|p	|[][,]{ }	|i	|i	|.
|[30][S]\drarr	|p	|i	|o	|n	|*	|*	|u	|m	|i	|*	|y	|*	|*	|[31][S]\darr	|w	|*	|e	|m	|*	|n	|.
|s	|e	|o	|[][,]{ }	|*	|*	|*	|s	|y	|a	|*	|[][,]{ }	|[32][S]\darr	|*	|ż	|r	|*	|j	|i	|[33][S]\darr	|g	|.
|u	|t	|n	|o	|[34][S]\darr	|*	|*	|z	|m	|[][,]{ }	|[35][S]\rarr	|p	|i	|s	|m	|o	|*	|z	|e	|m	|a	|.
|g	|t	|a	|k	|m	|*	|*	|*	|u	|r	|[36][S]\darr	|u	|n	|*	|u	|t	|[37][S]\darr	|a	|s	|i	|n	|.
|e	|i	|*	|u	|y	|*	|*	|*	|r	|o	|l	|b	|k	|*	|j	|n	|f	|n	|z	|e	|a	|.
|s	|s	|*	|n	|s	|*	|[38][S]\drarr	|g	|a	|z	|e	|l	|a	|[][,]{ }	|d	|o	|r	|k	|a	|s	|*	|.
|t	|a	|*	|a	|z	|*	|l	|*	|p	|w	|n	|i	|s	|[39][S]\darr	|z	|ś	|a	|a	|n	|i	|*	|.
|i	|*	|*	|*	|o	|*	|e	|*	|e	|o	|o	|c	|*	|b	|k	|ć	|n	|*	|y	|ą	|*	|.
|a	|*	|*	|*	|w	|*	|w	|*	|l	|j	|i	|z	|*	|ą	|i	|*	|t	|*	|*	|c	|*	|.
|*	|*	|*	|[40][S]\rarr	|a	|w	|a	|n	|t	|u	|r	|n	|i	|k	|*	|*	|i	|*	|*	|*	|*	|.
|*	|[41][S]\rarr	|f	|u	|t	|o	|r	|*	|a	|*	|*	|e	|*	|*	|*	|*	|s	|*	|*	|*	|*	|.
|*	|*	|*	|*	|e	|*	|e	|*	|*	|*	|*	|*	|*	|*	|[42][S]\rarr	|n	|e	|p	|a	|l	|*	|.
|*	|*	|*	|*	|*	|*	|k	|*	|[43][S]\rarr	|m	|ą	|c	|z	|y	|ń	|s	|k	|i	|*	|*	|*	|.
|*	|*	|*	|*	|*	|*	|*	|*	|*	|*	|*	|*	|*	|*	|*	|*	|*	|*	|*	|*	|*	|.\end{Puzzle}

\newpage

\begin{PuzzleClues}{\textbf{Poziome}\\}\Clue{1}{}{miejsce składania ofiar kultowych, główna część kościoła}
\Clue{2}{}{wydalenie cudzoziemca z kraju na podstawie decyzji administracyjnej}
\Clue{5}{}{Leuciscus cephalus - ryba z rodziny karpiowatych}
\Clue{7}{}{siedlisko charakterystyczne dla bezodpływowych niecek, występujące na glebach torfowych i murszowych z wodami gruntowymi na głębokości około 0,5 m; w warstwie drzewostanu dominuje sosna z niewielką domieszką brzozy omszonej lub świerka}
\Clue{9}{}{kuna domowa}
\Clue{10}{}{łow. sierść zająca lub królika}
\Clue{11}{}{mieszkanka Babic}
\Clue{12}{}{coś kwaśnego w smaku}
\Clue{13}{}{to, że coś jest archetypowe, stanowi pierwowzór, najstarszą wersję jakiegoś motywu, postaci, zdarzenia itp}
\Clue{14}{}{członek, działacz Młodzieży Wszechpolskiej}
\Clue{15}{}{skupisko, kępa krzewów lub młodych pędów drzew}
\Clue{18}{}{strój sportowy (body, kąpielówki, koszulka), czasem bielizna z trykotu}
\Clue{20}{}{w sporcie: spotkanie dwóch przeciwnych drużyn w celu rywalizacji; a w niektórych sportach: runda}
\Clue{23}{}{muzyka, do której tańczy się trapaka}
\Clue{25}{}{gatunek ssaka z rodziny człowiekowatych (Hominidae), jedyny występujący współcześnie przedstawiciel rodzaju Homo}
\Clue{27}{}{monarcha noszący tytuł mogoła}
\Clue{29}{}{kartonik, na którym jest napisana cena produktu}
\Clue{30}{}{mezon pi}
\Clue{35}{}{umiejętność wyrażania myśli znakami graficznymi, pisanie}
\Clue{38}{}{Gazella dorcas - ssak z rodziny krętorogich; zamieszkuje Północną Afrykę i Półwysep Arabski}
\Clue{40}{}{osoba, która ma pełne niebezpieczeństw i przygód życie}
\Clue{41}{}{w Rosji: pojedyncza, wiejska zagroda na słabo zaludnionym terenie}
\Clue{42}{}{demokratyczna republika federalna w Azji Południowej, w środkowej części Himalajów, granicząca na północy z Chinami i na południu, wschodzie i zachodzie z Indiami; bez dostępu do morza}
\Clue{43}{}{architekt (1874-1947), stosował formy secesji wiedeńskiej, konserwator kościoła Mariackiego}\end{PuzzleClues}

\begin{PuzzleClues}{\textbf{Pionowe}\\}\Clue{1}{}{woda}
\Clue{3}{}{Cygnus melancoryphus - gatunek ptaka z rodziny kaczkowatych (Anatidae); zamieszkuje południowe obszary Ameryki Południowej, najliczniej występuje w Patagonii i na Ziemi Ognistej}
\Clue{4}{}{rzadka choroba neurologiczna wywoływana, w której złogi lipopigmentu odkładająca się w komórkach nerwowych i tkankach organizmu}
\Clue{5}{}{taniec ludowy Anglii, Irlandii i Szkocji}
\Clue{6}{}{Carex fuliginosa - gatunek byliny z rodziny ciborowatych}
\Clue{8}{}{12 sztuk}
\Clue{10}{}{manatowate, manaty, lamantyny, Trichechidae - rodzina wodnych ssaków łożyskowych z rzędu syren; zamieszkują przybrzeżne rejony zachodniej części Atlantyku, od Florydy po północną Brazylię oraz baseny Amazonki i Orinoko}
\Clue{11}{}{w przenośni: cecha czegoś, co nie posiada wielkiej wartości, jest marne i nijakie}
\Clue{13}{}{miasto na Ukrainie nad jeziorem Saki}
\Clue{14}{}{owad z rzędu prostoskrzydłych}
\Clue{15}{}{córka}
\Clue{16}{}{rodzaj krytego powozu do podróży}
\Clue{17}{}{gatunek, w którym na równych prawach występują cechy dwóch lub trzech rodzajów literackich}
\Clue{19}{}{rozszerzenie pojęcia całki na funkcje o wartościach w przestrzeniach liniowo-topologicznych poprzez sprowadzenie do zagadnienia całkowalności złożeń funkcji z ciągłymi funkcjonałami liniowymi na rozważanej przestrzeni}
\Clue{20}{}{dział ekonomii zajmujący się aspektami gospodarczymi procesu rozwoju w krajach o niskim dochodzie, a w szczególności analizą istoty i przyczyn masowego ubóstwa oraz czynników, kierunków i narzędzi pobudzania rozwoju gospodarczego}
\Clue{21}{}{zatrudnienie bezrobotnego do wykonywania prac zleconych przez samorządy terytorialne, administrację rządową oraz instytucje użyteczności publicznej}
\Clue{22}{}{przen. miejsce wyizolowane z otoczenia}
\Clue{23}{}{makroekonomiczne prawo głoszące, że wraz ze wzrostem bezrobocia przymusowego spada PKB (PNB)}
\Clue{24}{}{na przykład przewrotność czyichś zamiarów}
\Clue{26}{}{gatunek drewna pozyskiwany z afrykańskich drzew z rodziny brezylkowatych Microberlinia brazzavillensis; wykorzystywane w szkutnictwie, meblarstwie i lutnictwie}
\Clue{27}{}{w zapasach: zwycięstwo poprzez położenie przeciwnika na łopatkach}
\Clue{28}{}{zupa jarzynowa na baranich kościach ze słodką kapustą i pomidorami}
\Clue{29}{}{Mymoorapelta - rodzaj dinozaura z grupy ankylozaurów, żyjący w okresie późnej jury na terenach Ameryki Północnej; długość ciała 2,7 m, wysokość 1 m, ciężar 200 kg}
\Clue{30}{}{namowa, propozycja, rada, wpływ na czyjeś przekonania, zachowanie}
\Clue{31}{}{język żmudzki - język należący do grupy języków bałtyckich, starszy dialekt współczesnego języka litewskiego}
\Clue{32}{}{osoba z narodu Inków, przedstawiciel ludu, który w czasach prekolumbijskich stworzył państwo w zachodniej części Ameryki Południowej, w okresie swego największego rozkwitu obejmowało tereny dzisiejszego Peru, Ekwadoru oraz częściowo Boliwii, Chile, Kolumbii i Argentyny}
\Clue{33}{}{właściwy okres obiegu Księżyca wokół Ziemi; miesiąc syderyczny}
\Clue{34}{}{Muridae -  rodzina gryzoni, w skład której wchodzi 140 rodzajów z ok. 650 gatunkami (w Polsce zamieszkuje 8 gatunków); pierwotnie zamieszkiwały Afrykę, Eurazję, Australię, teraz niektóre z myszowatych są rozprzestrzenione na całym świecie}
\Clue{36}{}{Etienne (1822-1900); mechanik francuski zbudował pierwszy użyteczny silnik spalinowy na mieszance gazu ziemnego i powietrza}
\Clue{37}{}{pilot czeski (1917-1940), podczas II wojny światowej w polskim dywizjonie 303, zestrzelił 7 maszyn}
\Clue{38}{}{narzędzie dentystyczne, które służy do usuwania zębów}
\Clue{39}{}{ślepak, duża muchówka napastująca zwierzęta, których krwią żywią się samice}\end{PuzzleClues}\newpage\section*{Krzyżówka 15}

\noindent\begin{Puzzle}{20}{26}|*	|*	|*	|*	|*	|*	|*	|*	|*	|[1][S]\drarr	|d	|i	|a	|b	|l	|i	|c	|a	|*	|*	|*	|.
|*	|*	|*	|*	|*	|[2][S]\drarr	|b	|e	|z	|g	|l	|u	|t	|e	|n	|o	|w	|i	|e	|c	|*	|.
|*	|*	|*	|*	|*	|r	|[3][S]\drarr	|b	|r	|o	|k	|*	|[4][S]\darr	|*	|[5][S]\darr	|[6][S]\darr	|*	|[7][S]\darr	|[8][S]\darr	|*	|*	|.
|*	|*	|[9][S]\darr	|*	|*	|u	|z	|[10][S]\darr	|*	|l	|*	|*	|p	|[11][S]\darr	|s	|n	|[12][S]\darr	|p	|a	|*	|*	|.
|*	|*	|l	|*	|*	|c	|b	|s	|[13][S]\drarr	|k	|l	|i	|o	|c	|h	|i	|n	|o	|l	|*	|*	|.
|*	|*	|a	|*	|*	|h	|i	|o	|f	|i	|*	|*	|z	|e	|a	|t	|i	|s	|g	|*	|*	|.
|*	|*	|n	|*	|*	|*	|ó	|e	|e	|p	|*	|*	|i	|l	|m	|k	|e	|t	|h	|[14][S]\darr	|*	|.
|*	|*	|d	|[15][S]\drarr	|w	|y	|r	|s	|z	|e	|c	|*	|o	|*	|i	|o	|p	|r	|e	|k	|*	|.
|*	|*	|s	|w	|[16][S]\darr	|*	|*	|t	|*	|r	|*	|*	|m	|*	|s	|s	|o	|o	|d	|o	|*	|.
|*	|*	|z	|y	|m	|*	|*	|*	|*	|k	|*	|*	|k	|*	|e	|k	|r	|m	|o	|l	|*	|.
|[17][S]\drarr	|m	|a	|s	|a	|[][,]{ }	|k	|a	|k	|a	|o	|w	|a	|*	|n	|r	|z	|a	|n	|o	|*	|.
|l	|[18][S]\drarr	|f	|o	|r	|s	|z	|u	|s	|*	|*	|*	|[][,]{ }	|*	|*	|z	|ą	|n	|i	|s	|*	|.
|i	|ł	|t	|k	|c	|*	|*	|*	|*	|*	|*	|*	|p	|*	|*	|e	|d	|t	|a	|t	|*	|.
|g	|ó	|*	|o	|o	|*	|[19][S]\drarr	|e	|n	|d	|u	|r	|o	|*	|*	|l	|n	|y	|*	|r	|*	|.
|u	|d	|*	|ś	|w	|*	|m	|*	|*	|[20][S]\rarr	|m	|i	|s	|s	|*	|n	|o	|z	|*	|y	|*	|.
|s	|ź	|*	|ć	|y	|*	|a	|*	|*	|*	|*	|*	|p	|*	|*	|e	|ś	|m	|*	|n	|*	|.
|t	|[][,]{ }	|*	|[][,]{ }	|[][,]{ }	|*	|k	|[21][S]\darr	|*	|[22][S]\rarr	|o	|s	|o	|b	|a	|*	|ć	|*	|*	|i	|*	|.
|r	|p	|*	|w	|d	|*	|r	|l	|*	|[23][S]\rarr	|w	|o	|l	|t	|a	|ż	|*	|*	|*	|n	|*	|.
|*	|o	|*	|z	|o	|*	|a	|i	|*	|[24][S]\rarr	|l	|a	|i	|c	|y	|z	|a	|c	|j	|a	|*	|.
|*	|l	|*	|g	|c	|[25][S]\rarr	|m	|a	|s	|a	|[][,]{ }	|a	|t	|o	|m	|o	|w	|a	|*	|*	|*	|.
|*	|i	|*	|l	|e	|*	|a	|o	|[26][S]\rarr	|p	|o	|p	|a	|r	|z	|e	|n	|i	|e	|*	|*	|.
|*	|c	|*	|ę	|n	|*	|*	|y	|*	|*	|*	|*	|*	|*	|*	|*	|*	|*	|*	|*	|*	|.
|*	|y	|*	|d	|t	|[27][S]\rarr	|s	|u	|b	|s	|t	|a	|n	|c	|j	|a	|l	|i	|z	|m	|*	|.
|*	|j	|*	|n	|*	|[28][S]\rarr	|p	|a	|n	|k	|i	|e	|w	|i	|c	|z	|*	|*	|*	|*	|*	|.
|*	|n	|*	|a	|[29][S]\rarr	|w	|i	|n	|y	|l	|e	|u	|m	|*	|*	|*	|*	|*	|*	|*	|*	|.
|*	|a	|*	|*	|*	|*	|*	|*	|*	|*	|*	|*	|*	|*	|*	|*	|*	|*	|*	|*	|*	|.
|*	|*	|*	|*	|*	|*	|*	|*	|*	|*	|*	|*	|*	|*	|*	|*	|*	|*	|*	|*	|*	|.\end{Puzzle}

\newpage

\begin{PuzzleClues}{\textbf{Poziome}\\}\Clue{1}{}{sprytna kobieta}
\Clue{2}{}{osoba, która nie je produktów zawierających gluten}
\Clue{3}{}{niewielka plamka, będąca zabezpieczniem papieru i druku}
\Clue{13}{}{organiczny związek chemiczny, chloro- i jodopochodna 8-hydroksychinoliny, stosowany jako środek o silnym i długotrwałym działaniu odkażającym i wysuszającym}
\Clue{15}{}{miasto w Bułgarii (okręg Michajłowgrad) na przedgórzu Starej Płaniny}
\Clue{17}{}{substancja powstająca w wyniku zmielenia i wysuszenia prażonych nasion kakaowca, uprzednio odtłuszczonych}
\Clue{18}{}{zaliczka, pieniądze wypłacone przed wykonaną pracą}
\Clue{19}{}{rodzaj motocykla, wykorzystywanego głownie w rajdach enduro}
\Clue{20}{}{zwyciężczyni konkursu piękności}
\Clue{22}{}{istota inteligentna; uosobienie}
\Clue{23}{}{różnica potencjałów elektrycznych między dwoma punktami obwodu elektrycznego lub pola elektrycznego; wyrażone w woltach napięcie elektryczne}
\Clue{24}{}{nadanie charakteru świeckiego; usunięcie wpływu, władzy duchowieństwa}
\Clue{25}{}{masa atomu}
\Clue{26}{}{uszkodzenie skóry i w zależności od stopni oparzenia także głębiej położonych tkanek lub narządów wskutek działania ciepła, żrących substancji chemicznych (stałych, płynnych, gazowych), prądu elektrycznego, promieni słonecznych - UV, promieniowania (RTG, UV i innych ekstremalnych czynników promiennych)}
\Clue{27}{}{pogląd filozoficzny, zgodnie z którym rzeczywiście istnieją tylko obiekty materialne i układy takich obiektów}
\Clue{28}{}{kompozytor i pianista (1857-1898); utwory na fortepian i pieśni}
\Clue{29}{}{tkanina jutowa lub konopna pokryta warstwą zmiękczonego i zabarwionego polichlorku winylu; wykładzina podłóg, stołów laboratoryjnych itp}\end{PuzzleClues}

\begin{PuzzleClues}{\textbf{Pionowe}\\}\Clue{1}{}{członkini drużyny w niektórych sportach zespołowych, której zadaniem jest uniemożliwienie zdobycia bramki (gola) zawodnikom drużyny przeciwnej}
\Clue{2}{}{wzmożone wykonywanie czynności, np. dokonywanie zakupów przez wiele osób w jednym czasie, ożywienie}
\Clue{3}{}{jedno z podstawowych pojęć matematycznych; kolekcja określonych obiektów, mnogość}
\Clue{4}{}{Fragaria vesca - gatunek rośliny z rodziny różowatych}
\Clue{5}{}{japoński instrument muzyczny strunowy szarpany; jest podobnej długości co gitara, ma jednak o wiele cieńszy, bezprogowy gryf i pudło rezonansowe w kształcie zbliżonym do prostokątnego, zaokrąglone, podobne do bębenka}
\Clue{6}{}{podgromada małży}
\Clue{7}{}{neoromantyzm - kierunek w muzyce II połowy XIX wieku (i początkach wieku XX), będący ostatnią fazą romantyzmu}
\Clue{8}{}{przeżywanie rozkoszy podczas doznawania cierpienia fizycznego}
\Clue{9}{}{pogardliwie o obrazie nie mającym wartości artystycznej}
\Clue{10}{}{miasto w środkowej Holandii, 40,4 tys. mieszkańców (1985r.)}
\Clue{11}{}{miejsce przeznaczenia}
\Clue{12}{}{cecha człowieka: brak ucziwości, moralności}
\Clue{13}{}{miasto w Maroku, dawna stolica kraju, ok. 426 tys. mieszkańców}
\Clue{14}{}{naturalnie występująca, uzyskiwana z siary (colostrum) mieszanina bogatych w prolinę polipeptydów}
\Clue{15}{}{pionowa odległość (wysokość) jakiegoś punktu względem punktu odniesienia innego niż poziom morza}
\Clue{16}{}{doktor będący adiunktem na uczelni, który po wydarzeniach z marca 1968 roku został mianowany docentem bez posiadania stopnia doktora habilitowanego}
\Clue{17}{}{eurazjatycki krzew o białych kwiatach i czarnych jagodach, powszechnie stosowany na żywopłoty}
\Clue{18}{}{łódź używana przez policję}
\Clue{19}{}{ażurowa tkanina dekoracyjna wykonana według starej techniki wiązania sznurków bez użycia igły, drutów czy szydełka}
\Clue{21}{}{miasto w Chinach (Jilin); duży ośrodek wydobycia węgla kamiennego, wielka elektrownia cieplna}\end{PuzzleClues}\newpage\section*{Krzyżówka 16}

\noindent\begin{Puzzle}{23}{32}|*	|*	|*	|*	|*	|[1][S]\drarr	|l	|o	|t	|[][,]{ }	|c	|z	|a	|r	|t	|e	|r	|o	|w	|y	|*	|*	|*	|*	|.
|*	|*	|[2][S]\darr	|*	|[3][S]\drarr	|s	|p	|o	|n	|g	|i	|o	|b	|l	|a	|s	|t	|*	|*	|*	|*	|*	|*	|[4][S]\darr	|.
|*	|[5][S]\darr	|j	|[6][S]\drarr	|n	|i	|a	|l	|a	|[][,]{ }	|g	|r	|z	|y	|w	|i	|a	|s	|t	|a	|*	|*	|*	|i	|.
|[7][S]\drarr	|j	|a	|g	|o	|d	|n	|i	|k	|*	|*	|*	|*	|*	|*	|*	|*	|*	|*	|[8][S]\darr	|*	|*	|*	|n	|.
|n	|a	|j	|d	|ż	|l	|*	|*	|*	|*	|*	|*	|*	|[9][S]\drarr	|w	|z	|ó	|r	|*	|o	|[10][S]\darr	|*	|*	|w	|.
|i	|n	|a	|a	|o	|i	|*	|*	|*	|[11][S]\rarr	|r	|o	|s	|t	|r	|y	|*	|*	|*	|c	|r	|*	|*	|e	|.
|e	|k	|*	|ń	|w	|s	|*	|*	|[12][S]\darr	|[13][S]\rarr	|l	|a	|b	|i	|r	|y	|n	|t	|*	|h	|o	|*	|*	|r	|.
|z	|e	|[14][S]\darr	|s	|n	|z	|*	|*	|o	|*	|[15][S]\rarr	|m	|e	|t	|a	|n	|i	|t	|*	|m	|ż	|*	|*	|s	|.
|m	|s	|n	|k	|i	|[][,]{ }	|*	|*	|t	|[16][S]\rarr	|p	|a	|c	|i	|o	|r	|k	|o	|w	|i	|e	|c	|*	|j	|.
|i	|k	|i	|a	|c	|j	|[17][S]\drarr	|w	|r	|ó	|b	|e	|l	|[][,]{ }	|s	|u	|d	|a	|ń	|s	|k	|i	|*	|a	|.
|e	|a	|e	|*	|t	|a	|ł	|[18][S]\darr	|ą	|[19][S]\rarr	|k	|l	|u	|c	|z	|*	|*	|*	|[20][S]\darr	|t	|*	|*	|[21][S]\darr	|[][,]{ }	|.
|n	|*	|p	|*	|w	|s	|a	|f	|b	|*	|[22][S]\rarr	|b	|o	|z	|o	|n	|*	|*	|s	|r	|*	|[23][S]\darr	|g	|t	|.
|n	|[24][S]\darr	|r	|*	|o	|k	|w	|u	|k	|[25][S]\rarr	|k	|o	|m	|e	|n	|d	|a	|*	|t	|z	|[26][S]\darr	|k	|z	|e	|.
|i	|e	|e	|*	|*	|i	|k	|n	|i	|*	|*	|[27][S]\rarr	|g	|r	|a	|n	|i	|c	|a	|*	|z	|u	|y	|r	|.
|k	|l	|c	|[28][S]\darr	|*	|n	|a	|k	|*	|*	|*	|*	|*	|w	|[29][S]\rarr	|w	|o	|d	|n	|i	|a	|k	|*	|m	|.
|*	|a	|y	|s	|*	|i	|*	|t	|*	|*	|[30][S]\rarr	|p	|i	|o	|n	|i	|e	|r	|*	|*	|g	|u	|[31][S]\darr	|i	|.
|[32][S]\drarr	|s	|z	|k	|ł	|o	|[][,]{ }	|o	|ł	|o	|w	|i	|a	|n	|e	|*	|*	|*	|*	|*	|r	|r	|s	|c	|.
|a	|m	|y	|ó	|[33][S]\drarr	|w	|y	|r	|w	|a	|n	|y	|*	|y	|*	|*	|*	|*	|*	|*	|z	|y	|e	|z	|.
|z	|o	|j	|r	|d	|y	|*	|[][,]{ }	|*	|[34][S]\rarr	|t	|o	|n	|*	|*	|*	|*	|*	|*	|*	|e	|d	|r	|n	|.
|b	|z	|n	|a	|*	|*	|[35][S]\rarr	|i	|r	|v	|i	|n	|g	|*	|*	|*	|*	|*	|*	|*	|b	|z	|c	|a	|.
|e	|a	|o	|*	|[36][S]\rarr	|t	|a	|n	|g	|o	|[][,]{ }	|a	|r	|g	|e	|n	|t	|y	|ń	|s	|k	|i	|e	|*	|.
|s	|u	|ś	|[37][S]\rarr	|d	|o	|k	|t	|o	|r	|[][,]{ }	|k	|o	|ś	|c	|i	|o	|ł	|a	|*	|a	|a	|*	|*	|.
|t	|r	|ć	|*	|[38][S]\rarr	|o	|l	|e	|j	|[][,]{ }	|s	|e	|z	|a	|m	|o	|w	|y	|*	|*	|*	|n	|*	|*	|.
|*	|*	|*	|*	|*	|*	|*	|n	|[39][S]\rarr	|o	|s	|i	|o	|ł	|[][,]{ }	|n	|u	|b	|i	|j	|s	|k	|i	|*	|.
|*	|*	|*	|[40][S]\rarr	|f	|a	|ł	|s	|z	|y	|w	|a	|[][,]{ }	|p	|o	|l	|ę	|d	|w	|i	|c	|a	|*	|*	|.
|[41][S]\rarr	|a	|n	|i	|m	|a	|c	|j	|a	|[][,]{ }	|k	|o	|m	|p	|u	|t	|e	|r	|o	|w	|a	|*	|*	|*	|.
|*	|*	|*	|*	|*	|[42][S]\rarr	|k	|o	|n	|f	|e	|k	|c	|j	|a	|*	|*	|*	|*	|[43][S]\darr	|*	|*	|*	|*	|.
|[44][S]\drarr	|k	|i	|e	|r	|o	|w	|n	|i	|k	|[][,]{ }	|p	|o	|c	|i	|ą	|g	|u	|*	|o	|*	|*	|*	|*	|.
|t	|*	|*	|*	|[45][S]\rarr	|p	|t	|a	|s	|z	|n	|i	|k	|[][,]{ }	|t	|y	|g	|r	|y	|s	|i	|*	|*	|*	|.
|a	|[46][S]\rarr	|f	|o	|r	|m	|u	|l	|a	|r	|z	|o	|w	|o	|ś	|ć	|*	|*	|*	|i	|*	|*	|*	|*	|.
|r	|[47][S]\rarr	|t	|o	|p	|o	|r	|n	|i	|c	|a	|[][,]{ }	|m	|a	|r	|m	|u	|r	|k	|o	|w	|a	|*	|*	|.
|o	|*	|*	|*	|[48][S]\rarr	|c	|z	|y	|n	|[][,]{ }	|l	|u	|b	|i	|e	|ż	|n	|y	|*	|ł	|*	|*	|*	|*	|.
|*	|*	|*	|*	|*	|*	|*	|*	|*	|*	|*	|*	|*	|*	|*	|*	|*	|*	|*	|*	|*	|*	|*	|*	|.\end{Puzzle}

\newpage

\begin{PuzzleClues}{\textbf{Poziome}\\}\Clue{1}{}{lot organizowany przez organizację (np. biuro podróży) lub prywatną osobę do wybranego miejsca}
\Clue{3}{}{komórka macierzysta komórki glejowej}
\Clue{6}{}{Tragelaphus angasii - gatunek średniej wielkości antylopy, ssaka parzystokopytnego z rodziny krętorogich; zamieszkuje południowo-wschodnią Afrykę (Malawi, Mozambik, Suazi, Zimbabwe i Republika Południowej Afryki)}
\Clue{7}{}{Paramythia montium - gatunek małego ptaka z rodziny jagodników (Paramythiidae); jedyny przedstawiciel rodzaju Paramythia; jest endemitem wysokich gór Nowej Gwinei}
\Clue{9}{}{rysunek, motyw, ozdoba}
\Clue{11}{}{ażurowy pokład na śródokręciu, na którym ustawione są łodzie ratunkowe}
\Clue{13}{}{część skrzeli niektórych ryb, umożliwia oddychanie powietrzem}
\Clue{15}{}{materiał wybuchowy stosowany w górnictwie}
\Clue{16}{}{modligroszek, Abrus - rodzaj roślin z rodziny bobowatych (Fabaceae)}
\Clue{17}{}{Passer cordofanicus - gatunek ptaka z rodziny wróblowatych (Passeridae)}
\Clue{19}{}{element konstrukcji; często ozdobny szczytowy kliniec łuku lub niektórych typów sklepienia; zwornik}
\Clue{22}{}{cząstka posiadająca spin całkowity}
\Clue{25}{}{budynek, w którym urzędują policjanci}
\Clue{27}{}{linia podziału, która zakreśla pewien obszar}
\Clue{29}{}{Cortinarius - rodzaj grzybów z rodziny zasłonakowatych}
\Clue{30}{}{osadnik, który jest lub był pierwszy na jakimś nowym terenie}
\Clue{32}{}{szkło o zawartości tlenku ołowiu przekraczającej 18\%}
\Clue{33}{}{taniec ludowy z przyśpiewkami charakterystyczny dla Mazowsza}
\Clue{34}{}{inna nazwa tuńczyka pospolitego}
\Clue{35}{}{(1783-1859), amerykański pisarz romantyczny, opowieści fantastyczne, szkice satyryczno-obyczajowe, biografie}
\Clue{36}{}{taniec towarzyski charakteryzujący się improwizacją i bliskością partnerów}
\Clue{37}{}{tytuł przyznawany świętym w sposób szczególny zasłużonym dla rozwoju chrześcijaństwa}
\Clue{38}{}{olej roślinny otrzymywany poprzez tłoczenie na zimno nasion sezamowych}
\Clue{39}{}{Equus asinus africanus - ssak z rodziny koniowatych (Equidae), podgatunek dzikiego osła; występuje w Egipcie i Sudanie}
\Clue{40}{}{rodzaj mięsa wołowego pochodzący z mięśnia utrzymującego kręgosłup w jego części krzyżowej}
\Clue{41}{}{dział grafiki komputerowej i animacji zajmujący się tworzeniem ruchomych obiektów z wykorzystaniem komputerów}
\Clue{42}{}{o lichej, niewiele wartej twórczości artystycznej, która jest przeznaczona do masowej sprzedaży}
\Clue{44}{}{pracownik kolejowy odpowiedzialny za obsługiwany pociąg od czasu jego przyjęcia do czasu jego zdania na danym odcinku drogi}
\Clue{45}{}{Cyclosternum fasciatum - gatunek średniej wielkości pająka z rodziny ptasznikowatych (Aviculariidae)}
\Clue{46}{}{to, że coś ma postać formularza}
\Clue{47}{}{Carnegiella strigata - gatunek ryby słodkowodnej z rodziny pstrążeniowatych (Gasteropelecidae)}
\Clue{48}{}{zachowanie seksualne niebędące obcowaniem płciowym, obecnie określane jakoinna czynność seksualna}\end{PuzzleClues}

\begin{PuzzleClues}{\textbf{Pionowe}\\}\Clue{1}{}{Amaurobius fenestralis - gatunek pająka z rodziny sidliszowatych}
\Clue{2}{}{cecha człowieka twardego, podziwianego, o mocnej osobowości}
\Clue{3}{}{bandytyzm (przy użyciu noża)}
\Clue{4}{}{zjawisko atmosferyczne polegające na wzroście temperatury powietrza wraz z wysokością}
\Clue{5}{}{lekceważąco o obywatelce Stanów Zjednoczonych}
\Clue{6}{}{zatoka Morza Bałtyckiego, u wybrzeży Polski i Federacji Rosyjskiej, główne porty: Gdańsk, Gdynia}
\Clue{7}{}{coś stałego, nie ulegającego zmianom w czasie}
\Clue{8}{}{oficer administracyjno-gospodarczy na statku}
\Clue{9}{}{Callicebus moloch - gatunek małpy szerokonosej; zamieszkuje północną Amerykę Południową, zasiedlając lasy w pobliżu bagien i niewielkich zbiorników wodnych}
\Clue{10}{}{rodzaj pościeli i ubrania dla noworodków, kawałek bawełny lub kołderka, w którą ściśle owija się dziecko w samej pieluszce lub już ubrane}
\Clue{12}{}{otręby}
\Clue{14}{}{brak precyzji, dokładności}
\Clue{17}{}{mebel do siedzenia w miejscach publicznych, element małej architektury}
\Clue{18}{}{funktor zdaniotwórczy, którego wartość zależy nie tylko od wartości logicznej, ale także od treści zdania}
\Clue{20}{}{talia, wcięcie w pasie, kibić}
\Clue{21}{}{rodzina dużych muchówek, których larwy rozwijają się w organizmie niektórych zwierząt parzyste kopytnych}
\Clue{23}{}{mąka (rzadziej: kasza) kukurydziana}
\Clue{24}{}{Elasmosaurus - rodzaj plezjozaurów o wyjątkowo długich szyjach, żyjących w górnej kredzie}
\Clue{26}{}{ślimak z rodziny zagrzebkowatych}
\Clue{28}{}{życie, zdrowie, bezpieczeństwo, własny interes}
\Clue{31}{}{dobre, życzliwe nastawienie względem kogoś}
\Clue{32}{}{minerał o włóknistej budowie używany do wyrobu materiałów ogniotrwałych, izolacyjnych i kwasoodpornych}
\Clue{33}{}{litera oznaczająca wymiar}
\Clue{43}{}{o osobie nierozgarniętej}
\Clue{44}{}{jadalne bulwy kolokazji}\end{PuzzleClues}\newpage\section*{Krzyżówka 17}

\noindent\begin{Puzzle}{21}{24}|*	|*	|*	|*	|*	|*	|*	|*	|*	|*	|*	|*	|*	|*	|*	|[1][S]\darr	|[2][S]\darr	|[3][S]\drarr	|c	|i	|s	|*	|.
|*	|*	|*	|[4][S]\darr	|*	|*	|*	|*	|*	|*	|*	|*	|*	|*	|*	|h	|g	|ł	|*	|[5][S]\darr	|*	|[6][S]\darr	|.
|*	|*	|*	|t	|*	|*	|[7][S]\drarr	|n	|a	|z	|i	|s	|t	|a	|*	|u	|o	|a	|*	|c	|*	|p	|.
|*	|*	|[8][S]\darr	|a	|*	|[9][S]\rarr	|k	|a	|p	|i	|c	|a	|*	|*	|*	|r	|s	|p	|*	|z	|[10][S]\darr	|i	|.
|*	|*	|i	|l	|[11][S]\drarr	|w	|a	|l	|u	|c	|i	|a	|r	|z	|*	|m	|p	|a	|*	|a	|o	|l	|.
|*	|[12][S]\drarr	|m	|i	|g	|*	|k	|[13][S]\drarr	|k	|l	|i	|s	|z	|k	|a	|*	|o	|c	|[14][S]\darr	|s	|p	|o	|.
|*	|o	|a	|a	|r	|*	|a	|s	|*	|*	|*	|*	|[15][S]\darr	|*	|*	|*	|d	|z	|f	|o	|r	|t	|.
|*	|r	|m	|*	|u	|[16][S]\rarr	|d	|o	|d	|a	|t	|e	|k	|[][,]{ }	|s	|m	|a	|k	|o	|w	|y	|*	|.
|*	|d	|*	|[17][S]\darr	|b	|*	|u	|b	|*	|*	|[18][S]\darr	|*	|w	|*	|*	|[19][S]\darr	|r	|a	|t	|n	|s	|*	|.
|*	|a	|*	|e	|i	|*	|*	|ó	|*	|[20][S]\rarr	|a	|l	|a	|g	|o	|a	|s	|*	|e	|i	|z	|*	|.
|*	|*	|[21][S]\darr	|w	|a	|*	|*	|r	|[22][S]\darr	|*	|n	|*	|l	|*	|[23][S]\darr	|n	|t	|*	|l	|k	|c	|*	|.
|*	|[24][S]\drarr	|k	|o	|n	|o	|w	|*	|n	|*	|n	|*	|i	|[25][S]\darr	|m	|a	|w	|*	|[][,]{ }	|[][,]{ }	|z	|*	|.
|*	|w	|u	|l	|i	|[26][S]\rarr	|r	|z	|e	|k	|a	|*	|f	|l	|a	|n	|o	|*	|l	|n	|k	|*	|.
|[27][S]\drarr	|s	|z	|u	|n	|o	|z	|a	|u	|r	|*	|[28][S]\darr	|i	|a	|r	|d	|[][,]{ }	|*	|o	|i	|a	|*	|.
|p	|c	|y	|c	|*	|[29][S]\drarr	|m	|a	|r	|c	|z	|a	|k	|*	|g	|*	|k	|*	|t	|e	|[][,]{ }	|*	|.
|o	|h	|n	|j	|*	|m	|*	|[30][S]\drarr	|o	|p	|e	|r	|a	|*	|r	|[31][S]\darr	|r	|*	|n	|d	|w	|*	|.
|l	|o	|o	|a	|*	|a	|[32][S]\darr	|k	|p	|*	|*	|i	|c	|[33][S]\darr	|a	|m	|a	|*	|i	|o	|a	|*	|.
|y	|d	|s	|*	|*	|l	|f	|r	|e	|*	|*	|a	|j	|c	|b	|u	|j	|*	|c	|k	|r	|*	|.
|p	|e	|t	|*	|*	|a	|r	|o	|p	|*	|*	|*	|a	|z	|i	|z	|o	|*	|z	|o	|g	|*	|.
|e	|k	|w	|*	|[34][S]\rarr	|k	|o	|k	|t	|a	|j	|l	|*	|o	|a	|y	|w	|*	|y	|n	|o	|*	|.
|m	|*	|o	|*	|*	|k	|n	|*	|y	|*	|*	|*	|*	|p	|n	|k	|e	|*	|*	|a	|w	|*	|.
|o	|*	|*	|*	|*	|a	|t	|[35][S]\rarr	|d	|y	|n	|a	|m	|i	|k	|a	|*	|*	|*	|n	|a	|*	|.
|n	|*	|*	|*	|*	|*	|o	|*	|*	|[36][S]\rarr	|b	|a	|r	|k	|a	|*	|*	|*	|*	|y	|*	|*	|.
|*	|*	|*	|*	|[37][S]\rarr	|a	|n	|g	|l	|o	|s	|a	|s	|*	|*	|*	|*	|*	|*	|*	|*	|*	|.
|*	|*	|*	|*	|*	|*	|*	|*	|*	|*	|*	|*	|*	|*	|*	|*	|*	|*	|*	|*	|*	|*	|.\end{Puzzle}

\newpage

\begin{PuzzleClues}{\textbf{Poziome}\\}\Clue{3}{}{dźwięk powstający przez podwyższenie dźwięku c o pół tonu}
\Clue{7}{}{osoba, będąca zwolennikiem ideologii nazistowskiej, zwłaszcza doktryny Narodowosocjalistycznej Niemieckiej Partii Robotników (NSDAP)}
\Clue{9}{}{fizyk radziecki (1894-1984); odkrył nadciekłość helu, nagroda Nobla}
\Clue{11}{}{osoba, która za czasów PRL prowadziła nielegalny obrót walutami, tzn. skupowała i sprzedawała dolary amerykańskie, a także inne waluty wymienialne i bony dolarowe PeKaO}
\Clue{12}{}{najkrótszy stan elektryczny przekazywany łączem telegraficznym}
\Clue{13}{}{błona światłoczuła wykorzystywana w fotografii; płaski, przezroczysty i elastyczny materiał, na którym rejestrowany jest obraz}
\Clue{16}{}{dodatek do żywności poprawiający jej walory smakowe}
\Clue{20}{}{stan w płn-wsch. Brazylii, nad Oceanem Atlantyckim, stolica Maceio, powierzchnia 27,7 tyś. km}
\Clue{24}{}{indianista norweski (1867-1848); prace na temat religii Indii}
\Clue{26}{}{powierzchniowy ciek wodny}
\Clue{27}{}{Shunosaurus - rodzaj zauropoda żyjący w okresie jury na terenach Azji; osiągał 12 metrów długości przy wadze 8 ton}
\Clue{29}{}{łow. młody zając urodzony w marcu}
\Clue{30}{}{instytucja kulturalna zajmująca się wystawianiem oper, zespół ludzi}
\Clue{34}{}{serwowany w specjalny sposób napój alkoholowy; mieszanina trunków z sokami, syropami, owocami, przyprawami, często też z lodem}
\Clue{35}{}{cecha dźwięków, instrumentów oraz kompozycji muzycznych określająca natężenie siły dźwięku}
\Clue{36}{}{śródlądowy statek wodny do przewozu towarów; najczęściej bez własnego napędu}
\Clue{37}{}{współcześnie człowiek wywodzący się z któregoś z narodów pochodzenia anglosaskiego (Brytyjczyk, Amerykanin, Australijczyk, Irlandczyk, Kanadyjczyk, Nowozelandczyk, Afrykaner itp.)}\end{PuzzleClues}

\begin{PuzzleClues}{\textbf{Pionowe}\\}\Clue{1}{}{tłum, chmara}
\Clue{2}{}{wyodrębniony granicami polityczno-ekonomicznymi na obszarze danego kraju organizm gospodarczy, w ramach którego działają podmioty gospodarcze niższego szczebla (przedsiębiorstwa, gospodarstwa domowe), podlegające władzy państwa}
\Clue{3}{}{kobieta, która coś łapie}
\Clue{4}{}{wielokrążek}
\Clue{5}{}{czasownik wyrażający czynność, która nie została ukończona}
\Clue{6}{}{osoba, której zadaniem jest doprowadzenie jakiegoś pojazdu, statku do celu, ułatwienie ruchu drogowego, wodnego lub powietrznego}
\Clue{7}{}{ptak egzotyczny; poszczególne gatunki tego ptaka klasyfikowane są w taksonomii biologicznej w obrębie rodziny kakadu (Cacatuidae)}
\Clue{8}{}{u sunnitów uczony, twórca jednej ze szkół teologicznych}
\Clue{10}{}{Herpes labialis - choroba ujawniająca się w postaci powierzchownych zapalnych pęcherzyków, uformowanych w skupiska, na granicy błony śluzowej jamy ustnej i warg, niekiedy wychodząc poza granicę rąbka czerwieni wargowej}
\Clue{11}{}{człowiek grubiański, taki, który zachowuje się niestosownie, np. mówi niestosowne rzeczy, jest głośny, niegrzeczny i prostacki}
\Clue{12}{}{wojsko tatarskie}
\Clue{13}{}{w prawosławiu i katolicyzmie wschodnim świątynia o szczególnym znaczeniu, wyróżniająca się rozmiarami i znaczeniem dla kultu w danym mieście lub regionie}
\Clue{14}{}{rozkładane siedzenie w samolocie}
\Clue{15}{}{ocena, pozwolenie (np. na dalszy udział w kolejnych etapach rywalizacji), które można uzyskać}
\Clue{17}{}{ciągły proces, polegający na stopniowych zmianach cech gatunkowych kolejnych pokoleń wskutek eliminacji przez dobór naturalny lub sztuczny części osobników (genotypów) z bieżącej populacji}
\Clue{18}{}{pierwszy amerykański satelita geodezyjny wystrzelony w 1962 r}
\Clue{19}{}{ur. 1905r, indyjski pisarz tworzący w języku angielskim i pendżabskim; „Kulis” - Leninowska Nagroda Pokoju}
\Clue{21}{}{stopień pokrewieństwa łączący członków rodziny w linii bocznej}
\Clue{22}{}{peptyd o funkcji mediatorów, syntetyzowany zarówno w neuronach, jak i w innych komórkach organizmu}
\Clue{23}{}{córka margrabiego}
\Clue{24}{}{schodek - każdy ze stopni, z których składają się schody}
\Clue{25}{}{w chemii: symbol lantanu}
\Clue{27}{}{postać z mitologii greckiej, syn Posejdona}
\Clue{28}{}{utwór wokalny głos solowy z akompaniamentem wchodzący w skład opery, operetki, oratorium, kantaty}
\Clue{29}{}{miasto i port w Malezji, nad cieśniną Malakka}
\Clue{30}{}{krocze, część ciała pomiędzy nogami}
\Clue{31}{}{wytwór aktywności artystycznej człowieka, przybierający postać przemyślanych szeregów dźwięków, śpiewanych lub granych na instrumentach; także: sama ta aktywność oraz zbiór utworów muzycznych}
\Clue{32}{}{front budynku, zwłaszcza gdy mowa o budynku zabytkowym z bardzo ozdobną fasadą}
\Clue{33}{}{niewielkich rozmiarów, wypukły, zwykle w kształcie walca element łączący, działający na zasadzie dopasowania kształtu wypustki w jednym elemencie do kształtu otworu w drugim}\end{PuzzleClues}\newpage\section*{Krzyżówka 18}

\noindent\begin{Puzzle}{17}{31}|*	|*	|*	|*	|*	|*	|*	|*	|*	|*	|*	|*	|*	|*	|[1][S]\darr	|*	|*	|*	|.
|*	|*	|[2][S]\drarr	|b	|e	|ł	|c	|h	|a	|t	|o	|w	|i	|a	|n	|i	|n	|*	|.
|[3][S]\drarr	|e	|s	|k	|a	|l	|a	|t	|o	|r	|*	|[4][S]\rarr	|z	|e	|i	|t	|z	|*	|.
|p	|[5][S]\darr	|a	|*	|*	|*	|[6][S]\drarr	|s	|o	|d	|a	|*	|[7][S]\drarr	|s	|e	|r	|m	|*	|.
|a	|d	|m	|*	|*	|*	|s	|*	|*	|*	|[8][S]\darr	|*	|g	|*	|d	|[9][S]\darr	|*	|*	|.
|w	|r	|o	|*	|[10][S]\rarr	|h	|e	|r	|o	|s	|t	|r	|a	|t	|o	|s	|*	|*	|.
|*	|o	|u	|*	|*	|*	|j	|*	|*	|*	|u	|*	|l	|[11][S]\darr	|z	|p	|[12][S]\darr	|[13][S]\darr	|.
|[14][S]\drarr	|g	|r	|y	|b	|o	|s	|z	|*	|*	|r	|*	|a	|k	|ó	|u	|r	|p	|.
|a	|a	|z	|[15][S]\darr	|*	|[16][S]\darr	|m	|[17][S]\rarr	|c	|o	|b	|*	|r	|a	|r	|s	|u	|a	|.
|n	|*	|e	|m	|*	|k	|o	|*	|[18][S]\drarr	|m	|o	|h	|e	|r	|*	|t	|b	|l	|.
|k	|*	|c	|b	|[19][S]\darr	|a	|g	|*	|c	|[20][S]\darr	|t	|*	|t	|m	|*	|*	|e	|c	|.
|s	|*	|z	|i	|m	|b	|r	|*	|h	|i	|*	|*	|a	|a	|*	|[21][S]\darr	|l	|ó	|.
|j	|*	|y	|n	|a	|w	|a	|*	|a	|k	|*	|*	|*	|*	|*	|s	|[][,]{ }	|w	|.
|o	|[22][S]\drarr	|w	|i	|c	|e	|m	|i	|n	|i	|s	|t	|e	|r	|*	|a	|b	|k	|.
|l	|s	|i	|*	|h	|*	|*	|[23][S]\darr	|s	|n	|*	|[24][S]\darr	|*	|[25][S]\darr	|*	|r	|i	|a	|.
|i	|t	|s	|[26][S]\drarr	|a	|z	|y	|d	|o	|t	|y	|m	|i	|d	|y	|n	|a	|*	|.
|t	|r	|t	|g	|j	|[27][S]\darr	|*	|r	|n	|o	|*	|o	|*	|o	|[28][S]\darr	|a	|ł	|*	|.
|y	|y	|n	|r	|r	|r	|[29][S]\darr	|ą	|*	|s	|*	|s	|*	|k	|c	|[][,]{ }	|o	|[30][S]\darr	|.
|k	|c	|i	|y	|o	|e	|p	|ż	|*	|*	|*	|s	|[31][S]\darr	|u	|i	|s	|r	|a	|.
|*	|h	|e	|z	|d	|p	|a	|e	|*	|*	|*	|*	|e	|m	|s	|y	|u	|k	|.
|*	|o	|n	|a	|*	|a	|s	|k	|*	|*	|*	|*	|o	|e	|o	|b	|s	|c	|.
|[32][S]\drarr	|w	|i	|k	|a	|r	|y	|*	|*	|*	|*	|*	|z	|n	|w	|e	|k	|e	|.
|f	|a	|e	|*	|*	|a	|w	|*	|*	|*	|*	|*	|y	|t	|a	|r	|i	|n	|.
|r	|n	|*	|*	|*	|c	|i	|[33][S]\rarr	|k	|a	|b	|a	|n	|*	|t	|y	|*	|c	|.
|a	|i	|*	|*	|*	|j	|z	|[34][S]\rarr	|b	|r	|o	|m	|o	|l	|e	|j	|*	|i	|.
|n	|e	|*	|[35][S]\drarr	|m	|a	|m	|b	|o	|*	|*	|*	|f	|*	|*	|s	|*	|k	|.
|c	|*	|*	|r	|*	|*	|*	|*	|[36][S]\rarr	|w	|a	|r	|i	|a	|t	|k	|a	|*	|.
|u	|*	|*	|a	|[37][S]\rarr	|m	|i	|n	|o	|c	|y	|k	|l	|i	|n	|a	|*	|*	|.
|z	|*	|[38][S]\rarr	|m	|i	|e	|r	|n	|i	|k	|o	|w	|i	|e	|c	|*	|*	|*	|.
|*	|[39][S]\rarr	|j	|u	|g	|s	|*	|*	|*	|*	|[40][S]\rarr	|b	|a	|j	|e	|w	|*	|*	|.
|[41][S]\rarr	|m	|e	|s	|s	|e	|l	|*	|*	|*	|*	|*	|*	|*	|*	|*	|*	|*	|.
|*	|*	|*	|*	|*	|*	|*	|*	|*	|*	|*	|*	|*	|*	|*	|*	|*	|*	|.\end{Puzzle}

\newpage

\begin{PuzzleClues}{\textbf{Poziome}\\}\Clue{2}{}{mieszkaniec Bełchatowa}
\Clue{3}{}{ruchome schody}
\Clue{4}{}{miasto w Niemczech (Saksonia Anhalt) nad Białą Elsterą; w okolicy wydobycie węgla brunatnego}
\Clue{6}{}{zwyczajowa nazwa węglanu sodu}
\Clue{7}{}{selektywny modulator receptora estrogenowego - ang. Selective Estrogen Receptor Modulators, lek należący do grupy leków działających za pośrednictwem receptorów estrogenowych, charakteryzujących się zróżnicowanym działaniem agonistycznym lub antagonistycznym w zależności od tkanki docelowej}
\Clue{10}{}{postać historyczna}
\Clue{14}{}{RUBASZNICA roślina zielna lub krzew z gruboszowatych, wiele gatunków w uprawie doniczkowej}
\Clue{17}{}{krępy koń średniego wzrostu z rasy anglo-normandzkiej}
\Clue{18}{}{rodzaj kudłatej włóczki (włóczka ta nie musi koniecznie być zrobiona z moheru - wełny kóz angorskich)}
\Clue{22}{}{urzędnik, zastępca ministra - kierownika ministerstwa}
\Clue{26}{}{pochodna tymidyny, w której grupa hydroksylowa w pozycji 3' zastąpiona jest grupą azydową; stosowana jako lek antyretrowirusowy, zwłaszcza do zwalczania zakażeń wirusem HIV}
\Clue{32}{}{WIKARIUSZ; w Kościele katolickim: ksiądz będący pomocnikiem proboszcza}
\Clue{33}{}{dzik}
\Clue{34}{}{farba stosowana w fotografice przy uzyskiwaniu obrazów metodą bromoleju}
\Clue{35}{}{taniec towarzyski pochodzący z Kuby, w metrum parzystym}
\Clue{36}{}{kobieta chora psychicznie}
\Clue{37}{}{antybiotyk należący do tetracyklin półsyntetycznych; często używana jest do leczenia trądziku młodzieńczego}
\Clue{38}{}{typ motyla (rodzina miernikowcowatych); nazwa pochodzi od ruchu gąsienic, które wyglądają tak, jakby odmierzały odległości}
\Clue{39}{}{rodzaj prymitywnej tuby}
\Clue{40}{}{biochemik radziecki ur. w 1903 r.; specjalista w dziedzinie biologii molekularnej}
\Clue{41}{}{architekt niemiecki (1853-1909), biurowce, domy mieszkalne i towarowe}\end{PuzzleClues}

\begin{PuzzleClues}{\textbf{Pionowe}\\}\Clue{1}{}{brak dozoru, niedopilnowanie czegoś, kogoś, zaniedbanie dozoru}
\Clue{2}{}{w psychologii: stałe dążenie do realizacji swojego potencjału, rozwijania talentów i możliwości, proces stawania się tym, kim się chce być (a nie tym, kim się jest), dążenie do wewnętrznej spójności, jedności z samym sobą, spełnienia swojego przeznaczenia lub powołania}
\Clue{3}{}{ptak z rodziny kurowatych (Phasianidae)}
\Clue{5}{}{sposób komunikowania się, kanał}
\Clue{6}{}{analogowy lub cyfrowy zapis drgań rejestrowanych przez sejsmometr}
\Clue{7}{}{zastygła, półstała masa powstała przez wygotowanie kości, ryb lub owoców}
\Clue{8}{}{ryba z rzędu płastug zamieszkująca dna mórz europejskich o długości do 1 m}
\Clue{9}{}{element broni palnej: metalowy języczek uruchamiający mechanizm w celu oddania strzału}
\Clue{11}{}{karman - w religii indyjskiej bilans tego, co się uczyniło poczas życia, decydujący o charakterze następnego wcielenia w (reinkarnacji)}
\Clue{12}{}{jednostka monetarna Republiki Białorusi równa 100 kopiejkom, wprowadzona do obiegu w roku 1992 po uzyskaniu przez Białoruś niepodległości}
\Clue{13}{}{pieszczoty wejścia do pochwy, okolic odbytu lub łechtaczki dłonią lub palcami partnera}
\Clue{14}{}{lek przeciwlękowy, lek anksjolityczny - lek psychotropowy, który przez działanie na przekaźnictwo impulsów nerwowych w ośrodkowym układzie nerwowym powoduje działanie zmniejszające lęk, niepokój, napięcie emocjonalne i objawy somatyczne towarzyszące tym stanom}
\Clue{15}{}{dawniej RIO MUNI: prowincja Gwinei Równikowej nad Zatoką Gwinejską, ośrodek administracyjny Bata}
\Clue{16}{}{miasto w środkowej Zambii, ośrodek eksploatacji hutnictwa metali nieżelaznych}
\Clue{18}{}{gatunek wielogłosowej francuskiej muzyki wokalnej}
\Clue{19}{}{eurazjatycki kopalny tygrys szablastozębny}
\Clue{20}{}{architekt z Attyki; Panteon w Atenach}
\Clue{21}{}{sarna azjatycka, Capreolus pygargus - gatunek ssaka parzystokopytnego z rodziny jeleniowatych, blisko spokrewniona z sarną europejską; zamieszkuje Azję północno-wschodnią i centralną (Rosja, Chiny, Kazachstan, Mongolia i Korea) - od zachodu, w górach Kaukazu, graniczy z zasięgiem występowania sarny europejskiej (Capreolus capreolus), sprowadzona do Polski, lecz wyginęła}
\Clue{22}{}{jedna z wad koni, uderzanie podkową jednej nogi o staw pęcinowy sąsiedniej}
\Clue{23}{}{pałąk drewniany lub metalowy, mniejszy od drąga}
\Clue{24}{}{miasto i port w Norwegii nad Oslofjorden, ośrodek administracyjny okręgu Ostfold}
\Clue{25}{}{utwór, który jest świadectwem epoki lub uwiecznieniem jakichś prawdziwych wydarzeń}
\Clue{26}{}{zabawka dziecięca, służąca do gryzienia, szczególnie w okresie ząbkowania}
\Clue{27}{}{w psychoanalizie: jeden z psychicznych mechanizmów obronnych}
\Clue{28}{}{Taxaceae - rodzina drzew i krzewów iglastych z rzędu cyprysowców; obejmuje 6 rodzajów z 28 gatunkami występującymi w Eurazji, północnej Afryce, Ameryce Północnej i Środkowej oraz na Nowej Kaledonii}
\Clue{29}{}{orientacja polityczna, w myśl której należy utrzymywać współpracę i dobre stosunki z Rosją, natomiast unikać nawiązywania tychże z Austrią i Niemcami; nurt popularny szczególnie w Polsce w czasach I wojny światowej}
\Clue{30}{}{zdrobniale: akcent - przen. podkreślenie czegoś, położenie nacisku na coś}
\Clue{31}{}{nadmiar granulocytów kwasochłonnych}
\Clue{32}{}{klucz francuski}
\Clue{35}{}{francuski filozof, logik i matematyk (1515-72); wszechstronny uczony epoki renesansu}\end{PuzzleClues}\newpage\section*{Krzyżówka 19}

\noindent\begin{Puzzle}{17}{31}|*	|*	|*	|[1][S]\darr	|[2][S]\drarr	|p	|r	|o	|s	|t	|o	|w	|ó	|d	|*	|[3][S]\drarr	|s	|*	|.
|*	|*	|*	|s	|k	|*	|[4][S]\drarr	|g	|e	|s	|z	|e	|f	|t	|*	|m	|*	|*	|.
|*	|*	|[5][S]\drarr	|k	|i	|f	|o	|z	|a	|*	|*	|*	|*	|*	|*	|a	|*	|*	|.
|*	|*	|t	|o	|n	|[6][S]\drarr	|d	|i	|a	|g	|n	|o	|s	|t	|y	|k	|*	|*	|.
|*	|*	|r	|k	|a	|k	|w	|*	|[7][S]\darr	|[8][S]\rarr	|k	|o	|l	|o	|n	|i	|a	|*	|.
|*	|*	|e	|*	|*	|a	|a	|[9][S]\rarr	|b	|o	|o	|t	|e	|s	|*	|a	|*	|*	|.
|*	|[10][S]\drarr	|p	|ó	|ł	|p	|r	|z	|e	|s	|t	|r	|z	|e	|ń	|*	|*	|*	|.
|*	|g	|*	|*	|*	|u	|s	|[11][S]\rarr	|l	|n	|i	|s	|k	|o	|*	|*	|[12][S]\darr	|*	|.
|*	|r	|[13][S]\rarr	|e	|k	|s	|t	|r	|a	|d	|y	|c	|j	|a	|*	|*	|r	|*	|.
|*	|i	|[14][S]\darr	|*	|*	|t	|w	|*	|*	|*	|*	|[15][S]\rarr	|m	|l	|e	|k	|o	|*	|.
|*	|l	|d	|*	|*	|a	|i	|*	|[16][S]\drarr	|b	|y	|d	|l	|ę	|*	|*	|z	|*	|.
|*	|l	|e	|*	|[17][S]\darr	|*	|e	|[18][S]\rarr	|m	|e	|n	|a	|c	|h	|a	|*	|s	|*	|.
|*	|*	|s	|[19][S]\drarr	|s	|e	|n	|*	|a	|*	|[20][S]\rarr	|l	|u	|i	|z	|y	|t	|*	|.
|*	|[21][S]\darr	|k	|r	|i	|[22][S]\rarr	|i	|n	|t	|e	|r	|r	|e	|x	|*	|*	|ę	|*	|.
|*	|p	|t	|u	|e	|*	|e	|[23][S]\drarr	|o	|d	|z	|i	|e	|ż	|*	|[24][S]\darr	|p	|[25][S]\darr	|.
|*	|r	|o	|c	|d	|*	|*	|k	|ł	|*	|*	|*	|*	|[26][S]\darr	|*	|ł	|[][,]{ }	|o	|.
|*	|z	|p	|h	|z	|*	|*	|r	|e	|[27][S]\darr	|*	|*	|[28][S]\darr	|p	|*	|a	|k	|k	|.
|*	|y	|[][,]{ }	|[][,]{ }	|i	|[29][S]\drarr	|n	|a	|k	|r	|ę	|t	|k	|a	|*	|ń	|w	|o	|.
|*	|b	|p	|b	|b	|a	|*	|t	|*	|a	|[30][S]\drarr	|d	|o	|n	|i	|c	|a	|*	|.
|*	|y	|u	|e	|a	|k	|*	|e	|*	|f	|t	|*	|d	|i	|*	|u	|r	|*	|.
|*	|w	|b	|z	|*	|w	|*	|r	|[31][S]\darr	|a	|o	|*	|e	|*	|[32][S]\darr	|c	|t	|*	|.
|*	|a	|l	|w	|*	|a	|*	|*	|j	|*	|*	|*	|k	|*	|c	|h	|y	|*	|.
|*	|j	|i	|i	|*	|m	|[33][S]\drarr	|p	|u	|m	|e	|k	|s	|*	|e	|[][,]{ }	|l	|*	|.
|*	|ą	|s	|z	|*	|a	|k	|*	|b	|*	|*	|*	|*	|*	|i	|e	|n	|*	|.
|[34][S]\drarr	|c	|h	|o	|j	|n	|o	|w	|i	|a	|n	|i	|n	|*	|b	|u	|y	|*	|.
|l	|y	|i	|w	|*	|i	|r	|*	|l	|*	|[35][S]\rarr	|o	|d	|w	|a	|l	|*	|*	|.
|i	|*	|n	|y	|*	|l	|z	|[36][S]\rarr	|e	|d	|a	|m	|*	|*	|*	|e	|*	|*	|.
|r	|*	|g	|*	|*	|a	|e	|*	|r	|*	|[37][S]\rarr	|u	|m	|i	|a	|r	|*	|*	|.
|a	|*	|*	|*	|*	|*	|n	|*	|*	|*	|[38][S]\rarr	|d	|ę	|t	|k	|a	|*	|*	|.
|*	|[39][S]\rarr	|m	|i	|ę	|k	|i	|s	|z	|[][,]{ }	|w	|o	|d	|n	|y	|*	|*	|*	|.
|[40][S]\rarr	|h	|a	|n	|o	|w	|e	|r	|*	|*	|*	|*	|*	|*	|*	|*	|*	|*	|.
|*	|*	|[41][S]\rarr	|k	|o	|t	|*	|*	|*	|*	|*	|*	|*	|*	|*	|*	|*	|*	|.\end{Puzzle}

\newpage

\begin{PuzzleClues}{\textbf{Poziome}\\}\Clue{2}{}{urządzenie mechaniczne służące do zamiany ruchu obrotowego na postępowy}
\Clue{3}{}{w chemii: symbol siarki}
\Clue{4}{}{każdy interes, biznes, transakcja}
\Clue{5}{}{łukowate wygięcie kręgosłupa w stronę grzbietową. Jeżeli takie wygięcie znajduje się w odcinku piersiowym lub lędźwiowym kręgosłupa i mieści się w granicach normy (dla kifozy piersiowej to 20 do 40 stopni), to jest to kifoza fizjologiczna (kifoza piersiowa, kifoza krzyżowa)}
\Clue{6}{}{osoba wydająca diagnozę medyczną; w praktyce będącą najczęściej lekarzem}
\Clue{8}{}{terytorium poza jakimś państwem zależne od tego państwa}
\Clue{9}{}{WOLARZ}
\Clue{10}{}{każda z dwóch części trójwymiarowej przestrzeni euklidesowej, na które dzieli tę przestrzeń płaszczyzna, wraz z tą płaszczyzną}
\Clue{11}{}{pole po zbiorze lnu}
\Clue{13}{}{wydanie władzom państwowym osoby przebywającej na terytorium państwa wydającego, podejrzanej o popełnienie przestępstwa na terytorium państwa zwracającego się z wnioskiem o ekstradycję lub w celu odbycia kary}
\Clue{15}{}{porcja mleka (gdy mowa o 1-4 sztukach), zazwyczaj jego opakowanie (butelka, karton), choć może to też być porcja podana w naczyniu, np. zamówiona w lokalu gastronomicznym}
\Clue{16}{}{człowiek podły, godny pogardy, pozbawiony kręgosłupa moralnego, bezmyślny}
\Clue{18}{}{duża menażka, metalowy pojemnik z pokrywką, służący do przenoszenia posiłków}
\Clue{19}{}{stan czynnościowy ośrodkowego układu nerwowego, cyklicznie pojawiający się i przemijający w rytmie okołodobowym, podczas którego następuje przejściowe wyłączenie świadomości i bezruch}
\Clue{20}{}{bojowy środek trujący powodujący oparzenia skóry i atakujący drogi oddechowe}
\Clue{22}{}{osoba sprawująca w Rzymie godność interrexa}
\Clue{23}{}{skóra żubra w gwarze łowieckiej}
\Clue{29}{}{łącznik w połączeniu śrubowym w formie pierścienia z naciętym na całej długości otworu gwintem}
\Clue{30}{}{zawartość donicy, naczynia do ucierania}
\Clue{33}{}{wulkaniczna skała magmowa zbudowana z porowatego szkliwa wulkanicznego powstałego z silnie gazującej, pienistej lawy}
\Clue{34}{}{mieszkaniec Chojnowa}
\Clue{35}{}{usunięta część masy formierskiej i rdzeniarskiej wybitej z form}
\Clue{36}{}{miasto w zach. Holandii nad zbiornikiem Ijesselmeer, ważny ośrodek handlowy rolniczego regionu}
\Clue{37}{}{powściągliwość, brak przesady, cecha osobowości}
\Clue{38}{}{noga, osoba słabo zorientowana w jakiejś dziedzinie}
\Clue{39}{}{tkanka wyspecjalizowana w gromadzeniu i przechowywaniu wody, zbudowana z cienkościennych komórek zawierających duże wakuole z zawartymi wewnątrz substancjami śluzowymi i pektynami, które pęczniejąc pod wpływem wody zatrzymują dużą jej ilość i zmniejszają szybkość jej oddawania}
\Clue{40}{}{formalnie Księstwo Brunszwiku i Lüneburga, potocznie Księstwo Hanoweru lub Księstwo Hanowerskie; państwo wchodzącym w skład Świętego Cesarstwa Rzymskiego}
\Clue{41}{}{w gwarze łowieckiej; zając}\end{PuzzleClues}

\begin{PuzzleClues}{\textbf{Pionowe}\\}\Clue{1}{}{nagła (taka, która nie jest stopniowa) zmiana}
\Clue{2}{}{miasto w środkowej części Egiptu nad Nilem, ośrodek handlu i rzemiosła}
\Clue{3}{}{wiecznie zielone, twardolistne zarośla przybrzeżnej strefy Morza Śródziemnego}
\Clue{4}{}{proces oddzielania się tkanek w sposób warstwowy}
\Clue{5}{}{deska z przybitymi listwami ułatwiająca przechodzenie po pochyłych wyrobiskach górniczych}
\Clue{6}{}{znane warzywo (taka kapusta rośnie w ogródku i można ją kupić pod tą nazwą w zieleniaku)}
\Clue{7}{}{miara ilości papieru (10 ryz po 20 libr)}
\Clue{10}{}{przenośne urządzenie do pieczenia mięsa, ryb i warzyw}
\Clue{12}{}{różnica pomiędzy trzecim i pierwszym kwartylem}
\Clue{14}{}{urządzenia i oprogramowanie umożliwiające desktop publishing}
\Clue{16}{}{ciężko upośledzona intelektualnie osoba, chora z powodu niedoboru jodu - kretynizmu/matołectwa}
\Clue{17}{}{miejsce, w którym znajduje się jakaś instytucja}
\Clue{19}{}{porozumienie między krajami, zezwalające na przekroczenie granicy lub pobyt na terenie obcego państwa bez obowiązku posiadania wizy}
\Clue{21}{}{ten, kto przybywa}
\Clue{23}{}{zagłębienie w szczycie wulkanu}
\Clue{24}{}{ścieżka w grafie, która przechodzi przez każdą jego krawędź dokładnie raz}
\Clue{25}{}{fachowa nazwa dziury w serze}
\Clue{26}{}{właścicielka zwierzęcia}
\Clue{27}{}{struktura podwodna powstała przez nagromadzenie szkieletów organizmów rafotwórczych (m.in. koralowców)}
\Clue{28}{}{jedna z form książki, w której karty są złączone (szyte, klejone) na jednym z brzegów nazywanym grzbietem książki; obecnie jest to najpopularniejsza forma książki}
\Clue{29}{}{ozdobne naczynie do polewania rąk używane między innymi w obrzędach religijnych}
\Clue{30}{}{nazwa zastępująca, ze względu na tabu kulturowe, ogół słownictwa związanego z seksem}
\Clue{31}{}{sprzedawca biżuterii i drobnych przedmiotów ozdobnych z drogich materiałów (zwł. z kamieni lub metali szlachetnych)}
\Clue{32}{}{miasto i port w Hondurasie nad Morzem Karaibskim, ośrodek departamentu Atlantida}
\Clue{33}{}{to, skąd coś lub ktoś się wywodzi}
\Clue{34}{}{Callionymus lyra - gatunek morskiej ryby okoniokształtnej z rodziny lirowatych}\end{PuzzleClues}\newpage\section*{Krzyżówka 20}

\noindent\begin{Puzzle}{25}{16}|*	|*	|*	|[1][S]\drarr	|s	|z	|c	|z	|ę	|k	|o	|t	|*	|*	|*	|*	|*	|*	|*	|*	|*	|*	|*	|*	|*	|*	|.
|*	|*	|*	|m	|[2][S]\drarr	|s	|z	|l	|i	|f	|i	|e	|r	|n	|i	|a	|*	|*	|[3][S]\drarr	|m	|o	|t	|y	|k	|a	|*	|.
|*	|*	|*	|y	|b	|*	|*	|*	|*	|*	|*	|[4][S]\rarr	|s	|z	|c	|z	|e	|r	|b	|o	|*	|*	|*	|*	|*	|*	|.
|*	|*	|*	|s	|a	|*	|*	|*	|*	|*	|*	|*	|*	|*	|[5][S]\darr	|*	|[6][S]\rarr	|h	|o	|l	|m	|e	|s	|*	|*	|*	|.
|*	|*	|*	|z	|r	|*	|[7][S]\drarr	|z	|i	|o	|ł	|a	|*	|*	|b	|[8][S]\rarr	|f	|l	|a	|c	|h	|a	|*	|*	|*	|*	|.
|*	|*	|*	|o	|w	|[9][S]\rarr	|k	|l	|i	|n	|*	|[10][S]\drarr	|i	|k	|a	|r	|*	|[11][S]\rarr	|s	|u	|z	|a	|n	|i	|*	|*	|.
|*	|*	|[12][S]\drarr	|w	|i	|n	|o	|[][,]{ }	|m	|a	|r	|k	|i	|[][,]{ }	|w	|i	|n	|o	|*	|*	|*	|*	|*	|*	|*	|*	|.
|*	|*	|w	|a	|e	|*	|r	|*	|[13][S]\rarr	|c	|z	|a	|c	|h	|ó	|r	|s	|k	|i	|*	|*	|*	|*	|*	|*	|*	|.
|*	|*	|u	|t	|n	|*	|s	|*	|*	|*	|*	|p	|[14][S]\rarr	|p	|ł	|o	|m	|y	|k	|ó	|w	|k	|a	|*	|*	|*	|.
|*	|*	|h	|e	|i	|*	|a	|*	|[15][S]\rarr	|b	|r	|i	|e	|f	|*	|*	|*	|*	|*	|*	|*	|*	|*	|*	|*	|*	|.
|*	|*	|u	|*	|e	|*	|r	|*	|[16][S]\rarr	|k	|a	|t	|a	|s	|t	|r	|o	|f	|i	|z	|m	|*	|*	|*	|*	|*	|.
|*	|*	|*	|*	|*	|*	|z	|[17][S]\rarr	|h	|a	|m	|u	|l	|e	|c	|[][,]{ }	|p	|r	|ó	|ż	|n	|i	|o	|w	|y	|*	|.
|*	|[18][S]\drarr	|e	|[][S]1	|[][S]2	|[][S]3	|*	|[19][S]\rarr	|j	|e	|d	|l	|i	|c	|a	|[][,]{ }	|d	|o	|u	|g	|l	|a	|s	|a	|*	|*	|.
|[20][S]\rarr	|h	|o	|n	|d	|a	|*	|*	|*	|[21][S]\rarr	|m	|a	|s	|a	|[][,]{ }	|p	|l	|a	|n	|c	|k	|a	|*	|*	|*	|*	|.
|*	|o	|*	|*	|*	|*	|[22][S]\rarr	|k	|w	|a	|d	|r	|a	|t	|[][,]{ }	|m	|a	|g	|i	|c	|z	|n	|y	|*	|*	|*	|.
|*	|l	|*	|*	|*	|[23][S]\rarr	|n	|i	|e	|p	|r	|z	|y	|t	|o	|m	|n	|o	|ś	|ć	|*	|*	|*	|*	|*	|*	|.
|*	|*	|*	|*	|*	|*	|*	|*	|*	|*	|*	|*	|*	|*	|*	|*	|*	|*	|*	|*	|*	|*	|*	|*	|*	|*	|.\end{Puzzle}

\newpage

\begin{PuzzleClues}{\textbf{Poziome}\\}\Clue{1}{}{głuchy dźwięk, który powstaje, kiedy uderzają o siebie twarde przedmioty}
\Clue{2}{}{zakład obróbki, polegającej na szlifowaniu przedmiotów, np. kryształów, szkła, kamieni}
\Clue{3}{}{narzędzie ręczne w postaci kawałka ostrego żelaza osadzonego na drewnianej rękojeści, służy m.in. do spulchniania ziemi}
\Clue{4}{}{gimnastyk białoruski, najbardziej utytułowany zawodnik ostatnich lat, m.in. sześciokrotny złoty medalista olimpijski z Barcelony, czterokrotny brązowy medalista z Atlanty}
\Clue{6}{}{geolog angielski (1890-1964); rozwinął geotektoniczną teorię prądów konwekcyjnych}
\Clue{7}{}{surowce pochodzenia roślinnego do sporządzania mieszanek ziołowych}
\Clue{8}{}{zawartość flachy, dużej butelki; tyle, ile się mieści we flasze}
\Clue{9}{}{wielościan, którego podstawą jest prostokąt, a ścianami bocznymi dwa trapezy równoramienne i dwa trójkąty równoramienne}
\Clue{10}{}{planetoida odkryta w 1949 r.; zbliża się do Ziemi na odległość ok. 7 ml km}
\Clue{11}{}{poeta perski, satyryk, zm.ok.H73r, utwory panegiryczne, satyry i parodie}
\Clue{12}{}{tanie wino (określenie nawiązuje sentymentalnie do dawnej marki popularnego polskiego trunku)}
\Clue{13}{}{Stanisław (1878-1954) malarz, profesor ASP w Warszawie}
\Clue{14}{}{kosmopolityczna sowa, żywi się gryzoniami, chroniona}
\Clue{15}{}{w branży reklamowej - dokument zawierający informacje potrzebne do rozpoczęcia kampanii reklamowej}
\Clue{16}{}{dekadencka postawa wyrażająca przeświadczenie o nieuniknionej, gwałtownej zagładzie obecnej formy świata i cywilizacji}
\Clue{17}{}{hamulec sterowany za pomocą rozrzedzonego powietrza}
\Clue{18}{}{skrót, alternatywna nazwa handlowa amarantu - barwnika spożywczego}
\Clue{19}{}{daglezja zielona, jedlica zielona, jedlica Menziesa, Pseudotsuga menziesii - gatunek drzewa z rodziny sosnowatych}
\Clue{20}{}{samochód marki Honda}
\Clue{21}{}{jednostka masy w naturalnym systemie jednostek}
\Clue{22}{}{tablica składająca się z n wierszy i n kolumn (n>2), w którą wpisano n2 różnych nie powtarzających się dodatnich liczb naturalnych w ten sposób, że suma liczb w każdym wierszu, w każdej kolumnie i w każdej przekątnej jest taka sama (tzw. suma magiczna)}
\Clue{23}{}{wyraz braku kontaktu z rzeczywistością, bycia myślami daleko}\end{PuzzleClues}

\begin{PuzzleClues}{\textbf{Pionowe}\\}\Clue{1}{}{Muridae -  rodzina gryzoni, w skład której wchodzi 140 rodzajów z ok. 650 gatunkami (w Polsce zamieszkuje 8 gatunków); pierwotnie zamieszkiwały Afrykę, Eurazję, Australię, teraz niektóre z myszowatych są rozprzestrzenione na całym świecie}
\Clue{2}{}{ubarwianie, zabarwianie - nadawanie czemuś jakiejś barwy przy użyciu substancji barwiących}
\Clue{3}{}{amerykański antropolog (1859-1943); antropologia kulturowa w USA}
\Clue{5}{}{przodek bydła domowego, ssak z rodziny krętorogich}
\Clue{7}{}{uzbrojony, prywatny statek handlowy upoważniony do prowadzenia wojny morskiej}
\Clue{10}{}{pomieszczenie w kościele przeznaczone na zebrania}
\Clue{12}{}{miasto w Chinach (Anhui) port nad Jangcy, ważny węzeł komunikacyjny}
\Clue{18}{}{HALL; mieszkalna sień w angielskim domu, także duży przedpokój, poczekalnie}\end{PuzzleClues}\newpage\section*{Krzyżówka 21}

\noindent\begin{Puzzle}{22}{31}|*	|*	|[1][S]\darr	|*	|*	|[2][S]\drarr	|a	|m	|i	|o	|d	|a	|r	|o	|n	|*	|*	|*	|*	|*	|*	|*	|*	|.
|[3][S]\rarr	|p	|o	|p	|i	|s	|*	|*	|[4][S]\darr	|*	|[5][S]\drarr	|p	|o	|ł	|ą	|c	|z	|e	|n	|i	|e	|*	|*	|.
|*	|*	|g	|*	|*	|a	|*	|*	|p	|[6][S]\rarr	|m	|a	|*	|*	|*	|[7][S]\drarr	|s	|t	|ó	|j	|k	|a	|*	|.
|*	|[8][S]\darr	|i	|[9][S]\rarr	|s	|k	|o	|m	|o	|r	|o	|c	|h	|*	|*	|r	|[10][S]\darr	|*	|[11][S]\darr	|*	|*	|*	|[12][S]\darr	|.
|*	|l	|e	|*	|*	|w	|*	|*	|n	|*	|d	|*	|*	|*	|*	|d	|k	|*	|b	|*	|[13][S]\darr	|[14][S]\darr	|p	|.
|*	|i	|ń	|*	|*	|a	|*	|[15][S]\rarr	|c	|o	|r	|o	|*	|[16][S]\darr	|*	|z	|o	|*	|u	|*	|ł	|k	|o	|.
|*	|c	|*	|*	|*	|*	|*	|*	|z	|[17][S]\darr	|a	|*	|*	|d	|*	|e	|n	|*	|d	|*	|ą	|o	|m	|.
|*	|z	|*	|[18][S]\drarr	|j	|u	|n	|i	|*	|p	|s	|*	|*	|r	|[19][S]\darr	|ń	|s	|[20][S]\darr	|z	|[21][S]\darr	|c	|r	|o	|.
|[22][S]\drarr	|b	|a	|ł	|t	|ó	|w	|*	|*	|r	|z	|[23][S]\drarr	|w	|a	|ł	|*	|u	|m	|i	|w	|z	|t	|c	|.
|k	|a	|[24][S]\drarr	|o	|b	|o	|w	|i	|ą	|z	|e	|k	|[][,]{ }	|p	|u	|b	|l	|i	|c	|z	|n	|y	|*	|.
|o	|[][,]{ }	|s	|s	|*	|*	|*	|*	|*	|o	|k	|a	|*	|a	|k	|*	|t	|o	|i	|m	|i	|k	|*	|.
|m	|n	|a	|z	|*	|*	|*	|[25][S]\darr	|*	|d	|[][,]{ }	|l	|[26][S]\darr	|c	|[][,]{ }	|*	|a	|d	|e	|a	|k	|o	|*	|.
|b	|i	|m	|a	|*	|*	|*	|p	|*	|o	|b	|e	|k	|z	|s	|*	|c	|o	|l	|c	|[][,]{ }	|s	|*	|.
|i	|e	|i	|*	|*	|*	|[27][S]\drarr	|o	|ś	|m	|l	|n	|a	|*	|k	|*	|j	|j	|*	|n	|d	|t	|*	|.
|n	|p	|z	|*	|*	|*	|t	|k	|*	|ó	|a	|d	|r	|*	|r	|*	|a	|a	|*	|i	|w	|e	|*	|.
|a	|a	|d	|*	|*	|*	|o	|ł	|*	|ż	|d	|a	|b	|*	|z	|*	|[][,]{ }	|d	|*	|a	|u	|r	|*	|.
|c	|r	|a	|*	|*	|*	|m	|a	|*	|d	|y	|r	|o	|*	|e	|*	|s	|[][,]{ }	|*	|c	|b	|y	|*	|.
|j	|z	|t	|*	|*	|*	|a	|d	|*	|ż	|*	|z	|a	|*	|l	|*	|p	|c	|*	|z	|i	|d	|*	|.
|a	|y	|*	|*	|*	|*	|s	|ó	|*	|e	|*	|[][,]{ }	|n	|*	|o	|*	|o	|i	|*	|[][,]{ }	|e	|*	|*	|.
|[][,]{ }	|s	|*	|*	|*	|*	|z	|w	|*	|*	|*	|g	|i	|[28][S]\darr	|w	|*	|ł	|e	|*	|o	|g	|*	|*	|.
|l	|t	|*	|*	|*	|*	|ó	|k	|*	|*	|*	|r	|o	|ż	|y	|*	|e	|n	|*	|p	|u	|*	|*	|.
|i	|a	|[29][S]\darr	|*	|*	|*	|w	|a	|*	|*	|*	|e	|n	|e	|*	|*	|c	|k	|*	|t	|n	|*	|*	|.
|n	|*	|k	|*	|*	|*	|*	|*	|*	|*	|*	|g	|*	|l	|*	|*	|z	|o	|*	|y	|o	|*	|*	|.
|i	|*	|u	|*	|*	|*	|*	|[30][S]\drarr	|f	|a	|g	|o	|t	|*	|*	|*	|n	|d	|*	|c	|w	|*	|*	|.
|o	|*	|k	|*	|*	|*	|[31][S]\rarr	|o	|p	|i	|e	|r	|d	|o	|l	|*	|a	|z	|*	|z	|y	|*	|*	|.
|w	|*	|i	|*	|*	|*	|*	|f	|*	|[32][S]\rarr	|w	|i	|a	|n	|e	|k	|*	|i	|*	|n	|*	|*	|*	|.
|a	|*	|e	|*	|[33][S]\rarr	|k	|o	|l	|e	|b	|k	|a	|*	|*	|*	|*	|*	|o	|*	|y	|*	|*	|*	|.
|*	|*	|ł	|*	|*	|[34][S]\rarr	|s	|a	|m	|o	|a	|ń	|c	|z	|y	|k	|*	|b	|*	|*	|*	|*	|*	|.
|*	|*	|k	|*	|*	|*	|*	|g	|[35][S]\rarr	|ł	|y	|s	|a	|k	|*	|*	|*	|y	|*	|*	|*	|*	|*	|.
|[36][S]\rarr	|h	|a	|ń	|s	|k	|a	|*	|*	|*	|[37][S]\rarr	|k	|o	|r	|s	|u	|ń	|*	|*	|*	|*	|*	|*	|.
|*	|*	|*	|[38][S]\rarr	|a	|m	|p	|e	|r	|o	|m	|i	|e	|r	|z	|*	|*	|*	|*	|*	|*	|*	|*	|.
|*	|*	|*	|[39][S]\rarr	|n	|a	|l	|a	|j	|c	|h	|*	|*	|*	|*	|*	|*	|*	|*	|*	|*	|*	|*	|.\end{Puzzle}

\newpage

\begin{PuzzleClues}{\textbf{Poziome}\\}\Clue{2}{}{lek zwiększający czas trwania potencjału błonowego; należy do III grupy leków antyarytmicznych zgodnie z podziałem według Vaughana Wiliamsa}
\Clue{3}{}{działanie zwaracające powszechną uwagę}
\Clue{5}{}{to, co łączy części całości (w tym: ogólnie nazwane miejsce połączenia); element łączący lub sposób połączenia}
\Clue{6}{}{jednostka prędkości odnosząca się do obiektów poruszających się w płynie lub poruszających się płynów}
\Clue{7}{}{stojący kołnierz okrywający szyję na całym jej obwodzie lub na znacznej części obwodu}
\Clue{9}{}{w średniowiecznej Rosji; wędrowny muzykant, śpiewak i kuglarz}
\Clue{15}{}{chór; nazwa stosowana w nutach}
\Clue{18}{}{hiszpański rzeźbiarz i malarz (1507-77) rzeźby, obrazy i freski}
\Clue{22}{}{wieś w Polsce położona w województwie świętokrzyskim, w powiecie ostrowieckim; siedziba gminy Bałtów}
\Clue{23}{}{ziemia lub kamienie usypane w podłużny nasyp, służące np. do powstrzymania wylewów rzek (wał przeciwpowodziowy) albo do celów obronnych (wał obronny, obwałowanie)}
\Clue{24}{}{zobowiązanie narzucane przedsiębiorstwom (np. energetycznym) przez kraje członkowskie w ogólnym interesie gospodarczym Wspólnoty Europejskiej}
\Clue{27}{}{ÓSMAK, ACHTEL}
\Clue{30}{}{drewniany, dęty instrument o niskiej skali składa się z dwóch połączonych ze sobą piszczałek z podwójnym stroikiem}
\Clue{31}{}{stanowcze i surowe pouczenie, reprymenda}
\Clue{32}{}{czystość przedmałżeńska; dziewictwo}
\Clue{33}{}{aleja obsadzona obustronnie drzewami lub pnączami, których pędy splatają się nad drogą tworząc sklepienie}
\Clue{34}{}{mieszkaniec Samoa, człowiek pochodzenia samoańskiego}
\Clue{35}{}{Gymnopilus P. Karst. - rodzaj grzybów należący do rodziny pierścieniakowatych; małe lub średniej wielkości grzyby, zazwyczaj rosnące na drewnie}
\Clue{36}{}{Ewelina z Rzewuskich (1801-82); żona H. Balzaka}
\Clue{37}{}{miasto na Ukrainie, stolica rejonu w obwodzie czerkaskim; nazwa stosowana oficjalnie do 1944 roku oraz obecnie potocznie}
\Clue{38}{}{przyrząd do pomiaru natężenia elektrycznego}
\Clue{39}{}{NALAJCHA; miasto w Mongolii na płd.-wsch. od Ułan Bator}\end{PuzzleClues}

\begin{PuzzleClues}{\textbf{Pionowe}\\}\Clue{1}{}{strzały z broni palnej}
\Clue{2}{}{worek na pieniądze lub drobiazgi, noszony zazwyczaj przy pasie}
\Clue{4}{}{mrożony lub gorący napój alkoholowy w podstawowej wersji przygotowywany z pięciu składników: herbaty, cukru, cytryny i dowolnego innego owocu oraz wina}
\Clue{5}{}{Cupido decolorata - motyl dzienny z rodziny modraszkowatych (podrodzina modraszki); jest to gatunek pontyjsko-śródziemnomorski dokonujący ekspansji w Europie Środkowej}
\Clue{7}{}{włókno; tkanka miękiszowa w łodygach roślin i drzew}
\Clue{8}{}{to liczba całkowita, niepodzielna przez 2}
\Clue{10}{}{instytucja demokracji bezpośredniej, której zadaniem jest wyrażanie przez członków danej społeczności opinii w określonej sprawie}
\Clue{11}{}{ktoś, kto rozbudza emocje i pobudza do czynu, inicjuje działanie, funkcjonowanie czegoś}
\Clue{12}{}{zawodnicy grający w drużynie na pozycji pomocników}
\Clue{13}{}{łącznik instalacyjny, który wyposażony jest w pojedynczy przycisk; dokonuje łączenia dwóch obwodów, działa na zasadzie przełącznika (sprężyna nie odbija z powrotem po puszczeniu przycisku)}
\Clue{14}{}{glikokortykosteroid, glikokortykoid, glikokortykosteryd, kortykosteroid - hormon produkowany w warstwach pasmowatej i siatkowatej kory nadnerczy pod wpływem ACTH, regulujący przemiany białek, węglowodanów i tłuszczów, stosowany jako lek o działaniu przeciwzapalnym, przeciwalergicznym i immunosupresyjnym}
\Clue{16}{}{kotwica o czterech wygiętych łapach używana do zakotwiczania małych łodzi}
\Clue{17}{}{część mózgu, która jest z przodu}
\Clue{18}{}{KLĘPA}
\Clue{19}{}{chrzęstny lub kostny element wchodzący w skład szkieletu trzewioczaszki kręgowców, występujący w różnej liczbie}
\Clue{20}{}{Meliphaga gracilis - gatunek ptaka z rodziny miodojadów (Meliphagidae), który zamieszkuje wschodnie wybrzeże Australii}
\Clue{21}{}{urządzenie wzmacniające sygnał optyczny (promieniowanie świetlne) bezpośrednio, bez konwersji na sygnał elektryczny}
\Clue{22}{}{wyrażenie, które stanowi sumę wektorów pomnożonych przez współczynniki skalarne}
\Clue{23}{}{kalendarz słoneczny wprowadzony w 1582 przez papieża Grzegorza XIII bullą Inter gravissimas}
\Clue{24}{}{określenie tzw. literatury drugiego obiegu wydawanej nielegalnie w ZSRR od 2. poł. lat 50}
\Clue{25}{}{NADBUDÓWKA}
\Clue{26}{}{jon organiczny, w którym ładunek dodatni jest skupiony na atomie węgla}
\Clue{27}{}{Tomaszów Lubelski, miasto w województwie lubelskim}
\Clue{28}{}{szczególny rodzaj układu koloidalnego, będący efektem koagulacji zolu}
\Clue{29}{}{lalka teatralna sztywno osadzona na kiju, którym porusza aktor animator, wyróżniająca się zamachowymi ruchami kończyn}
\Clue{30}{}{niemiecki obóz jeniecki, w którym trzymani byli oficerowie wzięci do niewoli w czasie działań wojennych lub okupacyjnych}\end{PuzzleClues}\newpage\section*{Krzyżówka 22}

\noindent\begin{Puzzle}{21}{27}|*	|*	|*	|*	|*	|*	|*	|*	|[1][S]\darr	|*	|*	|*	|*	|*	|*	|*	|*	|*	|*	|*	|[2][S]\darr	|[3][S]\darr	|.
|*	|*	|*	|[4][S]\rarr	|a	|l	|o	|e	|s	|[][,]{ }	|u	|z	|b	|r	|o	|j	|o	|n	|y	|*	|w	|a	|.
|*	|*	|*	|*	|*	|*	|[5][S]\rarr	|p	|e	|r	|s	|y	|f	|l	|a	|ż	|*	|*	|*	|*	|y	|n	|.
|*	|*	|[6][S]\rarr	|s	|z	|c	|z	|e	|n	|i	|a	|c	|z	|e	|k	|*	|*	|[7][S]\darr	|*	|*	|c	|o	|.
|*	|*	|*	|[8][S]\drarr	|c	|h	|o	|r	|o	|b	|a	|[][,]{ }	|u	|h	|l	|a	|*	|w	|*	|[9][S]\darr	|i	|m	|.
|*	|*	|[10][S]\rarr	|p	|t	|a	|s	|i	|a	|[][,]{ }	|g	|r	|y	|p	|a	|*	|*	|a	|[11][S]\darr	|b	|ą	|a	|.
|[12][S]\drarr	|w	|y	|s	|i	|ł	|e	|k	|*	|*	|*	|*	|*	|*	|*	|*	|*	|r	|r	|ą	|g	|l	|.
|j	|[13][S]\rarr	|s	|z	|a	|n	|t	|u	|n	|g	|*	|*	|*	|[14][S]\drarr	|m	|ł	|y	|n	|e	|k	|*	|i	|.
|a	|*	|[15][S]\darr	|c	|*	|[16][S]\drarr	|s	|m	|a	|c	|z	|l	|i	|w	|k	|a	|*	|a	|l	|*	|[17][S]\darr	|a	|.
|s	|*	|a	|z	|[18][S]\darr	|m	|*	|[19][S]\rarr	|k	|u	|k	|u	|r	|y	|d	|z	|a	|*	|i	|[20][S]\darr	|j	|[][,]{ }	|.
|z	|[21][S]\drarr	|w	|o	|t	|u	|m	|*	|*	|*	|*	|*	|[22][S]\rarr	|w	|i	|t	|z	|*	|k	|f	|a	|p	|.
|c	|o	|i	|ł	|u	|n	|[23][S]\drarr	|c	|y	|g	|a	|r	|n	|i	|c	|a	|*	|*	|w	|i	|f	|o	|.
|z	|p	|n	|a	|b	|s	|g	|*	|*	|*	|*	|[24][S]\rarr	|m	|a	|g	|n	|e	|s	|i	|k	|*	|l	|.
|u	|c	|i	|[][,]{ }	|u	|t	|a	|*	|*	|*	|*	|*	|*	|d	|*	|*	|[25][S]\darr	|[26][S]\darr	|a	|s	|[27][S]\darr	|a	|.
|r	|j	|o	|p	|s	|e	|r	|*	|*	|[28][S]\rarr	|s	|z	|t	|o	|k	|a	|d	|a	|*	|a	|k	|n	|.
|k	|a	|n	|e	|*	|r	|y	|*	|[29][S]\darr	|*	|*	|*	|*	|w	|*	|[30][S]\darr	|y	|k	|[31][S]\darr	|t	|r	|d	|.
|o	|[][,]{ }	|*	|r	|*	|*	|*	|*	|a	|*	|*	|*	|[32][S]\darr	|c	|[33][S]\darr	|w	|m	|c	|r	|y	|z	|a	|.
|j	|b	|*	|s	|[34][S]\drarr	|z	|e	|s	|p	|ó	|ł	|[][,]{ }	|z	|a	|d	|a	|n	|i	|o	|w	|y	|*	|.
|a	|a	|[35][S]\drarr	|k	|o	|ń	|[][,]{ }	|ś	|l	|ą	|s	|k	|i	|*	|y	|r	|i	|k	|z	|a	|k	|*	|.
|d	|r	|b	|a	|ś	|*	|*	|*	|e	|*	|*	|[36][S]\darr	|n	|*	|l	|g	|k	|*	|t	|*	|l	|*	|.
|[][,]{ }	|i	|o	|*	|r	|*	|*	|*	|g	|*	|*	|o	|g	|*	|*	|a	|*	|*	|r	|*	|i	|*	|.
|d	|e	|t	|*	|o	|*	|*	|*	|i	|*	|*	|c	|a	|*	|*	|c	|*	|*	|o	|*	|w	|*	|.
|u	|r	|f	|*	|d	|*	|*	|*	|e	|*	|*	|z	|n	|[37][S]\rarr	|s	|z	|l	|u	|p	|*	|e	|*	|.
|ż	|o	|o	|*	|e	|*	|[38][S]\rarr	|f	|r	|f	|*	|e	|a	|*	|*	|*	|*	|*	|n	|*	|*	|*	|.
|y	|w	|r	|*	|k	|*	|*	|*	|*	|*	|*	|r	|*	|*	|*	|*	|*	|*	|o	|*	|*	|*	|.
|*	|a	|t	|*	|*	|*	|*	|[39][S]\rarr	|l	|i	|m	|e	|t	|a	|[][,]{ }	|k	|w	|a	|ś	|n	|a	|*	|.
|*	|*	|y	|*	|*	|*	|*	|*	|*	|[40][S]\rarr	|s	|t	|a	|n	|o	|w	|o	|ś	|ć	|*	|*	|*	|.
|*	|*	|*	|*	|*	|*	|*	|*	|*	|*	|*	|*	|*	|*	|*	|*	|*	|*	|*	|*	|*	|*	|.\end{Puzzle}

\newpage

\begin{PuzzleClues}{\textbf{Poziome}\\}\Clue{4}{}{gatunek drzewa pochodzący z południowej Afryki}
\Clue{5}{}{wypowiedź o charakterze żartobliwym i ironicznym, ukrywająca szyderstwo pod pozorami wyszukanej uprzejmości lub udawanej powagi}
\Clue{6}{}{mała porcja wódki lub innego mocnego alkoholu (rzadko drinka) do wypicia na raz}
\Clue{8}{}{rzadka wada wrodzona serca charakteryzująca się częściowym lub całkowitym brakiem mięśnia sercowego prawej komory}
\Clue{10}{}{ostra choroba zakaźna występująca powszechnie u ptaków, wywołana przez typ A wirusa grypy}
\Clue{12}{}{trud, duży nakład pracy potrzebny do zrobienia czegoś}
\Clue{13}{}{półwysep w północno-wschodnich Chinach}
\Clue{14}{}{maszyna rolnicza do wstępnego sortowania i czyszczenia ziarna}
\Clue{16}{}{egzotyczny owoc (jagoda) pochodzący z rośliny o tej samej nazwie, wykorzystywany w kuchni jako owoc/warzywo}
\Clue{19}{}{Zea - rodzaj roślin należący do rodziny wiechlinowatych, plemienia kukurydzowych}
\Clue{21}{}{w Kościele katolickim symboliczny przedmiot na ołtarzu}
\Clue{22}{}{malarz szwajcarski (1400-1445); obrazy ołtarzowe; 'Cudowny połów ryb'}
\Clue{23}{}{pudełko na papierosy}
\Clue{24}{}{gadżet, który składa się z ozdoby i przytwierdzonego do niej magnesu, służący do przymocowywania (także do przymocowywania czegoś) do metalowych powierzchni}
\Clue{28}{}{długa szpada z długim końcem}
\Clue{34}{}{grupa osób powołanych do pracy w celu realizacji określonego zadania, będącego częścią większego projektu}
\Clue{35}{}{rasa koni powstała na bazie koni z terenów Dolnego i Górnego Śląska, odnosząca duże sukcesy jako konie zaprzęgowe w dyscyplinie powożenia; centrum hodowlanym jest Stado Ogierów Książ i działająca przy nim stadnina koni}
\Clue{37}{}{żaglowiec jednomasztowy posiadający dwa żagle (z grotem i fokiem)}
\Clue{38}{}{kod ISO 4217 franka francuskiego}
\Clue{39}{}{Citrus aurantifolia - gatunek rośliny należącej do rodziny rutowatych}
\Clue{40}{}{ustrój polityczny, w którym istnieją zdefiniowane grupy społeczne posiadające określone prawa i obowiązki}\end{PuzzleClues}

\begin{PuzzleClues}{\textbf{Pionowe}\\}\Clue{1}{}{pisarz chorwacki (1838-81), twórca chorwackiej powieści; „Złota dzieweczka”, „Bramka”, „Bunt chłopów”, opowiadania przekłady}
\Clue{2}{}{odległość między dwoma stanowiskami asekuracyjnymi wyznaczona przede wszystkim długością używanej liny oraz konfiguracją terenu i ilością sprzętu}
\Clue{3}{}{rzadki zespół wad wrodzonych, na który składają się jednostronna hipoplazja lub aplazja mięśni klatki piersiowej i brodawki sutkowej, krótkopalczastość i zrosty palców (syndaktylia) po tej samej stronie}
\Clue{7}{}{miasto w Bułgarii}
\Clue{8}{}{Apis mellifera meda - podgatunek pszczoły miodnej pochodzący z Iraku}
\Clue{9}{}{gatunek czapli o długości do 70 cm; w okresie godowym samiec wydaje buczący głos; Eurazja, płn. Afryka; chroniony}
\Clue{11}{}{przedmiot związany z czymś lub kimś świętym, najczęściej szczątki ciała świętego}
\Clue{12}{}{Coccyzus merlini - gatunek ptaka z rodziny kukułkowatych (Cuculidae), z podrodziny kukułek (Cuculinae)}
\Clue{14}{}{funkcjonariusz śledczy, pracownik służb śledczych}
\Clue{15}{}{miasto we Francji (Prowansja), nad Rodanem, ośrodek administracyjny departamentu Vaucluse}
\Clue{16}{}{MHUMHA; prowincja historyczna w płd Irlandii, powierzchnia 24,1 tyś. km , główne miasto Corcaigh}
\Clue{17}{}{rasa koni domowych pochodzących z Persji, bardzo wytrzymała i przydatna jako konie sportowe}
\Clue{18}{}{tuleja z soczewkami stanowiąca część przyrządu optycznego}
\Clue{20}{}{francuski malarz, rysownik i grafik (1732-1806) tematyka miłosna, portrety, pejzaże, rysunki}
\Clue{21}{}{rodzaj opcji egzotycznej, której wykonanie zależy od tego czy cena instrumentu bazowego osiagnie lub przekroczy ustalony poziom}
\Clue{23}{}{(1914-80) pisarz francuski, właściwie R. Kacev; „Korzenie nieba”}
\Clue{25}{}{przewód, komin}
\Clue{26}{}{zdrobniale: akt - część utworu scenicznego}
\Clue{27}{}{amerykańskie ptaki z rzędu wróblowatych, około 1000 gatunków}
\Clue{29}{}{ODKŁAD - pęd, z którego rozwija się nowa roślina}
\Clue{30}{}{czarny niedźwiedź z małym półksiężycem na piersi - Płw. Indyjski, Cejlon}
\Clue{31}{}{cecha człowieka: to, że ktoś myśli, rozważa różne rzeczy}
\Clue{32}{}{gatunek drewna pozyskiwany z afrykańskich drzew z rodziny brezylkowatych Microberlinia brazzavillensis; wykorzystywane w szkutnictwie, meblarstwie i lutnictwie}
\Clue{33}{}{gruba deska, także element betonowy lub gipsowy jako element ścienny stropowy}
\Clue{34}{}{przekształcona ściana zarodni, zaopatrująca przedrośle w substancje zapasowe}
\Clue{35}{}{buty z wysokimi cholewami, nieco wyższymi z przodu, używane do konnej jazdy}
\Clue{36}{}{zarośle roślin, szuwar przybrzeżny; zwykle w liczbie mnogiej}\end{PuzzleClues}\newpage\section*{Krzyżówka 23}

\noindent\begin{Puzzle}{20}{30}|*	|*	|*	|*	|*	|*	|[1][S]\darr	|[2][S]\darr	|*	|[3][S]\darr	|[4][S]\darr	|*	|[5][S]\drarr	|l	|o	|d	|ó	|w	|k	|a	|*	|.
|*	|[6][S]\darr	|[7][S]\darr	|*	|[8][S]\darr	|[9][S]\darr	|s	|d	|[10][S]\darr	|c	|k	|[11][S]\rarr	|p	|a	|l	|i	|w	|o	|d	|a	|*	|.
|[12][S]\drarr	|p	|r	|z	|y	|m	|i	|o	|t	|n	|o	|[][,]{ }	|o	|g	|r	|o	|d	|o	|w	|e	|*	|.
|n	|r	|a	|*	|p	|i	|e	|p	|e	|o	|s	|[13][S]\drarr	|w	|ę	|z	|e	|ł	|*	|*	|*	|[14][S]\darr	|.
|e	|o	|b	|*	|s	|ę	|r	|*	|l	|t	|z	|o	|i	|*	|*	|*	|[15][S]\darr	|*	|*	|*	|o	|.
|u	|d	|u	|*	|i	|c	|o	|*	|e	|a	|y	|r	|e	|*	|*	|*	|c	|*	|*	|*	|p	|.
|r	|u	|n	|*	|l	|z	|t	|*	|m	|[][,]{ }	|k	|k	|ś	|*	|[16][S]\rarr	|g	|o	|r	|i	|*	|i	|.
|o	|k	|e	|[17][S]\darr	|o	|a	|a	|*	|e	|k	|*	|*	|ć	|*	|[18][S]\drarr	|u	|b	|a	|w	|*	|e	|.
|p	|c	|k	|m	|n	|k	|[][,]{ }	|[19][S]\drarr	|c	|a	|ł	|y	|[][,]{ }	|t	|o	|n	|*	|*	|*	|*	|ń	|.
|s	|j	|*	|e	|*	|*	|z	|b	|h	|r	|*	|[20][S]\darr	|m	|*	|r	|[21][S]\darr	|[22][S]\darr	|[23][S]\darr	|*	|[24][S]\darr	|s	|.
|y	|a	|*	|t	|*	|*	|u	|a	|a	|d	|*	|w	|i	|[25][S]\darr	|z	|a	|s	|k	|*	|ś	|k	|.
|c	|*	|*	|a	|*	|*	|p	|r	|n	|y	|*	|e	|ł	|d	|e	|p	|z	|o	|[26][S]\darr	|p	|i	|.
|h	|*	|*	|s	|*	|*	|e	|w	|i	|n	|[27][S]\drarr	|k	|o	|z	|ł	|o	|w	|s	|k	|i	|*	|.
|i	|[28][S]\rarr	|c	|e	|p	|*	|ł	|n	|k	|a	|e	|i	|s	|w	|[][,]{ }	|r	|a	|m	|o	|e	|[29][S]\darr	|.
|a	|*	|*	|k	|*	|[30][S]\darr	|n	|i	|a	|l	|m	|e	|n	|o	|s	|t	|j	|o	|n	|w	|s	|.
|t	|*	|[31][S]\drarr	|w	|y	|m	|y	|k	|*	|n	|u	|r	|a	|n	|a	|a	|c	|g	|w	|[][,]{ }	|i	|.
|r	|*	|o	|o	|*	|i	|*	|*	|*	|a	|*	|a	|*	|o	|w	|c	|a	|o	|e	|g	|ł	|.
|i	|*	|d	|j	|[32][S]\rarr	|d	|e	|r	|y	|*	|*	|*	|*	|*	|a	|j	|r	|n	|r	|r	|a	|.
|a	|[33][S]\drarr	|w	|a	|l	|a	|b	|i	|a	|[][,]{ }	|p	|i	|ę	|k	|n	|a	|*	|i	|s	|e	|[][,]{ }	|.
|[][,]{ }	|p	|i	|[][,]{ }	|*	|s	|*	|[34][S]\rarr	|a	|w	|i	|o	|f	|o	|n	|*	|*	|a	|j	|g	|l	|.
|d	|e	|e	|c	|*	|*	|*	|[35][S]\drarr	|s	|t	|r	|o	|n	|n	|o	|ś	|ć	|*	|a	|o	|o	|.
|z	|r	|d	|h	|*	|*	|*	|f	|[36][S]\rarr	|m	|l	|e	|k	|o	|w	|i	|e	|c	|*	|r	|r	|.
|i	|f	|z	|i	|[37][S]\rarr	|m	|a	|r	|t	|r	|e	|*	|[38][S]\rarr	|m	|y	|l	|a	|i	|*	|i	|e	|.
|e	|o	|a	|ń	|[39][S]\drarr	|d	|r	|a	|m	|a	|t	|y	|k	|a	|*	|*	|*	|*	|*	|a	|n	|.
|c	|r	|l	|s	|z	|[40][S]\rarr	|i	|n	|t	|e	|r	|w	|i	|z	|j	|a	|*	|*	|*	|ń	|t	|.
|i	|a	|n	|k	|b	|[41][S]\rarr	|e	|c	|h	|o	|l	|o	|k	|a	|c	|j	|a	|*	|*	|s	|z	|.
|ę	|c	|o	|a	|i	|[42][S]\rarr	|s	|z	|m	|a	|r	|a	|g	|d	|ó	|w	|k	|a	|*	|k	|a	|.
|c	|j	|ś	|*	|ó	|*	|*	|y	|[43][S]\rarr	|g	|r	|z	|y	|b	|o	|l	|u	|b	|k	|i	|*	|.
|a	|a	|ć	|[44][S]\rarr	|r	|u	|s	|z	|n	|i	|k	|a	|r	|z	|*	|[45][S]\rarr	|ł	|u	|g	|*	|*	|.
|*	|*	|*	|*	|*	|[46][S]\rarr	|w	|a	|l	|u	|t	|o	|w	|o	|ś	|ć	|*	|*	|*	|*	|*	|.
|*	|[47][S]\rarr	|h	|o	|p	|a	|k	|*	|*	|*	|*	|*	|*	|*	|*	|*	|*	|*	|*	|*	|*	|.\end{Puzzle}

\newpage

\begin{PuzzleClues}{\textbf{Poziome}\\}\Clue{5}{}{gatunek kaczki z mórz północnych, w Polsce na przelotach}
\Clue{11}{}{osoba lekkomyślna, nieodpowiedzialna, beztroska}
\Clue{12}{}{Erigeron hybridus - gatunek roślin z rodziny astrowatych}
\Clue{13}{}{kolanko - zgrubienie pędu (źdźbła) roślin z rodziny wiechlinowatych (traw)}
\Clue{16}{}{miasto w środkowej części Gruzji,62 tys. mieszkańców (1586)}
\Clue{18}{}{spotkanie towarzyskie, zabawa}
\Clue{19}{}{jednostka odległości między dwoma dźwiękami, składa się z dwóch półtonów}
\Clue{27}{}{rzeźbiarz rosyjski (1753-1802) przedstawiciel klasycyzmu}
\Clue{28}{}{część młockarni, która służy do młócenia}
\Clue{31}{}{Arctosa - rodzaj pająka z rodziny pogońcowatych}
\Clue{32}{}{pisarz węgierski (1894-1977), ekspresjonizm, surrealizm; „Odpowiedź”, „Kochany teściu...”, „Miłość”}
\Clue{33}{}{walabia Parry’ego, Macropus parryi - gatunek ssaka z rodziny kangurowatych; zamieszkuje tereny od okolic Cooktown w Queensland do północnej części Nowej Południowej Walii}
\Clue{34}{}{urządzenie do porozumiewania się załogi samolotu, składające się z tuby, przewodu gumowego i słuchawek}
\Clue{35}{}{asymetria czynnościowa prawej i lewej strony ciała ludzkiego, która wynika z różnic w budowie i funkcjach obu półkul mózgowych}
\Clue{36}{}{Brosimum - rodzaj rośliny z rodziny morwowatych}
\Clue{37}{}{jezioro w Kanadzie połączone z Wielkim Jeziorem Niewolniczym}
\Clue{38}{}{(MY LAI) wieś w płd. Wietnamie; w 1968r wymordowanie ludności przez oddział amerykańskich żołnierzy}
\Clue{39}{}{teoria dzieła dramatycznego}
\Clue{40}{}{festiwal organizowany w latach 1977-1980 w Operze Leśnej w Sopocie jako przeciwwaga dla festiwalu Eurowizji}
\Clue{41}{}{system określania położenia przeszkód lub poszukiwanych obiektów w otoczeniu z użyciem zjawiska echa akustycznego}
\Clue{42}{}{Graydidascalus brachyurus - gatunek ptaka z rodziny papugowatych (Psittacidae), z podrodziny papug neotropikalnych (Arinae)}
\Clue{43}{}{BEDLISZKI; rodzina drobnych muchówek, groźne dla upraw pieczarek}
\Clue{44}{}{rzemieślnik wyrabiający ręczną broń palną}
\Clue{45}{}{stężony, wodny roztwór mocnego wodorotlenku}
\Clue{46}{}{cecha czegoś, co jest walutowe, np. transakcji}
\Clue{47}{}{ukraiński taniec ludowy utrzymany w metrum 2/4}\end{PuzzleClues}

\begin{PuzzleClues}{\textbf{Pionowe}\\}\Clue{1}{}{chłopiec, który stracił oboje rodziców (naturalnych lub zastępczych); forma rzadsza}
\Clue{2}{}{kod ISO 4217 peso dominikańskiego}
\Clue{3}{}{pojęcie w chrześcijaństwie oznaczające cnotę człowieka, która odgrywaja kluczową rolę, dlatego też nazywa się ją cnotą „kardynalną”; wszystkie inne zalety grupują się wokół tego rodzaju cnoty}
\Clue{4}{}{element oprogramowania różnych aplikacji internetowych (głównie sklepów), z których korzysta klient jako nabywca dóbr i usług; koszyk jest częścią oprogramowania, która umożliwia zebranie w jednym miejscu listy produktów, które klient wybrał do zakupu}
\Clue{5}{}{powieść, której tematem jest rozwój miłości dwojga ludzi}
\Clue{6}{}{proces produkowania filmu, nagrania muzycznego itp.; realizacja materiału filmowego, muzycznego itp}
\Clue{7}{}{jeden ze sposobów podziemnej eksploatacji złóż (wybieranie z zawałem)}
\Clue{8}{}{nazwa jednego z mezonów}
\Clue{9}{}{człowiek słabego charakteru, bojaźliwy i uległy}
\Clue{10}{}{dziedzina telekomunikacji zajmująca się przesyłaniem sygnałów sterujących na odległość}
\Clue{12}{}{dział neuropsychiatrii zajmujący się zaburzeniami wieku dziecięcego}
\Clue{13}{}{istota z rasy orków w literackim legendarium, wykreowanym przez J. R. R. Tolkiena; Tolkienowscy orkowie, wyhodowani z upodlonych elfów przez Morgotha, związani byli z obozem zła}
\Clue{14}{}{muzykolog i kompozytor (1870-1942); opery i pieśni, poematy symfoniczne; 'Zygmunt August'}
\Clue{15}{}{krępy koń średniego wzrostu z rasy anglo-normandzkiej}
\Clue{17}{}{Metasequoia glyptostroboides - gatunek drzewa należący do rodziny cyprysowatych, jedyny żyjący przedstawiciel rodzaju metasekwoja}
\Clue{18}{}{Aquila rapax rapax - nominatywny podgatunek ptaka wyróżniony w obrębie gatunku orzeł sawannowy (Aquila rapax); występuje na terenach od południowej Kenii i Demokratycznej Republiki Konga do RPA}
\Clue{19}{}{substancja nadająca barwę innej substancji pozbawionej barwy (przezroczystej, białej lub szarej) lub też zmieniająca barwę substancji posiadającej już jakąś barwę}
\Clue{20}{}{maczuga w kształcie pałki}
\Clue{21}{}{jedna ze zdolności, jaką w trakcie nauki mogli posiąść czarodzieje; polega na umiejętności przywołania do siebie przedmiotów nawet z dużej odległości za pomocą zaklęcia}
\Clue{22}{}{portier}
\Clue{23}{}{mitologiczne, religijne lub filozoficzne wyobrażenie o pochodzeniu świata - opowieść danej kultury o własnych korzeniach, jej koncepcja stworzenia świata}
\Clue{24}{}{tradycyjny, jednogłosowy śpiew liturgiczny Kościoła łacińskiego}
\Clue{25}{}{w górnictwie: warstwa wybieranej soli przy wybieraniu dwuwarstwowym}
\Clue{26}{}{w ekonomii i finansach: zmiana założeń umowy związanej ze zobowiązaniem finansowym tak, aby była ona korzystniejsza dla wierzyciela lub dłużnika}
\Clue{27}{}{nielotny ptak z rzędu kazuarów wysokości do 2m}
\Clue{29}{}{siła, jaka działa na cząstkę obdarzoną ładunkiem elektrycznym poruszającą się w polu elektromagnetycznym}
\Clue{30}{}{władca Frygii z VIII w. p.n.e., bohater wielu mitów greckich}
\Clue{31}{}{liczba odwiedzin notowana przez miejsce (często w przestrzeni wirtualnej), któremu odwiedziny przynoszą jakiś zysk}
\Clue{33}{}{wytworzenie się nieprawidłowego otworu na powierzchni narządu wewnętrznego, czyli jego przedziurawienie, pęknięcie}
\Clue{35}{}{system sprzedaży towarów, usług lub technologii, który jest oparty na ścisłej i ciągłej współpracy pomiędzy prawnie i finansowo odrębnymi i niezależnymi przedsiębiorstwami}
\Clue{39}{}{(często) PLUR. zespół przedmiotów gromadzonych, zbieranych ze względu na ich wartość artystyczną, naukową, historyczną; kolekcja}\end{PuzzleClues}\newpage\section*{Krzyżówka 24}

\noindent\begin{Puzzle}{23}{30}|*	|[1][S]\darr	|*	|*	|*	|*	|*	|*	|*	|*	|*	|*	|*	|*	|*	|*	|[2][S]\darr	|*	|*	|*	|[3][S]\darr	|*	|*	|*	|.
|*	|c	|[4][S]\drarr	|b	|i	|a	|ł	|o	|g	|a	|r	|d	|z	|i	|a	|n	|k	|a	|*	|*	|f	|*	|*	|[5][S]\darr	|.
|[6][S]\drarr	|s	|t	|e	|p	|n	|i	|a	|r	|k	|a	|[][,]{ }	|ś	|r	|e	|d	|n	|i	|a	|*	|i	|*	|[7][S]\darr	|o	|.
|u	|o	|r	|*	|*	|[8][S]\darr	|*	|*	|*	|*	|*	|*	|*	|*	|*	|*	|a	|[9][S]\darr	|[10][S]\darr	|*	|l	|*	|w	|s	|.
|z	|k	|i	|[11][S]\drarr	|l	|i	|d	|h	|o	|l	|m	|*	|[12][S]\darr	|*	|*	|*	|b	|e	|d	|[13][S]\darr	|i	|*	|a	|z	|.
|d	|*	|l	|d	|*	|m	|[14][S]\drarr	|m	|u	|s	|z	|t	|r	|a	|*	|[15][S]\drarr	|s	|t	|e	|m	|p	|e	|l	|*	|.
|o	|[16][S]\darr	|o	|*	|[17][S]\darr	|i	|f	|[18][S]\rarr	|o	|w	|e	|r	|o	|l	|*	|o	|t	|r	|t	|a	|i	|[19][S]\darr	|l	|*	|.
|l	|i	|f	|[20][S]\rarr	|s	|e	|r	|m	|o	|n	|i	|z	|m	|*	|*	|p	|r	|u	|e	|j	|n	|c	|a	|*	|.
|n	|n	|o	|*	|z	|l	|y	|*	|*	|[21][S]\rarr	|g	|r	|a	|b	|o	|ł	|u	|s	|k	|*	|k	|z	|r	|*	|.
|i	|f	|z	|*	|y	|i	|c	|[22][S]\rarr	|b	|a	|ż	|a	|n	|t	|*	|y	|p	|k	|t	|*	|a	|y	|o	|*	|.
|e	|l	|a	|*	|p	|n	|*	|[23][S]\rarr	|t	|a	|t	|k	|o	|*	|[24][S]\darr	|w	|*	|a	|o	|*	|*	|n	|o	|*	|.
|n	|a	|u	|*	|*	|*	|*	|*	|[25][S]\drarr	|j	|e	|d	|w	|a	|b	|*	|[26][S]\darr	|*	|r	|*	|*	|n	|[][,]{ }	|*	|.
|i	|c	|r	|[27][S]\drarr	|w	|a	|t	|y	|k	|a	|n	|k	|a	|*	|y	|*	|k	|[28][S]\darr	|*	|*	|[29][S]\darr	|i	|c	|*	|.
|e	|j	|*	|k	|*	|[30][S]\rarr	|t	|r	|u	|f	|l	|a	|*	|*	|l	|*	|a	|b	|*	|*	|k	|k	|z	|*	|.
|*	|a	|*	|o	|*	|*	|*	|*	|c	|*	|*	|[31][S]\darr	|*	|*	|i	|*	|p	|a	|*	|*	|e	|[][,]{ }	|a	|*	|.
|*	|[][,]{ }	|*	|d	|[32][S]\darr	|*	|*	|*	|[][,]{ }	|*	|[33][S]\rarr	|j	|e	|r	|n	|b	|e	|r	|g	|*	|n	|n	|r	|*	|.
|*	|b	|*	|*	|d	|*	|[34][S]\rarr	|v	|i	|p	|*	|ę	|*	|*	|a	|*	|r	|w	|*	|*	|o	|i	|n	|*	|.
|*	|a	|[35][S]\drarr	|p	|o	|d	|l	|e	|s	|z	|c	|z	|y	|k	|*	|*	|*	|y	|*	|*	|z	|e	|y	|*	|.
|*	|z	|c	|[36][S]\darr	|r	|[37][S]\darr	|*	|*	|l	|[38][S]\rarr	|t	|y	|r	|a	|n	|e	|k	|*	|*	|*	|o	|c	|*	|*	|.
|*	|o	|h	|l	|u	|f	|[39][S]\rarr	|t	|a	|j	|*	|k	|[40][S]\rarr	|z	|ł	|o	|t	|y	|[][,]{ }	|w	|i	|e	|k	|*	|.
|[41][S]\drarr	|w	|i	|e	|t	|r	|z	|e	|n	|i	|e	|[][,]{ }	|i	|l	|a	|s	|t	|e	|*	|*	|k	|n	|*	|*	|.
|z	|a	|r	|n	|a	|a	|*	|*	|d	|*	|[42][S]\rarr	|b	|e	|z	|r	|ą	|b	|e	|k	|*	|*	|o	|*	|*	|.
|a	|*	|g	|z	|*	|j	|[43][S]\rarr	|c	|z	|a	|p	|e	|l	|k	|a	|[][,]{ }	|z	|ł	|o	|t	|a	|w	|a	|*	|.
|p	|*	|i	|*	|*	|e	|*	|*	|k	|[44][S]\rarr	|a	|n	|o	|d	|a	|*	|*	|[45][S]\rarr	|a	|n	|t	|y	|k	|*	|.
|i	|*	|s	|[46][S]\rarr	|p	|r	|e	|m	|i	|a	|[][,]{ }	|g	|ó	|r	|s	|k	|a	|*	|*	|*	|*	|*	|[47][S]\darr	|*	|.
|a	|*	|*	|*	|*	|*	|*	|*	|*	|[48][S]\rarr	|k	|a	|m	|l	|o	|o	|p	|s	|*	|*	|*	|*	|p	|*	|.
|n	|*	|[49][S]\rarr	|k	|a	|n	|g	|u	|r	|[][,]{ }	|o	|l	|b	|r	|z	|y	|m	|i	|*	|*	|*	|*	|l	|*	|.
|*	|*	|*	|*	|*	|*	|*	|*	|*	|*	|[50][S]\rarr	|s	|m	|u	|ż	|k	|a	|[][,]{ }	|l	|e	|ś	|n	|a	|*	|.
|*	|*	|[51][S]\rarr	|a	|t	|t	|a	|c	|h	|[][S]é	|[][,]{ }	|k	|u	|l	|t	|u	|r	|a	|l	|n	|y	|*	|m	|*	|.
|*	|*	|*	|*	|[52][S]\rarr	|ż	|a	|g	|l	|o	|w	|i	|e	|c	|[][,]{ }	|s	|k	|a	|l	|a	|r	|*	|a	|*	|.
|*	|*	|*	|*	|*	|*	|*	|*	|*	|*	|*	|*	|*	|*	|*	|*	|*	|*	|*	|*	|*	|*	|*	|*	|.\end{Puzzle}

\newpage

\begin{PuzzleClues}{\textbf{Poziome}\\}\Clue{4}{}{mieszkanka Białogardu}
\Clue{6}{}{Eremias intermedia - gatunek gada z rodziny jaszczurkowatych, występujący w środkowej Azji}
\Clue{11}{}{ur. w 1921 r., szwedzki kompozytor i dyrygent; utwory orkiestrowe, kameralne, wokalne}
\Clue{14}{}{ćwiczenia wojskowe mające na celu sprawne wykonywanie rozkazów przez żołnierzy}
\Clue{15}{}{stojak drewniany będący elementami rusztowania}
\Clue{18}{}{strój damski o  fasonie przypominającym kombinezon}
\Clue{20}{}{pogląd Piotra Abelarda, wyrażający w sporze o uniwersalia stanowisko pośrednie (uniwesralia nie tylko istnieją w umyśle, ale nie są przedmiotami istniejącymi realnie)}
\Clue{21}{}{GRUBOD21ÓB}
\Clue{22}{}{mięso bażanta (także: mięso przygotowane do spożycia, jako że bażanta serwuje się zwykle w całości)}
\Clue{23}{}{tatuś, pieszczotliwie, zdrobniale o tacie, ojcu}
\Clue{25}{}{gładkość, cecha tego, co ma fakturę podobną do jedwabiu}
\Clue{27}{}{mieszkanka Watykanu, kobieta pochodzenia watykańskiego}
\Clue{30}{}{czekoladka, zwykle kulista bądź o nieregularnym kształcie, wykonana z czekolady i masła lub śmietany}
\Clue{33}{}{narciarz włoski, czterokrotny mistrz i trzykrotny wicemistrz olimpijski z Cortina d'Ampezzo, Squaw Valley i Innsbrucku}
\Clue{34}{}{peptydowy hormon składający się z 28 reszt aminokwasowych; u człowieka produkowany w jelitach (komórki D1), trzustce i niektórych strukturach mózgu}
\Clue{35}{}{krąp - słodkowodna ryba z rodziny karpiowatych}
\Clue{38}{}{przenośnie o dziecku, któremu inni ludzie łatwo ustępują lub pieszczotliwie o człowieku, który czasem objawia trudny charakter}
\Clue{39}{}{mieszkaniec Tajlandii, człowiek pochodzenia tajskiego}
\Clue{40}{}{czas rozkwitu, najpełniejszego rozwoju}
\Clue{41}{}{wietrzenie fizyczne zachodzące na skutek nasiąkania (hydracji) i wysychania (dehydracji) skał ilastych}
\Clue{42}{}{Pyramidula - mech z rodziny skrętkowatych, którego jedynym przedstawicielem jest objęty ochroną w Polsce bezrąbek czterokanciasty}
\Clue{43}{}{czapla złotawa, czapelka, Bubulcus ibis - gatunek dużego ptaka brodzącego z rodziny czaplowatych (Ardeidae)}
\Clue{44}{}{elektroda przyjmująca elektrody, w ogniwie galwanicznym jest elektrodą ujemną}
\Clue{45}{}{starożytność; okres w historii Bliskiego Wschodu, Europy i Afryki Północnej obejmujący dzieje tych regionów od powstania pierwszych cywilizacji do około V wieku n.e}
\Clue{46}{}{podjazd, miejsce, gdzie rozgrywa się sprint kolarzy pod górę w wyścigu kolarskim}
\Clue{48}{}{miasto w Kanadzie (Kolumbia Brytyjska) nad rzeką Thompson; ośrodek handlowy, węzeł kolejowy}
\Clue{49}{}{Macropus giganteus - duży torbacz z rodziny kangurowatych, endemit kontynentu australijskiego, jeden z najbardziej popularnych kangurów, często mylony z kangurem szarym, łapanym dla skór i mięsa już przez Aborygenów; zasięg jego występowania jest ograniczony do kontynentu australijskiego i wysp wchodzących w skład stanu Tasmania}
\Clue{50}{}{Sicista betulina - niewielki gatunek gryzonia z rodziny skoczkowatych, występujący  w Europie (od Skandynawii i Niemiec) po środkową Azję; w Polsce można go spotkać w północno-wschodnich rejonach oraz w Karpatach}
\Clue{51}{}{urzędnik oddelegowany w ambasadzie do zajmowania się sprawami kultury}
\Clue{52}{}{ryba słodkowodna z rodziny pielęgnicowatych}\end{PuzzleClues}

\begin{PuzzleClues}{\textbf{Pionowe}\\}\Clue{1}{}{malarz węgierski (1865-1961 ) wpływ postimpresjonizmu}
\Clue{2}{}{rasa koni gorącokrwistych umaszczonych wielobarwnie, zapoczątkowana w 1808 roku na bazie klaczy hiszpańskich}
\Clue{3}{}{granat produkowany i używany przez polską konspirację głównie w powstaniu warszawskim}
\Clue{4}{}{Trilophosaurus - roślinożerny gad z kladu Archosauromorpha, żyjącego w późnym triasie (karnik i noryk, około 225-215 milionów lat temu); zamieszkiwał obszary równikowej Pangei w jej zachodniej strefie}
\Clue{5}{}{obwodowe miasto w Kirgistanie w Kotlinie Fergańskiej}
\Clue{6}{}{dar do czegoś, predyspozycja przejawiana w jakiejś dziedzinie, talent}
\Clue{7}{}{Macropus bernardus - torbacz z rodziny kangurowatych; zamieszkuje lasy eukaliptusowe na Ziemi Arnhema}
\Clue{8}{}{duże osiedle na warszawskim Ursynowie}
\Clue{9}{}{przedstawicielka starożytnego ludu zamieszkującego Etrurię (w środkowej Italii)}
\Clue{10}{}{urządzenie do wykrywania i obserwacji jakiegoś zjawiska fizycznego}
\Clue{11}{}{litera alfabetu używana w numeracji porządkowej}
\Clue{12}{}{lekkoatletka rosyjska, mistrzyni olimpijska z Barcelony w biegu na 3 km}
\Clue{13}{}{przedstawiciel historycznego ludu Majów}
\Clue{14}{}{lekceważąco o Niemcu}
\Clue{15}{}{ruch cząsteczek wody wokół ciała stałego}
\Clue{16}{}{inflacja cen produktów konsumpcyjnych wyliczana po wyłączeniu niektórych z nich, np. z powodu niskiej stabilności cen}
\Clue{17}{}{Acipenser nudiventris - gatunek ryby z rodziny jesiotrowatych (Acipenseridae); szyp występuje w Morzu Czarnym, Morzu Azowskim oraz południowej części Morza Kaspijskiego i w Jeziorze Aralskim, w Dunaju tworzy niewędrowną formę słodkowodną, rozmnażającą się w Wagu, Cisie i Maruszy, pozostałe populacje szypów to ryby anadromiczne}
\Clue{19}{}{czynnik powodujący przesuwanie się krzywej popytu w górę lub w dół, np. moda, płeć, wiek, gusta}
\Clue{24}{}{wieloletnia, trwała roślina zielna}
\Clue{25}{}{rasa małego konia wyhodowana w Islandii, czasem wielkości kuców, większość rejestrów ras wspomina islandy jako konie; jedyna rasa koni w Islandii, popularna także  na świecie (spore populacje istnieją w Europie i Ameryce Północnej), ciągle używana do tradycyjnych prac gospodarczych w Islandii, do rekreacji, pokazów koni i wyścigów konnych}
\Clue{26}{}{KORSARZ, PIRAT; żeglarz walczący na własną rękę lub z upoważnienia króla uprawniony do łupienia wrogich statków}
\Clue{27}{}{ciąg znaków, które coś oznaczają, jeśli odczytać je według określonej zasady}
\Clue{28}{}{symbole, kolory charakterystyczne dla czegoś, utożsamiane z czymś, np. z państwem, instytucją lub klubem sportowym}
\Clue{29}{}{era, która rozpoczęła się ok. 66 mln lat temu (od wymierania kredowego) i trwa do dziś}
\Clue{31}{}{jeden z języków nowoindyjskich}
\Clue{32}{}{ryba atlantycka}
\Clue{35}{}{jezioro w Mongolii, wpływa do niego Dzabchan}
\Clue{36}{}{niemiecki geolog i podróżnik, badacz Afryki (1848-1925); przebył Afrykę w poprzek}
\Clue{37}{}{człowiek naiwny, taki, którego łatwo można oszukać, nabrać}
\Clue{41}{}{MYDLENIEC małe drzewo Ameryki Płd. uprawiane dla miąższu owoców zawierającego saponiny do produkcji mydła}
\Clue{47}{}{zabrudzenie, brudny ślad po czymś}\end{PuzzleClues}\newpage\section*{Krzyżówka 25}

\noindent\begin{Puzzle}{23}{28}|*	|*	|*	|*	|*	|*	|*	|*	|*	|*	|*	|*	|*	|*	|*	|*	|*	|*	|*	|[1][S]\darr	|*	|*	|[2][S]\darr	|*	|.
|*	|*	|*	|*	|*	|*	|*	|*	|*	|*	|*	|*	|[3][S]\drarr	|u	|i	|g	|e	|*	|*	|p	|*	|*	|k	|*	|.
|*	|*	|*	|*	|[4][S]\rarr	|r	|e	|j	|o	|w	|i	|e	|c	|*	|*	|*	|*	|[5][S]\darr	|*	|o	|*	|[6][S]\darr	|o	|*	|.
|*	|*	|*	|*	|*	|*	|[7][S]\rarr	|e	|u	|s	|t	|a	|z	|j	|a	|*	|*	|c	|*	|k	|*	|r	|m	|*	|.
|*	|*	|[8][S]\rarr	|p	|r	|z	|y	|g	|o	|t	|o	|w	|a	|n	|i	|e	|*	|e	|*	|r	|[9][S]\darr	|o	|ó	|*	|.
|*	|[10][S]\drarr	|m	|y	|s	|z	|[][,]{ }	|c	|y	|p	|r	|y	|j	|s	|k	|a	|*	|b	|[11][S]\darr	|y	|a	|z	|r	|*	|.
|*	|a	|[12][S]\rarr	|p	|r	|o	|j	|e	|k	|c	|y	|j	|n	|o	|ś	|ć	|*	|i	|p	|w	|k	|s	|k	|*	|.
|[13][S]\drarr	|p	|o	|d	|r	|o	|d	|z	|a	|j	|*	|*	|i	|[14][S]\drarr	|k	|a	|m	|o	|r	|y	|s	|t	|a	|*	|.
|e	|o	|[15][S]\drarr	|m	|a	|c	|i	|e	|r	|z	|[][,]{ }	|s	|k	|a	|l	|a	|r	|n	|a	|*	|j	|r	|*	|*	|.
|k	|s	|p	|[16][S]\drarr	|b	|r	|y	|c	|z	|e	|s	|y	|*	|n	|*	|[17][S]\darr	|*	|*	|w	|*	|o	|z	|*	|*	|.
|o	|t	|o	|p	|[18][S]\darr	|*	|*	|*	|*	|*	|*	|*	|[19][S]\rarr	|y	|a	|s	|s	|*	|o	|*	|m	|e	|*	|*	|.
|n	|e	|l	|ó	|t	|*	|*	|*	|*	|*	|*	|*	|*	|ż	|*	|c	|*	|*	|[][,]{ }	|*	|a	|ń	|*	|*	|.
|o	|l	|e	|ł	|r	|*	|*	|*	|*	|*	|*	|[20][S]\darr	|*	|e	|*	|h	|[21][S]\darr	|*	|p	|*	|t	|[][,]{ }	|*	|*	|.
|m	|*	|[][,]{ }	|g	|o	|*	|[22][S]\drarr	|c	|z	|e	|r	|p	|a	|k	|*	|o	|t	|*	|o	|*	|[][,]{ }	|k	|*	|*	|.
|i	|[23][S]\drarr	|s	|o	|c	|j	|o	|g	|r	|a	|f	|i	|a	|*	|*	|d	|r	|*	|w	|*	|a	|o	|*	|*	|.
|a	|r	|k	|l	|h	|*	|d	|*	|[24][S]\rarr	|o	|b	|r	|ą	|c	|z	|k	|a	|*	|i	|*	|r	|m	|*	|*	|.
|[][,]{ }	|a	|a	|f	|u	|[25][S]\drarr	|t	|r	|a	|n	|s	|i	|t	|*	|*	|o	|c	|*	|e	|*	|c	|o	|*	|*	|.
|p	|c	|l	|*	|s	|b	|w	|*	|[26][S]\darr	|*	|*	|[][S]-	|*	|*	|*	|w	|z	|*	|l	|*	|h	|r	|*	|*	|.
|o	|z	|a	|*	|*	|u	|a	|[27][S]\drarr	|s	|t	|e	|p	|ó	|w	|k	|a	|*	|[28][S]\darr	|a	|*	|i	|y	|*	|*	|.
|z	|y	|r	|*	|*	|i	|r	|t	|a	|*	|*	|i	|*	|*	|*	|n	|*	|s	|c	|*	|m	|[][,]{ }	|*	|*	|.
|y	|n	|n	|*	|*	|c	|z	|e	|d	|[29][S]\rarr	|d	|r	|ó	|b	|*	|i	|*	|o	|z	|*	|e	|s	|*	|*	|.
|t	|a	|e	|*	|[30][S]\rarr	|k	|a	|c	|z	|e	|n	|i	|c	|a	|*	|e	|*	|f	|o	|*	|d	|e	|*	|*	|.
|y	|*	|*	|*	|*	|*	|c	|z	|*	|*	|*	|*	|*	|*	|*	|*	|*	|i	|w	|*	|e	|r	|*	|*	|.
|w	|[31][S]\rarr	|r	|a	|u	|i	|z	|u	|c	|h	|i	|d	|y	|*	|*	|*	|*	|s	|e	|[32][S]\darr	|s	|c	|*	|*	|.
|n	|[33][S]\drarr	|p	|ł	|y	|n	|[][,]{ }	|s	|u	|r	|o	|w	|i	|c	|z	|y	|*	|t	|*	|g	|a	|a	|*	|*	|.
|a	|c	|*	|*	|*	|*	|m	|z	|*	|*	|[34][S]\rarr	|s	|z	|r	|e	|n	|i	|a	|w	|a	|*	|*	|*	|*	|.
|*	|e	|*	|*	|*	|*	|p	|k	|*	|*	|*	|*	|*	|*	|*	|*	|*	|*	|*	|z	|*	|*	|*	|*	|.
|*	|p	|*	|*	|*	|*	|[][S]3	|a	|*	|*	|*	|*	|*	|*	|*	|*	|*	|*	|*	|*	|*	|*	|*	|*	|.
|*	|*	|[35][S]\rarr	|r	|a	|z	|*	|*	|*	|*	|*	|*	|*	|*	|*	|*	|*	|*	|*	|*	|*	|*	|*	|*	|.\end{Puzzle}

\newpage

\begin{PuzzleClues}{\textbf{Poziome}\\}\Clue{3}{}{miasto w środkowej Gwinei}
\Clue{4}{}{wieś w Polsce położona w województwie wielkopolskim, w powiecie wągrowieckim}
\Clue{7}{}{długookresowe zmiany poziomu wód, spowodowane bilansem obiegu wody w przyrodzie, zmianami klimatycznymi}
\Clue{8}{}{uprzedzenie kogoś o czymś, uczynienie go gotowym na jakieś wydarzenie, wiadomość itp}
\Clue{10}{}{Mus cypriacus - gatunek małego ssaka z rodziny myszowatych; występuje jedynie na Cyprze}
\Clue{12}{}{to, że coś jest projekcyjne, cecha czegoś projekcyjnego}
\Clue{13}{}{grupa obiektów lub zjawisk jednego rodzaju, wyróżniana ze względu na pewne wspólne cechy}
\Clue{14}{}{członek kamory}
\Clue{15}{}{macierz diagonalna, w której wszystkie elementy na przekątnej głównej są równe}
\Clue{16}{}{spodnie do konnej jazdy, u dołu obcisłe, sznurowane, od kolan bufiaste}
\Clue{19}{}{styl w muzyce, łączący elementy współczesnej muzyki improwizowanej, jazzu, punk rocka i folku}
\Clue{22}{}{organ roboczy czerparek, koparek w postaci naczynia do przenoszenia materiałów sypkich}
\Clue{23}{}{nauka pomocnicza socjologii, zajmująca się opisem społeczeństwa i zachodzących w nim zjawisk}
\Clue{24}{}{obrączka z tworzyw sztucznych, często z wbudowanym nadajnikiem, zakładana na nogę ptakom w celu monitorowania ich liczebności i kierunków przemieszczania}
\Clue{25}{}{model samochodu dostawczego marki Ford produkowany od 1965 roku występujący w różnych typach nadwozi od dostawczego aż po 17-osobowego busa}
\Clue{27}{}{stepówka - but do stepowania; zwykle w liczbie mnogiej}
\Clue{29}{}{ptaki hodowane dla jaj, mięsa i pierza; kury, indyki, kaczki, perlice, gęsi}
\Clue{30}{}{Lepas - rodzaj stawonogów, należących do gromady wąsonogów}
\Clue{31}{}{Rauisuchidae - rodzina dużych drapieżnych triasowych archozaurów z rzędu rauizuchów (Rauisuchia); obejmuje największych i najbardziej zaawansowanych przedstawicieli rauizuchów}
\Clue{33}{}{płyn przenikający z komórek do jam ciała; składa się z wody, granulocytów, limfocytów, komórek wyściółki jam ciała, włóknika}
\Clue{34}{}{wieś w Polsce położona w województwie wielkopolskim, w powiecie poznańskim, w gminie Komorniki}
\Clue{35}{}{chwila, kiedy coś się zdarzyło/zdarzy, okoliczność}\end{PuzzleClues}

\begin{PuzzleClues}{\textbf{Pionowe}\\}\Clue{1}{}{pierwsza zesklerotyzowana para skrzydeł chrząszczy, owadów i pluskwiaków}
\Clue{2}{}{małe pomieszczenie, które najczęściej służy do przechowywania  czegoś}
\Clue{3}{}{naczynie, które służy do gotowania wody}
\Clue{5}{}{lekarstwo z witaminą C}
\Clue{6}{}{rozszerzenie lewej komory lub jednej z jam serca}
\Clue{9}{}{aksjomat głoszący, że każdy odcinek jest krótszy od pewnej wielokrotności długości każdego innego odcinka}
\Clue{10}{}{piłkarz, obecnie trener, prowadził reprezentację, ostatnio Wisła Kraków i Górnik Zabrze}
\Clue{11}{}{niepublikowane, wewnętrzne interpretacje, okólniki i wytyczne, o wątpliwej podstawie prawnej, które jednak wywierają realny wpływ na praktykę funkcjonowania administracji publicznej}
\Clue{13}{}{dział ekonomii zajmujący się opisem praw regulujących zjawiska ekonomiczne bez ich wartościowania i tworzeniem metod umożliwiających opisanie tych zjawisk}
\Clue{14}{}{przyprawa, owoc (rozłupnia) anyżu (biedrzeńca anyżu), znana także w medycynie jako źródło olejku o silnych właściwościach wykrztuśnych, rozkurczowych, wiatropędnych i odkażających}
\Clue{15}{}{przypisanie każdemu punktowi pewnego obszaru pewnej wielkości skalarnej (w matematyce - liczby, w fizyce zazwyczaj wielkości mianowanej)}
\Clue{16}{}{wkładany przez głowę sweter z sięgającym połowy szyi i przylegającym do niej kołnierzem (krótszym niż w przypadku golfu)}
\Clue{17}{}{sposób podejścia narciarskiego na stromy stok}
\Clue{18}{}{ślimak przodoskrzelny ze stożkowatą dekoracyjną muszlą używaną do wyrobu galanterii}
\Clue{20}{}{ostra w smaku mała papryczka, owoc rośliny o tej samej nazwie}
\Clue{21}{}{ptak wodny; poszczególne gatunki tego ptaka klasyfikowane są w taksonomii biologicznej w obrębie plemienia traczy (Mergini), w podrodzinie kaczek (Anatinae), w rodzinie kaczkowatych (Anatidae)}
\Clue{22}{}{przenośne urządzenie służące do katalogowania i odsłuchiwania plików dźwiękowych}
\Clue{23}{}{żartobliwie lub z sympatią o raku}
\Clue{25}{}{amerykańska marka samochodów osobowych, która obecnie należy do koncernu General Motors}
\Clue{26}{}{część łodzi rybackiej zawierająca wodę, przeznaczona do przechowywania złowionych ryb}
\Clue{27}{}{niewielka teczka - prostokątne, płaskie, ochronne opakowanie np. na dokumenty}
\Clue{28}{}{metaforycznie o człowieku, który manipuluje innymi, posługując się nieprawdziwymi argumentami i przesłankami}
\Clue{32}{}{paliwo}
\Clue{33}{}{proste narzędzie do ręcznego omłotu zbóż}\end{PuzzleClues}\newpage\section*{Krzyżówka 26}

\noindent\begin{Puzzle}{17}{29}|*	|*	|*	|*	|*	|*	|*	|[1][S]\darr	|*	|*	|*	|*	|*	|*	|*	|[2][S]\darr	|*	|*	|.
|*	|*	|*	|*	|*	|*	|*	|ś	|*	|[3][S]\rarr	|k	|o	|m	|a	|t	|s	|u	|*	|.
|*	|*	|*	|[4][S]\drarr	|a	|s	|y	|r	|y	|j	|s	|k	|i	|*	|[5][S]\darr	|u	|[6][S]\darr	|*	|.
|*	|*	|[7][S]\rarr	|k	|a	|i	|r	|u	|a	|n	|*	|*	|*	|*	|p	|s	|w	|*	|.
|*	|*	|*	|l	|[8][S]\drarr	|i	|n	|t	|a	|r	|s	|j	|a	|*	|o	|h	|y	|*	|.
|*	|*	|*	|a	|w	|*	|*	|a	|[9][S]\darr	|[10][S]\drarr	|d	|ż	|e	|j	|r	|a	|n	|*	|.
|*	|*	|*	|u	|a	|*	|*	|[][,]{ }	|s	|m	|*	|[11][S]\darr	|[12][S]\darr	|*	|o	|r	|a	|*	|.
|*	|*	|*	|z	|t	|*	|[13][S]\drarr	|s	|t	|o	|j	|a	|k	|*	|z	|n	|l	|*	|.
|*	|*	|*	|u	|e	|*	|d	|ł	|y	|d	|[14][S]\darr	|w	|l	|[15][S]\darr	|u	|i	|a	|*	|.
|*	|*	|*	|l	|r	|*	|e	|o	|l	|r	|a	|a	|i	|i	|m	|a	|z	|*	|.
|*	|*	|*	|a	|p	|*	|f	|n	|o	|o	|l	|r	|n	|k	|i	|*	|c	|*	|.
|*	|[16][S]\darr	|*	|[][,]{ }	|r	|*	|e	|e	|n	|l	|e	|*	|*	|a	|e	|*	|a	|*	|.
|*	|k	|*	|z	|o	|*	|r	|c	|*	|o	|g	|*	|*	|r	|n	|*	|*	|*	|.
|*	|a	|*	|a	|o	|[17][S]\darr	|e	|z	|[18][S]\drarr	|t	|r	|y	|b	|*	|i	|[19][S]\darr	|*	|*	|.
|*	|k	|*	|p	|f	|p	|n	|n	|p	|k	|i	|*	|[20][S]\darr	|*	|e	|b	|*	|*	|.
|*	|a	|*	|o	|*	|u	|t	|i	|o	|a	|a	|[21][S]\darr	|k	|*	|[][,]{ }	|e	|*	|*	|.
|*	|d	|[22][S]\rarr	|r	|w	|d	|*	|k	|m	|[][,]{ }	|*	|k	|o	|[23][S]\drarr	|p	|t	|*	|*	|.
|*	|u	|*	|o	|*	|ł	|*	|o	|n	|z	|*	|i	|r	|a	|ł	|o	|*	|*	|.
|*	|[][,]{ }	|[24][S]\darr	|w	|*	|o	|*	|w	|i	|i	|*	|e	|p	|n	|a	|n	|*	|*	|.
|*	|w	|a	|a	|*	|*	|*	|a	|k	|e	|*	|j	|u	|s	|c	|*	|*	|*	|.
|*	|y	|l	|*	|*	|*	|*	|*	|*	|l	|*	|d	|s	|a	|o	|*	|*	|*	|.
|[25][S]\drarr	|s	|u	|m	|a	|[][,]{ }	|z	|e	|r	|o	|w	|a	|*	|m	|w	|*	|*	|*	|.
|d	|p	|m	|*	|[26][S]\drarr	|t	|e	|r	|e	|n	|*	|*	|*	|b	|e	|*	|*	|*	|.
|i	|o	|i	|[27][S]\rarr	|s	|i	|o	|d	|ł	|o	|*	|*	|*	|l	|*	|*	|*	|*	|.
|e	|w	|n	|[28][S]\drarr	|t	|r	|o	|j	|a	|c	|z	|e	|k	|*	|*	|*	|*	|*	|.
|s	|a	|i	|s	|o	|[29][S]\rarr	|m	|u	|s	|z	|k	|a	|t	|*	|*	|*	|*	|*	|.
|e	|*	|u	|z	|w	|*	|*	|[30][S]\rarr	|q	|u	|i	|n	|e	|t	|*	|*	|*	|*	|.
|l	|*	|m	|l	|e	|*	|[31][S]\rarr	|a	|l	|b	|e	|r	|t	|a	|*	|*	|*	|*	|.
|*	|*	|*	|*	|*	|*	|[32][S]\rarr	|p	|i	|a	|s	|t	|a	|*	|*	|*	|*	|*	|.
|[33][S]\rarr	|f	|a	|k	|t	|o	|r	|i	|a	|*	|*	|*	|*	|*	|*	|*	|*	|*	|.\end{Puzzle}

\newpage

\begin{PuzzleClues}{\textbf{Poziome}\\}\Clue{3}{}{miasto w Japonii (Honsiu), nad Morzem Japońskim, uzdrowisko, gorące źródła}
\Clue{4}{}{język należący do grupy semickiej, którym posługuje się około 200 tys. Asyryjczyków zamieszkujących głównie Irak, Iran i Syrię}
\Clue{7}{}{miasto we wsch. Tunezji; ośrodek handlu i rzemiosła, miejsce kultu religijnego muzułmanów}
\Clue{8}{}{system zdobienia przedmiotów drewnianych znanych od starożytności}
\Clue{10}{}{Gazella subgutturosa - gatunek gazeli z rodziny krętorogich; występuje na pustynnych i półpustynnych obszarach Iranu, Afganistanu, zachodniego Pakistanu, południowej Mongolii i Chin, a także w Azerbejdżanie, Kazachstanie, Uzbekistanie, Kirgistanie, Tadżykistanie i Turkmenistanie}
\Clue{13}{}{wyspecjalizowany przedmiot, który służy do potrzymywania czegoś w ustalonej pozycji, często tak, jak jest dla tego czegoś bezpiecznie}
\Clue{18}{}{kategoria gramatyczna}
\Clue{22}{}{międzywojenny, turystyczno-szkolny, polski samolot}
\Clue{23}{}{punkt typograficzny - miara wielkości czcionek i innych elementów typograficznych }
\Clue{25}{}{w grach: suma stała, w której wypłaty graczy równoważą się (ich suma wynosi zero)}
\Clue{26}{}{przenośnie: arena wydarzeń}
\Clue{27}{}{część rzędu końskiego stanowiąca siedzenie przeznaczone dla jeźdźca}
\Clue{28}{}{przedmiot składający się z trzech części lub łatwy do podzielenia na trzy części}
\Clue{29}{}{muskat - rodzaj słodkiego, wzmacnianego białego wina}
\Clue{30}{}{(1803-75), francuski pisarz, historyk, polityk „.Napoleon”}
\Clue{31}{}{prowincja w płd-zach. Kanadzie, powierzchnia 661 tys.km2, główne miasta: Edmonton (stolica) i Calgary}
\Clue{32}{}{część koła przylegająca do czopa wału lub osi}
\Clue{33}{}{osada handlowa w krajach kolonialnych}\end{PuzzleClues}

\begin{PuzzleClues}{\textbf{Pionowe}\\}\Clue{1}{}{śruta stanowiąca pozostałość z nasion słonecznikowych, z których ekstrahuje się olej}
\Clue{2}{}{restauracja serwująca sushi}
\Clue{4}{}{minimalny procent poparcia wyborców uprawniający komitet wyborczy do uczestniczenia w rozdziale mandatów}
\Clue{5}{}{ugoda pomiędzy pracodawcą a grupą pracowników, na podstawie której ustala się wynegocjowane przez strony stawki wynagrodzeń}
\Clue{6}{}{ktoś, kto coś wynalazł}
\Clue{8}{}{nieprzemakalny płaszcz z tej tkaniny}
\Clue{9}{}{poliamidowe tworzywo sztuczne używane do wyrobu włókien syntetycznych bądź tkanina z tego włókna}
\Clue{10}{}{Eunymphicus uvaeensis - gatunek ptaka z rodziny papugowatych (Psittacidae), z podrodziny papug wschodnich (Psittaculinae)}
\Clue{11}{}{członek grupy Awarów - historycznego koczowniczego ludu zamieszkującego Europę w IV wieku}
\Clue{12}{}{porcja alkoholu, którą pije się rano, gdy ma się kaca w nadziei, że przejdzie}
\Clue{13}{}{główne koło orbitalne}
\Clue{14}{}{(1909-67), pisarz peruwiański, powieści w obronie praw Indian; „Złoty wąż”}
\Clue{15}{}{planetoida odkryta w 1949 r.; zbliża się do Ziemi na odległość ok. 7 ml km}
\Clue{16}{}{Cacatua sulphurea abbotti - podgatunek ptaka wyróżniony w obrębie gatunku kakadu żółtolica (Cacatua sulphurea)}
\Clue{17}{}{więzienie}
\Clue{18}{}{coś reprezentatywnego i ważnego dla danej kultury, o dużej wartości, np. historycznej}
\Clue{19}{}{materiał budowlany, kompozyt powstały ze zmieszania spoiwa (cementu) i wypełniacza (kruszywo) oraz ewentualnych domieszek nadających pożądane cechy}
\Clue{20}{}{zrąb - zasadnicza część czegoś}
\Clue{21}{}{w gwarze złodziei, przestępczej: kieszeń}
\Clue{23}{}{scena zbiorowa w przedstawieniu (teatralnym albo operowym)}
\Clue{24}{}{glin o czystości technicznej; stop zawierający różne ilości zanieczyszczeń, zależnie od metody otrzymywania}
\Clue{25}{}{wysokoprężny spalinowy silnik tłokowy}
\Clue{26}{}{(1811-96), pisarka amerykańska; „Chata wuja Toma”}
\Clue{28}{}{kod ISO 4217 lilangeni}\end{PuzzleClues}\newpage\section*{Krzyżówka 27}

\noindent\begin{Puzzle}{15}{28}|*	|[1][S]\drarr	|m	|u	|f	|t	|y	|*	|*	|*	|*	|[2][S]\drarr	|l	|e	|m	|*	|.
|[3][S]\drarr	|w	|a	|r	|k	|o	|c	|z	|*	|[4][S]\darr	|[5][S]\darr	|o	|[6][S]\darr	|[7][S]\darr	|*	|*	|.
|s	|o	|*	|[8][S]\darr	|[9][S]\drarr	|f	|j	|d	|*	|a	|z	|b	|n	|j	|*	|[10][S]\darr	|.
|u	|m	|[11][S]\darr	|p	|s	|[12][S]\drarr	|r	|a	|j	|t	|a	|r	|i	|a	|*	|w	|.
|m	|b	|g	|r	|e	|l	|*	|*	|*	|y	|j	|a	|c	|m	|*	|i	|.
|i	|a	|r	|z	|g	|o	|[13][S]\darr	|[14][S]\darr	|[15][S]\darr	|p	|ą	|z	|h	|a	|*	|e	|.
|k	|t	|u	|e	|m	|g	|s	|n	|p	|i	|k	|[][,]{ }	|o	|[][,]{ }	|*	|c	|.
|*	|[][,]{ }	|s	|r	|e	|i	|z	|a	|r	|a	|n	|k	|l	|p	|*	|z	|.
|[16][S]\drarr	|s	|z	|y	|n	|k	|a	|r	|z	|*	|i	|l	|s	|ł	|*	|k	|.
|k	|z	|k	|w	|t	|a	|r	|ó	|e	|*	|ę	|i	|o	|a	|[17][S]\darr	|o	|.
|u	|o	|i	|k	|a	|[][,]{ }	|p	|d	|ż	|*	|c	|n	|n	|s	|s	|[][,]{ }	|.
|r	|r	|[][,]{ }	|a	|c	|m	|a	|[][,]{ }	|y	|*	|i	|i	|*	|z	|t	|s	|.
|k	|s	|n	|*	|j	|a	|n	|w	|c	|*	|e	|c	|*	|c	|r	|k	|.
|ó	|t	|a	|*	|a	|t	|k	|y	|i	|*	|*	|z	|*	|z	|z	|r	|.
|w	|k	|[][,]{ }	|*	|*	|e	|a	|b	|e	|*	|*	|n	|[18][S]\darr	|o	|a	|z	|.
|k	|o	|w	|*	|*	|m	|*	|r	|*	|*	|*	|y	|ś	|w	|ł	|e	|.
|a	|n	|i	|*	|*	|a	|*	|a	|*	|*	|*	|*	|l	|a	|[][,]{ }	|l	|.
|*	|o	|e	|*	|*	|t	|[19][S]\darr	|n	|[20][S]\drarr	|f	|i	|g	|a	|*	|s	|o	|.
|*	|s	|r	|*	|*	|y	|s	|y	|m	|*	|*	|[21][S]\darr	|d	|[22][S]\darr	|a	|w	|.
|*	|y	|z	|*	|[23][S]\darr	|c	|z	|*	|o	|*	|[24][S]\darr	|k	|[][,]{ }	|c	|m	|e	|.
|*	|[][,]{ }	|b	|*	|c	|z	|p	|[25][S]\drarr	|l	|e	|s	|o	|t	|h	|o	|*	|.
|*	|k	|i	|[26][S]\drarr	|a	|n	|i	|m	|e	|*	|o	|t	|o	|a	|b	|*	|.
|*	|r	|e	|ł	|s	|a	|l	|u	|s	|*	|f	|e	|r	|t	|ó	|*	|.
|*	|e	|*	|a	|t	|*	|e	|r	|k	|*	|i	|w	|o	|a	|j	|*	|.
|*	|f	|*	|p	|i	|*	|c	|*	|i	|*	|s	|k	|w	|ń	|c	|*	|.
|*	|f	|*	|a	|n	|*	|z	|*	|n	|*	|t	|a	|y	|s	|z	|*	|.
|*	|t	|*	|c	|g	|*	|k	|*	|*	|*	|a	|*	|*	|k	|y	|*	|.
|*	|a	|*	|z	|*	|*	|a	|*	|*	|*	|*	|*	|*	|a	|*	|*	|.
|*	|*	|*	|*	|*	|*	|*	|*	|*	|*	|*	|*	|*	|*	|*	|*	|.\end{Puzzle}

\newpage

\begin{PuzzleClues}{\textbf{Poziome}\\}\Clue{1}{}{MUFTI; uczony, teolog, znawca islamu}
\Clue{2}{}{twórczość Lema}
\Clue{3}{}{wąski snop czegoś (np. światła lub cienia) lub pasmo pyłu, dymu pozostawiane przez coś}
\Clue{9}{}{kod ISO 4217 dolara Fidżi}
\Clue{12}{}{typ kawalerii używającej w walce przede wszystkim pistoletów; ukształtował się wraz z arkebuzerami w połowie XVI wieku w Niemczech, w związku z rozwojem broni palnej i związaną z tym utratą znaczenia ciężkiej jazdy}
\Clue{16}{}{właściciel szynku}
\Clue{20}{}{gest zaciśniętej pięści, który pokazuje się, gdy chce się komuś zakomunikować, że nie otrzyma tego, czego się spodziewał}
\Clue{25}{}{państwo, enklawa w Południowej Afryce}
\Clue{26}{}{japoński film animowany}\end{PuzzleClues}

\begin{PuzzleClues}{\textbf{Pionowe}\\}\Clue{1}{}{wombat australijski, wombat północny,  Lasiorhinus krefftii - krytycznie zagrożony wyginięciem gatunek torbacza z rodziny wombatowatych (Vombatidae), masowo poławiany ze względu na atrakcyjne futro osiągające wysokie ceny; obecnie żyje ok. 70 wombatów australijskich, zamieszkujących  Epping Forest National Park w środkowym Queensland}
\Clue{2}{}{całokształt obserwacji poczynionych przez lekarza i informacji zebranych przez niego podczas wywiadu lekarskiego z chorym oraz w wyniku badania fizykalnego, które charakteryzują daną chorobę u konkretnego pacjenta}
\Clue{3}{}{poziomy bal drewniany stosowany w dawnym budownictwie wiejskim}
\Clue{4}{}{zespół nieprawidłowych cech w budowie komórek}
\Clue{5}{}{drobne potknięcie w mówieniu, zaburzenie płynności wypowiadanych słów zwykle w postaci krótkiej przerwy lub w postaci nieumyślnego małego przejęzyczenia}
\Clue{6}{}{William, ojciec Bena (1872-1949) angielski malarz i grafik: martwe natury, pejzaże. portrety, plakaty}
\Clue{7}{}{u mięczaków przestrzeń między workiem trzewiowym a płaszczem}
\Clue{8}{}{czynność wykonywana przy uprawie roślin, polegająca na usuwaniu roślin, przerzedzaniu ich}
\Clue{9}{}{podział ciała zwierząt dwubocznie symetrycznych wzdłuż głównej osi na szereg mniej lub bardziej niezależnych morfologicznie i fizjologicznie odcinków (metamerów) o powtarzającym się, pierwotnie podobnym planie budowy}
\Clue{10}{}{u ryb kostnoszkieletowych, jedna z dwóch ruchomych płytek, zamykająca komory skrzelowe}
\Clue{11}{}{obietnica bez nadziei na jej spełnienie}
\Clue{12}{}{dział matematyki, który wyodrębnił się jako samodzielna dziedzina na przełomie XIX i XX wieku, wraz z dążeniem do dogłębnego zbadania podstaw matematyki}
\Clue{13}{}{proces, sytuacja, zjawisko, kiedy ktoś szarpie się z myślami, kłopocze się czymś, rozmyśla o ważnych, dotkliwych rzeczach i waha się w podejmowaniu decyzji i wyborów; rozterka}
\Clue{14}{}{istniejący w judaizmie i większości wyznań chrześcijańskich pogląd o szczególnym związku Żydów z Bogiem, który jest niezmienny i przekazany w Starym i Nowym Testamencie}
\Clue{15}{}{przejęcie się, doświadczenie silnych emocji w związku np. z jakimś wydarzeniem}
\Clue{16}{}{broń myśliwska o ręcznie naciąganych kurkach}
\Clue{17}{}{skuteczny strzał w kierunku własnej bramki, przypadkowe uderzenie lub zmiana toru lotu piłki, którego efektem jest zdobycie gola (punktu) dla drużyny przeciwnej}
\Clue{18}{}{przypowierzchniowa warstwa wody zaburzonej przez ruch jednostki nawodnej}
\Clue{19}{}{zdrobniale lub z uczuciem o szpilce; damskim bucie na wysokim, cienko zakończonym obcasie}
\Clue{20}{}{atłasowa tkanina bawełniana imitująca skórę, gęsta, gruba ale miękka, mocna, z wierzchu drapana i strzyżona, przez co trochę podobna do aksamitu, praktyczna do wielu celów (w zależności od druboźci: krawiectwo lub meblarstwo), trudno przemakalna}
\Clue{21}{}{zdrobniale o poziomym pręcie ściągającym elementy konstrukcji, np. przeciwległe mury, słupy czy filary, chroniącym je przed rozchyleniem}
\Clue{22}{}{zatoka Morza Łaptiewów między Płw Tajmur a azjatyckim wybrzeżem Federacji Rosyjskiej}
\Clue{23}{}{wędkarstwo rzutowe}
\Clue{24}{}{metaforycznie o człowieku, który manipuluje innymi, posługując się nieprawdziwymi argumentami i przesłankami}
\Clue{25}{}{kod ISO 4217 rupii maurytyjskiej}
\Clue{26}{}{człowiek, który coś łapie, np. Super Mario łapacz gwiazd}\end{PuzzleClues}\newpage\section*{Krzyżówka 28}

\noindent\begin{Puzzle}{18}{32}|*	|*	|*	|*	|*	|*	|*	|*	|*	|*	|*	|*	|[1][S]\drarr	|r	|u	|p	|i	|a	|*	|.
|*	|*	|*	|[2][S]\rarr	|p	|r	|o	|p	|o	|r	|z	|e	|c	|*	|[3][S]\drarr	|l	|ó	|d	|*	|.
|*	|*	|*	|*	|*	|[4][S]\drarr	|k	|o	|r	|e	|k	|t	|o	|r	|n	|i	|a	|*	|*	|.
|[5][S]\drarr	|d	|o	|l	|e	|s	|i	|e	|n	|i	|e	|*	|l	|*	|e	|*	|*	|*	|*	|.
|a	|*	|[6][S]\rarr	|u	|s	|z	|t	|y	|k	|*	|*	|*	|t	|[7][S]\drarr	|g	|b	|u	|r	|*	|.
|r	|[8][S]\darr	|[9][S]\rarr	|v	|e	|c	|t	|r	|a	|*	|*	|*	|*	|z	|a	|*	|*	|*	|*	|.
|*	|p	|[10][S]\rarr	|r	|o	|z	|e	|t	|k	|a	|*	|[11][S]\rarr	|r	|a	|t	|a	|n	|*	|*	|.
|[12][S]\drarr	|k	|o	|c	|i	|e	|[][,]{ }	|o	|k	|o	|*	|*	|*	|k	|y	|*	|*	|*	|*	|.
|p	|s	|[13][S]\rarr	|p	|u	|n	|k	|t	|[][,]{ }	|m	|e	|c	|z	|o	|w	|y	|*	|*	|*	|.
|o	|*	|*	|*	|*	|a	|*	|[14][S]\rarr	|g	|l	|u	|t	|e	|n	|i	|n	|a	|*	|*	|.
|c	|[15][S]\rarr	|r	|ó	|g	|*	|*	|*	|*	|*	|*	|*	|*	|[][,]{ }	|z	|*	|[16][S]\darr	|*	|*	|.
|h	|*	|[17][S]\drarr	|a	|c	|h	|o	|n	|d	|r	|y	|t	|*	|f	|m	|*	|g	|[18][S]\darr	|*	|.
|ł	|*	|n	|[19][S]\darr	|[20][S]\darr	|*	|*	|*	|[21][S]\drarr	|a	|d	|l	|e	|r	|*	|*	|r	|p	|*	|.
|a	|[22][S]\drarr	|o	|n	|t	|a	|r	|i	|o	|*	|*	|*	|*	|a	|*	|[23][S]\darr	|a	|a	|*	|.
|n	|a	|z	|u	|o	|*	|*	|[24][S]\darr	|g	|[25][S]\darr	|[26][S]\drarr	|p	|a	|n	|e	|k	|*	|p	|*	|.
|i	|l	|d	|r	|l	|*	|*	|b	|ł	|c	|s	|*	|[27][S]\darr	|c	|[28][S]\darr	|*	|*	|i	|*	|.
|a	|d	|r	|z	|k	|*	|[29][S]\darr	|i	|o	|z	|k	|*	|s	|i	|ż	|*	|[30][S]\darr	|e	|*	|.
|c	|o	|z	|e	|i	|[31][S]\drarr	|p	|u	|d	|e	|r	|*	|o	|s	|y	|*	|p	|r	|*	|.
|z	|p	|e	|c	|e	|s	|o	|r	|e	|c	|z	|[32][S]\darr	|l	|z	|ł	|[33][S]\darr	|l	|o	|*	|.
|[][,]{ }	|e	|[][,]{ }	|*	|n	|z	|w	|e	|k	|z	|y	|r	|i	|k	|a	|p	|e	|w	|*	|.
|g	|n	|t	|[34][S]\darr	|*	|c	|i	|t	|*	|o	|n	|a	|k	|a	|[][,]{ }	|o	|n	|y	|*	|.
|a	|t	|y	|w	|*	|z	|ś	|a	|*	|t	|e	|k	|a	|ń	|z	|s	|u	|[][,]{ }	|*	|.
|z	|o	|l	|i	|[35][S]\darr	|u	|l	|*	|*	|*	|c	|i	|m	|s	|ł	|t	|m	|t	|*	|.
|ó	|z	|n	|e	|p	|r	|a	|*	|*	|[36][S]\rarr	|z	|e	|s	|k	|o	|k	|*	|y	|*	|.
|w	|a	|e	|l	|o	|[][,]{ }	|n	|*	|*	|*	|k	|t	|k	|i	|t	|o	|[37][S]\darr	|g	|*	|.
|*	|*	|*	|k	|l	|l	|i	|*	|*	|[38][S]\darr	|a	|a	|*	|*	|a	|m	|s	|r	|*	|.
|*	|*	|*	|o	|b	|ą	|n	|*	|*	|w	|*	|*	|*	|*	|*	|u	|i	|y	|*	|.
|*	|*	|*	|l	|a	|d	|*	|[39][S]\rarr	|p	|o	|z	|i	|o	|m	|*	|n	|m	|s	|*	|.
|[40][S]\rarr	|b	|i	|u	|r	|o	|[][,]{ }	|m	|a	|k	|l	|e	|r	|s	|k	|i	|e	|*	|*	|.
|[41][S]\rarr	|i	|n	|d	|y	|w	|i	|d	|u	|a	|l	|i	|z	|m	|*	|z	|o	|*	|*	|.
|*	|*	|*	|*	|*	|y	|*	|[42][S]\rarr	|b	|l	|e	|z	|e	|r	|*	|m	|n	|*	|*	|.
|[43][S]\rarr	|y	|a	|r	|d	|*	|*	|*	|*	|*	|[44][S]\rarr	|s	|e	|i	|d	|*	|i	|*	|*	|.
|*	|*	|*	|*	|*	|*	|*	|*	|*	|*	|*	|*	|*	|*	|*	|*	|*	|*	|*	|.\end{Puzzle}

\newpage

\begin{PuzzleClues}{\textbf{Poziome}\\}\Clue{1}{}{nazwa jednostki monetarnej kilku państw azjatyckich}
\Clue{2}{}{PROPORCZYK barwny płat materiału przytwierdzony do drzewca lancy, oznaka bojowa pułku kawalerii}
\Clue{3}{}{tafla lodu, na której uprawiane są niektóre sporty zimowe, nieodzowna część lodowiska}
\Clue{4}{}{miejsce pracy korektorów, dział korekty}
\Clue{5}{}{planowe uzupełnienie zalesienia}
\Clue{6}{}{część szpondra bliższa podbrzuszu i okolicom mostka}
\Clue{7}{}{niemiły w obejściu człowiek, naburmuszony, aspołeczny, często też zarozumiały}
\Clue{9}{}{model samochodu osobowego klasy średniej produkowany przez General Motors pod marką Opel w latach 1988-2008}
\Clue{10}{}{koliście skupiona plecha niektórych wątrobowców}
\Clue{11}{}{giętkie drewno pozyskiwane z palmy rotang, używane do produkcji mebli}
\Clue{12}{}{efekt optyczny polegający na pojawianiu się wąskiej smugi (pasma) świetlnej przypominającej źrenicę kociego oka, która przemieszcza się po wypukłej i wypolerowanej powierzchni kamienia w trakcie jego obracania}
\Clue{13}{}{w niektórych sportach punkt, którego zdobycie przesądza o wyniku meczu na korzyść zdobywającego}
\Clue{14}{}{białko roślinne, składnik glutenu}
\Clue{15}{}{miejsce, w którym przecinają się dwie linie ograniczające boisko}
\Clue{17}{}{meteoryt kamienny złożony głównie z piroksenów i plagioklazów}
\Clue{21}{}{malarz i grafik (1895-1949) prace zaczerpnięte z folkloru żydowskiego oraz poświęcone zagładzie Żydów}
\Clue{22}{}{prowincja w środkowej Kanadzie, nad Wielkimi Jeziorami, Zatoką Hussana, powierzchnia 1,1 min km2, stolica Toronto}
\Clue{26}{}{Leccinum aurantiacum (Bull.) Gray - gatunek grzybów z rodziny borowikowatych; w Europie Środkowej jest szeroko rozprzestrzeniony, ale występuje nierównomiernie - w niektórych regionach jest dość częsty, w innych rzadki}
\Clue{31}{}{drobinki czegoś starte na gładką masę}
\Clue{36}{}{ostatnia faza skoku, w której skoczek dotyka ziemi}
\Clue{39}{}{wykładnik czegoś, stopień, pułap, informacja o wielkości, ilości, natężeniu itp}
\Clue{40}{}{wydzielona jednostka organizacyjna banku prowadząca działalność maklerską}
\Clue{41}{}{stanowisko w epistemologii i metodologii nauk, zgodnie z którym istnieją tylko byty jednostkowe (np. jednostki ludzkie)}
\Clue{42}{}{długi, luźny żakiet zwykle bez kołnierza i usztywnień}
\Clue{43}{}{JARD}
\Clue{44}{}{tytuł potomków Mahometa}\end{PuzzleClues}

\begin{PuzzleClues}{\textbf{Pionowe}\\}\Clue{1}{}{KOLT}
\Clue{3}{}{postawa negowania osiągnięć nauki, sztuki, innych dziedzin wiedzy}
\Clue{4}{}{szczęka; słowo młodzieżowe}
\Clue{5}{}{w chemii: symbol argonu}
\Clue{7}{}{związek wszystkich wspólnot życia konsekrowanego opierających się na Regule św. Franciszka z Asyżu}
\Clue{8}{}{autobus międzymiastowej komunikacji publicznej}
\Clue{12}{}{substancja chemiczna, której zadaniem jest utrzymywanie w danym przyrządzie próżni przez pochłanianie resztek gazów}
\Clue{16}{}{rozgrywka w rywalizacji, która się toczy, także: sposób gry}
\Clue{17}{}{przestrzeń w obrębie głowy, rynienka kostna otwierająca się w kierunku dolnym}
\Clue{18}{}{instytucja silna i prężna jedynie z pozoru, a w istocie słaba i podatna na porażkę w razie konfrontacji}
\Clue{19}{}{wieś w Polsce, położona w województwie podlaskim, w powiecie siemiatyckim}
\Clue{20}{}{dzieła, twórczość Tolkiena}
\Clue{21}{}{Scolytus - rodzaj chrząszcza z podrodziny kornikowatych}
\Clue{22}{}{pentoza należąca do aldoz}
\Clue{23}{}{kelwin - jednostka temperatury w układzie SI równa 1/273,16 temperatury termodynamicznej punktu potrójnego wody, oznaczana K}
\Clue{24}{}{rodzaj naczynia laboratoryjnego w kształcie szklanej rurki z podziałką i zamknięciem u dołu}
\Clue{25}{}{artysta rzeźbiący w metalu}
\Clue{26}{}{zdrobniale: skrzynka - opakowanie do transportu szklanych butelek standardowych kształtów i rozmiarów}
\Clue{27}{}{miasto we wschodniej części Rosji, w Kraju Permskim, na zachodnich stokach Uralu, przystań nad Kamą}
\Clue{28}{}{źródło korzyści, dojna krowa, kokosowy interes}
\Clue{29}{}{mieszkaniec Powiśla (dzielnicy Warszawy)}
\Clue{30}{}{ogół członków jakiejś instytucji, organizacji}
\Clue{31}{}{lekceważące określenie człowieka niezwiązanego z morzem, stale przebywającego na lądzie}
\Clue{32}{}{pozycja tenisisty w drużynie lub miejsce startowe w zawodach}
\Clue{33}{}{zmodyfikowana forma komunizmu}
\Clue{34}{}{bardzo wysoki człowiek; słowo żartobliwe}
\Clue{35}{}{polska rasa kur ogólnoużytkowych; prążkowane czarno-popielate upierzenie}
\Clue{37}{}{lekkoatletka włoska, mistrzyni olimpijska z Moskwy w skoku wzwyż}
\Clue{38}{}{głos ludzki w kompozycji muzycznej, śpiew}\end{PuzzleClues}\newpage\section*{Krzyżówka 29}

\noindent\begin{Puzzle}{25}{28}|*	|*	|*	|*	|[1][S]\drarr	|l	|ó	|d	|[][,]{ }	|p	|a	|k	|o	|w	|y	|*	|*	|*	|[2][S]\drarr	|t	|y	|g	|i	|e	|l	|*	|.
|*	|*	|*	|*	|a	|[3][S]\darr	|*	|*	|[4][S]\rarr	|w	|i	|c	|e	|b	|u	|r	|m	|i	|s	|t	|r	|z	|*	|*	|*	|*	|.
|*	|*	|*	|[5][S]\darr	|s	|w	|*	|*	|*	|*	|*	|[6][S]\darr	|*	|[7][S]\darr	|[8][S]\darr	|*	|*	|*	|z	|*	|*	|[9][S]\drarr	|c	|a	|b	|*	|.
|*	|*	|[10][S]\drarr	|c	|z	|a	|m	|a	|r	|a	|*	|p	|*	|o	|m	|[11][S]\rarr	|p	|u	|c	|h	|*	|p	|*	|*	|*	|*	|.
|*	|*	|c	|y	|a	|n	|[12][S]\rarr	|b	|r	|y	|g	|i	|d	|k	|a	|*	|[13][S]\rarr	|s	|z	|n	|u	|r	|*	|*	|*	|*	|.
|*	|*	|e	|g	|*	|*	|[14][S]\darr	|*	|*	|*	|*	|l	|*	|r	|n	|*	|[15][S]\drarr	|m	|a	|n	|y	|o	|k	|y	|*	|*	|.
|*	|[16][S]\rarr	|b	|a	|r	|d	|o	|n	|*	|*	|*	|a	|*	|ę	|u	|*	|k	|*	|w	|*	|*	|t	|[17][S]\darr	|*	|*	|*	|.
|[18][S]\rarr	|k	|u	|r	|a	|n	|t	|*	|*	|*	|*	|*	|*	|g	|f	|*	|r	|*	|i	|*	|[19][S]\darr	|e	|e	|[20][S]\darr	|*	|*	|.
|*	|*	|l	|n	|*	|[21][S]\rarr	|w	|a	|r	|n	|i	|k	|*	|*	|a	|*	|u	|[22][S]\drarr	|k	|ę	|p	|a	|*	|a	|*	|*	|.
|*	|*	|a	|i	|*	|*	|*	|[23][S]\rarr	|g	|a	|n	|g	|t	|o	|k	|*	|ż	|t	|*	|[24][S]\darr	|i	|z	|*	|ł	|*	|*	|.
|*	|*	|[][,]{ }	|c	|*	|*	|*	|*	|*	|*	|*	|*	|*	|[25][S]\darr	|t	|*	|a	|a	|*	|m	|a	|a	|*	|m	|*	|*	|.
|*	|*	|p	|z	|[26][S]\rarr	|e	|k	|w	|i	|n	|o	|k	|c	|j	|u	|m	|*	|m	|[27][S]\darr	|a	|n	|*	|*	|a	|*	|*	|.
|*	|*	|r	|k	|*	|*	|[28][S]\rarr	|t	|a	|v	|o	|l	|i	|e	|r	|e	|*	|a	|k	|z	|a	|*	|[29][S]\darr	|a	|*	|*	|.
|*	|[30][S]\drarr	|z	|a	|s	|ł	|o	|n	|a	|[][,]{ }	|d	|y	|m	|n	|a	|*	|*	|r	|o	|z	|*	|*	|s	|t	|*	|*	|.
|*	|m	|y	|*	|*	|[31][S]\rarr	|t	|r	|ą	|b	|o	|w	|c	|e	|*	|[32][S]\rarr	|w	|y	|r	|o	|b	|n	|i	|k	|*	|*	|.
|*	|a	|b	|*	|[33][S]\darr	|*	|[34][S]\rarr	|a	|r	|m	|a	|t	|o	|r	|*	|*	|*	|s	|a	|l	|*	|*	|e	|a	|*	|*	|.
|*	|c	|y	|[35][S]\rarr	|p	|ł	|y	|t	|a	|[][,]{ }	|o	|c	|e	|a	|n	|i	|c	|z	|n	|a	|*	|*	|r	|*	|*	|*	|.
|*	|z	|s	|[36][S]\darr	|a	|*	|*	|*	|*	|*	|*	|[37][S]\drarr	|m	|ł	|o	|t	|e	|k	|*	|*	|*	|*	|p	|*	|[38][S]\darr	|*	|.
|*	|e	|z	|m	|n	|*	|*	|*	|*	|*	|[39][S]\rarr	|s	|r	|*	|*	|[40][S]\rarr	|w	|a	|n	|n	|a	|*	|i	|*	|c	|*	|.
|*	|k	|o	|a	|a	|*	|*	|*	|*	|[41][S]\rarr	|c	|z	|e	|c	|h	|ł	|o	|*	|*	|*	|*	|*	|e	|*	|e	|*	|.
|*	|*	|w	|j	|m	|*	|[42][S]\rarr	|a	|n	|g	|i	|e	|l	|s	|k	|o	|ś	|ć	|*	|*	|*	|*	|ń	|*	|l	|*	|.
|*	|*	|a	|t	|a	|[43][S]\rarr	|c	|h	|r	|z	|ą	|s	|z	|c	|z	|[][,]{ }	|w	|o	|d	|n	|y	|*	|*	|*	|e	|*	|.
|*	|*	|*	|e	|*	|[44][S]\rarr	|o	|s	|t	|a	|t	|n	|i	|e	|[][,]{ }	|s	|ł	|o	|w	|o	|*	|*	|*	|*	|b	|*	|.
|*	|*	|[45][S]\rarr	|k	|o	|n	|c	|e	|n	|t	|r	|a	|t	|[][,]{ }	|s	|p	|o	|ż	|y	|w	|c	|z	|y	|*	|r	|*	|.
|*	|*	|*	|*	|*	|*	|[46][S]\rarr	|c	|h	|w	|o	|s	|t	|k	|a	|[][,]{ }	|w	|s	|p	|a	|n	|i	|a	|ł	|a	|*	|.
|*	|*	|*	|*	|*	|*	|*	|*	|[47][S]\rarr	|a	|n	|t	|y	|d	|a	|t	|o	|w	|a	|n	|i	|e	|*	|*	|n	|*	|.
|*	|*	|*	|[48][S]\rarr	|s	|i	|e	|w	|e	|c	|z	|k	|a	|[][,]{ }	|p	|u	|s	|t	|y	|n	|n	|a	|*	|*	|s	|*	|.
|*	|*	|*	|[49][S]\rarr	|g	|r	|z	|y	|b	|[][,]{ }	|p	|a	|s	|o	|ż	|y	|t	|n	|i	|c	|z	|y	|*	|*	|*	|*	|.
|*	|[50][S]\rarr	|j	|a	|r	|z	|y	|n	|o	|w	|a	|*	|*	|*	|*	|*	|*	|*	|*	|*	|*	|*	|*	|*	|*	|*	|.\end{Puzzle}

\newpage

\begin{PuzzleClues}{\textbf{Poziome}\\}\Clue{1}{}{wieloletnia, pływająca kra lodowa, występująca na obszarach polarnych Ziemi: Morzu Arktycznym na półkuli północnej i morzach otaczających Antarktydę}
\Clue{2}{}{stan bądź miejsce, w którym dochodzi do gwatownych zmian, lub w którym mieszają się różnorodne wpływy}
\Clue{4}{}{zastępca burmistrza}
\Clue{9}{}{powóz czterokołowy będący odmianą kabrioletu}
\Clue{10}{}{długie okrycie wierzchnie podszyte futrem, zapinane pod szyją noszone przez prałatów i zakonników w XVI w Polsce}
\Clue{11}{}{coś podobnego pod względem miękkości i lekkości do ptasiego puchu}
\Clue{12}{}{członkini Zakonu Świętej Brygidy}
\Clue{13}{}{wyrób powroźniczy o średnicy 4-8 mm}
\Clue{15}{}{rosyjski malarz i grafik (1859-1937) sceny rodzajowe, portrety}
\Clue{16}{}{dawny instrument muzyczny z grupy chordofonów, przypominający lirę, lutnię i mandolinę}
\Clue{18}{}{francuski taniec dworski popularny w XVI wieku}
\Clue{21}{}{w lodówkach - urządzenie służące do odparowywania czynnika chłodzącego}
\Clue{22}{}{niewielka wyspa rzeczna lub na jeziorze, porosła roślinami różnego typu}
\Clue{23}{}{miasto w Indiach, stolica stanu Sikkim; instytut tybetologii}
\Clue{26}{}{RÓWNONOC}
\Clue{28}{}{kraina we Włoszech nad Morzem Adriatyckim}
\Clue{30}{}{działanie przedsięwzięte w celu ukrycia prawdziwych intencji czy celów działającego}
\Clue{31}{}{Proboscidea - rząd dużych ssaków łożyskowych, do którego zaliczane są żyjące współcześnie słonie; zamieszkują Afrykę i Azję}
\Clue{32}{}{człowiek, który pracuje jako osoba wynajmowana do wykonywania różnych doraźnie potrzebnych prac}
\Clue{34}{}{osoba fizyczna lub prawna prowadząca eksploatację własnego lub cudzego statku}
\Clue{35}{}{płyta tektoniczna (płyta litosfery) znajdująca się w całości pod oceanem}
\Clue{37}{}{drobna, ruchoma część składowa różnych precyzyjnych mechanizmów}
\Clue{39}{}{symbol steradiana - jednostki uzupełniającej układu SI określającej wartość kąta bryłowego}
\Clue{40}{}{głębokie naczynie wykorzystywane w różnych dziedzinach przemysłu do różnych prac wymagających czynności typu: płukanie, barwienie, namaczanie}
\Clue{41}{}{koszula kobieca noszona w XVI/XVIII w}
\Clue{42}{}{zespół cech czegoś lub kogoś takiego jak w Anglii, także: stereotypowe cechy uznawane za właściwe Anglikom}
\Clue{43}{}{owad z rzędu chrząszczy, zamieszkujący środowiska wodne}
\Clue{44}{}{zakończenie czegoś, ostatnia wypowiedź w jakiejś kwestii, zdarzenie w jakimś ciągu}
\Clue{45}{}{produkty spożywcze poddane obróbce, mającej zmniejszyć ich objętość oraz wydłużyć trwałość}
\Clue{46}{}{Malurus splendens - gatunek małego ptaka z rodziny chwostkowatych (Maluridae), jeden z 13 przedstawicieli rodzaju Malurus}
\Clue{47}{}{prochronizm - opatrzenie wydarzenia datą wcześniejszą niż jemu właściwa}
\Clue{48}{}{Charadrius leschenaultii leschenaultii - nominatywny podgatunek ptaka wyróżniony w obrębie gatunku sieweczka pustynna (Charadrius leschenaultii)}
\Clue{49}{}{pasożyt należący do królestwa grzybów;  organizm cudzożywny, pobierający substancje odżywcze z roślin zielonych, zwierząt lub innych grzybów, pełniących wówczas funkcję żywicieli}
\Clue{50}{}{zupa z mieszanki kilku warzyw}\end{PuzzleClues}

\begin{PuzzleClues}{\textbf{Pionowe}\\}\Clue{1}{}{miejscowość w Federacji Rosyjskiej u podnóża Uralu Południowego, węzeł kolejowy}
\Clue{2}{}{ktoś, na kogo nie można liczyć; młoda, niedoświadczona i niekompetentna osoba}
\Clue{3}{}{sieć komputerowa znajdująca się na obszarze wykraczającym poza jedno miasto (bądź kompleks miejski)}
\Clue{5}{}{rodzaj lufki, wydłużony ustnik do papierosa}
\Clue{6}{}{włosowata struktura komórkowa; główną funkcją fimbrii jest ułatawianie bakterii przyleganie do innej kmórki}
\Clue{7}{}{jednostka administracyjna państwa}
\Clue{8}{}{przen. czynność, która jest czasochłonna i daje mierne efekty}
\Clue{9}{}{enzym proteolityczny - enzym z klasy hydrolaz katalizujący proteolizę oraz w większości przypadków hydrolizę wiązania estrowego}
\Clue{10}{}{rodzaj cebulki, wykorzystywany przy rozmnażaniu wegetatywnym roślin cebulowych, inaczej: cebulka przybyszowa}
\Clue{14}{}{ogólna teoria względności - teorii grawitacji formułowanej przez Alberta Einsteina w latach 1907-1915, a opublikowanej 20 marca 1916 roku; zgodnie z ogólną teorią względności, siła grawitacji wynika z lokalnej geometrii czasoprzestrzeni}
\Clue{15}{}{tyle, ile mieści się w kruży}
\Clue{17}{}{jednostka liczności fotonów}
\Clue{19}{}{mieszanina dyspersyjna, w której ośrodkiem rozpraszającym jest ciecz lub ciało stałe, a fazą rozproszoną gaz}
\Clue{20}{}{mieszkanka Ałam Aty}
\Clue{22}{}{ptak z rodziny pokrzewkowatych - Hiszpania, Francja, Włochy}
\Clue{24}{}{francuski malarz i grafik (1869-1954) reprezentant fowizmu; kompozycje figuralne, martwe natury, malowidła ścienne, rysunki}
\Clue{25}{}{generał}
\Clue{27}{}{ALKORAN; święta księga muzułmanów zawierająca nauki Mahometa; przepisy dogmatyczne, kultowe i rytualne}
\Clue{29}{}{ósmy miesiąc w roku według kalendarza gregoriańskiego, ma 31 dni}
\Clue{30}{}{słodki, miękki cukierek z masą kakaową w środku, oblany czekoladą i obsypany kolorową posypką}
\Clue{33}{}{letni, jasny kapelusz o płaskiej główce i szerokim rondzie wyplatany z włókien palmy, modny w XIX/XX w; dziś: kapelusz słomkowy o tym kształcie}
\Clue{36}{}{szeregowy członek załogi na statku i okręcie}
\Clue{37}{}{grupa obiektów, która składa się z 16 elementów}
\Clue{38}{}{duchowny odprawiający uroczyste nabożeństwo, prowadzący procesję}\end{PuzzleClues}\newpage\section*{Krzyżówka 30}

\noindent\begin{Puzzle}{20}{26}|*	|*	|*	|*	|*	|*	|*	|*	|*	|*	|*	|*	|*	|*	|[1][S]\drarr	|s	|o	|r	|e	|k	|*	|.
|*	|*	|*	|[2][S]\darr	|[3][S]\darr	|[4][S]\drarr	|s	|a	|r	|d	|e	|l	|a	|[][,]{ }	|k	|a	|p	|s	|k	|a	|*	|.
|*	|*	|*	|r	|c	|c	|[5][S]\rarr	|w	|i	|c	|e	|k	|u	|r	|a	|t	|o	|r	|*	|*	|*	|.
|*	|*	|*	|i	|z	|h	|[6][S]\rarr	|c	|z	|a	|r	|n	|a	|[][,]{ }	|m	|o	|w	|a	|*	|*	|*	|.
|*	|[7][S]\darr	|[8][S]\rarr	|n	|y	|l	|o	|n	|*	|*	|*	|*	|[9][S]\rarr	|c	|e	|n	|z	|u	|s	|*	|*	|.
|*	|a	|*	|*	|n	|a	|*	|[10][S]\rarr	|w	|i	|c	|e	|d	|y	|r	|e	|k	|t	|o	|r	|*	|.
|[11][S]\drarr	|g	|a	|z	|[][,]{ }	|p	|ł	|y	|n	|n	|y	|*	|[12][S]\rarr	|p	|a	|t	|y	|k	|*	|*	|*	|.
|ż	|u	|[13][S]\rarr	|e	|l	|a	|n	|o	|w	|e	|ł	|n	|a	|*	|[][,]{ }	|*	|*	|*	|*	|*	|*	|.
|a	|t	|[14][S]\drarr	|l	|u	|n	|a	|c	|j	|a	|*	|[15][S]\rarr	|p	|a	|w	|ę	|z	|a	|*	|*	|*	|.
|b	|i	|e	|*	|b	|i	|*	|[16][S]\rarr	|s	|z	|c	|z	|y	|p	|i	|o	|r	|e	|k	|*	|*	|.
|n	|[][,]{ }	|l	|*	|i	|e	|*	|*	|[17][S]\rarr	|s	|k	|r	|z	|y	|d	|ł	|o	|*	|*	|*	|[18][S]\darr	|.
|i	|z	|t	|*	|e	|[][,]{ }	|*	|*	|*	|*	|*	|*	|[19][S]\drarr	|b	|e	|ł	|k	|o	|t	|*	|o	|.
|c	|ł	|o	|*	|ż	|j	|*	|[20][S]\drarr	|p	|o	|g	|r	|o	|b	|o	|w	|i	|e	|c	|*	|w	|.
|a	|o	|n	|*	|n	|ę	|*	|z	|[21][S]\rarr	|a	|f	|e	|r	|a	|*	|*	|*	|*	|*	|*	|c	|.
|[][,]{ }	|c	|*	|*	|y	|z	|[22][S]\drarr	|a	|d	|m	|i	|r	|a	|l	|i	|c	|j	|a	|*	|*	|a	|.
|n	|i	|*	|[23][S]\darr	|*	|y	|b	|b	|[24][S]\drarr	|c	|h	|i	|n	|o	|l	|o	|n	|*	|*	|*	|[][,]{ }	|.
|a	|s	|[25][S]\drarr	|n	|a	|k	|ł	|u	|c	|i	|e	|*	|g	|[26][S]\drarr	|i	|l	|l	|e	|s	|*	|ś	|.
|w	|t	|v	|i	|*	|i	|ą	|ż	|o	|*	|*	|*	|u	|j	|*	|*	|*	|[27][S]\darr	|*	|*	|r	|.
|ę	|y	|a	|t	|*	|e	|d	|a	|c	|[28][S]\rarr	|c	|y	|t	|a	|d	|e	|l	|a	|*	|*	|u	|.
|d	|*	|i	|k	|*	|m	|[][,]{ }	|n	|k	|*	|[29][S]\rarr	|j	|a	|n	|k	|e	|s	|k	|i	|*	|b	|.
|*	|*	|s	|a	|*	|*	|m	|k	|n	|*	|*	|[30][S]\rarr	|n	|i	|c	|o	|l	|a	|i	|*	|o	|.
|*	|[31][S]\darr	|a	|*	|*	|*	|e	|a	|e	|*	|[32][S]\rarr	|c	|*	|c	|*	|*	|*	|d	|*	|*	|r	|.
|[33][S]\rarr	|a	|l	|b	|e	|r	|t	|*	|y	|[34][S]\rarr	|s	|e	|t	|k	|a	|*	|*	|e	|*	|*	|o	|.
|[35][S]\drarr	|b	|a	|s	|e	|d	|o	|w	|*	|*	|[36][S]\rarr	|k	|l	|i	|f	|*	|*	|m	|*	|*	|g	|.
|u	|a	|*	|[37][S]\rarr	|m	|y	|d	|l	|a	|n	|o	|ś	|ć	|*	|*	|*	|*	|i	|*	|*	|a	|.
|*	|*	|*	|[38][S]\rarr	|b	|ł	|y	|s	|k	|o	|l	|o	|t	|k	|a	|[][,]{ }	|m	|a	|ł	|a	|*	|.
|[39][S]\rarr	|t	|y	|t	|o	|ń	|*	|*	|[40][S]\rarr	|p	|e	|l	|i	|k	|a	|n	|y	|*	|*	|*	|*	|.\end{Puzzle}

\newpage

\begin{PuzzleClues}{\textbf{Poziome}\\}\Clue{1}{}{ryjówka; ssak wielkości myszy o długim tułowiu i wydłużonej głowie, owadożerny, niszczy szkodniki}
\Clue{4}{}{Engraulis capensis - gatunek morskiej ryby z rodziny sardelowatych (Engraulidae)}
\Clue{5}{}{zastępca kuratora w oświacie}
\Clue{6}{}{język, którym posługiwali się orkowie i inni słudzy Saurona w literackim legendarium, wykreowanym przez J. R. R. Tolkiena}
\Clue{8}{}{tkanina z nylonu}
\Clue{9}{}{spis ludności w danym kraju}
\Clue{10}{}{zastępca dyrektora - szefa, kierownika jakiejś instytucji}
\Clue{11}{}{ciecz, będąca skroplonym gazem}
\Clue{12}{}{kawałek gałęzi drzewa, który się oderwał, spadł na ziemię, mniejszy (cieńszy) niż kijek}
\Clue{13}{}{tkanina z mieszanych włókien elany i wełny}
\Clue{14}{}{odstęp czasu między dwoma kolejnymi nowiami Księżyca}
\Clue{15}{}{rodzaj tarczy w kształcie czworokąta, wysokiej (nawet do wysokości ramion dorosłego mężczyzny), z pionową wypukłością pośrodku}
\Clue{16}{}{znana (wręcz pospolita) przyprawa; najczęściej świeża (rzadko suszona) natka cebuli i szczypiorku}
\Clue{17}{}{w wojsku: boczna część ugrupowania bojowego (operacyjnego), odróżniana tradycyjnie od jego centrum}
\Clue{19}{}{niewyraźna mowa spowodowana jakimś utrudnieniem w wypowiadaniu słów}
\Clue{20}{}{dziecko, które urodziło się już po śmierci ojca}
\Clue{21}{}{przestępstwo, często dotyczące wyższych sfer lub osób zaufania publicznego, które ma potencjał medialny}
\Clue{22}{}{władze naczelne marynarki wojennej w niektórych państwach np. w Anglii}
\Clue{24}{}{chemioterapeutyk z grupy chinolonów o działaniu bakteriobójczym}
\Clue{25}{}{nabicie, nadzianie czegoś na szpikulec, ostry przedmiot}
\Clue{26}{}{(1895-1974), pisarz węgierski, pionier realizmu socjalistycznego; „Cisa płonie”, „Bitwa pod Teatrem Komediowym”, „Rapsodia karpacka”}
\Clue{28}{}{samodzielna twierdza panująca nad miastem}
\Clue{29}{}{odmiana języka angielskiego używana w Stanach Zjednoczonych Ameryki}
\Clue{30}{}{kompozytor niemiecki (1810-1849); opera komiczna 'Wesołe kumoszki z Windsoru'}
\Clue{32}{}{nazwa literowa pierwszego dźwięku w gamie, także od niej bierze oznaczenie tonacja, której toniką jest c}
\Clue{33}{}{polski zakonnik franciszkański, założyciel zgromadzenia albertynów i albertynek, powstaniec, malarz, święty Kościoła katolickiego, znany z pełnej poświęcenia pracy dla biednych i bezdomnych}
\Clue{34}{}{liczba 100, numer 100}
\Clue{35}{}{lekarz niemiecki (1799-1854); opisał zespół nadczynności tarczycy}
\Clue{36}{}{FALEZA; urwisty brzeg morski}
\Clue{37}{}{cecha czegoś, co pod względem aromatu przypomina mydło}
\Clue{38}{}{Phaps elegans - gatunek ptaka z rodziny gołębiowatych (Columbidae); występuje w Australii}
\Clue{39}{}{produkt tworzony z liści tytoniu, który można skonsumować, używać jako formy pestycydów lub jako lekarstwa}
\Clue{40}{}{Pelecanidae - rodzina ptaków z rzędu pelikanowych (Pelecaniformes)}\end{PuzzleClues}

\begin{PuzzleClues}{\textbf{Pionowe}\\}\Clue{1}{}{urządzenie elektroniczne, służące do nagrywania obrazu na taśmie wideo poprzez zamianę sygnałów optycznych na sygnały elektryczne}
\Clue{2}{}{dawna (bita w latach 1873-1919) jednostka zdawkowa w Japonii; 1/1000 jena, 1/100 sena, 1/10 funa}
\Clue{3}{}{zachowanie seksualne niebędące obcowaniem płciowym, obecnie określane jakoinna czynność seksualna}
\Clue{4}{}{mówienie za dużo, niepotrzebnie, ponad miarę}
\Clue{7}{}{Dasyprocta leporina - gatunek gryzonia z rodziny agutiowatych, żyjący w gęstych i wilgotnych lasach Brazylii, Gujany Francuskiej, Gujany, Surinamu, Trynidadu i Tobago i Wenezueli; introdukowany na Dominice, Grenadzie i Wyspach Dziewiczych}
\Clue{11}{}{Lophius piscatorius - ryba morska z rodziny żabnicowatych (Lophiidae), nazywana też diabłem morskim}
\Clue{14}{}{słone jezioro w Federacji Rosyjskiej na Nizinie Nadkaspijskiej, powierzchnia 152km , wydobycie soli}
\Clue{18}{}{racka, tucerna -  jedna z ras owcy hodowanej głównie na Węgrzech, użytkowana trojako - dostarczała mięsa, mleka i wełny; pierwotnie występowała na stepach i obszarach trawiastych}
\Clue{19}{}{Pongo - rodzaj dużych małp człekokształtnych; występuje jedynie w lasach deszczowych na Borneo i Sumatrze}
\Clue{20}{}{mieszkanka Zabuża}
\Clue{22}{}{błąd związany z zastosowaniem w metodzie numerycznej przybliżenia oryginalnego zagadnienia w sposób zmieniający jego właściwości matematyczne}
\Clue{23}{}{makaron pocięty na małe, cieniutkie paseczki}
\Clue{24}{}{osoba pochodząca ze wschodniego Londynu, posługująca się gwarą miejską nazywaną cockney}
\Clue{25}{}{fiński astronom i geodeta (1891-1971), autor koncepcji triangulacji satelitarnej}
\Clue{26}{}{ur. 1928r, dramaturg i dziennikarz, autor tekstów seriali telewizyjnych („Dom”),słuchowisk radiowych („Matysiakowie”)}
\Clue{27}{}{instytucja o charakterze państwowym lub społecznym, która zrzesza wybitnych artystów lub uczonych w celu współpracy}
\Clue{31}{}{ABAJA}
\Clue{35}{}{w chemii: symbol uranu}\end{PuzzleClues}\newpage\section*{Krzyżówka 31}

\noindent\begin{Puzzle}{21}{23}|*	|*	|*	|*	|*	|*	|*	|*	|*	|*	|*	|*	|*	|*	|*	|*	|*	|*	|*	|*	|[1][S]\darr	|*	|.
|*	|*	|*	|*	|*	|*	|[2][S]\darr	|*	|*	|[3][S]\drarr	|g	|ł	|a	|d	|ź	|*	|*	|*	|*	|*	|n	|*	|.
|*	|*	|*	|*	|*	|[4][S]\drarr	|m	|a	|g	|n	|o	|l	|i	|a	|[][,]{ }	|g	|ó	|r	|s	|k	|a	|*	|.
|*	|*	|*	|*	|*	|d	|l	|*	|[5][S]\rarr	|a	|l	|o	|g	|r	|a	|f	|*	|[6][S]\drarr	|f	|y	|t	|*	|.
|*	|*	|*	|*	|*	|ż	|e	|*	|[7][S]\rarr	|d	|y	|m	|n	|i	|k	|*	|[8][S]\drarr	|m	|o	|k	|o	|*	|.
|[9][S]\drarr	|h	|a	|j	|d	|u	|c	|y	|*	|w	|[10][S]\rarr	|a	|m	|u	|n	|i	|k	|e	|*	|*	|*	|*	|.
|k	|*	|[11][S]\rarr	|b	|e	|n	|z	|i	|*	|y	|[12][S]\drarr	|s	|z	|a	|r	|l	|o	|t	|k	|a	|*	|*	|.
|i	|[13][S]\darr	|*	|*	|*	|g	|n	|*	|*	|ż	|l	|*	|*	|[14][S]\rarr	|b	|a	|r	|r	|e	|l	|*	|*	|.
|e	|r	|[15][S]\drarr	|p	|o	|l	|i	|t	|y	|k	|a	|[][,]{ }	|s	|p	|ó	|j	|n	|o	|ś	|c	|i	|*	|.
|ł	|z	|s	|*	|*	|a	|k	|*	|*	|a	|n	|[16][S]\rarr	|p	|r	|e	|w	|e	|n	|c	|j	|a	|*	|.
|*	|e	|p	|*	|*	|*	|*	|*	|*	|[][,]{ }	|c	|*	|*	|[17][S]\drarr	|p	|o	|t	|o	|p	|*	|*	|*	|.
|[18][S]\rarr	|k	|a	|n	|t	|*	|*	|*	|[19][S]\rarr	|k	|e	|f	|i	|r	|*	|*	|*	|m	|*	|*	|*	|*	|.
|*	|o	|d	|[20][S]\drarr	|p	|r	|z	|y	|g	|o	|t	|o	|w	|a	|n	|i	|e	|*	|*	|*	|[21][S]\darr	|*	|.
|[22][S]\drarr	|t	|o	|m	|a	|s	|s	|i	|*	|n	|*	|*	|[23][S]\rarr	|s	|a	|u	|t	|i	|n	|*	|m	|*	|.
|g	|k	|c	|a	|*	|*	|*	|*	|*	|s	|[24][S]\rarr	|o	|b	|e	|d	|i	|e	|n	|c	|j	|a	|*	|.
|r	|a	|h	|r	|*	|[25][S]\darr	|[26][S]\drarr	|g	|a	|u	|s	|*	|*	|r	|*	|[27][S]\rarr	|p	|o	|l	|e	|r	|*	|.
|a	|[][,]{ }	|r	|i	|*	|t	|k	|*	|[28][S]\rarr	|m	|a	|r	|e	|*	|*	|*	|*	|*	|*	|[29][S]\darr	|g	|*	|.
|h	|p	|o	|a	|*	|e	|u	|[30][S]\rarr	|t	|e	|l	|e	|k	|o	|m	|u	|t	|a	|c	|j	|a	|*	|.
|a	|t	|n	|n	|*	|n	|z	|[31][S]\rarr	|m	|n	|o	|ż	|a	|r	|n	|i	|a	|*	|[32][S]\darr	|a	|t	|*	|.
|m	|a	|i	|i	|*	|r	|u	|*	|[33][S]\rarr	|t	|e	|l	|e	|s	|k	|o	|p	|*	|l	|m	|e	|*	|.
|k	|s	|a	|s	|*	|e	|*	|[34][S]\rarr	|k	|a	|t	|a	|t	|o	|n	|i	|a	|*	|w	|e	|*	|*	|.
|a	|i	|r	|t	|[35][S]\rarr	|k	|r	|ą	|p	|*	|*	|*	|*	|*	|*	|*	|*	|*	|e	|s	|*	|*	|.
|*	|a	|z	|a	|*	|*	|*	|*	|*	|*	|[36][S]\rarr	|g	|l	|e	|d	|i	|c	|z	|i	|a	|*	|*	|.
|*	|*	|*	|*	|*	|*	|*	|*	|*	|*	|*	|*	|*	|*	|*	|*	|*	|*	|*	|*	|*	|*	|.\end{Puzzle}

\newpage

\begin{PuzzleClues}{\textbf{Poziome}\\}\Clue{3}{}{wierzchnia, płaska część kowadła wykonana z gładkiej, twardej stali}
\Clue{4}{}{Magnolia fraseri - gatunek drzew, należący do rodziny magnoliowatych}
\Clue{5}{}{wariant litery}
\Clue{6}{}{malarz flamandzki (1611-61 ), martwe natury, często z trofeami myśliwskimi, kwiatami i owocami}
\Clue{7}{}{przewód wentylacyjny wystający ponad dachem}
\Clue{8}{}{Kerodon rupestris - gatunek gryzonia z rodziny marowatych, występujący we wschodniej Brazylii}
\Clue{9}{}{węgierscy i południowosłowiańscy partyzanci występujący zbrojnie przeciw Turkom}
\Clue{10}{}{piłkarz nigeryjski, napastnik C.F.Barcelona}
\Clue{11}{}{ur. 1937 r., dyrygent włoski, prowadził paryską 'Grand Opera'}
\Clue{12}{}{inna nazwa drinka z żubrówki tatanka}
\Clue{14}{}{BARYŁKA; beczułka, amerykańska jednostka objętości równa 1,6 hl}
\Clue{15}{}{polityka, której celem jest wspieranie działań prowadzących do wyrównania warunków ekonomicznych i społecznych w mniej rozwiniętych regionach Unii Europejskiej w formie dotacji na rozwój infrastruktury, potencjału gospodarczego i ludzkiego, innowacji i badań naukowych oraz zrównoważonego rozwóju}
\Clue{16}{}{zespół działań mających zapobiec niepożądanym skutkom}
\Clue{17}{}{wielka powódź obejmująca duży obszar}
\Clue{18}{}{twórczość Kanta, zbiór jego myśli i poglądów}
\Clue{19}{}{porcja kefiru, zazwyczaj jego opakowanie (kubeczek, kartonik), choć może to też być porcja podana w naczyniu, np. zamówiona w lokalu gastronomicznym}
\Clue{20}{}{czynność lub zespół czynności wykonywanych z myślą o jakimś innym przyszłym wydarzeniu, o czymś, co ma nastąpić}
\Clue{22}{}{fizykochemik ur. w 1912 r., profesor Politechniki Warszawskiej, skonstruował elektrolizer z obwodem wtórnym}
\Clue{23}{}{pływak rosyjski, mistrz olimpijski z Atlanty w skokach z wieży}
\Clue{24}{}{niezależny związek lóż wolnomularskich}
\Clue{26}{}{jednostka indukcji magnetycznej}
\Clue{27}{}{PACHOŁ, KNECHT; słupek na pokładzie lub nabrzeżu służący do cumowania}
\Clue{28}{}{angielski poeta i prozaik (1873-1956), liryka przesączona baśniową fantastyką}
\Clue{30}{}{dział telekomunikacji zajmujący się tworzeniem i likwidacją łączy telekomunikacyjnych}
\Clue{31}{}{szklarnia do uprawiania rozsady}
\Clue{33}{}{narzędzie, które służy do obserwacji odległych obiektów poprzez zbieranie promieniowania elektromagnetycznego (np. światła widzialnego)}
\Clue{34}{}{występowanie zwiększonej lub zmniejszonej aktywności ruchowej}
\Clue{35}{}{ryba z rodziny karpiowatych}
\Clue{36}{}{IGLICZNIA drzewo z rodziny motylkowatych o pniu i gałęziach pokrytych cierniami, w Polsce uprawiana w parkach}\end{PuzzleClues}

\begin{PuzzleClues}{\textbf{Pionowe}\\}\Clue{1}{}{organizacja polityczno-wojskowa powstała 24 sierpnia 1949 na mocy podpisanego 4 kwietnia 1949 Traktatu Północnoatlantyckiego}
\Clue{2}{}{Glaux L. - rodzaj roślin z rodziny pierwiosnkowatych}
\Clue{3}{}{różnica pomiędzy kwotą, jaką nabywca jest skłonny zapłacić za określoną ilość dobra, a kwotą jaką musi rzeczywiście zapłacić; jest miarą korzyści czerpanej z konsumpcji danej ilości dobra; stanowi jedną z kluczowych kategorii analizy w ekonomii dobrobytu}
\Clue{4}{}{miejsce ostrej rywalizacji}
\Clue{6}{}{taktomierz; przyrząd do regulowania szybkości tempa przy wykonaniu utworu muzycznego}
\Clue{8}{}{instrument dęty blaszany z grupy aerofonów ustnikowych, podobny do trąbki}
\Clue{9}{}{element tokarki lub szlifierki zakończony stożkiem}
\Clue{12}{}{mały nóż chirurgiczny o obustronnym ostrzu, posiadającym wybrzuszenie; stosowany do nacięć w celach drenażu}
\Clue{13}{}{Hyla avivoca - gatunek płaza bezogonowego z rodziny rzekotkowatych, zamieszkujący bagna}
\Clue{15}{}{sportowiec zajmujący się spadochroniarstwem zawodowo lub hobbystycznie}
\Clue{17}{}{maser pracujący jako generator drgań o częstotliwości radiowej}
\Clue{20}{}{zakonnik ze zgromadzenia marianistów}
\Clue{21}{}{miasto w Anglii nad Morzem Północnym, znany ośrodek wypoczynkowy}
\Clue{22}{}{bułka z grubo mielonej (razowej) mąki pszennej}
\Clue{25}{}{kretojeż bezogonowy, tenrek zwyczajny, Tenrec ecaudatus - gatunek ssaka z rodziny tenrekowatych, jedyny przedstawiciel rodzaju Tenrec; występuje na Madagaskarze, natomiast na Komorach, Seszelach, Mauritiusie i Reunionie został introdukowany początkowo jako pokarm dla pracowników plantacji}
\Clue{26}{}{lis workowaty; torbacz z rodziny pałanek}
\Clue{29}{}{zatoka między Płw Labradora a prowincją Ontario (Kanada) część Zatoki Hudsona}
\Clue{32}{}{dawna (1977-1999) jednostka zdawkowa w Angoli; 1/100 kwanzy}\end{PuzzleClues}\newpage\section*{Krzyżówka 32}

\noindent\begin{Puzzle}{21}{23}|*	|*	|[1][S]\drarr	|o	|d	|p	|r	|a	|w	|a	|[][,]{ }	|w	|a	|r	|u	|n	|k	|o	|w	|a	|*	|*	|.
|*	|[2][S]\rarr	|w	|ę	|g	|l	|ó	|w	|k	|a	|*	|*	|*	|*	|*	|[3][S]\drarr	|n	|a	|g	|e	|l	|*	|.
|*	|*	|y	|*	|*	|*	|*	|*	|[4][S]\rarr	|z	|j	|a	|d	|l	|i	|w	|o	|ś	|ć	|*	|*	|*	|.
|*	|[5][S]\drarr	|d	|u	|d	|a	|*	|*	|*	|*	|[6][S]\drarr	|p	|e	|t	|a	|r	|d	|a	|*	|*	|*	|*	|.
|*	|c	|a	|[7][S]\rarr	|s	|t	|e	|m	|p	|e	|l	|*	|*	|*	|[8][S]\drarr	|a	|n	|k	|r	|a	|*	|*	|.
|*	|h	|t	|*	|*	|*	|*	|*	|[9][S]\rarr	|p	|o	|d	|u	|s	|z	|k	|a	|*	|*	|[10][S]\darr	|*	|*	|.
|[11][S]\rarr	|l	|e	|n	|*	|[12][S]\rarr	|p	|r	|z	|y	|t	|y	|k	|*	|i	|*	|*	|*	|*	|p	|*	|*	|.
|[13][S]\rarr	|o	|k	|r	|e	|s	|*	|*	|*	|*	|n	|*	|*	|[14][S]\rarr	|a	|z	|y	|m	|u	|t	|*	|*	|.
|*	|r	|[][,]{ }	|[15][S]\drarr	|k	|u	|l	|a	|*	|[16][S]\rarr	|i	|n	|t	|e	|r	|w	|a	|ł	|*	|e	|*	|*	|.
|*	|e	|s	|k	|*	|[17][S]\rarr	|s	|p	|ł	|u	|k	|i	|w	|a	|n	|i	|e	|*	|*	|r	|*	|*	|.
|[18][S]\drarr	|k	|o	|ł	|p	|a	|c	|z	|e	|k	|*	|[19][S]\rarr	|u	|p	|i	|e	|k	|*	|*	|o	|*	|*	|.
|o	|[][,]{ }	|c	|a	|[20][S]\rarr	|g	|r	|u	|b	|o	|s	|k	|ó	|r	|c	|e	|*	|*	|*	|r	|[21][S]\darr	|*	|.
|p	|h	|j	|m	|*	|*	|*	|*	|[22][S]\rarr	|m	|i	|m	|o	|z	|a	|*	|*	|*	|*	|y	|t	|*	|.
|o	|e	|a	|c	|*	|*	|*	|*	|*	|*	|[23][S]\rarr	|b	|e	|z	|[][,]{ }	|c	|z	|a	|r	|n	|y	|*	|.
|s	|m	|l	|z	|*	|*	|*	|*	|*	|[24][S]\rarr	|r	|e	|m	|i	|z	|a	|*	|*	|*	|c	|r	|*	|.
|*	|a	|n	|u	|[25][S]\darr	|*	|*	|*	|*	|*	|*	|*	|[26][S]\rarr	|k	|ł	|ą	|b	|*	|*	|h	|o	|*	|.
|*	|t	|y	|c	|l	|*	|*	|*	|*	|*	|*	|[27][S]\rarr	|z	|r	|o	|ś	|l	|a	|k	|*	|l	|*	|.
|*	|y	|*	|h	|a	|[28][S]\rarr	|j	|a	|s	|k	|r	|a	|w	|o	|ś	|ć	|*	|*	|*	|*	|c	|*	|.
|*	|n	|*	|*	|i	|[29][S]\darr	|[30][S]\rarr	|o	|w	|a	|d	|o	|p	|y	|l	|n	|o	|ś	|ć	|*	|z	|*	|.
|*	|y	|*	|[31][S]\rarr	|s	|k	|a	|l	|a	|[][,]{ }	|r	|a	|n	|k	|i	|n	|e	|[][S]'	|a	|*	|y	|*	|.
|*	|*	|*	|*	|*	|u	|*	|[32][S]\rarr	|k	|o	|n	|s	|e	|r	|w	|a	|c	|j	|a	|*	|k	|*	|.
|*	|[33][S]\rarr	|z	|o	|o	|n	|o	|z	|a	|*	|[34][S]\rarr	|m	|e	|g	|a	|c	|e	|r	|o	|s	|*	|*	|.
|[35][S]\rarr	|d	|r	|e	|n	|a	|ż	|[][,]{ }	|m	|ó	|z	|g	|ó	|w	|*	|*	|*	|*	|*	|*	|*	|*	|.
|*	|*	|*	|*	|*	|*	|*	|*	|*	|*	|*	|*	|*	|*	|*	|*	|*	|*	|*	|*	|*	|*	|.\end{Puzzle}

\newpage

\begin{PuzzleClues}{\textbf{Poziome}\\}\Clue{1}{}{odprawa celna wolna od należności przywozowych}
\Clue{2}{}{wędka wykonana z włókna węglowego}
\Clue{3}{}{amerykański filozof i logik (1901-85); badanie wartości poznawczej twierdzeń naukowych}
\Clue{4}{}{cecha człowieka: zgryźliwość, złośliwość, uszczypliwość}
\Clue{5}{}{Gracz; ur. w 1941 r. malarz, uprawia malarstwo, rysunek i grafikę}
\Clue{6}{}{silny cios pięścią zwłaszcza w twarz}
\Clue{7}{}{znamię, charakterystyczna cecha, znak czegoś}
\Clue{8}{}{KOTEW; klamra spinająca mury}
\Clue{9}{}{miękkie podparcie dla głowy, rodzaj worka wypchanego czymś miękkim (np. pierzem, włosiem) używanego jako element pościeli}
\Clue{11}{}{jednoroczna roślina zielna lub bylina, uprawiana dla włókna otrzymywanego z łodyg i oleju z nasion, tzw. siemię}
\Clue{12}{}{wieś w województwie mazowieckim}
\Clue{13}{}{przedział czasowy}
\Clue{14}{}{kąt między płaszczyzną południka niebieskiego i płaszczyzną przez dany obiekt oraz przez zenit i nadir}
\Clue{15}{}{nabój wystrzeliwany z broni palnej}
\Clue{16}{}{różnice wysokości między dwoma dźwiękami współbrzmiącymi lub następującymi po sobie}
\Clue{17}{}{proces rzeźbotwórczy polegający na wymywaniu zwietrzeliny i gleby oraz ich przemieszczeniu w dół stoku przez wodę deszczową lub roztopową}
\Clue{18}{}{Panaeolus - rodzaj grzybów należący do rzędu pieczarkowców, jego przynależność do którejś z rodzin jest niepewna}
\Clue{19}{}{wypiek, wypiekanie, pieczenie}
\Clue{20}{}{Ceratomorpha - podrząd ssaków z rzędu nieparzystokopytnych; należą do niego rodziny: nosorożcowate i tapirowate}
\Clue{22}{}{drink zawierający równe ilości szampana (lub innego wina musującego) i soku z cytrusów (najczęściej pomarańczowego)}
\Clue{23}{}{Sambucus nigra - gatunek rośliny z rodziny piżmaczkowatych (Adoxaceae), dawniej zaliczany był także do rodziny bzowatych (Sambucaceae) i przewiertniowatych (Caprifoliaceae)}
\Clue{24}{}{pomieszczenia służące do przechowywania podczas postoju pojazdów strażackich i szynowych}
\Clue{26}{}{część ręki człowieka}
\Clue{27}{}{bliźnię, określane tak przez nierozłączność z drugim bliźnięciem}
\Clue{28}{}{duża intensywność koloru}
\Clue{30}{}{zapylanie kwiatów pyłkiem przenoszonym przez owady}
\Clue{31}{}{absolutna skala termometryczna opracowana przez Williama Rankine’a, w skali tej zero oznacza najniższą możliwą temperaturę, jaką może mieć kryształ doskonały, w którym ustały wszelkie drgania cząsteczek (zero bezwzględne); jest odpowiednikiem skali Kelwina dla stopni Fahrenheita}
\Clue{32}{}{różne sposoby przetwarzania i przechowywania żywności, których celem jest wydłużenie jej trwałości i zapewnienie bezpieczeństwa jej spożycia}
\Clue{33}{}{zakaźna lub pasożytnicza choroba zwierząt bądź przez zwierzęta tylko roznoszona, przenosząca się na człowieka poprzez kontakt bezpośredni lub surowce pochodzenia zwierzęcego, rzadziej drogą powietrzną}
\Clue{34}{}{wielki eurazjatycki jeleń z epoki lodowcowej o rozpiętości poroża do 4 m}
\Clue{35}{}{sytuacja mająca miejsce, gdy specjaliści wysokiej klasy skłaniani są do podejmowania pracy w krajach uprzemysłowionych oraz zapewnianie im są lepsze warunki ekonomiczne}\end{PuzzleClues}

\begin{PuzzleClues}{\textbf{Pionowe}\\}\Clue{1}{}{kwota pieniężna przeznaczane na świadczenia społeczne: ubezpieczenia, pomoc społeczną, świadczenia rodzinne i wydatki związane z polityką rynku pracy; także pieniądze przeznaczane na funkcjonowanie instytucji publicznych zarządzających tymi świadczeniami}
\Clue{3}{}{pojazd, który wskutek uszkodzeń lub starości przestał nadawać się do użytku}
\Clue{5}{}{pochodna hemu, która zawiera trójwartościowy atom żelaza (Fe3+) powstający z hemoglobiny bądź hemu pod wpływem działania bardzo silnych związków utleniających}
\Clue{6}{}{człowiek, który lata za pomocą jakiegoś pojazdu, zwykle: członek załogi, obsługi tego pojazdu}
\Clue{8}{}{układowa choroba nowotworowa układu chłonnego (chłoniak), zajmująca węzły chłonne i pozawęzłową tkankę limfatyczną}
\Clue{10}{}{Pterorhynchus - rodzaj pterozaura z rodziny ramforynchów; występował w jurze późnej na terenie współczesnej Mongolii Wewnętrznej i Chin}
\Clue{15}{}{osoba, która często kłamie, mówi nieprawdę}
\Clue{18}{}{zwierzę z dydelfów; drapieżnik wielkości kota}
\Clue{21}{}{mieszkaniec kraju związkowego Tyrol}
\Clue{25}{}{średniowieczna pieśń religijna}
\Clue{29}{}{historyczne narzędzie kary; obręcz lub obręcze żelazne składające się z dwóch zamykanych części, przytwierdzone łańcuchem lub powrozem do pręgierza, portalu ratusza lub kościoła, czasem przy wejściu do dworu, wkładane na szyję lub rękę skazańcom i zamykane na kłódkę}\end{PuzzleClues}\newpage\section*{Krzyżówka 33}

\noindent\begin{Puzzle}{21}{30}|*	|*	|*	|*	|*	|*	|*	|*	|*	|[1][S]\drarr	|s	|t	|y	|g	|a	|l	|*	|*	|[2][S]\darr	|*	|*	|*	|.
|*	|[3][S]\drarr	|z	|r	|o	|s	|t	|n	|i	|c	|z	|e	|k	|*	|[4][S]\darr	|*	|*	|*	|s	|*	|*	|*	|.
|*	|k	|[5][S]\drarr	|p	|s	|o	|w	|a	|c	|z	|[][,]{ }	|r	|ó	|ż	|a	|n	|y	|*	|o	|*	|*	|*	|.
|*	|o	|o	|[6][S]\drarr	|c	|o	|u	|r	|b	|e	|t	|*	|*	|*	|r	|*	|*	|[7][S]\darr	|d	|*	|*	|*	|.
|*	|ł	|k	|p	|[8][S]\darr	|[9][S]\darr	|*	|[10][S]\rarr	|l	|r	|d	|*	|*	|*	|a	|[11][S]\darr	|[12][S]\darr	|r	|a	|*	|[13][S]\darr	|[14][S]\darr	|.
|*	|a	|s	|a	|p	|c	|*	|*	|*	|w	|*	|*	|*	|*	|b	|f	|b	|o	|[][,]{ }	|*	|s	|t	|.
|[15][S]\drarr	|c	|y	|p	|r	|y	|s	|[][,]{ }	|g	|o	|w	|e	|n	|a	|*	|o	|i	|p	|o	|[16][S]\darr	|k	|v	|.
|p	|z	|d	|i	|a	|p	|[17][S]\drarr	|t	|a	|n	|i	|e	|c	|[][,]{ }	|b	|r	|z	|u	|c	|h	|a	|*	|.
|i	|k	|a	|e	|c	|r	|s	|*	|*	|a	|[18][S]\darr	|[19][S]\darr	|*	|[20][S]\darr	|*	|m	|o	|s	|z	|o	|l	|*	|.
|l	|o	|z	|r	|o	|*	|a	|[21][S]\darr	|*	|[][,]{ }	|o	|z	|*	|p	|*	|a	|n	|z	|y	|p	|e	|*	|.
|e	|w	|a	|[][,]{ }	|h	|[22][S]\darr	|l	|p	|[23][S]\drarr	|k	|r	|a	|t	|a	|*	|c	|*	|k	|s	|l	|ń	|*	|.
|r	|o	|*	|ś	|o	|e	|a	|e	|b	|a	|n	|b	|[24][S]\darr	|d	|*	|j	|*	|a	|z	|u	|[][,]{ }	|*	|.
|s	|*	|*	|c	|l	|g	|m	|ł	|a	|r	|i	|ł	|c	|w	|*	|a	|*	|[][,]{ }	|c	|r	|a	|*	|.
|*	|[25][S]\darr	|*	|i	|i	|i	|a	|n	|w	|t	|t	|o	|h	|a	|*	|[][,]{ }	|[26][S]\darr	|t	|z	|[][,]{ }	|w	|*	|.
|*	|f	|*	|e	|k	|p	|n	|o	|e	|k	|o	|c	|i	|n	|*	|m	|p	|y	|o	|o	|e	|*	|.
|*	|a	|[27][S]\darr	|r	|*	|c	|d	|r	|ł	|a	|z	|i	|p	|*	|[28][S]\darr	|ł	|ę	|r	|n	|b	|n	|*	|.
|[29][S]\drarr	|n	|o	|n	|a	|j	|r	|o	|n	|*	|u	|a	|p	|[30][S]\darr	|a	|y	|t	|r	|a	|r	|t	|*	|.
|k	|a	|r	|y	|*	|a	|a	|ż	|a	|[31][S]\darr	|c	|n	|e	|j	|c	|n	|l	|e	|*	|o	|u	|*	|.
|a	|t	|d	|*	|*	|n	|[][,]{ }	|c	|*	|e	|h	|i	|n	|a	|a	|a	|a	|ń	|[32][S]\darr	|ż	|r	|*	|.
|w	|y	|y	|*	|*	|i	|a	|e	|*	|u	|*	|n	|d	|r	|p	|*	|[][,]{ }	|s	|f	|n	|y	|*	|.
|a	|c	|n	|*	|*	|n	|n	|*	|*	|r	|*	|*	|a	|z	|u	|*	|o	|k	|u	|y	|n	|*	|.
|*	|z	|a	|*	|*	|*	|a	|*	|*	|y	|*	|*	|l	|y	|l	|[33][S]\rarr	|c	|a	|n	|*	|o	|*	|.
|[34][S]\drarr	|k	|r	|e	|d	|y	|t	|[][,]{ }	|s	|t	|u	|d	|e	|n	|c	|k	|i	|*	|k	|*	|w	|*	|.
|z	|a	|n	|[35][S]\rarr	|k	|r	|o	|c	|h	|m	|a	|l	|*	|k	|o	|*	|e	|*	|*	|*	|y	|*	|.
|n	|*	|o	|*	|*	|[36][S]\rarr	|l	|i	|s	|i	|c	|z	|k	|a	|*	|*	|k	|*	|*	|*	|*	|*	|.
|a	|*	|ś	|*	|[37][S]\rarr	|b	|i	|r	|u	|a	|n	|g	|*	|*	|*	|[38][S]\rarr	|o	|s	|o	|b	|a	|*	|.
|c	|*	|ć	|*	|*	|*	|j	|*	|*	|*	|*	|*	|[39][S]\rarr	|z	|j	|a	|w	|i	|s	|k	|o	|*	|.
|z	|*	|*	|*	|*	|[40][S]\rarr	|s	|y	|s	|t	|e	|m	|[][,]{ }	|t	|o	|n	|a	|l	|n	|y	|*	|*	|.
|e	|[41][S]\rarr	|k	|o	|r	|e	|k	|t	|y	|w	|n	|o	|ś	|ć	|*	|*	|*	|*	|*	|*	|*	|*	|.
|k	|*	|*	|[42][S]\rarr	|p	|t	|a	|s	|z	|n	|i	|k	|[][,]{ }	|z	|d	|o	|b	|i	|o	|n	|y	|*	|.
|*	|[43][S]\rarr	|d	|a	|c	|h	|*	|*	|*	|*	|*	|*	|*	|*	|*	|*	|*	|*	|*	|*	|*	|*	|.\end{Puzzle}

\newpage

\begin{PuzzleClues}{\textbf{Poziome}\\}\Clue{1}{}{środowisko wód podziemnych, zamieszkane przez specyficzną faunę (nazywaną stygonem, stygofauną)}
\Clue{3}{}{Zygodon - rodzaj mchów z rodziny szurpkowatych; mchy nadrzewne rosnące w poduszkowatych darniach}
\Clue{5}{}{Allantus cinctus - owad z rodziny pilarzowatych, szkodnik róż}
\Clue{6}{}{grupa artystów działających w XII-XIV w. w Rzymie}
\Clue{10}{}{kod ISO 4217 dolara liberyjskiego}
\Clue{15}{}{Cupressus goveniana - gatunek z rodziny cyprysowatych (Cupressaceae); pochodzi z Ameryki Północnej, gdzie występuje na kilku stanowiskach wyłącznie w stanie Kalifornia}
\Clue{17}{}{rodzaj tańca wykonywanego głównie (ale nie wyłącznie) przez kobiety, pochodzi z terenów Bliskiego Wschodu i Afryki Północnej, polegający na ruchach biodrami i mięśniami brzucha}
\Clue{23}{}{tyle ile mieści się w specyficznym sztywnym opakowaniu do transportowania butelek, w którym każda butelka ma wyodrębnione miejsce}
\Clue{29}{}{koszula ze sztucznego materiału, która nie wymaga prasowania, modna w późnym PRLu}
\Clue{33}{}{kod ISO 4217 dolara kanadyjskiego}
\Clue{34}{}{kredyt preferencyjny przeznaczony dla studentów; jest on wypłacany co miesiąc w czasie trwania roku akademickiego, a jego spłatę można rozpocząć najpóźniej dwa lata po ukończeniu studiów}
\Clue{35}{}{roztwór komercyjnie dostępnego krochmalu używany w celach przemysłowych, spożywczych, higienicznych}
\Clue{36}{}{zdrobniale: lisica - samica lisa}
\Clue{37}{}{niedźwiedź malajski; czarny niedźwiedź z białą podkową na piersi, roślino- i owadożerny}
\Clue{38}{}{kategoria gramatyczna czasownika określająca, jak podmiot zdania ma się do stanu lub czynności określonej przez orzeczenie}
\Clue{39}{}{wyjątkowo piękna kobieta}
\Clue{40}{}{system dźwiękowy porządkujący materiał dźwiękowy w ten sposób, że jedne dźwięki są zależne od innych na zasadzie swoistej hierarchii}
\Clue{41}{}{stopień, w jakim można skorygować wadę postawy}
\Clue{42}{}{Poecilotheria ornata - gatunek nadrzewnego ptasznika, występujący w rejonach górskich Sri Lanki}
\Clue{43}{}{przykrycie osłaniające pomieszczenia budowli przed opadami atmosferycznymi, słońcem, wiatrem itp}\end{PuzzleClues}

\begin{PuzzleClues}{\textbf{Pionowe}\\}\Clue{1}{}{sprzeciw wobec czegoś}
\Clue{2}{}{nieorganiczny związek chemiczny z grupy wodorowęglanów, wodorosól kwasu węglowego i sodu}
\Clue{3}{}{wieś w Polsce położona w województwie kujawsko-pomorskim, w powiecie nakielskim, w gminie Szubin}
\Clue{4}{}{mieszkaniec Arabii, człowiek pochodzenia arabskiego}
\Clue{5}{}{enzym katalizujący przenoszenie wodoru na tlen, w wyniku czego powstaje woda lub nadtlenek wodoru}
\Clue{6}{}{jeden z rodzajów wyrobów ściernych przeznaczonych do obróbki ściernej powierzchni przedmiotów wykonanych z takich materiałów jak drewno, metal, tworzywa sztuczne}
\Clue{7}{}{Discoglossus sardus - gatunek płaza bezogonowego z rodziny ropuszkowatych, występujący na wyspach Morza Tyrreńskiego - Sardynii, Korsyce, Monte Cristo i Isles d'Hyeres}
\Clue{8}{}{ktoś, kto bardzo dużo pracuje, jest pracowity}
\Clue{9}{}{azjatycka wyspa we wschodniej części Morza Śródziemnego, często traktowana jako część Bliskiego Wschodu, czasami jednak zaliczana do Europy, historycznie, kulturowo i politycznie stanowiąca część Europy}
\Clue{11}{}{w rugby - jedna z dwóch formacji w rugby}
\Clue{12}{}{żubr amerykański, Bison bison - duży ssak łożyskowy z rodziny krętorogich, rzędu parzystokopytnych, największy obecnie ssak Ameryki Północnej; zamieszkuje prerie oraz rzadkie, prześwietlone lasy Ameryki Północnej}
\Clue{13}{}{odmiana skalenia wykazująca efekt awenturyzacji przejawiających się w obserwowanych radialnie migotliwych refleksach w odcieniach czerwonawych, niekiedy zielonkawych lub żółtawych}
\Clue{14}{}{telewizja, dział telekomunikacji zajmujący się przekazywaniem ruchomego obrazu oraz dźwięku na odległość}
\Clue{15}{}{podpokładowa podpora ustawiona między dnem a pokładem statku wodnego}
\Clue{16}{}{Oplurus cyclurus - gatunek gada z rodziny madagaskarkowatych, występujący na Madagaskarze}
\Clue{17}{}{salamandra egejska, Lyciasalamandra helverseni - gatunek płaza ogoniastego z rodziny salamandrowatych, występujący na wyspach Kasos, Karpathos i Saria oraz w południowo-zachodniej Turcji}
\Clue{18}{}{Ornithosuchus - drapieżny wymarły archozaur z grupy Crurotarsi; żył w triasie około 220 mln lat temu na terenie dzisiejszej Szkocji}
\Clue{19}{}{mieszkaniec Zabłocia - wsi na Śląsku Cieszyńskim}
\Clue{20}{}{gatunek pieśni lub piosenki miłosnej pojawiający się w poezji mieszczańskiej w XVI i XVII wieku}
\Clue{21}{}{jeleniowate, Cervidae - rodzina ssaków z rzędu parzystokopytnych, do której należą zwierzęta o kostnym, pełnym porożu (w przeciwieństwie do rogów) ulegającym u większości gatunków corocznej zmianie; zamieszkują lasy, lasostepy, tereny bagniste i tundrę wszystkich kontynentów poza Australią, gdzie niektóre gatunki zostały introdukowane}
\Clue{22}{}{mieszkaniec Egiptu, człowiek pochodzenia egipskiego}
\Clue{23}{}{włókno okrywające nasiona drzew i krzewów tej rośliny stanowiące surowiec do wyrobu przędzy, nici i celulozy, także tkanina z tego włókna}
\Clue{24}{}{mężczyzna-striptizer, świadczy też inne usługi kobietom, umila czas}
\Clue{25}{}{kobieta/dziewczyna, która podchodzi do czegoś bardzo entuzjastycznie, uwielbia coś (także robić)}
\Clue{26}{}{część kabla zasilającego}
\Clue{27}{}{cecha zachowania, o ordynarnym, prostackim zachowaniu}
\Clue{28}{}{miasto i port w Meksyku (Guerrero) nad Oceanem Spokojnym, światowej sławy kąpielisko i ośrodek turystyczno-wypoczynkowy}
\Clue{29}{}{porcja kawy tj. napoju z zaparzonej kawy}
\Clue{30}{}{potrawa, dodatek z jarzyn}
\Clue{31}{}{to opracowana przez Rudolfa Steinera w roku 1911, inspirowana antropozofią sztuka poruszania się przy muzyce i odpowiednich tekstach (przypomina formą taniec)}
\Clue{32}{}{w sztuce - odmiana pop-artu}
\Clue{34}{}{mały znak}\end{PuzzleClues}\newpage\section*{Krzyżówka 34}

\noindent\begin{Puzzle}{21}{33}|*	|*	|*	|*	|*	|*	|[1][S]\darr	|*	|*	|*	|*	|*	|*	|*	|*	|*	|*	|*	|*	|*	|*	|*	|.
|*	|*	|*	|*	|*	|*	|d	|*	|*	|*	|*	|*	|*	|*	|*	|[2][S]\darr	|*	|*	|[3][S]\darr	|*	|*	|*	|.
|*	|*	|*	|*	|*	|*	|z	|*	|*	|*	|*	|*	|*	|*	|*	|b	|*	|*	|c	|*	|*	|*	|.
|*	|*	|*	|*	|*	|*	|i	|*	|*	|*	|*	|*	|*	|*	|*	|i	|*	|*	|h	|*	|*	|*	|.
|*	|*	|*	|*	|[4][S]\darr	|*	|e	|*	|*	|*	|[5][S]\darr	|*	|[6][S]\darr	|*	|*	|l	|*	|*	|r	|*	|*	|*	|.
|*	|*	|*	|*	|o	|*	|d	|*	|*	|*	|s	|*	|s	|*	|*	|e	|*	|*	|o	|*	|*	|*	|.
|*	|*	|*	|*	|k	|*	|z	|*	|*	|*	|o	|*	|a	|*	|*	|t	|*	|*	|m	|*	|*	|*	|.
|*	|*	|*	|*	|t	|*	|i	|*	|*	|*	|s	|*	|l	|*	|*	|[][,]{ }	|*	|*	|o	|*	|*	|*	|.
|*	|*	|*	|*	|a	|*	|c	|*	|*	|*	|[][,]{ }	|*	|a	|*	|*	|a	|*	|*	|s	|*	|*	|*	|.
|*	|*	|*	|*	|w	|*	|t	|*	|*	|*	|m	|*	|m	|*	|[7][S]\darr	|b	|*	|*	|o	|*	|*	|*	|.
|*	|*	|*	|*	|a	|*	|w	|*	|*	|*	|a	|*	|a	|[8][S]\drarr	|k	|o	|l	|u	|m	|n	|a	|*	|.
|*	|*	|*	|*	|[][,]{ }	|*	|o	|*	|*	|[9][S]\drarr	|j	|i	|n	|l	|o	|n	|g	|*	|[][,]{ }	|*	|[10][S]\darr	|*	|.
|*	|*	|*	|*	|w	|*	|[][,]{ }	|*	|*	|t	|o	|*	|d	|o	|n	|a	|*	|*	|s	|*	|p	|*	|.
|*	|*	|*	|*	|i	|*	|k	|*	|*	|r	|n	|*	|r	|g	|w	|m	|*	|*	|u	|*	|a	|*	|.
|*	|[11][S]\darr	|*	|*	|e	|*	|u	|*	|[12][S]\darr	|a	|e	|*	|a	|a	|e	|e	|*	|*	|b	|*	|r	|*	|.
|*	|h	|*	|*	|l	|*	|l	|*	|n	|w	|z	|*	|[][,]{ }	|n	|r	|n	|*	|*	|m	|*	|k	|*	|.
|*	|i	|*	|*	|k	|*	|t	|*	|o	|e	|o	|*	|m	|*	|g	|t	|*	|*	|e	|*	|[][,]{ }	|*	|.
|*	|p	|*	|*	|a	|*	|u	|*	|u	|r	|w	|*	|e	|*	|e	|o	|*	|*	|t	|*	|p	|*	|.
|*	|e	|*	|*	|n	|*	|r	|*	|m	|s	|y	|*	|k	|*	|n	|w	|*	|*	|a	|*	|r	|*	|.
|*	|r	|*	|*	|o	|*	|o	|*	|e	|*	|*	|*	|s	|*	|c	|y	|*	|*	|c	|*	|z	|*	|.
|*	|w	|*	|*	|c	|*	|w	|*	|n	|*	|*	|*	|y	|*	|j	|*	|*	|*	|e	|*	|e	|*	|.
|*	|e	|*	|[13][S]\drarr	|n	|i	|e	|w	|o	|l	|n	|i	|k	|*	|a	|*	|*	|*	|n	|*	|m	|*	|.
|*	|n	|*	|t	|a	|*	|*	|[14][S]\darr	|n	|*	|[15][S]\rarr	|r	|a	|k	|[][,]{ }	|b	|ł	|o	|t	|n	|y	|*	|.
|*	|t	|*	|r	|*	|*	|*	|p	|*	|*	|*	|*	|ń	|*	|b	|*	|*	|*	|r	|*	|s	|*	|.
|*	|y	|*	|e	|*	|*	|*	|u	|*	|*	|*	|*	|s	|*	|e	|*	|*	|*	|y	|*	|ł	|*	|.
|*	|l	|*	|p	|[16][S]\drarr	|k	|o	|ł	|e	|c	|z	|e	|k	|*	|t	|*	|*	|*	|c	|*	|o	|*	|.
|*	|a	|*	|*	|t	|*	|*	|k	|*	|*	|*	|*	|a	|*	|a	|*	|*	|*	|z	|*	|w	|*	|.
|[17][S]\rarr	|c	|z	|e	|r	|y	|m	|o	|j	|a	|*	|*	|*	|*	|*	|*	|*	|*	|n	|*	|y	|*	|.
|*	|j	|*	|*	|y	|*	|*	|w	|*	|*	|*	|*	|*	|*	|*	|*	|*	|*	|y	|*	|*	|*	|.
|*	|a	|*	|*	|p	|*	|*	|n	|*	|*	|*	|*	|*	|*	|*	|*	|*	|*	|*	|*	|*	|*	|.
|*	|*	|*	|*	|l	|*	|*	|i	|*	|*	|*	|*	|*	|*	|*	|*	|*	|*	|*	|*	|*	|*	|.
|*	|*	|*	|[18][S]\rarr	|e	|p	|o	|k	|a	|*	|*	|*	|*	|*	|*	|*	|*	|*	|*	|*	|*	|*	|.
|*	|*	|*	|*	|t	|*	|*	|*	|*	|*	|*	|*	|*	|*	|*	|*	|*	|*	|*	|*	|*	|*	|.
|*	|*	|*	|*	|*	|*	|*	|*	|*	|*	|*	|*	|*	|*	|*	|*	|*	|*	|*	|*	|*	|*	|.\end{Puzzle}

\newpage

\begin{PuzzleClues}{\textbf{Poziome}\\}\Clue{8}{}{element układu kierowniczego, który służy do przeniesienia momentu obrotowego z koła kierownicy na przekładnię kierowniczą, co zapewnia proporcjonalny skręt kół samochodu}
\Clue{9}{}{Yinlong - rodzaj prymitywnego ceratopsa, żyjący w okresie późnej jury na terenach Azji; osiągał około 1,2 metra długości}
\Clue{13}{}{ktoś, kto jest własnością innej osoby lub instytucji, która może nim dowolnie rozporządzać}
\Clue{15}{}{Astacus leptodactylus syn. Pontastacus leptodactylus - jeden z występujących w Polsce gatunków raków, skorupiak (Crustacea) z rzędu dziesięcionogów (Decapoda)}
\Clue{16}{}{zdrobniale: kołek - najczęściej drewniany, podłużny element, czasami ma zaostrzony koniec, może być np. częścią płotu}
\Clue{17}{}{Annona cherimola - gatunek drzewa z rodziny flaszowcowatych}
\Clue{18}{}{moment stanowiący punkt odniesienia w opisie liczbowym zjawiska astronomicznego}\end{PuzzleClues}

\begin{PuzzleClues}{\textbf{Pionowe}\\}\Clue{1}{}{zasób rzeczy nieruchomych i ruchomych wraz ze związanymi z nim wartościami duchowymi, zjawiskami historycznymi i obyczajowymi, uznawany za godny ochrony prawnej dla dobra społeczeństwa i jego rozwoju oraz przekazania następnym pokoleniom z uwagi na zrozumiałe i akceptowane wartości historyczne, patriotyczne, religijne, naukowe i artystyczne, mające znaczenie dla tożsamości i ciągłości rozwoju politycznego, społecznego i kulturalnego, dowodzenia prawd i upamiętniania wydarzeń historycznych, kultywowania poczucia piękna i wspólnoty cywilizacyjnej}
\Clue{2}{}{bilet uprawniający do odbycia określonej ilości przejazdów po cenie niższej niż regularna}
\Clue{3}{}{chromosom, w którym centromer położony jest w pobliżu środka chromosomu, ale nie w środku, co sprawia, że podczas metafazy i anafazy przybiera kształt litery L}
\Clue{4}{}{uroczyste obchody zmartwychwstania Jezusa w religiach chrześcijańskich}
\Clue{5}{}{zimny, emulsyjny sos na bazie oliwy z dodatkiem surowego żółtka}
\Clue{6}{}{Ambystoma mexicanum - endemiczny słodkowodny gatunek drapieżnego płaza ogoniastego z rodziny ambystomowatych (Ambystomatidae); cechą charakterystyczną tego gatunku jest zjawisko neotenii, czyli zdolność do rozmnażania się płciowego postaci larwalnej}
\Clue{7}{}{koncepcja dotycząca zależności pomiędzy średnią stopą wzrostu dochodu per capita a początkowym poziomem tego dochodu}
\Clue{8}{}{model samochodu osobowego klasy B produkowany od 2004 roku przez markę Dacia we współpracy z koncernem Renault, do którego należy marka}
\Clue{9}{}{ćwiczenie ujeżdżeniowe w jeździectwie - jest to ruch boczny, w którym głowa konia jest zwrócona w kierunku ruchu, łopatki przesuwają się po torze równoległym do ściany, zad skierowany jest do wewnątrz, a kończyny zewnętrzne krzyżują sięponad wewnętrznymi}
\Clue{10}{}{wyodrębniony zespół nieruchomości, w skład którego wchodzi również nieruchomość, na której znajduje się infrastruktura techniczna pozostała po restrukturyzowanym lub likwidowanym przedsiębiorcy}
\Clue{11}{}{nieprawidłowy, przyspieszony lub pogłębiony oddech}
\Clue{12}{}{u Kanta: rzecz sama w sobie, niepoznawalny przedmiot, manifestujący się przez zjawiska, niedostępny poznaniu zmysłowemu}
\Clue{13}{}{zawodowy żołnierz, często oceniany stereotypowo jako osoba o wojskowych (oszczędnych, szorstkich, ale i rubasznych, prostackich) manierach i konserwatywnych poglądach, posądzana o tępotę, upartość i ciężki charakter}
\Clue{14}{}{najwyższy stopień oficerski w grupie oficerów starszych}
\Clue{16}{}{w fotografice: obiektyw anastygmatyczny złożony z trzech członów}\end{PuzzleClues}\newpage\section*{Krzyżówka 35}

\noindent\begin{Puzzle}{24}{21}|*	|*	|*	|*	|*	|[1][S]\drarr	|b	|e	|n	|t	|o	|*	|*	|*	|*	|*	|*	|*	|*	|*	|*	|*	|*	|*	|*	|.
|*	|*	|*	|*	|*	|l	|[2][S]\drarr	|r	|a	|c	|h	|u	|n	|e	|k	|[][,]{ }	|c	|a	|ł	|k	|o	|w	|y	|*	|*	|.
|*	|*	|[3][S]\darr	|[4][S]\rarr	|k	|u	|s	|a	|k	|*	|*	|*	|*	|*	|*	|*	|*	|*	|*	|*	|*	|*	|*	|*	|*	|.
|*	|*	|s	|[5][S]\darr	|*	|c	|t	|[6][S]\darr	|*	|*	|*	|*	|*	|*	|*	|*	|*	|*	|*	|*	|*	|*	|*	|*	|*	|.
|*	|*	|k	|p	|[7][S]\darr	|e	|a	|ł	|*	|*	|[8][S]\drarr	|z	|j	|a	|d	|l	|i	|w	|o	|ś	|ć	|*	|[9][S]\darr	|*	|*	|.
|*	|*	|a	|a	|t	|r	|n	|a	|*	|[10][S]\rarr	|k	|o	|r	|p	|o	|r	|a	|c	|j	|a	|*	|*	|g	|*	|*	|.
|*	|[11][S]\drarr	|k	|l	|i	|n	|o	|p	|i	|r	|o	|k	|s	|e	|n	|*	|*	|*	|*	|*	|[12][S]\drarr	|c	|e	|l	|*	|.
|*	|p	|u	|l	|p	|a	|w	|a	|*	|*	|n	|*	|*	|*	|*	|*	|*	|*	|*	|*	|p	|*	|n	|*	|*	|.
|*	|r	|n	|a	|p	|*	|c	|c	|*	|[13][S]\rarr	|t	|u	|n	|g	|a	|r	|*	|*	|[14][S]\darr	|*	|o	|[15][S]\darr	|e	|*	|*	|.
|*	|z	|o	|*	|l	|*	|z	|z	|[16][S]\rarr	|g	|e	|h	|e	|n	|n	|a	|*	|*	|f	|*	|s	|k	|r	|*	|*	|.
|*	|e	|w	|*	|e	|[17][S]\rarr	|o	|k	|o	|l	|n	|i	|c	|a	|*	|*	|*	|*	|r	|*	|t	|a	|a	|*	|*	|.
|*	|r	|a	|*	|r	|*	|ś	|a	|[18][S]\rarr	|m	|e	|l	|i	|p	|o	|n	|y	|*	|a	|*	|ę	|p	|ł	|*	|*	|.
|*	|a	|t	|*	|*	|*	|ć	|*	|[19][S]\rarr	|g	|r	|o	|m	|a	|d	|k	|a	|r	|z	|*	|p	|i	|[][,]{ }	|*	|*	|.
|*	|b	|e	|*	|*	|*	|*	|*	|[20][S]\drarr	|t	|o	|k	|s	|e	|m	|i	|a	|*	|a	|*	|a	|t	|b	|*	|*	|.
|*	|i	|*	|*	|*	|*	|*	|*	|f	|[21][S]\rarr	|w	|i	|n	|y	|l	|e	|u	|m	|*	|*	|c	|a	|r	|*	|*	|.
|*	|a	|[22][S]\drarr	|n	|i	|e	|k	|r	|o	|p	|i	|e	|ń	|[][,]{ }	|d	|e	|l	|i	|k	|a	|t	|n	|y	|*	|*	|.
|*	|c	|w	|[23][S]\rarr	|l	|e	|v	|e	|r	|i	|e	|r	|*	|*	|*	|*	|*	|*	|*	|*	|w	|a	|g	|*	|*	|.
|[24][S]\rarr	|z	|a	|k	|r	|ę	|t	|*	|d	|[25][S]\rarr	|c	|a	|r	|s	|k	|i	|e	|[][,]{ }	|w	|r	|o	|t	|a	|*	|*	|.
|*	|*	|c	|*	|*	|*	|*	|*	|[][,]{ }	|*	|*	|*	|*	|*	|*	|*	|*	|*	|*	|*	|*	|*	|d	|*	|*	|.
|*	|*	|h	|*	|*	|*	|[26][S]\rarr	|s	|t	|e	|r	|o	|l	|o	|t	|k	|a	|*	|*	|[27][S]\rarr	|u	|t	|y	|k	|*	|.
|*	|[28][S]\rarr	|a	|m	|b	|a	|*	|*	|*	|*	|*	|*	|*	|*	|*	|*	|*	|*	|*	|*	|*	|*	|*	|*	|*	|.
|*	|*	|*	|*	|*	|*	|*	|*	|*	|*	|*	|*	|*	|*	|*	|*	|*	|*	|*	|*	|*	|*	|*	|*	|*	|.\end{Puzzle}

\newpage

\begin{PuzzleClues}{\textbf{Poziome}\\}\Clue{1}{}{rodzaj posiłku kuchni japońskiej, mający postać pojedynczej porcji na wynos, kupowanej w punktach gastronomicznych lub przygotowywanej w domu}
\Clue{2}{}{dział matematyki zajmujący się badaniem funkcji zmiennej rzeczywistej lub zespolonej w oparciu o podstawowe dla tej dyscypliny matematycznej pojęcia pochodnych i całek}
\Clue{4}{}{chrząszcz drapieżny o bardzo skróconych pokrywach, padlinożerny, grzybożerny lub humusożerny}
\Clue{8}{}{cecha produktu, który można zjeść}
\Clue{10}{}{rodzaj organizacji społecznej, zazwyczaj posiadającej osobowość prawną, której istotnym substratem są jej członkowie (korporanci)}
\Clue{11}{}{piroksen charakteryzujący się jednoskośną strukturą krystaliczną}
\Clue{12}{}{miejsce przeznaczenia}
\Clue{13}{}{gazotron z katodą torowaną wypełnioną argonem}
\Clue{16}{}{w judaizmie otchłań dla grzeszników}
\Clue{17}{}{Cyclanthus - rodzaj wieloletnich roślin naziemnopączkowych z rodziny okolnicowatych}
\Clue{18}{}{pszczoły bezżądłowe, pszczoły bezżądłe, Meliponini - plemię z rodziny pszczołowatych; należy do niego około 250 gatunków, występujących w krajach tropikalnej i subtropikalnej strefy Afryki, Azji, Australii i Ameryki}
\Clue{19}{}{członek ruchu ewangelickiego na Mazurach, powstałego w ramach protestu przeciwko germanizacji Mazurów}
\Clue{20}{}{krążenie we krwi toksyn bakteryjnych (jak przy błonicy i tężcu), zwierzęcych (przy ukąszeniu żmii) lub roślinnych; u ludzi stanowi główną przyczynę zgonów okołoporodowych matek i noworodków}
\Clue{21}{}{tkanina jutowa lub konopna pokryta warstwą zmiękczonego i zabarwionego polichlorku winylu; wykładzina podłóg, stołów laboratoryjnych itp}
\Clue{22}{}{Adiantum tenerum Sw. - gatunek paproci z rodziny adiantowatych lub orliczkowatych; występuje w Ameryce Środkowej od Florydy i północnego Meksyku na północy, do Wenezueli na południu}
\Clue{23}{}{astronom francuski (1811-77), prace z mechaniki nieba}
\Clue{24}{}{świr, szalona głowa, osoba, o której można powiedzieć, że jest zakręcona (zazwyczaj: na jakimś punkcie)}
\Clue{25}{}{główne drzwi w ikonostasie w cerkwiach, otwierane tylko podczas nabożeństwa}
\Clue{26}{}{ELEWON}
\Clue{27}{}{wypadnięcie maszyny z synchronizmu w elektrotechnice}
\Clue{28}{}{kryptyda, drapieżne zwierzę, które powoduje zniszczenia w przestrzeniach zurbanizowanych}\end{PuzzleClues}

\begin{PuzzleClues}{\textbf{Pionowe}\\}\Clue{1}{}{kanton w środku Szwajcarii, obszar 1,5 tyś. km2, stolica Lucerna}
\Clue{2}{}{cecha człowieka, który jest stanowczy, postępuje stanowczo}
\Clue{3}{}{Salticidae - kosmopolityczna rodzina pająków, które nie tkają sieci, tylko polują skacząc na swoją ofiarę; w skład rodziny, mającej minimum 65 mln lat, wchodzi ok. 5600 tysięcy gatunków przynależących do blisko 600 rodzajów, co czyni je najliczniejszą rodziną wśród pająków z ok. 13 proc. wszystkich gatunków; w Polsce stwierdzono występowanie 59 gatunków}
\Clue{5}{}{Schoenoplectus Palla - rodzaj roślin kłączowych z rodziny ciborowatych}
\Clue{6}{}{w sportach - rękawica do łapania czegoś (np. piłki, krążka itp.)}
\Clue{7}{}{hodowlana rasa angielskiego gołębia lotnego}
\Clue{8}{}{statek specjalnie wyposażony w prowadnice i przeznaczony do przewozu kontenerów, przy założeniu ich pionowego załadunku i wyładunku}
\Clue{9}{}{stopień wojskowy wprowadzony w Polsce w 1992 roku i oznaczany jedną gwiazdką na naramienniku}
\Clue{11}{}{ktoś, kto przerabia coś, zajmuje się przerabianiem czegoś, np. zdjęć, tekstów}
\Clue{12}{}{ironicznie: wypaczona postępowość, w której ludzie zatraceni są w dążeniu do realizacji założonych postępowych idei bez liczenia się ze stratami, jakie może to przynieść}
\Clue{14}{}{jest to cząstka budowy formalnej utworu, składająca się z dwóch lub kilku motywów, tworzących zamkniętą myśl muzyczną i określoną całość wyrazową}
\Clue{15}{}{komisja organizująca zawody sprotów wodnych}
\Clue{20}{}{ford z dawnego modelu T}
\Clue{22}{}{solandra, Acanthocybium solandri - gatunek morskiej ryby z rodziny makrelowatych (Scombridae)}\end{PuzzleClues}\newpage\section*{Krzyżówka 36}

\noindent\begin{Puzzle}{22}{27}|*	|*	|*	|*	|*	|*	|*	|[1][S]\drarr	|f	|o	|t	|o	|a	|u	|t	|o	|t	|r	|o	|f	|*	|*	|*	|.
|*	|*	|*	|[2][S]\rarr	|d	|o	|c	|h	|ó	|d	|[][,]{ }	|p	|a	|s	|y	|w	|n	|y	|*	|*	|*	|*	|*	|.
|*	|*	|[3][S]\darr	|*	|[4][S]\darr	|*	|*	|e	|*	|[5][S]\darr	|*	|[6][S]\drarr	|h	|a	|w	|a	|n	|a	|*	|*	|*	|*	|*	|.
|*	|*	|o	|*	|d	|*	|*	|r	|*	|s	|[7][S]\darr	|w	|*	|[8][S]\darr	|*	|*	|[9][S]\darr	|*	|*	|*	|*	|*	|*	|.
|*	|*	|b	|[10][S]\darr	|a	|*	|*	|e	|*	|o	|z	|a	|*	|ł	|[11][S]\rarr	|z	|b	|i	|t	|e	|ń	|*	|[12][S]\darr	|.
|*	|*	|j	|d	|r	|[13][S]\darr	|*	|t	|[14][S]\drarr	|b	|a	|r	|i	|a	|t	|r	|i	|a	|*	|[15][S]\darr	|*	|*	|i	|.
|*	|*	|e	|w	|[][,]{ }	|a	|*	|y	|p	|ó	|ś	|g	|*	|w	|*	|*	|r	|*	|*	|ś	|*	|*	|g	|.
|*	|[16][S]\drarr	|ż	|u	|b	|r	|[][,]{ }	|k	|a	|r	|p	|a	|c	|k	|i	|*	|ż	|[17][S]\darr	|*	|w	|*	|*	|l	|.
|*	|f	|d	|d	|o	|l	|*	|*	|p	|*	|i	|c	|*	|a	|*	|*	|e	|u	|*	|i	|*	|*	|i	|.
|*	|a	|ż	|z	|ż	|a	|*	|*	|i	|*	|e	|z	|*	|*	|*	|*	|*	|l	|*	|e	|*	|*	|s	|.
|*	|l	|a	|i	|y	|n	|*	|*	|e	|*	|w	|*	|*	|*	|*	|*	|*	|t	|*	|r	|[18][S]\darr	|*	|h	|.
|*	|a	|c	|e	|*	|d	|*	|*	|r	|*	|*	|*	|*	|*	|*	|[19][S]\darr	|*	|r	|*	|k	|k	|*	|m	|.
|*	|*	|z	|s	|*	|a	|[20][S]\drarr	|o	|z	|d	|o	|b	|a	|*	|*	|ż	|[21][S]\darr	|a	|*	|[][,]{ }	|l	|[22][S]\darr	|e	|.
|*	|[23][S]\darr	|*	|t	|*	|*	|a	|*	|y	|*	|*	|*	|[24][S]\darr	|*	|*	|a	|d	|m	|*	|k	|a	|n	|k	|.
|*	|ż	|*	|y	|*	|*	|m	|[25][S]\rarr	|s	|t	|a	|w	|k	|a	|[][,]{ }	|k	|w	|o	|t	|o	|w	|a	|*	|.
|*	|m	|*	|*	|[26][S]\drarr	|p	|i	|p	|k	|o	|w	|*	|l	|*	|[27][S]\darr	|i	|u	|n	|[28][S]\darr	|r	|i	|p	|*	|.
|*	|i	|*	|*	|h	|[29][S]\drarr	|c	|h	|o	|r	|o	|b	|a	|[][,]{ }	|p	|e	|r	|t	|h	|e	|s	|a	|*	|.
|*	|j	|*	|*	|i	|g	|i	|[30][S]\darr	|*	|[31][S]\darr	|*	|*	|s	|*	|i	|r	|o	|a	|a	|a	|z	|s	|*	|.
|*	|o	|*	|*	|p	|a	|*	|k	|*	|p	|*	|[32][S]\darr	|a	|*	|o	|i	|ż	|n	|l	|ń	|o	|t	|*	|.
|*	|w	|*	|*	|s	|l	|*	|a	|*	|r	|*	|ł	|*	|*	|n	|a	|e	|i	|o	|s	|w	|n	|*	|.
|*	|c	|*	|*	|y	|ó	|[33][S]\darr	|l	|*	|e	|*	|a	|*	|*	|i	|*	|k	|n	|g	|k	|i	|i	|*	|.
|*	|o	|*	|*	|b	|w	|s	|i	|*	|s	|*	|t	|*	|*	|e	|*	|*	|*	|e	|i	|e	|c	|*	|.
|*	|w	|[34][S]\rarr	|d	|e	|k	|a	|n	|a	|t	|*	|e	|*	|[35][S]\rarr	|r	|a	|p	|*	|n	|*	|c	|z	|*	|.
|*	|a	|*	|[36][S]\drarr	|m	|a	|l	|a	|w	|i	|j	|k	|a	|*	|*	|[37][S]\rarr	|t	|r	|e	|n	|*	|k	|*	|.
|[38][S]\rarr	|t	|i	|g	|a	|*	|a	|*	|*	|ż	|*	|*	|[39][S]\drarr	|g	|a	|z	|ó	|w	|k	|a	|*	|a	|*	|.
|*	|e	|*	|a	|*	|*	|*	|*	|*	|*	|[40][S]\rarr	|o	|w	|a	|d	|e	|k	|*	|*	|*	|*	|*	|*	|.
|*	|*	|[41][S]\rarr	|r	|e	|z	|o	|n	|a	|n	|s	|*	|u	|*	|*	|*	|*	|*	|*	|*	|*	|*	|*	|.
|*	|*	|*	|*	|*	|*	|*	|*	|*	|*	|*	|*	|*	|*	|*	|*	|*	|*	|*	|*	|*	|*	|*	|.\end{Puzzle}

\newpage

\begin{PuzzleClues}{\textbf{Poziome}\\}\Clue{1}{}{autotroficzny organizm pozyskujący pożywienie na drodze fotosyntezy}
\Clue{2}{}{dochód bez stałego angażowania własnej pracy}
\Clue{6}{}{hawańskie cygaro}
\Clue{11}{}{rozgrzewający napój rosyjski na bazie miodu, znany od XII wieku}
\Clue{14}{}{dziedzina medycyny zajmująca się leczeniem chorobliwej otyłości}
\Clue{16}{}{żubr węgierski, Bison bonasus hungarorum - podgatunek żubra, łożyskowca z rodziny krętorogich, rzędu parzystokopytnych; żył na terenie południowych Karpat i Siedmiogrodu, wymarł około 1790 lub w 1762 roku}
\Clue{20}{}{to, co przynosi dumę swym pięknem, coś, czego prymarną funkcją (powodem istnienia) nie jest estetyka, lecz może być uważane za (rozważane jako) przymiot}
\Clue{25}{}{stawka do zapłaty obliczana od wysokości podatku}
\Clue{26}{}{bułgarski kompozytor i pianista (1904-1974); dyrektor Opery Narodowej}
\Clue{29}{}{choroba, w której dochodzi do jałowej martwicy głowy kości udowej}
\Clue{34}{}{zwarta struktura duszpasterska obejmująca część diecezji, kilka parafii}
\Clue{35}{}{BOLEŃ}
\Clue{36}{}{mieszkanka Malawi, kobieta pochodzenia malawijskiego}
\Clue{37}{}{przedłużenie damskiej sukni ciągnące się z tyłu po ziemi}
\Clue{38}{}{azjatycki ptak z rodziny dzięciołowatych}
\Clue{39}{}{rodzaj leczniczej kąpieli, podczas której chory jest zanużony w wodzie z rozpuszczonym dwutlenkiem węgla; stosowana w leczeniu chorób serca}
\Clue{40}{}{owad}
\Clue{41}{}{reakcja na jakieś zdarzenie}\end{PuzzleClues}

\begin{PuzzleClues}{\textbf{Pionowe}\\}\Clue{1}{}{człowiek, który wyznaje lub głosi herezję, czyli poglądy sprzeczne z dogmatami wiary katolickiej}
\Clue{3}{}{osoba trenująca konie, zabierająca je na przejażdżkę, zapewniająca im niezbędny ruch}
\Clue{4}{}{pokarm, chleb}
\Clue{5}{}{w prawosławiu i katolicyzmie wschodnim świątynia o szczególnym znaczeniu, wyróżniająca się rozmiarami i znaczeniem dla kultu w danym mieście lub regionie}
\Clue{6}{}{czarny niedźwiedź z małym półksiężycem na piersi - Płw. Indyjski, Cejlon}
\Clue{7}{}{charakterystyczny śpiew lub melodia}
\Clue{8}{}{osoby siedzące w/na jednej ławce}
\Clue{9}{}{miasto w płn. części Litwy nad jeziorem Szyrwena}
\Clue{10}{}{dwudziesty dzień (najczęściej bieżącego lub przyszłego) miesiąca}
\Clue{12}{}{migowy język krasnoludów w literackim legendarium, wykreowanym przez J. R. R. Tolkiena}
\Clue{13}{}{międzynarodowy port lotniczy w Sztokholmie}
\Clue{14}{}{dokument}
\Clue{15}{}{Picea koraiensis - gatunek z rodziny sosnowatych}
\Clue{16}{}{wzniesienie, spiętrzenie czegoś powstałe wskutek działania jakichś sił, czynników, zjawisk}
\Clue{17}{}{ultramontanista - zwolennik ultramontaizmu}
\Clue{18}{}{człowiek, który gra na instrumencie klawiszowym}
\Clue{19}{}{powstanie chłopów francuskich w XIV wieku}
\Clue{20}{}{Giovanni, ur. w 1786r. włoski optyk i astronom - botanika mikroskopowa}
\Clue{21}{}{Dichodontium - rodzaj mchów z rodziny widłozębowatych}
\Clue{22}{}{kobieta, która dokonuje napaści, jest agresywna wobec kogoś}
\Clue{23}{}{Chauliodus - rodzaj drapieżnych ryb głębinowych z rodziny wężorowatych (Stomiidae)}
\Clue{24}{}{jedno z podstawowych pojęć matematycznych, zbiór w sensie matematycznym i logicznym}
\Clue{26}{}{Hypsibema - rodzaj roślinożernego dinozaura z nadrodziny hadrozauroidów (Hadrosauroidea); żył w okresie późnej kredy (83-71 mln lat temu) na terenach Ameryki Północnej; długość ciała 9 m, wysokość 4 m, ciężar 2,5 t}
\Clue{27}{}{członek komunistycznej organizacji młodzieżowej}
\Clue{28}{}{ogólna nazwa soli kwasu halogenowodorowego: fluorowodorowego, chlorowodorowego, bromowodorowego lub jodowodorowego}
\Clue{29}{}{potoczne określenie uroczystego występu, z jakiejś szczególnej okazji lub przed wyjątkową publicznością}
\Clue{30}{}{Viburnum - rodzaj roślin zaliczany w systemach APG do rodziny piżmaczkowatych (Adoxaceae), wcześniej wyodrębniany był w monotypową rodzinę kalinowatych (Viburnaceae) lub włączany był do przewiertniowatych (Caprifoliaceae)}
\Clue{31}{}{pochlebna opinia na temat kogoś lub czegoś cieszącego się uznaniem}
\Clue{32}{}{człowiek, który jest biedny (chodzi w połatanych ubraniach), ale nie jest postacią nieszczęśliwą, nie traktuje się go też jednak poważnie}
\Clue{33}{}{miasto w Maroku nad Oceanem Atlantyckim, tworzy z Rabatem prefekturę miejską; port lotniczy}
\Clue{36}{}{garnek pokaźnych rozmiarów}
\Clue{39}{}{jeden z głównych języków chińskich, używany powszechnie w Chinach}\end{PuzzleClues}\newpage\section*{Krzyżówka 37}

\noindent\begin{Puzzle}{23}{19}|*	|*	|*	|[1][S]\drarr	|g	|r	|a	|v	|e	|s	|*	|*	|*	|*	|[2][S]\drarr	|k	|r	|e	|m	|s	|*	|*	|*	|*	|.
|*	|*	|[3][S]\darr	|z	|*	|[4][S]\drarr	|r	|ó	|g	|[][,]{ }	|o	|b	|f	|i	|t	|o	|ś	|c	|i	|*	|*	|*	|*	|*	|.
|*	|*	|r	|m	|*	|d	|[5][S]\rarr	|s	|o	|c	|j	|a	|l	|d	|e	|m	|o	|k	|r	|a	|t	|a	|*	|*	|.
|*	|*	|e	|i	|*	|e	|*	|*	|[6][S]\drarr	|p	|ó	|ł	|s	|i	|o	|s	|t	|r	|a	|*	|*	|*	|*	|*	|.
|*	|*	|l	|a	|*	|s	|*	|[7][S]\darr	|ś	|[8][S]\rarr	|g	|r	|o	|s	|z	|*	|*	|[9][S]\drarr	|l	|o	|d	|i	|*	|*	|.
|*	|*	|i	|n	|[10][S]\darr	|k	|*	|a	|c	|[11][S]\rarr	|f	|i	|z	|j	|o	|g	|n	|o	|m	|i	|a	|*	|[12][S]\darr	|*	|.
|*	|[13][S]\drarr	|n	|a	|w	|a	|ł	|n	|i	|k	|i	|*	|*	|*	|f	|*	|*	|b	|*	|[14][S]\darr	|*	|*	|k	|*	|.
|*	|w	|g	|*	|y	|[][,]{ }	|*	|t	|e	|[15][S]\rarr	|o	|s	|t	|r	|i	|a	|*	|r	|*	|z	|*	|*	|a	|*	|.
|*	|y	|*	|[16][S]\darr	|t	|r	|*	|r	|r	|[17][S]\rarr	|b	|a	|b	|i	|a	|[][,]{ }	|d	|u	|p	|a	|*	|[18][S]\darr	|ł	|*	|.
|*	|s	|[19][S]\drarr	|g	|r	|o	|m	|o	|w	|ł	|a	|d	|c	|a	|*	|*	|*	|s	|*	|j	|*	|s	|m	|*	|.
|*	|p	|b	|n	|a	|z	|[20][S]\drarr	|p	|o	|l	|e	|r	|k	|a	|*	|*	|*	|*	|*	|ą	|*	|c	|u	|*	|.
|*	|a	|o	|i	|w	|d	|s	|o	|*	|*	|*	|[21][S]\rarr	|n	|u	|m	|i	|d	|y	|j	|c	|z	|y	|k	|*	|.
|*	|[][,]{ }	|a	|d	|n	|z	|e	|i	|*	|*	|*	|*	|*	|*	|*	|[22][S]\darr	|[23][S]\rarr	|e	|n	|z	|y	|m	|*	|*	|.
|*	|m	|r	|o	|o	|i	|r	|d	|*	|*	|[24][S]\rarr	|s	|t	|a	|w	|k	|a	|[][,]{ }	|c	|e	|l	|n	|a	|*	|.
|*	|a	|d	|s	|ś	|e	|a	|y	|*	|*	|*	|*	|*	|*	|*	|u	|[25][S]\drarr	|m	|a	|k	|*	|o	|*	|*	|.
|*	|n	|m	|z	|ć	|l	|j	|*	|*	|*	|*	|*	|*	|*	|*	|p	|k	|*	|*	|*	|*	|w	|*	|*	|.
|*	|*	|a	|*	|*	|c	|*	|*	|[26][S]\rarr	|h	|y	|d	|r	|o	|p	|l	|a	|n	|*	|*	|*	|a	|*	|*	|.
|*	|[27][S]\rarr	|n	|a	|c	|z	|y	|n	|i	|e	|[][,]{ }	|k	|u	|c	|h	|e	|n	|n	|e	|*	|*	|t	|*	|*	|.
|*	|*	|*	|[28][S]\rarr	|s	|a	|p	|r	|o	|f	|a	|g	|*	|*	|*	|t	|n	|[29][S]\rarr	|p	|a	|n	|e	|w	|*	|.
|[30][S]\rarr	|t	|r	|a	|m	|*	|*	|*	|*	|*	|*	|*	|*	|*	|*	|*	|*	|*	|*	|*	|*	|*	|*	|*	|.\end{Puzzle}

\newpage

\begin{PuzzleClues}{\textbf{Poziome}\\}\Clue{1}{}{ur. 1895r, angielski poeta i prozaik; „Ja Klaudiusz”, „Córka Homera”}
\Clue{2}{}{miasto w Austrii (Dolna Austria) port u ujścia rzeki Krems do Dunaju}
\Clue{4}{}{źródło nieskończonego bogactwa, dostatku i dobrobytu}
\Clue{5}{}{zwolennik socjaldemokracji lub członek partii socjaldemokratycznej}
\Clue{6}{}{klacz tej samej matki, ale innego ojca}
\Clue{8}{}{jednostka zdawkowa w Austrii; 1/100 szylinga austriackiego}
\Clue{9}{}{miasto we Włoszech (Lombardia) nad Addą; ośr. handlowy, ważny węzeł drogowy}
\Clue{11}{}{czyjeś oblicze, twarz, to, jak ktoś wygląda z twarzy}
\Clue{13}{}{Hydrobatinae - podrodzina ptaków z rodziny nawałnikowatych (Hydrobatidae) w rzędzie rurkonosych (Procellariiformes)}
\Clue{15}{}{bardzo wytrzymałe drewno pozyskiwane z wielu gatunków drzewa o tej samej nazwie, najczęściej jest to chmielograb wirginijski; wykorzystywane do produkcji mebli, szpul, narzędzi, części instrumentów muzycznych}
\Clue{17}{}{Crataegus - ludowa nazwa głogu, której użycie jest charakterystyczne dla południowych rejonów Polski między Krakowem a Kielcami, powiązana być może z kształtem owocu, kojarzonym z pośladkami}
\Clue{19}{}{ten, kto panuje nad piorunami; używane w stosunku do bóstw, przede wszystkim Zeusa-Jowisza}
\Clue{20}{}{przyrząd do wygładzania nierównej powierzchni}
\Clue{21}{}{mieszkaniec Numidii, starożytnego państwa w Afryce}
\Clue{23}{}{wielkocząsteczkowy, białkowy katalizator rekacji chemicznych dotyczących żywych organizmów; powstaje w każdym żywym organizmie, reguluje przebieg zachodzących w nim procesów}
\Clue{24}{}{stawka opłaty celnej}
\Clue{25}{}{jednoroczna roślina zielna, rzadziej bylina o owocu w postaci wielonasiennej torebki zwanej makówką}
\Clue{26}{}{WODNOSAMOLOT, WODNOPŁAT; samolot o specjalnej budowie mogący startować i lądować na wodzie}
\Clue{27}{}{pojemnik używany do przechowywania, przygotowywania potraw w kuchni, a także do ich spożywania.}
\Clue{28}{}{organizmy odżywiające się szczątkami organicznymi - żyją w glebie, ściółce leśnej itp.  np. dżdżownica}
\Clue{29}{}{płytkie metalowe naczynie do gotowania itp.; obecnie ma zastosowanie w warzelniach soli}
\Clue{30}{}{dolna belka wiązania dachowego}\end{PuzzleClues}

\begin{PuzzleClues}{\textbf{Pionowe}\\}\Clue{1}{}{widomy, powierzchniowy objaw choroby}
\Clue{2}{}{ruch religijny z XIX wieku inspirowany koncepcjami indyjskimi i wskazujący za cel osiągnięcie w procesie reinkarnacji tożsamości z bóstwem}
\Clue{3}{}{rodzaj półki, która podwieszana jest nad kuchnią lub stołemi; przeznaczona na kuchenne drobiazgi}
\Clue{4}{}{element wyposażenia samochodu umieszczony w przedniej części kabiny, poniżej przedniej szyby, zawierający tablicę rozdzielczą, regulatory i przełączniki}
\Clue{6}{}{martwe zwierzę}
\Clue{7}{}{małpy właściwe, Simiiformes, Anthropoidea - infrarząd ssaków naczelnych obejmujący małpy szerokonose i małpy wąskonose; małpy szerokonose występują w obydwu Amerykach i nazywane są małpami Nowego Świata, natomiast współcześnie żyjące małpy wąskonose znane są z Azji i Afryki, dawniej występowały również w Europie (obecnie tylko magot żyje na Gibraltarze) i nazywane są małpami Starego Świata}
\Clue{9}{}{tkanina lub dzianina używana do przykrycia stołu, zwykle na czas posiłków, pełniąca funkcję dekoracyjną i ochronną}
\Clue{10}{}{znajomość rzeczy, wykwintność cechująca znawcę}
\Clue{12}{}{przedstawiciel liczącego ok. 250 tysięcy osób narodu mongolskiego z grupy ojrackiej}
\Clue{13}{}{dependencja korony brytyjskiej znajdująca się na wyspie Man}
\Clue{14}{}{świecki symbol świąt Wielkiejnocy}
\Clue{16}{}{półpasożytnicza roślina zielna występująca w Polsce na łąkach i torfowiskach}
\Clue{18}{}{Dalatiidae - rodzina ryb chrzęstnoszkieletowych z rzędu koleniokształtnych (Squaliformes); scymnowate zamieszkują zimne i umiarkowane wody oceaniczne półkuli północnej i południowej, od Arktyki po Antarktydę}
\Clue{19}{}{angielski kolarz zawodowy, brązowy medalista olimpijski z Atlanty w jeździe indywidualnej na czas}
\Clue{20}{}{pałac muzułmański wzniesiony na centralnym planie z wewnętrznym dziedzińcem i dekoracyjnym portalem}
\Clue{22}{}{odcinek ronda oddzielający dwa sąsiadujące ze sobą tematy}
\Clue{25}{}{ur. w 1916r pisarka, książki dla dzieci i młodzieży o życiu górali; „Dujawica”, „Wantule”}\end{PuzzleClues}\newpage\section*{Krzyżówka 38}

\noindent\begin{Puzzle}{23}{31}|*	|*	|[1][S]\drarr	|z	|a	|s	|i	|e	|d	|l	|e	|n	|i	|e	|*	|[2][S]\drarr	|f	|*	|[3][S]\drarr	|h	|y	|i	|a	|*	|.
|*	|[4][S]\rarr	|l	|a	|m	|p	|a	|[][,]{ }	|o	|b	|r	|a	|z	|o	|w	|a	|*	|*	|p	|*	|[5][S]\darr	|[6][S]\darr	|*	|*	|.
|*	|[7][S]\rarr	|e	|k	|s	|p	|r	|e	|s	|*	|*	|*	|*	|*	|*	|n	|*	|*	|r	|*	|s	|s	|*	|*	|.
|*	|*	|p	|*	|[8][S]\darr	|[9][S]\rarr	|m	|a	|k	|a	|r	|o	|n	|i	|s	|t	|a	|*	|z	|*	|t	|t	|[10][S]\darr	|*	|.
|*	|*	|t	|*	|c	|*	|*	|*	|*	|*	|[11][S]\drarr	|k	|o	|s	|i	|a	|r	|z	|e	|*	|a	|a	|s	|*	|.
|*	|*	|o	|[12][S]\rarr	|h	|o	|f	|f	|*	|*	|e	|*	|*	|[13][S]\rarr	|t	|r	|z	|o	|s	|*	|l	|r	|a	|*	|.
|*	|[14][S]\drarr	|p	|a	|r	|y	|ż	|a	|n	|i	|n	|*	|*	|*	|[15][S]\darr	|e	|*	|*	|ł	|*	|a	|a	|t	|*	|.
|*	|k	|e	|*	|y	|[16][S]\rarr	|m	|i	|j	|a	|n	|k	|a	|*	|g	|s	|[17][S]\drarr	|s	|o	|n	|g	|*	|k	|*	|.
|*	|o	|l	|[18][S]\darr	|z	|*	|[19][S]\drarr	|d	|z	|i	|e	|w	|o	|j	|a	|*	|d	|*	|n	|*	|*	|[20][S]\darr	|a	|*	|.
|*	|t	|i	|p	|o	|[21][S]\drarr	|c	|h	|o	|r	|o	|b	|a	|[][,]{ }	|b	|r	|u	|g	|a	|d	|ó	|w	|*	|*	|.
|*	|l	|s	|r	|t	|w	|i	|*	|*	|*	|d	|*	|*	|*	|i	|*	|c	|*	|*	|*	|*	|s	|*	|*	|.
|*	|i	|[][,]{ }	|o	|y	|o	|ś	|*	|[22][S]\darr	|*	|a	|*	|[23][S]\darr	|[24][S]\darr	|n	|*	|h	|[25][S]\darr	|[26][S]\darr	|*	|*	|z	|*	|*	|.
|*	|n	|m	|e	|l	|l	|n	|*	|d	|*	|*	|[27][S]\rarr	|g	|n	|e	|t	|o	|w	|e	|*	|*	|e	|*	|*	|.
|*	|a	|o	|k	|*	|a	|i	|[28][S]\darr	|z	|*	|*	|[29][S]\darr	|r	|i	|t	|*	|w	|a	|l	|*	|*	|c	|*	|*	|.
|*	|[][,]{ }	|z	|s	|*	|*	|e	|s	|i	|[30][S]\darr	|*	|s	|y	|u	|[][,]{ }	|*	|y	|l	|e	|*	|*	|h	|*	|*	|.
|[31][S]\drarr	|k	|a	|p	|u	|ś	|n	|i	|a	|k	|*	|i	|z	|ń	|s	|[32][S]\darr	|[][,]{ }	|u	|k	|*	|*	|p	|*	|*	|.
|o	|ł	|i	|o	|*	|[33][S]\darr	|i	|e	|ł	|w	|[34][S]\darr	|ł	|a	|k	|t	|f	|o	|t	|t	|*	|*	|o	|*	|*	|.
|r	|o	|k	|r	|*	|ł	|e	|w	|o	|a	|s	|o	|c	|a	|o	|u	|j	|o	|r	|*	|*	|l	|*	|*	|.
|d	|d	|o	|t	|*	|o	|[][,]{ }	|k	|*	|s	|e	|w	|z	|*	|m	|n	|c	|w	|o	|*	|*	|a	|[35][S]\darr	|*	|.
|y	|z	|w	|o	|*	|ż	|p	|o	|*	|*	|g	|n	|e	|*	|a	|k	|i	|o	|e	|*	|*	|k	|ż	|*	|.
|n	|k	|y	|w	|[36][S]\drarr	|n	|a	|w	|a	|ł	|n	|i	|k	|*	|t	|c	|e	|ś	|n	|*	|*	|*	|ó	|*	|.
|a	|a	|*	|o	|p	|i	|r	|a	|*	|[37][S]\darr	|e	|a	|*	|*	|o	|j	|c	|ć	|e	|*	|[38][S]\darr	|*	|ł	|*	|.
|c	|*	|*	|ś	|e	|k	|c	|t	|*	|n	|r	|*	|*	|*	|l	|a	|*	|*	|r	|[39][S]\darr	|b	|[40][S]\darr	|w	|*	|.
|j	|[41][S]\darr	|*	|ć	|r	|*	|j	|e	|*	|a	|*	|*	|*	|*	|o	|[][,]{ }	|*	|*	|g	|w	|a	|r	|i	|*	|.
|a	|m	|*	|*	|m	|*	|a	|*	|*	|w	|*	|*	|*	|*	|g	|n	|[42][S]\rarr	|w	|e	|y	|d	|e	|n	|*	|.
|*	|e	|*	|*	|*	|[43][S]\rarr	|l	|o	|t	|a	|r	|y	|n	|g	|i	|a	|*	|*	|t	|z	|y	|k	|e	|*	|.
|[44][S]\drarr	|t	|i	|s	|c	|h	|n	|e	|r	|*	|*	|*	|*	|*	|c	|z	|*	|*	|y	|n	|l	|u	|k	|*	|.
|t	|r	|[45][S]\rarr	|z	|m	|i	|e	|r	|z	|a	|n	|i	|e	|*	|z	|w	|*	|*	|k	|a	|a	|r	|*	|*	|.
|o	|u	|[46][S]\rarr	|f	|u	|n	|*	|*	|[47][S]\rarr	|s	|p	|r	|a	|w	|n	|o	|ś	|ć	|*	|n	|r	|s	|*	|*	|.
|p	|m	|*	|*	|*	|*	|*	|*	|*	|*	|*	|*	|*	|*	|y	|w	|*	|*	|*	|i	|k	|*	|*	|*	|.
|i	|*	|*	|*	|*	|*	|*	|*	|*	|*	|*	|*	|*	|*	|*	|a	|*	|*	|*	|e	|a	|*	|*	|*	|.
|*	|*	|*	|*	|*	|*	|*	|*	|*	|*	|*	|*	|*	|*	|*	|*	|*	|*	|*	|*	|*	|*	|*	|*	|.\end{Puzzle}

\newpage

\begin{PuzzleClues}{\textbf{Poziome}\\}\Clue{1}{}{Zamieszkanie grupy ludzi na określonym terytornium, na którym wcześniej nie mieszkała}
\Clue{2}{}{dźwięk muzyczny, którego częstotliwość w oktawie razkreślnej wynosi 349,6 Hz.}
\Clue{3}{}{nowozelandzki ptak z rzędu wróblowatych}
\Clue{4}{}{lampa elektronowa wyposażona w ekran, na którym możliwe jest wyświetlenie obrazu za pomocą wiązki elektronów}
\Clue{7}{}{rodzaj broni palnej; sztucer podwójny o poziomym lub pionowym (bok-ekspres) ułożeniu luf gwintowanych}
\Clue{9}{}{człowiek, który używa makaronizmów}
\Clue{11}{}{nazwa 3gwiazd tworzących tzw. pas w gwiazdozbiorze Oriona}
\Clue{12}{}{etnograf (1829-94); współpracownik O. Kolberga}
\Clue{13}{}{woreczek na pieniądze}
\Clue{14}{}{mieszkaniec Paryża}
\Clue{16}{}{fragment trasy, miejsce specjalnie dostosowane, by mogły się tam mijać pojazdy jadące w przeciwnych kierunkach}
\Clue{17}{}{motyw jazzowy}
\Clue{19}{}{dorodna dziewczyna}
\Clue{21}{}{uwarunkowana genetycznie choroba o typie dziedziczenia autosomalnym dominującym, charakteryzująca się skłonnością do występowania napadowych zaburzeń rytmu o typie częstoskurczu komorowego, który może ustąpić samoistnie lub przekształcić się w migotanie komór i doprowadzić do nagłego zatrzymania krążenia i śmierci}
\Clue{27}{}{gniotowce, gniotowe, Gnetales - grupa roślin o różnej randze systematycznej w zależności od klasyfikacji i ujęcia systematycznego; posiadają pewne cechy przypominające rośliny okrytonasienne (Magnoliopsida), przez co niegdyś doszukiwano się wspólnego pochodzenia tych dwóch grup roślin}
\Clue{31}{}{zgrubiale: kapuśniaczek - ciepły deszcz}
\Clue{36}{}{ptak morski; poszczególne gatunki tego ptaka klasyfikowane są w taksonomii biologicznej w obrębie podrodziny nawałników (Hydrobatinae)}
\Clue{42}{}{malarz niderlandzki (1400-64) sceny religijne, portrety}
\Clue{43}{}{kraina historyczna i region gospodarczy w płn-wsch. Francji, główne miasta: Nancy, Metz}
\Clue{44}{}{dzieła, prace, dorobek Tischnera}
\Clue{45}{}{to, że ktoś do czegoś dąży, zmierza w jakimś kierunku}
\Clue{46}{}{dawna jednostka zdawkowa w Japonii; 10 rinów, 1/10 sena, 1/100 jena}
\Clue{47}{}{dobry stan, fakt, że jakieś urządzenie działa, jest na chodzie}\end{PuzzleClues}

\begin{PuzzleClues}{\textbf{Pionowe}\\}\Clue{1}{}{Leptopelis vermiculatus - gatunek płaza bezogonowego z rodziny artroleptowatych, spotykany w leśnych terenach Tanzanii}
\Clue{2}{}{najjaśniejsza gwiazda w gwiazdozbiorze Skorpiona}
\Clue{3}{}{w partii szachów: bierka}
\Clue{5}{}{w czasie II wojny światowej niemiecki obóz jeniecki dla szeregowców i podoficerów}
\Clue{6}{}{kobieta, która gdzieś pełni funkcję jakiegoś rodzaju nadzorcy, np. szefowej, strażniczki}
\Clue{8}{}{minerał z grupy krzemianów, zaliczany do grupy serpentynów (włóknistych)}
\Clue{10}{}{telewizja satelitarna, programy telewizyjne odbierane z satelity telewizyjnego za pomocą anteny satelitarnej}
\Clue{11}{}{siedmiosiatkowa lampa elektronowa}
\Clue{14}{}{kraina geograficzna w Sudetach Środkowych, której głównym miastem jest Kłodzko}
\Clue{15}{}{pomieszczenie będące miejscem pracy lekarza dentysty, wyposażone w narzędzia stomatologiczne}
\Clue{17}{}{ktoś, kto zainicjował istnienie jakiejś idei lub rzeczy}
\Clue{18}{}{przychylność wobec eksportu}
\Clue{19}{}{ciśnienie, jakie wywierałby dany składnik mieszaniny gazów, gdyby w tej samej temperaturze sam zajmował objętość całej mieszaniny}
\Clue{20}{}{członek, działacz Młodzieży Wszechpolskiej}
\Clue{21}{}{uczucie przejawiające się ochotą na coś, chceniem czegoś}
\Clue{22}{}{ARMATA}
\Clue{23}{}{gryzak dla dziecka}
\Clue{24}{}{pieszczotliwie o malutkim dziecku, zwykle dziewczynce}
\Clue{25}{}{cecha czegoś, co jest walutowe, np. transakcji}
\Clue{26}{}{pracownik branży elektroenergetycznej}
\Clue{28}{}{siewkowe, mewy-siewki, Charadriiformes - rząd ptaków z podgromady  Neornithes}
\Clue{29}{}{sala zaopatrzona w specjalistyczny sprzęt sportowy do treningu mięśni}
\Clue{30}{}{sytuacja, kiedy dobra atmosfera pryska i robi się drętwo}
\Clue{31}{}{podnoszenie osób świeckich do godności duchownych}
\Clue{32}{}{funkcja, która w wyniku podstawiania stałych nazwowych zmienia się w nazwę}
\Clue{33}{}{dawny urzędnik dworski, do którego należała opieka nad sypialniami króla lub możnowładcy}
\Clue{34}{}{niemiecki lekarz i fizyk (1704-77); zbudował prostą turbinę wodną}
\Clue{35}{}{pluskwiak różnoskrzydły częsty w zbożach, szkodnik}
\Clue{36}{}{najwyższy system paleozoiku}
\Clue{37}{}{część kościoła między prezbiterium a kruchtą}
\Clue{38}{}{gryzoń z rodziny myszowatych, pola i łąki Eurazji: szkodnik}
\Clue{39}{}{w znaczeniu religii}
\Clue{40}{}{apelacja - odwołanie się od wydanego wyroku sądowego}
\Clue{41}{}{podstawowy schemat określający wartość trwania nut oraz układ akcentów w obrębie taktu}
\Clue{44}{}{sassebi, Damaliscus lunatus - ssak z rodziny krętorogich; występuje w południowej i środkowej Afryce, zasiedlając sawanny, tereny zalewowe i zarośla}\end{PuzzleClues}\newpage\section*{Krzyżówka 39}

\noindent\begin{Puzzle}{20}{31}|*	|*	|*	|*	|[1][S]\drarr	|s	|t	|a	|r	|o	|o	|r	|m	|i	|a	|ń	|s	|k	|i	|*	|*	|.
|[2][S]\rarr	|b	|e	|r	|ż	|e	|r	|e	|t	|k	|a	|*	|*	|*	|*	|*	|[3][S]\darr	|*	|*	|[4][S]\darr	|*	|.
|*	|*	|*	|*	|y	|[5][S]\darr	|[6][S]\rarr	|s	|z	|c	|z	|e	|r	|b	|a	|k	|i	|*	|[7][S]\darr	|f	|[8][S]\darr	|.
|*	|*	|*	|*	|ł	|k	|*	|[9][S]\rarr	|t	|o	|k	|o	|f	|e	|r	|o	|l	|*	|c	|l	|w	|.
|*	|*	|*	|[10][S]\rarr	|a	|r	|p	|e	|g	|g	|i	|o	|n	|e	|*	|*	|u	|*	|a	|e	|y	|.
|*	|*	|[11][S]\darr	|*	|[][,]{ }	|e	|*	|*	|*	|[12][S]\rarr	|w	|i	|ś	|n	|i	|e	|w	|*	|r	|c	|k	|.
|*	|*	|p	|[13][S]\drarr	|s	|t	|a	|s	|e	|k	|*	|*	|*	|*	|*	|*	|i	|*	|t	|i	|a	|.
|[14][S]\rarr	|t	|u	|s	|z	|o	|n	|k	|a	|*	|*	|*	|*	|*	|[15][S]\drarr	|ż	|u	|l	|i	|k	|*	|.
|*	|[16][S]\darr	|s	|z	|y	|w	|*	|[17][S]\darr	|*	|*	|*	|*	|*	|*	|p	|*	|m	|*	|e	|[][,]{ }	|*	|.
|*	|t	|t	|p	|j	|a	|[18][S]\drarr	|s	|c	|h	|[][S]ö	|n	|b	|e	|r	|g	|*	|*	|r	|p	|*	|.
|*	|w	|y	|a	|n	|t	|j	|ł	|*	|*	|*	|[19][S]\darr	|[20][S]\drarr	|p	|o	|e	|t	|a	|*	|o	|*	|.
|*	|a	|n	|n	|a	|e	|e	|o	|*	|*	|*	|q	|r	|*	|t	|[21][S]\darr	|*	|*	|[22][S]\darr	|l	|*	|.
|*	|r	|n	|*	|[][,]{ }	|*	|l	|n	|*	|*	|*	|u	|e	|*	|r	|w	|*	|*	|l	|s	|*	|.
|*	|d	|i	|[23][S]\drarr	|w	|i	|c	|e	|d	|y	|r	|e	|k	|t	|o	|r	|*	|[24][S]\darr	|i	|k	|*	|.
|*	|e	|c	|z	|e	|*	|z	|c	|*	|*	|*	|g	|*	|*	|m	|ó	|[25][S]\darr	|l	|g	|i	|*	|.
|*	|[][,]{ }	|a	|a	|w	|*	|*	|z	|*	|*	|*	|*	|[26][S]\darr	|*	|b	|b	|e	|a	|o	|*	|*	|.
|*	|l	|[][,]{ }	|p	|n	|[27][S]\drarr	|g	|n	|i	|e	|w	|*	|o	|[28][S]\darr	|i	|l	|n	|n	|ń	|[29][S]\darr	|*	|.
|*	|ą	|k	|a	|ę	|z	|*	|i	|*	|*	|*	|[30][S]\drarr	|t	|a	|n	|i	|e	|c	|*	|u	|*	|.
|*	|d	|a	|c	|t	|a	|*	|c	|*	|*	|*	|a	|o	|k	|a	|c	|r	|e	|*	|j	|*	|.
|*	|o	|t	|h	|r	|g	|[31][S]\drarr	|a	|d	|i	|u	|n	|k	|t	|*	|z	|g	|t	|*	|ś	|*	|.
|*	|w	|o	|*	|z	|a	|a	|*	|*	|*	|*	|t	|*	|*	|*	|k	|i	|o	|*	|c	|*	|.
|*	|a	|l	|*	|n	|d	|p	|*	|[32][S]\rarr	|z	|d	|r	|ó	|j	|*	|a	|a	|g	|*	|i	|*	|.
|[33][S]\drarr	|n	|i	|e	|a	|k	|t	|u	|a	|l	|n	|o	|ś	|ć	|*	|*	|*	|ł	|*	|e	|*	|.
|r	|i	|c	|*	|*	|a	|*	|*	|*	|*	|[34][S]\rarr	|p	|r	|z	|e	|c	|h	|ó	|w	|*	|*	|.
|f	|e	|k	|*	|*	|*	|[35][S]\rarr	|ł	|a	|t	|w	|o	|ś	|ć	|*	|*	|*	|w	|*	|*	|*	|.
|*	|*	|a	|*	|[36][S]\rarr	|s	|z	|p	|i	|c	|e	|l	|*	|*	|*	|*	|*	|[][,]{ }	|*	|*	|*	|.
|*	|*	|*	|*	|*	|[37][S]\rarr	|d	|z	|i	|a	|ł	|o	|s	|z	|y	|n	|*	|s	|*	|*	|*	|.
|[38][S]\rarr	|w	|y	|r	|o	|b	|i	|s	|k	|o	|[][,]{ }	|g	|ó	|r	|n	|i	|c	|z	|e	|*	|*	|.
|*	|*	|*	|[39][S]\rarr	|w	|y	|m	|i	|e	|n	|n	|i	|k	|o	|w	|n	|i	|a	|*	|*	|*	|.
|[40][S]\rarr	|i	|c	|h	|t	|i	|o	|s	|t	|e	|g	|a	|*	|*	|[41][S]\rarr	|b	|e	|r	|i	|o	|*	|.
|*	|*	|*	|[42][S]\rarr	|h	|a	|s	|z	|y	|s	|z	|*	|*	|*	|*	|*	|*	|y	|*	|*	|*	|.
|*	|*	|*	|*	|*	|*	|*	|*	|*	|[43][S]\rarr	|r	|o	|z	|b	|r	|a	|t	|*	|*	|*	|*	|.\end{Puzzle}

\newpage

\begin{PuzzleClues}{\textbf{Poziome}\\}\Clue{1}{}{język dawnych Ormian, stara forma języka ormiańskiego}
\Clue{2}{}{skoczny taniec, popularny w XVI w., wywodzący się z Francji}
\Clue{6}{}{Xenarthra - grupa lądowych ssaków łożyskowych o zredukowanym uzębieniu, do których zaliczane są mrówkojady, leniwce i pancerniki - zwierzęta różniące się wyglądem zewnętrznym i trybem życia, ale o podobnej budowie anatomicznej; do czasów współczesnych przetrwało 30 gatunków występujących obecnie w obydwu Amerykach}
\Clue{9}{}{organiczny związek chemiczny należący do grupy witamin E}
\Clue{10}{}{sześciostrunowa gitara basowa}
\Clue{12}{}{wieś w Polsce położona w województwie mazowieckim, w powiecie siedleckim, w gminie Wiśniew}
\Clue{13}{}{Zeman (1843-1931), pisarz czeski, powieści i opowiadania z życia chłopów i proletariatu}
\Clue{14}{}{porcja tuszonki, konserwy mięsnej; określona ilość tego produktu, zazwyczaj puszka, konserwa}
\Clue{15}{}{w Łodzi i okolicach: chlebek turecki (rodzaj bułki w kształcie tzw. batona, czasem w tym regionie z dodatkiem kawy zbożowej zamiast melasy)}
\Clue{18}{}{kompozytor austriacki (1874-1951); główny przedstawiciel wiedeńskiej szkoły dodekafonicznej; utwory orkiestrowe, kameralne, fortepianowe, opera; 'Mojżesz i Aaron'}
\Clue{20}{}{twórca poezji, wierszy}
\Clue{23}{}{zastępca dyrektora - szefa, kierownika jakiejś instytucji}
\Clue{27}{}{reakcja emocjonalna na niepowodzenie lub wzburzenie, czasami z nastawieniem agresywnym, będącym reakcją na działanie interpretowane jako skierowana negatywnie akcja zaczepna}
\Clue{30}{}{utwór muzyczny, przy którym można tańczyć}
\Clue{31}{}{tytuł zawodowy nadawany pracownikom służby bibliotecznej, leśnictwa i muzealnictwa}
\Clue{32}{}{zdrojowisko - miejscowość uzdrowiskowa, kurort, w którym występują źródła wód leczniczych}
\Clue{33}{}{to, że coś jest nieaktualne, niezgodne z teraźniejszym stanem faktycznym}
\Clue{34}{}{to, że ktoś jest przechowywany, przechowanie kogoś lub czegoś}
\Clue{35}{}{cecha czegoś łatwego, czegoś, co nie sprawia trudności}
\Clue{36}{}{pogardliwie o osobie, która szpieguje i donosi w celu uzyskania jakichś korzyści}
\Clue{37}{}{wieś w Polsce położona w województwie dolnośląskim, w powiecie zgorzeleckim, w gminie Bogatynia}
\Clue{38}{}{przestrzeń w nieruchomości gruntowej lub górotworze powstała w wyniku robót górniczych}
\Clue{39}{}{pomieszczenie lub osobny budynek, w którym znajdują się urządzenia kontrolne systemów grzewczych; wymiennikownia ciepła}
\Clue{40}{}{rybopłaz, Ichthyostega - jeden z pierwszych przedstawicieli czworonogów, forma przejściowa między rybami a płazami, zaliczany do wymarłego rzędu Ichthyostegalia}
\Clue{41}{}{ur. 1925 r., kompozytor włoski, reprezentant awangardy}
\Clue{42}{}{KONOPIE}
\Clue{43}{}{rozstanie, koniec znajomości z kimś, porzucenie kogoś}\end{PuzzleClues}

\begin{PuzzleClues}{\textbf{Pionowe}\\}\Clue{1}{}{duże parzyste naczynie na szyi, odprowadzające krew z twarzy, jamy czaszki i szyi}
\Clue{3}{}{warstwa gleby, która znajduje się pod eluwium i zawiera liczne związki mineralne oraz osady próchnicze}
\Clue{4}{}{jeden z najprostszych instrumentów dętych, niewielka piszczałka z ustnikiem dzióbkowym i sześcioma wierzchnimi otworami palcowymi}
\Clue{5}{}{Talpidae - rodzina ssaków z rzędu Soricomorpha przystosowanych do życia pod ziemią; występują na półkuli północnej - w Europie, Azji i Ameryce Północnej, a jedynym przedstawicielem rodziny występującym współcześnie na terenie Polski jest kret europejski (Talpa europaea)}
\Clue{7}{}{żeglarz francuski, odkrywca Kanady (1491-1557); w poszukiwaniu drogi do Azji dotarł do Zatoki Św. Wawrzyńca}
\Clue{8}{}{BÓB, BOBIK - roślina zielna z bobowatych}
\Clue{11}{}{żaba katolicka, Notaden bennettii - gatunek australijskiego płaza bezogonowego z rodziny Limnodynastidae, zaliczany wcześniej do żółwinkowatych, a pierwotnie do ropuchowatych, którego młode osobniki cechują się charakterystycznym umaszczeniem z ciemnym krzyżem na jaskrawożółtym tle na grzbietowej stronie tułowia}
\Clue{13}{}{imponowanie komuś, budzenie w kimś zazdrości}
\Clue{15}{}{II czynnik krzepnięcia, glikoproteina znajdująca się w osoczu krwi, wytwarzana w wątrobie przy udziale witaminy K}
\Clue{16}{}{koniec jakieś sprawy niekorzystny dla kogoś, przynoszący przykre konsekwencje}
\Clue{17}{}{przedstawiciel pierwotniaków wodnych}
\Clue{18}{}{ciężarówka marki Jelcz}
\Clue{19}{}{królestwo z uniwersum cyklu powieści fantasyRiftwar Raymonda E. Feista}
\Clue{20}{}{przyrząd gimnastyczny}
\Clue{21}{}{samica wróbla}
\Clue{22}{}{(1879-1954), pisarz i działacz kulturalny związany ze Śląskiem; „Bery i bójki śląskie”}
\Clue{23}{}{jedna z liczb kwantowych przypisywanych kwarkom i leptonom; można go intuicyjnie utożsamić z rodzajem cząstki i z jej generacją}
\Clue{24}{}{Lampropeltis alterna - gatunek węża z rodziny połozowatych, występujący na południu USA i na północy Meksyku}
\Clue{25}{}{cecha kogoś żywotnego, energicznego, aktywnie działającego}
\Clue{26}{}{część czapki (zwłaszcza wojskowej)}
\Clue{27}{}{niejasna, tajemnicza sprawa}
\Clue{28}{}{współżycie seksualne dwojga (lub większej ilości) osób}
\Clue{29}{}{miejsce, w którym rzeka lub inny ciek kończy swój bieg, łącząc się z inną rzeką lub wpadając do jeziora, morza, oceanu}
\Clue{30}{}{zbiór dyscyplin, składających się na naukę o człowieku i jego kulturze; antropologia jest nauką interdyscyplinarną, wykorzystującą zwłaszcza (ale nie tylko) metody i narzędzia nauk humanistycznych i społecznych, a jej poszczególne dyscypliny badają zarówno zmienność cech anatomicznych i fizjologicznych człowieka w czasie i przestrzeni (antropologia fizyczna), jak i jego społeczno-kulturowe funkcjonowanie (antropologia kulturowa, etnologia)}
\Clue{31}{}{astronauta amerykański w wyprawie Atlantisa w 1991 r}
\Clue{33}{}{symbol oznaczający rutherford}\end{PuzzleClues}\newpage\section*{Krzyżówka 40}

\noindent\begin{Puzzle}{25}{26}|*	|*	|*	|*	|*	|*	|*	|*	|*	|*	|*	|*	|[1][S]\drarr	|ł	|ó	|d	|ź	|*	|*	|*	|*	|*	|*	|*	|*	|*	|.
|*	|*	|*	|*	|*	|*	|*	|*	|*	|*	|*	|[2][S]\rarr	|d	|a	|g	|u	|s	|a	|*	|*	|[3][S]\darr	|[4][S]\darr	|*	|*	|*	|*	|.
|*	|*	|*	|*	|*	|*	|*	|*	|*	|*	|[5][S]\rarr	|k	|r	|u	|k	|*	|*	|*	|*	|*	|b	|z	|*	|*	|*	|*	|.
|*	|*	|*	|*	|*	|*	|*	|*	|*	|*	|*	|*	|a	|*	|[6][S]\darr	|*	|*	|*	|*	|*	|u	|n	|*	|*	|*	|*	|.
|*	|*	|*	|*	|*	|*	|*	|*	|[7][S]\darr	|*	|*	|*	|g	|[8][S]\darr	|m	|*	|*	|*	|*	|*	|r	|a	|*	|*	|*	|*	|.
|*	|*	|*	|*	|*	|*	|[9][S]\drarr	|p	|o	|s	|t	|k	|o	|l	|o	|n	|i	|a	|l	|i	|z	|m	|*	|*	|*	|*	|.
|*	|*	|*	|*	|*	|*	|k	|[10][S]\rarr	|k	|r	|a	|*	|n	|i	|d	|*	|*	|*	|*	|*	|a	|i	|*	|*	|*	|*	|.
|[11][S]\drarr	|o	|s	|p	|o	|w	|a	|t	|o	|ś	|ć	|*	|*	|g	|e	|*	|*	|*	|*	|*	|n	|e	|*	|*	|*	|*	|.
|b	|*	|*	|*	|*	|*	|r	|*	|l	|*	|*	|*	|*	|a	|l	|*	|*	|*	|*	|*	|*	|n	|*	|*	|*	|*	|.
|i	|*	|*	|*	|*	|*	|m	|*	|i	|*	|*	|*	|*	|t	|a	|*	|*	|*	|*	|*	|*	|n	|*	|*	|*	|*	|.
|h	|*	|*	|*	|*	|*	|i	|*	|c	|*	|*	|*	|*	|u	|r	|*	|*	|*	|*	|*	|*	|o	|*	|*	|*	|*	|.
|a	|*	|*	|*	|*	|*	|d	|*	|z	|*	|*	|*	|*	|r	|s	|*	|*	|*	|*	|*	|*	|ś	|*	|*	|*	|*	|.
|ć	|*	|*	|*	|*	|*	|e	|*	|n	|*	|*	|*	|*	|a	|t	|*	|*	|*	|*	|*	|*	|ć	|*	|*	|*	|*	|.
|*	|*	|*	|*	|*	|*	|ł	|*	|o	|*	|*	|*	|*	|*	|w	|*	|*	|*	|*	|*	|*	|[][,]{ }	|*	|*	|*	|*	|.
|*	|*	|*	|*	|*	|*	|k	|*	|ś	|*	|*	|*	|*	|*	|o	|*	|*	|*	|*	|*	|*	|s	|*	|*	|*	|*	|.
|*	|*	|*	|*	|*	|*	|o	|*	|ć	|*	|*	|*	|*	|*	|[][,]{ }	|*	|*	|*	|*	|*	|*	|t	|*	|*	|*	|*	|.
|*	|*	|*	|*	|*	|*	|*	|*	|[][,]{ }	|*	|*	|*	|*	|*	|l	|*	|*	|*	|*	|*	|*	|a	|*	|*	|*	|*	|.
|*	|*	|*	|*	|*	|*	|*	|[12][S]\rarr	|f	|a	|r	|b	|a	|[][,]{ }	|o	|l	|e	|j	|n	|a	|*	|t	|*	|*	|*	|*	|.
|*	|*	|*	|*	|[13][S]\rarr	|n	|a	|d	|a	|j	|n	|i	|k	|[][,]{ }	|t	|e	|l	|e	|w	|i	|z	|y	|j	|n	|y	|*	|.
|*	|*	|*	|*	|*	|*	|*	|*	|k	|*	|*	|*	|*	|*	|n	|*	|*	|*	|*	|*	|*	|s	|*	|*	|*	|*	|.
|*	|*	|*	|*	|*	|*	|*	|*	|t	|*	|*	|*	|*	|*	|i	|*	|*	|*	|*	|*	|*	|t	|*	|*	|*	|*	|.
|*	|*	|*	|*	|*	|*	|*	|*	|y	|*	|*	|*	|*	|*	|c	|*	|*	|*	|*	|*	|*	|y	|*	|*	|*	|*	|.
|*	|*	|*	|*	|*	|*	|*	|*	|c	|*	|*	|*	|*	|*	|z	|*	|*	|*	|*	|*	|*	|c	|*	|*	|*	|*	|.
|*	|*	|*	|*	|*	|*	|*	|*	|z	|*	|*	|*	|*	|*	|e	|*	|*	|*	|*	|*	|*	|z	|*	|*	|*	|*	|.
|*	|*	|*	|*	|*	|*	|*	|*	|n	|*	|*	|*	|*	|*	|*	|*	|*	|*	|*	|*	|*	|n	|*	|*	|*	|*	|.
|*	|*	|*	|*	|*	|*	|*	|*	|a	|*	|*	|*	|*	|*	|*	|*	|*	|*	|*	|*	|*	|a	|*	|*	|*	|*	|.
|*	|*	|*	|*	|*	|*	|*	|*	|*	|*	|*	|*	|*	|*	|*	|*	|*	|*	|*	|*	|*	|*	|*	|*	|*	|*	|.\end{Puzzle}

\newpage

\begin{PuzzleClues}{\textbf{Poziome}\\}\Clue{1}{}{miasto w środkowej Polsce, siedziba władz województwa łódzkiego}
\Clue{2}{}{KORAKAN, RAGI}
\Clue{5}{}{pospolity ptak z rodziny krukowatych}
\Clue{9}{}{jeden z nurtów we współczesnej humanistyce, powiązany m.in. z marksizmem i feminizmem}
\Clue{10}{}{odłamek kruszejącej i cienkiej tafli lodu na rzece, jeziorze lub w morzu}
\Clue{11}{}{wirusowa choroba drzew pestkowych; plamy na liściach, wgłębione pierścienie lub ospowate zagłębienia na owocach}
\Clue{12}{}{farba, w której jako rozpuszczalnik do pigmentu stosuje się olej, zwykle lniany, orzechowy lub makowy}
\Clue{13}{}{nadajnik, który jest przeznaczony do wytwarzania i wysyłania energii o wielkiej częstotliwości; energia ta jest nadawana w postaci fal radiowych, zmodulowanych sygnałem wizyjnym i fonicznym otrzymywanym z ośrodka telewizyjnego}\end{PuzzleClues}

\begin{PuzzleClues}{\textbf{Pionowe}\\}\Clue{1}{}{gołąb domowy z grupy brodawczaków}
\Clue{3}{}{wysokie, bujne zarośla roślin zielnych, złożone przede wszystkim z chwastów: ostów i łopianów, występujące dawniej na stepach ukraińskich}
\Clue{4}{}{prawdopodobieństwo, że różnica pomiędzy badanymi grupami może pojawić się przypadkowo}
\Clue{6}{}{hobby i dziedzina sportu, w której buduje się latające modele szybowców, śmigłowców i samolotów, napędzane silnikami spalinowymi}
\Clue{7}{}{fakt, który w świetle właściwych przepisów prawa stanowi przesłankę powstania i istnienia roszczenia objętego powództwem w sprawie sądowej; termin prawny}
\Clue{8}{}{łącznik - symbol graficzny w notacji muzycznej służący do wydłużania wartości rytmicznych}
\Clue{9}{}{pojemnik lub naczynie skonstruowane w taki sposób, by zwierzęta - zwykle ptaki - mogły swobodnie pobierać z niego pokarm, który samoczynne opada z zasobnika w miarę jak jest wyjadany przez zwierzęta}
\Clue{11}{}{miasto w Bośni i Hercegowinie; ośrodek turystyczny}\end{PuzzleClues}\newpage\section*{Krzyżówka 41}

\noindent\begin{Puzzle}{24}{24}|*	|*	|*	|*	|[1][S]\drarr	|c	|*	|[2][S]\drarr	|p	|a	|r	|t	|e	|r	|*	|*	|[3][S]\darr	|*	|[4][S]\darr	|[5][S]\darr	|[6][S]\drarr	|a	|g	|a	|*	|.
|*	|[7][S]\rarr	|u	|h	|s	|e	|*	|b	|*	|[8][S]\drarr	|o	|r	|k	|*	|*	|[9][S]\darr	|n	|*	|d	|p	|o	|*	|*	|*	|[10][S]\darr	|.
|*	|*	|[11][S]\rarr	|s	|z	|l	|*	|a	|*	|ł	|*	|*	|[12][S]\rarr	|d	|o	|ż	|a	|*	|y	|l	|l	|*	|*	|*	|n	|.
|*	|*	|*	|[13][S]\rarr	|p	|i	|ę	|k	|n	|o	|s	|t	|k	|a	|*	|ó	|w	|*	|m	|a	|i	|*	|*	|*	|a	|.
|*	|*	|[14][S]\rarr	|w	|a	|ń	|k	|a	|*	|t	|[15][S]\darr	|[16][S]\darr	|[17][S]\rarr	|u	|k	|ł	|a	|d	|*	|t	|w	|*	|*	|*	|w	|.
|*	|[18][S]\rarr	|b	|a	|t	|i	|a	|r	|*	|e	|b	|k	|*	|*	|*	|w	|ł	|[19][S]\darr	|*	|y	|k	|[20][S]\darr	|*	|*	|i	|.
|*	|[21][S]\rarr	|m	|m	|[][,]{ }	|h	|g	|*	|*	|w	|u	|o	|*	|*	|*	|[][,]{ }	|n	|r	|*	|n	|a	|z	|*	|*	|l	|.
|*	|*	|[22][S]\rarr	|f	|i	|l	|c	|*	|[23][S]\darr	|s	|r	|m	|[24][S]\darr	|*	|[25][S]\drarr	|m	|i	|ę	|s	|o	|*	|w	|[26][S]\darr	|[27][S]\darr	|ż	|.
|*	|[28][S]\rarr	|n	|i	|s	|z	|a	|*	|n	|k	|s	|b	|m	|*	|n	|a	|k	|k	|*	|w	|[29][S]\darr	|o	|f	|o	|a	|.
|*	|[30][S]\rarr	|w	|o	|l	|e	|c	|*	|a	|i	|z	|a	|a	|*	|a	|l	|[][,]{ }	|a	|*	|a	|g	|l	|a	|b	|c	|.
|*	|*	|[31][S]\rarr	|m	|a	|z	|ł	|a	|m	|*	|t	|j	|s	|*	|w	|o	|o	|*	|*	|[][,]{ }	|w	|n	|u	|s	|z	|.
|*	|*	|*	|*	|n	|*	|[32][S]\darr	|*	|i	|*	|y	|n	|c	|[33][S]\darr	|a	|w	|b	|*	|[34][S]\drarr	|p	|e	|i	|n	|e	|*	|.
|*	|*	|*	|*	|d	|*	|ł	|*	|ę	|*	|n	|[][,]{ }	|a	|k	|ł	|a	|r	|[35][S]\darr	|m	|ł	|r	|e	|i	|r	|*	|.
|*	|*	|*	|*	|z	|*	|y	|*	|t	|*	|n	|r	|r	|r	|n	|n	|o	|ż	|u	|y	|*	|n	|s	|w	|*	|.
|[36][S]\drarr	|b	|a	|r	|k	|a	|s	|*	|n	|*	|i	|o	|p	|u	|i	|y	|ż	|o	|s	|t	|*	|i	|t	|a	|*	|.
|k	|*	|[37][S]\drarr	|f	|i	|v	|e	|[][,]{ }	|o	|[][S]'	|c	|l	|o	|c	|k	|*	|n	|ł	|z	|a	|*	|e	|y	|t	|*	|.
|o	|*	|t	|*	|*	|*	|n	|*	|ś	|*	|t	|n	|n	|z	|[][,]{ }	|*	|y	|ę	|k	|*	|[38][S]\darr	|[][,]{ }	|k	|o	|*	|.
|n	|*	|c	|*	|*	|*	|k	|*	|ć	|*	|w	|i	|e	|ę	|w	|*	|*	|d	|i	|[39][S]\darr	|g	|g	|a	|r	|*	|.
|w	|*	|*	|*	|[40][S]\rarr	|k	|o	|ń	|*	|*	|o	|c	|*	|*	|i	|*	|*	|n	|e	|j	|a	|r	|*	|[][,]{ }	|*	|.
|i	|[41][S]\rarr	|z	|w	|ó	|d	|*	|*	|*	|*	|*	|z	|*	|*	|l	|*	|*	|i	|t	|a	|b	|u	|[42][S]\darr	|s	|*	|.
|k	|*	|[43][S]\rarr	|l	|a	|o	|t	|a	|ń	|c	|z	|y	|k	|*	|s	|*	|*	|c	|e	|ś	|i	|p	|r	|t	|*	|.
|t	|*	|*	|[44][S]\rarr	|b	|r	|y	|l	|a	|n	|t	|*	|[45][S]\rarr	|k	|o	|l	|l	|a	|r	|*	|n	|o	|e	|a	|*	|.
|*	|[46][S]\rarr	|ł	|o	|m	|k	|a	|[][,]{ }	|z	|a	|c	|h	|o	|d	|n	|i	|a	|*	|*	|*	|e	|w	|n	|n	|*	|.
|*	|*	|*	|*	|*	|*	|*	|*	|*	|*	|*	|*	|*	|[47][S]\rarr	|a	|l	|b	|a	|c	|e	|t	|e	|*	|u	|*	|.
|*	|*	|*	|*	|*	|*	|*	|*	|*	|*	|[48][S]\rarr	|s	|t	|d	|*	|[49][S]\rarr	|p	|ł	|e	|ć	|*	|*	|*	|*	|*	|.\end{Puzzle}

\newpage

\begin{PuzzleClues}{\textbf{Poziome}\\}\Clue{1}{}{cyfra rzymska odpowiadająca 100}
\Clue{2}{}{ta kondygnacja, na której znajduje się główne wejście}
\Clue{6}{}{kururu, olbrzymia ropucha z gruczołami jadowymi w tyle głowy, Ameryka Południowa Środkowa}
\Clue{7}{}{pisarz niemiecki (1904-63), uczestnik wojny domowej w Hiszpanii; antyfaszystowskie powieści; „Patrioci”}
\Clue{8}{}{istota z rasy orków w literackim legendarium, wykreowanym przez J. R. R. Tolkiena; Tolkienowscy orkowie, wyhodowani z upodlonych elfów przez Morgotha, związani byli z obozem zła}
\Clue{11}{}{kod ISO 4217 lilangeni}
\Clue{12}{}{tytuł przysługujący władcom (najwyższym urzędnikom) Republiki Weneckiej i Genui}
\Clue{13}{}{piękny drobiazg służący do ozdoby}
\Clue{14}{}{żartobliwie o Rosjaninie}
\Clue{17}{}{w anatomii: zespół narządów współpracujących w wykonywaniu danej funkcji organizmu}
\Clue{18}{}{zwykle z sympatią i pobłażliwie: łobuz, chuligan; słowo z gwary lwowskiej}
\Clue{21}{}{milimetr słupa rtęci - pozaukładowa jednostka miary ciśnienia, równa ciśnieniu słupa rtęci o wysokości jednego milimetra w temperaturze 273,15 K (0 °C) przy normalnym przyspieszeniu ziemskim}
\Clue{22}{}{wyrób włókienniczy otrzymywany przez spilśnianie}
\Clue{25}{}{ujęte konwencjonalnie mięso większych zwierząt, które np. nie jest uważane za jedzenie postne}
\Clue{28}{}{wnęka w murze lub w ścianie}
\Clue{30}{}{wół - wykastrowany samiec bydła domowego z rodziny parzystokopytnych, trzymany w gospodarstwach i używany przez ludzi przy różnych pracach ze względu na zwoją znaczną wielkość i siłę}
\Clue{31}{}{wyrób z wafli i masy cukrowej}
\Clue{34}{}{miasto w Niemczech (Dolna Saksonia), port nad Kanałem Śródlądowym; rafineria ropy naftowej}
\Clue{36}{}{łódź rybacka o skośnym ożaglowaniu}
\Clue{37}{}{niewielkie spotkanie około siedemnastej, podczas którego pije się herbatę}
\Clue{40}{}{zwierzę pociągowe, wierzchowe lub juczne, udomowione około 3000 lat p.n.e}
\Clue{41}{}{przewód elektryczny o małej rezystencji}
\Clue{43}{}{przedstawiciel ludu Laotańczyków, którzy zamieszkują w Laosie, Kambodży, Tajlandii, USA i Francji}
\Clue{44}{}{w drukarstwie: stopień czcionki (jej wielkość), wynoszący 3 punkty}
\Clue{45}{}{pisarz słowacki (1793-1852), ideolog wspólnoty Słowian; „Córa Sławy”, „O literackiej wzajemności”}
\Clue{46}{}{Narthecium ossifragum - gatunek rośliny zielnej z rodziny łomkowatych}
\Clue{47}{}{miasto w Hiszpanii (Murqa), ośrodek administracyjny prowincji Albacete}
\Clue{48}{}{kod ISO 4217 waluty dobra}
\Clue{49}{}{przenośnie: cechy jakiegoś zjawiska, które sprawiają, że jest ono konotowane jako właściwe raczej dla mężczyzn lub raczej dla kobiet}\end{PuzzleClues}

\begin{PuzzleClues}{\textbf{Pionowe}\\}\Clue{1}{}{tradycyjna nazwa minerału stanowiącego chemicznie czystą, przezroczystą wielokrystaliczną odmianę kalcytu, wykształconą w postaci romboedrów}
\Clue{2}{}{gra karciana}
\Clue{3}{}{Oceanodroma hornbyi - gatunek ptaka z rodziny nawałników (Hydrobatidae)}
\Clue{4}{}{mieszanina gazów, czasami z domieszką drobnych ciał stałych, często stanowiąca zanieczyszczenie, mająca postać kłębiącej się w powietrzu chmury}
\Clue{5}{}{wyróżnienie na lokalnym rynku dla wydawnictwa fonograficznego za odpowiednio wysoką sprzedaż egzemplarzy (różne progi w różnych krajach, poza tym zależne od typu nośnika fonograficznego)}
\Clue{6}{}{odcień zieleni}
\Clue{8}{}{przedmiot szkolny lub uczony w ramach kursu, na którym opanowuje się podstawy języka łotewskiego}
\Clue{9}{}{Chrysemys picta - gatunek gada z podrzędu żółwi skrytoszyjnych z rodziny żółwi błotnych, jedyny przedstawiciel rodzaju Chrysemys, charakteryzujący się pancerzem barwy zielono-oliwkowej lub oliwkowej w różnych odcieniach (aż do czarnego) i występowaniem na tym tle na karapaksie i głowie żółtych lub pomarańczowych plam; żyje na obszarze od południowej Kanady do północnego Meksyku}
\Clue{10}{}{naczynie zawieszane na grzejnikach wypełnione wodą w celu nawilżenia powietrza}
\Clue{15}{}{dział rzemiosła obejmujący wyrabianie ozdób z bursztynu}
\Clue{16}{}{maszyna rolnicza służąca do zbioru zbóż i roślin okopowych, wykonująca jednocześnie pracę kilku wcześniej używanych maszyn lub maszyny i brygady roboczej}
\Clue{19}{}{chwytna część kończyny górnej}
\Clue{20}{}{zwolnienie określanej (w zależności od wielkości firmy) liczby osób w ciągu trzydziestu dni w danym przedsiębiorstwie}
\Clue{23}{}{gwałtowna i silna miłość}
\Clue{24}{}{biały ser świeży wyrabiany ze śmietanki z mleka krowiego, pochodzący z północnych Włoch, z Lombardii}
\Clue{25}{}{oceannik żółtopłetwy, nawałnik żółtopłetwy, Oceanites oceanicus - gatunek niewielkiego ptaka oceanicznego z rodziny nawałników (Hydrobatidae); zamieszkuje wody wokół Antarktydy}
\Clue{26}{}{dyscyplina zoologii zajmująca się opisem faun określonych terenów i okresów geologii; rozwinęła się bujnie od czasu Karola Linneusza, wraz z rozwojem systematyki organizmów}
\Clue{27}{}{model układu rzeczywistego, który wykorzystując pomiary wejść i wyjść tego układu dostarcza estymaty (wewnętrznego) stanu układu}
\Clue{29}{}{karabin skałkowy używany w dawnym wojsku polskim}
\Clue{32}{}{ukraiński kompozytor (1842-1912); pianista, dyrygent i pedagog}
\Clue{33}{}{pisklę kruka}
\Clue{34}{}{żołnierz uzbrojony w muszkiet, żołnierz specjalnych oddziałów francuskiej gwardii królewskiej w XVII/XVIII w}
\Clue{35}{}{drobny, nadrzewny gryzoń z rodziny popielicowatych, w Polsce bardzo rzadka, chroniona}
\Clue{36}{}{internat w XVII-XVIII w., prowadzony przy szkole}
\Clue{37}{}{w chemii: symbol technetu}
\Clue{38}{}{pomieszczenie w pałacu bez oddzielnego wyjścia na korytarz}
\Clue{39}{}{warzywo, dobrze znany gatunek fasoli o dużych ziarnach; ziarna odmiany fasoli wielokwiatowej Piękny Jaś}
\Clue{42}{}{pierwiastek 75 w układzie okresowym pierwiastków}\end{PuzzleClues}\newpage\section*{Krzyżówka 42}

\noindent\begin{Puzzle}{25}{27}|*	|*	|*	|*	|*	|*	|*	|*	|*	|*	|*	|*	|*	|*	|*	|*	|*	|*	|*	|*	|[1][S]\darr	|*	|*	|*	|*	|*	|.
|*	|*	|*	|*	|*	|[2][S]\drarr	|p	|i	|e	|r	|ś	|*	|*	|[3][S]\darr	|[4][S]\drarr	|r	|o	|b	|i	|g	|r	|o	|s	|z	|*	|*	|.
|*	|*	|*	|*	|*	|ł	|*	|*	|*	|*	|*	|*	|*	|k	|w	|*	|*	|*	|*	|*	|o	|*	|*	|*	|*	|*	|.
|*	|[5][S]\rarr	|n	|a	|t	|u	|r	|a	|*	|*	|*	|*	|*	|a	|ł	|*	|*	|*	|*	|[6][S]\darr	|z	|*	|*	|*	|*	|*	|.
|*	|*	|[7][S]\rarr	|b	|a	|k	|t	|e	|r	|i	|a	|[][,]{ }	|b	|r	|o	|d	|a	|w	|k	|o	|w	|a	|*	|*	|*	|*	|.
|*	|*	|*	|*	|*	|[][,]{ }	|*	|*	|*	|*	|*	|*	|*	|m	|d	|*	|*	|*	|*	|w	|ó	|*	|*	|*	|*	|*	|.
|*	|*	|*	|*	|*	|k	|*	|*	|*	|*	|*	|*	|*	|n	|a	|*	|*	|*	|*	|s	|j	|*	|*	|*	|*	|*	|.
|[8][S]\drarr	|z	|l	|e	|w	|o	|z	|m	|y	|w	|a	|k	|*	|i	|r	|*	|*	|*	|*	|i	|[][,]{ }	|*	|*	|*	|*	|*	|.
|p	|*	|*	|*	|*	|l	|*	|[9][S]\darr	|*	|*	|*	|*	|*	|k	|z	|*	|*	|*	|*	|c	|p	|*	|*	|*	|*	|*	|.
|a	|*	|*	|*	|*	|a	|*	|m	|*	|*	|*	|*	|*	|[][,]{ }	|*	|*	|*	|*	|*	|a	|o	|*	|*	|*	|*	|*	|.
|p	|*	|*	|[10][S]\drarr	|u	|n	|d	|e	|a	|d	|*	|*	|*	|a	|*	|*	|*	|*	|*	|[][,]{ }	|ś	|*	|*	|*	|*	|*	|.
|e	|*	|*	|ł	|*	|k	|*	|l	|*	|*	|*	|*	|*	|u	|*	|*	|*	|*	|*	|w	|r	|*	|*	|*	|*	|*	|.
|d	|*	|[11][S]\drarr	|e	|w	|o	|l	|u	|c	|j	|a	|*	|*	|t	|*	|*	|*	|*	|*	|i	|e	|*	|*	|*	|*	|*	|.
|a	|*	|b	|m	|*	|w	|*	|z	|*	|*	|*	|*	|*	|o	|*	|*	|*	|*	|*	|e	|d	|*	|*	|*	|*	|*	|.
|*	|*	|e	|k	|*	|y	|*	|y	|*	|*	|*	|*	|*	|m	|*	|*	|*	|*	|*	|c	|n	|*	|*	|*	|*	|*	|.
|*	|*	|n	|o	|*	|*	|*	|n	|*	|*	|*	|*	|*	|a	|*	|*	|*	|*	|*	|z	|i	|*	|*	|*	|*	|*	|.
|*	|*	|g	|w	|*	|*	|*	|a	|*	|*	|*	|*	|*	|t	|*	|*	|*	|*	|*	|n	|*	|*	|*	|*	|*	|*	|.
|*	|*	|a	|s	|*	|*	|*	|*	|*	|*	|*	|*	|*	|y	|*	|*	|*	|*	|*	|i	|*	|*	|*	|*	|*	|*	|.
|*	|*	|l	|z	|*	|*	|*	|*	|*	|*	|*	|*	|*	|c	|*	|*	|*	|*	|*	|e	|*	|*	|*	|*	|*	|*	|.
|*	|*	|i	|c	|*	|*	|*	|*	|*	|*	|*	|*	|*	|z	|*	|*	|*	|*	|*	|[][,]{ }	|*	|*	|*	|*	|*	|*	|.
|*	|*	|n	|z	|*	|*	|*	|*	|*	|*	|*	|*	|*	|n	|*	|*	|*	|*	|*	|z	|*	|*	|*	|*	|*	|*	|.
|*	|*	|a	|y	|*	|*	|*	|*	|*	|*	|*	|*	|*	|y	|*	|*	|*	|*	|*	|i	|*	|*	|*	|*	|*	|*	|.
|*	|*	|*	|z	|*	|*	|*	|*	|*	|*	|*	|*	|*	|*	|*	|*	|*	|*	|*	|e	|*	|*	|*	|*	|*	|*	|.
|*	|*	|*	|n	|*	|*	|*	|*	|*	|*	|*	|*	|*	|*	|*	|*	|*	|*	|*	|l	|*	|*	|*	|*	|*	|*	|.
|*	|*	|*	|a	|[12][S]\rarr	|r	|u	|d	|b	|e	|k	|i	|a	|[][,]{ }	|b	|ł	|y	|s	|k	|o	|t	|l	|i	|w	|a	|*	|.
|*	|*	|*	|*	|*	|*	|*	|*	|*	|*	|*	|*	|*	|*	|*	|*	|*	|*	|*	|n	|*	|*	|*	|*	|*	|*	|.
|*	|*	|*	|*	|*	|*	|*	|*	|*	|*	|*	|*	|*	|*	|*	|*	|*	|*	|*	|a	|*	|*	|*	|*	|*	|*	|.
|*	|*	|*	|*	|*	|*	|*	|*	|*	|*	|*	|*	|*	|*	|*	|*	|*	|*	|*	|*	|*	|*	|*	|*	|*	|*	|.\end{Puzzle}

\newpage

\begin{PuzzleClues}{\textbf{Poziome}\\}\Clue{2}{}{przednia górna część tułowia człowieka (\textasciitilde klatka piersiowa) i niektórych zwierząt}
\Clue{4}{}{pogardliwie o osobie, która stawia sobie za cel w życiu zdobywanie kapitału}
\Clue{5}{}{siła wyższa, rządząca prawami świata; często stosuje się zapis wielką literą}
\Clue{7}{}{bakteria korzeniowa - bakteria glebowa z rodzaju Rhizobium, Bradyrhizobium, Azorhizobium, Sinorhizobium żyjąca w symbiozie z roślinami motylkowatymi, które tworzą na swych korzeniach narośla, tzw. brodawki korzeniowe}
\Clue{8}{}{funkcjonalny odpowiednik umywalki (ujęcie wody) zamontowany w kuchni, służący głównie do mycia naczyń oraz produktów spożywczych i usuwania zbędnych płynów, pozostałych z gotowania potraw}
\Clue{10}{}{kategoria istot żywych w różnych mitologiach (np. w kulcie voodoo), przeniesiona do kultury popularnej i rozwinięta w fantasy; stwór, który jest ciałem niegdyś zmarłej istoty, której proces śmierci został zakłócony lub która została przywrócona do życia zwykle za pomocą czarów, rytuałów lub magicznych mikstur}
\Clue{11}{}{trudna figura gimnastyczna, akrobatyczna lub taneczna}
\Clue{12}{}{Rudbeckia fulgida - gatunek rośliny należący do rodziny astrowatych; pochodzi z Ameryki Północnej, rośnie dziko w stanach Georgia, Teksas i Floryda, do Europy sprowadzona została dopiero w XVIII w. przez ogrodników}\end{PuzzleClues}

\begin{PuzzleClues}{\textbf{Pionowe}\\}\Clue{1}{}{przeobrażenie, zmiana będąca elementem rozwoju}
\Clue{2}{}{część haczyka, między trzonkiem i grotem}
\Clue{3}{}{urządzenie automatyczne, które powoduje samoczynne opadanie paszy z zasobnika do koryta w miarę jak jest ona wyjadana przez zwierzęta}
\Clue{4}{}{w Polsce w średniowieczu, zarządca majątków możnowładców świeckich lub duchownych}
\Clue{6}{}{Helictotrichon sempervirens - gatunek rośliny z rodziny wiechlinowatych}
\Clue{8}{}{Citrus hystrix - gatunek cytrusa, rośliny z rodziny rutowatych}
\Clue{9}{}{kobieca postać z legend i folkloru}
\Clue{10}{}{region Beskidu Niskiego, zamieszkany przez Łemków}
\Clue{11}{}{tkanina półjedwabna o splocie płóciennym, podobna do rypsu}\end{PuzzleClues}\newpage\section*{Krzyżówka 43}

\noindent\begin{Puzzle}{21}{26}|*	|*	|*	|*	|*	|[1][S]\drarr	|t	|e	|n	|d	|e	|n	|c	|y	|j	|n	|o	|ś	|ć	|*	|*	|[2][S]\darr	|.
|*	|*	|*	|[3][S]\drarr	|h	|a	|n	|y	|s	|*	|[4][S]\drarr	|z	|i	|e	|l	|e	|n	|i	|e	|c	|*	|t	|.
|*	|*	|*	|m	|*	|k	|[5][S]\darr	|*	|*	|[6][S]\drarr	|k	|o	|m	|ó	|r	|k	|a	|*	|*	|*	|*	|i	|.
|*	|*	|*	|a	|[7][S]\darr	|l	|s	|*	|[8][S]\drarr	|m	|u	|r	|y	|[][,]{ }	|j	|e	|r	|y	|c	|h	|a	|*	|.
|*	|*	|*	|n	|b	|i	|t	|[9][S]\rarr	|k	|o	|k	|a	|r	|d	|k	|a	|*	|*	|[10][S]\darr	|*	|*	|*	|.
|*	|*	|[11][S]\drarr	|k	|e	|m	|i	|*	|o	|d	|u	|*	|*	|[12][S]\darr	|*	|*	|[13][S]\darr	|*	|m	|*	|*	|*	|.
|*	|[14][S]\darr	|c	|i	|r	|a	|l	|*	|z	|r	|ł	|*	|*	|m	|*	|*	|n	|[15][S]\drarr	|r	|g	|*	|*	|.
|*	|k	|h	|e	|n	|c	|b	|*	|i	|a	|c	|[16][S]\drarr	|h	|a	|k	|*	|i	|k	|ó	|[17][S]\darr	|[18][S]\darr	|*	|.
|*	|a	|i	|t	|o	|j	|i	|*	|o	|s	|z	|i	|[19][S]\darr	|l	|[20][S]\drarr	|s	|e	|r	|w	|i	|s	|*	|.
|*	|r	|ń	|*	|*	|a	|t	|*	|ł	|z	|e	|n	|k	|a	|s	|*	|c	|z	|k	|w	|t	|*	|.
|*	|e	|s	|*	|*	|*	|*	|*	|*	|e	|[][,]{ }	|t	|a	|j	|z	|*	|h	|y	|o	|o	|a	|*	|.
|*	|t	|k	|*	|*	|*	|*	|*	|*	|k	|j	|e	|w	|a	|a	|*	|l	|ż	|ż	|*	|n	|*	|.
|[21][S]\drarr	|k	|i	|c	|z	|*	|*	|*	|*	|*	|a	|r	|a	|l	|u	|*	|u	|a	|e	|[22][S]\darr	|i	|*	|.
|l	|a	|e	|*	|*	|*	|*	|*	|*	|*	|j	|e	|l	|a	|n	|[23][S]\darr	|j	|k	|r	|u	|c	|*	|.
|e	|[][,]{ }	|[][,]{ }	|*	|*	|*	|[24][S]\drarr	|f	|u	|r	|o	|s	|e	|m	|i	|d	|*	|[][,]{ }	|[][,]{ }	|n	|a	|*	|.
|n	|s	|c	|[25][S]\rarr	|s	|e	|m	|k	|o	|w	|*	|*	|r	|*	|s	|o	|*	|o	|w	|i	|*	|*	|.
|*	|z	|e	|*	|[26][S]\drarr	|k	|a	|d	|y	|k	|s	|y	|*	|*	|k	|n	|*	|g	|o	|w	|[27][S]\darr	|*	|.
|[28][S]\drarr	|p	|r	|z	|e	|s	|t	|ę	|p	|n	|o	|ś	|ć	|*	|a	|o	|[29][S]\darr	|r	|r	|e	|p	|*	|.
|o	|i	|e	|[30][S]\drarr	|m	|a	|k	|o	|w	|i	|a	|n	|i	|n	|*	|ś	|d	|o	|k	|r	|i	|*	|.
|k	|t	|m	|a	|m	|[31][S]\rarr	|a	|p	|s	|y	|d	|a	|*	|*	|*	|n	|y	|d	|o	|s	|k	|*	|.
|ó	|a	|o	|n	|e	|*	|*	|*	|*	|*	|*	|*	|*	|*	|*	|i	|p	|o	|w	|y	|i	|*	|.
|l	|l	|n	|g	|r	|*	|*	|[32][S]\drarr	|o	|p	|o	|ń	|c	|z	|y	|k	|o	|w	|a	|t	|e	|*	|.
|n	|n	|i	|a	|s	|[33][S]\drarr	|a	|s	|y	|m	|p	|t	|o	|t	|a	|*	|d	|y	|t	|e	|l	|*	|.
|o	|a	|e	|ż	|o	|b	|*	|z	|*	|*	|[34][S]\rarr	|s	|u	|r	|i	|m	|i	|*	|y	|t	|*	|*	|.
|ś	|*	|*	|*	|n	|a	|*	|o	|*	|[35][S]\rarr	|c	|o	|r	|d	|o	|b	|a	|*	|*	|*	|*	|*	|.
|ć	|*	|*	|*	|*	|t	|[36][S]\rarr	|p	|ł	|a	|t	|e	|c	|z	|e	|k	|*	|*	|*	|*	|*	|*	|.
|*	|*	|*	|*	|*	|*	|*	|*	|*	|*	|*	|*	|*	|*	|*	|*	|*	|*	|*	|*	|*	|*	|.\end{Puzzle}

\newpage

\begin{PuzzleClues}{\textbf{Poziome}\\}\Clue{1}{}{cecha sądów, poglądów - lansowanie pewnego stanu rzeczy, nieobiektywność}
\Clue{3}{}{określenie rdzennego mieszkańca dawnej pruskiej części Górnego Śląska}
\Clue{4}{}{część miasta Dusznik-Zdroju, w Górach Orlickich, na wysokości 800-960 m n.p.m.; ośrodek sportów zimowych, położony w województwie dolnośląskim, w powiecie kłodzkim}
\Clue{6}{}{jedno z ograniczonych pól w tabeli}
\Clue{8}{}{przeszkoda, którą można łatwo obalić}
\Clue{9}{}{sposób zawiązania sznurka, linki itp., który składa się z dwóch skrzyżowanych pętelek}
\Clue{11}{}{miasto i port w Finlandii przy ujściu rzeki Kemi od Zatoki Botnickiej, przemysł drzewny i celulozowo-papierniczy}
\Clue{15}{}{w chemii: symbol pierwiastka roentgen}
\Clue{16}{}{PNIAK; w gwarze łowieckiej górny kieł jelenia}
\Clue{20}{}{usługa np. naprawa}
\Clue{21}{}{cecha czegoś, co ma niską wartość, jest kiczowate}
\Clue{24}{}{organiczny związek chemiczny z grupy sulfonamidów, lek moczopędny i przeciwnadciśnieniowy, należący do grupy diuretyków pętlowych}
\Clue{25}{}{ur. w 1928 r., dyrygent; prowadzi głównie zagraniczne orkiestry symfoniczne oraz teatry operowe}
\Clue{26}{}{zatoka Oceanu Atlantyckiego u południowych wybrzeży Hiszpanii, do zatoki uchodzi Gwadalkiwir}
\Clue{28}{}{to, że jakieś działanie jest przestępne, tzn. ma znamiona przestępstwa lub zostało zakwalifikowane jako przestępstwo}
\Clue{30}{}{mieszkaniec Makowa Podhalańskiego}
\Clue{31}{}{punkt największego przybliżenia lub oddalenia orbity jednego ciała niebieskiego w stosunku do drugiego}
\Clue{32}{}{Encalyptaceae - rodzina mchów z rzędu opończykowców}
\Clue{33}{}{prosta, której odległość od krzywej maleje w miarę przesuwania się wzdłuż tej prostej}
\Clue{34}{}{mielone, rozdrobnione mięso ryby, najczęściej spożywane w formie paluszków}
\Clue{35}{}{miasto w środkowej Argentynie, jeden z głównych ośrodków przemysłowych, kulturalnych i naukowych Argentyny}
\Clue{36}{}{zdrobniale: płatek - płaski, cienki płat czegoś}\end{PuzzleClues}

\begin{PuzzleClues}{\textbf{Pionowe}\\}\Clue{1}{}{przystosowanie zachodzące w warunkach laboratoryjnych, do sztucznie wytwarzanych warunków}
\Clue{2}{}{w chemii: symbol tytanu}
\Clue{3}{}{nakładka na wiośle w miejscu styku z dulką}
\Clue{4}{}{zrzucany na kogoś kłopot lub problematyczny obowiązek}
\Clue{5}{}{pospolity minerał z gromady krzemianów, zaliczany do grupy zeolitów}
\Clue{6}{}{mały motyl dzienny; pospolity na terenach otwartych}
\Clue{7}{}{kanton w zachodniej Szwajcarii, w Alpach, na Wyżynie Szwajcarskiej i w górach Jura, stolica Berno}
\Clue{8}{}{konstrukcja drewniana o rozmaitym zastosowaniu, złożona z belki poziomej wspartej na czterech belkach ukośnie opartych na ziemi}
\Clue{10}{}{Myrmecobius fasciatus - gatunek torbacza, jedyny przedstawiciel rodzaju Myrmecobius i rodziny mrówkożerowatych, krytycznie zagrożony wyginięciem; zamieszkuje eukaliptusowe lasy południowo-zachodniej Australii}
\Clue{11}{}{grzeczność ponad miarę}
\Clue{12}{}{język drawidyjski, używany w Indiach na Wybrzeżu Malabarskim i Lakszadiwach przez około 30 mln mówiących; jest językiem urzędowym w stanie Kerala i na terytorium związkowym Lakszadiwów}
\Clue{13}{}{człowiek niechlujny, taki, który nie dba o czystość i estetykę tego, co robi lub sam jest niezadbany}
\Clue{14}{}{karetka pogotowia będąca własnością szpitala}
\Clue{15}{}{Araneus diadematus - gatunek pająka z rodzaju krzyżaka; nazwa pochodzi od charakterystycznego białego krzyża na odwłoku; występuje w Europie (w Polsce na terenie całego kraju) i Ameryce Północnej}
\Clue{16}{}{eufemistycznie: penis, członek męski}
\Clue{17}{}{miasto w płd.-zach. Nigerii, ośrodek handlu i rzemiosła}
\Clue{18}{}{przygraniczna strażnica}
\Clue{19}{}{ironicznie lub żartobliwie o chłopcu}
\Clue{20}{}{Indianka z plemienia Szaunisów}
\Clue{21}{}{włókno, przędza wytwarzane najczęściej ze lnu zwyczajnego}
\Clue{22}{}{najstarszy rodzaj uczelni o charakterze nietechnicznym, której celem jest przygotowanie kadr pracowników naukowych oraz kształcenie wykwalifikowanych pracowników}
\Clue{23}{}{część wielostrzałowej broni palnej, skąd pocisk przenoszony jest do komory}
\Clue{24}{}{źródło, przyczyna czegoś}
\Clue{26}{}{tenisista australijski, jeden z najbardziej utytułowanych tenisistów świata lat sześćdziesiątych, wygrał wszystkie najpoważniejsze turnieje w owym czasie}
\Clue{27}{}{kąpiel używana do piklowania skór w garbarstwie, składająca się z kwasu siarkowego, chlorku sodu i wody}
\Clue{28}{}{okrężność}
\Clue{29}{}{wers lub człon wersowy, złożony z dwóch stóp (jednostek rytmicznych)}
\Clue{30}{}{praca, zwłaszcza pracownika wykonującego działalność artystyczną}
\Clue{32}{}{nadrzewny ssak drapieżny o cennym futrze}
\Clue{33}{}{duża, drewniana, uzbrojona łódź przeznaczona do różnych prac pomocniczych we flocie, charakteryzująca się płaskim dnem i niewielkim zanurzeniem, opatrzona żaglem, jednomasztowa}\end{PuzzleClues}\newpage\section*{Krzyżówka 44}

\noindent\begin{Puzzle}{23}{27}|*	|*	|*	|*	|*	|*	|*	|*	|*	|*	|*	|[1][S]\drarr	|c	|i	|ą	|g	|[][,]{ }	|d	|a	|l	|s	|z	|y	|*	|.
|*	|*	|*	|*	|*	|*	|*	|[2][S]\rarr	|a	|z	|y	|m	|u	|t	|*	|[3][S]\darr	|*	|[4][S]\darr	|*	|*	|[5][S]\darr	|[6][S]\darr	|*	|*	|.
|*	|[7][S]\rarr	|g	|u	|z	|[][,]{ }	|z	|ł	|o	|ś	|l	|i	|w	|y	|*	|ł	|[8][S]\darr	|w	|*	|[9][S]\darr	|a	|d	|*	|*	|.
|*	|*	|*	|[10][S]\darr	|*	|*	|[11][S]\rarr	|j	|ę	|z	|y	|k	|[][,]{ }	|f	|r	|y	|z	|y	|j	|s	|k	|i	|*	|*	|.
|*	|*	|[12][S]\drarr	|s	|z	|l	|a	|r	|a	|*	|*	|r	|*	|*	|*	|ż	|r	|r	|*	|p	|w	|n	|*	|*	|.
|*	|*	|a	|z	|*	|[13][S]\drarr	|p	|a	|l	|i	|w	|o	|d	|a	|*	|k	|z	|a	|*	|a	|a	|o	|*	|*	|.
|*	|*	|l	|c	|*	|b	|*	|*	|*	|*	|*	|p	|[14][S]\darr	|*	|*	|a	|ą	|f	|*	|r	|l	|c	|*	|*	|.
|*	|[15][S]\darr	|t	|z	|*	|a	|[16][S]\darr	|[17][S]\darr	|[18][S]\darr	|*	|[19][S]\darr	|o	|n	|*	|*	|[][,]{ }	|d	|i	|*	|t	|u	|e	|*	|*	|.
|*	|b	|e	|u	|*	|r	|c	|m	|p	|*	|w	|w	|i	|*	|*	|g	|z	|n	|[20][S]\darr	|a	|n	|r	|*	|*	|.
|*	|o	|r	|r	|*	|d	|u	|a	|o	|*	|a	|i	|a	|*	|*	|i	|e	|o	|p	|ń	|g	|a	|*	|*	|.
|[21][S]\drarr	|a	|n	|o	|n	|*	|b	|ł	|k	|[22][S]\drarr	|c	|e	|l	|a	|*	|n	|n	|w	|o	|c	|*	|t	|*	|*	|.
|s	|[][,]{ }	|a	|s	|*	|*	|i	|ż	|a	|k	|h	|ś	|a	|*	|*	|e	|i	|a	|d	|z	|*	|*	|*	|*	|.
|z	|p	|c	|k	|*	|[23][S]\darr	|c	|o	|r	|m	|t	|ć	|[][,]{ }	|*	|*	|k	|e	|n	|t	|y	|*	|*	|*	|*	|.
|o	|a	|j	|o	|[24][S]\drarr	|s	|u	|r	|m	|i	|a	|*	|g	|[25][S]\darr	|*	|o	|*	|i	|l	|k	|[26][S]\darr	|*	|*	|*	|.
|t	|c	|a	|c	|p	|k	|l	|a	|*	|n	|*	|*	|ó	|c	|*	|l	|[27][S]\darr	|e	|e	|*	|e	|*	|*	|*	|.
|*	|y	|[][,]{ }	|z	|r	|ą	|u	|c	|*	|[][,]{ }	|[28][S]\rarr	|g	|r	|a	|*	|o	|o	|*	|n	|*	|k	|*	|*	|*	|.
|*	|f	|j	|k	|z	|p	|m	|z	|*	|r	|*	|*	|s	|n	|[29][S]\darr	|g	|r	|*	|e	|*	|s	|*	|*	|*	|.
|*	|i	|a	|o	|e	|s	|*	|e	|*	|z	|*	|*	|k	|t	|g	|i	|g	|[30][S]\darr	|k	|[31][S]\darr	|p	|*	|*	|*	|.
|*	|c	|k	|w	|p	|t	|*	|k	|*	|y	|*	|*	|a	|a	|a	|c	|a	|l	|[][,]{ }	|t	|o	|*	|*	|*	|.
|*	|z	|o	|a	|i	|w	|*	|*	|*	|m	|*	|*	|*	|d	|u	|z	|n	|i	|a	|e	|z	|*	|[32][S]\darr	|*	|.
|*	|n	|ś	|t	|ó	|o	|[33][S]\rarr	|p	|a	|s	|[][,]{ }	|m	|i	|e	|d	|n	|i	|c	|z	|n	|y	|*	|z	|*	|.
|*	|y	|c	|e	|r	|*	|*	|*	|*	|k	|*	|*	|*	|s	|i	|a	|z	|o	|o	|o	|c	|*	|i	|*	|.
|*	|*	|i	|*	|k	|[34][S]\drarr	|ł	|a	|c	|i	|n	|n	|i	|k	|*	|*	|a	|*	|t	|r	|j	|[35][S]\darr	|e	|*	|.
|*	|*	|o	|*	|a	|m	|*	|*	|*	|*	|*	|*	|*	|a	|*	|*	|c	|*	|u	|i	|a	|p	|l	|*	|.
|*	|*	|w	|*	|*	|i	|*	|*	|*	|*	|*	|*	|*	|*	|*	|*	|j	|*	|*	|n	|*	|i	|i	|*	|.
|*	|[36][S]\rarr	|a	|b	|i	|s	|y	|ń	|c	|z	|y	|k	|*	|*	|[37][S]\rarr	|f	|a	|l	|c	|o	|n	|e	|t	|*	|.
|*	|*	|*	|*	|*	|*	|*	|*	|*	|*	|*	|*	|*	|*	|*	|*	|*	|*	|*	|*	|*	|s	|z	|*	|.
|*	|*	|*	|*	|*	|*	|*	|*	|*	|*	|*	|*	|*	|*	|*	|*	|*	|*	|*	|*	|*	|*	|*	|*	|.\end{Puzzle}

\newpage

\begin{PuzzleClues}{\textbf{Poziome}\\}\Clue{1}{}{następna część, kontynuowanie}
\Clue{2}{}{kąt zawarty między północną częścią południka odniesienia a danym kierunkiem poziomym}
\Clue{7}{}{nowotwór utworzony z komórek o niskim zróżnicowaniu (niedojrzałych), o budowie znacznie odbiegającej od obrazu prawidłowych tkanek, charakteryzujący się atypią i szybkim wzrostem}
\Clue{11}{}{język, którym posługują się Fryzowie, składający się z wielu dialektów, często bardzo się od siebie różniących; po fryzyjsku mówi ok. 400 tys. osób, głównie w Holandii w prowincji Fryzja, a także w niemieckich landach Dolna Saksonia (region Saterland, region Fryzja Wschodnia) i Szlezwik-Holsztyn (region Fryzja Północna) oraz na Wyspach Fryzyjskich}
\Clue{12}{}{u sów: drobne pióra rosnące promieniście wokół oczu, zazwyczaj odmiennie ubarwione}
\Clue{13}{}{osoba lekkomyślna, nieodpowiedzialna, beztroska}
\Clue{21}{}{swojak, ktoś fajny, ktoś, kto ma ironiczne poczucie humoru, jednocześnie ktoś, kto nie jest nadmiernie przebojowy, osoba o cechach nerda, ktoś, z kim łatwo się utożsamiać innym podobnym młodym ludziom; nazwa wywodzi się od częstego bywania w internecie na tzw. anonimowych kanałach (chanach) i może stanowić informację o tym, że kogoś bawi zachowywanie anonimowości lub ktoś nie chce na potrzeby danej opowieści ujawniać tożsamości}
\Clue{22}{}{powieściopisarz hiszpański, ur. 1916r; „Rodzina Pascuala Duarte”, „Ul” - Nobel '89}
\Clue{24}{}{katalpa, Catalpa - rodzaj roślin należących do rodziny bignoniowatych; należy do niego 11 gatunków pochodzących z Azji i Ameryki Północnej}
\Clue{28}{}{zabawa towarzyska, która ma określone zasady i może wymagać rekwizytów}
\Clue{33}{}{część szkieletu w dolnej części tułowia zbudowana z kości miednicznych i kości krzyżowej połączonych spojeniem łonowym}
\Clue{34}{}{latynista - specjalista od łaciny, badacz tekstów łacińskich, współcześnie najczęściej filolog klasyczny lub filolog romański wyspecjalizowany w łacinie}
\Clue{36}{}{mieszkaniec Abisynii, człowiek pochodzenia abisyńskiego (etiopskiego)}
\Clue{37}{}{rzeźbiarz francuski (1716-91) pomnik Piotra I w Petersburgu}\end{PuzzleClues}

\begin{PuzzleClues}{\textbf{Pionowe}\\}\Clue{1}{}{powieść o bardzo małej objętości; dłuższe opowiadanie}
\Clue{3}{}{narzędzie medyczne do zabiegów ginekologicznych przeprowadzanych wewnątrz macicy}
\Clue{4}{}{nieszczerość, dążenie do celu kosztem innych przy zachowaniu pozorów szlachetności}
\Clue{5}{}{aparat tlenowy pozwalający na swobodne nurkowanie}
\Clue{6}{}{północnoamerykański i azjatycki ssak kopalny, na czaszce trzy pary kostnych wyrostków}
\Clue{8}{}{to, co się zdarza niezależnie od woli człowieka, co przynosi los}
\Clue{9}{}{mieszkaniec Sparty}
\Clue{10}{}{szczuroskoczki, Heteromyidae - rodzina ssaków z rzędu gryzoni; występują w rejonach na zachód od Missisipi w Ameryce Północnej, w Ameryce Środkowej i na północnym wschodzie Ameryki Południowej}
\Clue{12}{}{wymiana jednej głoski na inną, np. lód-lodu}
\Clue{13}{}{głosiciel, wieszcz}
\Clue{14}{}{Tragelaphus buxtoni - gatunek dużego ssaka parzystokopytnego z rodziny krętorogich, zaliczany do antylop, blisko spokrewniony z nialą grzywiastą i kudu; zamieszkuje  niewielkie obszary prowincji Arussi i Bale w Etiopii, na wysokościach 3000-4200 m n.p.m. na terenie Parku Narodowego Bale Mountains}
\Clue{15}{}{Candoia carinata - gatunek gada z rodziny dusicielowatych}
\Clue{16}{}{KUBIKULUM}
\Clue{17}{}{drobny, głównie denny skorupiak; niektóre służą jako pokarm dla ryb}
\Clue{18}{}{mleko wydzielane przez gruczoły mleczne kobiety w okresie laktacji}
\Clue{19}{}{czas (ok. 4 godz.) w ciągu, którego pełni służbę jedna zmiana załogi}
\Clue{20}{}{nieorganiczny związek chemiczny z grupy tlenków azotu, w którym azot jest na formalnym stopniu utlenienia I}
\Clue{21}{}{płytkie zagłębienie bezodpływowe algierskich oraz tunezyjskich pustyń piaszczystych, zazwyczaj z dnem pokrytym solą (stanowi słone jezioro okresowe)}
\Clue{22}{}{Cuminum cyminum - gatunek rośliny jednorocznej, należący do rodziny selerowatych}
\Clue{23}{}{cecha człowieka: przesadna oszczędność}
\Clue{24}{}{mięso przepiórki (także: mięso przygotowane do spożycia, jako że przepiórkę serwuje się zwykle w całości)}
\Clue{25}{}{BIELIKRASA}
\Clue{26}{}{wystawienie organizmu na działanie jakichś czynników}
\Clue{27}{}{działanie podejmowane w celu zorganizowania czegoś, urządzenie czegoś}
\Clue{29}{}{architekt hiszpański (1852-1926), stworzył własną odmianę modernizmu}
\Clue{30}{}{zewnętrzna część ściany}
\Clue{31}{}{słaby głos tenorowy o niewielkiej sile}
\Clue{32}{}{miasto w Niemczech (Saksonia-Anhalt), na płn. od Magdeburga}
\Clue{34}{}{hormon produkowany przez komórki Sertoliego gruczołów płciowych, powodujący podczas embriogenezy organizmów zwierzęcych płci męskiej zanik przewodów Müllera, które w przypadku braku działania hormonu w dalszym rozwoju embrionalnym przekształciłyby się w macicę i jajowody}
\Clue{35}{}{wyzwisko używane w stosunku do człowieka nikczemnego, małego}\end{PuzzleClues}\newpage\section*{Krzyżówka 45}

\noindent\begin{Puzzle}{23}{26}|*	|*	|*	|[1][S]\darr	|*	|*	|*	|*	|*	|*	|*	|*	|*	|*	|*	|*	|*	|[2][S]\drarr	|h	|o	|*	|[3][S]\darr	|[4][S]\darr	|*	|.
|*	|*	|*	|r	|*	|[5][S]\drarr	|p	|s	|a	|ł	|t	|e	|r	|z	|*	|*	|*	|c	|*	|*	|[6][S]\drarr	|d	|s	|*	|.
|*	|*	|*	|u	|[7][S]\rarr	|m	|a	|r	|k	|i	|z	|a	|*	|*	|*	|*	|*	|h	|[8][S]\darr	|*	|n	|r	|z	|[9][S]\darr	|.
|*	|*	|[10][S]\drarr	|n	|s	|u	|t	|a	|*	|*	|*	|[11][S]\rarr	|n	|o	|r	|f	|o	|l	|k	|*	|u	|a	|a	|s	|.
|*	|*	|k	|d	|[12][S]\rarr	|n	|a	|p	|a	|l	|e	|n	|i	|e	|c	|*	|*	|o	|o	|*	|ż	|b	|ł	|t	|.
|*	|*	|r	|*	|[13][S]\rarr	|s	|u	|b	|s	|t	|y	|t	|u	|c	|j	|a	|*	|r	|m	|[14][S]\darr	|e	|i	|*	|y	|.
|*	|*	|a	|[15][S]\rarr	|c	|z	|a	|r	|o	|d	|z	|i	|e	|j	|k	|a	|*	|e	|e	|h	|n	|n	|*	|l	|.
|*	|[16][S]\drarr	|j	|a	|s	|t	|r	|z	|ę	|b	|i	|a	|n	|k	|a	|*	|*	|k	|d	|o	|i	|k	|*	|i	|.
|*	|g	|e	|[17][S]\rarr	|s	|u	|z	|a	|f	|o	|n	|*	|*	|*	|*	|*	|*	|[][,]{ }	|i	|l	|e	|a	|[18][S]\darr	|s	|.
|*	|o	|r	|[19][S]\rarr	|o	|c	|e	|m	|b	|r	|o	|w	|a	|n	|i	|e	|*	|h	|a	|i	|c	|[][,]{ }	|w	|t	|.
|*	|d	|*	|*	|*	|z	|[20][S]\drarr	|a	|u	|d	|y	|t	|o	|r	|k	|a	|*	|e	|[][,]{ }	|d	|[][,]{ }	|l	|y	|k	|.
|*	|z	|[21][S]\rarr	|n	|i	|e	|p	|r	|a	|w	|o	|ś	|ć	|*	|*	|*	|*	|m	|s	|a	|ś	|i	|d	|a	|.
|*	|i	|*	|[22][S]\drarr	|e	|k	|o	|n	|o	|m	|i	|a	|[][,]{ }	|n	|o	|r	|m	|a	|t	|y	|w	|n	|a	|*	|.
|*	|n	|*	|w	|*	|*	|p	|*	|*	|*	|*	|[23][S]\rarr	|a	|t	|e	|i	|s	|t	|a	|*	|i	|o	|t	|*	|.
|*	|n	|*	|s	|[24][S]\drarr	|w	|i	|z	|y	|t	|ó	|w	|k	|a	|*	|*	|*	|y	|r	|*	|ń	|w	|e	|*	|.
|*	|i	|[25][S]\rarr	|p	|r	|z	|e	|d	|g	|ó	|r	|z	|e	|*	|*	|*	|*	|n	|o	|*	|s	|a	|k	|*	|.
|*	|k	|[26][S]\darr	|ó	|y	|[27][S]\drarr	|l	|e	|n	|i	|o	|w	|a	|t	|e	|*	|*	|y	|g	|*	|k	|*	|[][,]{ }	|*	|.
|*	|*	|z	|r	|t	|l	|i	|*	|[28][S]\drarr	|s	|a	|s	|a	|f	|r	|a	|s	|*	|r	|*	|i	|*	|s	|*	|.
|*	|[29][S]\darr	|a	|*	|*	|a	|c	|*	|s	|*	|*	|[30][S]\rarr	|p	|o	|r	|t	|e	|r	|e	|k	|*	|*	|o	|*	|.
|*	|h	|j	|*	|[31][S]\drarr	|j	|a	|g	|ł	|y	|*	|[32][S]\rarr	|a	|d	|w	|o	|k	|a	|c	|j	|a	|*	|c	|*	|.
|*	|r	|a	|*	|k	|t	|*	|*	|ó	|*	|*	|[33][S]\drarr	|k	|o	|p	|c	|z	|y	|k	|i	|*	|*	|j	|*	|.
|*	|a	|d	|*	|o	|h	|*	|*	|j	|*	|*	|a	|[34][S]\rarr	|r	|e	|t	|r	|o	|a	|k	|c	|j	|a	|*	|.
|[35][S]\drarr	|b	|e	|r	|l	|a	|c	|z	|*	|[36][S]\rarr	|m	|a	|g	|i	|e	|r	|k	|a	|*	|*	|*	|*	|l	|*	|.
|b	|i	|k	|*	|o	|*	|[37][S]\rarr	|p	|a	|a	|r	|l	|*	|*	|*	|*	|*	|[38][S]\rarr	|c	|a	|ł	|u	|n	|*	|.
|z	|a	|*	|[39][S]\rarr	|s	|i	|e	|ć	|[][,]{ }	|c	|i	|e	|p	|ł	|o	|w	|n	|i	|c	|z	|a	|*	|y	|*	|.
|d	|*	|*	|*	|*	|*	|*	|*	|*	|*	|*	|n	|*	|[40][S]\rarr	|f	|u	|k	|u	|y	|a	|m	|a	|*	|*	|.
|*	|*	|*	|*	|*	|*	|*	|*	|*	|*	|*	|*	|*	|*	|*	|*	|*	|*	|*	|*	|*	|*	|*	|*	|.\end{Puzzle}

\newpage

\begin{PuzzleClues}{\textbf{Poziome}\\}\Clue{2}{}{w chemii: symbol holmu}
\Clue{5}{}{zbiór 150 psalmów wchodzących w skład Starego Testamentu}
\Clue{6}{}{w chemii: symbol pierwiastka darmsztadt}
\Clue{7}{}{rodzaj składanego (najczęśiej półokrągłego w przekroju bocznym) daszka, osłaniającego okno lub wystawę sklepową}
\Clue{10}{}{ośrodek eksploatacji rud manganu i złota w płn.-zach części Ghanu na płd. od Tarły}
\Clue{11}{}{hrabstwo w Wlk. Brytanii (Anglia) nad Morzem Północnym, obszar 5,4 tyś.km2, główne miasto Norwich}
\Clue{12}{}{człowiek mający na coś ogromną ochotę}
\Clue{13}{}{reakcja chemiczna, w której atom, jon lub grupa atomów zostają podstawione innym atomem, jonem lub grupą atomów}
\Clue{15}{}{wróżka, kobieta odprawiająca czary, wykonująca sztuki magiczne}
\Clue{16}{}{mieszkanka Jastrzębia-Zdroju}
\Clue{17}{}{odmiana helikonu o znacznie szerszej czarze głosowej}
\Clue{19}{}{obmurowanie umacniające zbiornik wody, zwykle studni, fontanny lub brzeg rzeki, zwłaszcza: betonowy krąg studzienny}
\Clue{20}{}{kobieta pracująca w sądownictwie (wojskowym, świeckim, kościelnym), która przygotowuje materiał procesowy}
\Clue{21}{}{cecha tego, co jest niecne, nieprawe}
\Clue{22}{}{dział ekonomii zajmujący się analizą, doskonaleniem i upowszechnianiem strategii rozwoju, opracowanych i stosowanych przez rządy i organizacje międzynarodowe w krajach ubogich}
\Clue{23}{}{zwolennik ateizmu, osoba nie wierząca w Boga}
\Clue{24}{}{tabliczka informacyjna umieszczana na drzwiach lub przy drzwiach, zawierająca dane na temat osób przebywających (np. mieszkających, pracujących) w danym pomieszczeniu}
\Clue{25}{}{równinny, przeważnie lekko falisty obszar rozciągający się równolegle wzdłuż pasma górskiego}
\Clue{27}{}{rodzina muchówek; larwy żyją w glebie, podgryzają korzenie i nasadową część roślin}
\Clue{28}{}{drewno pozyskiwane z wielu gatunków drzewa o tej samej nazwie; używane do wyrobu szaf i kufrów na ubrania}
\Clue{30}{}{zdrobniale o piwie porter}
\Clue{31}{}{jaglana kasza; kasza z całych ziaren prosa, obłuskanych i polerowanych}
\Clue{32}{}{patronat i opieka władcy świeckiego nad Kościołem w średniowieczu}
\Clue{33}{}{konie podlaskie: miejscowa odmiana małych koni typu pogrubionego mierzyna}
\Clue{34}{}{pojęcie związane z paremią lex retro non agit wyrażającą zasadę zakazu nadawania normom prawnym mocy wstecznej}
\Clue{35}{}{gruby, ciepły but (często z futra); mógł być nakładany zimą na cieńsze obuwie lub noszony bezpośrednio na stopie}
\Clue{36}{}{charakterystyczna czapka węgierska}
\Clue{37}{}{miasto w płn.-zach. części Republiki Południowej Afryki, ośrodek regionu uprawy winorośli, drzew cytrusowych, turystycznych}
\Clue{38}{}{najczęściej czarna tkanina służąca do przykrywania zwłok, trumny, katafalku}
\Clue{39}{}{zespół urządzeń technicznych służących do transportu rurociągowego energii cieplnej od źródła ciepła do odbiorców, za pośrednictwem nośnika ciepła}
\Clue{40}{}{miasto w Japonii (płd. Honsiu), port nad Wewnętrznym Morzem Japońskim}\end{PuzzleClues}

\begin{PuzzleClues}{\textbf{Pionowe}\\}\Clue{1}{}{kompozytor polski pochodzący z Moraw (1889-1962); baryton}
\Clue{2}{}{pochodna hemu, która zawiera trójwartościowy atom żelaza (Fe3+) powstający z hemoglobiny bądź hemu pod wpływem działania bardzo silnych związków utleniających}
\Clue{3}{}{rodzaj drabiny, w której drewniane stopnie są powiązane liną}
\Clue{4}{}{szaleństwo, amok - stan psychiczny będący skutkiem silnych: namiętności, gniewu, radości, stan wielkiego podniecenia}
\Clue{5}{}{lufka, ustnik do papierosa}
\Clue{6}{}{Demodex phylloides - gatunek roztocza z rodzaju nużeńca}
\Clue{8}{}{jeden z podstawowych gatunków dramatu starożytnej Grecji, traktowany jako opozycyjny do tragedii, ukształtowany w V wieku p.n.e}
\Clue{9}{}{projektantka wnętrz lub mody}
\Clue{10}{}{bałtycki żaglowiec 1-3 masztowy używany w XV-XIX w., mniejszy od kogi}
\Clue{14}{}{amerykańska piosenkarka jazzowa stylu swing (1915-1959); mulatka, wybitna odtwórczyni ballad i bluesów}
\Clue{16}{}{cyferblat}
\Clue{18}{}{kwota pieniężna przeznaczane na świadczenia społeczne: ubezpieczenia, pomoc społeczną, świadczenia rodzinne i wydatki związane z polityką rynku pracy; także pieniądze przeznaczane na funkcjonowanie instytucji publicznych zarządzających tymi świadczeniami}
\Clue{20}{}{jeżyna sinojagodowa - gatunek jeżyny o płożących się pędach, owoce kwaśne z sinym nalotem}
\Clue{22}{}{spór - konflikt stanowisk, różnica zdań, interesów}
\Clue{24}{}{wycinanie lub rysowanie metoda żłobkowania motywów w metalu. drewnie i itp. ; przedmiot zdobiony w ten sposób}
\Clue{26}{}{pluskwiak różnoskrzydły, poluje na drobne owady i pająki}
\Clue{27}{}{węgierski kompozytor, etnograf i pedagog (1892-1963); opery, balety. utwory symfoniczne, kameralne}
\Clue{28}{}{opakowanie o pojemności co najmniej 50 cm3}
\Clue{29}{}{tytuł szlachecki}
\Clue{31}{}{olbrzymi posąg nadnaturalnej wielkości, np. Kolos Rodyjski}
\Clue{33}{}{w geologii: pierwszy wiek (w geochronologii) lub pierwsze piętro (w chronostratygrafii) środkowej jury}
\Clue{35}{}{kod ISO 4217 dolara Belize}\end{PuzzleClues}\newpage\section*{Krzyżówka 46}

\noindent\begin{Puzzle}{24}{23}|*	|*	|*	|*	|*	|[1][S]\drarr	|m	|y	|ś	|l	|*	|*	|*	|*	|[2][S]\darr	|[3][S]\drarr	|p	|i	|j	|a	|k	|*	|[4][S]\darr	|[5][S]\darr	|*	|.
|*	|*	|[6][S]\drarr	|m	|a	|k	|a	|k	|[][,]{ }	|j	|a	|w	|a	|j	|s	|k	|i	|*	|[7][S]\darr	|*	|*	|*	|r	|h	|*	|.
|*	|[8][S]\darr	|p	|[9][S]\rarr	|p	|o	|d	|o	|f	|i	|c	|e	|r	|[][,]{ }	|z	|a	|p	|r	|z	|ę	|g	|o	|w	|y	|*	|.
|[10][S]\drarr	|p	|i	|s	|a	|n	|o	|*	|*	|*	|*	|*	|*	|[11][S]\darr	|t	|p	|[12][S]\drarr	|d	|a	|u	|n	|*	|a	|d	|*	|.
|a	|o	|ę	|*	|[13][S]\rarr	|s	|z	|p	|r	|y	|c	|a	|*	|b	|y	|s	|l	|*	|k	|*	|*	|*	|n	|r	|*	|.
|w	|c	|t	|[14][S]\rarr	|k	|o	|ł	|o	|n	|i	|c	|e	|*	|ł	|f	|u	|u	|*	|a	|*	|*	|*	|d	|o	|*	|.
|e	|i	|r	|[15][S]\rarr	|e	|r	|k	|e	|l	|*	|*	|*	|*	|ą	|t	|ł	|t	|*	|z	|*	|[16][S]\darr	|*	|y	|f	|*	|.
|r	|ą	|u	|*	|[17][S]\rarr	|c	|h	|w	|y	|t	|a	|k	|*	|d	|*	|a	|*	|[18][S]\darr	|i	|*	|d	|*	|j	|o	|*	|.
|r	|g	|s	|[19][S]\rarr	|d	|j	|*	|*	|*	|*	|*	|*	|*	|[][,]{ }	|*	|*	|*	|r	|k	|*	|y	|*	|c	|r	|*	|.
|o	|[][,]{ }	|*	|[20][S]\drarr	|m	|u	|s	|e	|t	|t	|e	|*	|[21][S]\rarr	|p	|i	|*	|*	|z	|*	|[22][S]\darr	|k	|*	|z	|m	|*	|.
|i	|f	|*	|b	|*	|m	|[23][S]\rarr	|w	|i	|ę	|c	|i	|o	|r	|e	|k	|*	|e	|*	|m	|t	|*	|y	|i	|*	|.
|s	|i	|*	|ą	|[24][S]\darr	|*	|*	|*	|[25][S]\rarr	|k	|o	|z	|i	|o	|ł	|*	|*	|s	|*	|u	|a	|[26][S]\darr	|k	|n	|*	|.
|t	|z	|[27][S]\drarr	|k	|o	|m	|ó	|r	|k	|a	|[][,]{ }	|m	|a	|c	|i	|e	|r	|z	|y	|s	|t	|a	|*	|g	|*	|.
|a	|y	|c	|*	|c	|[28][S]\drarr	|k	|a	|s	|a	|[][,]{ }	|m	|i	|e	|s	|z	|k	|a	|n	|i	|o	|w	|a	|*	|*	|.
|*	|c	|h	|*	|z	|k	|*	|*	|*	|[29][S]\rarr	|d	|i	|k	|d	|i	|k	|*	|*	|*	|c	|r	|a	|*	|*	|*	|.
|*	|z	|a	|*	|y	|r	|*	|*	|*	|[30][S]\rarr	|h	|o	|ł	|u	|j	|*	|*	|*	|*	|a	|*	|t	|*	|*	|*	|.
|*	|n	|u	|*	|s	|a	|[31][S]\rarr	|b	|u	|n	|d	|e	|s	|r	|a	|t	|*	|*	|*	|l	|*	|a	|*	|*	|*	|.
|*	|y	|c	|*	|z	|k	|*	|*	|*	|*	|*	|[32][S]\rarr	|r	|a	|t	|r	|a	|k	|*	|*	|*	|r	|*	|*	|*	|.
|*	|*	|e	|*	|c	|u	|*	|*	|*	|*	|*	|*	|[33][S]\rarr	|l	|a	|w	|s	|o	|n	|i	|a	|*	|*	|*	|*	|.
|*	|[34][S]\drarr	|r	|o	|z	|s	|z	|c	|z	|e	|p	|i	|e	|n	|i	|e	|[][,]{ }	|w	|a	|r	|g	|i	|*	|*	|*	|.
|*	|b	|*	|*	|a	|*	|*	|*	|[35][S]\rarr	|m	|i	|s	|t	|y	|k	|a	|*	|*	|*	|*	|*	|*	|*	|*	|*	|.
|*	|i	|*	|[36][S]\rarr	|c	|z	|w	|o	|r	|o	|n	|ó	|g	|*	|*	|*	|*	|*	|*	|*	|*	|*	|*	|*	|*	|.
|*	|*	|[37][S]\rarr	|c	|z	|u	|j	|n	|i	|k	|[][,]{ }	|k	|o	|n	|t	|a	|k	|t	|o	|w	|y	|*	|*	|*	|*	|.
|[38][S]\rarr	|g	|i	|n	|*	|*	|*	|*	|*	|*	|*	|*	|*	|*	|*	|*	|*	|*	|*	|*	|*	|*	|*	|*	|*	|.\end{Puzzle}

\newpage

\begin{PuzzleClues}{\textbf{Poziome}\\}\Clue{1}{}{wytwór umysłu, efekt aktywności mózgu, który czasem przybiera kształt wrażenia bardziej konkretnego (np. sekwencji obrazów lub konkretnej reakcji skojarzeniowej na bodziec), a czasem jest bardziej nieuchwytny}
\Clue{3}{}{osoba uzależniona od alkoholu}
\Clue{6}{}{Macaca fascicularis - gatunek małpy wąskonosej z rodziny makakowatych; występuje w Azji Południowo-Wschodniej, od Zatoki Bengalskiej do Filipin}
\Clue{9}{}{podoficer, który prowadził zaprzęgi}
\Clue{10}{}{włoski rzeźbiarz, złotnik i architekt (1290-1349) okres gotyku}
\Clue{12}{}{rzeźbiarz i pedagog (1854-1922) rzeźby portretowe i dekoracyjne}
\Clue{13}{}{duża strzykawka}
\Clue{14}{}{Peripoda - zwierzęta morskie zaliczane do rozgwiazd (Asteroidea); kołonice mają ciało w kształcie krążka o średnicy ok. 1 cm, spłaszczonego w osi gębowo-przeciwgębowej (oralno-aboralnej), z kolcami szkieletowymi i nóżkami amulakralnymi na jego krawędziach}
\Clue{15}{}{węgierski kompozytor, dyrygent i pianista (1810-1893); twórca węgierskiej opery narodowej i hymnu państwowego}
\Clue{17}{}{element maszyn, który zwykle jest zawieszony na linie nośnej i służy do chwytania i przenoszenia różnych substancji i materiałów}
\Clue{19}{}{didżej}
\Clue{20}{}{starofrancuski typ komedii muzycznej}
\Clue{21}{}{stała matematyczna, w geometrii euklidesowej równa stosunkowi długości obwodu koła do długości jego średnicy}
\Clue{23}{}{mały więcierz, pułapka na ryby w kształcie cylindra}
\Clue{25}{}{samiec kozy a w języku łowieckim samiec sarny}
\Clue{27}{}{komórka, która ma zdolność do potencjalnie nieograniczonej liczby podziałów oraz do różnicowania się do innych typów komórek}
\Clue{28}{}{prowadzona przez banki instytucja umożliwiająca założenie immiennego rachunku oszczędnościowo-kredytowego i wzięcie kredytu kontraktowego w celu zakupu mieszkania}
\Clue{29}{}{Madoqua saltiana - gatunek małej, półpustynnej antylopy; żyje we wschodniej Afryce: na półpustyniach i sawannach Półwyspu Somalijskiego oraz w zachodniej części Etiopii}
\Clue{30}{}{ur,1916r, pisarz, więzień Oświęcimia; „Wiersze z obozu”, „Dom pod Oświęcimiem”, „Drzewo rodzi owoc”, „Koniec naszego świata”}
\Clue{31}{}{izba parlamentu w Niemczech, konstytucyjne przedstawicielstwo krajów związkowych}
\Clue{32}{}{pojazd gąsienicowy służący do ubijania śniegu na trasach narciarskich}
\Clue{33}{}{TURECZNIA hennowy krzew - tropikalny krzew uprawiany dla barwnika zwanego henną}
\Clue{34}{}{cheiloschisis - wada rozwojowa powstająca we wczesnym okresie embriogenezy, związana z nieprawidłowym rozwojem twarzoczaszki; rozszczepem nazywamy szczelinę lub przerwę powstałą na skutek niepołączenia się struktur formujących wargę}
\Clue{35}{}{coś tajemniczego, niesamowitego, nadprzyrodzonego}
\Clue{36}{}{płaz, gad, ssak lub ptak, przedstawiciel grupy tetrapodów}
\Clue{37}{}{elektrostykowy; czujnik w którym zmiana położenia trzpienia powoduje zmianę oporu elektrycznego}
\Clue{38}{}{anglosaska wódka jałowcowa}\end{PuzzleClues}

\begin{PuzzleClues}{\textbf{Pionowe}\\}\Clue{1}{}{organizacja zrzeszająca kilka podmiotów gospodarczych na określony czas, w konkretnym celu}
\Clue{2}{}{przedmiot o kształcie pałeczki lub pręta}
\Clue{3}{}{część kabiny oddzielona od statku powietrznego w razie awarii}
\Clue{4}{}{mieszkaniec Rwandy, człowiek pochodzenia rwandyjskiego (ruandyjskiego)}
\Clue{5}{}{wytwarzanie paliwa wysokooktanowego w procesie reformingu przy zastosowaniu katalizatora molibdenowo-glinowego}
\Clue{6}{}{potocznie o dwupokładowym autobusie}
\Clue{7}{}{miasto w płn. Egipcie, port w delcie Nilu}
\Clue{8}{}{pociąg do drugiego człowieka, pragnienie nawiązania z nim kontaktu fizycznego}
\Clue{10}{}{zwolennik awerroizmu}
\Clue{11}{}{w prawie - błąd wynikający ze złej interpretacji przepisów, może być powodem kasacji wyroku}
\Clue{12}{}{stop do lutowania}
\Clue{16}{}{o człowieku - tyran, despota, ktoś, kto nie liczy się ze zdaniem innych}
\Clue{18}{}{nieoficjalna nazwa państwa niemieckiego pod rządami NSDAP w latach 1933-1945}
\Clue{20}{}{bryła sztywna mająca możliwość obrotu wokół dowolnej osi stykającej lub ślizgającej się po powierzchni, mogąca wirować wokół środka masy}
\Clue{22}{}{forma teatralna, łącząca muzykę, piosenki, dialogi mówione i taniec}
\Clue{24}{}{ten, kto oczyszcza}
\Clue{26}{}{w hinduizmie: wcielenie (inkarnacja) bóstwa}
\Clue{27}{}{poeta angielski (1343-1400); „Opowieści kantenberyjskie”}
\Clue{28}{}{mieszkaniec Krakowa od pokoleń}
\Clue{34}{}{proces przekształcania danych w informacje, a informacji w wiedzę, która może być wykorzystana do zwiększenia konkurencyjności przedsiębiorstwa}\end{PuzzleClues}\newpage\section*{Krzyżówka 47}

\noindent\begin{Puzzle}{22}{29}|*	|*	|*	|*	|*	|*	|[1][S]\darr	|*	|*	|[2][S]\drarr	|h	|a	|t	|c	|h	|e	|r	|*	|*	|*	|*	|*	|*	|.
|*	|*	|[3][S]\darr	|*	|[4][S]\darr	|[5][S]\drarr	|c	|z	|e	|p	|i	|e	|c	|*	|[6][S]\darr	|[7][S]\darr	|*	|*	|*	|*	|*	|*	|*	|.
|*	|*	|a	|*	|f	|m	|i	|[8][S]\drarr	|r	|o	|c	|z	|n	|i	|k	|a	|r	|z	|*	|*	|*	|[9][S]\darr	|*	|.
|*	|*	|l	|*	|a	|u	|ę	|t	|*	|r	|[10][S]\drarr	|z	|a	|ł	|a	|m	|a	|n	|i	|e	|*	|w	|*	|.
|*	|*	|t	|*	|r	|ł	|ż	|e	|*	|t	|p	|[11][S]\darr	|[12][S]\darr	|*	|l	|a	|[13][S]\darr	|*	|*	|[14][S]\darr	|*	|e	|*	|.
|*	|[15][S]\darr	|e	|[16][S]\darr	|y	|y	|k	|r	|*	|e	|o	|c	|s	|*	|i	|d	|p	|*	|*	|w	|*	|k	|*	|.
|[17][S]\drarr	|g	|r	|o	|s	|*	|i	|y	|[18][S]\darr	|r	|d	|i	|i	|*	|n	|y	|r	|*	|*	|c	|*	|t	|*	|.
|k	|ó	|n	|s	|*	|[19][S]\darr	|[][,]{ }	|t	|b	|*	|h	|ś	|ó	|*	|i	|n	|z	|*	|*	|z	|*	|o	|*	|.
|o	|w	|a	|a	|*	|o	|s	|o	|u	|*	|a	|n	|d	|*	|n	|a	|ę	|*	|*	|a	|*	|r	|*	|.
|s	|n	|t	|[][,]{ }	|[20][S]\drarr	|b	|e	|r	|s	|a	|l	|i	|e	|r	|*	|*	|s	|*	|[21][S]\darr	|s	|*	|[][,]{ }	|*	|.
|t	|o	|y	|l	|z	|r	|n	|i	|o	|[22][S]\darr	|a	|e	|m	|[23][S]\drarr	|w	|a	|ł	|*	|s	|y	|*	|z	|*	|.
|r	|z	|w	|e	|a	|a	|*	|u	|n	|d	|ń	|n	|k	|p	|*	|*	|o	|*	|t	|[][,]{ }	|*	|a	|*	|.
|z	|j	|a	|ś	|k	|z	|*	|m	|i	|y	|s	|i	|a	|l	|*	|[24][S]\darr	|*	|*	|a	|w	|*	|c	|*	|.
|e	|a	|[][,]{ }	|n	|ł	|*	|*	|[][,]{ }	|*	|m	|k	|e	|*	|a	|*	|p	|*	|[25][S]\darr	|r	|[][,]{ }	|*	|z	|*	|.
|w	|d	|w	|a	|a	|*	|*	|a	|*	|k	|i	|*	|*	|t	|[26][S]\darr	|e	|*	|p	|o	|s	|*	|e	|*	|.
|a	|z	|y	|*	|d	|[27][S]\rarr	|ż	|u	|p	|a	|*	|*	|*	|e	|d	|t	|*	|o	|r	|i	|[28][S]\darr	|p	|*	|.
|[][,]{ }	|t	|k	|*	|[][,]{ }	|*	|*	|t	|*	|*	|*	|*	|[29][S]\darr	|r	|o	|r	|*	|z	|u	|o	|k	|i	|*	|.
|d	|w	|l	|[30][S]\drarr	|w	|i	|n	|o	|[][,]{ }	|g	|r	|o	|n	|o	|w	|e	|*	|a	|s	|d	|r	|o	|*	|.
|ł	|o	|u	|s	|z	|*	|*	|n	|*	|*	|*	|*	|y	|w	|ó	|l	|*	|ś	|k	|l	|ą	|n	|*	|.
|u	|*	|c	|z	|a	|[31][S]\drarr	|b	|o	|r	|z	|e	|ś	|l	|a	|d	|[][,]{ }	|ś	|w	|i	|e	|ż	|y	|*	|.
|g	|*	|z	|k	|j	|b	|[32][S]\rarr	|m	|a	|d	|a	|*	|o	|n	|*	|b	|*	|i	|*	|*	|e	|*	|*	|.
|o	|*	|a	|o	|e	|i	|*	|i	|[33][S]\rarr	|ś	|w	|i	|n	|i	|a	|r	|k	|a	|*	|*	|k	|*	|*	|.
|l	|*	|j	|p	|m	|n	|*	|c	|*	|*	|*	|*	|*	|e	|*	|u	|*	|t	|*	|*	|*	|[34][S]\darr	|*	|.
|i	|*	|ą	|s	|n	|d	|[35][S]\drarr	|z	|e	|s	|p	|ó	|ł	|*	|*	|n	|*	|*	|*	|*	|*	|r	|*	|.
|s	|*	|c	|k	|y	|a	|a	|n	|*	|*	|*	|*	|*	|[36][S]\rarr	|k	|a	|d	|z	|i	|d	|ł	|o	|*	|.
|t	|*	|a	|i	|*	|*	|s	|e	|*	|[37][S]\rarr	|m	|i	|n	|i	|a	|t	|u	|r	|k	|a	|*	|j	|*	|.
|n	|*	|*	|*	|*	|*	|*	|*	|[38][S]\rarr	|s	|a	|m	|p	|l	|i	|n	|g	|*	|*	|*	|*	|n	|*	|.
|a	|*	|*	|[39][S]\rarr	|p	|l	|a	|c	|e	|k	|[][,]{ }	|a	|m	|e	|r	|y	|k	|a	|ń	|s	|k	|i	|*	|.
|*	|*	|[40][S]\rarr	|s	|t	|r	|a	|s	|b	|u	|r	|c	|z	|y	|k	|*	|*	|*	|*	|*	|*	|k	|*	|.
|*	|*	|*	|*	|*	|*	|*	|*	|*	|*	|*	|*	|*	|*	|*	|*	|*	|*	|*	|*	|*	|*	|*	|.\end{Puzzle}

\newpage

\begin{PuzzleClues}{\textbf{Poziome}\\}\Clue{2}{}{hokeista, obrońca Pittsburgh Penguins}
\Clue{5}{}{kobiece nakrycie głowy z płótna bądź perkalu, przybrane falbankami i zawiązywane pod brodą, noszone dawniej przez mężatki i wdowy - CZEPEK}
\Clue{8}{}{osoba, która zajmuje się rocznikarstwem}
\Clue{10}{}{stan depresyjny, zmiana na gorsze}
\Clue{17}{}{malarz francuski (1771-1835) obrazy przedstawiające epopeję napoleońska, portrety}
\Clue{20}{}{w armii włoskiej: żołnierz oddziału piechoty zmotoryzowanej}
\Clue{23}{}{narzędzie uprawowe, którego zadaniem jest ugniatanie gleby, prowadzące do rozkruszenia brył i zmniejszenia stopnia jej porowatości}
\Clue{27}{}{kopalnia soli}
\Clue{30}{}{wino produkowane z winogron}
\Clue{31}{}{Pohlia cruda - gatunek mchu z rodziny prątnikowatych}
\Clue{32}{}{najczęściej w l. mn.; rodzaj gleby}
\Clue{33}{}{rasa owcy domowej, głównie kożuchowej}
\Clue{35}{}{grupa ludzi działających wspólnie, pracujących, wykonujacych razem inne czynności}
\Clue{36}{}{żywica pozyskiwana z drzew rodzaju kadzidla (Boswellia) rosnących na półpustyniach i obrzeżach pustyń północno-wschodniej Afryki oraz regionu Hadramaut na granicy Omanu i Jemenu}
\Clue{37}{}{zdrobniale: miniatura - krótki utwór literacki, muzyczny, teatralny itp}
\Clue{38}{}{użycie fragmentu wcześniej dokonanego nagrania muzycznego, zwanego samplem jako elementu tworzonego utworu przy użyciu specjalnego instrumentu zwanego samplerem (lub komputera)}
\Clue{39}{}{tradycyjna potrawa amerykańska, pulchny placek serwowany na słodko, zwykle jako śniadanie}
\Clue{40}{}{mieszkaniec Strasburga, człowiek pochodzący ze Strasburga}\end{PuzzleClues}

\begin{PuzzleClues}{\textbf{Pionowe}\\}\Clue{1}{}{męczący sen, który nie przynosi odpoczynku}
\Clue{2}{}{George ur. w 1920 r., chemik brytyjski, współtwórca chemii stanów wzbudzonych, laureat nagrody Nobla}
\Clue{3}{}{w logice - spójnikXOR, któremu w języku naturalnym odpowiada schemat zdaniowyalbo ..., albo ...}
\Clue{4}{}{muzułmański jeździec, profesjonalny wojownik (mameluk), wojownik-niewolnik gulam, odpowiednik europejskiego rycerza}
\Clue{5}{}{mięśnie widoczne u osoby dobrze zbudowanej (najczęściej dwugłowe ramienia)}
\Clue{6}{}{dzisiejszy Twer - miasto obwodowe w Rosji, port nad Wołgą, przy ujściu Twercy; nazwa Kalinin obowiązywała od 1931 do 1990 roku}
\Clue{7}{}{egzotyczny ptak o grubym dziobie z rzędu wróblowatych, barwne upierzenie; od Afryki, przez Indie do Australii}
\Clue{8}{}{fragment obszaru państwa, cieszący się pewną niezależnością od władzy centralnej, niejednokrotnie posiadający własne władze}
\Clue{9}{}{wektor, którego początek jest przyporządkowany jakiemuś stałemu punktowi}
\Clue{10}{}{gwara podhalańska - jedna z gwar dialektu małopolskiego, występująca na terenie Podhala}
\Clue{11}{}{nacisk krwi na ściany naczyń krwionośnych, przez które przepływa}
\Clue{12}{}{siódmy od przodu ząb}
\Clue{13}{}{w mostach, wiaduktach - konstrukcyjny element łączący dwie podpory (słupy, filary)}
\Clue{14}{}{obóz jeździecki}
\Clue{15}{}{chciwość, kombinatorstwo, cecha, postawa gównozjada - człowieka, który jest tak chytry, że prawdopodobnie byłby skłonny zjeść własne gówno (często sprawia wrażenie, że niewiele wystarcza mu do przeżycia); wąskie horyzonty, postawa osobista, w której nie mierzy się zbyt wysoko, także: lenistwo, abnegacja, szkodliwość społeczna}
\Clue{16}{}{Dolichovespula sylvestris - gatunek owada z rodziny osowatych (Vespidae); występuje w Europie i Azji}
\Clue{17}{}{Festuca guestphalica - gatunek trawy z rodziny wiechlinowatych}
\Clue{18}{}{włoski kompozytor i pianista (1866-1924); propagator neoklasycyzmu; utwory fortepianowe, orkiestrowe, opery, pieśni}
\Clue{19}{}{pogląd, jaki ktoś ma o kimś lub o czymś, oparty na posiadanym na ten temat wyobrażeniu}
\Clue{20}{}{zakład o wygrane pieniężne lub rzeczowe, również zakład zawarty za pośrednictwem totalizatorów i bukmacherów; termin prawny}
\Clue{21}{}{język słowiański, który na przestrzeni wieków rozwinął się w trzy języki narodowe: rosyjski, ukraiński i białoruski}
\Clue{22}{}{drobne cebule cebuli zwyczajnej użytkowane jako sadzonki}
\Clue{23}{}{pokrywanie metalu wartwą metalu szlachetniejszego}
\Clue{24}{}{Pterodroma solandri - gatunek ptaka z rodziny burzykowatych (Procellariidae)}
\Clue{25}{}{coś poza światem, przestrzeń poza światem}
\Clue{26}{}{w logice, skończony ciąg zdań uzasadnijący prawdziwość jakiegoś twierdzenia}
\Clue{28}{}{niewielki walec wykonany z czarnej, wulkanizowanej gumy służący do gry w hokeja na lodzie}
\Clue{29}{}{tkanina z tego włókna}
\Clue{30}{}{pogardliwie: niemiecki - język z grupy zachodniej rodziny języków germańskich}
\Clue{31}{}{opaska do podwiązywania wąsów}
\Clue{34}{}{bylina z rodziny gruboszowatych o żółtych kwiatach, niektóre gatunki uprawiane w ogródkach skalnych}
\Clue{35}{}{figura karciana}\end{PuzzleClues}\newpage\section*{Krzyżówka 48}

\noindent\begin{Puzzle}{16}{30}|*	|*	|*	|[1][S]\drarr	|o	|k	|o	|*	|[2][S]\drarr	|e	|s	|*	|*	|*	|*	|*	|*	|.
|*	|[3][S]\darr	|*	|n	|[4][S]\rarr	|k	|r	|o	|w	|i	|e	|ń	|c	|z	|a	|k	|*	|.
|[5][S]\drarr	|g	|r	|a	|n	|t	|o	|b	|i	|o	|r	|c	|a	|*	|*	|*	|*	|.
|d	|a	|[6][S]\darr	|t	|*	|*	|*	|[7][S]\darr	|e	|*	|[8][S]\drarr	|g	|a	|r	|y	|*	|*	|.
|o	|r	|d	|a	|*	|*	|*	|c	|l	|[9][S]\rarr	|f	|u	|j	|i	|a	|n	|*	|.
|m	|d	|i	|l	|[10][S]\drarr	|m	|i	|z	|o	|g	|i	|n	|i	|a	|*	|*	|*	|.
|[][,]{ }	|e	|a	|*	|s	|*	|[11][S]\darr	|w	|p	|*	|e	|*	|*	|[12][S]\darr	|*	|*	|*	|.
|w	|n	|l	|*	|u	|*	|ł	|o	|ł	|[13][S]\drarr	|d	|e	|r	|p	|*	|*	|*	|.
|c	|i	|o	|[14][S]\darr	|f	|*	|y	|r	|e	|d	|i	|*	|*	|s	|*	|*	|*	|.
|z	|a	|g	|e	|i	|[15][S]\drarr	|k	|a	|t	|o	|n	|*	|*	|e	|*	|*	|*	|.
|a	|*	|[][,]{ }	|l	|t	|r	|a	|k	|w	|m	|*	|*	|*	|u	|[16][S]\darr	|*	|*	|.
|s	|*	|o	|e	|*	|e	|c	|*	|c	|i	|*	|*	|*	|d	|k	|[17][S]\darr	|*	|.
|o	|*	|b	|m	|[18][S]\darr	|l	|z	|*	|e	|n	|[19][S]\darr	|*	|[20][S]\darr	|o	|a	|l	|*	|.
|w	|[21][S]\darr	|y	|e	|s	|i	|*	|*	|*	|i	|c	|*	|o	|p	|r	|e	|*	|.
|y	|p	|w	|n	|z	|g	|*	|*	|*	|k	|e	|*	|r	|o	|b	|n	|*	|.
|*	|ę	|a	|t	|c	|i	|*	|[22][S]\drarr	|c	|a	|l	|v	|a	|d	|o	|s	|*	|.
|*	|t	|t	|[][,]{ }	|z	|a	|[23][S]\rarr	|o	|d	|n	|o	|ś	|n	|i	|k	|*	|*	|.
|[24][S]\drarr	|l	|e	|g	|u	|n	|*	|s	|*	|i	|w	|*	|t	|u	|a	|*	|*	|.
|l	|a	|l	|r	|d	|t	|*	|a	|[25][S]\darr	|n	|n	|*	|e	|m	|t	|[26][S]\darr	|*	|.
|i	|[][,]{ }	|s	|z	|l	|*	|*	|d	|h	|*	|i	|*	|s	|*	|i	|a	|*	|.
|s	|a	|k	|e	|a	|*	|*	|[][,]{ }	|a	|*	|k	|*	|*	|*	|o	|t	|*	|.
|t	|b	|i	|j	|r	|*	|*	|e	|r	|*	|*	|*	|*	|*	|n	|m	|*	|.
|k	|o	|*	|n	|s	|[27][S]\darr	|*	|o	|m	|*	|*	|*	|*	|*	|*	|o	|*	|.
|o	|n	|*	|y	|t	|ł	|*	|l	|o	|*	|*	|*	|*	|*	|*	|m	|*	|.
|w	|e	|*	|*	|w	|o	|[28][S]\rarr	|i	|n	|h	|a	|m	|b	|a	|n	|e	|*	|.
|i	|n	|[29][S]\drarr	|s	|o	|p	|o	|c	|i	|a	|n	|i	|n	|*	|*	|t	|*	|.
|e	|c	|z	|*	|*	|a	|*	|z	|k	|*	|*	|[30][S]\rarr	|c	|h	|ó	|r	|*	|.
|c	|k	|r	|*	|*	|t	|[31][S]\rarr	|n	|a	|u	|k	|a	|*	|*	|*	|*	|*	|.
|*	|a	|a	|*	|*	|a	|*	|y	|*	|*	|*	|*	|*	|*	|*	|*	|*	|.
|*	|*	|z	|*	|*	|*	|*	|*	|*	|*	|*	|*	|*	|*	|*	|*	|*	|.
|*	|*	|*	|*	|*	|*	|*	|*	|*	|*	|*	|*	|*	|*	|*	|*	|*	|.\end{Puzzle}

\newpage

\begin{PuzzleClues}{\textbf{Poziome}\\}\Clue{1}{}{służba pełniona przez marynarzy na dziobie statku}
\Clue{2}{}{w chemii: symbol einsteinu}
\Clue{4}{}{metalicznie czarny chrząszcz, na głowie samca róg, żywi się odchodami ssaków, pożyteczny}
\Clue{5}{}{podmiot, który otrzymuje grant}
\Clue{8}{}{(1914-80) pisarz francuski, właściwie R. Kacev; „Korzenie nieba”}
\Clue{9}{}{FUCIEN - prowincja w płn-wsch. Chinach, stolica Fuzhou, powierzchnia 121,1 tyś. km2, 26,8 min mieszkańców}
\Clue{10}{}{nienawiść albo silne uprzedzenie w stosunku do płci żeńskiej}
\Clue{13}{}{bardzo głupi wyraz twarzy, często interpretowany jako wyraz zdziwienia lub nieogarniania}
\Clue{15}{}{twórczość, myśl społeczno-polityczna Katona}
\Clue{22}{}{wytrawny winiak francuski produkowany głównie w departamencie Calvados w Normandii poprzez destylację cydru}
\Clue{23}{}{odsyłacz - znak w tekście (gwiazdka, liczba, litera) kierujący czytelnika do objaśnienia, przypisu}
\Clue{24}{}{żołnierz legionów polskich z okresu I wojny światowej}
\Clue{28}{}{miasto w Mozambiku, nad Kanałem Mozambickim, ośrodek administracyjny prowincji Inhambane}
\Clue{29}{}{mieszkaniec Sopotu}
\Clue{30}{}{grupa śpiewaków}
\Clue{31}{}{morał, przestroga, nauczka}\end{PuzzleClues}

\begin{PuzzleClues}{\textbf{Pionowe}\\}\Clue{1}{}{prowincja we wschodniej części R.P.A, powierzchnia 87 tyś. km2, ośrodek administracyjny Pietermaritzburg}
\Clue{2}{}{miastugi - nazwa rodzaju słodkowodnych ryb (Polypterus) z rodziny wielopłetwcowatych (Polypteridae)}
\Clue{3}{}{PRZEPYSZLIN wiecznie zielony krzew lub drzewo o wonnych, okazałych kwiatach, uprawiany także w doniczkach}
\Clue{5}{}{obiekt (lub zespół obiektów) noclegowy przeznaczony i przystosowany do świadczenia wyłącznie lub głównie usług związanych z wczasami, mieszczący, oprócz pomieszczeń noclegowych, stołówkę, świetlicę, oraz inne urządzenia uatrakcyjniające pobyt }
\Clue{6}{}{kontakt między władzą państwową a organizacjami trzeciego sektora, który polega na wzajemnym przekazywaniu sobie informacji czy ustaleń dotyczących celów, instrumentów oraz strategii wdrażania polityki publicznej}
\Clue{7}{}{srebrna moneta bita za Zygmunta Augusta}
\Clue{8}{}{pisarz rosyjski (1892-1977), powieści psychologiczne, wspomnienia; „Porwanie Europy”, „Pierwsze porywy”, „Ognisko”}
\Clue{10}{}{dolna, widoczna z pomieszczenia, część stropu}
\Clue{11}{}{połykacz, człowiek, który daje pokazy tego, jak coś połyka lub udaje, że połyka}
\Clue{12}{}{wypustka plazmatyczna o zmiennym kształcie, powstająca w wyniku nacisku cytoplazmy na otaczającą ją błonę komórkową}
\Clue{13}{}{członek Zakonu Kaznodziejskiego}
\Clue{14}{}{zasadnicza część elektrycznego oporowego urządzenia grzejnego w którym dokonuje się przemiana energii elektrycznej w ciepło}
\Clue{15}{}{pogardliwie o osobie nadmiernie eksponującej swoją pobożność}
\Clue{16}{}{jon, w którym ładunek dodatni jest skupiony na atomie węgla}
\Clue{17}{}{miasto we Francji na płn.-zach od Lille, duży ośrodek polonijny, wydobycie węgla kamiennego}
\Clue{18}{}{sztuka chodzenia na szczudłach - umiejętność poruszania się nad ziemią na specjalnych kijach}
\Clue{19}{}{część przyrządu geodezyjnego, służąca do nakierowywania kątomierza na oddalony obiekt}
\Clue{20}{}{tenisista hiszpański, czołowy tenisista świata lat 70-tych}
\Clue{21}{}{obwód łączący zakończenie sieci telekomunikacyjnej z punktem dostępu do publicznej sieci telefonicznej}
\Clue{22}{}{osad utworzony przez wiatr podczas procesu eolicznego}
\Clue{24}{}{CYTRYNEK}
\Clue{25}{}{rodzaj funkcji matematycznej}
\Clue{26}{}{przyrząd do pomiaru prędkości parowania wody do atmosfery}
\Clue{27}{}{plaska część śmigła lotniczego}
\Clue{29}{}{w ogrodnictwie: tzw. komponent szczepienia; część szlachetnej rośliny (zazwyczaj jednoroczny pęd), którą naszczepia się na podkładce}\end{PuzzleClues}\newpage\section*{Krzyżówka 49}

\noindent\begin{Puzzle}{25}{23}|*	|*	|*	|*	|*	|*	|*	|*	|*	|*	|[1][S]\darr	|[2][S]\drarr	|ł	|y	|s	|k	|*	|*	|*	|*	|[3][S]\darr	|*	|*	|*	|[4][S]\darr	|*	|.
|*	|*	|*	|*	|*	|*	|[5][S]\drarr	|c	|z	|a	|p	|l	|a	|[][,]{ }	|n	|a	|d	|o	|b	|n	|a	|*	|*	|*	|p	|*	|.
|*	|*	|[6][S]\drarr	|h	|o	|y	|e	|r	|*	|*	|e	|e	|*	|[7][S]\rarr	|i	|s	|l	|a	|m	|i	|s	|t	|a	|*	|a	|*	|.
|*	|*	|i	|*	|[8][S]\darr	|[9][S]\darr	|k	|*	|[10][S]\drarr	|a	|r	|k	|a	|d	|a	|*	|[11][S]\rarr	|a	|r	|c	|t	|g	|*	|*	|r	|*	|.
|*	|[12][S]\darr	|l	|*	|ś	|w	|s	|*	|t	|*	|r	|[][,]{ }	|*	|[13][S]\rarr	|t	|a	|c	|z	|k	|a	|r	|z	|*	|*	|a	|*	|.
|*	|h	|u	|[14][S]\darr	|m	|y	|*	|*	|v	|*	|o	|p	|*	|[15][S]\rarr	|s	|z	|p	|e	|t	|n	|o	|ś	|ć	|*	|m	|*	|.
|*	|i	|z	|d	|i	|m	|*	|*	|*	|*	|n	|r	|*	|*	|*	|*	|[16][S]\rarr	|j	|a	|ś	|m	|i	|n	|*	|a	|*	|.
|*	|s	|j	|ł	|e	|i	|*	|[17][S]\rarr	|n	|i	|e	|z	|d	|a	|r	|s	|t	|w	|o	|*	|e	|*	|[18][S]\darr	|[19][S]\darr	|g	|*	|.
|[20][S]\drarr	|z	|a	|u	|r	|o	|p	|l	|i	|t	|*	|e	|[21][S]\rarr	|a	|w	|a	|n	|p	|o	|r	|t	|*	|s	|k	|n	|*	|.
|t	|p	|[][,]{ }	|g	|c	|t	|*	|[22][S]\rarr	|s	|p	|e	|c	|j	|a	|l	|i	|s	|t	|a	|*	|r	|[23][S]\darr	|i	|u	|e	|*	|.
|r	|a	|p	|o	|i	|y	|[24][S]\rarr	|t	|e	|r	|m	|i	|n	|[][,]{ }	|z	|a	|w	|i	|t	|y	|*	|p	|e	|t	|t	|*	|.
|z	|ń	|i	|p	|o	|*	|*	|*	|*	|*	|[25][S]\rarr	|w	|y	|j	|a	|d	|a	|c	|z	|*	|*	|l	|c	|a	|y	|*	|.
|e	|s	|e	|ł	|ż	|[26][S]\darr	|[27][S]\rarr	|l	|e	|j	|t	|n	|a	|n	|t	|*	|*	|*	|*	|*	|*	|u	|z	|s	|k	|*	|.
|c	|k	|n	|e	|e	|m	|[28][S]\drarr	|p	|i	|g	|w	|o	|w	|i	|e	|c	|[][,]{ }	|c	|h	|i	|ń	|s	|k	|i	|*	|*	|.
|i	|i	|i	|t	|r	|a	|w	|[29][S]\rarr	|r	|o	|z	|w	|ó	|j	|[][,]{ }	|r	|o	|d	|o	|w	|y	|*	|a	|[][,]{ }	|*	|*	|.
|*	|[][,]{ }	|ą	|w	|c	|n	|o	|*	|*	|[30][S]\rarr	|p	|o	|r	|o	|z	|u	|m	|i	|e	|n	|i	|e	|*	|ł	|*	|*	|.
|*	|p	|d	|i	|a	|d	|s	|[31][S]\rarr	|p	|r	|o	|t	|e	|k	|c	|j	|o	|n	|i	|z	|m	|*	|[32][S]\darr	|e	|*	|*	|.
|*	|t	|z	|e	|*	|y	|c	|*	|*	|*	|[33][S]\rarr	|w	|e	|n	|t	|a	|*	|*	|*	|*	|*	|*	|a	|b	|*	|*	|.
|*	|a	|a	|c	|*	|l	|h	|*	|*	|*	|[34][S]\rarr	|o	|r	|c	|z	|y	|c	|a	|*	|*	|*	|*	|v	|*	|*	|*	|.
|*	|s	|*	|*	|*	|i	|o	|*	|*	|*	|[35][S]\rarr	|r	|ó	|ż	|o	|w	|e	|[][,]{ }	|s	|ł	|o	|n	|i	|e	|*	|*	|.
|*	|z	|*	|*	|[36][S]\rarr	|o	|d	|l	|e	|g	|ł	|o	|ś	|ć	|*	|[37][S]\rarr	|b	|e	|a	|r	|d	|s	|l	|e	|y	|*	|.
|*	|e	|*	|*	|*	|n	|*	|*	|*	|*	|*	|w	|*	|*	|*	|*	|*	|[38][S]\rarr	|k	|o	|ź	|l	|a	|k	|*	|*	|.
|*	|k	|*	|*	|*	|*	|[39][S]\rarr	|u	|c	|h	|w	|y	|t	|[][,]{ }	|ś	|l	|i	|z	|g	|o	|w	|y	|*	|*	|*	|*	|.
|*	|*	|*	|*	|*	|*	|*	|*	|*	|*	|*	|*	|*	|*	|*	|*	|*	|*	|*	|*	|*	|*	|*	|*	|*	|*	|.\end{Puzzle}

\newpage

\begin{PuzzleClues}{\textbf{Poziome}\\}\Clue{2}{}{błysk, krótkie silne światło}
\Clue{5}{}{Egretta garzetta garzetta - nominatywny podgatunek czapli nadobnej (Egretta garzetta)}
\Clue{6}{}{histolog embriolog i lekarz (1834-1907); praca z histologii układu krwionośnego i tkanki łącznej}
\Clue{7}{}{naukowiec zajmujący się islamem}
\Clue{10}{}{element architektoniczny złożony z łuku i dwóch podpór}
\Clue{11}{}{funkcja odwrotna do funkcji tangens rozpatrywanej na przedziale (-?/2,? /2), oznacza to, że jeśli y = arctg(x), to, że tg(y) = x, przy czym y zawiera się w przedziale (-?/2,? /2)}
\Clue{13}{}{robotnik pchający taczkę, przewożący ciężary na taczce}
\Clue{15}{}{nacechowanie wulgarnością, brakiem przestrzegania norm obyczajowych}
\Clue{16}{}{Jasminum - rodzaj roślin z rodziny oliwkowatych}
\Clue{17}{}{niezręczność lub niedołęstwo}
\Clue{20}{}{Sauroplites - rodzaj dinozaura z rodziny ankylozaurów, żyjący w okresie wczesnej kredy na terenach wschodniej Azji; długość ciała 5-6 m, wysokość 1,7 m, ciężar 1 t}
\Clue{21}{}{PRZEDPORCIE; obszerny basen przy wejściu do portu, objęty falochronem}
\Clue{22}{}{lekarz specjalista, lekarz danej specjalizacji}
\Clue{24}{}{termin stanowczy o dużym rygorze prawnym;niepodjęcie określonej czynności przez uprawniony podmiot w okresie określonym tym terminem powoduje definitywne wygaśnięcie przysługującego podmiotowi prawa do tej czynności}
\Clue{25}{}{człowiek, który zna się na czymś bardzo dobrze, jest bardzo doświadczony}
\Clue{27}{}{stopień oficerki odpowiadający porucznikowi}
\Clue{28}{}{Pseudocydonia sinensis - gatunek rośliny wieloletniej z rodziny różowatych (Rosaceae)}
\Clue{29}{}{droga rozwoju rodowego, pochodzenie i zmiany ewolucyjne grupy organizmów, zwykle gatunków}
\Clue{30}{}{zgoda pomiędzy dwiema lub więcej stronami, mająca skutek prawny}
\Clue{31}{}{wzajemne popieranie się ludzi związanych pokrewieństwem, zażyłością lub interesami, dla osiągnięcia pozycji społecznej lub korzyści materialnych}
\Clue{33}{}{jednostka organizacyjna w krabonaryzmie}
\Clue{34}{}{część wozu, na której zawieszone są orczyki}
\Clue{35}{}{coś, co nie istnieje relanie, wymysł czyjegoś umysłu}
\Clue{36}{}{różnica wysokości pomiędzy dwoma dźwiękami}
\Clue{37}{}{malarka (1865-1940) od 1898 r. w Paryżu; przedstawicielka polskiego impresjonizmu, wybitna portrecistka}
\Clue{38}{}{rodzaj mocnego piwa dolnej fermentacji warzonego ze słodu jęczmiennego}
\Clue{39}{}{uchwyt umożliwiający przesuwanie się w nim przewodu elektrycznego}\end{PuzzleClues}

\begin{PuzzleClues}{\textbf{Pionowe}\\}\Clue{1}{}{lekkoatletka włoska, wicemistrzyni olimpijska z Atlanty w chodzie na 10 km}
\Clue{2}{}{substancja naturalna lub syntetyczna używana w chemioterapii nowotworów, działająca toksycznie na komórki nowotworowe}
\Clue{3}{}{przyrząd używany do mierzenia i określania położenia i ruchów ciał niebieskich}
\Clue{4}{}{ciało cechujące się paramagnetyzmem; umieszczone w zewnętrznym polu magnetycznym wykazuje się namagnesowaniem z kierunkiem zgodnym z kierunkiem pola zewnętrznego, co powoduje, że jest ono przyciągane przez magnes}
\Clue{5}{}{były chłopak, partner, mąż; rzeczownik rodzaju męskiego, nieodmienny}
\Clue{6}{}{zjawisko polegające na niepełnym lub opóźnionym uwzględnianiu inflacji w określaniu realnych zmian wartości i koncentrowaniu się na wartościach nominalnych a nie wielkościach realnych}
\Clue{8}{}{poplecznik lorda Voldemorta, czarodziej z cyklu książek o Harrym Potterze, który zalicza siebie do czarodziejów czystej krwi i nie ma szacunku dla mugoli ani dla innych czarodziejów, którzy żyją w zgodzie ze światem niemagicznym}
\Clue{9}{}{gwałtowny wyrzut treści pokarmowej na zewnątrz z żołądka (bądź z żołądka i jelit) poprzez przełyk i jamę ustną, w wyniku silnych skurczów mięśni brzucha, przepony oraz klatki piersiowej}
\Clue{10}{}{telewizja, dział telekomunikacji zajmujący się przekazywaniem ruchomego obrazu oraz dźwięku na odległość}
\Clue{12}{}{czeskie danie w postaci rolad z duszonej wołowiny}
\Clue{14}{}{waleń z podrzędu fiszbinowców; długość do 18 m; ceniony w połowach}
\Clue{18}{}{informacja, która jest błaha, nieuporządkowana, często chaotyczna}
\Clue{19}{}{człowiek oceniany bardzo negatywnie, wyzwisko}
\Clue{20}{}{trzeci dzień (najczęściej bieżącego lub przyszłego) miesiąca}
\Clue{23}{}{szkolna drobna ocena pozytywna lub podniesienie jej o pół stopnia}
\Clue{26}{}{obraz przedstawiający głowę Chrystusa (bez korony cierniowej) na chuście podtrzymywanej zwykle przez 2 anioły}
\Clue{28}{}{seria radzieckich trzyosobowych statków kosmicznych}
\Clue{32}{}{miasto w Hiszpanii (Stara Kastylia) u podnóża Gór Kastylijskich, ośrodek administracyjny prowincji Avila}\end{PuzzleClues}\newpage\section*{Krzyżówka 50}

\noindent\begin{Puzzle}{20}{33}|*	|*	|*	|*	|*	|*	|*	|*	|*	|*	|*	|*	|*	|*	|*	|*	|*	|*	|*	|*	|[1][S]\darr	|.
|*	|*	|*	|*	|*	|*	|*	|*	|*	|*	|*	|*	|*	|*	|*	|[2][S]\drarr	|d	|n	|o	|*	|k	|.
|*	|*	|*	|*	|*	|*	|*	|*	|*	|*	|*	|*	|*	|*	|*	|p	|[3][S]\darr	|*	|*	|[4][S]\darr	|u	|.
|*	|[5][S]\drarr	|d	|z	|i	|k	|o	|w	|o	|*	|*	|*	|*	|*	|*	|a	|p	|*	|*	|t	|c	|.
|*	|g	|*	|*	|*	|*	|*	|[6][S]\drarr	|s	|e	|r	|a	|p	|h	|i	|n	|e	|*	|*	|o	|[][,]{ }	|.
|*	|o	|*	|*	|*	|[7][S]\rarr	|t	|u	|ń	|c	|z	|y	|k	|*	|*	|e	|r	|*	|*	|l	|z	|.
|[8][S]\drarr	|m	|e	|t	|a	|t	|e	|k	|s	|t	|*	|*	|*	|*	|*	|w	|a	|*	|*	|k	|a	|.
|e	|o	|*	|[9][S]\drarr	|o	|d	|s	|ł	|o	|n	|i	|e	|n	|i	|e	|*	|h	|*	|*	|m	|n	|.
|t	|n	|[10][S]\darr	|k	|*	|[11][S]\rarr	|l	|a	|w	|e	|n	|t	|e	|r	|z	|*	|i	|*	|*	|i	|i	|.
|e	|*	|e	|o	|[12][S]\rarr	|j	|o	|d	|ł	|a	|[][,]{ }	|g	|ó	|r	|s	|k	|a	|*	|*	|c	|s	|.
|r	|[13][S]\drarr	|g	|r	|o	|c	|h	|[][,]{ }	|w	|ł	|o	|s	|k	|i	|*	|*	|*	|*	|*	|z	|k	|.
|*	|s	|z	|m	|*	|*	|[14][S]\rarr	|k	|a	|r	|y	|j	|c	|z	|y	|k	|*	|*	|*	|a	|a	|.
|*	|i	|y	|o	|*	|[15][S]\rarr	|g	|a	|l	|i	|c	|y	|j	|s	|k	|o	|ś	|ć	|*	|n	|r	|.
|*	|a	|s	|r	|*	|[16][S]\rarr	|k	|l	|a	|s	|a	|*	|*	|*	|*	|*	|*	|*	|*	|i	|i	|.
|*	|t	|t	|a	|[17][S]\rarr	|d	|e	|k	|a	|d	|e	|n	|t	|*	|[18][S]\drarr	|z	|r	|z	|y	|n	|*	|.
|*	|k	|e	|n	|*	|*	|[19][S]\darr	|u	|*	|[20][S]\drarr	|ś	|w	|i	|e	|t	|l	|i	|c	|a	|*	|*	|.
|*	|a	|n	|[][,]{ }	|*	|[21][S]\darr	|t	|l	|*	|h	|*	|*	|*	|*	|e	|[22][S]\darr	|*	|*	|*	|*	|*	|.
|*	|[][,]{ }	|c	|b	|[23][S]\drarr	|d	|i	|a	|l	|i	|z	|a	|t	|*	|s	|p	|*	|*	|*	|*	|*	|.
|*	|p	|j	|i	|w	|e	|g	|c	|*	|p	|*	|*	|*	|[24][S]\drarr	|t	|i	|l	|d	|e	|n	|*	|.
|*	|o	|a	|a	|o	|w	|a	|y	|*	|o	|*	|*	|[25][S]\darr	|ł	|[][,]{ }	|ę	|[26][S]\darr	|*	|*	|*	|*	|.
|*	|j	|l	|ł	|d	|i	|*	|j	|[27][S]\darr	|c	|*	|*	|m	|o	|k	|t	|w	|[28][S]\darr	|*	|*	|*	|.
|*	|ę	|i	|o	|a	|a	|[29][S]\drarr	|n	|i	|e	|z	|ł	|o	|ż	|o	|n	|o	|ś	|ć	|*	|*	|.
|*	|c	|s	|r	|*	|c	|s	|y	|s	|n	|*	|[30][S]\darr	|n	|n	|m	|a	|l	|w	|*	|*	|*	|.
|*	|i	|t	|z	|[31][S]\darr	|j	|a	|*	|l	|t	|[32][S]\darr	|m	|t	|i	|e	|s	|n	|i	|*	|*	|*	|.
|*	|o	|a	|y	|c	|a	|t	|*	|a	|a	|k	|a	|a	|c	|t	|t	|y	|a	|*	|*	|*	|.
|*	|w	|*	|t	|u	|*	|y	|*	|n	|u	|a	|d	|ż	|a	|o	|o	|[][,]{ }	|t	|*	|*	|*	|.
|*	|a	|[33][S]\darr	|n	|r	|*	|s	|*	|d	|r	|r	|d	|y	|*	|w	|l	|z	|[][,]{ }	|*	|*	|*	|.
|*	|*	|p	|y	|t	|*	|f	|*	|i	|*	|t	|e	|s	|*	|y	|a	|a	|d	|*	|*	|*	|.
|*	|*	|o	|*	|i	|[34][S]\rarr	|a	|g	|a	|p	|a	|n	|t	|*	|*	|t	|w	|y	|*	|*	|*	|.
|*	|[35][S]\rarr	|p	|u	|s	|z	|k	|a	|*	|*	|*	|*	|k	|*	|*	|k	|ó	|s	|*	|*	|*	|.
|*	|*	|*	|*	|s	|*	|c	|[36][S]\rarr	|p	|r	|a	|s	|a	|*	|*	|a	|d	|k	|*	|*	|*	|.
|*	|*	|*	|*	|*	|[37][S]\rarr	|j	|u	|d	|i	|n	|a	|*	|*	|*	|*	|*	|u	|*	|*	|*	|.
|*	|*	|[38][S]\rarr	|n	|e	|g	|a	|t	|o	|n	|*	|*	|*	|*	|*	|*	|*	|*	|*	|*	|*	|.
|[39][S]\rarr	|o	|b	|ł	|ę	|k	|*	|*	|*	|*	|*	|*	|*	|*	|*	|*	|*	|*	|*	|*	|*	|.\end{Puzzle}

\newpage

\begin{PuzzleClues}{\textbf{Poziome}\\}\Clue{2}{}{najniższa powierzchnia; zagłębienie w terenie lub w zbiorniku wodnym}
\Clue{5}{}{wieś w Polsce położona w województwie kujawsko-pomorskim, w powiecie toruńskim, w gminie Obrowo}
\Clue{6}{}{inst. klawiszowy, prototyp fisharmonii}
\Clue{7}{}{mięso z tuńczyka}
\Clue{8}{}{tekst (wypowiedź) lub element tekstu, który odnosi się do danego lub innego tekstu (innej wypowiedzi), wnosząc dodatkową informację o jego organizacji, znaczeniu, celu, genezie itp}
\Clue{9}{}{czynienie widocznym; odkrywanie; pokazywanie}
\Clue{11}{}{umywalnia kapłańska w zakrystii}
\Clue{12}{}{Abies lasiocarpa - gatunek z rodziny sosnowatych}
\Clue{13}{}{warzywo strączkowe; ziarno ciecierzycy pospolitej}
\Clue{14}{}{przedstawiciel narodu zamieszkującego Karię}
\Clue{15}{}{zestaw cech typowych dla Galicji i mieszkańca Galicji - regionu w Hiszpanii}
\Clue{16}{}{kategoria przedmiotów wyróżniających się ze względu na taką samą jakość}
\Clue{17}{}{rodzaj pesymisty}
\Clue{18}{}{produkt uboczny cięcia drewna, odpad przemysłu drzewiarskiego, liche deski}
\Clue{20}{}{pomieszczenie w szkole, pełniące funkcję świetlicy}
\Clue{23}{}{preparat służący do dializy}
\Clue{24}{}{tenisista ameryk., zwycięzca Wimbledonu w 1920 i 21 r}
\Clue{29}{}{prostota i nieskomplikowanie}
\Clue{34}{}{Agapanthus - rodzaj roślin z rodziny amarylkowatych (Amaryllidaceae)}
\Clue{35}{}{blaszany, walcowaty pojemnik, służący do przechowywania różnych przedmiotów, głównie żywności i płynów}
\Clue{36}{}{maszyna rolnicza do zbierania siana, słomy lub zielonki z wałów oraz prasowania podebranego materiału i wiązania go w bele lub wiązki}
\Clue{37}{}{pianistka radziecka (1899-1970); wybitna wykonawczyni utworów Bacha, Mozarta, Szostakowicza, Beethovena}
\Clue{38}{}{trwała cząstka elementarna (lepton) będąca jednym z elementów atomu, mająca ujemny ładunek elektryczny}
\Clue{39}{}{pręt wykonany ze stali, który służy do mocowania poprzecznej belki z izolatorami do słupa}\end{PuzzleClues}

\begin{PuzzleClues}{\textbf{Pionowe}\\}\Clue{1}{}{rasa kuców odkryta w Leh i Laddakh w obszarze Dżammu i Kaszmiru (północne Indie); zagrożona przez krzyżowanie z nieokreślonymi kucami, dlatego Departament Hodowli Zwierząt stanu Dżammu i Kaszmir rozpoczął program ochrony w gospodarstwie w Leh}
\Clue{2}{}{płytkie metalowe naczynie do gotowania itp.; obecnie ma zastosowanie w warzelniach soli}
\Clue{3}{}{ur. w 1947 r., pianista amerykański, uznany wykonawca utworów W.A. Mozarta}
\Clue{4}{}{mieszkaniec Tolkmicka}
\Clue{5}{}{barwny, hałaśliwy pochód}
\Clue{6}{}{klasyfikacja kosztów prowadzenia działalności gospodarczej w tzw. pozycje kalkulacyjne niezbędne do obliczenia kosztu wytworzenia produktu: koszty pośrednie i koszty bezpośrednie}
\Clue{8}{}{substancja wypełniająca wszechświat; przez filozofów greckich określany jako pierwotna materia, zdaniem XIX-wiecznych filozofów eter stanowił ośrodek, w którym rozchodzą się fale świetlne}
\Clue{9}{}{Phalacrocorax neglectus - gatunek ptaka z rodziny kormoranów (Phalacrocoracidae); zamieszkuje tereny południowo-zachodniej Afryki}
\Clue{10}{}{przedstawiciel egzystencjalizmu w filozofii}
\Clue{13}{}{zestaw pojęć używanych w zakresie jakiejś branży, nauki, dziedziny wraz z regułami ich użycia}
\Clue{18}{}{test służący badaniu uszkodzeń DNA oraz umożliwiający znalezienie sposobu na ich uniknięcie}
\Clue{19}{}{azjatycki ptak z rodziny dzięciołowatych}
\Clue{20}{}{istota posiadająca korpus i kończyny konia oraz głowę i tors mężczyzny}
\Clue{21}{}{w żeglarstwie: zboczenie z drogi, odchylenie od właściwego kierunku, błądzenie}
\Clue{22}{}{dziewczyna piętnastoletnia}
\Clue{23}{}{woda pitna}
\Clue{24}{}{sypialnia - pomieszczenie przeznaczone do spania}
\Clue{25}{}{monterka, kobieta pracująca przy montażu czegoś, np. maszyn, urządzeń}
\Clue{26}{}{profesja, której wykonywanie uprawnia do rozliczenia się za pomocą uproszczonych form opodatkowania (karty podatkowej lub ryczałtu)}
\Clue{27}{}{wyspa pochodzenia wulkanicznego na Oceanie Atlantyckim, w całości zajmuje ją państwo Islandia}
\Clue{28}{}{fikcyjny świat w kształcie dysku, w którym dzieje się akcja wieloksiągu fantasy autorstwa Terry'ego Pratchetta}
\Clue{29}{}{uczucie, poczucie zadowolenia, połączone często z odczuwaniem dumy, radości, spełnienia}
\Clue{30}{}{jezioro w Panamie, z jeziora wypływa rzeka Chagres}
\Clue{31}{}{amerykański konstruktor lotniczy i przemysłowiec (1878-1930)}
\Clue{32}{}{rodzaj dokumentu, środek płatniczy}
\Clue{33}{}{duchowny prawosławny i grekokatolicki, odpowiednik katolickiego księdza}\end{PuzzleClues}\newpage\section*{Krzyżówka 51}

\noindent\begin{Puzzle}{20}{32}|*	|[1][S]\drarr	|p	|r	|z	|e	|k	|r	|y	|c	|i	|e	|*	|*	|[2][S]\darr	|[3][S]\darr	|[4][S]\darr	|[5][S]\drarr	|y	|*	|*	|.
|[6][S]\rarr	|z	|a	|u	|r	|o	|p	|o	|d	|*	|*	|*	|*	|*	|c	|h	|r	|p	|*	|*	|*	|.
|[7][S]\rarr	|i	|z	|o	|p	|r	|e	|n	|a	|l	|i	|n	|a	|*	|h	|e	|i	|o	|*	|[8][S]\darr	|*	|.
|[9][S]\rarr	|e	|d	|y	|k	|u	|ł	|a	|*	|*	|*	|*	|[10][S]\drarr	|w	|a	|l	|c	|z	|y	|k	|*	|.
|*	|m	|*	|*	|*	|[11][S]\drarr	|z	|e	|f	|i	|r	|*	|t	|*	|m	|*	|h	|y	|*	|a	|*	|.
|*	|i	|[12][S]\rarr	|k	|i	|b	|i	|t	|k	|a	|*	|*	|w	|*	|*	|[13][S]\darr	|a	|t	|[14][S]\darr	|t	|*	|.
|*	|o	|*	|[15][S]\rarr	|f	|r	|y	|g	|i	|j	|s	|k	|i	|*	|[16][S]\rarr	|k	|r	|y	|p	|a	|*	|.
|[17][S]\drarr	|p	|a	|p	|r	|o	|c	|k	|i	|*	|[18][S]\drarr	|d	|e	|j	|*	|o	|d	|w	|e	|f	|*	|.
|b	|ł	|[19][S]\rarr	|ż	|y	|d	|*	|[20][S]\darr	|*	|[21][S]\darr	|g	|*	|r	|*	|*	|l	|*	|*	|r	|r	|*	|.
|r	|ó	|[22][S]\drarr	|m	|o	|n	|s	|t	|e	|r	|a	|[][,]{ }	|d	|z	|i	|u	|r	|a	|w	|a	|*	|.
|a	|d	|h	|*	|*	|i	|[23][S]\darr	|u	|*	|e	|c	|*	|z	|*	|[24][S]\darr	|s	|*	|*	|e	|k	|*	|.
|h	|*	|i	|*	|[25][S]\darr	|c	|r	|r	|[26][S]\darr	|i	|e	|*	|e	|*	|w	|z	|*	|[27][S]\darr	|r	|c	|*	|.
|m	|*	|s	|*	|p	|z	|a	|k	|s	|d	|k	|*	|n	|*	|o	|k	|*	|p	|s	|i	|*	|.
|s	|*	|t	|[28][S]\darr	|r	|a	|p	|o	|t	|*	|[][,]{ }	|*	|i	|[29][S]\drarr	|d	|o	|b	|r	|y	|*	|*	|.
|*	|[30][S]\darr	|o	|p	|o	|n	|t	|s	|r	|[31][S]\darr	|b	|*	|e	|d	|o	|w	|[32][S]\darr	|a	|j	|*	|*	|.
|[33][S]\drarr	|g	|r	|e	|c	|k	|i	|*	|a	|p	|r	|*	|[][,]{ }	|w	|p	|i	|p	|w	|n	|*	|*	|.
|k	|ł	|y	|t	|h	|a	|s	|[34][S]\darr	|d	|a	|u	|*	|p	|u	|r	|a	|ł	|o	|o	|*	|*	|.
|o	|o	|c	|r	|r	|*	|*	|f	|e	|s	|n	|*	|t	|k	|z	|n	|a	|[][,]{ }	|ś	|*	|*	|.
|p	|w	|z	|e	|o	|[35][S]\drarr	|v	|o	|l	|t	|a	|*	|o	|ó	|e	|i	|c	|o	|ć	|*	|*	|.
|r	|o	|k	|l	|n	|p	|*	|l	|l	|o	|t	|*	|l	|ł	|p	|n	|a	|b	|*	|*	|*	|.
|z	|c	|a	|[][,]{ }	|i	|r	|*	|a	|a	|f	|n	|*	|e	|k	|u	|*	|[][,]{ }	|y	|*	|*	|*	|.
|y	|i	|*	|b	|z	|z	|*	|*	|*	|o	|y	|*	|m	|a	|s	|[36][S]\darr	|m	|w	|[37][S]\darr	|*	|*	|.
|w	|s	|*	|i	|m	|y	|*	|*	|[38][S]\darr	|r	|*	|*	|e	|*	|z	|d	|i	|a	|w	|*	|*	|.
|n	|o	|*	|a	|*	|d	|*	|[39][S]\drarr	|c	|i	|n	|q	|u	|e	|c	|e	|n	|t	|o	|*	|*	|.
|i	|w	|[40][S]\darr	|ł	|[41][S]\darr	|z	|*	|k	|h	|u	|*	|*	|s	|[42][S]\darr	|z	|r	|i	|e	|s	|*	|*	|.
|c	|a	|s	|o	|g	|i	|*	|o	|ó	|m	|*	|*	|z	|p	|a	|p	|m	|l	|k	|*	|*	|.
|a	|t	|z	|s	|o	|a	|*	|t	|r	|*	|*	|*	|a	|l	|l	|*	|a	|s	|ó	|*	|*	|.
|*	|e	|a	|z	|r	|ł	|*	|w	|*	|*	|[43][S]\darr	|*	|*	|o	|n	|[44][S]\darr	|l	|k	|w	|*	|*	|.
|*	|*	|m	|y	|s	|*	|[45][S]\darr	|i	|*	|*	|m	|*	|[46][S]\rarr	|n	|o	|r	|n	|i	|k	|*	|*	|.
|*	|[47][S]\rarr	|b	|i	|e	|g	|a	|c	|z	|*	|a	|*	|*	|*	|ś	|ę	|a	|e	|a	|*	|*	|.
|*	|*	|o	|*	|t	|[48][S]\rarr	|p	|a	|j	|a	|c	|y	|k	|*	|ć	|k	|*	|*	|*	|*	|*	|.
|*	|*	|*	|*	|*	|*	|t	|*	|[49][S]\rarr	|l	|a	|y	|*	|*	|*	|a	|*	|*	|*	|*	|*	|.
|*	|*	|*	|*	|*	|*	|*	|*	|*	|*	|*	|*	|*	|*	|*	|*	|*	|*	|*	|*	|*	|.\end{Puzzle}

\newpage

\begin{PuzzleClues}{\textbf{Poziome}\\}\Clue{1}{}{rzecz, która przekrywa, tzn. przysłania coś, nachodząc na to}
\Clue{5}{}{w chemii: symbol itru}
\Clue{6}{}{dinozaur z rzędu dinozaurów gadziomiednicznych; zwierzę lądowe i roślinożerne}
\Clue{7}{}{wielofunkcyjny organiczny związek chemiczny z grupy katecholamin; lek o działaniu sympatykomimetycznym, działający najsilniej ze wszystkich katecholamin na receptory adrenergiczne}
\Clue{9}{}{niewielki motyw dekoracyjny w kształcie miniaturowej świątynki lub kapliczki, często z baldachimem, umieszczany w tympanonie, portalu lub jako sterczyna, zwieńczenie architektoniczne, niekiedy używany również jako dar fundacyjny dla obiektu sakralnego}
\Clue{10}{}{krótki walc, forma muzyczna}
\Clue{11}{}{ciepły łagodny wiatr}
\Clue{12}{}{kryty wóz gospodarski, czterokołowy, jednokonny, używany w Rosji; w okresie caratu również stosowany jako karetka do przewozu więźniów}
\Clue{15}{}{wymarły indoeuropejski język Frygów, używany w starożytnej Anatolii}
\Clue{16}{}{pejoratywnie o starej łodzi}
\Clue{17}{}{ur. w  1919 r.,  śpiewak (tenor); solista Opery Śląskiej i Teatru Wielkiego w Warszawie}
\Clue{18}{}{miasto w Rumunii, okręg Kluż; przemysł drzewny}
\Clue{19}{}{przedstawiciel ludu semickiego zamieszkującego w starożytności Palestynę}
\Clue{22}{}{jadalny, smaczny owoc (jagoda) rośliny o tej samej nazwie}
\Clue{29}{}{0.7 - skrócona forma wykrzyknieniadzień dobry używanego na powitanie}
\Clue{33}{}{przedmiot szkolny lub uczony w ramach kursu, na którym opanowuje się podstawy języka greckiego}
\Clue{35}{}{fizyk i fizjolog włoski (1745-1827); wynalazł elektrofor, odkrył gaz błotny, zbudował kondensator i ogniwo galwaniczne}
\Clue{39}{}{model miejskiego samochodu osobowego produkowanego w polskiej fabryce Fiata w Tychach od czerwca 1991 do września 1998 roku}
\Clue{46}{}{POLNIK - drobny gryzoń żerujący częściowo pod ziemią groźny szkodnik upraw}
\Clue{47}{}{duży chrząszcz drapieżny, pożyteczny, chroniony}
\Clue{48}{}{często w liczbie mnogiej: ćwiczenie, które polega na podskakiwaniu i wymachiwaniu rękami i nogami w taki sposób, że na zmianę albo ręce łączą się nad głową, a nogi są w rozkroku, albo nogi są połączone, a ręce przylegają do tułowia po bokach; ruch ćwiczącego przypomina sposób, w jaki porusza się drewniany pajacyk - marionetka}
\Clue{49}{}{pedagog niemiecki (1862-1926); pionier pedagogii eksperymentalnej w Niemczech}\end{PuzzleClues}

\begin{PuzzleClues}{\textbf{Pionowe}\\}\Clue{1}{}{płody rolne}
\Clue{2}{}{pogardliwie: chłop}
\Clue{3}{}{najlżejszy pierwiastek z grupy VIII układu okresowego pierwiastków}
\Clue{4}{}{Maurice, hokeista kanadyjski, ośmiokrotny zdobywca Pucharu Stanleya w latach 1946-59}
\Clue{5}{}{rodzaj małych organów w XI 1/XI 11 w}
\Clue{8}{}{starożytna jazda perska mająca konie osłonięte pancerzem}
\Clue{10}{}{twierdzenie geometryczne, wedle którego suma iloczynów długości przeciwległych sobie boków czworokąta ABCD, na którym można opisać okrąg, równa się iloczynowi długości jego przekątnych}
\Clue{11}{}{mieszkanka Brodnicy}
\Clue{13}{}{mieszkaniec Koluszek}
\Clue{14}{}{to, że ktoś ma skłonność do perwersji, przejawia zachowania perwersyjne}
\Clue{17}{}{utwór Brahmsa odtwarzany przez muzyka lub grupę muzyków}
\Clue{18}{}{gacek wielkouch, Plecotus auritus - gatunek ssaka z rzędu nietoperzy, który wraz z gackiem szarym stanowi trudną do rozróżnienia parę gatunków - do około roku 1960 gatunków tych nie rozróżniano, używając polskiej nazwy gacek wielkouch; występuje w całej Europie z wyjątkiem północnych i południowych jej krańców, w Polsce spotykany na terenie całego kraju}
\Clue{20}{}{strzelec we francuskich wojskach kolonialnych}
\Clue{21}{}{filozof szkocki (1710-96); twórca tzw. filozofii zdrowego rozsądku}
\Clue{22}{}{nauczycielka historii w szkole}
\Clue{23}{}{ur. w 1936 r., śpiewak operowy pochodzenia greckiego, solista opery łódzkiej}
\Clue{24}{}{cecha czegoś, co jest wodoprzepuszczalne - przepuszcza wodę}
\Clue{25}{}{antydatowanie - opatrzenie wydarzenia datą wcześniejszą niż jemu właściwa}
\Clue{26}{}{kompozytor włoski (1645-1682); przedstawiciel szkoły weneckiej; oratoria, kantaty, motety, madrygały, opery, utwory instrumentalne}
\Clue{27}{}{konstytucyjnie zagwarantowane prawo obywatela danego państwa, którego celem jest ochrona jego interesów; najważniejszymi z praw obywatelskich są: prawo do pracy, prawo do nauki i prawo wyborcze}
\Clue{28}{}{Pterodroma cervicalis - gatunek ptaka z rodziny burzykowatych (Procellariidae)}
\Clue{29}{}{BIEDA, BIEDKA, BIGA; pojazd o dwóch kołach}
\Clue{30}{}{Cephalotaxaceae - rodzina zaliczona w systemie Reveala do monotypowego rzędu głowocisowców (Cephalotaxales), w innych systemach zaliczana zwykle do rzędu cisowców (Taxales)}
\Clue{31}{}{w kościołach wczesnochrześcijańskich: pomieszczenie w prezbiterium przylegające symetrycznie do boków apsydy; stanowi zakrystię, zazwyczaj występuje podwójnie, po obu stronach ołtarza jako diakonikon i prothesis}
\Clue{32}{}{najniższa z możliwych płac, najczęściej ustalana na drodze prawnej przez dane państwo}
\Clue{33}{}{miasto w województwie świętokrzyskim, w powiecie sandomierskim, nad rzeką Koprzywianką, siedziba gminy miejsko-wiejskiej Koprzywnica}
\Clue{34}{}{duża skrzynia drewniana przeznaczona do gaszenia wapna palonego}
\Clue{35}{}{określona ilość czegoś, która przysługuje komuś}
\Clue{36}{}{osoba, która ma wyraz twarzy nazywany derpem, ma twarz, która zdecydowanie nie tchnie inteligencją; twarzowiec, przymuł}
\Clue{37}{}{miękkie, silnie unerwione nabrzmienie skórne u nasady dzioba niektórych ptaków}
\Clue{38}{}{w wielkich salach - balkon dla orkiestry}
\Clue{39}{}{ciężki metalowy hak, który wbija się w dno i utrzymuje statek w miejscu}
\Clue{40}{}{podziemny zbiornik zazwyczaj bezodpływowy, wykonany np. z betonu, do którego odprowadzane są ścieki z domowych urządzeń kanalizacyjnych}
\Clue{41}{}{element ubrania - osobny lub połączony z całością - kształtem nawiązujący do gorsetu bieliźnianego, opinający i lekko modelujący sylwetkę}
\Clue{42}{}{produkt uprawy roślin}
\Clue{43}{}{staropolska miara objętości zboża wynosząca 4 korce}
\Clue{44}{}{człowieku jako wykonawcy jakiejś pracy}
\Clue{45}{}{astronauta amerykański biorący udział w wyprawie Atlantisa w 1991 r}\end{PuzzleClues}\newpage\section*{Krzyżówka 52}

\noindent\begin{Puzzle}{22}{29}|*	|[1][S]\drarr	|s	|i	|l	|n	|i	|k	|[][,]{ }	|b	|l	|i	|ź	|n	|i	|a	|c	|z	|y	|*	|*	|*	|*	|.
|[2][S]\rarr	|k	|o	|ż	|u	|c	|h	|*	|[3][S]\drarr	|e	|w	|o	|l	|u	|c	|j	|a	|*	|*	|*	|*	|*	|*	|.
|*	|o	|*	|*	|*	|*	|*	|[4][S]\rarr	|h	|e	|l	|g	|o	|l	|a	|n	|d	|z	|k	|a	|*	|*	|*	|.
|*	|p	|*	|*	|[5][S]\rarr	|k	|o	|l	|i	|z	|j	|a	|[][,]{ }	|d	|r	|o	|g	|o	|w	|a	|*	|*	|[6][S]\darr	|.
|*	|i	|*	|*	|[7][S]\rarr	|k	|o	|n	|s	|u	|l	|o	|s	|t	|w	|o	|*	|[8][S]\darr	|[9][S]\darr	|*	|*	|*	|f	|.
|*	|e	|[10][S]\rarr	|n	|i	|e	|r	|o	|z	|u	|m	|n	|o	|ś	|ć	|*	|*	|c	|n	|*	|*	|[11][S]\darr	|i	|.
|[12][S]\drarr	|c	|i	|e	|m	|n	|e	|[][,]{ }	|p	|i	|e	|c	|z	|y	|w	|o	|*	|i	|i	|*	|[13][S]\darr	|m	|l	|.
|j	|*	|*	|[14][S]\darr	|*	|*	|*	|*	|a	|[15][S]\drarr	|k	|o	|ś	|a	|*	|*	|*	|ą	|e	|*	|i	|a	|m	|.
|i	|*	|*	|c	|*	|*	|[16][S]\drarr	|d	|n	|o	|*	|[17][S]\darr	|*	|*	|*	|*	|*	|g	|d	|*	|r	|c	|[][,]{ }	|.
|a	|[18][S]\drarr	|f	|u	|r	|t	|h	|*	|k	|b	|[19][S]\darr	|m	|[20][S]\darr	|*	|*	|[21][S]\darr	|*	|n	|z	|*	|g	|i	|o	|.
|o	|k	|[22][S]\darr	|g	|[23][S]\darr	|*	|a	|*	|a	|s	|t	|s	|r	|*	|*	|w	|[24][S]\darr	|i	|i	|*	|a	|e	|b	|.
|*	|o	|l	|*	|ł	|*	|f	|*	|*	|a	|o	|z	|y	|*	|*	|a	|p	|k	|e	|*	|[][,]{ }	|r	|y	|.
|[25][S]\drarr	|l	|i	|g	|o	|l	|*	|[26][S]\darr	|[27][S]\rarr	|d	|r	|a	|b	|*	|*	|r	|o	|[][,]{ }	|l	|*	|k	|z	|c	|.
|j	|b	|n	|*	|d	|*	|[28][S]\darr	|t	|*	|a	|*	|ł	|o	|*	|*	|u	|m	|m	|n	|*	|u	|[][,]{ }	|z	|.
|e	|a	|i	|[29][S]\darr	|z	|*	|p	|a	|[30][S]\darr	|[][,]{ }	|[31][S]\darr	|*	|ł	|*	|*	|n	|p	|a	|y	|*	|t	|r	|a	|.
|d	|[][,]{ }	|a	|a	|i	|*	|e	|m	|s	|e	|w	|*	|ó	|*	|[32][S]\darr	|e	|a	|n	|[][,]{ }	|*	|n	|z	|j	|.
|w	|l	|*	|g	|k	|[33][S]\darr	|i	|t	|a	|t	|r	|*	|w	|[34][S]\darr	|p	|k	|[][,]{ }	|e	|k	|[35][S]\darr	|e	|e	|o	|.
|a	|u	|*	|r	|*	|t	|r	|e	|b	|a	|z	|*	|[][,]{ }	|p	|o	|[][,]{ }	|ł	|w	|i	|z	|r	|c	|w	|.
|b	|t	|*	|e	|[36][S]\darr	|r	|e	|n	|e	|t	|ó	|[37][S]\darr	|z	|r	|t	|k	|y	|r	|e	|w	|o	|z	|y	|.
|i	|o	|[38][S]\drarr	|k	|w	|a	|s	|[][,]{ }	|j	|o	|d	|o	|w	|o	|w	|o	|d	|o	|r	|o	|w	|y	|*	|.
|e	|w	|p	|s	|e	|j	|k	|ś	|c	|w	|[][,]{ }	|d	|y	|z	|i	|n	|k	|w	|o	|d	|a	|w	|*	|.
|*	|n	|e	|o	|t	|k	|i	|w	|z	|a	|n	|p	|c	|e	|e	|i	|o	|y	|w	|n	|t	|i	|*	|.
|*	|i	|d	|f	|e	|o	|a	|i	|y	|*	|a	|o	|z	|l	|r	|e	|w	|*	|c	|i	|a	|s	|*	|.
|*	|c	|a	|i	|r	|t	|*	|a	|k	|*	|[][,]{ }	|c	|a	|i	|d	|c	|a	|*	|a	|c	|*	|t	|*	|.
|*	|z	|ł	|l	|a	|k	|*	|t	|*	|*	|d	|z	|j	|t	|z	|z	|*	|*	|*	|z	|*	|a	|*	|.
|*	|a	|ó	|i	|n	|a	|*	|*	|*	|*	|u	|y	|n	|y	|e	|n	|*	|*	|*	|o	|*	|*	|*	|.
|*	|*	|w	|a	|*	|*	|*	|*	|*	|*	|p	|n	|y	|z	|n	|y	|*	|*	|*	|ś	|*	|*	|*	|.
|*	|*	|k	|*	|*	|[39][S]\rarr	|d	|r	|u	|g	|i	|e	|*	|m	|i	|*	|*	|*	|*	|ć	|*	|*	|*	|.
|*	|*	|a	|*	|*	|[40][S]\rarr	|s	|o	|u	|c	|e	|k	|*	|*	|e	|*	|*	|*	|*	|*	|*	|*	|*	|.
|*	|*	|*	|*	|*	|*	|*	|*	|*	|*	|*	|*	|*	|*	|*	|*	|*	|*	|*	|*	|*	|*	|*	|.\end{Puzzle}

\newpage

\begin{PuzzleClues}{\textbf{Poziome}\\}\Clue{1}{}{dominujący napęd parowozów składający się z dwóch cylindrów w układzie niesprzężonym}
\Clue{2}{}{powłoka roślinna tworząca się zwykle na zbiornikach wodnych lub terenach podmokłych}
\Clue{3}{}{trudna figura gimnastyczna, akrobatyczna lub taneczna}
\Clue{4}{}{zatoka Morza Północnego u wybrzeży Niemiec}
\Clue{5}{}{zdarzenie drogowe, w którym uszkodzeniu uległo jedynie mienie; potocznie utożsamiana z wypadkiem drogowym}
\Clue{7}{}{konsul i jego żona}
\Clue{10}{}{to, że ktoś jest nierozumny, nie posiada rozumu, jest niemądry}
\Clue{12}{}{pieczywo o ciemnej ośródce, pieczone z ciemnej mąki}
\Clue{15}{}{węgierski kompozytor i pianista (1897-1984); opery, balety, pantomimy, oratoria, utwory symfoniczne}
\Clue{16}{}{pogardliwie: kompletny ignorant, zero; stosowane także w odniesieniu do osoby zgniłej moralnie}
\Clue{18}{}{miasto w Niemczech (Bawaria) w zespole miejskim Norymbergii, port nad kanałem Ren-Men-Dunaj}
\Clue{25}{}{popularna odmiana uprawna jabłoni domowej}
\Clue{27}{}{wysoki słup podtrzymujący rusztowanie}
\Clue{38}{}{roztwór wodny jodowodoru}
\Clue{39}{}{drugie danie, potrawa podawana (zazwyczaj) po zupie}
\Clue{40}{}{czeski malarz, grafik (1915-82); obrazował życie miasta}\end{PuzzleClues}

\begin{PuzzleClues}{\textbf{Pionowe}\\}\Clue{1}{}{wzniesienie terenu, często utworzone sztucznie przez człowieka (choć niekoniecznie)}
\Clue{3}{}{mieszkanka Hiszpanii, kobieta pochodzenia hiszpańskiego}
\Clue{6}{}{gatunek filmowy traktujący o sposobach postępowania, charakterystycznych zachowaniach społecznych, zwyczajach, nawykach, przyzwyczajeniach lub codzienności}
\Clue{8}{}{pojazd silnikowy o małej mocy przeznaczony do wykonywania niewielkich prac manewrowych i ciągnięcia lekkich pociągów}
\Clue{9}{}{kierowca, który rzadko prowadzi samochód}
\Clue{11}{}{macierz prostokątna, której elementami są liczby rzeczywiste ułożone w prostokąt}
\Clue{12}{}{jednostka zdawkowa w Chińskiej Republiki Ludowej; 1/10 yuana, 10 fenów}
\Clue{13}{}{Cotoneaster tomentosus - gatunek krzewu z rodziny różowatych}
\Clue{14}{}{zaprzęg konny składających się z sześciu (rzadziej czterech) koni dobranych maścią, wzrostem lub innymi cechami}
\Clue{15}{}{liczba etatów przewidzianych dla danej jednostki organizacyjnej}
\Clue{16}{}{zatoka morska między wybrzeżem a mierzeją}
\Clue{17}{}{księga liturgiczna w Kościele katolickim zawierająca tekst mszy i przepisy ich odprawiania}
\Clue{18}{}{narzędzie służące do lutowania}
\Clue{19}{}{krzywa, droga zakreślana w przestrzeni przez poruszające się ciało}
\Clue{20}{}{Pandion haliaetus haliaetus - nominatywny podgatunek ptaka wyróżniony w obrębie gatunku rybołów zwyczajny (Pandion haliaetus); występuje w niemal całej Europie, w południowej i środkowej Azji oraz w północnej Afryce. }
\Clue{21}{}{w logice wniosek wypływający z danego faktu}
\Clue{22}{}{sylwetka, figura człowieka; często: zgrabna sylwetka, szczupła figura}
\Clue{23}{}{głowonóg spokrewniony z kopalnymi amonitami}
\Clue{24}{}{część układu krwionośnego odpowiedzialna za pompowanie krwi od kończyn w górę, do serca; działa, kiedy organizm jest w ruchu}
\Clue{25}{}{ekskluzywna odzież z cienkich materiałów (takich jak jedwab)}
\Clue{26}{}{miejsce przebywania człowieka po śmierci; określenie przenośne}
\Clue{28}{}{amerykański krzew lub drzewko z rodziny kaktusowatych, pędy ulistnione, owoce niektórych gatunków jadalne}
\Clue{29}{}{odmiana wojeryzmu}
\Clue{30}{}{członek szczepu zamieszkującego w starożytności Sabę, państwo przedmuzułmańskie}
\Clue{31}{}{coś, co stanowi uporczywą przeszkodę}
\Clue{32}{}{urzędowy dokument, który coś potwierdza, poświadcza, o czymś zaświadcza}
\Clue{33}{}{kobieta, która dużo, szybko, zawzięcie mówi, gada bez sensu i nieustannie; papla, pleciuga}
\Clue{34}{}{nawracanie innych na swoją wiarę}
\Clue{35}{}{to, że coś jest zwodnicze - daje płonne nadzieje}
\Clue{36}{}{wojskowy, który długo służył i jest bardzo doświadczony (np. weterani wojny w Wietnamie)}
\Clue{37}{}{przerwa w czasie wykonywanych czynności w celu wyeliminowania uczucia zmęczenia}
\Clue{38}{}{maszyna drukarska do drukowania tekstów niedużego formatu, napędzana za pomocą pedałów}\end{PuzzleClues}\newpage\section*{Krzyżówka 53}

\noindent\begin{Puzzle}{19}{33}|*	|*	|[1][S]\darr	|[2][S]\darr	|[3][S]\drarr	|w	|o	|j	|*	|[4][S]\drarr	|ł	|a	|d	|o	|w	|a	|r	|k	|a	|*	|.
|*	|[5][S]\darr	|b	|d	|m	|*	|[6][S]\rarr	|d	|e	|s	|z	|c	|z	|ó	|w	|k	|a	|*	|*	|*	|.
|[7][S]\drarr	|p	|e	|z	|e	|t	|p	|e	|r	|o	|w	|i	|e	|c	|*	|*	|*	|*	|*	|*	|.
|a	|r	|z	|w	|t	|[8][S]\rarr	|w	|ł	|o	|c	|h	|a	|c	|z	|*	|*	|*	|*	|*	|*	|.
|k	|z	|t	|o	|y	|*	|[9][S]\rarr	|m	|a	|j	|e	|r	|a	|n	|e	|k	|*	|*	|*	|*	|.
|t	|e	|o	|n	|z	|*	|*	|[10][S]\rarr	|f	|a	|ł	|s	|z	|y	|w	|o	|ś	|ć	|*	|*	|.
|y	|ż	|r	|e	|a	|*	|[11][S]\rarr	|k	|o	|l	|a	|g	|e	|n	|o	|z	|a	|*	|*	|*	|.
|w	|u	|b	|c	|c	|*	|*	|*	|*	|i	|*	|[12][S]\darr	|*	|*	|*	|[13][S]\darr	|[14][S]\darr	|*	|*	|*	|.
|a	|w	|i	|z	|j	|*	|[15][S]\rarr	|p	|r	|z	|e	|c	|i	|n	|a	|c	|z	|*	|*	|*	|.
|[][,]{ }	|a	|k	|n	|a	|*	|*	|*	|*	|m	|*	|h	|*	|*	|*	|z	|n	|[16][S]\darr	|*	|*	|.
|t	|c	|[][,]{ }	|i	|*	|*	|*	|*	|*	|[][,]{ }	|*	|l	|*	|*	|*	|a	|a	|s	|*	|*	|.
|r	|z	|b	|k	|[17][S]\rarr	|o	|b	|o	|k	|n	|i	|e	|*	|*	|*	|r	|k	|y	|*	|*	|.
|w	|e	|a	|[][,]{ }	|*	|*	|*	|*	|*	|a	|*	|b	|[18][S]\darr	|*	|*	|c	|[][,]{ }	|n	|*	|*	|.
|a	|*	|m	|s	|*	|*	|*	|[19][S]\rarr	|h	|u	|r	|o	|n	|*	|*	|i	|s	|c	|*	|*	|.
|ł	|*	|b	|z	|[20][S]\rarr	|k	|r	|a	|s	|k	|o	|w	|e	|*	|*	|k	|t	|h	|*	|*	|.
|e	|*	|u	|t	|*	|[21][S]\drarr	|a	|l	|t	|o	|*	|i	|r	|*	|*	|ę	|e	|r	|*	|[22][S]\darr	|.
|*	|*	|s	|y	|[23][S]\darr	|m	|*	|*	|[24][S]\darr	|w	|*	|e	|w	|*	|[25][S]\darr	|s	|n	|o	|[26][S]\darr	|n	|.
|*	|*	|o	|w	|g	|o	|*	|*	|w	|y	|*	|c	|[][,]{ }	|*	|l	|*	|o	|t	|s	|i	|.
|*	|*	|w	|n	|a	|c	|*	|*	|a	|*	|*	|*	|s	|*	|a	|*	|g	|r	|z	|e	|.
|*	|*	|y	|y	|p	|[][,]{ }	|*	|*	|r	|*	|*	|*	|ł	|*	|z	|*	|r	|o	|a	|d	|.
|*	|*	|*	|*	|o	|w	|*	|*	|s	|*	|*	|*	|u	|*	|j	|*	|a	|n	|f	|o	|.
|*	|[27][S]\darr	|*	|[28][S]\darr	|w	|y	|*	|*	|z	|*	|*	|*	|c	|*	|o	|*	|f	|[][,]{ }	|r	|s	|.
|[29][S]\drarr	|k	|a	|p	|i	|t	|a	|n	|t	|*	|*	|*	|h	|*	|n	|*	|i	|p	|a	|y	|.
|u	|o	|*	|a	|c	|w	|*	|*	|a	|[30][S]\drarr	|p	|r	|o	|m	|*	|*	|c	|r	|n	|p	|.
|n	|m	|*	|r	|z	|ó	|*	|[31][S]\rarr	|t	|ł	|o	|*	|w	|*	|*	|*	|z	|o	|[][,]{ }	|i	|.
|d	|o	|*	|t	|k	|r	|*	|[32][S]\darr	|*	|a	|*	|*	|y	|*	|*	|*	|n	|t	|s	|a	|.
|a	|r	|*	|y	|a	|c	|*	|c	|*	|j	|*	|*	|*	|*	|*	|*	|y	|o	|p	|n	|.
|r	|ó	|*	|j	|*	|z	|*	|o	|*	|d	|*	|*	|*	|*	|*	|*	|*	|n	|i	|i	|.
|i	|w	|*	|n	|*	|a	|*	|w	|[33][S]\rarr	|a	|e	|r	|o	|b	|u	|s	|*	|o	|ż	|e	|.
|a	|*	|*	|y	|*	|*	|[34][S]\drarr	|l	|a	|k	|i	|e	|r	|n	|i	|c	|t	|w	|o	|*	|.
|*	|*	|*	|*	|*	|*	|g	|e	|*	|*	|*	|*	|*	|*	|*	|*	|*	|y	|w	|*	|.
|*	|*	|*	|*	|*	|*	|i	|y	|*	|*	|*	|*	|*	|*	|*	|*	|*	|*	|y	|*	|.
|*	|*	|*	|*	|*	|*	|l	|*	|*	|*	|*	|*	|*	|*	|*	|*	|*	|*	|*	|*	|.
|*	|*	|*	|*	|*	|*	|*	|*	|*	|*	|*	|*	|*	|*	|*	|*	|*	|*	|*	|*	|.\end{Puzzle}

\newpage

\begin{PuzzleClues}{\textbf{Poziome}\\}\Clue{3}{}{w średniowiecznej Polsce wojownik, rycerz}
\Clue{4}{}{urządzenie służące do wprowadzenia energii elektrycznej do  akumulatora elektrycznego}
\Clue{6}{}{opona deszczowa}
\Clue{7}{}{członek Polskiej Zjednoczonej Partii Robotniczej (PZPR)}
\Clue{8}{}{włochaty, kudłaty materiał (tkanina lub dzianina), zwykle o długich włoskach lub swego rodzaju frędzlach}
\Clue{9}{}{przyprawa; rozdrobnione ziele majeranku}
\Clue{10}{}{nieszczerość}
\Clue{11}{}{choroba, w której zmiana patologiczna pierwotnie występuje w tkance łącznej; termin wychodzący z użycia w medycynie}
\Clue{15}{}{człowiek, który wyrąbuje, przecina drogę w lesie}
\Clue{17}{}{futryna okienna}
\Clue{19}{}{jezioro w Kanadzie i USA, drugie pod względem wielkości w grupie Wielkich Jezior, powierzchnia 59,6 tyś. km2}
\Clue{20}{}{kraskowate, Coraciiformes - rząd ptaków z podgromady Neornithes}
\Clue{21}{}{altówka, wiola}
\Clue{29}{}{okręt admiralski}
\Clue{30}{}{statek wodny służący do przewozu osób i towarów}
\Clue{31}{}{odpowiednio dobrana ścieżka dźwiękowa do jakiegoś filmu, spektaklu teatralnego itp}
\Clue{33}{}{samolot szerokokadłubowy przeznaczony do przewozu dużej ilości pasażerów}
\Clue{34}{}{zawód lakiernika}\end{PuzzleClues}

\begin{PuzzleClues}{\textbf{Pionowe}\\}\Clue{1}{}{Dromiciops gliroides - południowoamerykański torbacz z rodziny beztorbikowatych, jedyny współcześnie żyjący przedstawiciel rzędu beztorbików; występuje w gęstych, górskich lasach Chile i Argentyny, zwłaszcza wśród bambusów Chusquea}
\Clue{2}{}{Adenophora stricta - gatunek rośliny z rodziny dzwonkowatych}
\Clue{3}{}{wzrost liczby mieszkańców wśród ludności danego obszaru; dotyczy to w szczególności Metysów}
\Clue{4}{}{materialistyczna teoria stworzona przez Karola Marksa i Fryderyka Engelsa głosząca, że socjalizm stanowi konieczne stadium rozwoju społecznego}
\Clue{5}{}{podrząd ssaków z rzędu parzystokopytnych; są roślinożerne, a ponieważ połykają pokarm słabo pogryziony, ulega on później przeżuciu, w chwilach, gdy zwierzęta się nie pasą}
\Clue{7}{}{część aktywów podmiotu gospodarczego o przewidywalnej zbywalności dłużej niż rok}
\Clue{12}{}{drewno pozyskiwane z drzewa chlebowca właściwego; wykorzystywane jest do rękodzieła artystycznego oraz do wyrobu czółen}
\Clue{13}{}{Succisa Haller - roślina należąca do rodziny szczeciowatych (Dipsacaceae)}
\Clue{14}{}{znak stosowany w systemie stenograficznym}
\Clue{16}{}{akcelerator cykliczny, rodzaj synchrotronu, przystosowany do przyspieszania protonów i antyprotonów}
\Clue{18}{}{VIII nerw czaszkowy unerwiający ucho wewnętrzne łączący je z mózgowiem}
\Clue{21}{}{zasób, który jest w posiadaniu przedsiębiorstwa (lub miasta, kraju, regionu itp.) i korzysta się z niego w celu wyprodukowania towaru lub świadczenia usług}
\Clue{22}{}{brak dostatecznej ilości snu, skutkujący zmęczeniem}
\Clue{23}{}{kobieta, która jedzie środkiem publicznej komunikacji na gapę, czyli bez ważnego biletu, bez uiszczonej opłaty za przejazd}
\Clue{24}{}{zdarzenie, którego uczestnicy pod okiem prowadzącego nabierają nowych umiejętności lub nabywają nowej wiedzy}
\Clue{25}{}{rodzaj bentosu obejmujący organizmy żyjące wśród glonów poroślowych i inkrustacji wapiennych}
\Clue{26}{}{Crocus aerius - gatunek rośliny z rodziny kosaćcowatych}
\Clue{27}{}{wieś w Polsce, w województwie dolnośląskim, w powiecie świdnickim, w gminie Świdnica}
\Clue{28}{}{człowiek, który należy do jakiejś partii politycznej}
\Clue{29}{}{przedstawiciel dużych glonów morskich z gromady brunatnic; rodzaj listownicy}
\Clue{30}{}{człowiek niegodziwy, godny pogardy}
\Clue{32}{}{(1618-67), poeta angielski, rojalista, liryki miłosne, anakreontyki, ody, eseje}
\Clue{34}{}{chroniony ptak z rzędu wróblowatych długości około 16 cm; żywi się głównie nasionami; Eurazja}\end{PuzzleClues}\newpage\section*{Krzyżówka 54}

\noindent\begin{Puzzle}{25}{22}|*	|*	|*	|[1][S]\drarr	|l	|o	|r	|i	|e	|n	|t	|*	|[2][S]\drarr	|r	|e	|t	|e	|n	|c	|j	|a	|*	|*	|*	|*	|*	|.
|*	|*	|[3][S]\drarr	|b	|o	|t	|u	|l	|i	|z	|m	|[][,]{ }	|d	|z	|i	|e	|c	|i	|ę	|c	|y	|*	|*	|*	|*	|*	|.
|*	|[4][S]\darr	|p	|e	|[5][S]\darr	|*	|*	|*	|*	|*	|[6][S]\drarr	|k	|u	|s	|a	|c	|z	|[][,]{ }	|p	|ł	|o	|w	|y	|*	|*	|*	|.
|*	|n	|r	|z	|s	|*	|[7][S]\darr	|*	|*	|*	|s	|[8][S]\drarr	|s	|k	|o	|c	|z	|e	|k	|*	|[9][S]\drarr	|m	|i	|g	|*	|*	|.
|*	|i	|z	|t	|k	|[10][S]\darr	|c	|*	|[11][S]\darr	|*	|a	|n	|z	|*	|*	|[12][S]\rarr	|i	|l	|j	|u	|s	|z	|y	|n	|*	|*	|.
|*	|e	|e	|r	|o	|a	|e	|[13][S]\rarr	|d	|u	|r	|i	|a	|n	|*	|*	|*	|[14][S]\drarr	|g	|r	|a	|e	|f	|e	|*	|*	|.
|*	|s	|g	|e	|c	|b	|n	|*	|e	|*	|i	|e	|[][,]{ }	|*	|[15][S]\drarr	|n	|a	|d	|b	|u	|d	|o	|w	|a	|*	|*	|.
|*	|p	|l	|ś	|z	|a	|z	|*	|f	|*	|n	|p	|c	|*	|g	|*	|*	|a	|[16][S]\darr	|[17][S]\darr	|ż	|*	|*	|*	|*	|*	|.
|*	|r	|ą	|c	|[][,]{ }	|k	|u	|*	|i	|*	|*	|e	|z	|*	|i	|[18][S]\darr	|*	|l	|b	|w	|d	|[19][S]\darr	|*	|*	|*	|*	|.
|[20][S]\rarr	|a	|d	|i	|d	|a	|s	|*	|l	|[21][S]\drarr	|m	|ł	|y	|n	|e	|k	|*	|m	|ó	|a	|ż	|t	|*	|*	|*	|*	|.
|*	|w	|a	|o	|ł	|ń	|[][,]{ }	|*	|a	|c	|*	|n	|ś	|[22][S]\darr	|s	|e	|*	|a	|r	|l	|a	|o	|*	|*	|*	|*	|.
|*	|n	|r	|w	|u	|c	|w	|*	|d	|u	|*	|o	|ć	|t	|e	|p	|*	|t	|[][,]{ }	|l	|d	|r	|*	|*	|*	|*	|.
|*	|o	|k	|o	|g	|z	|y	|*	|a	|k	|*	|s	|c	|r	|k	|l	|*	|y	|m	|e	|a	|[][,]{ }	|*	|*	|*	|*	|.
|*	|ś	|a	|ś	|o	|y	|b	|*	|*	|i	|[23][S]\drarr	|p	|o	|z	|i	|e	|w	|n	|i	|k	|*	|r	|*	|*	|*	|*	|.
|*	|ć	|*	|ć	|c	|k	|o	|*	|*	|e	|k	|r	|w	|e	|n	|r	|*	|k	|e	|*	|*	|e	|*	|*	|*	|*	|.
|*	|*	|*	|*	|z	|*	|r	|*	|*	|r	|o	|a	|a	|ź	|g	|*	|*	|a	|s	|*	|*	|g	|*	|*	|*	|*	|.
|*	|*	|*	|*	|u	|*	|c	|*	|*	|n	|l	|w	|*	|w	|*	|*	|*	|*	|z	|*	|*	|a	|*	|*	|*	|*	|.
|*	|*	|*	|[24][S]\rarr	|b	|e	|z	|i	|m	|i	|e	|n	|n	|o	|ś	|ć	|*	|[25][S]\rarr	|a	|r	|k	|t	|y	|k	|a	|*	|.
|*	|*	|*	|*	|y	|*	|y	|*	|*	|c	|c	|o	|*	|ś	|*	|[26][S]\rarr	|k	|a	|n	|e	|f	|o	|r	|a	|*	|*	|.
|*	|*	|*	|*	|*	|*	|*	|*	|*	|z	|*	|ś	|*	|ć	|[27][S]\rarr	|k	|r	|z	|y	|ż	|o	|w	|a	|*	|*	|*	|.
|*	|[28][S]\rarr	|p	|o	|m	|u	|r	|n	|i	|k	|*	|ć	|*	|*	|*	|*	|*	|*	|*	|*	|*	|y	|*	|*	|*	|*	|.
|*	|*	|*	|[29][S]\rarr	|l	|i	|c	|z	|m	|a	|n	|*	|*	|*	|*	|*	|*	|*	|*	|*	|*	|*	|*	|*	|*	|*	|.
|[30][S]\rarr	|m	|i	|l	|i	|w	|o	|l	|t	|*	|*	|*	|*	|*	|*	|*	|*	|*	|*	|*	|*	|*	|*	|*	|*	|*	|.\end{Puzzle}

\newpage

\begin{PuzzleClues}{\textbf{Poziome}\\}\Clue{1}{}{miasto we Francji (Bretania), ważny port nad Oceanem Atlantyckim; przemysł stoczniowy zbrojeniowy, samochodowy}
\Clue{2}{}{zapas wody, który jest zmagazynowany w gruncie lub na powierzchni ziemi (w zbiornikach sztucznych i naturalnych)}
\Clue{3}{}{rodzaj botulizmu, rozwijającego się u niemowląt na skutek namnażania się bakterii w organizmie}
\Clue{6}{}{Crypturellus bartletti - gatunek ptaka z rodziny kusaczy (Tinamidae)}
\Clue{8}{}{pluskwiak z rodziny szarańczowatych}
\Clue{9}{}{najkrótszy stan elektryczny przekazywany łączem telegraficznym}
\Clue{12}{}{radz. konstruktor lotniczy (1894-1977), twórca licznych samolotów typu Ił}
\Clue{13}{}{jadalny owoc (niesucha torebka) duriana właściwego, jeden z najdziwniejszych owoców świata}
\Clue{14}{}{okulista niemiecki (1828-70); jeden z twórców nowoczesnej okulistyki}
\Clue{15}{}{jedno z podstawowych pojęć marksistowskiej teorii rozwoju społecznego; wyobrażenia, sądy, idee przyjmujące formy prawne, polityczne, religijne, artystyczne, filozoficzne odzwierciedlające ogół stosunków kształtujących się na bazie ustroju ekonomicznego danego społeczeństwa}
\Clue{20}{}{but sportowy, przeznaczony do różnych form aktywnego wypoczynku lub jako element luźnego stroju codziennego}
\Clue{21}{}{mechaniczne urządzenie AGD służące do mielenia (rozdrabniania) suchych produktów spożywczych}
\Clue{23}{}{chwast z rodziny wargowatych}
\Clue{24}{}{anonimowość, nieujawnianie lub nieposiadanie imienia i nazwiska}
\Clue{25}{}{obszar polarny wokół bieguna płn, obszar 21 min km2}
\Clue{26}{}{podpora w formie kobiety z koszem na głowie, odmiana kariatydy}
\Clue{27}{}{mięso z zadniej (czy lędźwiowej) części półtuszy, bardzo cenione w kulinariach}
\Clue{28}{}{przedstawiciel zmiennokształtnej rasy fantastycznych postaci, które potrafią z formy ludzkiej przedzierzgnąć się w wielkiego czarnego ptaka; nazwa wywodzi się z tzw.Trylogii husyckiej A. Sapkowskiego}
\Clue{29}{}{pracownik portowy, który przelicza towar załadowywany na statek lub rozładowywany z niego}
\Clue{30}{}{jednostka napięcia elektrycznego}\end{PuzzleClues}

\begin{PuzzleClues}{\textbf{Pionowe}\\}\Clue{1}{}{brak treści, sensu, głębszej myśli, cecha czegoś błahego, pustego}
\Clue{2}{}{według wierzeń Kościoła rzymskokatolickiego - nieśmiertelny byt niematerialny, który po śmierci człowieka przebywa w miejscu, w którym pokutuje za swoje ziemskie grzechy i oczekuje na zbawienie}
\Clue{3}{}{urządzenie optyczne, które służy do oglądania przezroczy}
\Clue{4}{}{wadliwe funkcjonowanie np. urządzenia}
\Clue{5}{}{pingwin długoczuby, Eudyptes moseleyi -  gatunek ptaka z rodziny pingwinów (Spheniscidae), w oparciu o najnowsze badania wydzielony jako odrębny gatunek z gatunku Eudyptes chrysocome; zdecydowana większość populacji zamieszkuje archipelag Tristan da Cunha oraz wyspę Gough na południowym Oceanie Atlantyckim}
\Clue{6}{}{bojowy środek trujący o działaniu paralityczno-drgawkowym}
\Clue{7}{}{ograniczenie prawa wyborczego, które polega na przyznaniu obywatelowi prawa głosu w zależności od posiadanego majątku, wykształcenia itp}
\Clue{8}{}{długotrwały stan występowania pewnych ograniczeń w prawidłowym funkcjonowaniu człowieka}
\Clue{9}{}{muzułmański dywanik modlitewny}
\Clue{10}{}{mieszkaniec Abakanu}
\Clue{11}{}{przemarsz zorganizowanej grupy wojska, przelot samolotów lub przejazd pojazdów mechanicznych w szyku paradnym}
\Clue{14}{}{mieszkanka Dalmacji, kobieta pochodzenia dalmatyńskiego}
\Clue{15}{}{niemiecki pianista i pedagog (1895-1956); światowej sławy wykonawca utworów W. A. Mozarta}
\Clue{16}{}{bór, w którym gatunkami głównymi są drzewa iglaste, a domieszkowymi niektóre gatunki drzew liściastych}
\Clue{17}{}{Walewski; dyrygent, kompozytor i pedagog (1885-1944); założyciel Instytutu Muzycznego w Krakowie; utwory chóralne, poematy symfoniczne, kantaty, pieśni}
\Clue{18}{}{niemiecki astronom i matematyk (1571-1630) propagator idei Kopernika, odkrył ruchy planet}
\Clue{19}{}{zbiornik do uprawiania sportów wodnych}
\Clue{21}{}{zawartość cukierniczki, naczynia służącego do podawania i przechowywania cukru}
\Clue{22}{}{przytomność, przejaw braku senności, zmęczenia i innych stanów osłabiających uwagę i percepcję}
\Clue{23}{}{ostra część niektórych roślin, będąca wytworem epidermy, występująca na łodygach, liściach i owocach}\end{PuzzleClues}\newpage\section*{Krzyżówka 55}

\noindent\begin{Puzzle}{18}{28}|*	|*	|*	|*	|*	|*	|[1][S]\drarr	|g	|w	|ó	|ź	|d	|ź	|*	|*	|*	|*	|*	|*	|.
|*	|*	|*	|*	|*	|[2][S]\rarr	|k	|w	|a	|l	|i	|f	|i	|k	|a	|c	|j	|a	|*	|.
|*	|*	|*	|[3][S]\darr	|*	|[4][S]\drarr	|o	|k	|r	|e	|s	|*	|[5][S]\drarr	|b	|l	|a	|s	|k	|*	|.
|[6][S]\drarr	|l	|u	|k	|*	|p	|ł	|[7][S]\drarr	|n	|o	|s	|ó	|w	|k	|a	|*	|[8][S]\darr	|*	|*	|.
|o	|[9][S]\darr	|*	|o	|*	|ó	|n	|t	|*	|[10][S]\darr	|*	|[11][S]\drarr	|o	|d	|s	|u	|w	|*	|*	|.
|l	|l	|[12][S]\darr	|ń	|*	|ł	|i	|ł	|*	|k	|[13][S]\drarr	|g	|r	|u	|d	|k	|a	|*	|*	|.
|t	|ę	|t	|[][,]{ }	|*	|s	|e	|o	|[14][S]\rarr	|a	|d	|a	|k	|s	|*	|*	|r	|*	|*	|.
|*	|b	|o	|a	|*	|t	|r	|k	|*	|d	|z	|t	|o	|*	|*	|*	|i	|*	|*	|.
|*	|o	|k	|n	|*	|r	|z	|*	|*	|u	|w	|u	|l	|*	|[15][S]\darr	|*	|a	|*	|*	|.
|*	|r	|s	|g	|*	|u	|*	|*	|*	|k	|o	|n	|o	|*	|p	|[16][S]\darr	|t	|*	|*	|.
|[17][S]\drarr	|c	|y	|l	|i	|n	|d	|e	|r	|*	|n	|e	|t	|*	|o	|l	|k	|*	|*	|.
|p	|z	|k	|o	|[18][S]\drarr	|o	|r	|z	|ę	|s	|e	|k	|*	|*	|c	|e	|o	|*	|*	|.
|l	|a	|o	|a	|t	|w	|*	|*	|*	|[19][S]\darr	|c	|[][,]{ }	|[20][S]\darr	|*	|h	|g	|w	|*	|*	|.
|e	|n	|l	|r	|o	|c	|*	|*	|*	|c	|z	|i	|g	|*	|m	|n	|o	|*	|*	|.
|b	|i	|o	|a	|p	|e	|*	|[21][S]\drarr	|w	|e	|n	|t	|a	|*	|i	|i	|*	|*	|*	|.
|i	|n	|g	|b	|ó	|*	|[22][S]\drarr	|r	|u	|m	|i	|e	|n	|i	|e	|c	|*	|*	|*	|.
|s	|*	|i	|s	|r	|*	|o	|a	|[23][S]\darr	|e	|k	|r	|a	|*	|l	|z	|*	|*	|*	|.
|c	|*	|a	|k	|[][,]{ }	|*	|p	|j	|w	|n	|[][,]{ }	|o	|s	|*	|*	|a	|*	|*	|*	|.
|y	|[24][S]\darr	|*	|i	|w	|*	|a	|d	|o	|t	|b	|p	|z	|*	|*	|n	|*	|*	|*	|.
|t	|f	|[25][S]\darr	|*	|o	|*	|ł	|e	|r	|*	|u	|a	|e	|*	|*	|k	|*	|*	|*	|.
|*	|a	|s	|*	|j	|*	|*	|r	|e	|*	|l	|r	|*	|[26][S]\darr	|*	|a	|*	|*	|*	|.
|*	|ł	|p	|*	|e	|*	|[27][S]\darr	|*	|c	|[28][S]\darr	|l	|y	|[29][S]\darr	|a	|*	|*	|*	|*	|*	|.
|[30][S]\drarr	|d	|r	|e	|n	|*	|ł	|[31][S]\darr	|z	|s	|e	|c	|n	|l	|*	|*	|*	|*	|*	|.
|o	|a	|z	|[32][S]\darr	|n	|*	|o	|w	|e	|z	|y	|z	|a	|c	|*	|*	|*	|*	|*	|.
|s	|*	|ę	|g	|y	|[33][S]\rarr	|d	|y	|k	|t	|a	|n	|d	|o	|*	|*	|*	|*	|*	|.
|s	|*	|t	|ę	|*	|*	|y	|j	|*	|o	|*	|y	|ż	|c	|*	|*	|*	|*	|*	|.
|*	|*	|*	|s	|*	|*	|g	|e	|*	|s	|*	|*	|d	|k	|*	|*	|*	|*	|*	|.
|*	|[34][S]\rarr	|b	|i	|e	|g	|a	|c	|z	|*	|*	|*	|*	|*	|*	|*	|*	|*	|*	|.
|*	|*	|*	|*	|*	|*	|*	|*	|*	|*	|*	|*	|*	|*	|*	|*	|*	|*	|*	|.\end{Puzzle}

\newpage

\begin{PuzzleClues}{\textbf{Poziome}\\}\Clue{1}{}{wykonany z twardego metalu trzpień zaostrzony z jednej strony, z drugiej zakończony płaskim łbem}
\Clue{2}{}{uporządkowany podział, dzielenie czegoś według określonych cech}
\Clue{4}{}{miesiączka}
\Clue{5}{}{jaskrawość, jasność powierzchni; świecenie ciał astronomicznych}
\Clue{6}{}{zamykany otwór w pokładzie statku lub jego nadbudówkach (lub czołgu), pełniący różne funkcje}
\Clue{7}{}{ostra, zaraźliwa choroba psów, norek, lisów, wilków często śmiertelna}
\Clue{11}{}{ruch oddalenia narzędzia od przedmiotu obrabianego na obrabiarce}
\Clue{13}{}{w dermatologii: wykwit wyniosły ponad powierzchnię skóry, o różnych wymiarach, dość wyraźnym odgraniczeniu i innej niż otaczająca tkanka spoistości}
\Clue{14}{}{Addax nasomaculatus - gatunek ssaka parzystokopytnego z rodziny krętorogich, zaliczany do antylop, jedyny przedstawiciel rodzaju Addax; pierwotnie występował na całej Saharze: od Mauretanii, Maroka i Algierii, aż do Sudanu, a obecnie adaks należy do najrzadszych antylop}
\Clue{17}{}{rodzaj wysokiego, sztywnego kapelusza z główką w kształcie walca i wąskim rondem}
\Clue{18}{}{przedstawiciel protistów zwierzęcych z gromady orzęsków}
\Clue{21}{}{kiermasz, bazar}
\Clue{22}{}{napadowe zaczerwienienie skóry twarzy, zwłaszcza policzków, zazwyczaj jako reakcja psychosomatyczna na silne emocje}
\Clue{30}{}{SĄCZEK; rurka drenarska do odprowadzania wód gruntowych}
\Clue{33}{}{rodzaj sprawdzianu i ćwiczenia polegający na zapisaniu ze słuchu dyktowanego tekstu}
\Clue{34}{}{sportowiec, który zajmuje się bieganiem}\end{PuzzleClues}

\begin{PuzzleClues}{\textbf{Pionowe}\\}\Clue{1}{}{brezentowa ochrona otworu w pokładzie, przez który przechodzi maszt}
\Clue{3}{}{angloarab - jedna z ras koni gorącokrwistych, pochodząca od konia angielskiego skrzyżowanego z koniem arabskim, hodowana głównie we Francji, Wielkiej Brytanii oraz w Polsce; cechuje ją gorący temperament i inteligencja, dzieki czemu odnosi ogromne sukcesy w sporcie jeździeckim}
\Clue{4}{}{JELITODYSZCZE}
\Clue{5}{}{LOTOPAŁANKA; australijski torbacz wielkości wiewiórki}
\Clue{6}{}{okręg w południowej Rumunii}
\Clue{7}{}{element maszyny stanowiący szczelne, przesuwne zamknięcie cylindra, umożliwia zmianę objętości roboczej cylindra, dzięki połączeniu z mechanizmem korbowym, który nadaje mu ruch postępowo-zwrotny}
\Clue{8}{}{sytuacja kompletnego chaosu}
\Clue{9}{}{mieszkaniec Lęborka}
\Clue{10}{}{spadek pozostawiony bez dziedziców}
\Clue{11}{}{gatunek, którego przedstawiciele w ciągu swojego życia wielokrotnie przystępują do rozrodu lub biorą ciągły udział w procesie rozmnażania}
\Clue{12}{}{nauka interdyscyplinarna wyodrębniona z biologii, chemii, medycyny, medycyny weterynaryjnej, farmakologii i innych dziedzin}
\Clue{13}{}{Adenophora bulleyana - gatunek rośliny należący do rodziny dzwonkowatych}
\Clue{15}{}{w literaturze dawnej, poezji lub eufemistycznie - kac}
\Clue{16}{}{mieszkanka Legnicy}
\Clue{17}{}{głosowanie członków społeczności na określonym terytorium (np. kraju, województwa, gminy) w różnych sprawach mających związek z ich miejscem zamieszkania}
\Clue{18}{}{metafora konfliktu, walka z kimś o coś}
\Clue{19}{}{jedna z tkanek twardych zęba, zmineralizowana, o budowie podobnej do kości zbitej, pokrywająca zębinę korzeniową w zębach szkliwoguzkowych (wtedy koronę zęba pokrywa szkliwo) lub cały ząb szkliwofałdowy ssaków roślinożernych}
\Clue{20}{}{tylne, dolne krawędzie żuchwy konia, miejsce przyczepy mięśni żuchwowych}
\Clue{21}{}{osoba lub firma dążąca do przejęcia kontroli nad firmą}
\Clue{22}{}{materiał służący do opalania}
\Clue{23}{}{część ucha wewnętrznego, będąca fragmentem systemu służącego do utrzymania równowagi ciała, czyli błędnika błoniastego}
\Clue{24}{}{w geologii: wygięcie warstw skalnych}
\Clue{25}{}{wyposażenie niezbędne do funkcjonowania czegoś}
\Clue{26}{}{lotnik angielski (1892-1919) dokonał pierwszego w świecie przelotu wraz z Brownem przez Atlantyk}
\Clue{27}{}{główna oś pędu osiowców, na której osadzone są liście}
\Clue{28}{}{lepsze, wyjątkowe piwo rzemieślnicze, zwykle produkowane w małych ilościach, hit wśród birofilów}
\Clue{29}{}{kraina w środkowej części Arabii Saudyjskiej, powierzchnia około 1,4 min km}
\Clue{30}{}{miasto w środkowej Holandii; 48,8 tys. mieszkańców (1982 r.)}
\Clue{31}{}{nazwa małpy z grupy szerokonosych, żyjącej w lasach Ameryki Południowej; małpa ta swoją nazwę zawdzięcza wydawaniu głosów przypominających ryki i przeciągłe wycia}
\Clue{32}{}{grupa dużych ptaków łownych z rzędu blaszkodziobych}\end{PuzzleClues}\newpage\section*{Krzyżówka 56}

\noindent\begin{Puzzle}{19}{33}|*	|*	|*	|*	|*	|*	|*	|*	|*	|*	|[1][S]\darr	|[2][S]\darr	|*	|*	|*	|[3][S]\darr	|*	|*	|*	|*	|.
|*	|*	|*	|*	|*	|[4][S]\rarr	|g	|ł	|u	|p	|t	|a	|s	|*	|*	|m	|*	|[5][S]\darr	|*	|*	|.
|*	|*	|*	|*	|*	|*	|[6][S]\rarr	|e	|n	|e	|r	|g	|i	|a	|*	|i	|*	|h	|*	|*	|.
|*	|*	|*	|[7][S]\rarr	|c	|a	|r	|e	|z	|z	|a	|n	|d	|o	|*	|e	|*	|e	|*	|*	|.
|*	|*	|*	|*	|*	|*	|*	|*	|*	|*	|n	|o	|*	|*	|*	|j	|*	|ł	|*	|*	|.
|*	|*	|*	|*	|*	|*	|*	|*	|*	|*	|s	|s	|*	|*	|*	|s	|*	|m	|*	|[8][S]\darr	|.
|*	|*	|*	|*	|*	|*	|*	|*	|*	|*	|k	|t	|*	|*	|*	|c	|*	|[][,]{ }	|*	|l	|.
|*	|*	|*	|*	|*	|*	|*	|*	|*	|*	|r	|y	|*	|*	|*	|o	|*	|w	|*	|e	|.
|*	|*	|*	|*	|*	|*	|*	|*	|*	|*	|y	|c	|*	|*	|*	|w	|*	|i	|*	|k	|.
|*	|*	|*	|*	|*	|*	|[9][S]\darr	|*	|*	|*	|p	|y	|*	|*	|*	|y	|*	|e	|*	|k	|.
|*	|*	|*	|*	|*	|*	|ł	|*	|*	|*	|c	|z	|*	|*	|*	|[][,]{ }	|*	|l	|*	|i	|.
|*	|*	|*	|*	|*	|*	|a	|*	|*	|*	|j	|m	|*	|*	|*	|p	|*	|k	|*	|[][,]{ }	|.
|*	|*	|*	|*	|*	|[10][S]\drarr	|p	|i	|ó	|r	|a	|*	|*	|*	|*	|l	|*	|i	|*	|k	|.
|*	|*	|*	|*	|*	|t	|i	|[11][S]\darr	|*	|*	|*	|[12][S]\drarr	|b	|u	|g	|a	|j	|*	|*	|r	|.
|*	|*	|*	|*	|[13][S]\drarr	|o	|d	|k	|a	|ż	|a	|c	|z	|*	|*	|n	|*	|*	|*	|ą	|.
|*	|*	|*	|*	|a	|k	|u	|u	|*	|*	|*	|y	|*	|*	|*	|[][,]{ }	|*	|*	|*	|ż	|.
|*	|[14][S]\darr	|*	|*	|r	|a	|c	|r	|*	|*	|*	|f	|*	|*	|*	|o	|*	|*	|*	|o	|.
|*	|p	|*	|*	|r	|r	|h	|o	|*	|*	|*	|r	|*	|*	|*	|d	|*	|*	|*	|w	|.
|*	|i	|*	|*	|h	|k	|*	|b	|*	|*	|*	|a	|*	|*	|*	|b	|*	|*	|*	|n	|.
|*	|e	|*	|*	|e	|a	|*	|r	|*	|*	|*	|*	|*	|*	|*	|u	|*	|[15][S]\darr	|*	|i	|.
|*	|r	|*	|*	|n	|[][,]{ }	|*	|o	|*	|*	|*	|*	|*	|*	|*	|d	|*	|n	|*	|k	|.
|*	|s	|*	|*	|i	|w	|[16][S]\rarr	|d	|a	|r	|l	|i	|n	|g	|t	|o	|n	|i	|a	|*	|.
|*	|i	|*	|*	|u	|i	|*	|y	|*	|[17][S]\darr	|*	|*	|*	|*	|*	|w	|*	|e	|*	|*	|.
|*	|c	|*	|*	|s	|e	|*	|*	|*	|c	|*	|*	|*	|*	|*	|y	|*	|t	|*	|*	|.
|*	|z	|*	|*	|*	|l	|*	|*	|*	|z	|*	|*	|*	|*	|*	|*	|*	|a	|*	|*	|.
|*	|k	|*	|[18][S]\rarr	|l	|o	|r	|y	|s	|a	|[][,]{ }	|b	|i	|a	|ł	|o	|o	|k	|a	|*	|.
|[19][S]\rarr	|a	|d	|w	|e	|n	|t	|y	|z	|m	|*	|*	|*	|*	|*	|*	|*	|t	|*	|*	|.
|*	|*	|*	|[20][S]\rarr	|p	|o	|n	|s	|*	|b	|*	|*	|*	|*	|*	|*	|*	|o	|*	|*	|.
|*	|*	|*	|*	|*	|ż	|*	|*	|*	|u	|*	|*	|*	|*	|*	|*	|*	|w	|*	|*	|.
|*	|*	|*	|*	|*	|o	|*	|*	|*	|ł	|*	|*	|*	|*	|*	|*	|*	|n	|*	|*	|.
|*	|*	|*	|*	|*	|w	|*	|*	|*	|*	|*	|*	|*	|*	|*	|*	|*	|o	|*	|*	|.
|[21][S]\rarr	|d	|w	|u	|n	|a	|s	|t	|o	|l	|e	|c	|i	|e	|*	|*	|*	|ś	|*	|*	|.
|*	|*	|*	|*	|*	|*	|*	|*	|*	|*	|*	|*	|*	|*	|*	|*	|*	|ć	|*	|*	|.
|*	|*	|*	|*	|*	|*	|*	|*	|*	|*	|*	|*	|*	|*	|*	|*	|*	|*	|*	|*	|.\end{Puzzle}

\newpage

\begin{PuzzleClues}{\textbf{Poziome}\\}\Clue{4}{}{żartobliwie o osobie o niedużych zdolnościach intelektualnych lub takiej, która nierozsądnie postępuje}
\Clue{6}{}{cecha kogoś żywotnego, energicznego, aktywnie działającego}
\Clue{7}{}{określenie wykonawcze; pieszczotliwie}
\Clue{10}{}{łow. gruba i dłuższa szczecina na grzbiecie dorosłego dzika}
\Clue{12}{}{kępa drzew lub krzewów}
\Clue{13}{}{substancja, która służy do odkażania, dezynfekowania}
\Clue{16}{}{Darlingtonia - rodzina roślin z rodziny kapturnicowatych}
\Clue{18}{}{Psitteuteles versicolor - gatunek ptaka z rodziny papugowatych (Psittacidae), z podrodziny papug wschodnich (Psittaculinae)}
\Clue{19}{}{Kościoły i inne wspólnoty chrześcijańskie wywodzące się lub nawiązujące do XIX-wiecznego amerykańskiego przebudzenia eschatologiczno-mesjanistycznego w łonie ewangelikalizmu, któremu przewodził William Miller}
\Clue{20}{}{Jean, ur. w 1761 r. astronom francuski, odkrywca wielu komet, dyrektor obserwatorium we Florencji}
\Clue{21}{}{12. rocznica}\end{PuzzleClues}

\begin{PuzzleClues}{\textbf{Pionowe}\\}\Clue{1}{}{zapisywanie (np. ze słuchu), przepisywanie w innej konwencji tekstów, w sposób, który jest jakąś interpretacją fonetyczną dźwięków języka}
\Clue{2}{}{pogląd filozoficzny, według którego obecnie niemożliwe jest  dowiedzenie się, czy Bóg lub bogowie istnieją, czy też nie}
\Clue{3}{}{akt prawa miejscowego określający granice zewnętrzne gruntów przeznaczonych do odbudowy obiektów budowlanych zniszczonych lub uszkodzonych w wyniku osunięcia ziemi}
\Clue{5}{}{rodzaj średniowiecznego hełmu rycerskiego typu zamkniętego}
\Clue{8}{}{stosunkowo szybki okręt artyleryjski o wyporności od 3000 do 5600 ton}
\Clue{9}{}{kontroler biletów, zazwyczaj w komunikacji miejskiej}
\Clue{10}{}{tokarka przystosowana do jednoczesnej pracy wieloma nożami}
\Clue{11}{}{koralniki, Callaeidae - rodzina ptaków z rzędu wróblowych, obejmująca kilka gatunków ptaków, występujących wyłącznie w Nowej Zelandii}
\Clue{12}{}{rodzaj wzoru, ornamentu, np. na góralskim ubraniu}
\Clue{13}{}{astrofizyk i fizykochemik szwedzki (1839-1927), nagroda Nobla w 1903 r}
\Clue{14}{}{zdrobniale: pierś - klatka piersiowa}
\Clue{15}{}{cecha kogoś, kto jest nietaktowny}
\Clue{17}{}{zbrojny oddział tatarski}\end{PuzzleClues}\newpage\section*{Krzyżówka 57}

\noindent\begin{Puzzle}{24}{31}|*	|*	|[1][S]\drarr	|d	|z	|i	|e	|r	|z	|y	|k	|[][,]{ }	|d	|w	|u	|b	|a	|r	|w	|n	|y	|*	|*	|*	|*	|.
|[2][S]\rarr	|p	|s	|t	|r	|o	|k	|ó	|w	|k	|a	|[][,]{ }	|n	|a	|d	|o	|b	|n	|a	|*	|*	|[3][S]\darr	|*	|*	|*	|.
|*	|[4][S]\rarr	|t	|u	|r	|e	|c	|k	|o	|j	|ę	|z	|y	|c	|z	|n	|o	|ś	|ć	|*	|[5][S]\darr	|s	|*	|*	|*	|.
|[6][S]\drarr	|b	|r	|a	|m	|k	|a	|[][,]{ }	|k	|o	|n	|t	|a	|k	|t	|o	|w	|a	|*	|*	|i	|t	|*	|*	|*	|.
|w	|[7][S]\rarr	|z	|m	|i	|e	|n	|n	|a	|[][,]{ }	|z	|a	|l	|e	|ż	|n	|a	|*	|[8][S]\drarr	|d	|n	|a	|*	|[9][S]\darr	|*	|.
|c	|*	|ę	|*	|*	|[10][S]\darr	|*	|*	|*	|*	|*	|*	|*	|*	|*	|*	|*	|*	|ł	|[11][S]\darr	|s	|ł	|*	|e	|*	|.
|i	|[12][S]\drarr	|p	|r	|e	|z	|e	|n	|t	|e	|r	|k	|a	|*	|*	|*	|[13][S]\drarr	|m	|u	|s	|t	|a	|n	|g	|*	|.
|ą	|ł	|l	|[14][S]\rarr	|d	|r	|y	|b	|l	|i	|n	|g	|*	|*	|*	|*	|p	|*	|s	|r	|y	|*	|*	|k	|*	|.
|g	|a	|i	|*	|[15][S]\darr	|z	|*	|[16][S]\darr	|*	|*	|*	|[17][S]\rarr	|p	|o	|l	|k	|a	|*	|z	|e	|t	|*	|*	|*	|*	|.
|a	|w	|c	|*	|m	|u	|*	|d	|*	|[18][S]\drarr	|e	|g	|e	|r	|*	|*	|s	|[19][S]\darr	|c	|b	|u	|[20][S]\darr	|*	|*	|*	|.
|c	|a	|a	|*	|o	|t	|*	|e	|[21][S]\rarr	|p	|a	|r	|n	|a	|s	|*	|t	|d	|z	|r	|c	|f	|*	|[22][S]\darr	|*	|.
|z	|[][,]{ }	|[][,]{ }	|*	|t	|a	|*	|p	|*	|r	|*	|*	|*	|[23][S]\darr	|*	|*	|o	|r	|y	|z	|j	|i	|*	|s	|*	|.
|*	|p	|p	|*	|y	|*	|[24][S]\darr	|o	|*	|o	|*	|[25][S]\darr	|*	|l	|*	|*	|r	|ę	|k	|a	|a	|l	|[26][S]\darr	|z	|*	|.
|*	|r	|o	|[27][S]\drarr	|w	|y	|r	|z	|e	|c	|z	|e	|n	|i	|e	|*	|a	|t	|[][,]{ }	|n	|[][,]{ }	|e	|r	|c	|*	|.
|*	|z	|l	|k	|*	|*	|e	|y	|*	|e	|*	|t	|[28][S]\darr	|d	|*	|*	|ł	|w	|i	|k	|k	|m	|ó	|z	|*	|.
|*	|y	|s	|w	|*	|*	|n	|c	|[29][S]\drarr	|s	|i	|e	|m	|i	|ę	|[][,]{ }	|k	|a	|n	|a	|r	|o	|w	|e	|*	|.
|[30][S]\drarr	|s	|k	|a	|j	|l	|a	|j	|t	|*	|*	|r	|o	|n	|[31][S]\darr	|[32][S]\darr	|a	|[][,]{ }	|d	|[][,]{ }	|e	|n	|n	|r	|*	|.
|w	|i	|a	|d	|*	|[33][S]\darr	|r	|a	|e	|*	|[34][S]\darr	|[][,]{ }	|c	|o	|t	|a	|*	|p	|y	|a	|d	|[][,]{ }	|o	|k	|*	|.
|e	|ę	|*	|r	|[35][S]\darr	|ż	|d	|*	|r	|*	|w	|n	|*	|w	|r	|d	|*	|a	|g	|n	|y	|t	|z	|l	|*	|.
|n	|g	|*	|a	|r	|y	|*	|*	|m	|*	|i	|a	|*	|o	|ę	|h	|*	|w	|o	|d	|t	|a	|n	|i	|*	|.
|d	|ł	|*	|t	|u	|w	|[36][S]\darr	|*	|y	|*	|g	|f	|[37][S]\darr	|*	|b	|e	|*	|i	|w	|y	|o	|n	|a	|n	|*	|.
|a	|y	|[38][S]\rarr	|u	|c	|i	|o	|s	|*	|*	|i	|t	|a	|*	|i	|z	|*	|k	|y	|j	|w	|i	|c	|a	|*	|.
|*	|c	|*	|r	|h	|k	|k	|*	|[39][S]\darr	|[40][S]\rarr	|l	|o	|k	|a	|c	|j	|a	|*	|*	|s	|a	|m	|z	|[][,]{ }	|*	|.
|*	|h	|*	|a	|*	|*	|a	|*	|s	|*	|i	|w	|t	|*	|k	|a	|[41][S]\rarr	|r	|a	|k	|*	|b	|n	|p	|*	|.
|*	|*	|*	|[][,]{ }	|*	|*	|p	|*	|h	|*	|a	|y	|*	|*	|i	|*	|*	|*	|*	|a	|*	|a	|o	|i	|*	|.
|[42][S]\rarr	|n	|i	|g	|e	|l	|*	|*	|o	|*	|*	|*	|*	|*	|*	|*	|*	|*	|*	|*	|*	|r	|ś	|a	|*	|.
|[43][S]\rarr	|d	|r	|a	|b	|i	|n	|o	|w	|i	|e	|c	|[][,]{ }	|m	|r	|o	|c	|z	|n	|y	|*	|s	|ć	|s	|*	|.
|*	|[44][S]\rarr	|p	|u	|c	|h	|a	|r	|*	|*	|*	|*	|*	|*	|[45][S]\rarr	|o	|d	|c	|i	|n	|e	|k	|*	|k	|*	|.
|[46][S]\drarr	|g	|ę	|s	|t	|o	|ś	|ć	|[][,]{ }	|k	|r	|y	|t	|y	|c	|z	|n	|a	|*	|*	|*	|i	|*	|o	|*	|.
|d	|[47][S]\rarr	|p	|s	|z	|e	|n	|i	|e	|c	|[][,]{ }	|l	|e	|ś	|n	|y	|*	|*	|*	|*	|*	|*	|*	|w	|*	|.
|*	|[48][S]\rarr	|z	|a	|s	|ł	|o	|n	|a	|[][,]{ }	|d	|y	|m	|n	|a	|*	|[49][S]\rarr	|g	|o	|m	|ó	|ł	|k	|a	|*	|.
|*	|*	|*	|*	|*	|*	|*	|*	|*	|*	|*	|*	|*	|*	|*	|*	|*	|*	|*	|*	|*	|*	|*	|*	|*	|.\end{Puzzle}

\newpage

\begin{PuzzleClues}{\textbf{Poziome}\\}\Clue{1}{}{Laniarius bicolor - gatunek ptaka  z rodziny dzierzbików (Malaconotidae)}
\Clue{2}{}{Utetheisa pulchella - gatunek motyla z rodziny niedźwiedziówkowatych, o pięknym ubarwieniu z rysunkiem w formie prostokątnych czerwonych i czarnych plamek; lata od końca maja do sierpnia oraz we wrześniu i w październiku}
\Clue{4}{}{to, że ktoś jest tureckojęzyczny; mówi po turecku}
\Clue{6}{}{gol strzelony przez drużynę przegrywającą różnicą dwóch bramek}
\Clue{7}{}{w matematyce i statystyce - zmienna, która jest wynikiem wcześniejszych operacji}
\Clue{8}{}{cząsteczka kwasu deoksyrybonuklinowego, rozumiana jako nośnik informacji genetycznej}
\Clue{12}{}{kobieta zapowiadająca, prowadząca programy w telewizji}
\Clue{13}{}{dziki koń żyjący na preriach}
\Clue{14}{}{w piłce i hokeju: prowadzenie piłki lub krążka blisko ziemi (lodowiska) z częstą zmianą kierunku dla zmylenia i wyminięcia przeciwnika}
\Clue{17}{}{rytmiczny, skoczny utwór, do którego można tańczyć polkę}
\Clue{18}{}{miasto w płn. Węgrzech, ośrodek administracyjny komitatu Heves, u podnóża Gór Bukowych}
\Clue{21}{}{pasmo górskie w środkowej Grecji, z najwyższym szczytem wznoszącym się na wysokość 2457 m n.p.m}
\Clue{27}{}{trudność, konieczność odmówienia sobie czegoś}
\Clue{29}{}{ziarno rośliny zielnej z rodziny wiechlinowatych}
\Clue{30}{}{oszklone okienko w pokładzie statku lub dachu kabiny, które przepuszcza światło z zewnątrz}
\Clue{38}{}{element architektoniczny w postaci poziomej, zwykle profilowanej listwy wystającej przed lico muru, która chroni elewację budynku przed ściekającą wodą deszczową}
\Clue{40}{}{jakiś wyodrębniony obszar w niektórych grach komputerowych, część wykreowanej rzeczywistości}
\Clue{41}{}{jakieś negatywne zjawisko, coś, co jest równie szkodliwe i niszczące, jak poważna choroba}
\Clue{42}{}{miasto w Republice Południowe Afryki (Transwal) koło Johannesburga; eksploatacja złota}
\Clue{43}{}{drabinowiec torfowy, Cinclidium stygium - gatunek mchu należący do rodziny merzykowatych (Mniaceae), dawniej klasyfikowany w rodzinie drabinowcowatych (Cinclidiaceae); występuje w strefie klimatów chłodnych Eurazji i Ameryki Północnej, tworzy czarniawe lub brunatne darnie wysokości 10-12 cm; roślina bardzo rzadka, objęta ścisłą ochroną gatunkową w Polsce}
\Clue{44}{}{objętość płynu, który mieści się w pucharze (naczyniu)}
\Clue{45}{}{wyodrębniona część czegoś, mająca kształt podłużny}
\Clue{46}{}{taką średnią gęstość materii nierelatywistycznej, jaką miałby Wszechświat o zerowej krzywiźnie i płaskiej geometrii przestrzeni}
\Clue{47}{}{Melampyrum sylvaticum - gatunek roślin należący do rodziny zarazowatych}
\Clue{48}{}{działanie przedsięwzięte w celu ukrycia prawdziwych intencji czy celów działającego}
\Clue{49}{}{łow. samiec zwierząt rogatych, który okresowo nie ma poroża}\end{PuzzleClues}

\begin{PuzzleClues}{\textbf{Pionowe}\\}\Clue{1}{}{Koeleria grandis - gatunek roślin z rodziny wiechlinowatych}
\Clue{3}{}{w informatyce - fragment kodu źródłowego, który nie może się zmienić}
\Clue{5}{}{przedsiębiorstwo, którego działalność polega na przyjmowaniu depozytów lub innych funduszy podlegających zwrotowi od klientów oraz na udzielaniu kredytów na własny rachunek}
\Clue{6}{}{mięsień służący do wciągania jakiejś części ciała, np. korpusu do muszli (u muszlowców)}
\Clue{8}{}{łuszczak indygo, Passerina cyanea - gatunek ptaka z rodziny kardynałów (Cardinalidae), wcześniej zaliczany do łuszczaków (Fringillidae) lub trznadlowatych (Emberizidae)}
\Clue{9}{}{kompozytor niemiecki (1901-83); opery, balety, utwory orkiestrowe}
\Clue{10}{}{składka}
\Clue{11}{}{Anas puna - gatunek ptaka z rodziny kaczkowatych (Anatidae)}
\Clue{12}{}{instytucja władzy sądowniczej, w skład której wchodzą zwykli obywatele, a nie sędziowie zawodowi}
\Clue{13}{}{rodzaj udramatyzowanej kolędy wykonywanej dawniej przez wędrownych muzyków i żaków, w której religijna tematyka bożonarodzeniowa poszerzona została o warstwę obyczajową}
\Clue{15}{}{wyrazista rytmicznie grupa dźwięków, zwrot melodyczny}
\Clue{16}{}{proces samooczyszczania się środowiska z zanieczyszczeń i trujących związków}
\Clue{18}{}{przebieg następujących po sobie i powiązanych przyczynowo zmian}
\Clue{19}{}{Torpedo torpedo - gatunek ryby chrzęstnoszkieletowej z rodziny drętwowatych (Torpedinidae); występuje we wschodnim Atlantyku od południowej części Zatoki Biskajskiej do Angoli oraz w Morzu Śródziemnym}
\Clue{20}{}{Philemon plumigenis - gatunek ptaka z rodziny miodojadów (Meliphagidae) obejmującej gatunki występujące w Australii, Nowej Gwinei, wschodniej Indonezji i Nowej Kaledonii}
\Clue{22}{}{Ammophila sabulosa - owad z rodziny grzebaczowatych występujący na suchych terenach Europy Środkowej, głównie w lasach sosnowych i na wrzosowiskach; szczerklina piaskowa  ma czarne ubarwienie, tylko stylik i dwa następne segmenty odwłoka są czerwone}
\Clue{23}{}{miasto w europejskiej części Federacji Rosyjskiej, na płd.-zach. od Moskwy; przemysł taboru kolejowego}
\Clue{24}{}{(1864-1910), pisarz francuski, powieści, dramaty psychologiczno-obyczajowe; „Marchewka”}
\Clue{25}{}{mieszanina węglowodorów (głównie alkanów) pochodząca z procesu rektyfikacji ropy naftowej o średniej temperaturze wrzenia 70-90 °C}
\Clue{26}{}{synonimia}
\Clue{27}{}{kwadratura interpolacyjna z węzłami będącymi zerami wielomianu P n+1}
\Clue{28}{}{wytrzymałość danego materiału}
\Clue{29}{}{w starożytnym Rzymie łaźnie publiczne, kompleks obiektów ulokowanych na rozległym terenie, dostępnych dla wszystkich bezpłatnie i o określonych godzinach}
\Clue{30}{}{inżynier, budowniczy (1863-1948); projektował i budował port w Gdyni}
\Clue{31}{}{sztangista, brązowy medalista olimpijski z Meksyku w wadze koguciej}
\Clue{32}{}{zmiana polegająca na włączeniu się do czegoś - w jakieś działania lub struktury}
\Clue{33}{}{larwa niektórych raków dziesięcionogich charakterystyczna dla krabów; składnik planktonu}
\Clue{34}{}{kolacja wigilijna, świąteczna wieczerza, którą je się z okazji Bożego Narodzenia}
\Clue{35}{}{poruszanie się przechodniów i pojazdów po szlakach komunikacyjnych}
\Clue{36}{}{dolna krawędź połaci dachowej wystająca poza lico ściany budowli}
\Clue{37}{}{obrzęd, oficjalna uroczystość}
\Clue{39}{}{angielskie widowisko artystyczno - rozrywkowe o charakterze rewiowym}
\Clue{46}{}{litera alfabetu używana w numeracji porządkowej}\end{PuzzleClues}\newpage\section*{Krzyżówka 58}

\noindent\begin{Puzzle}{23}{29}|*	|*	|[1][S]\drarr	|n	|o	|w	|o	|g	|r	|ó	|d	|[][,]{ }	|w	|i	|e	|l	|k	|i	|*	|*	|*	|*	|[2][S]\darr	|*	|.
|[3][S]\rarr	|o	|w	|o	|c	|a	|r	|s	|t	|w	|o	|*	|*	|*	|*	|*	|*	|*	|[4][S]\drarr	|b	|l	|o	|k	|*	|.
|*	|[5][S]\darr	|i	|[6][S]\drarr	|w	|l	|e	|w	|*	|[7][S]\drarr	|k	|a	|p	|u	|c	|h	|a	|*	|i	|*	|*	|*	|o	|*	|.
|*	|z	|e	|a	|*	|[8][S]\drarr	|k	|a	|r	|m	|a	|t	|a	|*	|*	|*	|*	|*	|n	|*	|[9][S]\darr	|[10][S]\darr	|r	|*	|.
|*	|e	|l	|k	|*	|i	|[11][S]\darr	|[12][S]\rarr	|k	|a	|r	|l	|o	|v	|a	|c	|*	|[13][S]\darr	|i	|*	|l	|s	|s	|*	|.
|[14][S]\drarr	|s	|o	|s	|[][,]{ }	|m	|a	|j	|o	|n	|e	|z	|o	|w	|y	|*	|*	|b	|c	|*	|i	|z	|z	|*	|.
|h	|t	|k	|a	|*	|p	|r	|[15][S]\rarr	|n	|a	|s	|a	|d	|a	|*	|*	|*	|o	|j	|[16][S]\darr	|t	|e	|e	|*	|.
|a	|a	|r	|m	|[17][S]\drarr	|l	|o	|g	|o	|p	|e	|d	|k	|a	|*	|[18][S]\drarr	|t	|r	|a	|p	|e	|r	|*	|*	|.
|l	|w	|o	|i	|k	|i	|w	|[19][S]\drarr	|s	|o	|l	|o	|w	|i	|e	|c	|*	|ó	|t	|a	|r	|o	|*	|*	|.
|l	|[][,]{ }	|t	|t	|o	|k	|a	|l	|*	|u	|*	|[20][S]\darr	|*	|[21][S]\darr	|[22][S]\darr	|i	|[23][S]\darr	|w	|y	|d	|a	|k	|*	|*	|.
|*	|p	|n	|k	|ł	|a	|n	|e	|*	|r	|*	|ł	|[24][S]\darr	|a	|ż	|a	|j	|i	|w	|w	|t	|o	|*	|*	|.
|*	|r	|o	|a	|n	|n	|a	|x	|*	|i	|[25][S]\darr	|ą	|s	|s	|ó	|s	|e	|e	|a	|a	|u	|ś	|*	|*	|.
|*	|z	|ś	|*	|i	|t	|*	|u	|*	|*	|p	|c	|e	|z	|ł	|t	|l	|c	|[][,]{ }	|n	|r	|ć	|*	|*	|.
|*	|e	|ć	|*	|e	|[][,]{ }	|*	|s	|[26][S]\drarr	|r	|o	|z	|r	|y	|w	|k	|a	|*	|o	|*	|a	|[][,]{ }	|*	|*	|.
|*	|c	|*	|*	|r	|i	|*	|*	|d	|*	|z	|n	|b	|k	|[][,]{ }	|a	|r	|*	|b	|*	|[][,]{ }	|g	|*	|*	|.
|*	|i	|*	|*	|z	|s	|*	|*	|ł	|*	|i	|i	|i	|*	|f	|r	|a	|*	|y	|[27][S]\darr	|o	|a	|*	|*	|.
|*	|w	|*	|*	|[][,]{ }	|t	|*	|[28][S]\darr	|u	|*	|o	|k	|s	|*	|l	|z	|n	|*	|w	|s	|d	|l	|*	|*	|.
|*	|l	|*	|*	|m	|o	|*	|k	|g	|*	|m	|[][,]{ }	|t	|*	|o	|*	|g	|*	|a	|t	|p	|a	|*	|*	|.
|*	|o	|*	|*	|a	|t	|*	|l	|o	|*	|[][,]{ }	|c	|y	|[29][S]\darr	|r	|*	|*	|*	|t	|e	|u	|k	|*	|*	|.
|*	|t	|*	|*	|r	|n	|*	|u	|s	|*	|g	|i	|k	|p	|y	|*	|*	|*	|e	|r	|s	|t	|*	|*	|.
|*	|n	|*	|*	|y	|y	|*	|c	|z	|*	|ł	|e	|a	|ó	|d	|*	|[30][S]\darr	|[31][S]\darr	|l	|o	|t	|y	|*	|*	|.
|*	|i	|*	|[32][S]\darr	|n	|*	|*	|z	|o	|*	|o	|c	|*	|ł	|y	|*	|r	|k	|s	|w	|o	|c	|*	|*	|.
|*	|c	|*	|d	|a	|[33][S]\darr	|[34][S]\drarr	|k	|w	|*	|ś	|z	|*	|k	|j	|*	|e	|o	|k	|n	|w	|z	|*	|*	|.
|*	|z	|*	|e	|r	|n	|s	|a	|a	|*	|n	|o	|*	|o	|s	|*	|j	|n	|a	|o	|a	|n	|*	|*	|.
|*	|y	|*	|s	|s	|o	|k	|*	|t	|*	|o	|w	|*	|k	|k	|*	|o	|t	|*	|ś	|*	|a	|*	|*	|.
|*	|*	|*	|s	|k	|r	|l	|*	|e	|*	|ś	|y	|[35][S]\rarr	|s	|i	|l	|n	|o	|ś	|ć	|*	|*	|*	|*	|.
|*	|[36][S]\rarr	|p	|o	|i	|m	|e	|k	|*	|*	|c	|*	|*	|*	|*	|*	|*	|*	|*	|*	|*	|*	|*	|*	|.
|*	|*	|*	|u	|*	|a	|p	|[37][S]\rarr	|o	|f	|i	|c	|e	|r	|[][,]{ }	|f	|l	|a	|g	|o	|w	|y	|*	|*	|.
|*	|*	|*	|s	|*	|*	|*	|*	|*	|*	|*	|*	|*	|*	|*	|*	|*	|*	|*	|*	|*	|*	|*	|*	|.
|*	|*	|*	|*	|*	|*	|*	|*	|*	|*	|*	|*	|*	|*	|*	|*	|*	|*	|*	|*	|*	|*	|*	|*	|.\end{Puzzle}

\newpage

\begin{PuzzleClues}{\textbf{Poziome}\\}\Clue{1}{}{miasto w północno-zachodniej Rosji nad rzeką Wołchow; stolica obwodu nowogrodzkiego (przydomek „Wielki” przywrócono oficjalnie w 1998 roku)}
\Clue{3}{}{uprawa drzew i krzewów owocowych dla pozyskania owoców}
\Clue{4}{}{w siatkówce: gra obronna przy siatce, blokada, bloking}
\Clue{6}{}{otwór w zbiorniku służący do wlewania płynów}
\Clue{7}{}{kasa, szmal, forsa, środki materialne, to, czym się płaci}
\Clue{8}{}{członek odłamu muzułmańskiej sekty izmalitów powstałej w 877 r. w Iranie}
\Clue{12}{}{miasto w Chorwacji; przemysł maszynowy, chemiczny, drzewny; węzeł kolejowy}
\Clue{14}{}{zimny, emulsyjny sos na bazie oliwy z dodatkiem surowego żółtka}
\Clue{15}{}{zespół organów mowy ponad krtanią}
\Clue{17}{}{kobieta, która jest logopedą; kobieta zajmująca się badaniami w zakresie diagnozy stanu rozwoju mowy i terapii zaburzeń komunikacji człowieka na różnych etapach rozwoju}
\Clue{18}{}{but roboczy, górski albo trekkingowy, cechujący się grubą podeszwą z protektorem i wysokością do łydki}
\Clue{19}{}{SŁONACZEK; gat. stawonoga}
\Clue{26}{}{działalność mająca na celu dostarczenie przyjemności poprzez relaks}
\Clue{34}{}{skrót, symbol jednostki - kilowata}
\Clue{35}{}{właściwość ludzi i/lub rzeczy odznaczająca się dużą mocą, wyrażająca się w intensywnym działaniu}
\Clue{36}{}{klasa synsemantycznych wyrazów bądź morfemów rządzących przypadkiem poprzedzającego je rzeczownika}
\Clue{37}{}{żołnierz pomocniczy dowódcy}\end{PuzzleClues}

\begin{PuzzleClues}{\textbf{Pionowe}\\}\Clue{1}{}{efekt mnożenia przez liczbę naturalną}
\Clue{2}{}{miasto w województwie warmińsko-mazurskim, w powiecie kętrzyńskim, siedziba gminy miejsko-wiejskiej Korsze}
\Clue{4}{}{jeden z elementów demokracji bezpośredniej, umożliwiający ściśle określonej przez prawo grupie obywateli, posiadających pełnię praw wyborczych wystąpić z inicjatywą ustawodawczą}
\Clue{5}{}{pojazd wojskowy z działem, wykorzystywany do zwalczania statków powietrznych przeciwnika}
\Clue{6}{}{aksamitek, Tagetes - rodzaj roślin jednorocznych należący do rodziny astrowatych, który obejmuje ponad 40 gatunków; dość popularna uprawna roślina ozdobna}
\Clue{7}{}{najgłębsze jezioro w Nowej Zelandii, na Wyspie Południowej, powierzchnia 142 km2, głębokość do 443 m}
\Clue{8}{}{implikant prosty zawierający co najmniej jeden minterm nie występujący w żadnym innym implikancie prostym}
\Clue{9}{}{tanie publikacje przeznaczone dla ludu, popularne zwłaszcza w XIX wieku}
\Clue{10}{}{kąt pomiędzy kierunkiem od obserwatora do obiektu astronomicznego a płaszczyzną równika galaktycznego}
\Clue{11}{}{ryba z rodziny kostnojęzykowych (Osteoglossidae) z rodzajów Osteoglossum i Scleropages}
\Clue{13}{}{wieś w Polsce położona w województwie wielkopolskim, w powiecie poznańskim, w gminie Kórnik, w dolinie rzeki Głuszynki, 5 km na północny zachód od Kórnika}
\Clue{14}{}{HOL}
\Clue{16}{}{szesnastowieczny taniec włoski, utrzymany w wolnym tempie}
\Clue{17}{}{kołnierz o kwadratowym kształcie, zwykle biały, opadający na plecy; ma naszyte granatowe tasiemki (lub białe jeśli jest granatowy)}
\Clue{18}{}{osoba, która wykonuje czynności związane z wypiekiem ciast i ciastek}
\Clue{19}{}{marka luksusowych samochodów osobowych, należąca do japońskiego koncernu Toyota Motor Corporation}
\Clue{20}{}{łącznik, w którym do gaszenia łuku zastosowano ciecz}
\Clue{21}{}{ludowy pieśniarz i poeta w krajach wschodu}
\Clue{22}{}{Pseudemys floridana - gatunek gada z rodziny żółwi błotnych}
\Clue{23}{}{Ratufa bicolor - gatunek ssaka z rodziny wiewiórkowatych, jeden z jej największych przedstawicieli; występuje na Półwyspie Indochińskim, Jawie, Sumatrze, Bali oraz w części Indii, żyje na drzewach}
\Clue{24}{}{filologia serbska; dziedzina nauki, której przedmiotem badań jest serbska literatura i język serbski}
\Clue{25}{}{wielkość będąca porównawczą miarą głośności dźwięku w odniesieniu do głośności tonu wzorcowego, wyrażona w fonach}
\Clue{26}{}{Osmundaceae - rodzina paproci, jej przedstawiciele wytwarzają bardzo duże liście (do 2 m długości), sprawiają wrażenie, jakby miały pnie, są to jednak nasady liści}
\Clue{27}{}{podatność statku latającego na wykonywanie określonych zmian ruchu, kursu, itp}
\Clue{28}{}{uskok zwierzęcia w bok w gwarze łowieckiej}
\Clue{29}{}{łatwopalne paliwo}
\Clue{30}{}{administracyjnie wydzielona część miasta}
\Clue{31}{}{zbiór dokonań, zarówno pozytywnych (szczególnie zawodowych, sportowych, arytystycznych), jak i negatywnych, np. konfliktów z prawem}
\Clue{32}{}{damska bielizna dzienna}
\Clue{33}{}{normalka, coś normalnego, np. normalna sytuacja}
\Clue{34}{}{lokal użyteczności publicznej, obiekt handlowy, przeznaczony do detalicznej sprzedaży towarów}\end{PuzzleClues}\newpage\section*{Krzyżówka 59}

\noindent\begin{Puzzle}{24}{27}|*	|*	|*	|*	|*	|*	|[1][S]\drarr	|p	|e	|r	|s	|z	|e	|r	|o	|n	|*	|[2][S]\drarr	|ł	|e	|b	|*	|*	|*	|*	|.
|*	|*	|[3][S]\rarr	|s	|i	|a	|t	|k	|a	|*	|*	|*	|*	|[4][S]\rarr	|p	|a	|r	|s	|e	|k	|*	|*	|[5][S]\darr	|[6][S]\darr	|[7][S]\darr	|.
|*	|*	|*	|[8][S]\rarr	|p	|r	|a	|w	|d	|z	|i	|w	|o	|ś	|ć	|*	|*	|u	|*	|*	|[9][S]\darr	|*	|r	|p	|j	|.
|*	|[10][S]\darr	|*	|*	|*	|*	|l	|*	|*	|*	|*	|[11][S]\drarr	|g	|a	|d	|a	|c	|z	|*	|*	|n	|*	|z	|o	|e	|.
|[12][S]\drarr	|c	|z	|ł	|o	|w	|i	|e	|k	|[][,]{ }	|g	|u	|m	|a	|*	|*	|*	|a	|*	|*	|e	|*	|e	|n	|r	|.
|j	|z	|[13][S]\drarr	|m	|o	|d	|*	|[14][S]\rarr	|a	|n	|o	|r	|e	|k	|t	|y	|k	|*	|*	|*	|d	|[15][S]\darr	|ź	|o	|z	|.
|ę	|e	|s	|*	|*	|[16][S]\drarr	|s	|p	|e	|c	|j	|a	|l	|n	|y	|*	|*	|*	|*	|*	|b	|n	|n	|c	|y	|.
|z	|s	|i	|*	|[17][S]\drarr	|ż	|ó	|ł	|w	|[][,]{ }	|a	|n	|n	|a	|m	|s	|k	|i	|*	|*	|a	|u	|i	|n	|k	|.
|y	|a	|l	|*	|c	|ó	|*	|[18][S]\rarr	|t	|a	|h	|i	|t	|i	|*	|[19][S]\drarr	|k	|r	|e	|o	|l	|s	|k	|i	|*	|.
|k	|l	|n	|[20][S]\darr	|a	|ł	|*	|*	|[21][S]\darr	|*	|*	|n	|*	|[22][S]\rarr	|a	|w	|e	|r	|o	|a	|*	|a	|*	|c	|*	|.
|[][,]{ }	|n	|i	|z	|r	|t	|[23][S]\rarr	|k	|l	|i	|n	|i	|k	|a	|[][,]{ }	|a	|b	|o	|r	|c	|y	|j	|n	|a	|*	|.
|k	|i	|k	|n	|b	|o	|[24][S]\drarr	|p	|o	|ż	|y	|t	|e	|c	|z	|n	|o	|ś	|ć	|*	|*	|r	|*	|[][,]{ }	|*	|.
|l	|a	|[][,]{ }	|a	|o	|o	|b	|[25][S]\darr	|w	|*	|*	|*	|[26][S]\darr	|*	|*	|*	|*	|*	|[27][S]\darr	|*	|*	|y	|*	|m	|*	|.
|i	|*	|z	|k	|n	|k	|i	|g	|e	|[28][S]\darr	|*	|*	|k	|[29][S]\drarr	|b	|e	|r	|*	|w	|*	|*	|t	|*	|i	|[30][S]\darr	|.
|n	|*	|a	|[][,]{ }	|a	|[][,]{ }	|m	|ę	|l	|j	|[31][S]\darr	|*	|y	|c	|[32][S]\rarr	|k	|o	|k	|i	|l	|k	|a	|*	|r	|n	|.
|g	|[33][S]\drarr	|b	|o	|r	|z	|e	|ś	|l	|a	|d	|[][,]{ }	|n	|i	|t	|e	|c	|z	|k	|a	|*	|*	|*	|i	|i	|.
|o	|t	|u	|c	|a	|ł	|t	|*	|*	|n	|i	|*	|o	|ę	|[34][S]\rarr	|b	|a	|n	|i	|a	|k	|*	|*	|k	|w	|.
|ń	|r	|r	|h	|*	|o	|a	|[35][S]\darr	|[36][S]\darr	|o	|p	|*	|s	|ż	|*	|*	|*	|*	|p	|*	|*	|*	|*	|i	|e	|.
|s	|a	|t	|r	|[37][S]\darr	|t	|l	|l	|k	|w	|l	|*	|u	|k	|[38][S]\rarr	|s	|y	|r	|e	|n	|a	|*	|*	|n	|l	|.
|k	|f	|o	|o	|k	|o	|*	|a	|o	|i	|o	|*	|r	|o	|[39][S]\rarr	|s	|z	|o	|d	|o	|n	|*	|*	|a	|i	|.
|i	|i	|w	|n	|o	|p	|[40][S]\darr	|k	|k	|e	|d	|*	|a	|ś	|*	|*	|[41][S]\rarr	|w	|y	|b	|i	|e	|g	|*	|z	|.
|*	|e	|y	|n	|l	|l	|c	|k	|o	|c	|o	|*	|*	|ć	|[42][S]\rarr	|g	|a	|d	|s	|d	|e	|n	|*	|[43][S]\darr	|m	|.
|*	|n	|*	|y	|e	|a	|h	|o	|s	|*	|k	|*	|*	|*	|[44][S]\rarr	|l	|u	|s	|t	|r	|a	|t	|o	|r	|*	|.
|*	|i	|*	|*	|n	|m	|y	|l	|z	|*	|i	|*	|*	|[45][S]\drarr	|b	|o	|s	|s	|a	|[][,]{ }	|n	|o	|v	|a	|*	|.
|*	|e	|*	|[46][S]\rarr	|d	|y	|l	|i	|n	|a	|*	|[47][S]\rarr	|r	|a	|u	|d	|a	|*	|*	|*	|*	|*	|*	|z	|*	|.
|*	|*	|*	|*	|r	|*	|u	|t	|i	|*	|*	|[48][S]\rarr	|k	|l	|i	|n	|*	|*	|*	|*	|*	|*	|*	|*	|*	|.
|[49][S]\rarr	|m	|o	|r	|a	|*	|s	|*	|k	|*	|[50][S]\rarr	|s	|ó	|l	|[][,]{ }	|w	|[][,]{ }	|o	|k	|u	|*	|*	|*	|*	|*	|.
|*	|*	|*	|*	|*	|*	|*	|*	|*	|*	|*	|*	|*	|*	|*	|*	|*	|*	|*	|*	|*	|*	|*	|*	|*	|.\end{Puzzle}

\newpage

\begin{PuzzleClues}{\textbf{Poziome}\\}\Clue{1}{}{ciężki koń pociągowy rasy francuskiej używany w rolnictwie i transporcie}
\Clue{2}{}{umiejętność picia alkoholu}
\Clue{3}{}{naciągnięta plecionka sznurowa przedzielająca na dwie części plac tenisowy, siatkówkowy}
\Clue{4}{}{jednostka odległości używana w astronomii, odległość, dla której paralaksa roczna wynosi 1 sekundę łuku}
\Clue{8}{}{podobieństwo do rzeczywistości}
\Clue{11}{}{człowiek, którego zadaniem jest mówienie, wypowiadanie się gdzieś, człowiek, który jest oddelegowany do wypowiadania się na jakiś temat}
\Clue{12}{}{człowiek, który potrafi się ponadprzeciętnie rozciągać}
\Clue{13}{}{rezonans własny pomieszczenia}
\Clue{14}{}{osoba cierpiąca na anoreksję}
\Clue{16}{}{człowiek, który odbiega od normy pod względem intelektualnym, zwykle opóźniony w rozwoju umysłowym}
\Clue{17}{}{Mauremys annamensis - gatunek żółwia z rodziny batagurowatych}
\Clue{18}{}{język z grupy języków polinezyjskich, używany przez około 120 tys. mówiących, głównie na wyspach Polinezji Francuskiej (z wyjątkiem archipelagu Markizów), gdzie ma status języka urzędowego (obok języka francuskiego)}
\Clue{19}{}{jeden z etnicznych języków mieszanych, rozwiniętych na bazie języków europejskich kolonistów}
\Clue{22}{}{OSKOMIAN niskie drzewo uprawiane w strefie tropikalnej; smaczne owoce}
\Clue{23}{}{ośrodek medyczny, w którym można jawnie i zgodnie z prawem dokonać aborcji}
\Clue{24}{}{to, że coś jest pożyteczne, przynosi korzyści, przydaje się}
\Clue{29}{}{WŁOŚNICA, CZUMIZA, GOMI jednoroczna roślina zbożowa uprawiana na paszę na Dalekim Wschodzie}
\Clue{32}{}{niewielka forma odlewnicza wykonana z trwałego materiału}
\Clue{33}{}{Pohlia filum - gatunek mchu z rodziny prątnikowatych}
\Clue{34}{}{pojemnik na płyny}
\Clue{38}{}{rodzina polskich samochodów osobowych i dostawczych produkowanych w latach 1957-1972 przez Fabrykę Samochodów Osobowych w Warszawie, a od 1972 do 1983 przez Fabrykę Samochodów Małolitrażowych w Bielsku-Białej}
\Clue{39}{}{biały, gęsty, słodki sos typowy dla kuchni francuskiej}
\Clue{41}{}{obszar, w którym skoczkowie narciarscy wyhamowują po lądowaniu}
\Clue{42}{}{miasto w USA (Alabama) nad rzeką Coosa; wydobycie węgla kamiennego, rud żelaza}
\Clue{44}{}{urzędnik, który dokonywał opisu dóbr króla}
\Clue{45}{}{styl latynoskiego tańca towarzyskiego w rytmie bossa novy}
\Clue{46}{}{płot wykonany z dylów}
\Clue{47}{}{litewska pieśń pogrzebowa, duma tren}
\Clue{48}{}{maszyna prosta w przekroju będąca trójkątem równoramiennym}
\Clue{49}{}{kłopotliwe do sfotografowania tło}
\Clue{50}{}{coś, co sprawia kłopot, przyczyna niedogodności}\end{PuzzleClues}

\begin{PuzzleClues}{\textbf{Pionowe}\\}\Clue{1}{}{twarde drewno pozyskiwane z  drzewa Erythrophleum ivorense z rodziny bobowatych; wykorzystywane do produkcji mebli, podłóg, schodów, podkładów, mostów}
\Clue{2}{}{starożytne miasto w płd.zach. Iranie, od VI w p.n.e. stolica administracyjna państwa perskiego Tunezji}
\Clue{5}{}{sprzedawca mięsa}
\Clue{6}{}{Aotus trivirgatus - gatunek małpy szerokonosej z rodziny ponocnic, jednen z nielicznych gatunków małp prowadzących nocny tryb życia; zamieszkują najwyższe piętra drzew w zwrotnikowych i podzwrotnikowych lasach Ameryki Południowej}
\Clue{7}{}{owadożerny ptak podobny do jaskółki, latają z prędkością do 170 km/godz., w Polsce chroniony}
\Clue{9}{}{czeski kompozytor, dyrygent i altowiolista (1874-1930); utwory orkiestrowe, kameralne, operetki, opery}
\Clue{10}{}{zakład przemysłowy, który specjalizuje się w czesaniu włókien z roślin i zwierząt (wełny)}
\Clue{11}{}{minerał promieniotwórczy z gromady tlenków, należący do minerałów rzadkich}
\Clue{12}{}{sztuczny język stworzony na potrzeby serialuStar Trek}
\Clue{13}{}{silnik doczepiany do rufy łodzi motorowej}
\Clue{15}{}{członek muzułmańskiej sekty, odłamu szyitów, założonej w IX w}
\Clue{16}{}{Metriopelia aymara - gatunek ptaka z rodziny gołębiowatych (Columbidae)}
\Clue{17}{}{spaghetti z sosem na bazie jajek, boczku i dojrzewającego sera (pecorino)}
\Clue{19}{}{sieć komputerowa znajdująca się na obszarze wykraczającym poza jedno miasto (bądź kompleks miejski)}
\Clue{20}{}{symbol oznaczający jaka firma ma prawa do danego towaru}
\Clue{21}{}{James (1819-91), poeta amerykański, przeciwnik niewolnictwa}
\Clue{24}{}{układ warstwowy dwóch spojonych ze sobą metali będących mechaniczną całością}
\Clue{25}{}{idiotka, głupia, naiwna kobieta}
\Clue{26}{}{CYNOZURA; Gwiazda Polarna}
\Clue{27}{}{autor treści zamieszczanych w Wikipedii}
\Clue{28}{}{krzewinka z motylkowatych o żółtych kwiatach, z pędów i kwiatów otrzymuje się żółty barwnik}
\Clue{29}{}{cecha działania, czynności - zwykle ruchu: to, że coś jest powolne, świadczy o ociężałości (np. zmęczeniu) wykonującego}
\Clue{30}{}{ujednolicanie społeczeństwa pod względem ekonomicznym i społecznym}
\Clue{31}{}{Diplodocidae - rodzina zauropodów, której nazwa oznaczadwubelkowe; obejmowała ona jedne z największych istot żywych wszech czasów, takie jak Diplodocus czy Supersaurus}
\Clue{33}{}{fakt, że trafiło się w cel (w sporcie, grach lub prawdziwej walce)}
\Clue{35}{}{jest to proces polegający na wchodzeniu magmy w skorupę ziemską, bez wychodzenia na jej powierzchnię}
\Clue{36}{}{staroświeckie nakrycie głowy zamężnych kobiet rosyjskich, noszone na uroczyste okazje do stroju ludowego}
\Clue{37}{}{Coriandrum - rodzaj roślin z rodziny selerowatych}
\Clue{40}{}{płyn w jelicie, powstały ze strawionego pokarmu, wchłaniany przez naczynia limfatyczne kosmków jelitowych}
\Clue{43}{}{cios, gorzkie, bolesne doświadczenie życiowe}
\Clue{45}{}{kod ISO 4217 waluty lek}\end{PuzzleClues}\newpage\section*{Krzyżówka 60}

\noindent\begin{Puzzle}{24}{26}|*	|*	|[1][S]\darr	|[2][S]\drarr	|a	|c	|h	|a	|r	|d	|*	|[3][S]\drarr	|d	|e	|p	|o	|r	|t	|a	|c	|j	|a	|*	|*	|*	|.
|*	|*	|p	|a	|*	|*	|*	|*	|*	|[4][S]\rarr	|i	|n	|s	|t	|r	|u	|m	|e	|n	|t	|a	|c	|j	|a	|*	|.
|*	|[5][S]\darr	|r	|s	|*	|*	|[6][S]\darr	|*	|[7][S]\rarr	|c	|z	|e	|r	|w	|o	|n	|y	|[][,]{ }	|g	|r	|z	|y	|b	|*	|*	|.
|[8][S]\drarr	|r	|o	|t	|a	|*	|g	|[9][S]\rarr	|w	|i	|e	|l	|k	|a	|[][,]{ }	|p	|ł	|y	|t	|a	|*	|[10][S]\darr	|[11][S]\darr	|*	|*	|.
|p	|o	|m	|r	|[12][S]\drarr	|f	|a	|t	|a	|ł	|a	|s	|z	|k	|i	|*	|*	|*	|*	|*	|*	|k	|e	|[13][S]\darr	|*	|.
|r	|z	|o	|o	|z	|[14][S]\darr	|ł	|*	|*	|[15][S]\drarr	|p	|o	|w	|ł	|o	|k	|a	|*	|[16][S]\darr	|*	|[17][S]\darr	|a	|g	|c	|*	|.
|z	|p	|t	|b	|a	|z	|ą	|*	|[18][S]\darr	|k	|[19][S]\drarr	|n	|a	|c	|z	|ó	|ł	|e	|k	|*	|p	|n	|o	|i	|*	|.
|e	|r	|o	|l	|m	|i	|ź	|[20][S]\drarr	|b	|o	|a	|*	|*	|[21][S]\drarr	|c	|i	|l	|i	|o	|p	|a	|t	|i	|a	|*	|.
|d	|a	|r	|e	|e	|n	|[][,]{ }	|g	|a	|m	|l	|*	|*	|d	|*	|*	|*	|*	|c	|*	|s	|*	|s	|n	|*	|.
|r	|w	|*	|m	|k	|j	|t	|r	|z	|p	|u	|[22][S]\drarr	|m	|u	|t	|u	|a	|l	|i	|z	|m	|*	|t	|g	|*	|.
|z	|a	|*	|*	|*	|a	|r	|z	|i	|l	|w	|r	|[23][S]\darr	|m	|*	|[24][S]\rarr	|i	|v	|a	|l	|o	|*	|k	|s	|*	|.
|e	|[][,]{ }	|*	|*	|[25][S]\darr	|n	|z	|a	|n	|a	|i	|o	|w	|k	|*	|*	|*	|*	|[][,]{ }	|*	|*	|*	|a	|u	|*	|.
|ź	|d	|*	|[26][S]\darr	|k	|t	|e	|n	|*	|n	|u	|s	|e	|a	|[27][S]\darr	|[28][S]\drarr	|l	|a	|m	|p	|k	|a	|*	|*	|*	|.
|n	|o	|*	|o	|a	|r	|w	|k	|*	|a	|m	|s	|k	|*	|f	|s	|[29][S]\rarr	|b	|a	|l	|d	|w	|i	|n	|*	|.
|i	|k	|*	|r	|k	|o	|n	|a	|*	|c	|[][,]{ }	|*	|t	|*	|o	|u	|[30][S]\drarr	|i	|m	|b	|i	|r	|*	|*	|*	|.
|a	|t	|[31][S]\rarr	|k	|i	|p	|a	|*	|*	|j	|r	|*	|o	|*	|o	|w	|s	|[32][S]\drarr	|a	|l	|e	|f	|*	|*	|*	|.
|c	|o	|*	|*	|*	|*	|*	|[33][S]\rarr	|l	|a	|z	|a	|r	|e	|t	|*	|i	|s	|*	|*	|*	|*	|*	|*	|*	|.
|z	|r	|[34][S]\rarr	|l	|u	|z	|i	|n	|o	|*	|e	|*	|*	|*	|*	|*	|a	|k	|*	|[35][S]\darr	|*	|*	|*	|*	|*	|.
|*	|s	|*	|*	|[36][S]\rarr	|a	|m	|u	|n	|i	|c	|j	|a	|[][,]{ }	|j	|ą	|d	|r	|o	|w	|a	|*	|*	|*	|*	|.
|[37][S]\rarr	|k	|w	|a	|s	|[][,]{ }	|t	|ł	|u	|s	|z	|c	|z	|o	|w	|y	|[][,]{ }	|o	|m	|e	|g	|a	|[][S]-	|[][S]6	|*	|.
|*	|a	|[38][S]\darr	|[39][S]\drarr	|c	|i	|e	|m	|i	|e	|n	|i	|e	|c	|*	|*	|k	|b	|[40][S]\rarr	|n	|i	|l	|i	|n	|*	|.
|*	|*	|g	|d	|*	|*	|*	|[41][S]\rarr	|k	|l	|e	|s	|z	|c	|z	|*	|u	|k	|*	|e	|*	|*	|*	|*	|*	|.
|[42][S]\rarr	|i	|r	|u	|n	|*	|*	|*	|[43][S]\rarr	|k	|*	|*	|*	|*	|*	|*	|c	|a	|[44][S]\rarr	|t	|o	|n	|*	|*	|*	|.
|[45][S]\drarr	|m	|u	|s	|t	|e	|r	|r	|o	|l	|a	|*	|*	|*	|*	|*	|z	|*	|*	|*	|*	|*	|*	|*	|*	|.
|t	|*	|p	|z	|*	|*	|[46][S]\rarr	|g	|a	|z	|[][,]{ }	|w	|u	|l	|k	|a	|n	|i	|c	|z	|n	|y	|*	|*	|*	|.
|e	|*	|a	|a	|*	|*	|*	|*	|*	|*	|*	|[47][S]\rarr	|n	|a	|c	|z	|y	|ń	|k	|o	|*	|*	|*	|*	|*	|.
|*	|*	|*	|*	|*	|[48][S]\rarr	|m	|i	|t	|s	|u	|b	|i	|s	|h	|i	|*	|*	|*	|*	|*	|*	|*	|*	|*	|.\end{Puzzle}

\newpage

\begin{PuzzleClues}{\textbf{Poziome}\\}\Clue{2}{}{pisarz francuski (1899-1974), komedie bulwarowe, dramaty psychologiczne; „Idiotka”}
\Clue{3}{}{wydalenie cudzoziemca z kraju na podstawie decyzji administracyjnej}
\Clue{4}{}{opracowanie utworu instrumenty zespołu w formie partytur; orkiestracja}
\Clue{7}{}{gatunek grzybów z rodziny borowikowatych o kapeluszu z odcieniami czerwieni}
\Clue{8}{}{formuła przysięgi}
\Clue{9}{}{osiedle złożone z bloków wybudowanych w technice wielkiej płyty}
\Clue{12}{}{żartobliwie o ubraniach, strojach, ciuszkach, zwykle damskich}
\Clue{15}{}{warstwa materiału wytworzona lub nałożona na powierzchnię przedmiotu w celach zdobniczych, ochronnych i innych}
\Clue{19}{}{dawna ozdobna przepaska damska noszona na czole}
\Clue{20}{}{długi, wąski szal z futra, strusich piór lub puchu}
\Clue{21}{}{choroba z grupy chorób, w których patogenezie odgrywa rolę nieprawidłowa funkcja rzęsek i wici}
\Clue{22}{}{doktryna złagodzonego socjalizmu, oparta na zasadzie wzajemności w przeciwstawianiu do walki konkurencyjnej}
\Clue{24}{}{fińskie miasto na płd. od jeziora Inari}
\Clue{28}{}{znicz - szklane, metalowe bądź plastikowe naczynie wypełnione stearyną, w której zatopiony jest knot; zapalane na grobach w celu uczczenia pamięci  zmarłych}
\Clue{29}{}{pisarz amerykańskie 924-87), czołowy przedstawiciel literatury murzyńskiej; „Inny kraj”, „Na spotkanie człowieka”, „Następnym razem pożar”}
\Clue{30}{}{warzywo, korzeń (kłącze) imbiru lekarskiego}
\Clue{31}{}{drewniane lub metalowe oczko do przeprowadzania szotów żagli przednich i obciągania ich ku dołowi}
\Clue{32}{}{pierwsza litera alfabetów semickich}
\Clue{33}{}{szpital wojskowy, zwłaszcza polowy, przeznaczony do opatrywania rannych na polu walki oraz izolacji i leczenia chorych żołnierzy}
\Clue{34}{}{duża wieś kaszubska o charakterze małomiasteczkowym w Polsce, położona w województwie pomorskim, w powiecie wejherowskim, w gminie Luzino, nad rzeką Bolszewką w pobliżu drogi krajowej nr 6}
\Clue{36}{}{amunicja zawierająca ładunek jądrowy materiału wybuchowego w głowicach pocisków artyleryjskich i rakietowych, bomb lotniczych, torped oraz w minach}
\Clue{37}{}{nienasycony kwas tłuszczowy, którego ostatnie wiązanie podwójne znajduje się przy szóstym od końca atomie węgla łańcucha węglowodorowego}
\Clue{39}{}{Dictyna - rodzaj pająka z rodziny ciemieńcowatych}
\Clue{40}{}{pisarz rosyjski ur. 1908r, szkice i opowiadania o tematyce wojennej i społeczno-obyczajowej; „Okrucieństwo”}
\Clue{41}{}{drobny pajęczak z rzędu roztoczy, pasożyt kręgowców, których krwią się żywi}
\Clue{42}{}{miasto w Hiszpanii (Baskonia) nad Zatoką Biskajską przejście graniczne z Francją}
\Clue{43}{}{w chemii: symbol potasu}
\Clue{44}{}{w językoznawstwie: sposób akcentowania danej samogłoski (gł. rozpatrywany w językach tonalnych - tam jest czynnikiem znaczącym, tworzy opozycje), który polega na zmianie wysokości brzmienia sylaby, intonacja}
\Clue{45}{}{lista załogi}
\Clue{46}{}{gaz wydobywający się ze stożka wulkanicznego, przeważnie podczas erupcji}
\Clue{47}{}{bardzo cienkie, podskórne naczynie krwionośne}
\Clue{48}{}{samochód marki Mitsubishi}\end{PuzzleClues}

\begin{PuzzleClues}{\textbf{Pionowe}\\}\Clue{1}{}{substancja, która po dodaniu do katalizatora znacznie przyspiesza szybkość reakcji}
\Clue{2}{}{duży krater po meteorycie zniszczony znacznie wskutek długotrwałych procesów erozyjnych}
\Clue{3}{}{niemiecki filozof i prawnik (1882-1927); zajmował się teorią poznania i analizą pojęć etycznych}
\Clue{5}{}{praca naukowa napisana w celu zdobycia stopnia doktora}
\Clue{6}{}{pień nerwu błędnego}
\Clue{8}{}{ptak z rzędu wróblowatych, naśladuje zasłyszane głosy; Ameryka Płn}
\Clue{10}{}{pisarz niemiecki ur. w 1926r; „Pobyt”, „Aula”, „Stopka redakcyjna”}
\Clue{11}{}{niewielki, jednoosobowy pojazd konny, przeznaczony na krótkie trasy, popularny w drugiej połowie XVIII wieku}
\Clue{12}{}{ruchoma część odtylcowej broni palnej}
\Clue{13}{}{JIANGSU - prowincja we wschodnich Chinach, obszar 102,2 tyś. km2, ośrodek administracyjny Nankin}
\Clue{14}{}{Zinjanthropus boisei, Paranthropus boisei - gatunek wymarłego hominida, przedstawiciel australopiteków masywnych, który zamieszkiwał Afrykę ok. 2,3-1,2 mln lat temu}
\Clue{15}{}{akt zgody, porozumienia, stanowiący zakończenie zatargu, kończący spór, proces lub zapobiegający procesowi}
\Clue{16}{}{kobieta będąca miłośniczką kotów}
\Clue{17}{}{kilka zwiniętych i przewiązanych pasemek przędzy}
\Clue{18}{}{dermatolog francuski (1807-78); podał pierwsze opisy wielu chorób skórnych i pasożytniczych}
\Clue{19}{}{osady rzeczne}
\Clue{20}{}{tost - przypieczona kromka chleba}
\Clue{21}{}{DUMA}
\Clue{22}{}{sir Alec (1800-1862) angielski żeglarz, odkrywca Ziemi Wiktorii}
\Clue{23}{}{organizm przenoszący pasożyta lub drobnoustrój zakaźny}
\Clue{25}{}{PERSYMONA, HEBANOWIEC wschodnie drzewo owocowe uprawiane głównie w Chinach, Japonii dla dużych, deserowych owoców}
\Clue{26}{}{istota z rasy orków w literackim legendarium, wykreowanym przez J. R. R. Tolkiena; Tolkienowscy orkowie, wyhodowani z upodlonych elfów przez Morgotha, związani byli z obozem zła}
\Clue{27}{}{STOPA}
\Clue{28}{}{część cyklu pracy silnika tłokowego}
\Clue{30}{}{w gimnastyce: siedzenie na ugiętych nogach}
\Clue{32}{}{narzędzie do czyszczenia, wygładzania, skrobania}
\Clue{35}{}{przedstawiciel ludu celtyckiego zamieszkującego w starożytności Bretanię}
\Clue{38}{}{grupa ludzi, zbiorowość, której członkowie połączeni są jakąś więzią, jakąś relacją}
\Clue{39}{}{osoba (lub grupa ludzi) odgrywająca główną, przewodnią rolę w jakimś przedsięwzięciu lub strukturze}
\Clue{45}{}{symbol telluru w układzie okresowym}\end{PuzzleClues}


\end{document}
